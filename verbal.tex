\documentclass[11pt]{article}
\usepackage{standalone} \newif\ifstandlone \standalonetrue
\usepackage[left=1.75in, right=1.75in, top=1.25in, bottom=1.25in]{geometry}
\geometry{letterpaper}
\usepackage{graphicx}
\usepackage{enumitem}
%\usepackage{amssymb}
\usepackage{amsmath}
\usepackage{epstopdf}
\usepackage{verbatim}
\usepackage{setspace}
\usepackage{natbib}
\setcitestyle{aysep={}}
\usepackage{hyperref}
\usepackage{url}
\synctex=1

\DeclareSymbolFont{symbolsC}{U}{txsyc}{m}{n}
\DeclareMathSymbol{\strictif}{\mathrel}{symbolsC}{74}
\DeclareMathSymbol{\boxright}{\mathrel}{symbolsC}{128}

\newenvironment{squote}{%
\begin{spacing}{1}
       	\begin{list}{}{%
\setlength{\labelwidth}{0pt}%
\rightmargin\leftmargin%
}
\item\relax
}{%
\end{list}%
\end{spacing}
}

\title{Introduction}
\author{Alexander A. Dunn}
\begin{document}
\ifstandalone
\maketitle
\begin{spacing}{1.25}
\fi

%\section{What are we doing?}
I believe that there are chairs.  I believe that there are desks, and
desk lamps, and doors, and doorways, and houses, and gardens, and
plants, and people.  Such things, and many others, are commonly termed
`ordinary things'.  This term is extremely vague in its application,
but is taken to refer to macroscopic objects, such as those listed
above, that are parts of our everyday lives.

Many philosophers have denied that ordinary things exist.  Until
recently, such a denial was generally a consequence of the
philosopher's views on other matters.  If a philosopher claimed that
there was no external world, or that the world was not at all like it
appears, then they might deny that there were any physical things, or
any things that exist outside the mind, or anything at all.  It would
follow from such a claim that there would be no ordinary things, like
chairs.  But the philosopher would not be specifically interesting in
denying that chairs exist.  They were interested in denying that {\em
  anything} exists; the denial of chairs was a minor consequence.

In the past 30 years, however, philosophers have begun to construct
arguments specifically aiming to show that there are no ordinary
things.  (Peter Unger was one of the first, with the aptly titled
paper, ``There are no ordinary things''.)  These philosophers do not
deny that there is an external world, or that it contains many
physical things; these propositions are readily granted to be true.
But the philosophers are unwilling to admit that such a world
does---or even possibly could---contain chairs.

Most philosophers making this sort of claim admit that it is strange
and unintuitive.  But they believe that the benefits of denying the
existence of ordinary things outweighs the costs.  Different
philosophers cite different benefits: consistency with regard to our
notion of composition, theoretical simplicity, or greater coherency
among our beliefs.  

The benefits do not out-weight the cost.  Moreover, I am unable to
imagine that any argument could convince me that there are no ordinary
things.  I believe that any argument that has the nonexistence of
chairs as a consequence is flawed.  Whether or not we can immediately
identify the flaw in the argument, the fact that it entails a
falsehood shows that something has gone amiss.

It will be objected that this is merely a fact about myself; other
philosophers are perfectly willing to deny that there are chairs.  (It
is another fact about me that I doubt that they really believe that
there are no chairs.)  It may be argued that the fact that I consider
``there are chairs'' to be true regardless of arguments against its
truth shows that I consider it to be, in some sense, a conceptual
truth.  It may be further argued that, since there are philosophers
willing to deny that ``there are chairs'' is true, what I mean by
``there are chairs'' is something different than what these
philosophers mean by ``there are chairs''.  We may be thought to be
using our words in different ways.

\section{English ontology}
Some philosophers maintain that the recent disputes over the existence
of ordinary things are {\em merely verbal disputes}.  Suppose one
philosopher claims that nothing is part of something else.  This
philosopher will say things like ``there are no chairs'' Suppose
another philosopher rejects this view.  This philosopher will says
``there are chairs''.  It seems obvious that these philosophers are
disagreeing.  But according to some, they are not disagreeing at all.
The philosopher Eli Hirsch claims that our two philosophers do not
mean the same thing by ``there are''.  To see why this is so, he
explains that the philosopher who denies that there are chairs will
willingly admit that there are {\em things arranged chairwise} (these
things may be atoms or whatever partless `simples' there actually
are).  Whenever the second philosopher says something like ``there is
a chair here'', the first philosopher will say something like ``there
are things arranged chairwise here''.  The first philosopher should
therefore assume that what the second philosopher {\em means} by
``there is a chair here'' {\em is} ``there are things arranged
chairwise here''.  Doing this allows the first philosopher to
interpret the second as saying something true, rather than something
false.

Whenever the first philosopher says, ``there are no chairs'', this is
because she claims that nothing is a part of anything and that chairs,
if they existed, would have parts.  Whenever she says ``there are no
chairs'', therefore, the second philosopher (who believes that there
are chairs and have parts) will say ``there are no chairs that have no
parts''.  The second philosopher should assume that what the first
philosopher {\em means} by ``there are no chairs'' {\em is} ``there
are no chairs that have no parts''.  Doing this allows the second
philosopher to interpret the first as saying something true, rather
than something false.

Hirsch's reason for advancing these arguments is to show that
philosophers who deny that there are chairs are not actually speaking
English.  According to Hirsch, it is a conceptual truth (or something
like it) in English that there are chairs.  When a philosopher denies
that there are chairs, therefore, we can assume one of two things.
Either the philosopher is not speaking English, or the philosopher is
saying something obviously false.  Hirsch claims that the latter
assumption is not {\em charitable}.  We should charitably assume that
other speakers do not say obviously false things; we should therefore
assume that someone who says ``there are no chairs'' is not speaking
English.

Some philosophers who deny that there are chairs have granted Hirsch
his conclusion and admitted that when they say things like ``there are
no chairs'', they are not speaking English.  The most notable such
philosopher is Ted Sider, who advocates that philosophers (like
himself) who are tempted to deny that there are chairs should {\em
  stipulate} a new language in which to conduct their inquiries.
Sider urges that philosophers should speak ``Ontologese'', a special
language that is not subject to interpretive charity:

\begin{squote}
{[}The philosophers{]} should stipulate that their quantifiers are to be
understood as theoretical terms (and so are not subject to the same
level of metasemantic pressure from charity that governs terms like
`sofa' and `game') that stand for whatever joint-carving notion is in
the vicinity \citeyearpar[9]{sider2011b}.
\end{squote}

Sider claims that when a speaker of ``Ontologese'' says ``there are no
chairs'', they may not be interpreted as meaning ``there are no chairs
that have no parts'' or any other such interpretation that an English
speaker might apply so as to make what they said true.  Sider hopes
that when {\em he} says ``there are no chairs'', it will be true (if
it is true) because a ``joint-carving'' quantifier does not range over
chairs.

I'm not sure that I fully understand what Sider is trying to do here.
When at the beginning of this Introduction I said ``there are
chairs'', I was writing in English.  The only language with which I
have any proficiency is English; I don't know how to make that claim
in any other language.  If what I said was true---if ``there are
chairs'' is true in English---then there are chairs.  That is the
claim I want to defend.  If Sider is willing to give it to me, then my
job is done.

Perhaps I don't understand how Sider is going to invent a new language
in which to do metaphysics.  I don't see how he can stipulate that the
quantifiers in ``Ontologese'' should pick out the most
``joint-carving'' notion in the vicinity, whatever that may be.  It's
as if he introduces the name `Heerriet' to pick out whichever person
named `Harriet' is nearest to the speaker.  One can't thereby talk
about Heerriet.  I can certainly say things like ``I wonder if
Heerriet remembered to call her mother on her birthday'', but neither
I nor my audience have any idea who I'm talking about (unless we both
happen to know who Heerriet currently is).  Likewise, I simply don't
understand how Sider expects to know what he's talking about when he
uses his 'new' quantifier phrases.  When I use quantifier phrases like
``there are'' in ``there are chairs'', I know perfectly well what I
mean.  Regarding chairs, there are some (many) of them.

Moreover, Hirsch's claim that these disputes are verbal is not
obviously true.  Hirsch relies on the idea that if two propositions
have equivalent truth-conditions, then (given one or more unspecified
conditions), they {\em mean} the same thing.  Truth-conditional
theories of meaning face many difficulties, so the success of Hirsch's
project is conditional on the resolution of those difficulties.

\section{Inexplicable beliefs}
Henceforth I will assume that the debate over whether there are chairs
is conducted in English.  But even some philosophers who agree to this
will deny that ``there are chairs'' is a conceptual truth.  Such
philosophers will object that so far, the only objection I have raised
against the view that there are no chairs is that I cannot bring
myself to believe it.

However, there is another reason to resist their conclusions, one that
is independent of my inability to believe that there are no chairs.
As we will see in Section \ref{stroud}, philosophers who deny that
there are chairs have a difficult time explaining why we believe that
there are chairs.  To the extent that they cannot explain why we hold
this belief (and others concerning ordinary things), we have reason to
suspect that their denials might be unfounded.

Even if we show that there are problems with the arguments of
philosophers who deny that ordinary things exist, we have not thereby
proved that they {\em do} exist.  The philosophers who say that there
are no chairs are motivated by a number of questions about the nature
of ordinary things.  For example, why are there chairs and tables, but
not chair-tables (single objects composed of an adjacent table and
chair)?  If chairs are physical things, then they are made up of atoms
(or even smaller things); is there a determinate number of this
microscopic objects in a given chair, and can we know what the number
is?

These and other questions have troubled some philosophers to the point
that they choose to deny that there are chairs at all.  If there are
no chairs, then there is no question of why there are chairs but not
chair-tables; if there are no chairs then there is no question of how
many atoms compose them.  But such a position is incorrect, because
there {\em are} chairs.  This thesis, therefore, is an attempt to
begin to answer some of the difficult questions about chairs and other
ordinary things.  I may be compelled to give some strange and
implausible answers of my own.  But no matter how strange, if they do
not have the consequence that there are no chairs, then I will
consider myself to have succeeded.

\ifstandalone
\end{spacing}
\bibliography{everything}
\bibliographystyle{ChicagoReedweb}
\fi
\end{document}
