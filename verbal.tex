\documentclass[11pt]{article}
\usepackage{standalone} \newif\ifstandlone \standalonetrue
\usepackage[left=1.75in, right=1.75in, top=1.25in, bottom=1.25in]{geometry}
\geometry{letterpaper}
\usepackage{graphicx}
\usepackage{enumitem}
%\usepackage{amssymb}
\usepackage{amsmath}
\usepackage{epstopdf}
\usepackage{verbatim}
\usepackage{setspace}
\usepackage{natbib}
\setcitestyle{aysep={}}
\usepackage{hyperref}
\usepackage{url}
\synctex=1

\DeclareSymbolFont{symbolsC}{U}{txsyc}{m}{n}
\DeclareMathSymbol{\strictif}{\mathrel}{symbolsC}{74}
\DeclareMathSymbol{\boxright}{\mathrel}{symbolsC}{128}

\newenvironment{squote}{%
\begin{spacing}{1}
       	\begin{list}{}{%
\setlength{\labelwidth}{0pt}%
\rightmargin\leftmargin%
}
\item\relax
}{%
\end{list}%
\end{spacing}
}

\title{English ontology}
\author{Alexander A. Dunn}
\begin{document}
\ifstandalone
\maketitle
\begin{spacing}{1.25}
\fi

% \section{What are we doing?}
I believe that there are chairs.  I believe that there are desks, and
desk lamps, and doors, and doorways, and houses, and gardens, and
plants, and people.  Such things, and many others, are commonly termed
`ordinary things'.  This term is extremely vague in its application,
but is taken to refer to macroscopic objects, such as those listed
above, that are parts of our everyday lives.

Many philosophers have denied that ordinary things exist.  Until
recently, such a denial was generally a consequence of the
philosopher's views on other matters.  If a philosopher claimed that
there was no external world, or that the world was not at all like it
appears, then they might deny that there were any physical things, or
any things that exist outside the mind, or anything at all.  It would
follow from such a claim that there would be no ordinary things, like
chairs.  But the philosopher would not be specifically interesting in
denying that chairs exist.  They were interested in denying that {\em
  anything} exists; the denial of chairs was a minor consequence.

In the past 30 years, however, philosophers have begun to construct
arguments specifically aiming to show that there are no ordinary
things.  (Peter Unger was one of the first, with the aply titled
paper, ``There are no ordinary things''.)  These philosophers do not
deny that there is an external world, or that it contains many
physical things; these propositions are readily granted to be true.
But the philosophers are unwilling to admit that such a world
does---or even possibly could---contain chairs.

Most philosophers making this sort of claim admit that it is strange
and unintuitive.  But they believe that the benefits of denying the
existence of ordinary things outweighs the costs.  Different
philosophers cite different benefits: consistency with regard to our
notion of composition, theoretical simplicity, or greater coherency
among our beliefs.  

The benefits do not outweight the cost.  Moreover, I am unable to
imagine that any argument could convince me that there are no ordinary
things.  I believe that any argument that has the nonexistence of
chairs as a consequence is flawed.  Whether or not we can immediately
identify the flaw in the argument, the fact that it entails a
falsehood shows that something has gone amiss.

It will be objected that this is merely a fact about myself; other
philosophers are perfectly willing to deny that there are chairs.  (It
is another fact about me that I doubt that they really believe that
there are no chairs.)  However, there is another reason to resist
their conclusions, one that is independent of my inability to believe
that there are no chairs.  As we will see in Section \ref{stroud},
philosophers who deny that there are chairs have a difficult time
explaining why we believe that there are chairs.  To the extent that
they cannot explain why we hold this belief (and others concerning
ordinary things), we have reason to suspect that their denials might
be unfounded.

Even if we show that there are problems with the arguments of
philosophers who deny that ordinary things exist, we have not thereby
proved that they {\em do} exist.  The philosophers who say that there
are no chairs are motivated by a number of questions about the nature
of ordinary things.  For example, why are there chairs and tables, but
not chair-tables (single objects composed of an adjecent table and
chair)?  If chairs are physical things, then they are made up of atoms
(or even smaller things); is there a determinate number of this
microscopic objects in a given chair, and can we know what the number
is?

These and other questions have troubled some philosophers to the point
that they choose to deny that there are chairs at all.  If there are
no chairs, then there is no question of why there are chairs but not
chair-tables; if there are no chairs then there is no question of how
many atoms compose them.  But such a position is incorrect, because
there {\em are} chairs.  This thesis, therefore, is an attempt to
begin to answer some of the difficult questions about chairs and other
ordinary things.  I may be compelled to give some strange and
implausible answers of my own.  But no matter how strange, if they do
not have the consequence that there are no chairs, then I will
consider myself to have succeeded.

\ifstandalone
\end{spacing}
\bibliography{everything}
\bibliographystyle{ChicagoReedweb}
\fi
\end{document}
