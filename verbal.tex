\documentclass[11pt]{article}
\usepackage{standalone} \newif\ifstandlone \standalonetrue
\usepackage[left=1.75in, right=1.75in, top=1.25in, bottom=1.25in]{geometry}
\geometry{letterpaper}
\usepackage{graphicx}
\usepackage{enumitem}
%\usepackage{amssymb}
\usepackage{amsmath}
\usepackage{epstopdf}
\usepackage{verbatim}
\usepackage{setspace}
\usepackage{natbib}
\setcitestyle{aysep={}}
\usepackage{hyperref}
\usepackage{url}
\synctex=1

\DeclareSymbolFont{symbolsC}{U}{txsyc}{m}{n}
\DeclareMathSymbol{\strictif}{\mathrel}{symbolsC}{74}
\DeclareMathSymbol{\boxright}{\mathrel}{symbolsC}{128}

\newenvironment{squote}{%
\begin{spacing}{1}
       	\begin{list}{}{%
\setlength{\labelwidth}{0pt}%
\rightmargin\leftmargin%
}
\item\relax
}{%
\end{list}%
\end{spacing}
}

\title{What do I mean when I say there are chairs?}
\author{Alexander A. Dunn}
\begin{document}
\ifstandalone
\maketitle
\begin{spacing}{1.25}
\fi

\label{verbal}
Philosophers such as Eli Hirsch have argued that metaphysical
disputes---such as whether there are chairs---are verbal disputes that
have no real import.  Hirsch claims that ``there are chairs'' is
obviously true in English.  If a philosopher says ``there are no
chairs'' on the grounds that there is nothing in the world but
partless atoms, Hirsch will {\em interpret} them as meaning something
like ``there are no partless atoms that are also chairs''.  This
interpretation allows Hirsch to maintain that both philosophers are
saying true things, and are not really disagreeing at all.  Ted Sider
and others have replied by attempting to invent a new language,
Ontologese, in which it is not obviously that ``there are chairs'' is
true.  In this section I will argue that Sider's `Ontologese gambit'
cannot work, but that Hirsch's argument is flawed as well.

\section{Verbal disputes}
\label{hirsch}
% Is Sider's Ontologese just physics?
Some philosophers maintain that the recent disputes over the existence
of ordinary things are {\em merely verbal disputes}.  Suppose one
philosopher claims that nothing is part of something else.  This
philosopher will say things like ``it is not true that there are
chairs'' Suppose another philosopher rejects this view.  This
philosopher will says ``there are chairs''.  It seems obvious that
these philosophers are disagreeing.  But according to some, they are
not disagreeing at all.  

Eli Hirsch claims that our two philosophers are engaged in a {\em
  verbal dispute}.  A verbal dispute is one that is somehow not
substantive; Hirsch's paradigm case of a verbal dispute is over
whether glasses are cups:

\begin{squote}
  I know someone, whom I'll call $A$, who claimed that a standard
  drinking glass is a cup.  ``Just as a cat is a kind of animal,'' she
  said,``a glass is a kind of cup.''  Everyone else whom I've asked
  about this agrees with me that a glass is not a cup.  Clearly, this
  dispute is, in some sense, merely about
  language~(\citeyear[69]{hirsch2005}).
\end{squote}

To see that this dispute is verbal, Hirsch instructs us to do the
following:

\begin{enumerate}
  \item Take what each disputant says.
  \item Postulate a community that agrees with that disputant.
  \item Interpret each community's language so that the relevant
    utterances come out true.
  \item Interpret each disputant as speaking the language of their
    community.
\end{enumerate}

For example, when $A$ says ``There is a cup on the table'', Hirsch
would say instead ``There is a cup or there is a glass on the table''.
Postulating a community that agrees with $A$ (the $A$-community) and
imagining a community that agrees with Hirsch (the $H$-community), we
may assign the truth-conditions this way:
\begin{enumerate}[itemindent=25pt, label=(T)]
    \item ``There is a cup on the table'' is true in $A$-English iff
    ``There is a cup or glass on the table'' is true in $H$-English
\end{enumerate}
($A$ is the dialect of English spoken in the $A$-community and $H$ is
the dialect spoken in the $H$-community.)

One might object that we have not shown that $A$ and $H$ are in fact
speaking different languages.  ``All you have shown,'' $H$
might say, ``is that we can imagine them speaking a language in which
what they say is true.  But you have not shown that they {\em are}
speaking such a language.  What $A$ says sounds like normal English to
me, and in English, what she says is simply false.v''

What justifies us in postulating $A$- and $H$-English is simply that
what $A$ {\em means} by `cup' is not what $H$ means by `cup'.  $A$
uses `cup' to refer to all the things that $H$ uses `cup' to refer to,
but she also uses `cup' to refer to those things that $H$ refers to
exclusively by `glass'.  What $A$ means by `cup' is what $H$ means by
`cup or glass'.

It is this difference in {\em meaning}, as well as in
truth-conditions, that allows us to postulate `$A$-English' and
`$H$-English' and to conclude that the dispute between $A$ and Hirsch
is verbal. If we understand $A$ to mean by ``There is a cup on the
table'' what we mean by ``There is a cup or a glass on the table'',
then $A$ is not saying something false.  We thought that they were
disagreeing over what was on the table, but they simply meant
different things by their words.

Hirsch does not explicitly claim that meaning is reducible to
truth-conditions, but he is clearly relying on a close connection
between the two:

\begin{squote}
When I speak throughout this paper about interpreting a language this
is always to be understood in the narrow sense of assigning truth
conditions.  I leave it open what there is to understand a language
beyond knowing the truth conditions of its sentences, but, whatever
this additional element may be, it will have a bearing on my argument
only insofar as it might affect the plausibility of certain
truth-condition assignments \citeyearpar[72]{hirsch2005}.
\end{squote}

Having given this warning, Hirsch speaks freely of meaning instead of
mere truth-conditions.  When imaging himself as David Lewis
interpreting Roderick Chisholm, he suggests that we ``reject the
assumption that the RC-speakers [Roderick Chisholm's `community
  language'] mean what we [speakers of the David Lewis language]
mean'' \citeyearpar[76]{hirsch2005} and advocates ``{\em semantically
  restricted quantifiers}'' \citeyearpar[76, his
  emphasis]{hirsch2005}.  When discussing these `RC' quantifiers, he
goes on to say this:

\begin{squote}
The RC-speakers will, of course, make the platitudinous disquotational
assertion, ``If something exists it is referred to by the word
`something'.''  Given what they {\em mean} by ``something'' this
sentence is trivially true \citeyearpar[77, my emphasis]{hirsch2005}.
\end{squote}

Without committing Hirsch to exactly the following thesis, I think he
would accept some modification of:
\begin{enumerate}[itemindent=25pt, label=(V)]
    \item If $\square$($p$ is true iff $q$ is true), then $p$ and $q$ mean
    the same thing. \label{v}
\end{enumerate}

This is a stronger thesis than Hirsch needs to accept.  Moreover, it
is probably not true; it seems to entail that there is only one
necessary proposition.  But {\em something like this} underlies
Hirsch's argument.

\subsection{Charity}
\label{charity}
But Hirsch's conclusion that $A$ means by `cup' what he means by `cup
or glass' does not follow from~\ref{v} alone.  To see why this is so,
recall the dispute between $A$ and $H$ over whether a glass is a cup.
Suppose that $H$ accepts~\ref{v}.  He might say, ``We are both
speaking English.  In English, `cup' does not mean the same as `cup or
glass'. Therefore, by {\em modus ponens}, it is not necessarily true
that `There is a cup on the table' is true iff `There is a cup or
glass on the table' is true.  For when there is a glass on the table,
the latter proposition is true and the former false.  And yet $A$
insists on treating these as somehow identical.  She affirms one if
and only if she affirms the other.  Evidently, she is deeply
confused.''

$H$ could accuse $A$ of making fundamental mistakes about language or
perception, and $A$ could level the same accusation at $H$.  But
Hirsch thinks that this is a poor way of understanding the
debate.  Instead of supposing that ``the other has some incurably
irrational tendency to make a priori mistakes about what they perceive
in front of their faces''~\citep[78]{hirsch2005}, we should pursue a
policy of {\em interpretive charity}:

\begin{squote}
Why is it plausible to suppose that in the $A$-language the word
`cup' doesn't mean what it means in our language, so that the sentence
`A glass is a cup'' is true in that language?  The basic answer to
this question comes out of a widely accepted principle of linguistic
interpretation that has often been called the principle of charity.''
This principle, put very roughly, says that, other things being equal,
an interpretation is plausible to the extent that its effect is to
make many of the community's shared assertions come out true or at
least reasonable~(\citeyear[71]{hirsch2005}).
\end{squote}

We can see the correctness of this principle by imagining a resolution
to the dispute between $A$ and $H$.  Any neutral arbitrator should
sit them down and explain things thus:  ``Now $A$, you said that just
as a cat is a kind of animal, a glass is a kind of cup.  $H$, you
probably disagree; you think cups and glasses are like cats and
dogs.  But given that $A$ thinks of cups and glasses like she does, you
should remember when she says `cup', that she just means anything that
you'd call either a cup or a glass.  And $A$, when $H$ talks about
cups, remember that he means only the cups that aren't glasses.''

Unless $A$ and $H$ are simply looking for something to bicker about,
they will agree that they each mean these different things by `cup';
having recognized this, the argument dissolves.  The only question
remaining is which meaning is shared by the majority of English
speakers~\citep[70]{hirsch2005}.

Hirsch diagnoses verbal disputes by applying his principle of
interpretive charity alongside a version of~\ref{v}.  If he can
interpret the propositions of two disputants so as to make all come
out true, and if these equivalences in truth-conditions correspond
with equivalent meanings, then Hirsch has shown a dispute to be
verbal.  Unfortunately, while this method works well for his test case
involving $A$ and $H$, it does not appear to succeed when applied to
the metaphysical disputes that are his primary subjects.  He does
manage to interpret the apparently conflicting propositions of the
competing metaphysicians so that neither contradicts the other;
however, his truth-conditional interpretations fail to preserve
meaning.

Consider Hirsch's analysis of the dispute between a
four-dimensionalist and a mereological essentialist.  Hirsch uses
David Lewis and Roderick Chisholm as mascots for these respective
positions.  We are to suppose that Chisholm (hereafter referred to as
`$RC$') and Lewis (`$DL$') are sitting at a table.  Upon the table is
a pencil. $DL$ claims that objects have temporal parts, and that any
set of objects and/or temporal parts has a fusion; that is, there is
an object composed of the objects in that set.  $RC$, on the other
hand, claims that objects do not have temporal parts (there are no
such things); the only physical objects are masses of matter.

$DL$ and $RC$ obviously have different things to say about the pencil
on the table.  $DL$ claims that a temporal part of the eraser from
$t_{1}$ fuses with a temporal part of the wood from $t_{2}$; thus $DL$
says that ``There is something on the table that is pink, then brown.
$RC$ denies this asserting that ``There is nothing on the table that
is pink, then brown''.  Both, however, agree that ``There is something
that is pink, then something that is brown''.

Hirsch imagines himself as a four-dimensionalist trying to interpret
$RC$, and then as a mereological essentialist interpreting $DL$.  He
claims that from the point of view of $DL$, the quantifiers in
$RC$-English are {\em semantically restricted}; ``the rough idea seems
to be that the range of the $RC$-quantifiers excludes any physical
object that is composed of matter but is not itself a mass of
matter''~\citep[76]{hirsch2005}.

Hirsch then adopts the perspective of $RC$.  He finds that speakers of
$DL$-English consider the sentence ``There is first something that is
$F$ and later there is something that is $G$'' to be `` (a priori
necessarily) equivalent'' to ``There is something that is first $F$
and later $G$''.  He says that ``we should make the charitable
assumption that in $DL$-English these sentences really are
equivalent''~(\citeyear[78]{hirsch2005}).

Given the mereological axioms that $DL$ has adopted, it is
uncontroversially true that whenever there is something that is pink,
then something that is brown, they fuse to create something that is
first pink and then brown.  Given $RC$'s doctrines, it is also
true that everything (every physical thing) he claims to exist is a
mass of matter.

Having completing his `charitable' interpretation, Hirsch applies his
version of~\ref{v} and claims that these truth-equivalent propositions
{\em mean} the same thing.  He claims that $DL$ means the same thing
by ``There is first something that is $F$ and later there is something
that is $G$'' and by ``There is something that is first $F$ and later
$G$.''  He also claims that when $RC$ says ``something'' he means
``something that is either a mass of matter or is not composed of
matter''~(\citeyear[76]{hirsch2005}).

Hirsch concludes that when $DL$ says ``There is something here that is
first pink and then brown'', he should be taken to mean that there is
something that is pink and then something that is brown.  And $RC$ can
agree with that.  Hirsch also claims that when Chisholm says that
``there is nothing here that is first pink and then brown'' he means
that there is no mass of matter that is first pink and then
brown.  $DL$ will not deny that.  So Hirsch concludes that $DL$ and
$RC$ are engaged in a verbal dispute; they are simply talking past
each other.

Hirsch's analysis of $DL$ is dubious.  $DL$ will of course admit that
these sentences are truth-conditionally equivalent, but we can imagine
him saying ``Do they mean the same thing?  Well, no.  The second
proposition entails that there is one thing that is pink then brown;
the first does not (in fact, it generates an implicature in the other
direction).''

If that seems dubious, Hirsch's analysis of $RC$ seems downright
false.  ``Does `something' {\em mean} `something that is a mass of
matter or not composed of matter'?''  $RC$ might ask.  ``Of course
not!  If I meant that, then by ``there is not something that is not a
mass of matter'' all I would mean would be ``there is not something
that is a mass of matter that is not a mass of matter''.  That's
trivially true, and thoroughly uninteresting.  But I'm not speaking in
tautologies; I'm expounding a controversial metaphysical thesis;
namely, that {\em everything that exists is either a mass of matter or
  is not composed of matter}.  Only after having done some rigorous
metaphysics can we affirm that `Something exists' is true iff `Some
mass of matter or immaterial object exists'.  That claim reports a
discovery about the world, not about what I mean by my words.''

\subsection{Hostile interpretations}
Hirsch's claim, that ontological disputes like the above are merely
verbal, relies on a controversial theory of meaning.  If a
truth-conditional theory of meaning is correct (or largely so), then
Hirsch's interpretations of the disputing metaphysicians would also be
correct.  But there would still be a sense in which the verbal
disputes as to whether there are chairs differs from the disputes as
to whether glasses are cups.  Above I said that two people arguing
over whether glasses are cups will agree that they do not mean the
same thing by `cup'.  They will agree that they are engaged in a
verbal dispute.

The metaphysicians are not so cooperative.  Even after Hirsch has
diagnosed their dispute as verbal, the disputants maintain that they
are {\em not} engaged in a verbal dispute.  Hirsch's interpretations
are therefore \emph{hostile}.  They diverge substantially from the
expectations of the speakers.  Even if our ontologists tell Hirsch
``we're \emph{not} engaged in a verbal dispute'', Hirsch will be
unmoved:

\begin{squote}
The presumption of charity is supposed to be an a priori principle
that is partially constitutive of linguistic meaning.  Insofar as the
disputing ontologists assert the sentence, ``We are not engaged in a
verbal dispute,'' this sentence will figure, together with all the
other asserted sentences, in arriving at the most charitable
interpretation.  I would suspect that meta-level, quasi-technical
(self-aggrandizing) assertions probably have low priority as
supplicants for charity.  In any case, it can't be seriously suggested
that the charitable presumption in favor of the correctness of this
one assertion threatens to trump the presumption in favor of all of
the other assertions made by the ontologists (\citeyear{hirsch2008}).
\end{squote}

As long as Hirsch can produce truth-condition assignments that make
the relevant assertions of both sides true, there seems to be nothing
they can do to convince him that they are having a real argument
(other than convince him that his truth-conditional theory of meaning
is mistaken).

\subsection{Ontologese}
Or isn't there?  Sider's strategy for dealing with Hirsch is to
stipulate---in concert with other metaphysicians---that they use
`$\exists$' and `$\forall$' in a special sense:

\begin{squote}
{[}The philosophers{]} should stipulate that their quantifiers are to be
understood as theoretical terms (and so are not subject to the same
level of metasemantic pressure from charity that governs terms like
`sofa' and `game') that stand for whatever joint-carving notion is in
the vicinity (\citeyear[9]{sider2011b}).
\end{squote}

By explicitly \emph{intending} to mean by `$\exists$' and `$\forall$'
whatever `joint-carving' notions are `in the vicinity', Sider hopes to
evade Hirsch's argument from interpretive charity.  The idea is that
charity can be put on hold.  Sider hopes that when {\em he} says
``there are no chairs'', it will be true (if it is true) only because
a ``joint-carving'' quantifier does not range over chairs.

\subsection{Sider's retreat}
The first thing to notice about Sider's argument is that it involves
conceding that {\em Hirsch's theory of meaning is correct}.  Sider
implicitly grants that some sort of truth-conditional theory of
meaning is true.  There is no other reason why Sider should feel the
need to build a language with the specific purpose of avoiding
interpretive charity.  As I pointed out above, Hirsch's
conclusion---that metaphysical disputes are verbal---requires not only
a principle of interpretive charity, but a truth-conditional theory of
meaning---something like \ref{v}.

A principle of charity alone cannot secure Hirsch's conclusion.  When
Hirsch charitably interprets the metaphysicians, it is because his
`charitable' interpretations involve only the assignment of
truth-conditions that he can claim that the metaphysicians mean
different things.  We say above that Hirsch recommends David Lewis and
other universalists to interpret Roderick Chisholm, when he says
`something' to mean `something that is a mass of matter or not
composed of matter'.  This is a bizarre and uncharitable
interpretation, {\em unless} we assume a truth-conditional theory of
meaning.

I am not sure that a truth-conditional theory of meaning is correct.
There are a number of powerful arguments against such theories, and I
do not know how to argue against them.  Sider makes an unnecessary
concession when he allows Hirsch to build his thesis on a
truth-conditional theory of meaning.

That being said, I simply do not understand how `Ontologese' is
supposed to work.

\subsection{Intention and reference}
Sider places a heavy emphasis on his notion of `joint-carving'.  He
seems to think that one can, when one uses a word, {\em intend} to use
it to `pick out' a `joint-carving' meaning.  His example involving
simultaneity was, I think, supposed to show this.  He says elsewhere
that

\begin{squote}
a highly joint-carving interpretation can ``trump'' the superior
charity of rival interpretations.  Now, this prediction is in some
cases correct, especially for ``theoretical'' terms---terms that are,
intuitively, {\em intended} to stand for joint-carving meanings.  When
a term like `mass' is introduced in physics, it's intended to stand
for a fundamental physical magnitude, and so if there's a
joint-carving property in the vicinity then that property is meant by
`mass', even if it doesn't quite fit the physicists' theory of `mass'
(\citeyear[32]{sider2011d}).
\end{squote}

It's not clear to me exactly what it means to intend to have one's
words interpreted in a `highly joint-carving' way.  If one intends to
be understood in a certain way, then generally one has some idea as to
what this amounts to.

I can say ``I intend to use `Harriet' to designate Harriet Tubman.''
If I go on to say things like ``Harriet was a very brave woman'', my
audience (if they know who Harriet Tubman is) recognize my intention
to refer to Harriet Tubman; communication is thereby secured.  But
suppose instead that I say ``I intend to use `Heerriet' to designate
whichever person named `Harriet' is nearest to me.''  I can't thereby
talk about Heerriet.  I can certainly say things like ``I wonder if
Heerriet remembered to call her mother on her birthday'', but neither
I nor my audience have any idea who I'm talking about (unless we both
happen to know who Heerriet currently is).  Likewise, I simply don't
understand how Sider expects to know what he's talking about when he
uses his `new' quantifier phrases.

Suppose this exchange were to occur:

\stage{Sophists}{(in unison)}{We do not mean `love' such that it is
  obviously true that two people can fall in love.  Rather, we intend
  to use `love' such that it is an open question whether or not two
  people can fall in love.  Let our uses of the word pick out whatever
  `joint-carving' notion is in the vicinity.}

\stage{Sophist 1}{}{Two people can fall in love!}

\stage{Sophist 2}{}{Two people cannot fall in love!}

\stage{Sophist 3}{}{We're not engaged in a verbal dispute!}

I'm inclined to say that our sophists simply do not know what they are
talking about.  Unless they have {\em some} idea of what they (and
each other) mean by `love', this is not even a verbal dispute---it is
nonsense.  (Of course, if we continue to listen we may discover that
it {\em is} just a verbal dispute.  One of the sophists might declare that
true love can only occur in stories, or something.)

I'm dubious, therefore, that one can simply {\em stipulate} that they
mean ``the most `joint-carving' meaning in the area, whatever it
happens to be''.  One must have {\em some} idea of what one is talking
about, at the risk of talking nonsense.  This brings us back to the
ontologists:

\stage{Ontologists}{(in unison)}{We do not mean `$\exists$' and
  `$\forall$' such that it is obviously true that there exist chairs.
  Rather, we intend to use `$\exists$' and `$\forall$' such that it is
  an open question whether or not chairs exist.  Let our uses of the
  quantifiers pick out whatever `joint-carving' notion is in the
  vicinity.}

\stage{Ontologist 1}{}{There exist chairs!}

\stage{Ontologist 2}{}{There do not exist chairs!}

\stage{Ontologist 3}{}{We're not engaged in a verbal dispute!}

If these ontologists are to be taken as saying anything at all, they
must have some idea what they mean by `$\exists$' and `$\forall$'.
They cannot simply appeal to whatever `joint-carving' notion is in the
vicinity.  That would be analogous to me stipulating I will use `John'
to refer to the physically closest person named `John'.  I simply
cannot do this, unless I have some idea who that is.  Likewise, the
ontologists must have some idea of what they mean by their
quantifiers.  And according to Hirsch, if the ontologists are using
the quantifiers with some intelligible meaning, figuring out what that
meaning is requires interpretive charity.

Suppose that, having made their resolution to speak Ontologese, the
ontologists go on much as before, taking sides over whether or not
there are chairs and people.  Despite their protestations, Hirsch will
conclude that what they mean by their quantifiers has not changed:

\begin{squote}
If one philosopher talks like a typical organicist and another talks
like a typical common sense ontologist then, despite their
protestations that they are both speaking the philosophically best
language [Ontologese], it's probably more plausible to hold that the
first is speaking O$\ast$-English and the second is speaking
C$\ast$-English (\citeyear[12]{hirsch2008}).
\end{squote}

Again, because Hirsch can continue to make charitable truth-condition
assignments, he will conclude that, unlike the scientists' debate over
simultaneity, the ontological disputes remain verbal.

If the ontologists, having announced their intention to speak
Ontologese, no longer sound like organicists and nihilists but rather
use the quantifiers in wholly unexpected ways, then Hirsch may simply
decide that they no longer know what they are talking about.  Like my
attempted reference to the closest `John', the ontologists have no
clear idea of what they mean.  They are no longer making sense.

\subsection{An English ontology}
I have argued that Sider makes two mistakes in his reply to Hirsch.
The first is to concede to Hirsch a controversial theory of meaning.
The second is to claim that we can intend to refer to things---like
Heerriet and `joint-carving' quantifiers---that we know nothing about.
 When Sider attempts to use the `Ontologese' quantifiers, he does not
 actually know what he is talking about.

I do not subscribe to a truth-conditional theory of meaning, so I
reject Hirsch's motivation for interpreting metaphysical utterances as
being spoken in different languages.  I think it is most charitable
for the philosophers to interpret each other as speaking English.  I
have (or so I claim) been writing in English this whole time.  When I
said ``there are chairs'', that was part of an English sentence.  If
``there are chairs'' is true in English, then there are chairs.

Sider was attempting to use his quantifiers in a wholly mysterious
way.  I, on the other hand, am using them with their ordinary English
meaning.  When I use quantifier phrases like ``there are'' in ``there
are chairs'', I know perfectly well what {\em I} mean.  Regarding
chairs, there are some (many) of them.

%% I'm not sure that I fully understand what Sider is trying to do here.
%% When at the beginning of this Introduction I said ``there are
%% chairs'', I was writing in English.  The only language with which I
%% have any proficiency is English; I don't know how to make that claim
%% in any other language.  If what I said was true---if ``there are
%% chairs'' is true in English---then there are chairs.  That is the
%% claim I want to defend.  If Sider is willing to give it to me, then my
%% job is done.

\section{Lessons}
Henceforth I will assume that the debate over whether there are chairs
is conducted in English.  But even some philosophers who agree to this
will deny that ``there are chairs'' is a conceptual truth.  Nor will
they admit that it is obviously true (they are denying that it is
true, after all).  Such philosophers will object that so far, the only
objection I have raised against the view that there are no chairs is
that I cannot bring myself to believe it.

However, there is another reason to resist their conclusions, one that
is independent of my inability to believe that there are no chairs.
As we will see in Section \ref{stroud}, philosophers who deny that
there are chairs have a difficult time explaining why we believe that
there are chairs.  To the extent that they cannot explain why we hold
this belief (and others concerning ordinary things), we have reason to
suspect that their denials might be unfounded.

Moreover, just as the philosophers cannot deny that I believe that
there are chairs, they cannot (they should not) deny that it {\em
  seems} obviously true that there are chairs.  It is `intuitively
true' that there are chairs.  The philosophers accordingly have ways
of undermining this intuition.  They try to show that although the
unreflective individual might think that it is obvious that there are
chairs, careful consideration will show her that it is far from
obvious.  After I try to show that these philosophers cannot explain
why we believe that there are chairs, I will then explain why we
should resist their attempts to undermine the obviousness of the fact
that there are chairs.

\ifstandalone
\end{spacing}
\bibliography{everything}
\bibliographystyle{ChicagoReedweb}
\fi
\end{document}
