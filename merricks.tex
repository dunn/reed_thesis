\documentclass[11pt]{article}
\usepackage{standalone} \newif\ifstandlone \standalonetrue
\usepackage[left=1.75in, right=1.75in, top=1.25in, bottom=1.25in]{geometry}
\geometry{letterpaper}
\usepackage{graphicx}
\usepackage{enumitem}
%\usepackage{amssymb}
\usepackage{amsmath}
\usepackage{epstopdf}
\usepackage{verbatim}
\usepackage{setspace}
\usepackage{natbib}
\setcitestyle{aysep={}}
\usepackage{hyperref}
\usepackage{url}
\synctex=1

\DeclareSymbolFont{symbolsC}{U}{txsyc}{m}{n}
\DeclareMathSymbol{\strictif}{\mathrel}{symbolsC}{74}
\DeclareMathSymbol{\boxright}{\mathrel}{symbolsC}{128}

\newenvironment{squote}{%
\begin{spacing}{1}
       	\begin{list}{}{%
\setlength{\labelwidth}{0pt}%
\rightmargin\leftmargin%
}
\item\relax
}{%
\end{list}%
\end{spacing}
}

\title{``Nearly as good as true''}
\author{Alexander A. Dunn}
\begin{document}
\ifstandalone
\maketitle
\begin{spacing}{1.25}
\fi

\noindent Trenton Merricks' explanation of why we believe that there
are chairs is much more plausible that van Inwagen's.  However, the
plausibility of Merricks' explanation relies on his ability to
undermine our reasons to think that it is obviously true that there
are chairs.  I will argue that his attempt to show that ``there are
chairs'' is not obviously true fails; consequently, his explanation of
why we believe that there are chairs fails too.  

The lesson of van Inwagen's nihilism was that the Special Composition
Question is a poor question.  The lesson of Merricks' nihilism is that
`composition' is itself a poor concept.  Talking about when things do
or do not `compose' another things is an unhelpful and confusing way
to talking, which tends to make nihilism seem more plausible than it
is.

\section{Merricks and the indispensability of ordinary concepts}
\label{merricks}
Trenton Merricks comes to the same metaphysical conclusions as does
van Inwagen.  That is, he claims that there are no physical objects
other than human beings.  However, he comes to this conclusion through
a different path of reasoning.  (I have not studied these arguments
yet, so I will skip that part for now.)

Despite the fact that Merricks has a different motivation for his
nihilism, we can pose the same question to him as we posed to van
Inwagen.  Why, if there are no chairs, do we believe that there are
chairs?  Happily, Merricks addresses our concern.  Even more happily,
he has a better explanation than van Inwagen.  But sadly, even if his
explanation is good, it will probably be undermined by Unger's
arguments for nihilism.

\subsection{Nearly as good as true}
\label{near}
Merricks claims that `folk' beliefs, such as the belief that there are
chairs, are false, but nonetheless are {\em nearly as good as true}.
What does this mean?

\begin{squote}
People who believe in unicorns [or ghosts] are few and far between.
And those few are generally unjustified.  On the other hand, people
who believe in statues are legion.  And they are generally justified
in so believing.  Given the truth of eliminativism [what I have been
  calling nihilism], we might ask {\em why} the belief in statues is
more common, and more commonly justified, than the belief in unicorns.

The answer is that statue beliefs are nearly as good as true.  For, so
I claim here, {\em atoms arranged statuewise} often play a key role in
producing, and grounding the justification of, the belief that statues
exist.  In general, a false belief's being nearly as good as true
explains how {\em reasonable} people come to hold it.  And, relatedly,
its being nearly as good as true can ground its justification.
Because the belief that unicorns exist is not nearly as good as true
(i.e.\ because there are no things arranged unicornwise), there is no
similar explanation of its production or similar reason to think it is
justified (\citeyear[171--172]{merricks2001a}).
\end{squote}

To say that something is ``nearly as good as true'' seems to be
equivalent to saying that it is `loosely true', or `true for practical
purposes'.  In each case, the proposition in question is false, but it
is somehow close enough to the truth for a given purpose or situation.
For example, suppose we have decided to buy a fake holiday tree for
the holidays this year.  We are looking at a number of different fake
trees.  I point to one and say ``that is a nice tree''.  What I have
said is false; that is not a tree.  It is a fake tree.  But what I
mean---and what my audience recognizes me to mean---is that it is a
nice {\em fake} tree.  We both know that we are looking at fake trees;
there is no point qualifying every use of `tree' with `fake'.  When I
say ``that is a nice tree'', therefore, what I say is quite sufficient
to allow for successful communication. despite being false.  Merricks
claims that propositions expressed by things like ``there are chairs''
are also loosely true.  They are false, but are nonetheless good
enough for certain purposes.

Initially, this seems like a bizarre claim.  After all, Merricks is
claiming that chairs {\em necessarily} do not exist.  According to
Merricks, ``chairs exist'', given its current meaning, could {\em
  never} be true.  If the proposition expressed by ``chairs exist'' is
necessarily false, how could it nonetheless be ``nearly as good as
true''?

\subsection{The necessary connection}
\label{connection}
Merricks' argument relies on a very close conceptual connection
between ``chair'' and ``chairwise'' (and likewise for all ordinary
terms).  Despite claiming that chairs are impossible, Merricks admits
that we understand perfectly what chairs {\em would} be, if they
existed.  Because we understand the concept of `chair', we can
recognize {\em actually existing} things that are arranged
`chairwise':

\begin{squote}
The folk concept of \emph{statue} plays a role in determining which
atomic arrangements are statuewise. I would even go so far as to say
that if \emph{being arranged statuewise} were not derivative upon
folk-ontological concepts\,\ldots something would be amiss
(\citeyear[8]{merricks2001a}).
\end{squote}

For Merricks, to know what things are actually arranged statue- or
chairwise requires knowing what things would compose a chair, if
chairs were possible:

\begin{squote}
Atoms are \emph{arranged statuewise} if and only if they both have the
properties and also stand in the relations to microscopica upon which,
if statues existed, those atoms' \emph{composing a statue} would
non-trivially supervene (\citeyear[4]{merricks2001a}).
\end{squote}

When we look at Peter Unger's arguments for nihilism, we will see that
this close conceptual connection between terms like `chairwise' and
`chair' will cause trouble for Merricks.  But is Merricks' position
plausible anyway?  Does he have a good explanation of why we believe
in chairs?

Merricks' explanation is, at least, plausible.  One reason is the
structure of the explanation.  Recall that our explanation of why we
believe that there are chairs (or statues) is that, first, there are
chairs, and, second, we see that there are chairs (or learn that there
are chairs through a similarly reliable mechanism).  

Merricks' definition of `nearly as good as true' allows us to produce
a parallel explanation:

\begin{squote}
Any folk-ontological claim of the form `F exists' is \emph{nearly as
  good as true} if and only if (i) `F exists' is false and (ii) there
are things arranged F-wise. So, for example, `the statue \emph{David}
exist' is nearly as good as true because (it is false and) there are
some things arranged Davidwise (\citeyear[171]{merricks2001a}).
\end{squote}

We may now say on behalf of Merricks that we believe that there are
chairs (and statues) because, first, there are things arranged
chairwise and, second, we see that there are things arranged
chairwise.

The structure of the two explanations is analogous, but there is an
apparent disanalogy in the content of the two.  The disanalogy does
not favor Merricks.  For it is easy enough to see why there being
chairs, and us seeing that there are chairs, would cause us to believe
that there are chairs.  But it is less obvious why there being things
arranged chairwise, and us seeing that there are things arranged
chairwise, would cause us to believe {\em not} that there are things
arranged chairwise, but that there are {\em chairs}.

(While it is certainly true that we believe that there are chairs, I
am not sure if all or even most of us {\em also} believe that there
are things arranged chairwise.  Let us suppose for now that we do.)

The close conceptual connection between `chair' and `chairwise' is
very important for Merricks.  It is this {\em connection} that is
doing the explanatory work.  The only thing that can explain why there
being things arranged chairwise would cause us to believe that there
are chairs is this connection between the concepts.  The existence of
things arranged chairwise, and the belief that there are things
arranged chairwise, is supposed to cause the {\em additional} belief
that there are chairs.  How does this happen?

I believe that Merricks' answer would go something like this: certain
arrangements of things---chairwise arrangements, statuewise
arrangements, and all ordinary arrangements---play important roles in
our lives.  These arrangements of things are of interest to us, so we
have developed words that allow us to refer to them.  For whatever
reason---sociological, psychological, or otherwise---we think of each
arrangement as a single thing, rather than as things.  Words like
`chair' and `statue' reflect this (incorrect) view of the world.

This is more than Merricks says himself.  I have not found a passage
in which he explicitly describes the nature of the conceptual
connection between concepts like `chair' and `chairwise'.  But take
this quote:

\begin{squote}
{[}Consider{]} the claim that the atoms arranged my-neighbour's-dogwise
and the-top-half-of-the-tree-in-my-backyardwise compose an
object\ldots{}it won't do to defend this claim with nothing more than `I
can \emph{just see} the object composed of the atoms arranged
dog-and-treetopwise'. Part of why this won't do, presumably, is that
one's visual evidence would be the same \emph{whether or not} those
atoms composed something (\citeyear[8--9]{merricks2001a}).
\end{squote}

He seems to be pointing out here that whether we see an arrangement of
things as composing an object or not depends more on our own interests
than features of the things themselves.  We have words for chairs and
statues because things arranged chairwise and statuewise interest us.
We don't have a word for things arranged ``my-neighbor's-dogwise and
the-top-half-of-the-tree-in-my-backyardwise'' because such an
arrangement does not hold much interest for us.  But each of these
arrangements exist, and it seems arbitrary to say that the chairwise
and statuewise arrangements compose chairs and statues while the other
arrangement composes nothing.

Merricks would explain why we believe that there are things arranged
my-neighbor's-dogwise and the-top-half-of-the-tree-in-my-backyardwise
thus: there are things arranged my-neighbor's-dogwise and
the-top-half-of-the-tree-in-my-backyardwise, and we see that there are
things so arranged.  This is exactly the same explanation that I would
give.

Now Merricks explains why we believe that there are chairs thus: there
are things arranged chairwise, and we see that there are things
arranged chairwise.  {\em And incidentally, due to our own human
  peculiarities, we have found it convenient to refer to and think
  about things arranged chairwise as if they were single objects}.

\section{Dogbushes}
\label{dogbush}
This is a somewhat plausible explanation of why we would belief that
there are chairs if there were not.  It is certainly much better than
van Inwagen's.  But I think that it fails.  I think that when we look
closer at Merricks' attempts to motivate nihilism, we will see that
they do not support nihilism at all.  If anything they support a
version of {\em universalism}.

Merricks observes that one might object to nihilism simply by saying,
``I just {\em see} the chair!''  He claims that if this objection
moves us, we should think about this analogous objection, which he
finds much less moving:

\begin{squote}
{[}Consider{]} the claim that the atoms arranged my-neighbour's-dogwise
and the-top-half-of-the-tree-in-my-backyardwise compose an
object\ldots{}it won't do to defend this claim with nothing more than `I
can \emph{just see} the object composed of the atoms arranged
dog-and-treetopwise'. Part of why this won't do, presumably, is that
one's visual evidence would be the same \emph{whether or not} those
atoms composed something (\citeyear[8--9]{merricks2001a}).
\end{squote}

It does indeed seem plausible to say that the top half of a tree and
my neighbor's dog do not `compose' anything.  But I think this is
incorrect, and I think we find it plausible only because of the word
`compose'.

Recall the bliger story that van Inwagen used to motivate his version
of nihilism (section \ref{prop-ont}).  A bliger was supposed to be
four monkeys, an owl, and a sloth, who arrange themselves into a
temporary symbiotic configuration.  Van Inwagen thought we would agree
that bligers did not exist.  Like Merricks, he brought in the language
of `composition' to make his claim more plausible.  Van Inwagen says
that it is not true that ``six animals arranged in bliger fashion
compose anything, and that is what I mean to deny when I say that
there are no bligers'' \citeyearpar[104]{inwagen1995}.

But as we saw, it is simply false that there are no bligers:

\begin{squote}
\ldots {\em of course} there are bligers in [van Inwagen's] story.
Bligers are what the story is about.  The zoologists do not report
that there are no bligers.  Rather they tell us what a bliger is.
They explain that a bliger is not a single large carnivorous animal
but a transient symbiotic union of six animals
\citep[704]{rosenberg1993}.
\end{squote}

Once we drop the language of composition, it becomes obviously true
that there are bligers.  Van Inwagen relied too heavily on the
metaphor of `composition', and as a result concluded that ``there are
bligers'' was not only less than obviously true, but actually false.

Merricks makes the same mistake in his passage above.  Following van
Inwagen, let's give our `composite' object a name.  For simplicity's
sake, I'm also going to change the example slightly.  Let `dogbush'
designate a tree that is within 3 meters of a dog.  Merricks would now
ask us if there is anything `composed' of a tree and a dog when they
are within 3 meters of each other.  But let's drop the language of
`composition' and simply ask, ``are there dogbushes?''

{\em Of course there are dogbushes.}  There are trees, and there are
dogs within 3 meters of those trees.  That's all it takes to make
``there are dogbushes'' true.  How could ``there are dogbushes'' be
false; how could there be no dogbushes?  Are there {\em no} dogs
within 3 meters of a tree?  There obviously are, and so it is
obviously true that there are dogbushes.  

Merricks relies on the idea of `composition' to undermine the
obviousness of the fact that there are dogbushes.  He then asks why,
given we are supposed to have admitted that a tree and a dog do not
compose anything, why we should think that things arranged chairwise
compose anything.  If the tree and the dog do not stand in the correct
relations to each other such that `composition' takes place, why would
things arranged chairwise stand in such a relation?

But again, if we drop the language of `composition', things become
clearer.  For {\em of course} there are chairs.  There are things
arranged chairwise.  That's all it takes to make ``there are chairs''
true.  How could ``there are chair'' be false; how could there be no
chairs?  Are there {\em no} things arranged chairwise?  There
obviously are, and so it is obviously true that there are chairs.

\subsection{The meaning and truth-conditions of `chair'}
\label{meaning}
I think I know how Merricks would object.  He might take me to be
accusing him of contradiction; for how can there be things arranged
chairwise but no chairs?  If there is the one, there is the other.
His response to this sort of objection is as follows:

\begin{squote}
`There are married bachelors' is no explicitly formally contradictory,
  but it is contradictory in some quite straightforward sense.  And
  one might object that `there are atoms arranged statuewise but no
  statues' is contradictory in the same way.  For as `bachelor' means
  someone who is, among other things, unmarried, so---the objector
  insists---`there are [composite] statues' {\em just means} that
  there are some things arranged statuewise.  Because of its
  contradictory nature, we should not take seriously an ontology
  according to which there are married bachelors.  Likewise, this
  objection concludes, we should not take seriously the
  eliminativist's ontology, with its atoms arranged statuewise but no
  statues \citeyearpar[13]{merricks2001a}.
\end{squote}

Merricks points out that ``\,`There are statues' does {\em not} mean
only that there are some things arranged statuewise\,\ldots this is
simply not a plausible claim about ordinary meaning''
\citeyearpar[13]{merricks2001a}.  And this seems quite true.  But I do
not need to claim that ``there are statues'' (or chairs) {\em means}
that there are things arranged statuewise (or chairwise).  All I need
to claim is that the following is true:

\begin{squote}
`there are chairs' is true if and only if `there are things arranged
  chairwise' is true.
\end{squote}

(If, as some claim, meaning is reducible to truth-conditions, then I
would be committed to the additional claim that `there are chairs' and
`there are things arranged chairwise' mean the same thing.  But if
Merricks does not hold this thesis about meaning, then he has no
grounds to resist the distinction between two propositions meaning the
same thing and their being truth-conditionally equivalent.  It is only
the latter that I claim.)

Nihilists like Trenton Merricks have assumed that what is required for
``there are chairs'' to be true is something more than what is
required for ``there are things chairwise'' to be true.  But this is a
mistake.  Ted Sider makes this mistake in a recent paper on parthood:

\begin{squote}
Consider Nihilo, god and creator of a world comprised solely of
subatomic particles.  On the first day Nihilo creates some particles
and arranges them in beautiful but lifeless patterns.  But he becomes
lonely, so on the second day he creates some minions (or rather,
particles arranged minion-wise).  On the third day he tries to teach
his minions to speak.  But this goes badly.  The minions aren't very
bright, and are slow to catch on to Nihilo's talk of subatomic
particles and their physical states.  So on the fourth day he teaches
them an easier way to speak.  Whenever an electron is bonded (in a
certain way) to a proton, he teaches them to say ``there is a hydrogen
atom''; whenever some subatomic particles are arranged chairwise he
teaches them to say ``there is a chair'', and so on.  (Pretend that
electrons and protons have no proper parts.)

When the minions utter sentences like ``there is a hydrogen atom'', do
they speak falsely?  They do if their language is the same as the
language I used to describe the example, since I described Nihilo as
having created a world comprised solely of subatomic particles
\citeyearpar[7]{sider2011c}.
\end{squote}

Sider seems to be thinking that he can stipulate that the
`metaphysical laws' of his imagined world make it impossible for there
to be minions.  But he has clearly not succeeded, for there are
minions right there in his story!  Nihilo created minions.  In
Nihilo's world, ``there are minions'' is true.  So is ``there are
hydrogen atoms''.  When the minions utter such sentences, what they
say is true.  It is true even if their language is the same language
as Sider used to describe the example (the language is English; recall
section \ref{english} above).  For there to be minions, nothing is
required over and above there being things arranged minionwise.
``There are minions'' is true iff ``there are things arranged
minionwise'' is true.

Above I claimed that Merricks' explanation of why we believe that
there are chairs is this: there are things arranged chairwise, and we
see that there are things arranged chairwise.  {\em And incidentally,
  due to our own human peculiarities, we have found it convenient to
  refer to and think about things arranged chairwise as if they were
  single objects}.  Merricks argued that just because things arranged
chairwise interest us, we should not therefore suppose that there are
chairs.  What interests us should not be a guide to what exists.  But
now there is an obvious counter against this move by Merricks.  Just
because dogbushes do {\em not} interest us, we should not therefore
suppose that there are not dogbushes.  (What interests us should not
be a guide to what exists.)

\subsection{Universalism}
\label{universalism}
I said in section \ref{scq-ans} that I would postpone discussion of
universalism until later.  It it time now to discuss it.  For I seem
to be committing myself to some version of universalism.  The
formulation I cited above was this:

\begin{description}
\item[Universalism] Necessarily, for any $x$s, there is an object
  composed of the $x$s iff no two of the $x$s overlap
  \citep[227]{markosian1998a}.
\end{description}

Much of the resistance to this thesis (at least, much of my early
resistance) probably stems from its wording; as I argued above, the
language of `composition' can lead us to deny obvious truths, such as
that there are chairs and dogbushes.  If we are aware of this and
ignore the troublesome word `composition', then universalism seems
extremely plausible, much more so than nihilism.

Ned Markosian's objection to universalism provides another example of
how the language of `composition' obscures the plausibility of
universalism:

\begin{squote}
There is what seems to me a fatal objection to Universalism:
Universalism entails that there are far more composite objects than
common sense intuitions can allow.  To give just one example,
Universalism entails that the following sentence is true:\,\ldots
There is an object composed of (i) London Bridge, (ii) a certain
sub-atomic particle located far beneath the surface of the moon, and
(iii) Cal Ripken, Jr.  My intuitions tell me that there is no such
object, and I suspect that the intuitions of the man on the street
would agree with mine on this point \citeyearpar[228]{markosian1998a}.
\end{squote}

I suspect that Markosian's intuitions are led astray because he relies
so heavily on the language of `composition'.  As in the case of the
dog and the tree, it does seem unintuitive that the tree and dog
`compose' something else.  But it is obvious that there are
dogbushes.  Let us therefore drop the language of `composition' from
Markosian's example and see if that makes things clearer.

Let `Lumpkin Junior' designate the London Bridge, a particle in the
moon, and Cal Ripkin, Jr.  Does Lumpkin Junior exist?

Let us introduce another term so as to bring the analogy closer.  A
`lumpkin' is the London Bridge, a sub-atomic particle in the moon, and
a retired major league baseball player.  Now we can ask ``are there
lumpkins?''  And, as in the case of the dogbush, {\em of course there
  are}.  If there were no lumpkins, there would have to be either no
former ball-players, no London Bridge, or no particles in the moon.
Given that there are all these things, there are obviously lumpkins.
Lumpkin Junior is one of many lumpkins.

I cannot resist including another example, partly for its silliness
and partly in the hope that it will bring out further the way in which
terms like `composition' are misleading.  The Blues Brothers have a
song called ``Rubber Biscuit''.  In it, they refer to a `wish
sandwich'.  ``A wish sandwich,'' they say, ``is the kind of a sandwich
where you have two slices of bread and {\em wish} you had some meat.''
This term having been introduced, one can truthfully say, ``I had a
wish sandwich the other day.''  All that is required for this to be
true is that she had two slices of bread and wished she had some meat.
There was then a wish sandwich.  Should we say that there was some
`composite' thing, `composed of' two slices of bread and a wish?  We
can talk this way, I suppose, but it serves little purpose but to
confuse ourselves.

\subsection{Deflationary metaphysics}
\label{deflate}
Kathrin Koslicki has an interesting objection to universalist theses
such as the one I appear committed to.  Her objection amounts to this:
if every `collection' of objects (such as the London Bridge, a
particle in the moon, and Cal Ripkin, Jr.) is a thing in its own
right, then metaphysics becomes uninteresting.  There is no longer any
debate about whether chairs or dogbushes are more `real' or have a
stronger claim to existence.  They both (obviously) exist, and the
difference between chairs and dogbushes is not ontological but
conceptual: `chair' is more embedded in our talk, and so chairs have
greater importance to {\em us}.  But metaphysically, or ontologically,
chairs and dogbushes are on the same level.  There is no sense in
which chairs exist and dogbushes do not.

In the quoted material below, Koslicki is criticizing a version of
four-dimensionalism that Sider has previously defended.  Sider's
position was that any collection of objects-at-times is a thing in its
own right.  Sider calls these things `fusions'.  For example, a chair
is a fusion of a large number of {\em temporal part} of things (wood
molecules, or atoms, or simples).  Each thing (wood molecule, atom, or
simple) is a fusion of {\em its} temporal parts.  Each temporal part
of the chair is also a thing (a fusion).

I take no stance on whether objects have temporal parts, or rather
`endure' through time.  But Koslicki's comments are relevant to most
versions of universalism:

\begin{squote}
There is room, in Sider's theory, for {\em some} genuine ontological
disagreements: for example, the universalist, the nihilist and the
holder of the intermediary position genuinely disagree over how many
and which fusions that exist.  But the only genuine ontological
disagreements for which there is room, in Sider's world, are ones that
concern disagreements over `bare' fusions, so to speak.  What has
happened to the houses, trees, people, and cars, the familiar concrete
objects of common-sense, whose persistence this account set out to
analyze?  There are no `deep' ontological facts as to whether a given
fusion should count as a house or not\,\ldots

[By claiming that there can be genuine ontological disputes,] Sider is
guilty of a bit of false advertising: his account is really a way of
saying that, at the end of the day, there is no interesting {\em
  ontological} story to be told about the persistence of our familiar
concrete objects of common-sense; whatever there is to say about the
persistence of houses, trees, people and cars concerns the
organization of our conceptual household
\citeyearpar[124--125]{koslicki2003}.
\end{squote}

Koslicki seems to think that we ought to be able to find some
ontological difference between ``the familiar concrete objects of
common-sense'' and things like dogbushes or chairs-at-times.  But as I
remarked above, why should what interests us (familiar objects like
chairs) be a guide to what exists?  The conclusion that ``the
persistence [and other properties] of houses, trees, people and cars
concerns the organization of our conceptual household'' seems to be a
most welcome one.

In section \ref{lessons-v} I considered Jay Rosenberg's claim that
the Special Composition Question is the wrong question to be asking.
Rosenberg's position seems to be that there is {\em no} answer to the
Special Composition Question.  Rather, he thinks what it takes to
`compose' something depends on what that something is---making a chair
is not like making a pie.

I agree that the Special Composition Question is, in a sense, the
wrong question.  But this is not because I think there is no answer to
is; rather I think there is no {\em interesting} answer.  Once we set
aside misleading terms like `composition', we see that an answer much
like universalism is obviously correct.  There may be interesting
questions in the vicinity---When are we willing to call something a
chair?  What conditions must be fulfilled?---but these are, as
Koslicki observes, not ontological questions anymore.  They are
questions about our ``conceptual household.''

\section{Lessons}
What we have learned from examining Merricks' arguments is not that
there are no chairs.  What we have learned is that the language of
`composition' can fool us into thinking that there are no dogbushes
(or bligers).  Worse, it can undermine our confidence that ``there are
chairs'' is obviously true.  It may be that talk of things `composing'
other things gives the impression that `composition' is something that
things {\em do}---as if they were gathering or attaching themselves
together.  If we avoid terms like `composition', we can see that there
is no motivation for Merricks' nihilism, other than his thesis of
causal over-determination.  And given that his conclusion---that there
are no chairs---is obviously untrue, we should suspect that causal
efficacy is not required for ``there are chairs'' to be true.  Rather,
all that is required for ``there are chairs'' to be true is that
``there are things arranged chairwise'' be true.

In this section I have tentatively supported a version of
universalism.  However, in the next section we will examine a number
of powerful objections to universalism.

\ifstandalone
\end{spacing}
\bibliography{everything}
\bibliographystyle{ChicagoReedweb}
\fi
\end{document}
