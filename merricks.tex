\documentclass[11pt]{article}
\usepackage{standalone} \newif\ifstandlone \standalonetrue
\usepackage[left=1.75in, right=1.75in, top=1.25in, bottom=1.25in]{geometry}
\geometry{letterpaper}
\usepackage{graphicx}
\usepackage{enumitem}
%\usepackage{amssymb}
\usepackage{amsmath}
\usepackage{epstopdf}
\usepackage{verbatim}
\usepackage{setspace}
\usepackage{natbib}
\setcitestyle{aysep={}}
\usepackage{hyperref}
\usepackage{url}
\synctex=1

\DeclareSymbolFont{symbolsC}{U}{txsyc}{m}{n}
\DeclareMathSymbol{\strictif}{\mathrel}{symbolsC}{74}
\DeclareMathSymbol{\boxright}{\mathrel}{symbolsC}{128}

\newenvironment{squote}{%
\begin{spacing}{1}
       	\begin{list}{}{%
\setlength{\labelwidth}{0pt}%
\rightmargin\leftmargin%
}
\item\relax
}{%
\end{list}%
\end{spacing}
}

\title{``Nearly as good as true''}
\author{Alexander A. Dunn}
\begin{document}
\ifstandalone
\maketitle
\begin{spacing}{1.25}
\fi

\section{Merricks and the indispensability of ordinary concepts}
\label{merricks}
Trenton Merricks comes to the same metaphysical conclusions as does
van Inwagen.  That is, he claims that there are no physical objects
other than human beings.  However, he comes to this conclusion through
a different path of reasoning.  (I have not studied these arguments
yet, so I will skip that part for now.)

Despite the fact that Merricks has a different motivation for his
nihilism, we can pose the same question to him as we posed to van
Inwagen.  Why, if there are no chairs, do we believe that there are
chairs?  Happily, Merricks addresses our concern.  Even more happily,
he has a better explanation than van Inwagen.  But sadly, even if his
explanation is good, it will probably be undermined by Unger's
arguments for nihilism.

\subsection{Nearly as good as true}
\label{near}
Merricks claims that `folk' beliefs, such as the belief that there are
chairs, are false, but nonetheless are {\em nearly as good as true}.
What does this mean?

\begin{squote}
People who believe in unicorns [or ghosts] are few and far between.
And those few are generally unjustified.  On the other hand, people
who believe in statues are legion.  And they are generally justified
in so believing.  Given the truth of eliminativism [what I have been
  calling nihilism], we might ask {\em why} the belief in statues is
more common, and more commonly justified, than the belief in unicorns.

The answer is that statue beliefs are nearly as good as true.  For, so
I claim here, {\em atoms arranged statuewise} often play a key role in
producing, and grounding the justification of, the belief that statues
exist.  In general, a false belief's being nearly as good as true
explains how {\em reasonable} people come to hold it.  And, relatedly,
its being nearly as good as true can ground its justification.
Because the belief that unicorns exist is not nearly as good as true
(i.e.\ because there are no things arranged unicornwise), there is no
similar explanation of its production or similar reason to think it is
justified (\citeyear[171--172]{merricks2001a}).
\end{squote}

To say that something is ``nearly as good as true'' seems to be
equivalent to saying that it is `loosely true', or `true for practical
purposes'.  In each case, the proposition in question is false, but it
is somehow close enough to the truth for a given purpose or situation.
For example, suppose we have decided to buy a fake holiday tree for
the holidays this year.  We are looking at a number of different fake
trees.  I point to one and say ``that is a nice tree''.  What I have
said is false; that is not a tree.  It is a fake tree.  But what I
mean---and what my audience recognizes me to mean---is that it is a
nice {\em fake} tree.  We both know that we are looking at fake trees;
there is no point qualifying every use of `tree' with `fake'.  When I
say ``that is a nice tree'', therefore, what I say is quite sufficient
to allow for successful communication. despite being false.  Merricks
claims that propositions expressed by things like ``there are chairs''
are also loosely true.  They are false, but are nonetheless good
enough for certain purposes.

Initially, this seems like a bizarre claim.  After all, Merricks is
claiming that chairs {\em necessarily} do not exist.  According to
Merricks, ``chairs exist'', given its current meaning, could {\em
  never} be true.  If the proposition expressed by ``chairs exist'' is
necessarily false, how could it nonetheless be ``nearly as good as
true''?

\subsection{The necessary connection}
\label{connection}
Merricks' argument relies on a very close conceptual connection
between ``chair'' and ``chairwise'' (and likewise for all ordinary
terms).  Despite claiming that chairs are impossible, Merricks admits
that we understand perfectly what chairs {\em would} be, if they
existed.  Because we understand the concept of `chair', we can
recognize {\em actually existing} things that are arranged
`chairwise':

\begin{quote}
The folk concept of \emph{statue} plays a role in determining which
atomic arrangements are statuewise. I would even go so far as to say
that if \emph{being arranged statuewise} were not derivative upon
folk-ontological concepts\,\ldots something would be amiss
(\citeyear[8]{merricks2001a}).
\end{quote}

For Merricks, to know what things are actually arranged statue- or
chairwise requires knowing what things would compose a chair, if
chairs were possible:

\begin{quote}
Atoms are \emph{arranged statuewise} if and only if they both have the
properties and also stand in the relations to microscopica upon which,
if statues existed, those atoms' \emph{composing a statue} would
non-trivially supervene (\citeyear[4]{merricks2001a}).
\end{quote}

When we look at Peter Unger's arguments for nihilism, we will see that
this close conceptual connection between terms like `chairwise' and
`chair' will cause trouble for Merricks.  But is Merricks' position
plausible anyway?  Does he have a good explanation of why we believe
in chairs?

Merricks' explanation is, at least, plausible.  One reason is the
structure of the explanation.  Recall that our explanation of why we
believe that there are chairs (or statues) is that, first, there are
chairs, and, second, we see that there are chairs (or learn that there
are chairs through a similarly reliable mechanism).  

Merricks' definition of `nearly as good as true' allows us to produce
a parallel explanation:

\begin{quote}
Any folk-ontological claim of the form `F exists' is \emph{nearly as
  good as true} if and only if (i) `F exists' is false and (ii) there
are things arranged F-wise. So, for example, `the statue \emph{David}
exist' is nearly as good as true because (it is false and) there are
some things arranged Davidwise (\citeyear[171]{merricks2001a}).
\end{quote}

We may now say on behalf of Merricks that we believe that there are
chairs (and statues) because, first, there are things arranged
chairwise and, second, we see that there are things arranged
chairwise.

The structure of the two explanations is analogous, but there is an
apparent disanalogy in the content of the two.  The disanalogy does
not favor Merricks.  For it's easy enough to see why there being
chairs, and us seeing that there are chairs, would cause us to believe
that there are chairs.  But it is less obvious why there being things
arranged chairwise, and us seeing that there are things arranged
chairwise, would cause us to believe {\em not} that there are things
arranged chairwise, but that there are {\em chairs}.

(While it is admitted by all that we believe that there are chairs, I
am not sure if all or even most of us {\em also} believe that there
are things arranged chairwise.  Let us suppose for now that we do.)

The close conceptual connection between `chair' and `chairwise' is
very important for Merricks.  It is this {\em connection} that is
doing the explanatory work.  The only thing that can explain why there
being things arranged chairwise would cause us to believe that there
are chairs is this connection between the concepts.  The existence of
things arranged chairwise, and the belief that there are things
arranged chairwise, is supposed to cause the {\em additional} belief
that there are chairs.  How does this happen?

I believe that Merricks' answer would go something like this: certain
arrangements of things---chairwise arrangements, statuewise
arrangements, and all ordinary arrangements---play important roles in
our lives.  These arrangements of things are of interest to us, so we
have developed words that allow us to refer to them.  For whatever
reason---sociological, psychological, or otherwise---we think of each
arrangement as a single thing, rather than as things.  Words like
`chair' and `statue' reflect this (incorrect) view of the world.

This is more than Merricks says himself.  I have not found a passage
in which he explicitly describes the nature of the conceptual
connection between concepts like `chair' and `chairwise'.  But take
this quote:

\begin{quote}
{[}Consider{]} the claim that the atoms arranged my-neighbour's-dogwise
and the-top-half-of-the-tree-in-my-backyardwise compose an
object\ldots{}it won't do to defend this claim with nothing more than `I
can \emph{just see} the object composed of the atoms arranged
dog-and-treetopwise'. Part of why this won't do, presumably, is that
one's visual evidence would be the same \emph{whether or not} those
atoms composed something (\citeyear[8--9]{merricks2001a}).
\end{quote}

He seems to be pointing out here that whether we see an arrangement of
things as composing an object or not depends more on our own interests
than features of the things themselves.  We have words for chairs and
statues because things arranged chairwise and statuewise interest us.
We don't have a word for things arranged ``my-neighbor's-dogwise and
the-top-half-of-the-tree-in-my-backyardwise'' because such an
arrangement does not hold much interest for us.  But each of these
arrangements exist, and it seems arbitrary to say that the chairwise
and statuewise arrangements compose chairs and statues while the other
arrangement composes nothing.

Merricks would explain why we believe that there are things arranged
my-neighbor's-dogwise and the-top-half-of-the-tree-in-my-backyardwise
thus: there are things arranged my-neighbor's-dogwise and
the-top-half-of-the-tree-in-my-backyardwise, and we see that there are
things so arranged.  This is exactly the same explanation that I would
give.

Now Merricks explains why we believe that there are chairs thus: there
are things arranged chairwise, and we see that there are things
arranged chairwise.  {\em And incidentally, due to our own human
  peculiarities, we have found it convenient to refer to and think
  about things arranged chairwise as if they were single objects}.

\subsection{And yet I believe that there are chairs}
\label{yet}
This is a somewhat plausible explanation of why we would belief that
there are chairs if there were not.  It is certainly much better than
van Inwagen's.  But it does not {\em move} me.  I do not feel inclined
to say that chairs do not exist.  I still believe that there are
chairs.  Whether it has to do with composition (I think this is where
van Inwagen stumbles), or with the meaning of `chair', I am convinced
that Merricks has {\em somehow} gone astray.

Moreover, this argument presupposes that universalism is false.  If
universalism is true, then there {\em is} something composed of the
things arranged my-neighbor's-dogwise and
the-top-half-of-the-tree-in-my-backyardwise.  I will postpone
discussion of universalism until after we look at Peter Unger's
version.  I am doing this because his arguments for the conclusion
that there are no ordinary things threatens to undermine Merricks'
position anyway.

Unger's attempts to show that ordinary concepts like `chair' are {\em
  incoherent} and do not apply to anything in the world (thus he
concludes that there are no ordinary things).  I think that if his
arguments succeed in showing concepts like `chair' to be incoherent,
they will show that concepts like `chairwise' are too.  If a
`chairwise arrangement of simples' is an incoherent notion, then there
can be no such things.  If there are no chairwise arrangements of
simples, then proposition like ``there are chairs'' cannot be nearly
as good as true.  We will look at Unger's argument next.  I will
discuss how it relates to Merricks in section \ref{sorites-m}.

\ifstandalone
\end{spacing}
\bibliography{everything}
\bibliographystyle{ChicagoReedweb}
\fi
\end{document}
