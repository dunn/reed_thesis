\documentclass[11pt]{article}
\usepackage{standalone} \newif\ifstandlone \standalonetrue
\usepackage[left=1.75in, right=1.75in, top=1.25in, bottom=1.25in]{geometry}
\geometry{letterpaper}
\usepackage{graphicx}
\usepackage{enumitem}
%\usepackage{amssymb}
\usepackage{amsmath}
\usepackage{epstopdf}
\usepackage{verbatim}
\usepackage{setspace}
\usepackage{natbib}
\setcitestyle{aysep={}}
\usepackage{hyperref}
\usepackage{url}
\synctex=1

\DeclareSymbolFont{symbolsC}{U}{txsyc}{m}{n}
\DeclareMathSymbol{\strictif}{\mathrel}{symbolsC}{74}
\DeclareMathSymbol{\boxright}{\mathrel}{symbolsC}{128}

\newenvironment{squote}{%
\begin{spacing}{1}
       	\begin{list}{}{%
\setlength{\labelwidth}{0pt}%
\rightmargin\leftmargin%
}
\item\relax
}{%
\end{list}%
\end{spacing}
}

\title{``Nearly as good as true''}
\author{Alexander A. Dunn}
\begin{document}
\ifstandalone
\maketitle
\begin{spacing}{1.25}
\fi

\section{Merricks and the indispensibility of ordinary concepts}
\label{merricks}
Trenton Merricks comes to the same metaphysical conclusions as does
van Inwagen.  That is, he claims that there are no physical objects
other than human beings.  However, he comes to this conclusion through
a different path of reasoning.  (I have not studied these arguments
yet, so I will skip that part for now.)

Despite the fact that Merricks has a different motivation for his
nihilism, we can pose the same question to him as we posed to van
Inwagen.  Why, if there are no chairs, do we believe that there are
chairs?  Happily, Merricks addresses our concern.  Even more happily,
he has a better explanation than van Inwagen.  But sadly, even if his
explanation is good, it will probably be undermined by Unger's
arguments for nihilism.

\subsection{Nearly as good as true}
\label{near}
Merricks claims that `folk' beliefs, such as the belief that there are
chairs, are false, but nonetheless are {\em nearly as good as true}.
What does this mean?

\begin{squote}
People who believe in unicorns [or ghosts] are few and far between.
And those few are generally unjustified.  On the other hand, people
who believe in statues are legion.  And they are generally justified
in so believing.  Given the truth of eliminativism [what I have been
  calling nihilism], we might ask {\em why} the belief in statues is
more common, and more commonly justified, than the belief in unicorns.

The answer is that statue beliefs are nearly as good as true.  For, so
I claim here, {\em atoms arranged statuewise} often play a key role in
producing, and grounding the justification of, the belief that statues
exist.  In general, a false belief's being nearly as good as true
explains how {\em reasonable} people come to hold it.  And, relatedly,
its being nearly as good as true can ground its justification.
Because the belief that unicorns exist is not nearly as good as true
(i.e.\ because there are no things arranged unicornwise), there is no
similar explanation of its production or similar reason to think it is
justified (\citeyear[171--172]{merricks2001a}).
\end{squote}

To say that something is ``nearly as good as true'' seems to be
equivalent to saying that it is `loosely true', or `true for practical
purposes'.  In each case, the proposition in question is false, but it
is somehow close enough to the truth for a given purpose or situation.
For example, suppose we have decided to buy a fake holiday tree for
the holidays this year.  We are looking at a number of different fake
trees.  I point to one and say ``that is a nice tree''.  What I have
said is false; that is not a tree.  It is a fake tree.  But what I
mean---and what my audience recognizes me to mean---is that it is a
nice {\em fake} tree.  We both know that we are looking at fake trees;
there is no point qualifying every use of `tree' with `fake'.  When I
say ``that is a nice tree'', therefore, what I say is quite sufficient
to allow for successful communication. despite being false.  Merricks
claims that propositions expressed by things like ``there are chairs''
are also loosely true.  They are false, but are nonetheless good
enough for certain purposes.

Initially, this seems like a bizarre claim.  After all, Merricks is
claiming that chairs {\em necessarily} do not exist.  According to
Merricks, ``chairs exist'', given its current meaning, could {\em
  never} be true.  If the proposition expressed by ``chairs exist'' is
necessarily false, how could it nonetheless be ``nearly as good as
true''?

Merricks relies on the very close conceptual connection between
``chair'' and ``chairwise'' (and likewise for all ordinary terms).
Despite claiming that chairs are impossible, Merricks admits that we
understand perfectly what chairs {\em would} be, if they existed.
Because we understand the concept of `chair', we can recognize {\em
  actually existing} things that are arranged `chairwise':

\begin{quote}
The folk concept of \emph{statue} plays a role in determining which
atomic arrangements are statuewise. I would even go so far as to say
that if \emph{being arranged statuewise} were not derivative upon
folk-ontological concepts\,\ldots something would be amiss
(\citeyear[8]{merricks2001a}).
\end{quote}

For Merricks, to know what things are actually arranged statue- or
chairwise requires knowing what things would compose a chair, if
chairs were possible:

\begin{quote}
Atoms are \emph{arranged statuewise} if and only if they both have the
properties and also stand in the relations to microscopica upon which,
if statues existed, those atoms' \emph{composing a statue} would
non-trivially supervene (\citeyear[4]{merricks2001a}).
\end{quote}

When we look at Peter Unger's arguments for nihilism, we will see that
this close conceptual connection between terms like `chairwise' and
`chair' will cause trouble for Merricks.  Unger's arguments for a
version of nihilism aim to prove that our ordinary concepts like
`chair' are {\em incoherent} and do not apply to anything in the world
(thus he concludes that there are no ordinary things).  As we will
see, if his arguments succeed in showing concepts like `chair' to be
incoherent, they will show that concepts like `chairwise' are too.  If
a `chairwise arrangement of simples' is an incoherent notion, then
there can be no such things.  If there are no chairwise arrangements
of simples, then proposition like ``there are chairs'' cannot be
nearly as good as true.  We will look at Unger's argument and see how
it relates to Merricks' in section \ref{sorites-m}.

\ifstandalone
\end{spacing}
\bibliography{everything}
\bibliographystyle{ChicagoReedweb}
\fi
\end{document}
