\documentclass[11pt]{article}
\usepackage{standalone} \newif\ifstandlone \standalonetrue
\usepackage[left=1.75in, right=1.75in, top=1.25in, bottom=1.25in]{geometry}
\geometry{letterpaper}
\usepackage{graphicx}
\usepackage{enumitem}
%\usepackage{amssymb}
\usepackage{amsmath}
\usepackage{epstopdf}
\usepackage{verbatim}
\usepackage{setspace}
\usepackage{natbib}
\setcitestyle{aysep={}}
\usepackage{hyperref}
\usepackage{url}
\synctex=1

\DeclareSymbolFont{symbolsC}{U}{txsyc}{m}{n}
\DeclareMathSymbol{\strictif}{\mathrel}{symbolsC}{74}
\DeclareMathSymbol{\boxright}{\mathrel}{symbolsC}{128}

\newenvironment{squote}{%
\begin{spacing}{1}
       	\begin{list}{}{%
\setlength{\labelwidth}{0pt}%
\rightmargin\leftmargin%
}
\item\relax
}{%
\end{list}%
\end{spacing}
}

\title{``Nearly as good as true''}
\author{Alexander A. Dunn}
\begin{document}
\ifstandalone
\maketitle
\begin{spacing}{1.25}
\fi

\label{merricks}
Trenton Merricks comes to the same metaphysical conclusions as does
van Inwagen.  That is, he claims that there are no physical objects
other than human beings.  However, he comes to this conclusion through
a different path of reasoning.  (I have not studied his arguments yet,
so I will skip that part for now.)

\section{Explaining our beliefs}
\label{m-exp}
Despite the fact that Merricks has a different motivation for his
nihilism, we can pose the same question to him as we posed to van
Inwagen.  Why, if there are no chairs, do we believe that there are
chairs?  Happily, Merricks addresses our concern.  Even more happily,
he has a better explanation than van Inwagen.  But sadly, even if his
explanation is good, it will probably be undermined by Unger's
arguments for nihilism.

\section{Nearly as good as true}
\label{near}
Merricks claims that `folk' beliefs,\footnote{He distinguishes `folk'
  beliefs from his own (presumably sophisticated) beliefs.  He claims
  that when he expresses a proposition by saying ``there is a chair'',
  {\em he} says something {\em true}.  I will examine this
  extraordinary claim in section \ref{folk}.} such as the belief that
there are chairs, are false, but nonetheless are {\em nearly as good
  as true}.  What does this mean?

\begin{squote}
People who believe in unicorns [or ghosts] are few and far between.
And those few are generally unjustified.  On the other hand, people
who believe in statues are legion.  And they are generally justified
in so believing.  Given the truth of eliminativism [what I have been
  calling nihilism], we might ask {\em why} the belief in statues is
more common, and more commonly justified, than the belief in unicorns.

The answer is that statue beliefs are nearly as good as true.  For, so
I claim here, {\em atoms arranged statuewise} often play a key role in
producing, and grounding the justification of, the belief that statues
exist.  In general, a false belief's being nearly as good as true
explains how {\em reasonable} people come to hold it.  And, relatedly,
its being nearly as good as true can ground its justification.
Because the belief that unicorns exist is not nearly as good as true
(i.e.\ because there are no things arranged unicornwise), there is no
similar explanation of its production or similar reason to think it is
justified (\citeyear[171--172]{merricks2001a}).
\end{squote}

To say that something is ``nearly as good as true'' seems to be
equivalent to saying that it is `loosely true'.  In both cases, the
proposition in question is false, but it is close enough to the truth
for a given purpose or situation.  An example of a proposition being
loosely true might be this:

\stage{Billy}{(pointing)}{Golly that's a nice tree!}

\stage{Betsy}{}{Yes it is, but it's not a tree.}

\stage{Billy}{}{Whatd'ya mean not a tree?  Look, needles!  Ow!}

\stage{Betsy}{}{That's a fake tree made out of aluminum.}

\stage{Billy}{}{Ow!  Close enough!}

When he says ``that'', Billy is indicating a fake tree.  Betsy knows
that the fake tree is fake, and so knows that what Billy
says---``that's a nice tree''---is false.  Why then does she agree to
a falsehood?  Presumably because the falsehood is loosely true.  The
object demonstrated is not a tree, but it is a fake tree, and let us
suppose that it is, in fact, nice.  Because it resembles a nice tree
in these relevant ways, Betsy recognizes what object Billy intends to
refer to (the fake tree, not a real tree elsewhere) and recognizes
that `nice' nonetheless applies.

It is relatively easy to understand how Billy is taken in---how he
comes to believe that there is a nice tree in front of him when there
is really no such thing.  He sees a rather convincing fake tree, and
mistakes it for a real tree.

The question is now whether there is an analogy here with the
nihilistic treatment of ordinary objects.  Can we understand what goes
on during putative communication about chairs in a way that neither
commits us to the existence of chairs nor fails to explain why our
communicators believe that there are chairs?

\stage{Child}{}{What's that?}

\stage{Parent}{}{That's a chair.  It's a very pretty chair, isn't it?}

\stage{Trenton Merricks}{(runs in)}{Well, yes, but it's not a chair!}

\stage{Parent}{}{What?}

According to the nihilists, there are no chairs, so nobody can refer
to them.  If the parent has indicated anything at all with ``that's a
chair'', then presumably she has referred to a chairwise arrangement
of simples.
%
%\footnote{Unger, unlike some nihilistic
%  philosophers (see \citet{Sider2011c}), does not rule out the
%  possibility of composite objects entirely.  Rather he maintains that
%  our current ordinary term are incoherent.  He might therefore allow
%  that there could exist a composite object that resembles a chair but
%  has precisely demarcated boundaries.  Such an object would, of
%  course, have much stricter persistence-conditions than would an
%  ordinary chair; it may cease to existence with the addition or
%  removal of a single atom.}
%
\ Why then would Merricks agree?  Presumably because what the parent
says is loosely true.  The objects demonstrated do not compose a
chair, but they are arranged chairwise and the effect is pretty.
Because the arrangement resembles a pretty chair in these relevant
ways, Merricks recognizes what objects the parent and child mean to
refer to (the simples arranged chairwise, not a metaphysically
impossible chair) and recognizes that `pretty' nonetheless applies.

I doubt that these cases are as analogous as they may appear, however.

First, the {\em kind} of mistake seems to be different.  Billy
mistakes the fake tree for a real tree perhaps because he has only
ever seen real trees, or very few fake trees.  His mistake is that of
classifying something unfamiliar as something familiar.  He (naturally
enough) assumes that the object in front of him is the same sort of
object that he has come across in the past.  He doesn't have {\em
  experience} with fake trees.

This is not the sort of mistake made by the parent with regard to the
chair.  Here we are to suppose that she is mistaking something
`familiar'---chairwise simples or chair-stages---for something she has
never seen before.  This sort of mistake happens, of course (people
think they see ghosts) but our case seems different.  The parent isn't
getting tricked by some aspect of this particular situation, and so
mis-classifying the object in front of her due to some lapse.  We're
suppose to believe that she has {\em invariably} mistaken simples
(arranged chairwise) for chairs.  She doesn't believe she has {\em
  ever} seen the former, when in fact she has {\em never} seen the
latter.

This leads into the second disanalogy.  When Betsy shows Billy that
the fake tree is not a real tree, and explains to him how the two
things differ, he recognizes his mistake.  He can then (with practice)
go about in the world and correctly identify real trees and fake
trees, as they arise.

This is not quite what Merricks is attempting to teach {\em us}.  It's
not that we have been running two different things together (real and
fake trees); he thinks we have been mis-categorizing groups of one
kind of thing (simples arranged chairwise) as single things (chairs)
of a very different kind.  And unlike the case of Billy and Betsy,
this is not a mistake we are inclined to recognize as such, even when
Merricks points it out to us.  Unless Billy is exceptionally thick, he
will soon agree that what he thought was such a nice tree was really
not a tree at all.  But one does not have to be particularly dense to
deny that what she has been calling a chair is not, in fact, a chair.

The point in bringing out these two disanalogies is the same.  In the
case of the fake tree, there is a perfectly familiar story to tell
about where the false belief came from, but there is no corresponding
story in the chair case.  Billy thinks he's seeing a tree because he's
seen lots of trees and not very many fake trees.  Thus when he comes
across a convincing fake tree, he believes that it is a real tree.
The parent has (according to the nihilist) seen lots of chairwise
arrangements but no chairs.  So why does she believe that {\em every}
chairwise arrangement is actually a chair?  Billy can be taught to
distinguish real trees from fake trees.  But not only are there no
real chairs for the parent to distinguish from chairwise arrangements,
but she will probably not let Merricks convince her that she is really
looking at just a chairwise arrangement.  Where does this unshakable
belief come from?

We might try to skew the analogy.  Suppose Billy has grown up in a
fake tree factory, and has never seen a real tree.  All the workers
refer to the fake trees as ``trees'', so Billy assumes that fake trees
{\em are} trees.  Now he meets Betsy:

\stage{Billy}{(pointing)}{Golly that's a nice tree!}

\stage{Betsy}{}{Yes it is, but it's not a tree.}

\stage{Billy}{}{Whatd'ya mean not a tree?  Look, needles!  Ow!}

\stage{Betsy}{}{That's a fake tree made out of aluminum.}

\stage{Billy}{}{Ow!  What?}

The ensuing attempts by Betsy to explain to Billy what a real tree is
will no doubt be frustrating and useless.  To Billy, a tree just {\em
  is} an aluminum construct.  The only way Betsy can convince Billy is
by taking him outside and showing him real trees.  Only once he learns
what a real tree is can he understand why fake trees are not real
trees.

When Merricks tries to explain to the parent that a real chair is
(necessarily) nonexistent, he will probably have little success.  He
cannot of course {\em show} the parent a real chair to illustrate the
difference.  He presumably has to make her understand {\em why} chairs
are necessarily impossible before she will agree that what she has
been referring to as chairs are not real chairs, and will no longer
believe that there are chairs.  And this is something few parents will
do.

\section{The real problem}
These considerations are not conclusive.  There might be an
explanation as to why chairwise arrangements of simples cause us to
believe that there are chairs.  It seems possible to maintain that
there is some feature of human psychology that causes us to do this.
Maybe it would be too difficult for us to think of the world as mere
arrangements of simples, so our brains present the world to us in
terms of objects composed from simples.

However, even if such an explanation were available, Merricks'
position faces another challenge.  Peter Unger's arguments for a
version of nihilism aim to prove that our ordinary concepts like
`chair' are {\em incoherent} and do not apply to anything in the world
(thus he concludes that there are no ordinary things).  As we will
see, if his arguments succeed in showing concepts like `chair' to be
incoherent, they will show that concepts like `chairwise' are too.  If
a `chairwise arrangement of simples' is an incoherent notion, then
there can be no such things.  If there are no chairwise arrangements
of simples, then proposition like ``there are chairs'' cannot be
nearly as good as true.  We will look at Unger's argument and see how
it relates to Merricks' in section \ref{sorites-m}.

\ifstandalone
\end{spacing}
\bibliography{everything}
\bibliographystyle{ChicagoReedweb}
\fi
\end{document}
