\documentclass[11pt]{article}
\usepackage{standalone} \newif\ifstandlone \standalonetrue
\usepackage[left=1.75in, right=1.75in, top=1.25in, bottom=1.25in]{geometry}
\geometry{letterpaper}
\usepackage{graphicx}
\usepackage{enumitem}
\usepackage{amssymb}
\usepackage{amsmath}
\usepackage{tipa}
\usepackage{epstopdf}
\usepackage{verbatim}
\usepackage{setspace}
\usepackage{natbib}
\setcitestyle{aysep={}}
\usepackage{url}
\synctex=1
\usepackage{hyperref}

%% \newcounter{dcount}
%% \newcommand{\ditem}[1]{%
%%   \item[#1] \refstepcounter{dcount}\label{#1}
%% }
%% \newcommand{\dref}[1]{\hyperref[#1]{#1}}

\DeclareSymbolFont{symbolsC}{U}{txsyc}{m}{n}
\DeclareMathSymbol{\strictif}{\mathrel}{symbolsC}{74}
\DeclareMathSymbol{\boxright}{\mathrel}{symbolsC}{128}

\newenvironment{squote}{%
\begin{spacing}{1}
       	\begin{list}{}{%
\setlength{\labelwidth}{0pt}%
\rightmargin\leftmargin%
}
\item\relax
}{%
\end{list}%
\end{spacing}
}

\title{``Nearly as good as true''}
\author{Alexander A. Dunn}
\begin{document}
\ifstandalone
\maketitle
\begin{spacing}{1.25}
\fi

\section{Merricks and the indispensability of ordinary concepts}
\label{universe}

\noindent Trenton Merricks' explanation of why we believe that there
are chairs is much more plausible that van Inwagen's.  However, the
plausibility of Merricks' explanation relies on his ability to
undermine our reasons to think that it is obviously true that there
are chairs.  I will argue that his attempt to show that ``there are
chairs'' is not obviously true fails; consequently, his explanation of
why we believe that there are chairs fails too.  

The lesson of van Inwagen's nihilism was that the Special Composition
Question is perhaps a poor question.  The lesson of Merricks' nihilism
is that `composition' is itself a tricky notion.  It is useful, when
formally describing a thing and its parts, to describe the thing as
`composed' of its parts.  But talking about when things do or do not
`compose' another things is an unintuitive way to talking, which tends
to make nihilism seem more plausible than it is.

\section{How does Merricks explain our beliefs?}
\label{merricks}
Trenton Merricks comes to the same metaphysical conclusions as does
van Inwagen.  That is, he claims that there are no physical objects
other than human beings.  However, he comes to this conclusion through
a different path of reasoning.  (I have not studied these arguments
yet, so I will skip that part for now.)

Despite the fact that Merricks has a different motivation for his
nihilism, we can pose the same question to him as we posed to van
Inwagen.  Why, if there are no chairs, do we believe that there are
chairs?  Happily, Merricks addresses our concern.  Even more happily,
he has a better explanation than van Inwagen.  He explains why, if
nihilism is true, we might nonetheless believe that there are chairs.
But when giving his explanation, he presupposes that {\em
  universalism} is false (see sections \ref{scq-ans} and
\ref{universalism}).  Universalism, like nihilism, seems to contradict
certain of our beliefs, but Merricks' explanation can also explain
why, if universalism is true, we nonetheless hold these certain
beliefs.  Merricks' explanation does not therefore provide any
advantage to nihilism as opposed to universalism, and the latter is
far more plausible.

\subsection{Nearly as good as true}
\label{near}
Merricks claims that `folk' beliefs, such as the belief that there are
chairs, are false, but nonetheless are {\em nearly as good as true}.
What does this mean?

\begin{squote}
People who believe in unicorns [or ghosts] are few and far between.
And those few are generally unjustified.  On the other hand, people
who believe in statues are legion.  And they are generally justified
in so believing.  Given the truth of eliminativism [what I have been
  calling nihilism], we might ask {\em why} the belief in statues is
more common, and more commonly justified, than the belief in unicorns.

The answer is that statue beliefs are nearly as good as true.  For, so
I claim here, {\em atoms arranged statuewise} often play a key role in
producing, and grounding the justification of, the belief that statues
exist.  In general, a false belief's being nearly as good as true
explains how {\em reasonable} people come to hold it.  And, relatedly,
its being nearly as good as true can ground its justification.
Because the belief that unicorns exist is not nearly as good as true
(i.e.\ because there are no things arranged unicornwise), there is no
similar explanation of its production or similar reason to think it is
justified (\citeyear[171--172]{merricks2001a}).
\end{squote}

To say that something is ``nearly as good as true'' seems to be
equivalent to saying that it is `loosely true', or `true for practical
purposes'.  In each case, the proposition in question is false, but it
is somehow close enough to the truth for a given purpose or situation.
For example, suppose we have decided to buy a fake holiday tree for
the holidays this year.  We are looking at a number of different fake
trees.  I point to one and say ``that is a nice tree''.  What I have
said is false; that is not a tree.  It is a fake tree.  But what I
mean---and what my audience recognizes me to mean---is that it is a
nice {\em fake} tree.  We both know that we are looking at fake trees;
there is no point qualifying every use of `tree' with `fake'.  When I
say ``that is a nice tree'', therefore, what I say is quite sufficient
to allow for successful communication. despite being false.  Merricks
claims that propositions expressed by things like ``there are chairs''
are also loosely true.  They are false, but are nonetheless good
enough for certain purposes.

Initially, this seems like a bizarre claim.  After all, Merricks is
claiming that chairs {\em necessarily} do not exist.  According to
Merricks, ``chairs exist'', given its current meaning, could {\em
  never} be true.  If the proposition expressed by ``chairs exist'' is
necessarily false, how could it nonetheless be ``nearly as good as
true''?

\subsection{The conceptual connection}
\label{connection}
Merricks' argument relies on a very close conceptual connection
between ``chair'' and ``chairwise'' (and likewise for all ordinary
terms).  Despite claiming that chairs are impossible, Merricks admits
that we understand perfectly what chairs {\em would} be, if they
existed.  Because we understand the concept of `chair', we can
recognize {\em actually existing} things that are arranged
`chairwise':

\begin{squote}
The folk concept of \emph{statue} plays a role in determining which
atomic arrangements are statuewise. I would even go so far as to say
that if \emph{being arranged statuewise} were not derivative upon
folk-ontological concepts\,\ldots something would be amiss
(\citeyear[8]{merricks2001a}).
\end{squote}

For Merricks, to know what things are actually arranged statue- or
chairwise requires knowing what things would compose a statue or a
chair, if such things were possible:

\begin{squote}
Atoms are \emph{arranged statuewise} if and only if they both have the
properties and also stand in the relations to microscopica upon which,
if statues existed, those atoms' \emph{composing a statue} would
non-trivially supervene (\citeyear[4]{merricks2001a}).
\end{squote}

Merricks' explanation of why we believe that there are chairs relies
on this conceptual connection.  It also is structurally similar to the
explanation we gave in section \ref{intro-beliefs}.  Recall that our
explanation of why we believe that there are chairs (or statues) is
that, first, there are chairs, and, second, we see that there are
chairs (or learn that there are chairs through a similarly reliable
mechanism).

Merricks' definition of `nearly as good as true' allows us to produce
a parallel explanation.  His definition is this:

\begin{squote}
Any folk-ontological claim of the form `F exists' is \emph{nearly as
  good as true} if and only if (i) `F exists' is false and (ii) there
are things arranged F-wise. So, for example, `the statue \emph{David}
exist' is nearly as good as true because (it is false and) there are
some things arranged Davidwise (\citeyear[171]{merricks2001a}).
\end{squote}

We may now say on behalf of Merricks that we believe that there are
chairs (and statues) because, first, there are things arranged
chairwise and, second, we see that there are things arranged
chairwise.

The structure of the two explanations is analogous, but there is an
apparent disanalogy in the content of the two.  The disanalogy does
not favor Merricks.  For it is easy enough to see why there being
chairs, and us seeing that there are chairs, would cause us to believe
that there are chairs.  But it is less obvious why there being things
arranged chairwise, and us seeing that there are things arranged
chairwise, would cause us to believe {\em not} that there are things
arranged chairwise, but that there are {\em chairs}.

(While it is certainly true that we believe that there are chairs, I
am not sure if all or even most of us {\em also} believe that there
are things arranged chairwise.  Let us suppose for now that we do.)

The close conceptual connection between `chair' and `chairwise' is
very important for Merricks.  It is this {\em connection} that is
doing the explanatory work.  The only thing that can explain why there
being things arranged chairwise would cause us to believe that there
are chairs is this connection between the concepts.  The existence of
things arranged chairwise, and the belief that there are things
arranged chairwise, is supposed to cause the {\em additional} belief
that there are chairs.  How does this happen?

I believe that Merricks' answer would go something like this: certain
arrangements of things---chairwise arrangements, statuewise
arrangements, and all ordinary arrangements---play important roles in
our lives.  These arrangements of things are of interest to us, so we
have developed words that allow us to refer to them.  For whatever
reason---sociological, psychological, or otherwise---we think of each
arrangement as a single thing, rather than as things.  Words like
`chair' and `statue', being singular, reflect this (incorrect) view of
the world.  We are, in a sense, fooled by grammar.

This is more than Merricks says himself.  I have not found a passage
in which he explicitly describes the nature of the conceptual
connection between concepts like `chair' and `chairwise', and explains
why, from our belief that there are things arranged chairwise, we
invariably infer that there are chairs.  But I think he would endorse
something like this.  In the first chapter of his book, he claims that
whether there is a statue or merely things arranged statuewise is not
an empirical question.  He claims that were there not a statue and
merely things arranged chairwise, our ``visual evidence'' would be the
same.  He supports this claim with an analogy:

\begin{squote}
{[}Consider{]} the claim that the atoms arranged my-neighbour's-dogwise
and the-top-half-of-the-tree-in-my-backyardwise compose an
object\ldots{}it won't do to defend this claim with nothing more than `I
can \emph{just see} the object composed of the atoms arranged
dog-and-treetopwise'. Part of why this won't do, presumably, is that
one's visual evidence would be the same \emph{whether or not} those
atoms composed something (\citeyear[8--9]{merricks2001a}).
\end{squote}

He relies here on his assumption that we do not believe that there is
a thing composed of a dog and some of a tree (I will challenge this
assumption below).  Later he says that it is (at least possibly)
arbitrary to claim that there are statues but not dog-tree things:
``we ought to see that the only difference between arbitrary sums and
statues is a matter of conventional wisdom and local custom''
\citeyearpar[75]{merricks2001a}.  He seems sympathetic to the idea
that the reason we believe that there are statues, and not dog-tree
composites, is due to our conventional speech practices: ``it is at
least somewhat plausible that atoms arranged statuewise are united not
by composing something but, instead and in part, by how we speak and
think'' \citeyearpar[121]{merricks2001a}.

On this picture, whether we see an arrangement of things as composing
an object or not depends more on our own interests than features of
the things themselves.  We have words for chairs and statues because
things arranged chairwise and statuewise interest us.  We don't have a
word for things arranged ``my-neighbor's-dogwise and
the-top-half-of-the-tree-in-my-backyardwise'' because such an
arrangement does not hold much interest for us.  But each of these
arrangements exist, and it seems arbitrary to say that the chairwise
and statuewise arrangements compose chairs and statues while the other
arrangement composes nothing.

Merricks might explain why we believe that there are things arranged
my-neighbor's-dogwise and the-top-half-of-the-tree-in-my-backyardwise
thus: there are things arranged my-neighbor's-dogwise and
the-top-half-of-the-tree-in-my-backyardwise, and we see that there are
things so arranged.  This is exactly the same explanation that I would
give.

Now Merricks explains why we believe that there are chairs thus: there
are things arranged chairwise, and we see that there are things
arranged chairwise.  {\em And incidentally, due to our own human
  peculiarities, we have found it convenient to refer to and think
  about things arranged chairwise as if they were single objects}.

\section{Strange objects and visual evidence}
\label{dogbush}
This is a somewhat plausible explanation of why we would belief that
there are chairs if there were not.  It is certainly much better than
van Inwagen's.  But I think that it fails.  I think that when we look
closer at Merricks' attempts to motivate nihilism, we will see that
they do not support nihilism at all.  If anything they support a
version of {\em universalism}.

Merricks observes that one might object to nihilism simply by saying,
``I just {\em see} the chair!''  He claims that if this objection
moves us, we should think about an analogous objection, which he finds
much less moving:

\begin{squote}
Whether atoms arranged statuewise compose a statue is analogous to
whether atoms arranged my-neighbour's-dogwise and
the-top-half-of-the-tree-in-my-backyardwise compose an object\,\ldots
it would not do to support an affirmative answer to the latter
question simply by saying `I can just see that object'
\citeyearpar[73]{merricks2001a}.
\end{squote}

It does indeed seem initially plausible to say that the top half of a
tree and my neighbor's dog do not `compose' anything.  But I think
this is ultimately incorrect.

Recall the bliger story that van Inwagen used to motivate his version
of nihilism (section \ref{prop-ont}).  A bliger was supposed to be
four monkeys, an owl, and a sloth, who arrange themselves into a
temporary symbiotic configuration.  Van Inwagen thought we would agree
that bligers did not exist.  Like Merricks, he brought in the language
of `composition' to make his claim more plausible.  Van Inwagen says
that it is not true that ``six animals arranged in bliger fashion
compose anything, and that is what I mean to deny when I say that
there are no bligers'' \citeyearpar[104]{inwagen1995}.

But as we saw, it is simply false that there are no bligers:

\begin{squote}
\ldots {\em of course} there are bligers in [van Inwagen's] story.
Bligers are what the story is about.  The zoologists do not report
that there are no bligers.  Rather they tell us what a bliger is.
They explain that a bliger is not a single large carnivorous animal
but a transient symbiotic union of six animals
\citep[704]{rosenberg1993}.
\end{squote}

The only reason we might be tempted to say that there are no bligers
is that van Inwagen presents the question in an unintuitive way.  He
asks us if there is some thing, some object, that is composed of the
other six animals.  This gives one the impression that, were there to
be such a thing, it would be another animal (a seventh); were there
such a thing, it should somehow pop out at us.  But all we see when we
picture the scene are the six animals together, so we feel that van
Inwagen might be right.  There is no {\em other} thing.  But if we ask
the question in a more intuitive manner, things become clearer.
Rather than ask if there is some thing composed of such and such other
things, we simply ask, ``are there bligers?''  And of course there
are.  Van Inwagen relied too heavily on the term `composition', and as
a result concluded that ``there are bligers'' was not only less than
obviously true, but actually false.

Merricks makes the same mistake in his passage above.  Imagine if he
had said, ``Consider five discontinuous islands.  One cannot argue
that they compose some further thing by simply saying `I just see
it!'\,'' If these five islands are an archipelago, then one {\em can}
say ``I just see the archipelago!''  {\em Of course} there are
archipelagos.  They are, as one might put it, `scattered objects'.
The archipelago is made up of a number of separate islands, but it is
nonetheless a thing.  It is an archipelago.  Now let us suppose there
is an archipelago in the Mediterranean Sea (this example is adapted
from \citeyearpar{hawthorne2008}).  This archipelago is called the
Roman Archipelago, due to the fact that there are a number of Roman
ruins on one of its islands.  There are several research camps on the
islands, where archaeologists dig for artifacts.  Their researches
result in a surprising discovery: one of the islands {\em is} a Roman
ruin.  What was thought to a rocky and curiously shaped island is in
fact a massive collapsed temple.  Further investigation reveals that
another island is made up of the bones of an ancient sea-dragon, and
another island is a crashed UFO.

Despite these extraordinary circumstances, it is nonetheless true that
the Roman archipelago exists.  It just happens to be composed of
several islands, a Roman ruin, a pile of old bones, and an alien
spacecraft.  To say the Roman Archipelago does not exist would mean
that these things are {\em not} sitting in the Mediterranean Sea.  (Of
course I made this story up, so the Roman Archipelago in fact doesn't
exist; but it does in the story.)

If Merricks or someone else asks us ``could scattered islands, Roman
ruins, old bones and alien spacecraft ever compose anything?'' we
should reply ``{\em of course}''.  Now take this example:

\begin{quote}
Pranksters break into a museum to install joke pieces of art.  One one
wall they put up a bathroom mirror and towel ring (complete with
towel).  Under the mirror they put a little sign reading ``Wash your
hands''.  The installation is accepted as art by the gullible curator,
who gets an equally gullible journalist to write about it.  {\em Wash
  Your Hands} quickly becomes a valuable piece of art---valuable
enough that art thieves target it.  They break into the museum in
order to steal {\em Wash Your Hands}, but trip an alarm and are forced
to flee.  All they get away with is the towel.  In the morning the
guards tell the curator that part of {\em Wash Your Hands} is missing.
The curator orders them to remove the rest of the piece and informs
crestfallen visitors that {\em Wash Your Hands} is no longer in the
museum's collection.
\end{quote}

Here, the only point at which is it true to say that {\em Wash Your
  Hands} is not in the museum is when it is finally removed.  Someone
who claimed that it was {\em never} in the museum because it doesn't
exist would be saying something quite clearly false.  Thus if Merricks
asks us ``do mirrors and towels ever compose anything?'' we should say
``{\em of course!}\,''

In these two examples, it is clear that the things in question really
do exist.  Nobody will deny that there are archipelagos and works of
art without having first been moved by a philosophical argument.  But
it may be that people {\em will} deny that there are things composed
of the tops of trees and dogs, even before hearing an argument.

Call the things composed of dogs and treetops `dogbushes'.  For
example, in a park that contains one tree and one dog, there is also
one dogbush.  Is it {\em obvious} that there are dogbushes?  Is it
just as obvious as that there are archipelagos and chairs and pieces
of crappy modern art like the {\em Wash Your Hands}?  If not, why?
What is the difference between things like archipelagos and things
like dogbushes?

One obvious difference is that things like archipelagos interest us.
I argued above that Merricks motivates his nihilism by drawing our
attention to the role of tradition and convention in our talk.  We
have a word for archipelagos because they {\em matter} to us.  We
don't have a word for dogbushes because they {\em don't} matter.
Merricks argued, in effect, that since we are not inclined to say that
there are dogbushes, and since there is no metaphysical difference
between dogbushes and archipelagos, we should not be inclined to say
that there are archipelagos.

But we can reverse Merricks' argument.  Since we {\em are} inclined to
say that there are archipelagos, and since there is no metaphysical
difference between archipelagos and dogbushes, we should not be
inclined to deny that there are dogbushes.

%% \paragraph{The universe} \label{world}
%% What would it mean to say that the universe doesn't exist?

\subsection{Visual evidence}
\label{visual}
Trenton Merricks introduced dogbushes---things composed of treetops
and dogs---as part of a reply to an argument against nihilism.  The
argument against nihilism was, essentially, that there are chairs and
things because we {\em see} them.  Merricks' reply is that if this is
a good argument, then it should be just as good an argument to claim
that there are dogbushes because we see them, too.  Merricks hopes
that our unwillingness to say that there are dogbushes will lead us
to drop this argument against nihilism.  He wants us to agree that
whether or not there are chairs is not settled by what we see.  He
claims that  that our ``visual evidence'' would be the same whether
or not there were chairs.

But I think Merricks is wrong; our visual evidence would not be the
same if there were no chairs.

Indeed, it is easy to think that Merricks is {\em obviously} wrong,
that he is making some sort of conceptual mistake.  One might take him
to mean that an otherwise empty room would look the same whether or
not there was a chair in it; a chair could be added or removed from
the room without changing our `visual evidence'.  This would indeed be
a bizarre claim.  If the removal of a chair does not affect our visual
evidence, that would have to mean that we are hallucinating the chair,
or there is some other major perceptual error.  But this is not how
Merricks wants to be understood:

\begin{squote}
When someone claims to have seen a statue [or a chair], I am {\em not}
likely to suspect that he has hallucinated, been the victim of a
prank, or mistaken something for a statue that, upon further
inspection, he would agree was not really a statue after all
\citeyearpar[2]{merricks2001a}.
\end{squote}

In a room with a chair, Merricks is not disagreeing over the
arrangement of matter in the room.  He is claiming that the
arrangement does not compose a chair.  He agrees even that the matter
is arranged chairwise, but will not admit that {\em composition}
occurs, and that there is a {\em thing} made up of the matter.

At this point the natural reply is ``but all it means for there to be
a chair is for there to be that arrangement of matter!  You can't have
one without the other.  You can't say `there are things arranged
chairwise by there is no chair'.''

Merricks might therefore take me to be accusing him of `linguistic
contradiction'.  He might think that I think that `there are chairs'
means `there are things arranged chairwise'; to affirm the one and
deny the other would be absurd.  His response to this sort of
objection is as follows:

\begin{squote}
`There are married bachelors' is not explicitly formally
  contradictory, but it is contradictory in some quite straightforward
  sense.  And one might object that `there are atoms arranged
  statuewise but no statues' is contradictory in the same way.  For as
  `bachelor' means someone who is, among other things, unmarried,
  so---the objector insists---`there are [composite] statues' {\em
    just means} that there are some things arranged statuewise.
  Because of its contradictory nature, we should not take seriously an
  ontology according to which there are married bachelors.  Likewise,
  this objection concludes, we should not take seriously the
  eliminativist's ontology, with its atoms arranged statuewise but no
  statues \citeyearpar[13]{merricks2001a}.
\end{squote}

But Merricks points out that ``\,`There are statues' does {\em not}
mean only that there are some things arranged statuewise\,\ldots this
is simply not a plausible claim about ordinary meaning''
\citeyearpar[13]{merricks2001a}.  According to Merricks, `there are
statues' means something like `there are things such that they are
statues', or `there are things such that they are composed of other
things and fit our folk criteria for statuehood'.  It does {\em not}
mean `there are things arranged statuewise'.  The claim is that `there
are statues' may therefore be false, even if there are things arranged
statuewise.

This may all be true, but it is not quite to the point.  The meaning
of phrases like `there are statues' or `there are chairs' depend
(presumably) on the meaning of the constituent words.  The meaning of
`there are chairs' depends on the meaning of `there are' and `chairs'.
As I emphasized in section \ref{verbal}, `there are' is a normal
English quantifier phrase.  A proposition of the form `there are $x$s'
is true if and only if there are $x$s.  The truth or falsity of the
proposition depends on what the plural variable represents.  The truth
or falsity of `there are chairs' depends on the meaning of `chairs'.

What is the meaning of `chair' and `chairs'?  I do not think it is
possible to give non-circular necessary and sufficient conditions for
being a chair.  That is, I do not think there is a true proposition of
the form ``$x$ is a chair if and only if\,\ldots '', where the right
side of the biconditional makes no mention of `chair' or `chairwise'.
There is simply too much variety among chairs to allow for an
exclusive and exhaustive definition:

\begin{squote}
When one says chair, one thinks vaguely of an average chair.  But
collect individual instances, think of arm-chairs and reading chairs,
and dining-room chairs and kitchen chairs, chairs that pass into
benches, chairs that cross the boundary and become settees, dentists'
chairs, thrones, opera stalls, seats of all sorts, those miraculous
fungoid growths that cumber the floor of the Arts and Crafts
Exhibition, and you will perceive what a lax bundle in fact is this
simple straightforward term.  In co-operation with an intelligent
joiner I would undertake to defeat any definition of chair or
chairishness that you gave me. \citep[384--385]{wells1904}.
\end{squote}

Fortunately the word `chair' is perfectly meaningful nonetheless.
This is because it, like `pig' and many other words for ordinary
things, is defined {\em ostensively}:

\begin{squote}
We learn the word `pig', as we learn the vast majority of words for
ordinary things, ostensively---by being told, in the presence of the
animal, `\emph{That} is a pig' \citep[121]{austin1964}.
\end{squote}

We learn the meaning of the word `chair' by engaging in exchanges like
these:

\stage{Child}{(pointing)}{Mother, what is that?}

\stage{Parent}{}{That is a chair, dear.}

Because the meaning of `chair' is not given through a formal
definition, its meaning must apparently be given through its use.  The
extension of `chair'---what objects `chair' applies to---depends (at
least in part) upon the use we make of the word.  The meaning of
`chair' is therefore bound up with its use in demonstrative
statements.  So while `there are chairs' may not mean `there are
things arranged chairwise', `there are chairs' {\em does} mean `there
are {\em those} sorts of things':

\stage{Nihilist}{}{There are no chairs.}

\stage{Child}{(pointing at a chair)}{There are none of those?}

Merricks does not disagree over what matter is in the room.  And we
learn the meaning of `chair' by making demonstrative reference to this
matter.  We do not fix the meaning of chair in isolation, then go
around seeing if anything satisfies our definition; it is not simply
an {\em open question} as to whether there are chairs.  The meaning of
`chair' is determined by its use.  Our use of chair involves
demonstrative statements like ``that is a chair'' when in the presence
of what Merricks would refer to as `things arranged chairwise'.  The
concept of composition plays {\em no} role in the determination of the
meaning of `chair'.

The proposition `there are chairs' might not {\em mean} `there are
things arranged chairwise', but they are truth-conditionally
equivalent:

\begin{squote}
`There are chairs' is true if and only if `there are things arranged
  chairwise' is true.
\end{squote}

(If, as some claim, meaning is reducible to truth-conditions, then I
would be committed to the additional claim that `there are chairs' and
`there are things arranged chairwise' mean the same thing.  But if
Merricks does not hold this thesis about meaning, then he has no
grounds to resist the distinction between two propositions meaning the
same thing and their being truth-conditionally equivalent.  It is only
the latter that I claim.)

The claim above can be generalized:

\begin{description}
  \item[The $F$ existence principle] `There is an $F$'' is true if
    and only if ``There are things arranged $F$-wise'' is
    true. \label{fwise}
\end{description}

Our `visual evidence' would {\em not} be the same if there were no
chairs.

%% To say that there is no chair in the room means
%% that there is nothing of {\em that} sort in the room.  

Likewise, to say that there are no archipelagos entails that there are
no islands grouped together in the relevant way.  To say that there is
not the {\em Wash Your Hands}, entails that the mirror and towel ring
and towel are not where they are in the museum.

Imagine if van Inwagen had claimed that the visual evidence of his
imagined farmers would be the same whether or not there were bligers.
This would be nonsense, for to say there are no bligers means that
owls, monkeys, and sloths never form their symbiotic unions.  And,
{\em by hypothesis}, they do form such unions.  Farmers see these
unions.  Farmers would not see these unions if there were no bligers.
Their visual evidence would certainly not be the same.

%% The same point applies to dogbushes.  I believe there are such things
%% because I {\em see} them.  In making this claim I am, apparently,
%% alone:

%% \begin{squote}
%% There are many philosophers who believe in arbitrary sums like the
%% `dog-and-treetop', but none of them---not one---defends the existence
%% of such things on merely perceptual grounds. No one says we should
%% believe that such an object exists simply because we can see it or
%% simply because we can hear it (gnawing on a bone while rustling its
%% leaves) \citep[74]{merricks2001a}.
%% \end{squote}

%% Merricks claims rather that, if we believe in such things, our belief
%% should be defended on {\em philosophical} grounds.  ``Merely
%% perceptual grounds'' are not good enough.  The point of this, of
%% course, is to undermine our belief that there dogbushes.  Once
%% Merricks has convinced us that there are no dogbushes, the trap is
%% sprung.  For what, really, is the difference between dogbushes and
%% ordinary things like chairs, statues, and archipelagos?  Merricks
%% thinks ``we ought to see that the only difference between arbitrary
%% sums and statues is a matter of conventional wisdom and local custom.
%% Once this is pointed out, one is no longer justified in believing that
%% statues exist merely because one can supposedly see them''
%% \citeyearpar[75]{merricks2001a}.  On the contrary, I say that once
%% this is pointed out, one is no longer justified in believing that
%% dogbushes {\em don't} exist.  For our visual evidence would {\em not}
%% be the same if there were none.

%% Likewise, if there were no chairs, our visual evidence would certainly
%% not be the same.  If there were no chairs, there would not be any
%% things arranged chairwise either.  And Merricks agrees that there are
%% things arranged chairwise.  Now how could ``there are chair'' be
%% false; how could there be no chairs?  Are there {\em no} things
%% arranged chairwise?  There obviously are, and so it is obviously true
%% that there are chairs.

Nihilists like Trenton Merricks have assumed that what is required for
``there are chairs'' to be true is something more than what is
required for ``there are things chairwise'' to be true.  But this is a
mistake.  Ted Sider makes this mistake in a recent paper on parthood:

\begin{squote}
Consider Nihilo, god and creator of a world comprised solely of
subatomic particles.  On the first day Nihilo creates some particles
and arranges them in beautiful but lifeless patterns.  But he becomes
lonely, so on the second day he creates some minions (or rather,
particles arranged minion-wise).  On the third day he tries to teach
his minions to speak.  But this goes badly.  The minions aren't very
bright, and are slow to catch on to Nihilo's talk of subatomic
particles and their physical states.  So on the fourth day he teaches
them an easier way to speak.  Whenever an electron is bonded (in a
certain way) to a proton, he teaches them to say ``there is a hydrogen
atom''; whenever some subatomic particles are arranged chairwise he
teaches them to say ``there is a chair'', and so on.  (Pretend that
electrons and protons have no proper parts.)

When the minions utter sentences like ``there is a hydrogen atom'', do
they speak falsely?  They do if their language is the same as the
language I used to describe the example, since I described Nihilo as
having created a world comprised solely of subatomic particles
\citeyearpar[7]{sider2011c}.
\end{squote}

Sider seems to be thinking that he can stipulate that the
`metaphysical laws' of his imagined world make it impossible for there
to be minions.  But he has clearly not succeeded, for there are
minions right there in his story!  Nihilo created minions.  For there
to be minions, nothing is required over and above there being things
arranged minionwise.  ``There are minions'' is true if and only if
``there are things arranged minionwise'' is true.

In fact, it is {\em impossible} to create a world ``comprised solely
of subatomic particles''.  If they can be moved around, then we (or
Nihilo) can use them to make {\em things}, like minions.  If they
cannot be moved around, it still seems perfectly obvious that there
are things they compose.  Just as we have scattered objects like
archipelagos and artworks, why shouldn't we assume that there are
scattered objects comprised of subatomic particles?

At one point Merricks attempts to motivate his nihilism by claiming
that ``whether atoms arranged statuewise compose a statue is not
straightforwardly empirical'' \citeyearpar[9]{merricks2001a}.  After
reminding us that our visual evidence would be the same whether or not
things arranged statuewise composed a statue, he writes:

\begin{squote}
The fundamental question is not so much whether some particular
alleged statue exists.  That question might---sceptical scenarios
aside---seem to be a matter of just looking and seeing.  The issue is
rather whether, in general, atoms arranged statuewise compose a
statue.  But it seems that this question of metaphysical necessity
cannot be decided, one way or another, simply by a trip to the museum
or a ride down Monument Avenue.  It must be decided on philosophical
grounds \citeyearpar[9]{merricks2001a}.
\end{squote}

I disagree.  The question ``whether things arranged statuewise compose
a statue'' is equivalent to the question whether there are statues.
If the answer to one is ``yes'', the answer to the other is also
``yes''.  The answer to the latter is ``yes''.  So the answer to the
former is ``yes''.  The answer to both is ``yes'', and this just {\em
  is} an empirical question that can be settled ``by a trip to the
museum''.

\section{Universalism}
\label{universalism}
Above I claimed that Merricks' explanation of why we believe that
there are chairs is something like this: there are things arranged
chairwise, and we see that there are things arranged chairwise.  {\em
  And incidentally, due to our own human peculiarities, we have found
  it convenient to refer to and think about things arranged chairwise
  as if they were single objects}.  I attributed to Merricks the idea
that just because things arranged chairwise interest us, we should not
therefore suppose that there are chairs.  What interests us should not
be a guide to what exists.  But now there is an obvious counter
against this move.  Just because dogbushes do {\em not} interest us,
we should not therefore suppose that there are not dogbushes: ``We can
think of Reality as like an over-crowded attic, some of its contents
interesting, and most merely junk.  There is no need to deny the junk;
we can simply leave it to gather dust'' \citep[167]{thomson1998a}.

We seem to be presented with something like this argument:

\begin{enumerate}[ref=(\arabic*)]
  \item Chairs exist. \label{u-1}
  \item Things that do not differ from chairs (or archipelagos, or
    works of art) in metaphysically significant ways also exist.
  \item Dogbushes do not differ from chairs in metaphysically
    significant ways.
  \item {\em Therefore}, dogbushes exist. \label{u-4}
\end{enumerate}

I'm guessing Merricks would deny the conclusion \ref{u-4} and so, by
{\em modus tollens}, deny one or more premises (and we have seen that
he denies \ref{u-1}).  But I would affirm the premises and so, by {\em
  modus ponens}, affirm the conclusion.

\subsection{What is universalism?}
I said in section \ref{scq-ans} that I would postpone discussion of
universalism until later.  It it time now to discuss it.  For I seem
to be committing myself to some version of universalism.  The
formulation I cited above was this:

\begin{description}
\item[Universalism] Necessarily, for any $x$s, there is an object
  composed of the $x$s iff no two of the $x$s overlap
  \citep[227]{markosian1998a}.
\end{description}

Much of the resistance to this thesis (at least, much of my early
resistance) probably stems from its wording; as I argued above, the
unintuitive language of `composition' can lead us to deny obvious
truths, such as that there are chairs and dogbushes.  Ned Markosian's
objection to universalism provides another example of how the term
`composition' obscures the plausibility of universalism:

\begin{squote}
There is what seems to me a fatal objection to Universalism:
Universalism entails that there are far more composite objects than
common sense intuitions can allow.  To give just one example,
Universalism entails that the following sentence is true:\,\ldots
There is an object composed of (i) London Bridge, (ii) a certain
sub-atomic particle located far beneath the surface of the moon, and
(iii) Cal Ripken, Jr.  My intuitions tell me that there is no such
object, and I suspect that the intuitions of the man on the street
would agree with mine on this point \citeyearpar[228]{markosian1998a}.
\end{squote}

Perhaps Markosian's intuitions are led astray because of the term
`composition'.  Even in the case of the archipelago, it seems somewhat
unintuitive to say that the islands, ruins, bones, and spacecraft
`compose' something else.  But it is obvious that there are
archipelagos.  Let us therefore drop `composition' from Markosian's
example and see if that makes things clearer.

Let `Lumpkin Junior' designate the thing composed of the London
Bridge, a particle in the moon, and Cal Ripkin, Jr.  Does Lumpkin
Junior exist?

Let us introduce another term so as to bring the analogy closer.  A
`lumpkin' is something composed of the London Bridge, a sub-atomic
particle in the moon, and a retired major league baseball player.  Now
we can ask ``are there lumpkins?''  If we already believe that there
are chairs, archipelagos, works like the {\em Wash Your Hands}, and
even dogbushes, why {\em shouldn't} there be lumpkins?

\subsection{Begging the question}
\label{beg}
The most straightforward objection to the above is that I am begging
the question.  For I claim that it is {\em true} (if misleading) to
say that Lumpkin Junior is ``that object composed of the London
Bridge, a particle of the moon, and Cal Ripkin, Jr.''  So when
Markosian denies that Lumpkin Junior exists, my `counterargument' is
just that it does exist!

Above, I attempted to make it plausible that things like dogbushes and
lumpkins exists.  I claimed that the wording used by Markosian and
others is unintuitive and causes us to come to incorrect conclusions.
I think that if we formulate the question more naturally (``are there
lumpkins?''), it is intuitively true that there {\em are} such things.
I think this is an appropriate way to proceed, given that Markosian's
argument {\em against} lumpkins is an appeal to intuitions as well.
He says is that ``[his] intuitions tell [him] that there is no such
object, and [he] suspect[s] that the intuitions of the man on the
street would agree with [his] on this point''
\citeyearpar[228]{markosian1998a}.  I suggest that his intuitions are
led astray by an awkward wording.

Take the following argument:

\begin{enumerate}
  \item There are archipelagos.
  \item If there are archipelagos, then it is possible that there are
    things composed of scattered islands, ruins, bones, and
    spacecraft.
  \item It is possible that there are things composed of scattered
    islands, ruins, bones, and spacecraft.
\end{enumerate}

This seems perfectly unobjectionable.  There are archipelagos.
Consequences of that fact that may seem objectionable---such as that
it is possible that there are things composed of scattered islands,
ruins, bones, and spacecraft---should be seen as less objectionable
{\em because they follow from an intuitively obvious premise}.

One could instead start from the bottom.  One could say, ``well, it
doesn't seem to me as if there could be things composed of scattered
islands, ruins, bones, and spacecraft.  So perhaps there are no
archipelagos after all.''  But I think it is more appropriate to begin
from the top.  The phrasing ``there are archipelagos'' is more natural
and common than ``there are things composed of such-and-such'';
therefore our intuitive judgments regarding it should be more
accurate.  When we arrive at technical formulations involving terms
like `composition', our intuitions may be less reliable.  Even if the
technical formulation seems intuitively false, if it follows from one
that seems clearly true, then we should not reject it unless we have
good independent reasons for doing so.

Markosian's objection to universalism requires starting from the
bottom rather than the top:

\begin{enumerate}
  \item There are lumpkins.
%  \item There are archipelagos.
  \item If there are lumpkins, then there are things composed of the
    London Bridge, bits of the moon, and retired baseball players.
%  \item If there are dogbushes, then it is possible that there are
%    things composed of scattered islands, ruins, bones, and
%    spacecraft.
  \item There are things composed of the London Bridge, bits of the
    moon, and retired baseball players.
%  \item It is possible that there are things composed of scattered
%    islands, ruins, bones, and spacecraft.
\end{enumerate}

Markosian finds it implausible to say that are things composed of the
London Bridge, bits of the moon, and retired baseball players, and so,
by {\em modus tollens}, denies that there are lumpkins.  I start from
the top.  Given the other bizarre and scattered objects we have seen
to exist, I do not find it implausible to say that there are lumpkins.
By {\em modus ponens}, I claim that there are things composed of the
London Bridge, bits of the moon, and retired baseball players.

As a methodological principle, I am claiming that we should begin by
affirming what seems intuitively true, and---unless there are very
strong reasons to do otherwise---accept what follows from these
apparent truths.

Take, for example, the Reed College women's rugby team.  There is,
obviously, such a team (whether or not it has College funding).  The
rugby team exists.  Having established this, there are various
consequences.  For instance, each player is part of the team.  The
team is made up of the players and the coach.  Expressing this
formally, we might say that there is some thing (the team) composed of
the $x$s (the players and coach).  If this formal treatment is
equivalent to the informal ``there is a team'', and if the informal
phrasing is uncontroversial, the formal phrasing should not be
controversial either.

Merricks has a related example:

\begin{squote}
Consider whether `the Crew of the USS {\em Enterprise}' is a plural
referring expression---akin to `Locke, Berkeley, and Hume'---or,
instead, the name of a single large object with each crew member as a
proper part.  Note, in fact, that there are two questions here.
First, there is the semantic question of what `the Crew of the USS
{\em Enterprise}' is supposed to mean.  Second, there is the
metaphysical question of whether there really is a big physical object
that has all and only the crew members as its parts (at one level of
decomposition), a scattered object that weighs as much as the sum of
the weights of those people taken individually.

I am not sure how to answer the first question.  But, I say, the
answer to the second question is `no'.  Some philosophers would
disagree.  No matter.  The point here---in this section of this
chapter---is not to settle either the metaphysical or the semantic
dispute surrounding `the Crew of the USS {\em Enterprise}'.  It is,
rather, that such disputes are neither here nor there with respect to
everyday uses of `the Crew of the USS {\em Enterprise}'.  `The Crew of
the USS {\em Enterprise}' will continue to perform its ordinary duties
regardless of how or whether the semantic and metaphysical disputes
get settled \citeyearpar[10]{merricks2001a}.
\end{squote}

I will discuss plural referring expressions in a moment.  But first I
want to point out that the last sentence of this quote by Merricks
does not seem to be true.  Suppose I believed that the crew of the
{\em Enterprise}, were it to exist, would be a thing composed of the
crewmembers, {\em and on those grounds} I denied that the crew exists.
If the crew does not exist, if there is no crew, then that could only
mean that the ship was unmanned.  If there are crewmembers, there is a
crew.  If there is a crew, then there are crewmembers.  If Merricks or
anyone denies that there is a crew, what would it mean for them to say
that `the crew' ``will continue to perform its ordinary duties''?

\subsection{Plural referring expressions}
\label{plural-ref}
In the quoted material above, Merricks suggests that terms like `the
crew of the USS {\em Enterprise}' may not be singular terms, but
plural referring expressions.  For example, `the Dunns' refers
plurally to me and the other members of my family.  I say things like
``the Dunns are fine people''; the term obviously does not function as
a singular term.  The suggestion is that `the crew' behaves similarly.
When I say that the crew exists, it would then {\em not} follow that
there is a {\em thing} composed of the crewmembers.  Saying ``the crew
exists'' would instead be equivalent to saying ``the crewmembers all
exist''.

I do not think that this a plausible claim.  Recall the analogy I
tried to draw between `the crew' and `the Dunns'.  On closer
inspection, this analogy appears weak.  A stronger analogy would be
between a term like `the crew' and a term like `the Dunn family'.
`The Dunn family' is {\em not} a plural referring expression.  It is
used to refer to a {\em thing}.  If I talk about `the Dunn family', I
would say things like, ``The Dunn family is waning'', or ``The Dunn
family must regain its political power''.  The term `the Dunn family'
is a singular term that designates a thing---the family.

`The crew' appears to behave like `the Dunn family' and not like `the
Dunns'.  We say things like ``there is a skeleton crew on board'', or
``the crew is small for such a large ship'', and ``the crew is
abandoning the ship''.  We so also say things like ``the crew {\em
  are} abandoning the ship'', but this may be a case of {\em
  non-literal} speech; `the crew' is being used non-literally to refer
to the crewmembers.  If this is not non-literal speech, then it seems
most likely that `the crew' is {\em ambiguous}: it can be used to
refer either to {\em the crew} or to the crew{\em members}.  In the
former case, `the crew' is used as a singular term.

Similar considerations apply to terms like `team' as well.  `The Reed
College women's rugby team' is a singular term, for it behaves in the
same ways as do `the crew' and `the Dunn family'.  We say things like
``The Reed College women's rugby team is going to win'', or ``The Reed
College women's rugby team is in Seattle this weekend''.  However,
team names are often used (whether non-literally or not) to refer to
the team-members, rather than to the team itself.  This is often due to
pluralized team names.  The Reed College women's rugby team is called
``The Badass Sparkle Princesses''.  This leads us to say things like
``The Badass Sparkle Princesses are on a losing streak''.  Here we are
led by the plural construction to---perhaps unconsciously---use the
term to refer not to the team itself but to the players.  The Badass
Sparkle Princesses {\em is} a rugby team, but it is far more natural
to say that the Badass Sparkle Princesses are rugby players.

I will henceforth assume that terms like `crew', `family', and `team'
are not plural referring expressions, but rather singular terms that
designate things---crews, families, and teams.  (Similar
considerations will convince us that `lumpkin' is a singular term too,
and not a plural referring expression.)

\subsection{What are groups?}
\label{group}
I have suggested that families, crews, and other {\em groups} are,
strictly speaking, things.  But one might object that there is no {\em
  need} to suppose that there are things called groups; we can
identify families, crews, courts, etc.\ with {\em sets}, and avoid the
`ontological clutter' that would result from the introduction of
groups.  Groups, it may be said, are really no different than sets.
When we speak of a group of people, we are actually referring to the
set of which they are members.

But there are some reasons why it seems incorrect to identify groups
with sets.  Take the Supreme Court.  It seems that any attempt to
identify the Supreme Court with the set of the Supreme Court justices
will not succeed.  This is because the membership of the Supreme Court
changes over time, while the members of a set do not.  The set
containing the 1990 justices is a {\em different} set from the set
containing the 2012 justices, but the 2012 Supreme Court is not a
different entity than the 1990 Court.  (We may of course say things
like ``it's a different court now'', but by that we mean only that it
is composed of different people, and so may rule differently---note
that we do {\em not} say ``it's a different Court now''.)

If one grants that groups such as the Supreme Court are not sets, it
may be objected that they are therefore mereological sums, like chairs
and people (see section \ref{tech}).  But

\begin{squote}
membership in the Supreme Court is very different from
the part-whole relation on material objects.  The part-whole relation
on material objects is a transitive relation.  Thus if one identified
the Supreme Court with a material object and Justice Breyer with a
part of it, then one would be forced to conclude that Justice Breyer's
arm must be a part of the Supreme Court as well.  Yet, it is plain
that Justice Breyer's arm is neither a part nor a member of the
Supreme Court \citep[136--137]{uzquiano2004a}.
\end{squote}

If the Supreme Court were a mereological sum, it would behave very
strangely.  What its parts would be on a given occasion would depend
on the appointment decisions of the President.  (If we accept a
`four-dimensional' version of universalism---see section \ref{4d}---,
then objects have {\em temporal} as well as spatial parts.  There
would then be a mereological sum of the parts of the justices that
existed during their appointments.  This would be an object whose
existence would not depend on the President.)

There is at least some motivation to posit a new {\em kind} of thing
that is not a set or a sum.  This new kind is the group.  In section
\ref{parts} will will look at three theories that attempt to make
room for groups.  Unfortunately, they make room for an incredible
amount of other things as well.

\section{Lessons}
\label{lessons-m}
What we have learned from examining Merricks' arguments is not that
there are no chairs.  What we have learned is that unintuitive terms
like `composition' can fool us into thinking that there are no chairs.
It may be that talk of things `composing' other things gives the
impression that `composition' is something that things {\em do}---as
if they were gathering or attaching themselves together.  But ``there
are some things such that they compose a chair'' is just a technical
way of saying ``there is a chair''.  And all that is required for
there to be a chair is for there to be things arranged chairwise.  If
we recognize this, we can see that there is no motivation for
Merricks' nihilism, other than his thesis of causal
over-determination.  And given that his conclusion---that there are no
chairs---is obviously untrue, we should suspect that causal efficacy
is not required for ``there are chairs'' to be true.  Rather, all that
is required for ``there are chairs'' to be true is that ``there are
things arranged chairwise'' be true.

Despite the fact that `composition' is an unintuitive term, it is very
useful when formalizing metaphysical claims.  Once facts like ``there
are chairs'' are established, then questions regarding composition may
be asked.  These are such questions as: What are the parts of a chair?
How do they compose the chair?  Do the parts of the chair change over
time?  How?

Additionally, if we accept that there are chairs what other things are
there?  If there are groups, and if groups are a different {\em kind}
of thing than are chairs, then how many kinds of things are there?

We will address these questions in the next section.
\ifstandalone
\end{spacing}
\bibliography{everything}
\bibliographystyle{ChicagoReedweb}
\fi
\end{document}
