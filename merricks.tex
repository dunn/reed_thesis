\documentclass[11pt]{article}
\usepackage{standalone} \newif\ifstandlone \standalonetrue
\usepackage[left=1.75in, right=1.75in, top=1.25in, bottom=1.25in]{geometry}
\geometry{letterpaper}
\usepackage{graphicx}
\usepackage{enumitem}
%\usepackage{amssymb}
\usepackage{amsmath}
\usepackage{tipa}
\usepackage{epstopdf}
\usepackage{verbatim}
\usepackage{setspace}
\usepackage{natbib}
\setcitestyle{aysep={}}
\usepackage{url}
\synctex=1
\usepackage{hyperref}

%% \newcounter{dcount}
%% \newcommand{\ditem}[1]{%
%%   \item[#1] \refstepcounter{dcount}\label{#1}
%% }
%% \newcommand{\dref}[1]{\hyperref[#1]{#1}}

\DeclareSymbolFont{symbolsC}{U}{txsyc}{m}{n}
\DeclareMathSymbol{\strictif}{\mathrel}{symbolsC}{74}
\DeclareMathSymbol{\boxright}{\mathrel}{symbolsC}{128}

\newenvironment{squote}{%
\begin{spacing}{1}
       	\begin{list}{}{%
\setlength{\labelwidth}{0pt}%
\rightmargin\leftmargin%
}
\item\relax
}{%
\end{list}%
\end{spacing}
}

\title{``Nearly as good as true''}
\author{Alexander A. Dunn}
\begin{document}
\ifstandalone
\maketitle
\begin{spacing}{1.25}
\fi

\noindent Trenton Merricks' explanation of why we believe that there
are chairs is much more plausible that van Inwagen's.  However, the
plausibility of Merricks' explanation relies on his ability to
undermine our reasons to think that it is obviously true that there
are chairs.  I will argue that his attempt to show that ``there are
chairs'' is not obviously true fails; consequently, his explanation of
why we believe that there are chairs fails too.  

The lesson of van Inwagen's nihilism was that the Special Composition
Question is probably a poor question.  The lesson of Merricks'
nihilism is that `composition' is itself a tricky notion.  It is
useful, when formally describing a thing and its parts, to describe
the thing as `composed' of its parts.  But talking about when things
do or do not `compose' another things is an unintuitive way to
talking, which tends to make nihilism seem more plausible than it is.

\section{Merricks and the indispensability of ordinary concepts}
\label{merricks}
Trenton Merricks comes to the same metaphysical conclusions as does
van Inwagen.  That is, he claims that there are no physical objects
other than human beings.  However, he comes to this conclusion through
a different path of reasoning.  (I have not studied these arguments
yet, so I will skip that part for now.)

Despite the fact that Merricks has a different motivation for his
nihilism, we can pose the same question to him as we posed to van
Inwagen.  Why, if there are no chairs, do we believe that there are
chairs?  Happily, Merricks addresses our concern.  Even more happily,
he has a better explanation than van Inwagen.  But sadly, even if his
explanation is good, it will probably be undermined by Unger's
arguments for nihilism.

\subsection{Nearly as good as true}
\label{near}
Merricks claims that `folk' beliefs, such as the belief that there are
chairs, are false, but nonetheless are {\em nearly as good as true}.
What does this mean?

\begin{squote}
People who believe in unicorns [or ghosts] are few and far between.
And those few are generally unjustified.  On the other hand, people
who believe in statues are legion.  And they are generally justified
in so believing.  Given the truth of eliminativism [what I have been
  calling nihilism], we might ask {\em why} the belief in statues is
more common, and more commonly justified, than the belief in unicorns.

The answer is that statue beliefs are nearly as good as true.  For, so
I claim here, {\em atoms arranged statuewise} often play a key role in
producing, and grounding the justification of, the belief that statues
exist.  In general, a false belief's being nearly as good as true
explains how {\em reasonable} people come to hold it.  And, relatedly,
its being nearly as good as true can ground its justification.
Because the belief that unicorns exist is not nearly as good as true
(i.e.\ because there are no things arranged unicornwise), there is no
similar explanation of its production or similar reason to think it is
justified (\citeyear[171--172]{merricks2001a}).
\end{squote}

To say that something is ``nearly as good as true'' seems to be
equivalent to saying that it is `loosely true', or `true for practical
purposes'.  In each case, the proposition in question is false, but it
is somehow close enough to the truth for a given purpose or situation.
For example, suppose we have decided to buy a fake holiday tree for
the holidays this year.  We are looking at a number of different fake
trees.  I point to one and say ``that is a nice tree''.  What I have
said is false; that is not a tree.  It is a fake tree.  But what I
mean---and what my audience recognizes me to mean---is that it is a
nice {\em fake} tree.  We both know that we are looking at fake trees;
there is no point qualifying every use of `tree' with `fake'.  When I
say ``that is a nice tree'', therefore, what I say is quite sufficient
to allow for successful communication. despite being false.  Merricks
claims that propositions expressed by things like ``there are chairs''
are also loosely true.  They are false, but are nonetheless good
enough for certain purposes.

Initially, this seems like a bizarre claim.  After all, Merricks is
claiming that chairs {\em necessarily} do not exist.  According to
Merricks, ``chairs exist'', given its current meaning, could {\em
  never} be true.  If the proposition expressed by ``chairs exist'' is
necessarily false, how could it nonetheless be ``nearly as good as
true''?

\subsection{The necessary connection}
\label{connection}
Merricks' argument relies on a very close conceptual connection
between ``chair'' and ``chairwise'' (and likewise for all ordinary
terms).  Despite claiming that chairs are impossible, Merricks admits
that we understand perfectly what chairs {\em would} be, if they
existed.  Because we understand the concept of `chair', we can
recognize {\em actually existing} things that are arranged
`chairwise':

\begin{squote}
The folk concept of \emph{statue} plays a role in determining which
atomic arrangements are statuewise. I would even go so far as to say
that if \emph{being arranged statuewise} were not derivative upon
folk-ontological concepts\,\ldots something would be amiss
(\citeyear[8]{merricks2001a}).
\end{squote}

For Merricks, to know what things are actually arranged statue- or
chairwise requires knowing what things would compose a chair, if
chairs were possible:

\begin{squote}
Atoms are \emph{arranged statuewise} if and only if they both have the
properties and also stand in the relations to microscopica upon which,
if statues existed, those atoms' \emph{composing a statue} would
non-trivially supervene (\citeyear[4]{merricks2001a}).
\end{squote}

When we look at Peter Unger's arguments for nihilism, we will see that
this close conceptual connection between terms like `chairwise' and
`chair' will cause trouble for Merricks.  But is Merricks' position
plausible anyway?  Does he have a good explanation of why we believe
in chairs?

Merricks' explanation is, at least, plausible.  One reason is the
structure of the explanation.  Recall that our explanation of why we
believe that there are chairs (or statues) is that, first, there are
chairs, and, second, we see that there are chairs (or learn that there
are chairs through a similarly reliable mechanism).  

Merricks' definition of `nearly as good as true' allows us to produce
a parallel explanation:

\begin{squote}
Any folk-ontological claim of the form `F exists' is \emph{nearly as
  good as true} if and only if (i) `F exists' is false and (ii) there
are things arranged F-wise. So, for example, `the statue \emph{David}
exist' is nearly as good as true because (it is false and) there are
some things arranged Davidwise (\citeyear[171]{merricks2001a}).
\end{squote}

We may now say on behalf of Merricks that we believe that there are
chairs (and statues) because, first, there are things arranged
chairwise and, second, we see that there are things arranged
chairwise.

The structure of the two explanations is analogous, but there is an
apparent disanalogy in the content of the two.  The disanalogy does
not favor Merricks.  For it is easy enough to see why there being
chairs, and us seeing that there are chairs, would cause us to believe
that there are chairs.  But it is less obvious why there being things
arranged chairwise, and us seeing that there are things arranged
chairwise, would cause us to believe {\em not} that there are things
arranged chairwise, but that there are {\em chairs}.

(While it is certainly true that we believe that there are chairs, I
am not sure if all or even most of us {\em also} believe that there
are things arranged chairwise.  Let us suppose for now that we do.)

The close conceptual connection between `chair' and `chairwise' is
very important for Merricks.  It is this {\em connection} that is
doing the explanatory work.  The only thing that can explain why there
being things arranged chairwise would cause us to believe that there
are chairs is this connection between the concepts.  The existence of
things arranged chairwise, and the belief that there are things
arranged chairwise, is supposed to cause the {\em additional} belief
that there are chairs.  How does this happen?

I believe that Merricks' answer would go something like this: certain
arrangements of things---chairwise arrangements, statuewise
arrangements, and all ordinary arrangements---play important roles in
our lives.  These arrangements of things are of interest to us, so we
have developed words that allow us to refer to them.  For whatever
reason---sociological, psychological, or otherwise---we think of each
arrangement as a single thing, rather than as things.  Words like
`chair' and `statue', being singular, reflect this (incorrect) view of
the world.  We are, in a sense, fooled by grammar.

This is more than Merricks says himself.  I have not found a passage
in which he explicitly describes the nature of the conceptual
connection between concepts like `chair' and `chairwise', and explains
why, from our belief that there are things arranged chairwise, we
invariably infer that there are chairs.  But I think he would endorse
something like this.  In the first chapter of his book, he claims that
whether there is a statue or merely things arranged statuewise is not
an empirical question.  He claims that were there not a statue and
merely things arranged chairwise, our ``visual evidence'' would be the
same.  He supports this claim with an analogy:

\begin{squote}
{[}Consider{]} the claim that the atoms arranged my-neighbour's-dogwise
and the-top-half-of-the-tree-in-my-backyardwise compose an
object\ldots{}it won't do to defend this claim with nothing more than `I
can \emph{just see} the object composed of the atoms arranged
dog-and-treetopwise'. Part of why this won't do, presumably, is that
one's visual evidence would be the same \emph{whether or not} those
atoms composed something (\citeyear[8--9]{merricks2001a}).
\end{squote}

He relies here on his assumption that we do not believe that there is
a thing composed of a dog and some of a tree (I will challenge this
assumption below).  Later he says that it is (at least possibly)
arbitrary to claim that there are statues but not dog-tree things:
``we ought to see that the only difference between arbitrary sums and
statues is a matter of conventional wisdom and local custom''
\citeyearpar[75]{merricks2001a}.  He seems sympathetic to the idea
that the reason we believe that there are statues, and not dog-tree
composites, is due to our conventional speech practices: ``it is at
least somewhat plausible that atoms arranged statuewise are united not
by composing something but, instead and in part, by how we speak and
think'' \citeyearpar[121]{merricks2001a}.

On this picture, whether we see an arrangement of things as composing
an object or not depends more on our own interests than features of
the things themselves.  We have words for chairs and statues because
things arranged chairwise and statuewise interest us.  We don't have a
word for things arranged ``my-neighbor's-dogwise and
the-top-half-of-the-tree-in-my-backyardwise'' because such an
arrangement does not hold much interest for us.  But each of these
arrangements exist, and it seems arbitrary to say that the chairwise
and statuewise arrangements compose chairs and statues while the other
arrangement composes nothing.

Merricks might explain why we believe that there are things arranged
my-neighbor's-dogwise and the-top-half-of-the-tree-in-my-backyardwise
thus: there are things arranged my-neighbor's-dogwise and
the-top-half-of-the-tree-in-my-backyardwise, and we see that there are
things so arranged.  This is exactly the same explanation that I would
give.

Now Merricks explains why we believe that there are chairs thus: there
are things arranged chairwise, and we see that there are things
arranged chairwise.  {\em And incidentally, due to our own human
  peculiarities, we have found it convenient to refer to and think
  about things arranged chairwise as if they were single objects}.

\section{Dogbushes}
\label{dogbush}
This is a somewhat plausible explanation of why we would belief that
there are chairs if there were not.  It is certainly much better than
van Inwagen's.  But I think that it fails.  I think that when we look
closer at Merricks' attempts to motivate nihilism, we will see that
they do not support nihilism at all.  If anything they support a
version of {\em universalism}.

Merricks observes that one might object to nihilism simply by saying,
``I just {\em see} the chair!''  He claims that if this objection
moves us, we should think about an analogous objection, which he finds
much less moving:

\begin{squote}
Whether atoms arranged statuewise compose a statue is analogous to
whether atoms arranged my-neighbour's-dogwise and
the-top-half-of-the-tree-in-my-backyardwise compose an object\,\ldots
it would not do to support an affirmative answer to the latter
question simply by saying `I can just see that object'
\citeyearpar[73]{merricks2001a}.
\end{squote}

It does indeed seem plausible to say that the top half of a tree and
my neighbor's dog do not `compose' anything.  But I think this is
incorrect, and I think we find it plausible only because of the word
`compose'.

Recall the bliger story that van Inwagen used to motivate his version
of nihilism (section \ref{prop-ont}).  A bliger was supposed to be
four monkeys, an owl, and a sloth, who arrange themselves into a
temporary symbiotic configuration.  Van Inwagen thought we would agree
that bligers did not exist.  Like Merricks, he brought in the language
of `composition' to make his claim more plausible.  Van Inwagen says
that it is not true that ``six animals arranged in bliger fashion
compose anything, and that is what I mean to deny when I say that
there are no bligers'' \citeyearpar[104]{inwagen1995}.

But as we saw, it is simply false that there are no bligers:

\begin{squote}
\ldots {\em of course} there are bligers in [van Inwagen's] story.
Bligers are what the story is about.  The zoologists do not report
that there are no bligers.  Rather they tell us what a bliger is.
They explain that a bliger is not a single large carnivorous animal
but a transient symbiotic union of six animals
\citep[704]{rosenberg1993}.
\end{squote}

The only reason we might be tempted to say that there are no bligers
is that van Inwagen presents the question in an unintuitive way.  He
asks us if there is some thing, some object, that is composed of the
other six animals.  This gives one the impression that, were there to
be such a thing, it would be another animal (a seventh); were there
such a thing, it should somehow pop out at us.  But all we see when we
picture the scene are the six animals together, so we feel that van
Inwagen might be right.  There is no {\em other} thing.  But if we ask
the question in a more intuitive manner, things become clearer.
Rather than ask if there is some thing composed of such and such other
things, we simply ask, ``are there bligers?''  And of course there
are.  Van Inwagen relied too heavily on the term `composition', and as
a result concluded that ``there are bligers'' was not only less than
obviously true, but actually false.

Merricks makes the same mistake in his passage above.  Following van
Inwagen, let's give our composite object a name.  For simplicity's
sake, I'm also going to change the example slightly.  I hereby
introduce `dogbush' as a term designating trees and dogs that are
within 3 meters of each other.  This term applies to situations like
these:

\begin{itemize}
  \item At the park, there are six trees, each with a (single) dog
    sleeping under it.  There are six dogbushes in the
    park. \label{park}
  \item In the above example, if one dog walked away from its tree,
    there would be one less dogbush.
  \item If that dog then walked under another tree (where another dog
    was sleeping), we would again have six dogbushes.
\end{itemize}

Van Inwagen and Merricks might now ask us if there is object composed
of a tree and a dog when they are within 3 meters of each other.  But
let's set aside the language of `composition' and simply ask, ``are
there dogbushes?''

{\em Of course} there are dogbushes.  There are trees, and there are
dogs within 3 meters of those trees.  That's all it takes to make
``there are dogbushes'' true.  How could ``there are dogbushes'' be
false; how could there be no dogbushes?  Are there {\em no} dogs
within 3 meters of a tree?  There obviously are, and so it is
obviously true that there are dogbushes.  It is correct to say that a
dogbush is a thing composed of a dog and a tree, but to rely on that
formulation invites misinterpretation.

\subsection{Visual evidence}
\label{visual}
Merricks claims that our ``visual evidence'' would be the same whether
or not there were dogbushes, but this is simply false: if there were
no dogbushes, that would mean that there were no dogs within 3 meters
of trees.  Imagine if van Inwagen had claimed that the visual evidence
of his imagined farmers would be the same whether or not there were
bligers.  This would be nonsense, for if there were no bligers then
that would mean that owls, monkeys, and sloths never formed their
symbiotic unions.  And, {\em by hypothesis}, they do form such
unions.  Farmers see these unions.  Farmers would not see these unions
if there were no bligers.  Their visual evidence would certainly not
be the same.

The same point applies to dogbushes.  I believe there are dogbushes
because I {\em see} them.  In making this claim I am, apparently,
alone:

\begin{squote}
There are many philosophers who believe in arbitrary sums like the
`dog-and-treetop', but none of them---not one---defends the existence
of such things on merely perceptual grounds. No one says we should
believe that such an object exists simply because we can see it or
simply because we can hear it (gnawing on a bone while rustling its
leaves) \citep[74]{merricks2001a}.
\end{squote}

Merricks claims rather that, if we believe in such things, our belief
should be defended on {\em philosophical} grounds.  ``Merely
perceptual grounds'' are not good enough.  The point of this, of
course, is to undermine our belief that there dogbushes.  Once
Merricks has convinced us that there are no dogbushes, the trap is
sprung.  For what, really, is the difference between dogbushes and
ordinary things like statues?  Merricks thinks ``we ought to see that
the only difference between arbitrary sums and statues is a matter of
conventional wisdom and local custom.  Once this is pointed out, one
is no longer justified in believing that statues exist merely because
one can supposedly see them'' \citeyearpar[75]{merricks2001a}.  On the
contrary, I say that once this is pointed out, one is no longer
justified in believing that dogbushes {\em don't} exist.  For our
visual evidence would {\em not} be the same if there were none.
(Recall my sample uses of `dogbush' in section \ref{park} above, in
which there were six, and then later five, dogbushes in the park.  If
at any point there were none, that would be because the dogs had all
left their trees.)

Likewise, if there were no chairs, our visual evidence would certainly
not be the same.  If there were no chairs, there would not be any
things arranged chairwise either.  And Merricks agrees that there are
things arranged chairwise.  Now how could ``there are chair'' be
false; how could there be no chairs?  Are there {\em no} things
arranged chairwise?  There obviously are, and so it is obviously true
that there are chairs.

Similar examples are readily available:

\paragraph{The accidental art} \label{towel}
Pranksters break into a museum to install joke pieces of art.  One one
wall they put up a bathroom mirror and towel ring (complete with
towel).  Under the mirror they put a little sign reading ``Wash your
hands''.  The installation is accepted as art by the gullible curator,
who gets an equally gullible journalist to write about it.  {\em Wash
  Your Hands} quickly becomes a valuable piece of art---valuable
enough that art thieves target it.  They break into the museum in
order to steal {\em Wash Your Hands}, but trip and alarm and are
forced to flee.  All they get away with is the towel.  In the morning
the guards tell the curator that part of {\em Wash Your Hands} is
missing.  The curator orders them to remove the rest of the piece and
informs crestfallen visitors that {\em Wash Your Hands} is no longer
in the museum's collection.

Here, the only point at which is it true to say that {\em Wash Your
  Hands} is not in the museum is when it is finally removed.  Someone
who claimed that it was {\em never} in the museum because it doesn't
exist would be saying something quite clearly false.

\paragraph{The curious archipelago} \label{roman}
This example is based on one by John Hawthorne
\citeyearpar{hawthorne2008}.  Archipelagos obviously exist.  There are
many archipelagos in the world.  They are, as one might put it,
``scattered objects''.  The archipelago is made up of a number of
separate islands, but it is nonetheless a thing.  It is an
archipelago.  Now there is (let us suppose) an archipelago in the
Mediterranean Sea.  This archipelago is called the Roman Archipelago,
due to the fact that there are a number of Roman ruins on one of its
islands.  There are several research camps on the islands, where
archaeologists dig for artifacts.  Their researches result in a
surprising discovery: one of the islands {\em is} a Roman ruin.  What
was thought to a rocky and curiously shaped island is in fact a
massive collapsed temple.  Further investigation reveals that another
island is made up of the bones of an ancient sea-dragon, and another
island is a crashed UFO.

Despite these extraordinary circumstances, it is nonetheless true that
the Roman archipelago exists.  It just happens to be composed of
several islands, a Roman ruin, a pile of old bones, and an alien
spacecraft.  To say the Roman Archipelago does not exist would mean
that these things are {\em not} sitting in the Mediterranean Sea.  (It
so happens that I made this story up, so the Roman Archipelago in fact
doesn't exist, but it does in the story.)

\paragraph{The universe} \label{world}
What would it mean to say that the universe doesn't exist?

\subsection{The meaning and truth-conditions of `chair'}
\label{meaning}
I am claiming that if things like dogbushes and wacky archipelagos did
not exist, that would mean that there were no dogs near trees or no
such objects in the (fictional) Mediterranean.  Merricks might
therefore take me to be accusing him of `linguistic contradiction'.
He might think that I think that `there are chairs' means `there are
things arranged chairwise'; to affirm the one and deny the other would
be absurd.  His response to this sort of objection is as follows:

\begin{squote}
`There are married bachelors' is no explicitly formally contradictory,
  but it is contradictory in some quite straightforward sense.  And
  one might object that `there are atoms arranged statuewise but no
  statues' is contradictory in the same way.  For as `bachelor' means
  someone who is, among other things, unmarried, so---the objector
  insists---`there are [composite] statues' {\em just means} that
  there are some things arranged statuewise.  Because of its
  contradictory nature, we should not take seriously an ontology
  according to which there are married bachelors.  Likewise, this
  objection concludes, we should not take seriously the
  eliminativist's ontology, with its atoms arranged statuewise but no
  statues \citeyearpar[13]{merricks2001a}.
\end{squote}

Merricks points out that ``\,`There are statues' does {\em not} mean
only that there are some things arranged statuewise\,\ldots this is
simply not a plausible claim about ordinary meaning''
\citeyearpar[13]{merricks2001a}.  And this seems quite true.  But I do
not need to claim that ``there are statues'' (or chairs) {\em means}
that there are things arranged statuewise (or chairwise).  All I need
to claim is that the following is true:

\begin{squote}
`There are chairs' is true if and only if `there are things arranged
  chairwise' is true.
\end{squote}

(If, as some claim, meaning is reducible to truth-conditions, then I
would be committed to the additional claim that `there are chairs' and
`there are things arranged chairwise' mean the same thing.  But if
Merricks does not hold this thesis about meaning, then he has no
grounds to resist the distinction between two propositions meaning the
same thing and their being truth-conditionally equivalent.  It is only
the latter that I claim.)

The claim above can be generalized:

\begin{description}
  \item[The $F$ existence principle] `There is an $F$'' is true if
    and only if ``There are things arranged $F$-wise'' is
    true. \label{fwise}
\end{description}

Nihilists like Trenton Merricks have assumed that what is required for
``there are chairs'' to be true is something more than what is
required for ``there are things chairwise'' to be true.  But this is a
mistake.  Ted Sider makes this mistake in a recent paper on parthood:

\begin{squote}
Consider Nihilo, god and creator of a world comprised solely of
subatomic particles.  On the first day Nihilo creates some particles
and arranges them in beautiful but lifeless patterns.  But he becomes
lonely, so on the second day he creates some minions (or rather,
particles arranged minion-wise).  On the third day he tries to teach
his minions to speak.  But this goes badly.  The minions aren't very
bright, and are slow to catch on to Nihilo's talk of subatomic
particles and their physical states.  So on the fourth day he teaches
them an easier way to speak.  Whenever an electron is bonded (in a
certain way) to a proton, he teaches them to say ``there is a hydrogen
atom''; whenever some subatomic particles are arranged chairwise he
teaches them to say ``there is a chair'', and so on.  (Pretend that
electrons and protons have no proper parts.)

When the minions utter sentences like ``there is a hydrogen atom'', do
they speak falsely?  They do if their language is the same as the
language I used to describe the example, since I described Nihilo as
having created a world comprised solely of subatomic particles
\citeyearpar[7]{sider2011c}.
\end{squote}

Sider seems to be thinking that he can stipulate that the
`metaphysical laws' of his imagined world make it impossible for there
to be minions.  But he has clearly not succeeded, for there are
minions right there in his story!  Nihilo created minions.  In
Nihilo's world, ``there are minions'' is true.  So is ``there are
hydrogen atoms''.  When the minions utter such sentences, what they
say is true.  It is true even if their language is the same language
as Sider used to describe the example (the language is English; recall
section \ref{english} above).  For there to be minions, nothing is
required over and above there being things arranged minionwise.
``There are minions'' is true if and only if ``there are things
arranged minionwise'' is true.

At one point Merricks attempts to motivate his nihilism by claiming
that ``whether atoms arranged statuewise compose a statue is not
straightforwardly empirical'' \citeyearpar[9]{merricks2001a}.  After
reminding us that our visual evidence would be the same whether or not
things arranged statuewise composed a statue, he writes:

\begin{squote}
The fundamental question is not so much whether some particular
alleged statue exists.  That question might---sceptical scenarios
aside---seem to be a matter of just looking and seeing.  The issue is
rather whether, in general, atoms arranged statuewise compose a
statue.  But it seems that this question of metaphysical necessity
cannot be decided, one way or another, simply by a trip to the museum
or a ride down Monument Avenue.  It must be decided on philosophical
grounds \citeyearpar[9]{merricks2001a}.
\end{squote}

I disagree.  The question ``whether things arranged statuewise compose
a statue'' is equivalent to the question whether there are statues.
If the answer to one is ``yes'', the answer to the other is also
``yes''.  The answer to the latter is ``yes''.  So the answer to the
former is ``yes''.  The answer to both is ``yes'', and this just {\em
  is} an empirical question that can be settled ``by a trip to the
museum''.

\section{Universalism}
\label{universalism}
Above I claimed that Merricks' explanation of why we believe that
there are chairs is something like this: there are things arranged
chairwise, and we see that there are things arranged chairwise.  {\em
  And incidentally, due to our own human peculiarities, we have found
  it convenient to refer to and think about things arranged chairwise
  as if they were single objects}.  I attributed to Merricks the idea
that just because things arranged chairwise interest us, we should not
therefore suppose that there are chairs.  What interests us should not
be a guide to what exists.  But now there is an obvious counter
against this move.  Just because dogbushes do {\em not} interest us,
we should not therefore suppose that there are not dogbushes.  (What
interests us should not be a guide to what exists.)

I said in section \ref{scq-ans} that I would postpone discussion of
universalism until later.  It it time now to discuss it.  For I seem
to be committing myself to some version of universalism.  The
formulation I cited above was this:

\begin{description}
\item[Universalism] Necessarily, for any $x$s, there is an object
  composed of the $x$s iff no two of the $x$s overlap
  \citep[227]{markosian1998a}.
\end{description}

Much of the resistance to this thesis (at least, much of my early
resistance) probably stems from its wording; as I argued above, the
unintuitive language of `composition' can lead us to deny obvious
truths, such as that there are chairs and dogbushes.  Ned Markosian's
objection to universalism provides another example of how the term
`composition' obscures the plausibility of universalism:

\begin{squote}
There is what seems to me a fatal objection to Universalism:
Universalism entails that there are far more composite objects than
common sense intuitions can allow.  To give just one example,
Universalism entails that the following sentence is true:\,\ldots
There is an object composed of (i) London Bridge, (ii) a certain
sub-atomic particle located far beneath the surface of the moon, and
(iii) Cal Ripken, Jr.  My intuitions tell me that there is no such
object, and I suspect that the intuitions of the man on the street
would agree with mine on this point \citeyearpar[228]{markosian1998a}.
\end{squote}

I suspect that Markosian's intuitions are led astray because he relies
so heavily on the term `composition'.  As in the case of the dog and
the tree, it does seem unintuitive that the tree and dog `compose'
something else.  But it is obvious that there are dogbushes.  Let us
therefore drop `composition' from Markosian's example and see if that
makes things clearer.

Let `Lumpkin Junior' designate the London Bridge, a particle in the
moon, and Cal Ripkin, Jr.  Does Lumpkin Junior exist?

Let us introduce another term so as to bring the analogy closer.  A
`lumpkin' is the London Bridge, a sub-atomic particle in the moon, and
a retired major league baseball player.  Now we can ask ``are there
lumpkins?''  And, as in the case of the dogbush, {\em of course there
  are}.  If there were no lumpkins, there would have to be either no
former ball-players, no London Bridge, or no particles in the moon.
Given that there are all these things, there are obviously lumpkins.
Lumpkin Junior is one of many lumpkins.

I cannot resist including another example, partly for its silliness
and partly in the hope that it will bring out further the way in which
terms like `composition' are misleading.  The Blues Brothers have a
song called ``Rubber Biscuit''.  In it, they refer to a `wish
sandwich'.  ``A wish sandwich,'' they say, ``is the kind of a sandwich
where you have two slices of bread and {\em wish} you had some meat.''
This term having been introduced, one can truthfully say, ``I had a
wish sandwich the other day.''  All that is required for this to be
true is that she had two slices of bread and wished she had some meat.
There was then a wish sandwich.  Should we say that there was some
`composite' thing, `composed of' two slices of bread and a wish?  We
can talk this way, I suppose, but it serves little purpose but to
confuse us.

%% Rather than risk being mislead by unintuitive formulations of
%% universalism, I will rely on and defend the $F$ existence principle
%% from \ref{fwise}: ``There is an $F$'' is true if and only if
%% ``There are things arranged $F$-wise'' is true.

\subsection{Begging the question}
\label{beg}
The most straightforward objection to the above is that I am begging
the question.  For I claim that it is {\em true} (if misleading) to
say that Lumpkin Junior is ``that object composed of the London
Bridge, a particle of the moon, and Cal Ripkin, Jr.''  So when
Markosian denies that Lumpkin Junior exists, my `counterargument' is
just that it does exist!

Above, I attempted to make it plausible that Lumpkin Junior exists.  I
claimed that the wording used by Markosian and others is unintuitive
and causes us to come to incorrect conclusions.  I think that if we
formulate the question more naturally (``are there lumpkins?''), it is
intuitively true that there {\em are} such things.  I think this is an
appropriate way to proceed, given that Markosian's argument {\em
  against} lumpkins is an appeal to intuitions as well.  He says is
that ``[his] intuitions tell [him] that there is no such object, and
[he] suspect[s] that the intuitions of the man on the street would
agree with [his] on this point'' \citeyearpar[228]{markosian1998a}.  I
suggest that his intuitions are led astray by an awkward wording.

I think Markosian and I are going through the following argument in
different directions:

\begin{enumerate}
  \item There are dogbushes
  \item If there are dogbushes, then they are things composed of dogs
    and trees.
  \item There are things composed of dogs and trees.
\end{enumerate}

Markosian begins at the bottom.  He finds it implausible to say that
there are things composed of dogs and trees, and so, by {\em modus
  tollens}, denies that there are dogbushes.  I start from the top.  I
find it perfectly plausible to say that there are dogbushes.  By {\em
  modus ponens}, I claim that there are things composed of dogs and
trees.

I think it is more appropriate to begin from the top.  As I have said,
the phrasing ``there are dogbushes'' is more natural; therefore our
intuitive judgments regarding it should be more accurate.  When we
arrive at technical formulations like ``there is some thing such that
it is composed of a dog and a tree'', our intuitions will be less
reliable.  Even if that proposition seems intuitively false, if it
follows from one that seems clearly true, then we should not reject
it unless we have good independent reasons for doing so.

Another example: the Reed College women's rugby team.  There is,
obviously, such a team (whether or not it has College funding).  The
rugby team exists.  Having established this, there are various
consequences.  For instance, each player is part of the team.  The
team is made up of the players and the coach.  Expressing this
formally, we might say that there is some thing (the team) composed of
the $x$s (the players and coach).  If this formal treatment is
equivalent to the informal ``there is a team'', and if the informal
phrasing is uncontroversial, the formal phrasing should not be
controversial either.

Merricks has a related example:

\begin{squote}
Consider whether `the Crew of the USS {\em Enterprise}' is a plural
referring expression---akin to `Locke, Berkeley, and Hume'---or,
instead, the name of a single large object with each crew member as a
proper part.  Note, in fact, that there are two questions here.
First, there is the semantic question of what `the Crew of the USS
{\em Enterprise}' is supposed to mean.  Second, there is the
metaphysical question of whether there really is a big physical object
that has all and only the crew members as its parts (at one level of
decomposition), a scattered object that weighs as much as the sum of
the weights of those people taken individually.

I am not sure how to answer the first question.  But, I say, the
answer to the second question is `no'.  Some philosophers would
disagree.  No matter.  The point here---in this section of this
chapter---is not to settle either the metaphysical or the semantic
dispute surrounding `the Crew of the USS {\em Enterprise}'.  It is,
rather, that such disputes are neither here nor there with respect to
everyday uses of `the Crew of the USS {\em Enterprise}'.  `The Crew of
the USS {\em Enterprise}' will continue to perform its ordinary duties
regardless of how or whether the semantic and metaphysical disputes
get settled \citeyearpar[10]{merricks2001a}.
\end{squote}

I will discuss plural referring expressions in a moment.  But first I
want to point out that the last sentence of this quote by Merricks
does not seem to be true.  Suppose I believed that the crew of the
{\em Enterprise}, were it to exist, would be a thing composed of the
crewmembers, {\em and on those grounds} I denied that the crew exists.
If the crew does not exist, if there is no crew, then that could only
mean that the ship was unmanned.  If there are crewmembers, there is a
crew.  If there is a crew, then there are crewmembers.  If Merricks or
anyone denies that there is a crew, what would it mean for them to say
that `the crew' ``will continue to perform its ordinary duties''?

\subsection{Plural referring expressions}
In the quoted material above, Merricks suggests that terms like `the
crew of the USS {\em Enterprise}' may not be singular terms, but
plural referring expressions.  For example, `the Dunns' refers
plurally to me and the other members of my family.  I say things like
``the Dunns are fine people''; the term obviously does not function as
a singular term.  The suggestion is that `the crew' behaves similarly.
When I say that the crew exists, it would then {\em not} follow that
there is a {\em thing} composed of the crewmembers.  Saying ``the crew
exists'' would instead be equivalent to saying ``the crewmembers all
exist''.

I do not think that this a plausible claim.  Recall the analogy I
tried to draw between `the crew' and `the Dunns'.  On closer
inspection, this analogy appears weak.  A stronger analogy would be
between a term like `the crew' and a term like `the Dunn family'.
`The Dunn family' is {\em not} a plural referring expression.  It is
used to refer to a {\em thing}.  If I talk about `the Dunn family', I
would say things like, ``The Dunn family is waning'', or ``The Dunn
family must regain its political power''.  The term `the Dunn family'
is a singular term that designates a thing---the family.

`The crew' appears to behave like `the Dunn family' and not like `the
Dunns'.  We say things like ``there is a skeleton crew on board'', or
``the crew is small for such a large ship'', and ``the crew is
abandoning the ship''.  Were we to say things like ``the crew {\em
  are} abandoning the ship'', this would most likely be a case of
non-literal speech; `the crew' is being used non-literally to refer to
the crewmembers.

The phenomenon of non-literal talk can lead us astray when thinking
about terms like `team' as well.  `The Reed College women's rugby
team' is a singular term, for it behaves in the same ways as do `the
crew' and `the Dunn family'.  We say things like ``The Reed College
women's rugby team is going to win'', or ``The Reed College women's
rugby team is in Seattle this weekend''.  However, non-literal speech
is more commonly used with teams than with terms like `family' or
`crew'.  This is often due to pluralized team names.  The Reed College
women's rugby team is called ``The Badass Sparkle Princesses''.  This
leads us to say things like ``The Badass Sparkle Princesses are on a
losing streak''.  Here we are led by the plural construction
to---perhaps unconsciously---use the term non-literally, referring not
to the team itself but to the players.  The Badass Sparkle Princesses
{\em is} a rugby team, but it is far more natural (yet not literally
true) to say that the Badass Sparkle Princesses are rugby players.

I will henceforth assume that terms like `crew', `family', and `team'
are not plural referring expressions, but rather singular terms that
designate things---crews, families, and teams.  (Similar
considerations will convince us that `dogbush' is a singular term too,
and not a plural referring expression.)

\section{Parthood and the language of composition}
\label{part}
Things like teams, crews, and families are indeed {\em things}.  They
are not disguised references to plurals, nor are they ``mere
collections'' \citep[29]{inwagen2009}.  Moreover, things like teams
are things with {\em parts}.  The rugby players are each {\em part} of
the Reed College women's rugby team.  The team is made up of---it is
composed of---the players.

When I say that the players are part of the team, or that the
crewmembers are part of the crew, or that I am part of my family, is
that use of `part' the same as when I say that the tree is part of the
dogbush, or that the seat is part of the chair?  Are {\em any} of
these uses of `part' the same?

Technical notions of composition are often defined in terms of
parthood.  How I am using the term `part' will therefore influence how
I construct formal equivalences for propositions like ``there are
chairs'', ``there are dogbushes'', and ``there are teams''.  Is the
appropriate formalization for each of these the same?  Can chairs,
dogbushes, and teams each `fit in' to the schema ``there is an $x$
such that it is composed of the $y$s''?  Or is `composition' one of
many {\em operations} that `produce' things?

\subsection{Van Inwagen's notion of parthood}
\label{van-part}
Van Inwagen defines his technical notion of composition (see section
\ref{scq}) in terms of a largely intuitive notion of parthood.  Van
Inwagen's interest, however, is restricted to `material' objects
(objects made exclusively of quarks and protons, or whatever the basic
atoms of the physical world turn out to be).  While he goes on to use
`part' only in reference to material objects, he recognizes that the
term has much wider application:

\begin{squote}
Parthood will occupy a central place in the present study of material
objects.  It is therefore worth noting that the word `part' is applied
to many things besides material objects.  We have already noted that
submicroscopic objects like quarks and protons are at least not clear
cases of material objects; nevertheless, every material object would
seem pretty clearly to have quarks and protons as \emph{parts}, and,
it would seem, in exactly the same sense of \emph{part} as that in
which a paradigmatic material object might have another paradigmatic
material object as a part.  A ``part,'' therefore, need not be a thing
that is clearly a material object.  Moreover, the word `part' is
applied to things that are clearly \emph{not} material objects---or at
least it is on the assumption that these things really exist and that
apparent reference to them is not a mere manner of speaking.  A stanza
is a part of a poem; Botvinnik was in trouble for part of the game;
the part of the curve that lies below the x-axis contains two minima;
parts of his story are hard to believe\,\ldots\,such examples can be
multiplied indefinitely.  Does this word `part' mean the same thing
when we speak of parts of cats, parts of poems, parts of games, parts
of curves, and parts of stories \citeyearpar[18--19]{inwagen1995}?
\end{squote} 

Van Inwagen suggests that `part' does have a number of different
meanings.  Later he says that ``there is one relation called
`parthood' whose field comprises material objects\,\ldots\,another
relation called `parthood' defined on events, another still defined on
stories, yet another defined on curves, and so on''
\citeyearpar[19]{inwagen1995}.

One reason why we might resist this conclusion is that it appears to
rule out the silly example of the wish sandwich.  A wish sandwich,
recall, is the kind of sandwich where you have two slices of bread and
wish you had some meat.  The slices of bread and the wish for meat are
all parts of the wish sandwich.  But if they are parts of the sandwich
in different ways, then in what sense did I use `part' in the
preceding sentence?  If it was part$_b$---the parthood relation for
foodstuffs---then the sentence was false, because a wish does not
partake in that sort of relation.  If the relation was that of
parthood$_w$---parthood for wishes---then the sentence would again be
false, because {\em that} relation does not govern foodstuffs.  I
could instead say ``the slices of bread are part$_b$ of the sandwich
and the wish is part$_w$ of the sandwich'', but it still seems to me
that the original sentence is {\em true}.  Might this suggest that
there is really just one parthood relation that both foodstuffs and
wishes partake in?

Moreover, our modified sentence still faces a difficulty.  For the
notion of composition is generally defined in terms of parthood.
Since van Inwagen's technical definition of composition is given in
terms of his notion of parthood, `composition' for van Inwagen can
only apply to things whose parts are all material things.  Van
Inwagen's notion of composition cannot make sense of the wish
sandwich, or any thing with both material and non-material parts.
(Van Inwagen does not see this as a disadvantage; he finds mysterious
the idea that there could be something composed of ``you and I and the
number two'' \citeyearpar[20]{inwagen1995}.)

We could, perhaps, define `composition' in terms of not just one
parthood relation but all of them.  Composition would take into
account all possible ways there are of being a part.  Kit Fine has
proposed a theory of parthood that takes seriously the possibility
that there are a plurality of different parthood relations.

\section{Fine's theory of part}
\label{fine}
Fine agrees with van Inwagen that the notion of parthood should not be
reserved only for material things:

\begin{squote}
Philosophers have often supposed the notion of part only has proper
application to material things or the like and that its application
to abstract objects such as sets or properties is somehow improper
and not sanctioned by ordinary use.  But I suspect that this is
something of a philosopher’s myth.  We happily talk of a sentence
being composed of words and of the words being composed of
letters---and not just the sentence and work tokens, mind, but also
the types.  And similarly, a symphony (and not just its performance)
will be composed of movements, a play of acts, a proof of steps.  I
wonder how many of these philosophers have said such things as ``this
paper is in three parts.''  When they have, then I very much doubt
that they would have any inclination, as ordinary speakers of the
language, to add ``but not, of course, in a strict or literal sense of
the term''; and the intended reference here is not primarily---or
perhaps not at all---to the tokens of the paper but to the type of
which they are the tokens.  The evidence concerning our ordinary talk
of part is mixed and complicated, but it does not seem especially to
favor taking material things to be the only true relata of the
relation \citeyearpar[561]{fine2010}.
\end{squote}

But even if one accepts the idea that there are things other than
`material things' that have parts, one might object that they are
still all parts in the same sense.  This might be the parthood
relation of classical mereology, or it might be some other, general
relation.  Moreover, one who maintained the univocality of parthood
can still concede that there are different ways of being a part.  But
for the believer in the univocality of parthood---the `monist'---these
are only {\em derivative} kinds of parthood.  For example, for any
given mereological sum, there are bigger and smaller parts of it. But
these are bigger and smaller parts of the same {\em kind}.  The
pluralist goes further and claims that there are parts of different
kinds. These different kinds are not derivative but \emph{basic}; they
are ``not definable in terms of other ways of being a part''
\citep[561]{fine2010}.  Fine gives a number of reasons to think that
there might be different basic parthood relations:

\begin{squote}
Now, on the face of it, there would appear to be a wide variety of
basic ways in which one object can be a part of another.  The letter
`n' would appear to be a part of the expression `no', for example, and
a particular pint of milk part of a particular quart; and if these two
relations of part are not themselves basic (perhaps through being
restricted to expressions or quantities), there would appear to be
basic relations of part that hold between `n' and `no' or the pint and
the quart.  It is also plausible that the way in which `n' is a part
of `no' is different from the way in which the pint is a part of the
quart.  For if the two ways were the same, then how could it be that
two pints were only capable of composing a single quart, while the two
letters `n' and `o' were capable of composing two expressions, `no'
and `on' \citeyearpar[562]{fine2010}?
\end{squote}

The parthood relation for sets is again different.  The set containing
the only the letters `n' and `o' has the letters as parts.  When the
letters are parts of a set, their order is irrelevant, but when the
letters are parts of a word, order matters; hence `no' and `on'.  The
parthood relation for sets is also different from the parthood
relation for quantities (of milk):

\begin{squote}
If four quarts compose a gallon the pints which compose the quarts
will compose the gallon in the same way in which they compose the
quarts, whereas, if four sets compose a further set the members of the
sets will not compose the further set in the same way in which they
compose the component sets.  Thus we would now appear to have three
different basic ways in which one object can be a part of another
(pint/gallon, letter/word, and member/set); and once these cases have
been granted, it is plausible that there will be many more
\citeyearpar[562]{fine2010}.
\end{squote}

One might, of course, refuse to grant these cases.  But one would have
to refuse them {\em all}; for if it can be established that there are
even two different (basic) ways of being a part, then the pluralist
position is established.  Once it is established that there are at
least two ways of being a part, it becomes much more plausible that
there might be three ways, or more.  Fine therefore attempts to
motivate the idea that the members of a set are, quite literally,
parts of the set.

\subsection{Parts of sets}
\label{sets}
The first objection is that while parthood is supposed to be
transitive, the membership relation of sets is not.  The letter `n' is
a member of the set \{`n',\{`n',`o'\}\}, but `o' is not.  The
objection claims that sets have {\em members}, not parts, and that
Fine has confused the two.

But while it is true that the membership relation is not the parthood
relation, this is no reason to think that sets do not have parts.  A
given set will have certain members---the $x$s---and certain
parts---the $y$s---and only sometimes will the $x$s and the $y$s be
the very same things.  The set \{`n',\{`n',`o'\}\} has two members
but three parts.  The parthood relation for sets can even be defined
in set-theoretic terms:

\begin{squote}
It may well be thought that the way in which a member is a part of a
set is given, not by the membership relation itself, but by the
ancestral of the membership relation, where this is the relation that
holds between $x$ and $y$ when $x$ is a member of $y$ or a member of a
member of $y$ or a member of a member of a member of $y$, and so on
\citep[563]{fine2010}.
\end{squote}

A second objection is that talk of parthood in connection with things
like sets is somehow metaphorical or non-literal.  We saw above that
van Inwagen admits that many different things are said to have parts.
However, he qualifies this in two ways.  First, he seems to have
doubts (or at least is sympathetic with those who have doubts) as to
whether the non-material things that are said to have parts really
exist:

\begin{squote}
The word `part' is applied to things that are clearly \emph{not}
material objects---or at least it is on the assumption that these
things really exist and that apparent reference to them is not a mere
manner of speaking \citep[19]{inwagen1995}.
\end{squote}

If there are no such things as tennis matches or poems or papers, then
of course they do not have parts.  But I think it is obviously true
that there are such things.  This being so, what does it mean to say
that they have parts?  This is where van Inwagen's second
qualification comes in.  For he suggests not only that the `parts' of
tennis matches and poems are parts in a different way than are the
parts of a table, but that these different relations of parthood are
only tenuously connected.  Van Inwagen says that the various relations
of parthood (if such there be) are connected only by the ``unity of
analogy'' \citeyearpar[19]{inwagen1995}.  If the only similarity
between the parthood relation for poems and the parthood relation for
chairs is that they share the `analogy' of parthood, then is there
anything important or interesting about `parts' of poems?  Is the
parthood relation for sets likewise only interesting because of the
analogy with the parthood relation for chairs?

At least in the case of parthood for sets, the notion does not appear
to be wholly metaphorical:

\begin{squote}
In the case of set-membership, there would appear to be nothing that
might plausibly be taken to indicate that the talk of part-whole is
not to be taken literally. A set is indeed composed of or built up
from its members, and we should add that we may meaningfully
talk---and in the intended way---of \emph{replacing} one member of a
set with another.  Thus Aristotle in the set \{Plato, Aristotle\} may
be replaced with Socrates to obtain the set \{Plato, Socrates\}, with
the given set becoming a different set from what it was. In the case
of sets, our conception of members as parts seems to extend all the
way \citep[564]{fine2010}.
\end{squote}

But the second worry raised by van Inwagen remains.  Why should we
think that there is any {\em real} similarity between these different
parthood relations, other than the fact that we call them all
`parthood'?

\subsection{Operationalism}
\label{operation}
Fine's theory of {\em operationalism} helps answer this worry.
Various {\em operations} produce different things---mereological
summation produces mereological sums or fusions, the set-builder
produces sets, and so forth.  Parts are therefore {\em things} that
have been `combined', through one or more such operations, into a
single {\em thing}.  What is common to all parthood relations is that
from each set of parts is produced a {\em whole}.  This may simply be
metaphorical, but it is nonetheless an accurate description of the
result of the composition operator.  From parts (letters, atoms) we
can `make' something new (a word, a set, a chair).  What ties together
all the ways of being a part is that they are involved in an operation
that produces a single thing from a number of things:

\begin{squote}
In formulating the principles of mereology, it has been usual to take
the relation of part-whole or some associated relation (such as
overlap) as primitive.  But I believe that, in formulating a more
general theory, it is important to take the operation of composition
as primitive rather than the more familiar relation of part-whole.  In
the case of classical mereology, the operation of composition will
take some objects into the sum or fusion of those objects, while, in
the set-theoretic case, it will take some objects into the set of
those objects; and, in general, the operation of composition will be
the characteristic means (summation, set-builder, and so on) by which
a given kind of whole is formed from its parts \citep[565]{fine2010}.
\end{squote}

Each way of being a part can then be defined in terms of the related
composition operation:

\begin{squote}
Once given a compositional operation, a corresponding relation of part
may be defined in two steps.  We say first that $x$ is a component of
$y$ if $y$ is the result of applying $\sum$ to $x$ or to $x$ and some
other objects.  In other words, $y$ should be of the form $\sum
(x_{1}, x_{2}, \mathellipsis )$, where at least one of $x_1$, $x_2,
\mathellipsis$ is $x$.  Thus when $\sum$ is mereological summation the
components of an object will be mere parts, and where $\sum$ is the
set-builder the components of an object will be its members.  We may
then define $x$ to be a part of $y$ if there is a sequence of objects
$x_1$, $x_2, \mathellipsis x_n$, $n$ \textgreater{} $0$, for which $x
= x_1$, $y = x_n$, and $x_i$ is a component of $x_{i+1}$ for $i = 1$,
$2, \mathellipsis, n-1$. The parts of an object are the object itself,
or its components, or the components of the components, and so on
\citep[567--568]{fine2010}.
\end{squote}

The parthood relation for summation can therefore be seen to exhibit
reflexivity, transitivity and anti-symmetry:

\begin{description}
\item[Reflexivity] Each object is a part of itself.
\item[Transitivity] If $x$ is a part of $y$ and $y$ of $z$, then $x$
  is a part of $z$.
\item[Anti-symmetry] $x$ is a part of $y$ and $y$ of $x$ only when $x
  = y$ \citep[568]{fine2010}.
\end{description}

But not all definitions of parthood that issue from a composition
operator will exhibit these features:

\begin{squote}
When the underlying operation is summation, each object will be a part
of itself, since the unit sum of any object is the object itself, but
when the underlying operation is the set-builder, no object will be a
part of itself, since no object is ever an ancestral member of itself
\citep[569]{fine2010}.
\end{squote}

\subsection{Principles}
\label{principle}
Each composition operation will, according to Fine, be governed by
various principles:

\begin{squote}
I believe that the principles governing the basic forms of composition
will conform to a general template.  Variations in the principles for
the different forms of composition will then arise from variations in
how the template is to be filled in.  The template will comprise two
broad categories of principle---the {\em formal} and the {\em
  material} (though not quite in the sense of Husserl).  Among the
formal principles, we may distinguish between those that provide
conditions of application for the operation and those that provide
identity conditions; among the material principles, we may distinguish
between those that provide conditions for the presence of a whole (in
space and time or at a world) and those that specify the descriptive
character of the whole.  The presence conditions, in their turn, may
concern either the existence of the whole or its extension
\citeyearpar[569--570]{fine2010}.
\end{squote}

\paragraph{Formal principles}
The formal principles govern when composition occurs and when two
products of a composition operation are identical:

\begin{description}
  \item[Application] The application conditions are ``the conditions
    under which there are wholes of a given sort---which, on the
    operational approach, is a matter of stating when the result of
    applying the compositional operation to various objects will be
    defined'' \citep[570]{fine2010}.  For the summation operation, the
    application conditions are very lax: for any $n$ physical objects
    (where $n$ \textgreater{} 1), there is a thing composed of those
    objects.  The application conditions for the set-builder will be
    more limited; and other operations will be more restricted
    still. \label{fine-app}
  \item[Identity] Setting out identity conditions on Fine's
    operational approach ``is a matter of stating when a whole formed
    in one way by means of the compositional operation is the same as
    a given object or a whole that has been formed in some other way''
    \citeyearpar[570]{fine2010}.  For example, suppose $y$ is the
    result of applying the summation operator to $x$, and suppose
    $y^\prime$ is the result of applying the set-builder to $x$.  If
    $y \neq y^\prime$, it will be because of some difference in the
    respective operations that produced these composites.  As we will
    see below, the summation operator is defined such that its
    application to a single thing ($x$) produces that very thing.  The
    set-builder, however, produces the singleton of $x$, which is not
    $x$.  And whereas applying the set-builder to nothing produces the
    null set, the result of applying the summation operator to nothing
    might be undefined---nothing would result.
\end{description}

\paragraph{Material principles}
As above, there are two subcategories:

\begin{description}
  \item[Presence] Fine claims that ``there are two fundamentally
    different ways in which an object might be present in space or
    time; it may \emph{exist} in space or time, or it may be
    \emph{extended} (or \emph{located}) in space or time. Thus a
    material thing will exist in time but be extended in space while
    an event will be extended in both space and time''
    \citeyearpar[570]{fine2010}.  Whether or not this is actually true
    seems to depend on whether a `three-dimensional' theory of
    persistence is correct.  However, that question will remain unasked.
  \item[Character] ``The character conditions will tend to have a much
    more ad hoc character than the other conditions that we have
    considered. The color of a house, for example, is the color of its
    siding; the color of an egg, the color of its shell; the color of
    a pencil, the color of its lead. In the case of the `intrinsic'
    character of a thing---such as its mass or color---the character
    of the whole will be some sort of function of the character of the
    parts. But the function in question will vary from case to case''
    \citep[571]{fine2010}.  What we decide is the color of a house
    will turn on our concept `house', rather than anything about the
    house itself (anything beyond its having that color to {\em some}
    extent).  When the composition operation is summation, interesting
    questions also arise about weight; is the weight of a sum equal to
    the combined weights of the parts?  This seems intuitively so, but
    if someone's parts weight 150 pounds, and therefore they weight
    150, it does not follow that when they (and their parts) stand on
    a scale, the scale reads 300 pounds.  Figuring out what to say in
    these cases may ultimately feel ad hoc.
\end{description}

\subsection{Fine's pluralist account of classical mereology}
\label{classical}
Of the principles sketched above, Fine gives most attention to the
identity conditions for composition operations.  The composition
operation used as a paradigm is the summation operation of classical
mereology.  Fine's exposition of identity conditions for sums relies
on the notion of `regularity':

\begin{squote}
Call an identity condition $s = t$ {\em regular} if the variables
appearing in $s$ and in $t$ are the same.  Thus $\sum (x, y) = \sum
(y, x)$ is regular while $\sum (x, y) = x$ is not
\citeyearpar[572]{fine2010}.
\end{squote}

With this notion in hand, Fine proposes this condition for identity of
sums:

\begin{description}
  \item[Summative Identity] $s = t$ whenever `$s = t$' is a regular
    identity \citeyearpar[572]{fine2010}.
\end{description}

One particularly interesting aspect of this condition is that it
entails four more principles of the summation operation:

\begin{description}
  \item[Absorption] $\sum (\mathellipsis, x, x, \mathellipsis,
    \mathellipsis, y, y, \mathellipsis, \mathellipsis = \sum (
    \mathellipsis, x, \mathellipsis, y, \mathellipsis )$;
\item[Collapse] $\sum (x) = x$;
\item[Leveling] $\sum (\mathellipsis, \sum (x, y, z, \mathellipsis ),
  \mathellipsis, \sum (u, v, w, \mathellipsis ), \mathellipsis ) = \sum
  (\mathellipsis, x, y, z, \mathellipsis, \mathellipsis, u, v, w,
  \mathellipsis, \mathellipsis )$;
\item[Permutation] $\sum (x, y, z, \mathellipsis ) = \sum (y, z, x,
  \mathellipsis )$ (and similarly for all other permutations)
  \citep[573]{fine2010}.
\end{description}

We can define other compositional identity criteria (e.g., sequences)
in terms of which of these principles apply to their compositional
operation.  But we may also devise new principles by which we may then
define new types of composition:

\begin{squote}
We should note that there would appear to be no good reason to require
that the defining principles for the various operations should be
limited to the particular principles (C [collapse], L [leveling], A
[absorption], and P [permutation]) that we used in characterizing
sums; for any set of regular identities would appear to be equally
well suited to defining a basic form of composition, so long as they
conform to Anti-cyclicity.  Indeed, I would conjecture that any such
set of principles in fact will correspond to a form of composition and
a corresponding form of whole.  How the resulting forms of composition
and whole might be organized is an interesting question, but it should
be apparent that the approach will lead to an infinitude of forms of
composition, each differing from one another in how exactly the
identity of the resulting wholes is to be
determined. \citep[575--576]{fine2010}.
\end{squote}

It is at this point that the importance of Fine's theory becomes
obvious.  Above I stressed that things like teams and families are
really {\em things}; moreover I made this claim as part of an attempt
to motivate a sort of universalistic outlook on metaphysics.  I argued
that the term `composition' was potentially misleading, but that it
was nevertheless correct to say that things like dogbushes, wish
sandwiches, and teams are composed of their parts.  But now it is
apparent that `composition' will mean something different when applied
to each of these things.  Each thing will be the product of a
different composition operation.

Fine's theory reveals new {\em kinds} of universalism.  One might be
committed to the existence of dogbushes---and so to unrestricted
mereological composition---but deny the existence of teams, groups,
crews, and families.  Or one might defend unrestricted composition of
groups while claiming a restriction on mereological composition.

Below I will look at how a definition of the composition operator for
groups might be formulated.  But I will first return to Fine's theory.

\subsection{Hybrid parts}
\label{hybrid}
Above, we saw that one objection to the idea of sets having parts was
that parthood is transitive and set-membership is not.  Moreover it
was supposed (rather plausibly) that the only reason we think that
sets have parts is {\em because} they have members; it is the members
of a set that we are tempted to call parts.  But, the objection goes,
it is a mistake to think of members as part.  I am the only member of
my singleton (the singleton of $x$ is the set resulting from applying
the set-builder to $x$ alone).  My hand, for instance, is not a member
of my singleton.  But my hand is a part of me.  If I was a part of my
singleton, then---because parthood is transitive---my hand would be a
part of my singleton.  And if that means that my hand is a {\em
  member} of my singleton, that is clearly wrong.

Fine points out, of course, that the objection makes the mistake of
supposing that something (me, my hand) can be a part in only one way
(in this case, through set-membership).  Once we recognize that there
are a plurality of ways of being a part, it becomes clear that my hand
is part of the set in one way, but not in another:

\begin{squote}
Given the specific relations of part, we may derive various {\em
  hybrid} relations of part.  Suppose, for example, that we are given
the relations of set-theoretic and mereological part---which we may
designate as \textepsilon -part and $m$-part. We may then take one
object to be an \textepsilon ,$m$-part of another if it is an
\textepsilon -part or an $m$-part or an $m$-part of an \textepsilon
-part or an \textepsilon -part of an $m$-part, or an $m$-part of an
\textepsilon -part of an $m$-part, and so on. More generally, if $K$
is a family of specific ways of being a part, we may take an object to
be a {\em K-part} of another if $x$ and $y$ can be linked by
relationships of $k$-part for $k$ in $K$ \citep[579]{fine2010}.
\end{squote}

My hand is a \textepsilon ,$m$-part of my singleton, but not a
\textepsilon -part.

By conjoining every way of being a part, we arrive at the most general
notion of part:

\begin{squote}
Among the hybrid relations of part, of special interest is the
relation of $K$-part where $K$ is the family of {\em all} the specific
ways of being a part.  This is the relation of $K$-part that holds
between two objects when they may be linked by relationships of
$k$-part without restriction on $k$.  We might call it the {\em
  general} relation of part, and it is a relation that holds between
$x$ and $y$ whenever $x$ is in any way whatever a part of $y$
\citep[580]{fine2010}.
\end{squote}

\subsection{Generating kinds}
\label{generate}
On this theory, what kind a thing is depends on what operation
produced it.  If a chair or a dogbush is a mereological sum, then this
is because they are produced by the summation operation.  The Dunn
family is `produced' by the family operation.  Groups are produced by
the group operation (see section \ref{group}).

But there is a difficulty to be avoided here.  As we saw in section
\ref{classical}, the mereological sum of a single thing $x$ is just
$x$.  Therefore there is a sense in which every physical thing,
including every simple, is a mereological sum, for the application of
the summation operation would just produce that thing.  To avoid this
consequence Fine introduces the notion of a {\em generative}
application of an operation:

\begin{squote}
We might say that the application $y = \Gamma (x_1, x_2, x_3,
\mathellipsis )$ of an operation $\Gamma$ is {\em generative} if there
is an explanation of the identity of $y$ as $\Gamma (x_1, x_2, x_3,
\mathellipsis )$; and we might say that the operation $\Gamma$ is
itself {\em generative} if it permits a generative application. Thus
both the set-builder and the operation of predication will be
generative in this sense \citeyearpar[582]{fine2010}.
\end{squote}

Whether or not the summation operation is generative depends on the
things it is being applied to.  When summing a dog and a tree, it is
generative; when summing a dog by itself, it is not.

For any operation, there will be things it applies to that it cannot
produce.  The summation operator fuses simples, but cannot produce
them; the set-builder combines many things that it cannot produce
(like letters).  For any given operation, there is a `level 0'
consisting of the things that the operator itself cannot produce:

\begin{squote}
We suppose that certain objects are simply given.  These are the
objects whose identity does not require an explanation in terms of
$\Gamma$.  Thus, when $\Gamma$ is the set-builder, they are the
objects that are not sets and, when $\Gamma$ is summation, they are
the objects that are not sums or, rather, the objects that do not need
to be seen as sums.

We now `generate' objects in stages.  At stage 0 are the givens; at
stage 1, we add the objects that result from a single application of
the generative operation $\Gamma$ to the givens \citep[583]{fine2010}.
\end{squote}

An application can now be identified as generative in a strong or a
weak sense:

\begin{description}
  \item[Strong generative application] Also called `strict' by Fine, a
    ``[strong] generative application of $\Gamma$ to the objects $x_1,
    x_2, \mathellipsis$ can now be defined as one in which $y = \Gamma
    (x_1, x_2, \mathellipsis )$ is of a higher level than each of
    $x_1, x_2, \mathellipsis$'' \citeyearpar[584]{fine2010}.  For
    example, summing the simples $x$ and $y$ to produce the fusion $z$
    would be a strong generative application of the summation
    operator; the simples are level 0 and $z$ is level 1.  Summing two
    composites, or a composite and a simple, would not be strongly
    generative; one or both of the parts would be the same level (1)
    as the product.
  \item[Weak generative application] To illuminate this notion Fine
    introduces another, that of a {\em putative generative
      application}: ``Let us say, in the first place, that $y = \Gamma
    (x_1, x_2, \mathellipsis )$ is a putative generative application
    of $\Gamma$ if $y$ is of a higher or of the same level as each of
    $x_1, x_2, \mathellipsis$.  This gives us the notions of a
    putative prior component and of a putative prior in the usual way.
    We now say that the application $y = \Gamma (x_1, x_2,
    \mathellipsis )$ of $\Gamma$ is a {\em weak} generative
    application if it is the putative generative application and if
    $y$ is not putatively prior to any of $x1, x2, \mathellipsis$.  We
    can get from $x_1, x_2, \mathellipsis$ to $y$ without an ascent in
    level but not from $y$ to any of $x_1, x_2, \mathellipsis$''
    \citeyearpar[584]{fine2010}.
\end{description}

Applying the summation operator to a simple is neither strongly nor
weakly generative.  It is not strongly generative because the result
is a simple, which is at level 0---the same level as its part
(itself).  It is not weakly generative because the result of the
operation is putatively prior to its parts.

\section{Building groups}
\label{group}
%[cf. \citet{uzquiano2004a}]

I this section I will attempt to define a composition operator for
groups.  One might ask why this is necessary.  Groups, it might be
objected, are really no different than sets.  When we speak of a group
of people, we are actually referring to the set of which they are
members.

This is a mistake, however; groups differ from sets in important ways.
The Supreme Court, for instance, is a group.  It is a special group
with unique political attributes, but it is a group nonetheless.  And
any attempt to identify the Supreme Court with the set of the Supreme
Court justices will not succeed.  It will not succeed because the
membership of the Supreme Court changes over time, while the members
of a set do not.  The set containing the 1990 justices is a {\em
  different} set from the set containing the 2012 justices, but the
2012 Supreme Court is not a different entity than the 1990 Court.  (We
may of course say things like ``it's a different court now'', but by
that we mean only that it is composed of different people, and so may
rule differently.)

If one grants that groups such as the Supreme Court are not sets, it
may still be objected that they are therefore simply mereological
sums.  But 

\begin{squote}
membership in the Supreme Court is very different from
the part-whole relation on material objects.  The part-whole relation
on material objects is a transitive relation.  Thus if one identified
the Supreme Court with a material object and Justice Breyer with a
part of it, then one would be forced to conclude that Justice Breyer's
arm must be a part of the Supreme Court as well.  Yet, it is plain
that Justice Breyer's arm is neither a part nor a member of the
Supreme Court \citep[136--137]{uzquiano2004a}.
\end{squote}

If the Supreme Court were a mereological sum, it would behave very
strangely.  What its parts would be on a given occasion would depend
on the appointment decisions of the President.  (If we accept a
`four-dimensional' version of universalism, then objects have {\em
  temporal} as well as spatial parts.  There would then be a
mereological sum of the parts of the justices that existed during
their appointments.  This would be an object whose existed would not
depend on the President.)

There is at least some motivation to posit a new {\em kind} of thing
that is not a set or a sum.  This new kind is the group.

\subsection{The place of the group in Fine's template}
\label{group-temp}
In section \ref{classical} above, Kit Fine showed how the summation
operator relates to four different properties: Collapse, Leveling,
Absorption, and Permutation.

\begin{description}
  \item[Absorption] $\sum (\mathellipsis, x, x, \mathellipsis,
    \mathellipsis, y, y, \mathellipsis, \mathellipsis = \sum (
    \mathellipsis, x, \mathellipsis, y, \mathellipsis )$;
\item[Collapse] $\sum (x) = x$;
\item[Leveling] $\sum (\mathellipsis, \sum (x, y, z, \mathellipsis ),
  \mathellipsis, \sum (u, v, w, \mathellipsis ), \mathellipsis ) = \sum
  (\mathellipsis, x, y, z, \mathellipsis, \mathellipsis, u, v, w,
  \mathellipsis, \mathellipsis )$;
\item[Permutation] $\sum (x, y, z, \mathellipsis ) = \sum (y, z, x,
  \mathellipsis )$ (and similarly for all other permutations)
  \citep[573]{fine2010}.
\end{description}

Sums have all four properties, while sets have only Permutation and
Absorption.  We can begin to define our group operator by thinking
about which of these properties it has.

I tentatively suggest that groups mimick sets with regard to these
four properties.  Groups possess Absorption due to the fact that one
cannot be twice a member of the same group.  Groups possess
Permutation, since there is no `order' with regard to membership of a
group (there may be {\em temporal} order---I joined the group
first!---but that is not the same thing).  Groups do not possess
Collapse, since (I am inclined to think) a group can have a single
member without thereby {\em being} that member.  If, for example, a
task force is created and only one individual assigned to it, the
findings of the task force will be of the {\em task force} and not of
the individual.  Groups do not possess Leveling either, since there
can be groups made up of groups.

So far we have seen that groups are quite similar to sets.  But there
are some differences.  As we have seen, groups can change their
membership, while sets cannot.  Additionally, I suggest that the
application conditions (see section \ref{fine-app}) for groups is very
difference from that of sets.  While sets can have more or less
anything as members (letters, people, other sets), I propose that {\em
  groups may be composed only of living things or other groups}.

This claim is made on intuitive grounds.  It simply seems odd to talk
about a group of rocks, or a group of sets.  Talk of groups implies
some sort of activity, and so it is more natural to use `group' to
refer to people and other animals.  Even a `group' of trees is not
wholly bizarre.

Another point on application conditions: I think that group
composition is {\em unrestricted} among living things and groups.
That is, for any set of living things and/or groups, there is a group
of them.  Any restriction seems arbitrary.

\subsection{Members-at-times}
Another important difference between sets and groups is that while the
members of a set are necessarily so, the members of a group may change
over time.  We should therefore think of the group composition
operator (the group-builder) as operating {\em not} on (living) things
but on things-at-times.  (I hope this does not presuppose
four-dimensionalism.)  The group-builder for the Supreme Court takes
the various justices during the times of their service and produces
the group---the Supreme Court---from those people-at-times.

This might presuppose {\em eternalism}.  If the group-builder makes
the Supreme Court `in one go', then future justices (people-at-times)
must already exist in some sense.  How else could the group-builder
operate on them?

One way to avoid presupposing eternalism would be to introduce the
notion of a {\em dynamic} operator.  A dynamic group operator is in
some sense constantly building (or re-building) the group.  It could
thus actively operate only on present things.

\subsection{Identity conditions}
cf Fine

\subsection{Kinds of groups}
I said above the Supreme Court is a special kind of group.  What does
this mean?  The Supreme Court certainly has more power than other
groups.  Might it be {\em more} than just a group?  I am going to
argue that the Supreme Court is really just a group, but that we (the
United States and the world) have ascribed to it various social or
conventional characteristics which set it apart from other groups.

The best reason I can think of for considering the Supreme Court to be
simply a group is this: what else could it be?  I have already had to
argue for the recognition of groups as an `ontological category';
should I now claim that {\em courts} are yet another category?  What
about teams, clubs, and mililtias?  There are many different kinds of
groups, and it seems very strange to suppose that there must be a
separate ontological category for each.  This is especially so because
the difference between these groups are largely {\em conventional}.
The differences have to do with the social role and importance of the
respective groups, not some deep metaphysical difference.  Whether a
given group is a team, club, militia, or court depends on {\em us},
not on the world.  We decide what kind of a group it is.

But now another question arises.  How does a `generic' group {\em
  become} a team, club, militia or court?  Are groups `baptized' in a
certain way (through official recognition, or some other ritual)?  Or
is it a more vague kind of status that depends on how the group is
perceived, or how its members think of it?

The case of the Supreme Court is, at least, relatively
straightforward.  The Supreme Court was established by the Consitution
and so has legal recognition.  It would take a constitutional
amendment to dissolve the Supreme Court.  Were that to happen, the
group that was the Supreme Court would be just another group.  (But
would its membership change?  My first impression is that the
membership would be `frozen' at the time of dissolution.)

Other kinds of groups can be seen to follow a similar pattern.  Teams
are instituted more or less officially (professional and college teams
on one end, neighborhood pick-up teams on the other) but have clear
enough membership criteria and 

\section{What else is there?}
[Figure out how, if at all, Fine's theory can be used to explain the
  wish sandwich]

If there is a plurality of kinds of composition, then what position
should we take on the putative results of these composition
operations?  Should we embrace universal universalism and claim that
they all exist?  Or should we be more cautious?

\begin{comment}
One is a question about the `vocality' of existence.  Are there
difference senses of `exist', or is it univocal?  ``Many
philosophers,'' says van Inwagen, ``have thought that `there is' and
`exists' mean one thing when they are applied to material objects, and
another when they are applied to, say, minds, and yet another when
they are applied to (or withheld from) supernatural beings, and one
more thing again when applied to abstractions like numbers or
possibilities'' \citeyearpar[236]{inwagen1998}.  Van Inwagen denies
this, claiming that there one sense of `exist', which is equivalent to
the quantifier $\exists$.  If van Inwagen is right about this, then,
if ``there is a chair'' is true, and if ``there is a set'' is true,
both can be expressed as ``$\exists x$\,\ldots ''

I will not contest this claim by van Inwagen.  That is, I will suppose
that existence is univocal, and maintain that both chairs and sets
exist in the very same sense.

%%%%%%%

Perhaps, as Jay Rosenberg suggested (see section \ref{lessons-v}), we
must have recourse to the empirical sciences.  For any given $y$s, we
may suppose that they compose a thing---that is the metaphysical
lesson we may draw.  But the question of what kind of thing they
compose is not a metaphysical question.  It is an empirical question.
\end{comment}

\section{Deflationary metaphysics}
Kathrin Koslicki has an interesting objection to universalist theses
such as the one I appear committed to.  Her objection amounts to this:
if every `collection' of objects (such as the London Bridge, a
particle in the moon, and Cal Ripkin, Jr.) is a thing in its own
right, then metaphysics becomes uninteresting.  There is no longer any
debate about whether chairs or dogbushes are more `real' or have a
stronger claim to existence.  They both (obviously) exist, and the
difference between chairs and dogbushes is not ontological but
conceptual: `chair' is more embedded in our talk, and so chairs have
greater importance to {\em us}.  But metaphysically, or ontologically,
chairs and dogbushes are on the same level.  There is no sense in
which chairs exist and dogbushes do not.

In the quoted material below, Koslicki is criticizing a version of
four-dimensionalism that Sider has previously defended.  Sider's
position was that any collection of objects-at-times is a thing in its
own right.  Sider calls these things `fusions'.  For example, a chair
is a fusion of a large number of {\em temporal part} of things (wood
molecules, or atoms, or simples).  Each thing (wood molecule, atom, or
simple) is a fusion of {\em its} temporal parts.  Each temporal part
of the chair is also a thing (a fusion).

I take no stance on whether objects have temporal parts or rather
`endure' through time.  Moreover, if we accept Fine's theory of parts
then we reject the idea that there is just one composition operation;
the operation that produces fusions is one among many.  But Koslicki's
comments are relevant nonetheless:

\begin{squote}
There is room, in Sider's theory, for {\em some} genuine ontological
disagreements: for example, the universalist, the nihilist and the
holder of the intermediary position genuinely disagree over how many
and which fusions that exist.  But the only genuine ontological
disagreements for which there is room, in Sider's world, are ones that
concern disagreements over `bare' fusions, so to speak.  What has
happened to the houses, trees, people, and cars, the familiar concrete
objects of common-sense, whose persistence this account set out to
analyze?  There are no `deep' ontological facts as to whether a given
fusion should count as a house or not\,\ldots

[By claiming that there can be genuine ontological disputes,] Sider is
guilty of a bit of false advertising: his account is really a way of
saying that, at the end of the day, there is no interesting {\em
  ontological} story to be told about the persistence of our familiar
concrete objects of common-sense; whatever there is to say about the
persistence of houses, trees, people and cars concerns the
organization of our conceptual household
\citeyearpar[124--125]{koslicki2003}.
\end{squote}

Koslicki seems to think that we ought to be able to find some
ontological difference between ``the familiar concrete objects of
common-sense'' and things like dogbushes or chairs-at-times.  But as I
remarked above, why should what interests us (familiar objects like
chairs) be a guide to what exists?  The conclusion that ``the
persistence [and other properties] of houses, trees, people and cars
concerns the organization of our conceptual household'' seems to be a
most welcome one.

However, there {\em is} an ontological difference between some things,
if no between chairs and dogbushes.  One lesson of Kit Fine's theory
of parts is that mereological sums are not the only kind of composite
thing.  There are sets as well, and groups, and sequences, and perhaps
infinitely many other types of thing.  The difference between a set
and a sum is an ontological difference.  Within each type, however,
we must rely on our own conceptual `scheme' to organize things.

In section \ref{lessons-v} I considered Jay Rosenberg's claim that
the Special Composition Question is the wrong question to be asking.
Rosenberg's position seems to be that there is {\em no} answer to the
Special Composition Question.  Rather, he thinks what it takes to
`compose' something depends on what that something is---making a chair
is not like making a pie.

This insight of Rosenberg's can be connected with the insights of
Fine's theory.  If we understand `composition' in the Special
Composition Question to mean {\em mereological composition}, then
Rosenberg was wrong if he held that there is no correct answer to the
Special Composition Question.  It seems intuitively true that
mereological composition is unrestricted.  But if take `composition'
in the Special Composition Question to be $K$-composition---any
composition operator at all---then Rosenberg was {\em right} that
there is no answer.  What the application conditions are for a
composition operator depends on {\em which} composition operation is
being applied.  

Moreover, determining these application conditions, and determining
identity conditions, and determining the other properties of these
various composition operations is a task for metaphysics.  The field
is not then so barren as Koslicki seems to have feared.  But it is
true that perhaps the most interesting questions---When are we willing
to call something a chair, and why?  What conditions must be
fulfilled?---are not ontological questions anymore.  They are
questions about our ``conceptual household.''

\begin{comment}
\section{Lessons}
What we have learned from examining Merricks' arguments is not that
there are no chairs.  What we have learned is that the language of
`composition' can fool us into thinking that there are no dogbushes
(or bligers).  Worse, it can undermine our confidence that ``there are
chairs'' is obviously true.  It may be that talk of things `composing'
other things gives the impression that `composition' is something that
things {\em do}---as if they were gathering or attaching themselves
together.  If we avoid terms like `composition', we can see that there
is no motivation for Merricks' nihilism, other than his thesis of
causal over-determination.  And given that his conclusion---that there
are no chairs---is obviously untrue, we should suspect that causal
efficacy is not required for ``there are chairs'' to be true.  Rather,
all that is required for ``there are chairs'' to be true is that
``there are things arranged chairwise'' be true.

In this section I have tentatively supported a version of
universalism.  However, in the next section we will examine a number
of powerful objections to all versions of universalism.

\end{comment}

\ifstandalone
\end{spacing}
\bibliography{everything}
\bibliographystyle{ChicagoReedweb}
\fi
\end{document}
