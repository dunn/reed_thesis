\documentclass[11pt]{standalone} \newif\ifstandlone \standalonetrue
\usepackage{standalone} 
\usepackage[left=1.75in, right=1.75in, top=1.25in, bottom=1.25in]{geometry}
\geometry{letterpaper}
\usepackage{graphicx}
\usepackage{enumitem}
%\usepackage{amssymb}
\usepackage{amsmath}
\usepackage{epstopdf}
\usepackage{verbatim}
\usepackage{setspace}
\usepackage{natbib}
\setcitestyle{aysep={}}
\usepackage{hyperref}
\usepackage{url}
\synctex=1

\DeclareSymbolFont{symbolsC}{U}{txsyc}{m}{n}
\DeclareMathSymbol{\strictif}{\mathrel}{symbolsC}{74}
\DeclareMathSymbol{\boxright}{\mathrel}{symbolsC}{128}

\newenvironment{squote}{%
\begin{spacing}{1}
       	\begin{list}{}{%
\setlength{\labelwidth}{0pt}%
\rightmargin\leftmargin%
}
\item\relax
}{%
\end{list}%
\end{spacing}
}

\title{Nearly as good as true}
\author{Alexander A. Dunn}
\begin{document}
\ifstandalone
\maketitle
\begin{spacing}{1.25}
\fi

\section{Van Trenton}
Trenton Merricks comes to the same metaphysical conclusions as does
van Inwagen.  That is, he claims that there are no physical objects
other than human beings.  However, he comes to this conclusion through
a different path of reasoning.  (I have not studied his arguments yet,
so I will skip that part for now.)

\section{Explaing our beliefs}
Despite the fact that Merricks has a different motivation for his
nihilism, we can pose the same question to him as we posed to van
Inwagen.  Why, if there are no chairs, do we believe that there are
chairs?  Happily, Merricks addresses our concern.  Even more happily,
he has a better explanation than van Iwagen.  Sadly, even if his
explanation is good, it will probably be undermined by Unger's
arguments for nihilism.

\subsection{Nearly as good as true}
Merricks claims that `folk' beliefs,\footnote{He distinguishes `folk'
  beliefs from his own (presumably sophisticated) beliefs.  He claims
  that when he expresses a proposition by saying ``there is a chair'',
  {\em he} says something {\em true}.  I will examine this
  extraordinary claim in section \ref{folk}.} such as the belief that
there are chairs, are false, but nonetheless are {\em nearly as good
  as true}.  What does this mean?

\begin{squote}
People who believe in unicorns [or ghosts] are few and far between.
And those few are generally unjustified.  On the other hand, people
who believe in statues are legion.  And they are generally justified
in so believing.  Given the truth of eliminativism [what I have been
  calling nihilism], we might ask {\em why} the belief in statues is
more common, and more commonly justified, than the belief in unicorns.

The answer is that statue beliefs are nearly as good as true.  For, so
I claim here, {\em atoms arranged statuewise} often play a key role in
producing, and grounding the justification of, the belief that statues
exist.  In general, a false belief's being nearly as good as true
explains how {\em reasonable} people come to hold it.  And, relatedly,
its being nearly as good as true can ground its justification.
Because the belief that unicorns exist is not nearly as good as true
(i.e.\ because there are no things arranged unicornwise), there is no
similar explanation of its production or similar reason to think it is
justified (\citeyear[171--172]{merricks2001}).
\end{squote}

\ifstandalone
\end{spacing}
\bibliography{everything}
\bibliographystyle{ChicagoReedweb}
\fi
\end{document}
