\documentclass[11pt]{article}
\usepackage{standalone} \newif\ifstandlone \standalonetrue
\usepackage[left=1.75in, right=1.75in, top=1.25in, bottom=1.25in]{geometry}
\geometry{letterpaper}
\usepackage{graphicx}
\usepackage{enumitem}
\usepackage{amssymb}
\usepackage{amsmath}
\usepackage{tipa}
\usepackage{epstopdf}
\usepackage{verbatim}
\usepackage{setspace}
\usepackage{natbib}
\setcitestyle{aysep={}}
\usepackage{url}
\synctex=1
\usepackage{hyperref}

\newenvironment{squote}{%
\begin{spacing}{1}
       	\begin{list}{}{%
\setlength{\labelwidth}{0pt}%
\rightmargin\leftmargin%
}
\item\relax
}{%
\end{list}%
\end{spacing}
}

\title{``Nearly as good as true''}
\author{Alexander A. Dunn}
\begin{document}
\ifstandalone
\maketitle
\begin{spacing}{1.25}
\fi

\section{How does Merricks explain what we believe?}
\label{universe}
\label{merricks}
Trenton Merricks, like van Inwagen, claims that there are no physical
objects other than human beings.  However, he comes to this conclusion
through a different path of reasoning.  He claims, roughly, that
positing ordinary things (excluding people) is causally redundant;
everything that ordinary things are said to do can be described in
terms of their parts. (The details are unimportant; what matters is
how Merricks explains why we nonetheless believe that there are
ordinary things.)

Despite the fact that Merricks has a different motivation for his
nihilism, we can pose the same question to him as we posed to van
Inwagen.  Why, if there are no chairs, do we believe that there are
chairs?  Happily, Merricks addresses our concern.  Even more happily,
he has a better explanation than van Inwagen.  He explains why, if
nihilism is true, we might nonetheless believe that there are chairs.
But his strategy presupposes that universalism is false (see Sections
\ref{scq-ans} and \ref{universalism}).  Universalism, like nihilism,
seems to contradict certain of our beliefs, but Merricks' strategy can
also explain why, if universalism is true, we nonetheless hold these
certain beliefs.  Merricks' strategy does not therefore provide
nihilism any advantage over universalism, and universalism is
intuitively more plausible than nihilism.

\subsection{Nearly as good as true}
\label{near}
Merricks claims that ``folk'' beliefs, such as the belief that there
are chairs, are false, but nonetheless are {\em nearly as good as
  true}.  What does this mean?

\begin{squote}
People who believe in unicorns [or ghosts] are few and far between.
And those few are generally unjustified.  On the other hand, people
who believe in statues are legion.  And they are generally justified
in so believing.  Given the truth of eliminativism [what I have been
  calling `nihilism'], we might ask {\em why} the belief in statues is
more common, and more commonly justified, than the belief in unicorns.

The answer is that statue beliefs are nearly as good as true.  For, so
I claim here, {\em atoms arranged statuewise} often play a key role in
producing, and grounding the justification of, the belief that statues
exist.  In general, a false belief's being nearly as good as true
explains how {\em reasonable} people come to hold it.  And, relatedly,
its being nearly as good as true can ground its justification.
Because the belief that unicorns exist is not nearly as good as true
(i.e.\ because there are no things arranged unicornwise), there is no
similar explanation of its production or similar reason to think it is
justified (\citeyear[171--172]{merricks2001a}).
\end{squote}

To say that a proposition is ``nearly as good as true'' seems to mean
that while it is false, it is nonetheless somehow close enough to the
truth for a given purpose or situation.  For example, suppose we have
decided to buy a fake holiday tree for the holidays this year.  We are
looking at a number of different fake trees.  I point to one and say
``that is a nice tree''.  What I have said is false; that is not a
tree.  It is a fake tree.  But what I mean---and what my audience
recognizes me to mean---is that it is a nice {\em fake} tree.  We both
know that we are looking at fake trees; there is no point in saying
``fake tree'' every time.  When I say ``that is a nice tree'',
therefore, what I say is quite sufficient to allow for successful
communication, despite being false.  Merricks claims that propositions
expressed by things like ``there are chairs'' are also loosely true.
They are false, but are nonetheless good enough for certain purposes.

Initially, this seems like a bizarre claim.  After all, Merricks is
claiming that chairs {\em necessarily} do not exist.  According to
Merricks, ``there are chairs'', given its current meaning, could {\em
  never} be true.  If the proposition expressed by ``there are
chairs'' is necessarily false, how could it nonetheless be ``nearly as
good as true''?

\subsection{The conceptual connection}
\label{connection}
Merricks' argument relies on a very close conceptual connection
between `chair' and `chairwise' (and likewise for all ordinary terms).
Despite claiming that chairs are impossible, Merricks admits that we
understand perfectly what chairs {\em would} be, if they existed.
Because we understand the concept of `chair', we can recognize {\em
  actually existing} things that are arranged chairwise:

\begin{squote}
The folk concept of \emph{statue} plays a role in determining which
atomic arrangements are statuewise. I would even go so far as to say
that if \emph{being arranged statuewise} were not derivative upon
folk-ontological concepts\,\ldots something would be amiss
(\citeyear[8]{merricks2001a}).
\end{squote}

For Merricks, to know what things are actually arranged statue- or
chairwise requires knowing what things would compose a statue or a
chair, if such things were possible:

\begin{squote}
Atoms are \emph{arranged statuewise} if and only if they both have the
properties and also stand in the relations to microscopica upon which,
if statues existed, those atoms' \emph{composing a statue} would
non-trivially supervene (\citeyear[4]{merricks2001a}).
\end{squote}

Merricks' explanation of why we believe that there are chairs relies
on this conceptual connection.  It also is structurally similar to the
explanation we gave in Section \ref{intro-beliefs}.  Recall that our
explanation of why we believe that there are chairs (or statues) is
that, first, there are chairs, and, second, we see that there are
chairs (or learn that there are chairs through a similarly reliable
mechanism).

Merricks' definition of `nearly as good as true' allows us to produce
a parallel explanation.  His definition is this:

\begin{squote}
Any folk-ontological claim of the form ``F exists'' is \emph{nearly as
  good as true} if and only if (i) ``F exists'' is false and (ii)
there are things arranged F-wise. So, for example, ``the statue
\emph{David} exists'' is nearly as good as true because (it is false
and) there are some things arranged Davidwise
(\citeyear[171]{merricks2001a}).
\end{squote}

We may now say on behalf of Merricks that we believe that there are
chairs (and statues) because, first, there are things arranged
chairwise and, second, we see that there are things arranged
chairwise.

The structure of the two explanations is analogous, but there is an
apparent disanalogy in the content of the two.  The disanalogy does
not favor Merricks.  For it is easy enough to understand why there
being chairs, and us seeing that there are chairs, would cause us to
believe that there are chairs.  But it is less obvious why there being
things arranged chairwise, and us seeing that there are things
arranged chairwise, would cause us to believe {\em not} that there are
things arranged chairwise, but that there are {\em chairs}.

(While it is certainly true that we believe that there are chairs, I
am not sure if all or even most of us {\em also} believe that there
are things arranged chairwise.  Let us suppose for now that we do.)

The close conceptual connection between `chair' and `chairwise' is
very important for Merricks.  It is this {\em connection} that is
doing the explanatory work.  The only thing that can explain why there
being things arranged chairwise would cause us to believe that there
are chairs is this connection between the concepts.  The existence of
things arranged chairwise, and the belief that there are things
arranged chairwise, is supposed to cause the {\em additional} belief
that there are chairs.  How does this happen?

Merricks' answer appears to go something like this: chairwise
arrangements, statuewise arrangements, and other ordinary arrangements
of things play important roles in our lives.  These arrangements of
things are of interest to us, so we have developed words that allow us
to refer to them.  For whatever reason---historical, psychological, or
otherwise---we think of each arrangement as a single thing, rather
than as things.  Words like `chair' and `statue', being singular,
reflect this (incorrect) view of the world.  We are, in a sense,
fooled by grammar.

This is more than Merricks says himself.  I have not found a passage
in which he explicitly describes the nature of the conceptual
connection between concepts like `chair' and `chairwise', and explains
why, from our belief that there are things arranged chairwise, we
invariably infer that there are chairs.  But I think he would endorse
something like this.  In the first chapter of his book, he claims that
whether there is a statue or merely things arranged statuewise is not
an empirical question.  He claims that were there not a statue and
merely things arranged chairwise, our ``visual evidence'' would be the
same.  He supports this claim with an analogy:

\begin{squote}
{[}Consider{]} the claim that the atoms arranged
my-neighbour's-dogwise and the-top-half-of-the-tree-in-my-backyardwise
compose an object\ldots it won't do to defend this claim with nothing
more than ``I can \emph{just see} the object composed of the atoms
arranged dog-and-treetopwise''. Part of why this won't do, presumably,
is that one's visual evidence would be the same \emph{whether or not}
those atoms composed something (\citeyear[8--9]{merricks2001a}).
\end{squote}

He assumes, of course, that we do not believe that there is a thing
composed of a dog and some of a tree.  Later he suggests that it is
arbitrary to claim that there are statues but not dog-tree things:
``we ought to see that the only difference between arbitrary sums and
statues is a matter of conventional wisdom and local custom''
\citeyearpar[75]{merricks2001a}.  He seems sympathetic to the idea
that the reason we believe that there are statues, and not dog-tree
composites, is due to our conventional speech practices: ``it is at
least somewhat plausible that atoms arranged statuewise are united not
by composing something but, instead and in part, by how we speak and
think'' \citeyearpar[121]{merricks2001a}.

On this picture, whether we see an arrangement of things as composing
an object or not depends more on our own interests than features of
the things themselves.  We have words for chairs and statues because
things arranged chairwise and statuewise interest us.  We don't have a
word for things arranged ``my-neighbor's-dogwise and
the-top-half-of-the-tree-in-my-backyardwise'' because such an
arrangement does not hold much interest for us.  But each of these
arrangements exist, and it seems arbitrary to say that the chairwise
and statuewise arrangements compose chairs and statues while the other
arrangement composes nothing.

Merricks might explain why we believe there are things arranged
my-neighbor's-dogwise and the-top-half-of-the-tree-in-my-backyardwise
thus: there are things arranged my-neighbor's-dogwise and
the-top-half-of-the-tree-in-my-backyardwise, and we see that there are
things so arranged.  This is exactly the same explanation that I would
give.

Now Merricks explains why we believe that there are chairs thus: there
are things arranged chairwise, and we see that there are things
arranged chairwise.  {\em And incidentally, due to our own human
  peculiarities, we have found it convenient to refer to and think
  about things arranged chairwise as if they were ``chairs''---single
  unified objects}.

\section{Strange objects}
\label{dogbush}
This is a somewhat plausible explanation of why we would believe that
there are chairs if there were not.  It is certainly much better than
van Inwagen's.  But I think that it fails.  I think that when we look
closer at Merricks' attempts to motivate nihilism, we will see that
they do not support nihilism at all.  If anything they support a
version of {\em universalism}.

Merricks observes that one might object to nihilism simply by saying,
``I just {\em see} the chair!''  He claims that if this objection
moves us, we should think about an analogous objection, which he finds
much less moving:

\begin{squote}
Whether atoms arranged statuewise compose a statue is analogous to
whether atoms arranged my-neighbour's-dogwise and
the-top-half-of-the-tree-in-my-backyardwise compose an object\,\ldots
it would not do to support an affirmative answer to the latter
question simply by saying ``I can just see that object''
\citeyearpar[73]{merricks2001a}.
\end{squote}

It does indeed seem initially plausible to say that the top half of a
tree and my neighbor's dog do not compose anything.  But I think this
is ultimately incorrect.

Recall the bliger story that van Inwagen used to motivate his version
of nihilism (Section \ref{prop-ont}).  A bliger was supposed to be
four monkeys, an owl, and a sloth, who arrange themselves into a
temporary symbiotic configuration.  Van Inwagen thought we would agree
that bligers did not exist.  He claimed that it is not true that ``six
animals arranged in bliger fashion compose anything, and that is what
I mean to deny when I say that there are no bligers''
\citeyearpar[104]{inwagen1995}.

But as we saw, it is simply false that there are no bligers:

\begin{squote}
\ldots {\em of course} there are bligers in [van Inwagen's] story.
Bligers are what the story is about.  The zoologists do not report
that there are no bligers.  Rather they tell us what a bliger is.
They explain that a bliger is not a single large carnivorous animal
but a transient symbiotic union of six animals
\citep[704]{rosenberg1993}.
\end{squote}

We might be tempted to say that there are no bligers because van
Inwagen presents the question in an unintuitive way.  He asks us if
there is some thing, some object, that is composed of the other six
animals.  This gives one the impression that, were there to be such a
thing, it would perhaps be another animal (a seventh); were there such
a thing, it should somehow pop out at us.  But all we see when we
picture the scene are the six animals together, so we feel that van
Inwagen might be right.  There is no {\em other} thing.  But if we
phrase the question differently, things become clearer.  Rather than
ask if there is some thing composed of such and such other things, we
simply ask, ``are there bligers?''  And of course there are.  Van
Inwagen's use of the word `composition' led our intuitions astray.

Merricks makes the same mistake in his passage above.  Imagine if he
had said, ``Consider five discontinuous islands.  One cannot argue
that they compose some further thing by simply saying `I just see
it!'\,'' If these five islands are an archipelago, then one {\em can}
say ``I just see the archipelago!''  {\em Of course} there are
archipelagos.  They are, as one might put it, {\em scattered objects}.
The archipelago is made up of a number of separate islands, but it is
nonetheless a thing.  It is an archipelago.  Now let us suppose there
is an archipelago in the Mediterranean Sea (this example is adapted
from \citet{hawthorne2008}).  This archipelago is called the Roman
Archipelago, due to the fact that there are a number of Roman ruins on
one of its islands.  There are several research camps on the islands,
where archaeologists dig for artifacts.  Their researches result in a
surprising discovery: one of the islands {\em is} a Roman ruin.  What
was thought to a rocky and curiously shaped island is in fact a
massive collapsed temple.  Further investigation reveals that another
island is made up of the bones of an extinct sea monster, and another
island is a crashed \textsc{ufo}.

Despite these extraordinary circumstances, it is nonetheless true that
the Roman Archipelago exists.  It just happens to be composed of
several islands, a Roman ruin, a pile of old bones, and an alien
spacecraft.  To say the Roman Archipelago does not exist would entail
that these things are {\em not} sitting in the Mediterranean Sea.  (Of
course I made this story up, so the Roman Archipelago in fact doesn't
exist; but it does in the story.)

If Merricks or someone else asks us ``could scattered islands, Roman
ruins, old bones and alien spacecraft ever compose anything?'' we
should reply ``{\em of course}''.  Now take this example:

\begin{quote}
Pranksters break into a museum to install joke pieces of art.  One one
wall they put up a bathroom mirror and towel ring (complete with
towel).  Under the mirror they put a little sign reading ``Wash your
hands''.  The installation is accepted as art by the gullible curator,
who gets an equally gullible journalist to write about it.  {\em Wash
  Your Hands} quickly becomes a valuable piece of art---valuable
enough that art thieves target it.  They break into the museum in
order to steal {\em Wash Your Hands}, but trip an alarm and are forced
to flee.  All they get away with is the towel.  In the morning the
guards tell the curator that part of {\em Wash Your Hands} is missing.
The curator orders them to remove the rest of the piece and informs
crestfallen visitors that {\em Wash Your Hands} is no longer in the
museum's collection.
\end{quote}

Here, the only point at which is it true to say that {\em Wash Your
  Hands} is not in the museum is when it is finally removed.  Someone
who claimed that it was {\em never} in the museum because it doesn't
exist would be saying something quite clearly false.  Thus if Merricks
asks us ``do mirrors and towels ever compose anything?'' we should say
``{\em of course!}\,''

In these two examples, it is clear that the things in question really
do exist.  Nobody will deny that there are archipelagos and works of
art without having first been moved by a philosophical argument.  But
it may be that people {\em will} deny that there are things composed
of the tops of trees and dogs, even before hearing an argument.

Call the things composed of dogs and treetops `dogbushes'.  For
example, in a park that contains one tree and one dog, there is also
one dogbush.  Is it {\em obvious} that there are dogbushes?  Is it
just as obvious as that there are archipelagos and chairs and the {\em
  Wash Your Hands}?  If not, why?  What is the difference between
things like archipelagos and things like dogbushes?

One obvious difference is that things like archipelagos interest us.
I argued above that Merricks motivates his nihilism by drawing our
attention to the role of tradition and convention in our talk.  We
have a word for archipelagos because they {\em matter} to us.  We
don't have a word for dogbushes because they {\em don't} matter.
Merricks argued, in effect, that since we are not inclined to say that
there are dogbushes, and since there is no metaphysical difference
between dogbushes and archipelagos, we should not be inclined to say
that there are archipelagos.

But we can reverse Merricks' argument.  Since we {\em are} inclined to
say that there are archipelagos, and since there is no metaphysical
difference between archipelagos and dogbushes, we should not be
inclined to deny that there are dogbushes.

\section{Universalism}
\label{universalism}
I claimed in Section \ref{connection} that Merricks' explanation of
why we believe that there are chairs is something like this: there are
things arranged chairwise, and we see that there are things arranged
chairwise.  {\em And incidentally, due to our own human peculiarities,
  we have found it convenient to refer to and think about things
  arranged chairwise as if they were ``chairs''---single unified
  objects}.  I attributed to Merricks the idea that just because
things arranged chairwise interest us, we should not therefore suppose
that there are chairs.  What interests us should not be a guide to
what exists.  But now it is obvious how we should reply to Merricks.
Just because dogbushes do {\em not} interest us, we should not
therefore suppose that there are not dogbushes.  In this spirit,
Judith Thomson suggests that we ``think of Reality as like an
over-crowded attic, some of its contents interesting, and most merely
junk.  There is no need to deny the junk; we can simply leave it to
gather dust'' \citep[167]{thomson1998a}.  This is the intuition behind
universalism, one of the answers to the Special Composition Question
(Section \ref{scq}):

\begin{description}
\item[Universalism] Necessarily, for any $x$s, there is an object
  composed of the $x$s if and only if no two of the $x$s overlap
  \citep[227]{markosian1998a}.
\end{description}

The above considerations suggest an argument of this sort:

\begin{enumerate}[ref=(\arabic*)]
  \item Chairs exist. \label{u-1}
  \item Things that do not differ from chairs (or archipelagos, or
    works of art) in metaphysically significant ways also exist.
  \item Dogbushes do not differ from chairs in metaphysically
    significant ways.
  \item {\em Therefore}, dogbushes exist. \label{u-4}
\end{enumerate}

I imagine that Merricks would deny the conclusion \ref{u-4} and so, by
{\em modus tollens}, deny one or more premises (and we have seen that
he denies \ref{u-1}).  But I affirm the premises and so, by {\em modus
  ponens}, affirm the conclusion.

This argument helps us see what is wrong with Ned Markosian's response
to the Special Composition Question.  Markosian defends what he calls
`brutal composition'.  The thesis of brutal composition is that, while
there is indeed no ``no true, non-trivial, and finitely long answer to
[the Special Composition Question]''
\citeyearpar[214]{markosian1998a}, this is not because we should refer
questions of composition to the empirical sciences.  Rather, whether
or not some things compose another is simply a {\em brute fact}.

This is a clever reply, but whether true or not I do not think it does
the work that Markosian expects it to.  He presents his theory as
``consistent with standard, pre-philosophical intuitions about the
universe's composite objects'' \citeyearpar[211]{markosian1998a}.  But
his theory will only be consistent with such intuitions if, first, it
is a brute fact that all of the things we ordinarily take to exist
(tables, chairs, etc.) do in fact exist, and, second, that it is a
brute fact that the things that we don't take to exist don't in fact
exist.  But why should we expect there to be a {\em metaphysical}
difference between things that interest us and things that don't?  The
chance that the brute facts of composition happen to line up with our
(or Markosian's) intuitions seems to be incredibly low.

But accepting the above argument for universalism has some strange
consequences that are not immediately apparent.  Ned Markosian brings
out such a consequence in this passage:

\begin{squote}
There is what seems to me a fatal objection to Universalism:
Universalism entails that there are far more composite objects than
common sense intuitions can allow.  To give just one example,
Universalism entails that the following sentence is true:\,\ldots
There is an object composed of (i) London Bridge, (ii) a certain
sub-atomic particle located far beneath the surface of the moon, and
(iii) Cal Ripken, Jr.  My intuitions tell me that there is no such
object, and I suspect that the intuitions of the man on the street
would agree with mine on this point \citeyearpar[228]{markosian1998a}.
\end{squote}

If this is a compelling objection, it is because such an object (call
it `Lumpkin') does not interest us in the least.  As Merricks observed
(see Section \ref{connection}), the things that we believe to exist
are largely the things that interest us.  We believe that there are
archipelagos; van Inwagen's imaginary farmers believe that there are
bligers.  If we do not believe that there are dogbushes or Lumpkins,
this may be because they do not interest us.

Suppose Markosian wrote this instead:

\begin{squote}
Universalism entails that the following sentence is true:\,\ldots
There is an object composed of (i) an island, (ii) a Roman ruin, and
(iii) the bones of a sea monster.
\end{squote}

But this is just the Roman Archipelago I mentioned in Section
\ref{dogbush}.  We are (or should be) happy to admit that it exists.
If the Roman Archipelago exists, and if it does not differ from the
Lumpkin in any metaphysically significant ways, why shouldn't we admit
that the Lumpkin exists?  Of course we don't {\em care} about the
Lumpkin.  We have no need to refer to it; it doesn't matter to our
lives.  But why should we expect---as Markosian seems to---that our
intuitions should perfectly track what exists?

\section{Lessons, part 2}
\label{lessons-m}
What we have learned from examining Merricks' arguments is not that
there are no chairs.  What we have learned is that since there {\em
  are} chairs, and since dogbushes do not differ from chairs in
metaphysically significant ways, there are therefore also dogbushes.

If we agree that there are chairs and archipelagos and dogbushes and
the Lumpkin, however, new questions arise. For example: What are the
parts of a chair?  How do they compose the chair?  Do the parts of the
chair change over time?  How?  We will address these questions in the
next section.  

\ifstandalone
\end{spacing}
\bibliography{everything}
\bibliographystyle{ChicagoReedweb}
\fi
\end{document}
