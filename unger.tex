\documentclass[11pt]{article}
\usepackage{standalone} \newif\ifstandlone \standalonetrue
%\documentclass[11pt]{article}
\usepackage[left=1.75in, right=1.75in, top=1.25in, bottom=1.25in]{geometry}
\geometry{letterpaper}
\usepackage{graphicx}
%\usepackage{tipa}
%\usepackage{exaccent}
%\usepackage{txfonts}
%\usepackage{pxfonts}
\usepackage{enumitem}
%\usepackage{amssymb}
\usepackage{amsmath}
\usepackage{epstopdf}
\usepackage{setspace}
\usepackage{natbib}
\setcitestyle{aysep={}}
\usepackage{hyperref}
		
\synctex=1

\DeclareSymbolFont{symbolsC}{U}{txsyc}{m}{n}
\DeclareMathSymbol{\strictif}{\mathrel}{symbolsC}{74}
\DeclareMathSymbol{\boxright}{\mathrel}{symbolsC}{128}

\newenvironment{squote}{%
	\begin{quote}\begin{singlespace}%
	}{%
	\end{singlespace}\end{quote}}

\newcommand{\stager}[4]%
{%
	\begin{spacing}{1}%
	\vspace{0pt}
		\begin{description}[style=nextline, noitemsep, parsep=0pt, topsep=0pt, leftmargin=15mm, itemindent=-10mm, font=\mdseries]
			\item[\textsc{#1} \emph{#2}] #3
			\item[]%
			\begin{flushright}#4\end{flushright}
		\end{description}%
	\end{spacing}%
}

\newcommand{\stage}[3]%
{%
	\begin{spacing}{1}%
	\vspace{0pt}
		\begin{description}[style=nextline, parsep=0pt, leftmargin=15mm, itemindent=-10mm, font=\mdseries]
			\item[\textsc{#1} \emph{#2}] #3
		\end{description}%
	\end{spacing}%
}

\newenvironment{inq}{\vspace{0pt}%
	\begin{list}{}%
	{\setlength\labelwidth{0pt}%
	\setlength\leftmargin{2.5\oddsidemargin}%
	\setlength\rightmargin{\leftmargin}}
	\begin{spacing}{1}
	\item[]%
	}{
	\end{spacing}
	\end{list}
	\vspace{10pt}
	%\noindent%
	}

\title{Brute facts and arbitrary objects}
\author{Alex Dunn}
\begin{document}

\ifstandalone
\maketitle
\begin{spacing}{1.5}
\fi

%\noindent In section~\ref{deny} we found that an attempt to deny the existence of ordinary things such as tables and chairs cannot succeed unless an explanation is given for our successful communication using the terms `table' and `chair'. Unger's nihilistic thesis failed to explain what we are really thinking and talking about when we take ourselves to be thinking and talking about chairs. We found no reason to suppose that we are not, in fact, thinking and talking about tables and chairs. If we are able to think and talk about tables and chairs, then it seems to follow that tables and chairs exist. (Remember that the fact that ghosts do not exist gives us reason to believe that we do not think and talk about ghosts.)
%
%Peter van Inwagen has developed a more sophisticated thesis denying the existence of tables and chairs. Like Unger, he claims that (necessarily) there are simply no such things as tables and chairs in the world. But unlike Unger, he does not claim that when we take ourselves to be thinking and talking about such things, we are thinking and talking about nothing at all. Or at least ``when people say things in the ordinary business of life by uttering sentences that start `There are chairs\,\ldots\,' or `There are stars\,\ldots\,', they very often say things that are literally true''~\citep[102]{inwagen1995}. One would generally assume that if such statements are true, then it follows that chairs and stars exist. But van Inwagen denies that chairs and stars exist. How can he claim, then, that what was said was true? As we mentioned in section~\ref{paraphrase}, van Inwagen attempts to show that the statements in question can be {\em paraphrased}---they can be reformulated to show that they have no ``ontological committments''. According to van Inwagen, one can assert that there is a chair without being committed to the existence of chairs. This section will be devoted to assessing the plausibility of this claim.
%
%Section~\ref{comp} will summarize the motivation for van Inwagen's denial. Section~\ref{paravan} will introduce and criticize van Inwagen's paraphrasing strategy.
%
%\section{Composition}
%\label{comp}
%The Special Composition Question was given a precise formulation by van Inwagen, who finds that ``the metaphysically puzzling features of material objects are connected in deep and essential ways with metaphysically puzzling features of the constitution of material objects by their parts''~\citep[18]{inwagen1995}. A ready example is the Ship of Theseus: the planks and rigging and sails (and every part of the ship) are replaced as they individually wear out. These replacements happen each by themselves; the entire ship (nor even large sections) are swapped out all at once. But eventually no part of the original ship remains. And yet we would commonly say that it is still the same ship. But why should we think that the present ship is identical with a past ship with which it shares no parts?
%
%\subsection{The question}
%Answering the question ``why is this ship identical with that past ship?'' requires first figuring out why (and how) these planks and rigging and sails (et.\ al.) compose a ship in the first place. Van Inwagen asks ``in what circumstances do planks\footnote{For simplicity's sake, van Inwagen ignores the rigging and sails.} compose (add up to, form) something?''~(\citeyear[21]{inwagen1995}) For some $x$s, then, van Inwagen asks us to consider when
%\begin{equation}
%\exists x\ \text{the}\ x\text{s compose}\ y
%\end{equation}
%is true.%
%\footnote{Van Inwagen explains in some detail how plural referring expressions (like ``the planks'') can be given a logical formalization (\citeyear[23--28]{inwagen1995}), but suffice to say they work just as one would expect.}%
%%
%\ Less formally, van Iwagen asks: ``suppose one had certain (nonoverlapping) objects, the $x$s, at one's disposal; what would one have to do---what {\em could} one do---to get the $x$s to compose something?''~(\citeyear[31]{inwagen1995})
%
%({\em Composition} is a technical term for van Inwagen. He understands it thus: ``the $x$s compose $y$'' means that ``the $x$s are all parts of $y$ and no two of the $x$s overlap and every part of $y$ overlaps at least one of the $x$s\,\ldots\,a thing {\em overlaps} a thing---or: they overlap---if they have a common part''~(\citeyear[29]{inwagen1995}). For van Inwagen, everything is a part of itself; some $x$ is a {\em proper} part of some $y$ only if $x \neq y$.)
%
%\subsection{The usual answers}
%There are several prominent answers to the Special Composition Question, including the following:\footnote{These formulations are from~\citet{markosian1998a}.}
%\begin{description}
%	\item[Nihilism] Necessarily, for any $x$s, there is an object composed of the $x$s iff there is only one of the $x$s, i.e., the only objects that exist are simples.~(\citeyear[219]{markosian1998a})%
%	%
%	\footnote{\label{flip} Note that this is not clearly Unger's view. He denies that people, apples, cheese, tables, chairs, and other ``ordinary things'' are nonexistence but he does not, as far as I know, take a stand on whether anything at all exists. His view can be (flippantly) summarized thus: ``if we have a word for it, it doesn't exist.''}
%	%\footnote{\label{gunk} Of course, it may be that the world is not fundamentally particulate, and is filled not with simples but with `gunk'; see \citet{schaffer2003}. Nihilism (and van Inwagen's second condition below) can be formulated to take this possibility into account: ``for any quantity of gunk, there is nothing composed of it.''}%
%	%
%	\item[Universalism] Necessarily, for any $x$s, there is an object composed of the $x$s iff no two of the $x$s overlap.~(\citeyear[227]{markosian1998a})
%	\item[Van Inwagenism] Necessarily, for any $x$s, there is an object composed of the $x$s iff either (i) the activity of the $x$s constitutes a life or (ii) there is only one of the $x$s.~(\citeyear[221]{markosian1998a})
%\end{description}
%
%As we have seen, any version of nihilism that does explain our putative communication about tables and chairs is false.
%
%Whether or not universalism is false is a more difficult question. As we saw in section~\ref{many13p}, any version of universalism that entails the existence of millions of chairs where we assumed there to be only one is false. This is because such a thesis would entail that we are unable to successfully communicate about ordinary things; we would not be able to know that were were talking about the {\em same thing} (not things) as our audience. It may be that a version of universalism without this consequence is true; but we will leave that possibility aside for now.
%
%Van Inwagen examines and rejects a number of answers to the Special Composition Question that would entail the existence of tables and chairs. Some are too strong: `some $x$s compose a $y$ iff the $x$s are in contact' would entail that two people shaking hands will result in a new object coming into being. Some are too stong in some ways and too weak in others: `some $x$s compose a $y$ iff the $x$s are fastened together' would entail that two people being glued together would result in a new object; and it would deny that an object can be composed without fastening its parts together (such as when building a house of cards). The only answer van Inwagen finds consistent is what we have dubbed {\em van Inwagenism}, which entails that tables and chairs do not exist.
%
%However, without a good explanation of what our beliefs about tables and chairs are really about, the fact that van Inwagenism entails the nonexistence of tables and chairs only shows that van Inwagenism is false. Happily, though, van Inwagen recognizes this and is prepared with a paraphrasing strategy aimed to show that the beliefs that we take to be about tables and chairs are really about something else.
%
%\section{Van Inwagen's paraphrasing strategy}
%\label{pigletwise}
%Van Inwagen distances himself from the kind of resolute denial we saw Unger attempting in section~\ref{unger}. Unger maintained that terms like `chair' are incoherent; were this so, a statement involving the phrase ``There is a chair\,\ldots '' could surely not be true. Van Inwagen, on the other hand, admits that ``when people say things in the ordinary business of life by uttering sentences that start `There are chairs\,\ldots\,' or `There are stars\,\ldots\,', they very often say things that are literally true''~\cite[102]{inwagen1995}. One would generally assume that if what people say with ``There are chairs\,\ldots '' and the like are true, then chairs exist. But van Inwagen denies this entailment.
%
%How can van Inwagen maintain this? He claims that one can also say, truly, ``There are simples arranged chairwise\,\ldots '' without committing oneself to the existence of chairs. He could, therefore, claim that when someone says ``There is a chair\,\ldots '' she {\em means} ``There are simples arranged chairwise.'' This is, of course, a bold hypothesis about the speech practices of ordinary speakers. Certainly very few speakers would, if asked, affirm that what they meant to say had anything to do with simples; they would say that when they said that there was a chair, they meant just that. Van Inwagen recognizes that this is not a viable position: ``The only thing I have to say about what the ordinary man really means by `There are two valuable chairs in the next room' is that he really means that there are two valuable chairs in the next room''~(\citeyear[106]{inwagen1995}).
%
%One might then assume that van Inwagen is thinking in analogy with Russell. He could attempt to claim that, despite the surface appearance of language (``There is a chair\,\ldots ''), the underlying logical form does not make any mention of chairs (or tables); the offending concept is analyzed away, leaving ``There are simples arranged chairwise\,\ldots ''. Van Inwagen notes that his ``suggested technique of paraphrasing enables [him] to escape some of the more embarrassing consequences of [his] position. When someone says ``Some tables are heavier than some chairs,'' there is obviously something right about what he says. [His] technique of paraphrasis enables [him] to capture what it is that is right about what he says''~(\citeyear[111]{inwagen1995}). However, on the very next page he admits that the ordinary language proposition and his paraphrased version are different propositions: ``When the ordinary man utters the sentence `Some chairs are heavier than some tables' (in an appropriate context, and so on and so on), he expresses a certain proposition, and one that is almost certainly true. But I do not claim that this proposition {\em is} the proposition that, for some $x$s, those $x$s are arranged chairwise and for some $y$s, those $y$s are arranged tablewise, and the $x$s are heavier than the $y$s''~(\citeyear[112]{inwagen1995}). So van Inwagen is not making an appeal to some notion of `logical form'. But then what is the purpose of the paraphrasing project?
%
%Van Inwagen attempts to justify his method of paraphrasis by asserting the following parallels between the original and paraphrased propositions:
%\begin{enumerate}[label=(\Alph*)]
%	\item The paraphrase describes the same fact as the original. \label{para-a}
%	\item The paraphrase, unlike the original, does not even appear to imply that there are any objects that occupy chair-receptacles.
%	\item The paraphrase is neutral with respect to competing metaphysical theories, {\em viz}. the ``received'' theory, that there are objects that occupy chair-receptacles, and the theory I have proposed, according to which there are no such objects. \label{para-c}
%	\item The original, though it doubtless does not express the same proposition as the paraphrase, has the feature ascribed to the paraphrase in \ref{para-c}: It is neutral with respect to the question whether there are objects that fit exactly into chair-receptacles~(\citeyear[113]{inwagen1995}).
%\end{enumerate}
%I am rather dubious as to the truth of \ref{para-a}, but I am quite sure that \ref{para-c} is false, and van Inwagen's thesis appears to depend on it.
%
%\subsection{Propositions and ontological committment}
%Let us review the situation so as to appreciate the mess van Inwagen has gotten himself into. First, he agrees that when someone says thinks like ``There is a chair\,\ldots '' they mean just that. Second, he admits that his `paraphrases' of such propositions are not so faithful to the original that they can be called the same proposition; the original and the paraphrase are two different propositions. Third, he claims nonetheless that {\em neither} the original nor the paraphrase entail the existence of chairs.
%
%This may strike one as obviously untrue. After all, does not the proposition expressed by my saying ``There are simples arranged chairwise\,\ldots '' entail the existence of simples? If van Inwagen says that there are simples arranged chairwise, and means just that, then it would appear to follow that there are simples. Van Inwagen's argument relies rather heavily on the assumption that simples exist.%
%%
%\footnote{Ted Sider takes him to task for this assumption~(\citeyear{sider1993}), claiming that the possibility of `gunk'---the possibility that the matter of the world is not fundamentally particulate but infinitely divisible---falsifies van Inwagen's thesis. I think it may be possible for van Inwagen to adapt to a gunky world (he might be able to claim that nothing exists but organisms, who are composed of other organisms and/or gunk), but I think van Inwagen's thesis is false either way.}
%%
%\ How, then, can he claim that when someone says ``There is a chair\,\ldots '' and means just that, that the proposition they express does not entail the existence of chairs?
%
%To defend his claim, van Inwagen appeals to his `Copernican analogy':
%\begin{squote}
%I accept the Copernical Hypothesis. One day you hear me say, ``It was cooler in the garden after the sun had moved behind the elms.'' You say, ``You see, you can't consistently maintain your Copernicanism outside the astronomer's study. You say that the sun moved behind the elms; yet, according to your official theory, the sun does not move.'' I reply that the proposition I expressed by saying ``It was cooler in the garden after the sun had moved behind the elms'' is consistent with the Copernican Hypothesis~(\citeyear[101]{inwagen1995}).
%\end{squote}
%That is, van Inwagen claims that the proposition he expressed with ``It was cooler in the garden after the sun had moved behind the elms'' does not entail that the sun actually moved. And he argues that this is analogous to our talk of chairs: most propositions expressed with ``There is a chair\,\ldots '' do not entail that chairs actually exist.
%
%First, does ``The sun moved behind the elms'' entail that the sun moved? I am inclined to say that it does. Fortunately, the sun does actually move, so the proposition van Inwagen expressed was most probably true. But if Copernicus had been right in claiming that the Sun was stationary, then the proposition would have been false. Even were this so, however, the proposition would still be `loosely true' (more on this below).
%
%But let us suppose that ``The sun moved behind the elms'' does not entail that the sun moved. Does the analogy with chairs hold? Van Inwagen's frequent example of a statement that does not entail the existence of chairs is ``There are two very valuable chairs in the next room''. If this does not entail that chairs exist, then what about ``There are two valuable chairs left in the world'' or ``There are at least two chairs in the world'' or ``There are at least two chairs'' or simply ``There are chairs''? Van Inwagen appears committed to the claim that ``There are chairs'' does not entail that there are chairs.
%
%It is here that the disanalogy becomes apparent. For even if ``The sun moved behind the elms'' does not entail that the sun moved, it does entail that the sun exists. Of course, van Inwagen will deny this; he claims that the sun doesn't exist. Here, however, we will appeal to our own analogy, hinted at above: if ``The sun moved behind the trees'' entails neither that the sun moved not that the sun exists, then how can van Inwagen maintain that ``There are simples arranged chairwise'' entail that there are simples, or that they are arranged chairwise? He has given us no reason to believe one and not the other.
%
%\subsection{Loose truth, again}
%When criticizing Peter Unger's nihilism, we imagined a defense of his thesis based on the notion of `loose truth'. Our Unger partisan claimed that while such claims as ``There is a chair\,\ldots '' are invariably false (because incoherent), they may be loosely true. Unfortunately for Unger, his defender was unable to give a loose-truthmaker for these supposed loose truths.
%
%Might van Inwagen also appeal to loose truths? He admits it as a last-ditch possiblility:
%\begin{squote}
%I can say this [that ``There are chairs\,\ldots '' can be true yet not entail that there are chairs] because I accept certain theses in the philosophy of language. I can say this because I accept certain theses in the philosophy of language. Some people, I suppose, would reject these theses. These people would say that when I said\,\ldots\,`The sun moved behind the elms,' I said something false. even if I did accept the austere philosophy of language that ascribes falsity to typical utterances of `The sun moved behind the elms', I could nevertheless respond to `Moore's gambit' in a way that is very much like the way I have responded to it. If someone maintains that `The sun moved behind the elms' expresses a falsehood, he must still have some way to distinguish between this sentence and those sentences (like `The sun exploded' and `The sun turned green') that the vulgar would regard as the sentences that expressed falsehoods about the sun\,\ldots\,[if I took this line,] I should not be willing to say that people who uttered things like `There are two valuable chairs in the next room' very often said what was true. I should be willing to say only that they very often say what might be treated as a truth for all practical purposes~(\citeyear[102--103]{inwagen1995}).
%\end{squote}
%We can distinguish the given propositions based on the fact that ``The sun moved behind the elms'' has a loose-truthmaker, while others do not:
%\begin{squote}
%Owing to a change in the relative positions and orientations of the earth and the sun, it came to pass that a straight line drawn between the sun and this point (which is on the surface of the earth) would have passed through the elms~\citep[112--113]{inwagen1995}.
%\end{squote}
%
%But now what is the loose-truthmaker for ``There are chairs''? Presumably van Inwagen will say that it is the fact that there are simples arranged chairwise. For this to be a legitimate move, however, van Inwagen needs to give a non-circular definition of `chairwise'. The loose-truthmaker for the sun's movement contained no appeal to movement; the movement in question was defined in other terms. Likewise, the loose-truthmaker for the chair's existence must not make a covert appeal to chairs.
%
%\subsection{Chair(wise)}
%We are demanding that van Inwagen give us definitions of `chairwise' and `chair' that are not definitions in terms of each other. Van Inwagen claims to be able to do this; responding to criticism by Jay Rosenberg, he says that ``it is easy to see how to define `chairwise' in terms of `chair' without supposing that there are any chairs. Let a ``chair'' be defined as an object that has the properties $C_{1}, C_{2},\,\dots\,C_{n}$''~(\citeyear[719]{inwagen1993b}). In order for the definitions to be non-circular, these properties must not include things like ``is a chair'', ``is shaped chairwise'', etc.
%
%I am dubious that van Inwagen can provide us with necessary and sufficient property-list definitions of chair and chairwise that are not circular. And I am not alone in my skepticism:
%\begin{squote}
%When one says chair, one thinks vaguely of an average chair. But collect individual instances, think of arm-chairs and reading chairs, and dining-roomchairs and kitchen chairs, chairs that pass into benches, chairs that cross the boundary and become settees, dentists' chairs, thrones, opera stalls, seats of all sorts, those miraculous fungoid growths that cumber the floor of the Arts and Crafts Exhibition, and you will perceive what a lax bundle in fact is this simple straightforward term. In co-operation with an intelligent joiner I would undertake to defeat any definition of chair or chairishness that you gave me. Chairs just as much as individual organisms, just as much as mineral and rock specimens, are unique things---if you know them well enough you will find an individual differenc even in a set of machine-made chairs---and	it is only because we do not possess minds of unlimited capacity, because our brain has only a limited number of pigeon-holes for our correspondences with an unlimited universe of objective uniques, that we have to delude ourselves into the belief that there is a chairishness in this species common to and distinctive of all chairs~\citep[384--385]{wells1904}.
%\end{squote}
%
%Without a proposal as to how we can capture all these types of chairs under an exclusive definition (leaving none out and bringing nothing else in), I find the idea that `chair' has no necessry and sufficient property-list definition; it is a `family resemblance'-type concept.

%\footnote{What we don't know is what exactly Unger and van Inwagen mean by their denials of the existence of ordinary things. Hirsch justifiably complains, ``I simply have no idea of what van Inwagen means when he says such things as, `There are trees but there are no apples (though it would of course be correct to say in the ordinary business of life that there are a lot of apples around here).' It seems to me that I don't grasp the sense of his words `There are no apples'\,''~(\citeyear[690]{hirsch1993}).}

\section{Other answers that are false}
This section will mostly involve me registering agreement with Markosian on the insufficiency of other proposed answers to the SCQ. For example:
\begin{description}
	\item[Fastenation] Necessarily, for any $x$s, there is an object composed of the $x$s iff the $x$s are fastened together.~\citep[223]{markosian1998a}
\end{description}
No general notion of `fasten' will make this plausible. ``Suppose that van Inwagen and I shake hands, and suppose that just as we do so, our hands becomes paralyzed [or super-glued together], so that we cannot pull them apart. Then, according to Fastenation, there is a new composite object in the world''~\citep[224]{markosian1998a}.\footnote{This is based on an example of van Inwagen's~(\citeyear[57--58]{inwagen1995}.} But we do not recognize van Inwagen and Markosian as composing an object; `van Mark', thankfully, does not emerge. Fastenation is not then sufficient. Nor is it necessary; I can create a house of cards without fastening any of them together.

Another approach is this:
\begin{description}
	\item[The Serial Response] The correct answer to the SCQ is an instance of this schema: There is an object composed of the $x$s iff {\em either} the $x$s are F1s and related by R1, {\em or} the $x$s are F2s and are related by R2, {\em or}\,\ldots\,the $x$s are Fns and are related by Rn.~\citep[230]{markosian1998a}
\end{description}
But seeing as we have not {\em one} plausible relation between objects that is necessary and sufficient for composition, it seems unlikely that we can find a series of them, limited in scope though they may be. (For what are the necessary and sufficient conditions for a deck of cards composing a house of cards?)

\section{Arbitrariness}
Where does this leave us? [stuff about realism]

\ifstandalone
%\csname standaloneignore\endcsname
\bibliography{everything}%\endcsname
\bibliographystyle{ChicagoReedweb}%\endcsname
\end{spacing}%\endcsname
\fi
\end{document}%\endcsname

%\onlyifstandalone{\end{spacing} \end{document}}