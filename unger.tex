\documentclass[11pt]{standalone} \newif\ifstandlone \standalonetrue
\usepackage{standalone}
\usepackage[left=1.75in, right=1.75in, top=1.25in, bottom=1.25in]{geometry}
\geometry{letterpaper}
\usepackage{graphicx}
\usepackage{enumitem}
%\usepackage{amssymb}
\usepackage{amsmath}
\usepackage{epstopdf}
\usepackage{setspace}
\usepackage{natbib}
\setcitestyle{aysep={}}
\usepackage{hyperref}
		
\synctex=1

\DeclareSymbolFont{symbolsC}{U}{txsyc}{m}{n}
\DeclareMathSymbol{\strictif}{\mathrel}{symbolsC}{74}
\DeclareMathSymbol{\boxright}{\mathrel}{symbolsC}{128}

\newenvironment{squote}{%
	\begin{quote}\begin{singlespace}%
	}{%
	\end{singlespace}\end{quote}}

\newcommand{\stager}[4]%
{%
	\begin{spacing}{1}%
	\vspace{0pt}
		\begin{description}[style=nextline, noitemsep,
                    parsep=0pt, topsep=0pt, leftmargin=15mm,
                    itemindent=-10mm, font=\mdseries]
			\item[\textsc{#1} \emph{#2}] #3
			\item[]%
			\begin{flushright}#4\end{flushright}
		\end{description}%
	\end{spacing}%
}

\newcommand{\stage}[3]%
{%
	\begin{spacing}{1}%
	\vspace{0pt}
		\begin{description}[style=nextline, parsep=0pt,
                    leftmargin=15mm, itemindent=-10mm, font=\mdseries]
			\item[\textsc{#1} \emph{#2}] #3
		\end{description}%
	\end{spacing}%
}

\newenvironment{inq}{\vspace{0pt}%
	\begin{list}{}%
	{\setlength\labelwidth{0pt}%
	\setlength\leftmargin{2.5\oddsidemargin}%
	\setlength\rightmargin{\leftmargin}}
	\begin{spacing}{1}
	\item[]%
	}{
	\end{spacing}
	\end{list}
	\vspace{10pt}
	%\noindent%
	}

\title{Brute facts and arbitrary objects}
\author{Alex Dunn}
\begin{document}
\ifstandalone
\maketitle
\begin{spacing}{1.5}
\fi

\section{Unger's nihilism}
\label{unger}
Just as resolutely as we would deny the existence of ghosts, so Peter
Unger has denied the existence of all `ordinary things'---such things
as ``tables and chairs and spears\,\ldots\,swizzle sticks and
sousaphones\,\ldots\,stones and rocks and twigs, and also tumbleweeds
and fingernails''~(\citeyear[117]{unger1979}).  He does not consider
them merely `subjective' as opposed to objectively so---like van
Inwagen, he claims that they simply do not exist.  He comes to this
conclusion from a different direction, however.  As we will see, van
Inwagen's denial of the existence of `ordinary things' is a
consequence of his theory of composition (under what conditions some
things compose another thing).  Unger, on the other hand, claims that
terms for ordinary things, like `chair', are {\em incoherent}.  Unger
claims that incoherent terms cannot apply to anything in the world;
therefore he concludes that there are no chairs (or any other ordinary
thing).

Unger has two largely independent motivations for his claim that terms
like `chair' are incoherent.  One is the sorites paradox, and the
other is the `problem of the many'.

\subsection{Sorites paradoxes}
\label{sorites}
A typical instance of the sorites paradox begins by having us imagine
some ordinary object; let us use a heap of sand.  Now suppose we
remove a single grain of sand.  If we were inclined to believe that
the initial quantity of sand did in fact constitute a heap, then after
the removal of a single grain, we should presumably still have a heap
(albeit a slightly smaller one).  It seems very implausible to think
that one grain of sand more or less could {\em ever} make a difference
as to whether something is or is not a heap.

But having conceded (a) that there is a heap and (b) that the removal
of a single grain cannot make the difference as to whether a quantity
of sand is a heap, we have unwittingly put our foot in it.  For if the
removal of a single grain {\em never} transforms a heap into a
non-heap, then by repeatedly removing one grain after another, we will
eventually find ourselves with a heap that consists of no sand at
all.  But it seems absurd to suppose that there could be a heap of
sand that is composed of no sand---indeed, of nothing whatsover.

This is the sorites paradox.  While a heap is a useful example,
because it is so ill-defined, similar problems appear to afflict all
ordinary things.  Unger illustrates the difficulty for stones:

\begin{squote}
Consider a stone, consisting of a certain finite number of atoms.  If
we or some physical process should remove one atom, without
replacement, then there is left that number minus one, presumably
constituting a stone still\,\ldots after another atom is removed,
there is that original number minus two; so far, so good.  But after
that certain number has been removed, in similar stepwise fashion,
there are no atoms at all in the situation, while we must still be
supposing that there is a stone there.  But as we have already agreed,
if there is a stone present, then there must be some atoms\,\ldots I
suggest that any adequate response to this contradiction must
include\,\ldots the denial of the existence of even a single
stone.~\citep[121--122]{unger1979}
\end{squote}
Unger understands this dilemma to apply across the board, and
correspondingly argues that we should deny the existence of even a
single ordinary thing.

\subsection{The problem of the many}
\label{many}
The `problem of the many', as Unger terms this second difficulty for
ordinary things, follows a similar line of reasoning.  If we consider
an ordinary thing---a cloud, for instance---it is presumably composed
of molecules.  There is probably then a set of molecules, the members
of which compose the cloud.  Call that set $S$.  Now consider $S_1$.
This is a set of molecules that includes all of the members of $S$ as
well as one additional molecule.  Do the members of $S_1$ compose a
cloud?  Surely they are just as well suited to do so.  Now consider
$S_2$\,\ldots

Because these numerous 'candidates' are equally (or nearly equally)
well suited to be clouds, we seem forced to conclude that there are
either many clouds where we supposed there to be one, or rather no
clouds at all:

\begin{squote}
No matter where we start, the complex first chosen has nothing
objectively in its favor to make it a better candidate for cloudhood
than so many of its overlappers are.  Putting the matter somewhat
personally, each one's claim to be a cloud is just as good, no better
and no worse, than each of the many others.  And, by all odds, each
complex has \emph{at least} as good a claim as any still further real
entity in the situation.  So, either \emph{all} of \emph{them} make it
or else \emph{nothing} does; in this real situation, either there are
many clouds or else there really are no clouds at all
\citep[415--??]{unger1980a}.
\end{squote}

The problem of the many can also arise by considering the {\em
  boundary} of a given cloud.  It is natural to suppose that a cloud
has a determinate boundary.  But if we look at the edge of the cloud,
where we suppose the boundary to be, ``we may find, side by side, or
themselves overlapping, a great many potential boundaries for
clouds\,\ldots if our alleged typical item {[}the cloud{]} is indeed
a typical cloud, then many of these candidates, millions at least, do
not fail to be clouds altogether but are clouds of some
sort'' \citep[420--421]{unger1980a}.

The pattern of argumentation is the same for both approaches.  For a
given cloud, a certain set of members or a certain boundary is
supposed, and it is argued that a set or boundary that differs
minimally from the original must also compose our bound a cloud.  The
new set or boundary does not appear to differ from the original in any
relevant way; there seems no principled reason to deny that if the
first set's members compose a cloud, the second set's members do too.
And since there are a great deal of very similar sets and boundaries,
we find ourselves threatened with a plurality of clouds.

And of course, Unger does not rest content with applying the problem
of the many to clouds.  All ordinary objects get the same treatment;
he concludes that either there are a great many of them, or there are
none at all.  He claims, predictably, that the latter disjunct is
preferable.

\section{So what's the problem?}
We find ourselves wanting to hold three theses, which appear mutually
inconsistent:

\begin{enumerate}
  \item There is at least one chair (stone, cloud).
  \item If a chair (stone, cloud) exists, it must be composed of
    matter.
  \item If a chair (stone, cloud), exists, the removal of a single
    molecule (or otherwise insignificant quantity of matter) from it
    cannot destroy it or cause it to cease to exist.
\end{enumerate}

We seem to be clearly caught in a paradox; the only question is where
we have gone wrong.

But have we, in fact, gone wrong?  Peter Unger thinks that we are
right on target:

\begin{squote}
While Eubulides' contribution has often been labeled `the sorites
paradox', there is nothing here which is a paradox in any
philosophically important sense\,\ldots Accepting our negative
conclusions here does not mean important logical trouble for us; we
only think we have troubles while we refuse to admit their validity
(\citeyear[145]{unger1979}).
\end{squote}

Our situation is only paradoxical, says Unger, while we unreflectingly
cling to the first thesis.  If, however, we come to see that there are
no chairs (stones, clouds), then we happily escape paradox: if there
are no chairs (stones, clouds) to begin with, we do not have to worry
about what the addition or removal of small amounts of matter would do
to them; nor do we need concern ourselves with what they would be made
of.

But things are not quite so simple.  First, adding to the
implausibility of Unger's view, he must deny that our use of ordinary
terms like `chair' (`stone', `cloud') follow any sort of pattern or
display any competence at all.  Second, even if we manage to swallow
that consequence, Unger has no explanation as to why we believe that
there are chairs (stones, clouds).

\subsection{Competence and correctness}
\label{comp}
Setting aside whether or not expressions of propositions like ``that's
a chair'' are ever \emph{true}, it seems right to say that there are
at least correct and incorrect uses of the terms.  For a word like
`chair' (`stone', `cloud') we generally do not say that a child has
learned how to use it until she is capable of deploying it in certain
ways.  We admit that she understands what `chair' (`stone', `cloud')
means or what a chair (stone, cloud) is when she displays a certain
competence with the term.  If instead of using `chair' to refer to
chairs she used it to refer to dogs or people, we would say that she
is confused and attempt to correct her use.

But Unger maintains that this is all an illusion, and that there is no
such thing as the correct or incorrect use of a term like `chair'
(`stone', `cloud'):

\begin{squote}
Concerning words and kinds, now, we might say this.  First, we might
say that it is in connection with \emph{semantics} that our reasonings have
what are their most obvious implications and, second, that their most
obvious semantic implications concern certain \emph{sortal nouns}, namely,
those which purport to denote ordinary things.  Thus, it appears quite
obvious to us now that there will be no application to things for such
nouns as `stone' and `rock', `twig' and `log', `planet' and `sun',
`mountain' and `lake', `sweater' and `cardigan', `telescope' and
`microscope', and so on, and so forth.  Simple positive sentences
containing these terms will never, given their current meanings,
express anything true, correct, accurate, etc., or even anything which
is anywhere close to being any of those things
(\citeyear[148]{unger1979}).
\end{squote}

This seems simply bizarre.  On what grounds, then, do parents correct
their children with respect to their use of ordinary terms?  Are they
compelled by some irrational force to consider certain utterances
correct and others incorrect?  One may question whether or not we use
ordinary term entirely consistently, but it seems false to say that,
necessarily, we {\em never} use (or have used, or will use) ordinary
terms in correct, as opposed to incorrect, ways.  As we shall see,
color words are susceptible to the sorites paradox as well, but that
should not make us think that we cannot use `red' successfully:

\begin{squote}
It is\,\ldots\,unclear how far our use of e.g. the vocabulary of
colours \emph{is} consistent.  The descriptions given of awkward cases
may vary from occasion to occasion.  Besides that, the notion of using
a predicate consistently would appear to require some objective
criteria for variation in relevant respects among items to be
described in terms of it; but what is distinctive about observational
predicates is exactly the lack of such criteria.  So it would be
unwise to lean too heavily, as though it were a matter of hard fact,
upon the consistency of our employment of colour predicates.  What,
however, may be depended upon is that our use of these predicates is
largely \emph{successful}; the expectations which we form on the basis
of others' ascriptions of colour are not usually disappointed.
Agreement is generally possible about how colours are to be described;
and this, of course, is equivalent to saying that others \emph{seem}
to use colour predicates in a largely consistent way
\citep[361]{wright1975}.
\end{squote}

It is because we do seem to use color and other words in a consistent
way that the sorites paradoxes are troubling.  Unger's conclusion is
that there is no such consistency.  But where do we get this
impression of consistency in the first place?

I am not sure how Unger would reply (though I'll try to think of
something).  Moreover, in the next section we will see that, having
denied the existence of ordinary things, Unger is incapable of
accounting for our believing in the existence of ordinary things.  It
is this explanatory deficiency which, I think, constitutes a larger
barrier to the acceptability of his thesis.

\section{Beliefs in things}
\label{beliefs}
Having denied that ordinary things exist, Unger should explain why we
nonetheless believe that there are chairs and other ordinary things.
Van Inwagen and Merricks claimed that our beliefs are both caused and
justified by arrangements of simples.  Although Merricks denies that
beliefs like ``there is a fine chair'' are strictly true, he agrees with
van Inwagen that they `get something right' in a way that beliefs like
``there is a dancing chair'' do not.  Their explanation for our
belief that there are chairs is that they are caused (and justified)
by a `nearby' or somehow related fact---that there are simples
arranged chairwise.  

Unger does not pursue this line.  He thinks our belief that there are
chairs, like our beliefs that there are other ordinary things, are not
justified, `nearly as good as true', or even coherent.  Unger says
that ``terms for ordinary things are incoherent [and] cannot apply to
anything real''~\citep[147]{unger1979}.

A consequence of this is that our language and thought concerning all
such things is directed toward {\em nothing at all}: ``it may well be
that I have never {\em thought of} any stones at all, or tables, or
even human hands.  If that is so, then it would seem that {\em a
  fortiori} I do not {\em know} anything {\em about these entities},
however commonly I might otherwise
suppose''~(\citeyear[458]{unger1980a}).

This all seems very strange.  Concerning ghosts, ``it is difficult
even to find a fully coherent belief that might be exposed as false;
we discover, at best, obscurity or perhaps confusion\,\ldots\,do we
really understand what sort of thing a ghost is supposed to
be''~\citep[76]{stroud2000a}?  If someone tries to tell me about the
ghost that visited him the previous night, it does not seem unjust to
say that he doesn't really know what he is talking about.  But can
this line be extended to some of the most common objects of
experience?

When we denied the existence of ghosts, we denied also others' beliefs
in them.  We did not, however, deny that people have beliefs which
they take to be about ghosts.  But we were able to show that these
beliefs were not {\em about} ghosts; in most cases they were about
nothing at all.  Likewise, Unger cannot deny that we have beliefs that
we take to be about tables, chairs, and all the other things that he
denies exist.  If our beliefs about tables and chairs are really
beliefs about nothing at all, then what causes us to form these
beliefs?  Why do we believe in ordinary things to begin with?

\subsection{Causes of belief}
\label{unger-cause}
People who believe in ghosts probably do so because they have
unreflectively embraced the superstitions of their culture.  They may
initially come to believe that ghosts exist on the testimony of other
people---older siblings, perhaps---or by reading too many ghost
stories.  Much as Catherine in Jane Austen's {\em Northanger Abbey}
jumps to the most macabre conclusions as a result of having absorbed
too many gothic novels, so might a gullible reader of ghost stories go
on to interpret such innocent phenomena as reflections of the moon as
ghostly assailants.  Those of us who have not taken our cues from
fiction would be more likely to recognize such phenomena as tricks of
the light.  Even if we were to see something that was definitely {\em
  not} a trick of the light, we would sooner attribute it to an
hallucination than countenance the possibility of ghosts.  Suppose
{\em you} saw what you took to be a ghost in an empty, well-lit room.
Most of us would still, even if presented with such a vision, {\em
  refuse to believe in ghosts}.  This is because we know that the
probability of there being such spirits is far less than the
probability of us experiencing cracks in our sanity.  Undermining my
belief that ghosts don't exist would require a great deal---for
example, my friend and I both apparently seeing the {\em same} ghost
at the same time, and knowing that we were each experiencing the same
vision.  (Even then, we would want further confirmed sightings to
convince us that we weren't, in fact, crazy.)

If this is an accurate characterization of our beliefs concerning
ghosts, it is a very different characterization than one we might give
of how we learn about and come to believe in chairs.  Chairs are not
something that children learn about from stories.  Rather, we learn
about chairs by coming across them in the world.  A child probably
learns what a chair is as an answer to the question, ``What is {\em
  that?}\,''

Let us suppose that the child is pointing at a chair in the center of
a well-lit room containing no other furniture.  The chair is clearly
visible.  If someone were to believe they were pointing at a ghost in
a similarly well-lit situation, we could safely assume that they would
be experiencing a hallucination.  But clearly the child is not
hallucinating.  There is {\em something} (or some things) in the
center of the room; what Unger wants to deny is that there is a {\em
  chair} there.

Unger would admit, I think, that there is a concentration of matter,
arranged in a certain way, in the center of the room.  (Unger
presumably also denies the existence of rooms, so this would have to
be expressed differently, but never mind.)  He sees it just as well
as we do; it's not as though Unger can't see straight.  All he's
saying is that what he is looking at is not a chair.

But if all that is in the center of the room is a mass of matter, {\em
  why do we believe that there is a chair there?}  To say that there
is a chair in the center of the room would, according to Unger, be
neither true, nor accurate, nor correct, nor ``anything which is
anywhere close to being any of those things''
(\citeyear[148]{unger1979}).  So where on earth do we get the idea
that there is a chair there?

Unger himself gives no answers.  Having denied the existence of all
ordinary things, he makes no attempt to explain why we have so many
false beliefs or what gives the impression of coherency to our use of
them in communication.  He seems almost to revel in the strangeness of
his position:

\begin{squote}
Now, it must of course be admitted that these arguments
undermine the possibility of any endeavor I should try to propose, or
even the putative thought that I should propose anything, just as all
of my putative essay is undermined.  But even so, I shall
(incoherently) propose that what we have now to do is invent new
expressions which are not inconsistent ones, and by means of which we
may, to some significant extent, think coherently about concrete
reality (\citeyear[544]{unger1980b}).
\end{squote}

\ldots

\ifstandalone
\bibliography{everything}
\bibliographystyle{ChicagoReedweb}
\end{spacing}
\fi
\end{document}
