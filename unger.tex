%\documentclass[11pt]{article}
%\usepackage[left=1.75in, right=1.75in, top=1.25in, bottom=1.25in]{geometry}
%\geometry{letterpaper}
%\usepackage{graphicx}
%%\usepackage{tipa}
%%\usepackage{exaccent}
%%\usepackage{txfonts}
%%\usepackage{pxfonts}
%\usepackage{enumitem}
%%\usepackage{amssymb}
%\usepackage{amsmath}
%\usepackage{epstopdf}
%\usepackage{setspace}
%\usepackage{natbib}
%\setcitestyle{aysep={}}
%\synctex=1
%
%\DeclareSymbolFont{symbolsC}{U}{txsyc}{m}{n}
%\DeclareMathSymbol{\strictif}{\mathrel}{symbolsC}{74}
%\DeclareMathSymbol{\boxright}{\mathrel}{symbolsC}{128}
%
%\newenvironment{squote}{\begin{quote}\begin{singlespace}}{\end{singlespace}\end{quote}}
%
%\newcommand{\stager}[4]%
%{%
%	\begin{spacing}{1}%
%	\vspace{0pt}
%		\begin{description}[style=nextline, noitemsep, parsep=0pt, topsep=0pt, leftmargin=15mm, itemindent=-10mm, font=\mdseries]
%			\item[\textsc{#1} \emph{#2}] #3
%			\item[]%
%			\begin{flushright}#4\end{flushright}
%		\end{description}%
%	\end{spacing}%
%}
%
%\newcommand{\stage}[3]%
%{%
%	\begin{spacing}{1}%
%	\vspace{0pt}
%		\begin{description}[style=nextline, parsep=0pt, leftmargin=15mm, itemindent=-10mm, font=\mdseries]
%			\item[\textsc{#1} \emph{#2}] #3
%		\end{description}%
%	\end{spacing}%
%}
%
%\title{Brute facts and arbitrary objects}
%\author{Alexander A. Dunn}
%\begin{document}
%\maketitle
%\begin{spacing}{1.5}

\chapter{Brute facts and arbitrary objects}
\chaptername{Brute Facts}
	\fancyhead[CE]{\textsc{Chapter\ \thechapter}}%
	\fancyhead[CO]{\textsc{\thetitle}}

%\fancyhead[LE,RO]{\thepage} 
%\fancyhead[CE]{\scshape \thechapter} 
%\fancyhead[CO]{\scshape Chapter 1}

\section{Is a ship more than its planks (and rigging, and sails)?}
The Special Composition Question was brought into focus by Peter van Inwagen. He finds that ``the metaphysically puzzling features of material objects are connected in deep and essential ways with metaphysically puzzling features of the constitution of material objects by their parts''~\citep[18]{inwagen1995}. A ready example is the Ship of Theseus: the planks and rigging and sails (and every part of the ship) are replaced as they individually wear out. These replacements happen each by themselves; the entire ship (nor even large sections) are swapped out all at once. But eventually no part of the original ship remains. And yet we would commonly say that it is still the same ship. But in virtue of what is the present ship identical with a past ship with which it shares no parts?

\subsection{The Special Composition Question}
Answering the question ``why is this ship identical with that past ship?'' requires first figuring out why (and how) these planks and rigging and sails (et.\ al.) compose a ship in the first place. Van Inwagen asks ``in what circumstances do planks\footnote{For simplicity's sake, van Inwagen ignores the rigging and sails.} compose (add up to, form) something?''~(\citeyear[21]{inwagen1995}) For some $x$s, then, van Inwagen asks us to consider when
\begin{equation}
\exists x\ \text{the}\ x\text{s compose}\ y
\end{equation}
is true.%
\footnote{Van Inwagen explains in some detail how plural referring expressions (like ``the planks'') can be given a logical formalization (\citeyear[23--28]{inwagen1995}), but suffice to say they work just as one would expect.} %
%
Less formally, van Iwagen asks: ``suppose one had certain (nonoverlapping) objects, the $x$s, at one's disposal; what would one have to do---what {\em could} one do---to get the $x$s to compose something?''~(\citeyear[31]{inwagen1995})

\subsection{The usual answers}
There are several prominent answers to the Special Composition Question, including the following:\footnote{These formulations are from~\citet{markosian1998a}.}
\begin{description}
	\item[Nihilism] Necessarily, for any $x$s, there is an object composed of the $x$s iff there is only one of the $x$s, i.e., the only objects that exist are simples~(\citeyear[219]{markosian1998a})%
		\footnote{Of course, it may be that the world is not fundamentally particulate, and is filled not with simples but with `gunk'; see \citet{schaffer2003}. Nihilism (and van Inwagen's second condition below) can be formulated to take this possibility into account: ``for any quantity of gunk, there is nothing composed of it.''}%
	\item[Van Inwagenism] Necessarily, for any $x$s, there is an object composed of the $x$s iff either (i) the activity of the $x$s constitutes a life or (ii) there is only one of the $x$s~(\citeyear[221]{markosian1998a}).
	\item[Universalism] Necessarily, for any $x$s, there is an object composed of the $x$s iff no two of the $x$s overlap~(\citeyear[227]{markosian1998a}).
\end{description}

Markosian argues that, from among these and all other possibilities, no satisfactory answer to the Special Composition Question can be found. He concludes that cases of composition are simply brute facts with no explanation connecting them to other cases of composition.

I think that Markosian does show that, {\em for a realist},%
\footnote{At least for a realist who believes that things that exist would exist in a world without us. [The question ``what would there be in a world without us?'' will be raised before this chapter.]} %
%
there is no good answer to the SCQ. If Markosian is a realist, he must admit that the fact that my desk exists is just a brute fact. It will likewise be a brute fact that there is not a desk missing a leg that exists in the exact location as my real desk (minus the missing leg).

This position accords well with common sense, but it smacks of arbitrariness. Below I will [maybe] argue that facts about what things do and do not exist is not arbitrary in this way. I think facts about what is in the world are not arbitrary in a way that a standard realist view of existence cannot account for. First, however, we ought to look at the answers to the SCQ that Markosian rejects.

Markosian tosses Nihilism with a refreshing lack of ado: ``according to Nihilism, there are no physical objects composed of many parts. So Nihilism seems to entail that you and I do not exist. And we can't have that''~(\citeyear[220]{markosian1998a}).

This will satisfy most of us, but Peter Unger won't take kindly to such a dismissal. Fortunately I think there is a fuller answer as to why Nihilism is wrong. My objection applies equally well to van Inwagen's version of Nihilism,\footnote{Markosian has an additional argument against van Inwagen, accusing him of countenancing vague objects~(\citeyear[222--223]{markosian1998a}).} which allows for living things to exist. (Spoiler: Nihilism is incoherent.)

\section{Nihilism (and why it's wrong)}
As explained above, Nihilism is the theory that no composite objects exist. If the universe is fundamentally particulate, then Nihilism claims that only simples (things with no proper parts) exist; if the universe is constituted of infinitely divisible gunk, then Nihilism claims either that only gunk exists, or perhaps nothing at all (because there are no simples; gunk is composed of more gunk).

Van Inwagen thinks that Nihilism is almost correct; other than simples (or gunk), nothing exists except for living organisms.%
%
\footnote{Unger is at least sympathetic to the idea that living things {\em do} exist~(\citeyear[151]{unger1979}); van Inwagen, as we have seen, makes much of this possibility.} %
%
But even with this generous exception, I think van Inwagenism is incoherent for the same reasons that normal Ungerian Nihilism is.

\subsection{What are they talking about?}
Peter Unger is the most vocal proponent of metaphysical nihilism. This is the view that very few, if any, of the objects we normally think exist actually do exist. He denies the existence of such things as ``tables and chairs and spears\,\ldots\,swizzle sticks and sousaphones\,\ldots\,stones and rocks and twigs, and also tumbleweeds and fingernails''~(\citeyear[117]{unger1979}). He draws this conclusion from an application of the sorites paradox:
\begin{squote}
Consider a stone, consisting of a certain finite number of atoms. If we or some physical process should remove one atom, without replacement, then there is left that number minus one, presumably constituting a stone still\,\ldots\,after another atom is removed, there is that original number minus two; so far, so good. But after that certain number has been removed, in similar stepwise fashion, there are no atoms at all in the situation, while we must still be supposing that there is a stone there. But as we have already agreed, if there is a stone present, then there must
be some atoms\,\ldots\,I suggest that any adequate response to this contradiction must include\,\ldots\,the denial of the existence of even a single stone.~\citep[121--122]{unger1979}
\end{squote}
This seems crazy. To say that there are no ordinary things is to say that our ``terms for ordinary things are incoherent [and] cannot apply to anything real''~\citep[147]{unger1979}. A consequence of this is that most of our language and thought is directed toward {\em nothing at all}: ``when we are under the impression that we are thinking about an object in the world\,\ldots\,our impression is mistaken''~(\citeyear[149]{unger1979}).

Peter van Inwagen agrees that there are no ordinary things, but tries to make it seem less bizarre. Indeed, he admits that ``when people say things in the ordinary business of life by uttering sentences that start `There are chairs\,\ldots\,' or `There are stars\,\ldots\,', they very often say things that are literally true''~\cite[102]{inwagen1995}.%
%
\footnote{He argues (persuasively) that ``there is an $x$'' is equivalent to ``$\exists x$'' in \citet{inwagen1998}.} %
%
Great! So there are chairs after all, then? Sadly, no. What van Inwagen is getting at is that {\em we}, when saying things like ``Some chairs are heavier than some tables'', {\em actually mean} this instead: ``There are $x$s that are arranged chairwise and there are $y$ that are arranged tablewise and the $x$s are heavier than the $y$s''~(\citeyear[109]{inwagen1995}).

\subsection{What do they mean by it?}
This seems crazy. For a start, what is it for something to be arranged chairwise? Can the nihilists define this arrangement? To be arranged chairwise is just to be arranged in the shape of a chair. If we ask, ``what is the shape of a chair?'' the nihilist can only say, ``the same shape as simples arranged chairwise'', which tells us nothing. And remember, we're supposing that chairs don't exist; how helpful could it be, really, to explain the shape of something by reference to something that does not exist?%
%
\footnote{Not only are there (supposedly) no chairs, but there never have been: ``certain kinds are never instanced''~\citep[147]{unger1979}.} %
%
The nihilist may say, ``there are no chairs, but that does not prevent simples from being arranged chairwise.'' Yet if there are no chairs, then how do we know what a chairwise arrangement {\em looks like}? ``We have access to the Form of the Chair through our rational intuition.'' Perhaps. But the problem is that the form of the chair is, by the nihilist's own lights, the form of a {\em metaphysically impossible} object. (Unger seems to think they're {\em logically} impossible.) It seems similar to a florist being asked to arrange flowers in the shape of a round square. ``What does a round square look like?'' {\em Nothing}.

\subsection{Do {\em they} know what they're talking about?}
If the nihilists agree that their analysis of `chairwise' is circular, they might just try pointing. As an answer to the question ``what is it to be arranged chairwise?'' van Inwagen may demonstrate the chair and say ``to be like that!'' He could avoid the painful word `chair' altogether.

For example, suppose van Inwagen takes a fancy to our chair. He might try bargaining for it by saying, ``That's a fine batch of simples! I wonder how much they cost?'' If we're feeling peevish we will reply, ``which simples do you mean?'' And he will say ``Those ones in the corner!'' Which ones? ``The ones arranged chairwise!'' And of course we reply ``what do you mean `chairwise'?'' If van Inwagen says, ``like a chair, dummy!'' then we're back where we started.

He might just growl, ``you know exactly which simples I mean.'' And of course we do. {\em But only because we know what chairs look like}. If the nihilist is going to actually talk as if chairs didn't exist, he'll have to give up this `chairwise' nonsense. But then how will he bargain for the chair? Can he say anything more than, ``I want some of the simples over there''? {\em We} know what he's talking about, because we know he really does want that chair. But imagine two nihilists talking to each other:

\stage{Unger}{}{I'm hungry. Are there any simples?}
\stage{van Inwagen}{}{Check behind the gunk.}

The point is that nihilism would make most communication {\em impossible}. Unger and van Inwagen {\em can't} consistently be nihilists and ask if there's any food in the cupboard. If, as Unger claims, these ordinary terms are incoherent, then how could one refer with them? Referential communication involves the speaker using terms to make salient certain things, which, if communication is successful, the hearer recognizes as the objects of the speaker's referential intention. If someone tries to communicate with a nihilist using terms such as `chair' and `food', can the nihilist have {\em any idea} what the speaker is trying to refer to using these words for allegedly impossible objects? It would have to be something like our saying ``that is a nice round square one, don't you think?''%Reference cannot succeed because there is, according to the nihilists, {\em nothing to refer to} other than simples.
%
%cf. Stroud and Davidson on representing the beliefs of others

The nihilist is claiming that our ordinary terms purport to designate impossible things and so are incoherent. But the nihilist will have to explain how we manage to have so much successful communication using terms that apparently should have no possible application, ever. Our talk about chairs cannot be paraphrased into talk about simples because, as the nihilists are so insistent on pointing out, there is no precise definition of chair; hence the sorites paradox. And yet we {\em do} communicate successfully using terms like `chair'.\footnote{I want to set this down as an axiom: \textsc{Communication occurs.}} If Unger replies that communication does not, in general, occur, he would have to mean that, among other areas of discourse, most science is empty of content. At this point we can ignore him; nihilism is itself incoherent.

\section{Universalism (everyone, everywhere, everything)}
If Nihilism is false, then there are at least some composite objects. Now that we've opened the door for tables and chairs, the easiest answer to the SCQ is to let everything in. I get my desk back, but I also get the desk that's coextensive with mine, but missing a leg. We also get ``an object composed of (i) London Bridge, (ii) a certain sub-atomic particle located far beneath the surface of the moon, and (iii) Cal Ripken, Jr.''~\citep[228]{markosian1998a} %
%We might also find that a wish sandwich isn't just a figment of the Blues Brothers' imaginations (``a wish sandwich is a kind of a sandwich where you have two slices of bread and you {\em wish} you had some meat'').
This is what we might call an embarrassment of riches.

Is this strangeness enough to reject Universalism? Probably; I don't believe that there are billions of objects in my room that are all indistinguishable from my desk. (If there were, how would I know which one {\em was} my desk?)

\subsection{Can Universalism find room for me?}
But van Inwagen is not satisfied. He wants to show that Universalism also entails his own non-existence.\footnote{A consequence which, for the rest of us, might count as a virtue of the theory.} His argument is that if we are physical organisms, then at a given time $t_{1}$ we are composed of some $x$s. These $x$s are the atoms that compose the organism that is identical with us. But we are gaining and shedding parts all the time; van Inwagen guesses that ten years after $t_{1}$, we are composed of all new atoms. (We resemble the Ship of Theseus in this way.) If at $t_{1}$ our body was the set of atoms that constituted us (the organism), then ten years later we are scattered across the world. The present organism is someone else.

This would be absurd: a theory that promises existence for all, but fails to account for living things! However, I don't think this is a damning objection. Why not take an idea from van Inwagen himself, and reformulate Universalism thus?
\begin{description}
	\item[Superduperuniversalism] Necessarily, for any $x$s, there is an object composed of the $x$s iff either (i) no two of the $x$s overlap or (ii) the activity of the $x$s constitutes a life.\footnote{Condition (ii) is from \citet[82]{inwagen1995}. A life for van Inwagen is something like ``an unimaginably complex self-maintaining storm of atoms''~(\citeyear[87]{inwagen1995}). But nothing important depends on this; Universalism is false and van Inwagenism, being a variant of Nihilism, is incoherent.}
\end{description}
Even if this is coherent, it still leaves us saddled with a lot of objects that we didn't ask for. And while it accounts for the changes that organisms undergo over time, it still entails that, due to the shedding of atoms, the desk I am sitting at now is not the same desk as five minutes ago. It might be possible to add {\em another} condition to the definition of Superduperuniversalism, but I can only handle so much absurdity in one definition. Universalism is false.

\section{Other answers that are false}
oink

%\bibliography{everything}
%\bibliographystyle{ChicagoReedweb}
%
%\end{spacing}
%\end{document}
