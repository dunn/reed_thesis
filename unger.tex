%\documentclass[11pt]{article}
%\usepackage[left=1.75in, right=1.75in, top=1.25in, bottom=1.25in]{geometry}
%\geometry{letterpaper}
%\usepackage{graphicx}
%%\usepackage{tipa}
%%\usepackage{exaccent}
%%\usepackage{txfonts}
%%\usepackage{pxfonts}
%\usepackage{enumitem}
%%\usepackage{amssymb}
%\usepackage{amsmath}
%\usepackage{epstopdf}
%\usepackage{setspace}
%\usepackage{natbib}
%\setcitestyle{aysep={}}
%\synctex=1
%
%\DeclareSymbolFont{symbolsC}{U}{txsyc}{m}{n}
%\DeclareMathSymbol{\strictif}{\mathrel}{symbolsC}{74}
%\DeclareMathSymbol{\boxright}{\mathrel}{symbolsC}{128}
%
%\newenvironment{squote}{\begin{quote}\begin{singlespace}}{\end{singlespace}\end{quote}}
%
%\newcommand{\stager}[4]%
%{%
%	\begin{spacing}{1}%
%	\vspace{0pt}
%		\begin{description}[style=nextline, noitemsep, parsep=0pt, topsep=0pt, leftmargin=15mm, itemindent=-10mm, font=\mdseries]
%			\item[\textsc{#1} \emph{#2}] #3
%			\item[]%
%			\begin{flushright}#4\end{flushright}
%		\end{description}%
%	\end{spacing}%
%}
%
%\newcommand{\stage}[3]%
%{%
%	\begin{spacing}{1}%
%	\vspace{0pt}
%		\begin{description}[style=nextline, parsep=0pt, leftmargin=15mm, itemindent=-10mm, font=\mdseries]
%			\item[\textsc{#1} \emph{#2}] #3
%		\end{description}%
%	\end{spacing}%
%}
%
%\title{What are these nihilists talking about?}
%\author{Alexander A. Dunn}
%\begin{document}
%\maketitle
%\begin{spacing}{1.5}
\chapter{There is no real answer to the Special Composition Question}

\section{Who are these nihilists?}
Peter Unger is the most vocal proponent of metaphysical nihilism. This is the view that very few, if any, of the objects we normally think exist actually do exist.%
%
\footnote{Unger is at least sympathetic to the idea that living things {\em do} exist~(\citeyear[151]{unger1979}); Peter van Inwagen makes much of this exception in his book {\em Material Beings}.}%
 %
He denies the existence of such things as ``tables and chairs and spears\,\ldots\,swizzle sticks and sousaphones\,\ldots\,stones and rocks and twigs, and also tumbleweeds and fingernails''~(\citeyear[117]{unger1979}). He draws this conclusion from an application of the sorites paradox:
\begin{squote}
Consider a stone, consisting of a certain finite number of atoms. If we or some physical process should remove one atom, without replacement, then there is left that number minus one, presumably constituting a stone still\,\ldots\,after another atom is removed, there is that original number minus two; so far, so good. But after that certain number has been removed, in similar stepwise fashion, there are no atoms at all in the situation, while we must still be supposing that there is a stone there. But as we have already agreed, if there is a stone present, then there must
be some atoms\,\ldots\,I suggest that any adequate response to this contradiction must include\,\ldots\,the denial of the existence of even a single stone.~\citep[121--122]{unger1979}
\end{squote}
This seems crazy. To say that there are no ordinary things is to say that our ``terms for ordinary things are incoherent [and] cannot apply to anything real''~\citep[147]{unger1979}. A consequence of this is that most of our language and thought is directed toward {\em nothing at all}: ``when we are under the impression that we are thinking about an object in the world\,\ldots\,our impression is mistaken''~(\citeyear[149]{unger1979}).

Peter van Inwagen agrees that there are no ordinary things, but tries to make it seem less bizarre. Indeed, he admits that ``when people say things in the ordinary business of life by uttering sentences that start `There are chairs\,\ldots\,' or `There are stars\,\ldots\,', they very often say things that are literally true''~\cite[102]{inwagen1995}.\footnote{He argues (persuasively) that ``there is an $x$'' is equivalent to $\exists x$ in \citet{inwagen1998}.} Great! So there are chairs after all, then? Sadly, no. What van Inwagen is getting at is that {\em we}, when saying things like ``Some chairs are heavier than some tables'', {\em actually mean} this instead: ``There are $x$s that are arranged chairwise and there are $y$ that are arranged tablewise and the $x$s are heavier than the $y$s''~(\citeyear[109]{inwagen1995}).

\section{What do they mean by it?}
This seems crazy. For a start, what is it for something to be arranged chairwise? Can the nihilists define this arrangement? To be arranged chairwise is just to be arranged in the shape of a chair. If we ask, ``what is the shape of a chair?'' the nihilist can only say, ``the same shape as simples arranged chairwise'', which tells us nothing. And remember, we're supposing that chairs don't exist; how helpful could it be, really, to explain the shape of something by reference to something that does not exist?%
%
\footnote{Not only are there (supposedly) no chairs, but there never have been: ``certain kinds are never instanced''~\citep[147]{unger1979}.}%
 %
The nihilist may say, ``there are no chairs, but that does not prevent simples from being arranged chairwise.'' Yet if there are no chairs, then how do we know what a chairwise arrangement {\em looks like}? ``We have access to the Form of the Chair through our rational intuition.'' Perhaps. But the problem is that the form of the chair is, by the nihilist's own lights, the form of a {\em metaphysically impossible} object. (Unger seems to think they're {\em logically} impossible.) It seems similar to a florist being asked to arrange flowers in the shape of a round square. ``What does a round square look like?'' {\em Nothing}.

\section{Do {\em they} know what they're talking about?}
If the nihilists agree that their analysis of `chairwise' is circular, they might just try pointing. As an answer to the question ``what is it to be arranged chairwise?'' Unger may demonstrate the chair and say ``to be like that!'' He could avoid the painful word `chair' altogether.

For example, suppose Unger takes a fancy to our chair. He might try bargaining for it by saying, ``That's a fine batch of simples! I wonder how much they cost?'' If we're feeling peevish we will reply, ``which simples do you mean?'' And he will say ``Those ones in the corner!'' Which ones? ``The ones arranged chairwise!'' And of course we reply ``what do you mean `chairwise'?'' If Unger says, ``like a chair, dummy!'' then we're back where we started.

He might just growl, ``you know exactly which simples I mean.'' And of course we do. {\em But only because we know what chairs look like}. If the nihilist is going to actually talk as if chairs didn't exist, he'll have to give up this `chairwise' nonsense. But then how will he bargain for the chair? Can he say anything more than, ``I want some of the simples over there''? {\em We} know what he's talking about, because we know he really does want that chair. But imagine two nihilists talking to each other:

\stage{Unger}{}{I'm hungry. Are there any simples?}
\stage{van Inwagen}{}{Check behind the gunk.}

The point is that real nihilism would make most communication {\em impossible}. Unger and van Inwagen {\em can't} consistently be nihilists and ask if there's any food in the cupboard. If, as Unger claims, these ordinary terms are incoherent, then how could one refer with them? Referential communication involves the speaker using terms to make salient certain things, which, if communication is successful, the hearer recognizes as the objects of the speaker's referential intention. If someone tries to communicate with a nihilist using terms such as `chair' and `food', can the nihilist have {\em any idea} what the speaker is trying to refer to using these words for allegedly impossible objects? It would have to be something like our saying ``that is a nice round square one, don't you think?''%Reference cannot succeed because there is, according to the nihilists, {\em nothing to refer to} other than simples.
%
%cf. Stroud and Davidson on representing the beliefs of others

The nihilist is claiming that our ordinary terms purport to designate impossible things and so are incoherent. But the nihilist will have to explain how we manage to have so much successful communication using terms that apparently should have no possible application, ever. Our talk about chairs cannot be paraphrased into talk about simples because, as the nihilists are so insistent on pointing out, there is no precise definition of chair; hence the sorites paradox. And yet we {\em do} communicate successfully using terms like `chair'.\footnote{I want to set this down as an axiom: \textsc{Communication occurs.}} If Unger replies that communication does not, in general, occur, he would have to mean that, among other areas of discourse, most science is empty of content. At this point we can ignore him.

%\bibliography{everything}
%\bibliographystyle{ChicagoReedweb}
%
%\end{spacing}
%\end{document}
