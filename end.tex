\documentclass[11pt]{article}
\usepackage{standalone} \newif\ifstandlone \standalonetrue
\usepackage[left=1.75in, right=1.75in, top=1.25in, bottom=1.25in]{geometry}
\geometry{letterpaper}
\usepackage{graphicx}
\usepackage{enumitem}
%\usepackage{amssymb}
\usepackage{amsmath}
\usepackage{epstopdf}
\usepackage{verbatim}
\usepackage{setspace}
\usepackage{natbib}
\setcitestyle{aysep={}}
\usepackage{hyperref}
\usepackage{url}
\synctex=1

\DeclareSymbolFont{symbolsC}{U}{txsyc}{m}{n}
\DeclareMathSymbol{\strictif}{\mathrel}{symbolsC}{74}
\DeclareMathSymbol{\boxright}{\mathrel}{symbolsC}{128}

\newenvironment{squote}{%
\begin{spacing}{1}
\begin{list}{}{%
\setlength{\labelwidth}{0pt}%
\rightmargin\leftmargin%
}
\item\relax
}{%
\end{list}%
\end{spacing}
}

\title{The End: Oh Make it Stop}
\author{Alexander A. Dunn}
\begin{document}
\ifstandalone
\maketitle
\begin{spacing}{1.5}
\fi

I have argued for a number of claims in the preceding sections.

First, debates in metaphysics such as the one I have been engaged in
are conducted in English (or French, or German) and not in
`Ontologese' or some other pseudo-language.  If it is `really' or
`fundamentally' the case that there are no chairs, then it is true in
English that there are no chairs.

Second, philosophers who deny that there are chairs appear unable to
explain why we nonetheless believe that there are chairs.  Trenton
Merricks, who denies that there are chairs, has an explanation of why
we believe nonetheless that there are chairs.  He claims that because
``things arranged chairwise'' matter to us, we have introduced the
word `chair' to refer to them; we are fooled by the singular nature of
the word `chair' and come to think that there is some single {\em
  thing} that we are referring to, when in fact there is not.  His
explanation, however, is equally compatible with universalism: the
claim that for every set of things, there is some other thing they
compose.  And universalism is a much more plausible thesis than the
nihilism of Merricks.

Third, if we assume that universalism is true then we have a choice to
make.  We can either adopt a `plurality' theory that posits a huge
plurality of things (and different {\em kinds} of things), or we can
adopt a version of {\em essentialism}, maintaining that, strictly
speaking, things do not change their parts over time.  I have
suggested that while both routes are defensible, the essentialist
theory avoids some of the excesses of co-location that plague the
plurality theories while offering some neat solutions to problems of
personal identity over time.  However, neither solution appears to
offer a ready answer to the mental problem of the many: how it is that
with many mereological sums that are all equally `suitable' to be
identified with the one and only thinking thing in a particular
location, one and only one of those sums is a thinking thing? \\

In the Introduction and in Section \ref{stroud}, I emphasized that my
opposition to metaphysical nihilism was based, largely, on the fact
that it is {\em obviously true} that there are chairs.  I claimed to
be arguing for what is clearly so, and rejecting what is clearly not.

But surely, mereological essentialism is not {\em obviously} true.
Some would say it is obviously false.  In either case, I can no longer
claim to be arguing for what is clearly so.

But not everything is obvious in metaphysics.  (This itself should be
obvious.)  There are certain things that are obviously true, and many
other things that are not; but they are no less true.  It is obvious
that there are chairs; given that, what are they like?  What sort of
thing are they?  Can they change their parts?  If these questions have
answers, they are not obvious.

Moreover we seem to be stuck between two possibilities, both of which
might be decried as obviously false.  If there are chairs, and all the
other things that universalism entails, then either things change
their parts or they do not.  If things change their parts, then (again
ignoring four-dimensionalism) there must be very many co-located
things.  If someone finds this clearly not true, then they may be
forced to conclude that things do not change their parts.

This thesis therefore ends with no fully-formed theory.  I have, in
effect, offered a disjunction: either a plurality theory or an
essentialist theory is correct. I have no decisive intuitions on this
score.  I appreciate the minimalism of the essentialist solution,
which dissolves problems of persistence and identity over time.  But I
recognize its strangeness, and see also the strengths of theories that
posit pluralities of things.

\ifstandalone
\end{spacing}
\bibliography{everything}
\bibliographystyle{ChicagoReedweb}
\fi
\end{document}
