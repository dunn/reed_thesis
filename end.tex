\documentclass[11pt]{article}
\usepackage{standalone} \newif\ifstandlone \standalonetrue
\usepackage[left=1.75in, right=1.75in, top=1.25in, bottom=1.25in]{geometry}
\geometry{letterpaper}
\usepackage{graphicx}
\usepackage{enumitem}
%\usepackage{amssymb}
\usepackage{amsmath}
\usepackage{epstopdf}
\usepackage{verbatim}
\usepackage{setspace}
\usepackage{natbib}
\setcitestyle{aysep={}}
\usepackage{hyperref}
\usepackage{url}
\synctex=1

\DeclareSymbolFont{symbolsC}{U}{txsyc}{m}{n}
\DeclareMathSymbol{\strictif}{\mathrel}{symbolsC}{74}
\DeclareMathSymbol{\boxright}{\mathrel}{symbolsC}{128}

\newenvironment{squote}{%
\begin{spacing}{1}
\begin{list}{}{%
\setlength{\labelwidth}{0pt}%
\rightmargin\leftmargin%
}
\item\relax
}{%
\end{list}%
\end{spacing}
}

\title{The End: Oh Make it Stop}
\author{Alexander A. Dunn}
\begin{document}
\ifstandalone
\maketitle
\begin{spacing}{1.5}
\fi

I have argued for a number of claims in the preceding sections.

First, debates in metaphysics such as the one I have been engaged in
are conducted in English (or French, or German) and not in
``Ontologese'' or some other pseudo-language.  If it is ``really'' or
``fundamentally'' the case that there are no chairs, then ``there are
no chairs'' is true in English.

Second, philosophers who deny that there are chairs have some
difficulty explaining why we nonetheless believe that there are
chairs.  Van Inwagen's explanation fails outright.  Trenton Merricks
claims that because things arranged chairwise matter to us, we have
introduced the word `chair' to refer to them; we are fooled by the
singular nature of the word `chair' and come to think that there is
some single {\em thing} that we are referring to, when in fact there
is not.  This explanation, however, is equally compatible with
universalism: the claim that for every set of things, there is some
other thing they compose.  And universalism is a much more plausible
thesis than the nihilism of Merricks.

Third, if we assume that universalism is true then we have a choice to
make.  We can either adopt a ``plurality theory'' that posits a huge
number of things (and possibly different {\em kinds} of things), or we
can adopt a version of essentialism, maintaining that, strictly
speaking, things do not change their parts over time.  I have
suggested that the essentialist theory avoids some of the excesses of
co-location that plague the plurality theories while offering some
neat solutions to problems of personal identity over time.  But
neither route is obviously superior, and both are defensible.\\

In the Introduction and in Section \ref{stroud}, I emphasized that my
opposition to metaphysical nihilism was based, largely, on the fact
that it is {\em obviously true} that there are chairs.  I claimed to
be arguing for what is clearly so, and rejecting what is clearly not.

But surely, mereological essentialism is not {\em obviously} true.
Some would say it is obviously false.  In either case, I can no longer
claim to be arguing for what is clearly so.

But not everything is clear in metaphysics.  ({\em This} is obviously
true.)  There are a few things that are obviously true; many other
things are not, but they are no less true.  It is obvious that there
are chairs; given that, what are they like?  What sort of thing are
they?  Can they change their parts?  If these questions have answers,
they are not obvious.

Moreover we seem to be forced to choose between two possibilities,
both of which might be decried as obviously false.  If there are
chairs, and all the other things that universalism entails, then
either things change their parts or they do not.  If things change
their parts, then (again ignoring four-dimensionalism) there must be
very many co-located things.  Someone who takes this to be false may
be forced to conclude that things do not change their parts.

This thesis therefore ends with no fully-formed theory.  I have
offered a disjunction: either a plurality theory or an essentialist
theory is correct. I have no decisive intuitions here.  I appreciate
the minimalism of the essentialist solution, which dissolves problems
of persistence and identity over time.  But I recognize its
strangeness, and see also the strengths of theories that posit
pluralities of things.

\ifstandalone
\end{spacing}
\bibliography{everything}
\bibliographystyle{ChicagoReedweb}
\fi
\end{document}
