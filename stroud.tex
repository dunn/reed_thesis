\documentclass[11pt]{article}
\usepackage{standalone} \newif\ifstandlone \standalonetrue
\usepackage[left=1.75in, right=1.75in, top=1.25in, bottom=1.25in]{geometry}
\geometry{letterpaper}
\usepackage{graphicx}
\usepackage{enumitem}
\usepackage{amssymb}
\usepackage{amsmath}
\usepackage{verbatim}
\usepackage{epstopdf}
\usepackage{setspace}
\usepackage{natbib}
\setcitestyle{aysep={}}
\usepackage%[colorlinks=true, citecolor=blue, linkcolor=black]%
{hyperref}

\synctex=1

\DeclareSymbolFont{symbolsC}{U}{txsyc}{m}{n}
\DeclareMathSymbol{\strictif}{\mathrel}{symbolsC}{74}
\DeclareMathSymbol{\boxright}{\mathrel}{symbolsC}{128}

\newcommand{\stager}[4]%
{%
	\begin{spacing}{1}%
	\vspace{0pt}
		\begin{description}[style=nextline, noitemsep,
                    parsep=0pt, topsep=0pt, leftmargin=15mm,
                    itemindent=-10mm, font=\mdseries]
			\item[\textsc{#1} \emph{#2}] #3
			\item[]%
			\begin{flushright}#4\end{flushright}
		\end{description}%
	\end{spacing}%
}

\newcommand{\stage}[3]%
{%
	\begin{spacing}{1}%
	\vspace{0pt}
		\begin{description}[style=nextline, parsep=0pt,
                    leftmargin=15mm, itemindent=-10mm, font=\mdseries]
			\item[\textsc{#1} \emph{#2}] #3
		\end{description}%
	\end{spacing}%
}

\newenvironment{squote}{%
	\begin{spacing}{1}
	\begin{list}{}{%
	\setlength{\labelwidth}{0pt}%
	\rightmargin\leftmargin%
	}
	%\begin{singlespace}%
	\item\relax
	}{%
	%\end{singlespace}%
	\end{list}%
	\end{spacing}
	}

\newenvironment{inq}{\vspace{0pt}%
	\begin{list}{}%
	{\setlength\labelwidth{0pt}%
	\setlength\leftmargin{2.5\oddsidemargin}%
	\setlength\rightmargin{\leftmargin}}
	\begin{spacing}{1}
	\item[]%
	}{
	\end{spacing}
	\end{list}
	\vspace{10pt}
	%\noindent%
	}

\title{Why do you think that?}
\author{Alexander A. Dunn}
\begin{document}
\ifstandalone
\maketitle
\begin{spacing}{1.5}
\fi
\label{stroud}

A nihilistic metaphysical thesis should be accompanied by an
explanation of why people nonetheless believe that there are chairs
and other ordinary things.  Peter van Inwagen and Trenton Merricks
each have the beginnings of such an explanation.  The explanation
offered by van Inwagen is flawed, but Merricks' is promising.  I
expand on what I take to be Merricks explanation of why we believe
that there are chairs, and conclude that is is successful, given one
other assumption.  That assumption is the denial of metaphysial
universalism.  Howver, universalism is independently plausible, and
{\em its} unintuitive consequences can be satisfactorily explained
using an adaptation of Merricks' own theory.

\section{Explaining the beliefs of others}
\label{intro-beliefs}
\noindent Many people have false beliefs.  They believe things that
misrepresent (in some sense) how the world is.  For example, some
people believe that ghosts exist.  These people each hold a false
belief, for it is not true that ghosts exist.  There are no ghosts in
the world.  Despite this fact---that there are no ghosts---some people
believe that there are.  Why?  What explanation can we give as to why
someone believes a falsehood like this?

In explaining why someone holds a belief, we appeal to {\em reasons}.
Even people who hold beliefs that we may consider irrational (like the
belief that there are ghosts) have reasons for holding these beliefs.
They may not be good reasons; someone might believe that there are
ghosts because her older sister told her that there are ghosts, or
because she read ghost stories as a child and took them seriously.
Someone who believes in ghosts might even think that she has {\em
  seen} a ghost.  This too would be a false belief; there are no
ghosts, so nobody can have seen one.  But here too there will be a
reason why she holds this false belief.  Perhaps she saw a strange
play of light on a distant wall, or the reflection of the moon
filtered through an attic window.  What she actually saw was perhaps
one of these things, but she somehow took what she saw to be a ghost.
Probably she already believed that there were ghosts, and so, when
confronted with a deceptive or confusing sight, was predisposed to
form the mistaken belief that she was seeing a ghost.

Here and in what follows, when I say that there is a reason why
someone believes something, I mean that there is some {\em cause} that
produced the belief.  Above, I told a causal story about why the
person who believes that she saw a ghost holds that belief.  She had
been told that there were ghosts by a person who she thought
trustworthy, so she came to believe that there are ghosts.  Holding
that belief caused her to be predisposed to interpret unusual
phenomena as ghosts.  This disposition caused her to believe that she
was seeing a ghost when she saw a reflection of the moon.

My use of `reason', therefore, should be taken in this causal sense.
There are other ways that people use the word `reason'.  If someone
asks ``What reason do you have to believe that $((P \rightarrow Q )
\wedge P) \rightarrow Q$?''  I might reply that it is a theorem of
first-order logic.  Here I am not telling a causal story.  I am rather
{\em justifying} my belief that $((P \rightarrow Q ) \wedge P)
\rightarrow Q$.  But in this case it is perfectly correct to say that
I am giving a reason as to why I hold a belief.  It is just not a {\em
  causal} reason.  A causal reason would be something like the
following: $((P \rightarrow Q ) \wedge P) \rightarrow Q$ is true, and
I have done the proof.

(Another example: suppose someone falsely believes that $((P
\rightarrow Q ) \wedge Q) \rightarrow P$ is a theorem of first-order
logic.  There will be some (causal) reason why they hold this belief;
probably they attempted to deduce it from no premises and believe that
they succeeded.  There will, in turn, be a reason why they hold {\em
  this} false belief; maybe they were not concentrating on the proof
steps, or they forgot certain rules of deduction.)

An example involving an apparently obviously true belief might help
clarify the distinction between causal reason and justifying reasons.
If someone were to ask me why I believe that the sky is blue during
the day, my immediate answer would probably be ``well, because it
is!''  There's not much else I can say to {\em justify} my belief.
But this not a {\em causal} explanation.  The fact that something is
true (the sky {\em is} blue) does not cause me to believe it.
Otherwise I would believe every truth, and I do not.  There are
doubtless many truths that I do not believe.  There must therefore be
another (causal) reason why I believe that the sky is blue, other than
the fact that the sky is blue.

I believe that the sky is blue because, first, it is blue, and second,
I have {\em seen} that it is blue.  My vision is generally reliable
(or at least seems to be), so the fact that my eyes `tell' me
something is good reason to believe it.  The same is true of my other
senses: they are generally reliable, so the fact that they `tell' me
something is a good reason to believe it.  It does not follow that it
is {\em true}, however (though no doubt we believe that it is true);
our eyes can be deceived.

A skeptic might claim that we cannot rule out the possibility that we
are {\em constantly} deceived.  They attempt to undermine the
reliability of our senses.  I will not be addressing such arguments.
Rather, in what follows I will examine arguments that deny (or appear
to deny) that many of our beliefs about `ordinary things' are true.
The philosophers making these denials do not claim that our eyes are
unreliable sources of information.  Their arguments are metaphysical
rather than epistemic; they deny that certain objects are {\em
  possible}.  

For example, Trenton Merricks believes that chairs do not exist.  He
relies on a number of metaphysical arguments to motivate this claim.
If he is right, however, then it seems to follow from this that
beliefs like ``there are chairs'' are necessarily false.  I, however,
believe that there are chairs.  Even if Merricks is right, and my
belief is (necessarily) false, it still seems to be the case that
there are reasons why I believe this.

If someone were to ask {\em me} why I believe that there are chairs, I
would probably answer ``because there are, and I have seen them (and
sat upon them)!''  It seems obviously true, just like the fact that
the sky is blue.  I have seen lots of chairs, and I can't have been
confused or deceived {\em every} time.

Nonetheless, Trenton Merricks, Peter van Inwagen and other
philosophers say that I am mistaken.  They claim that I have not in
fact seen lots of chairs, though I may believe that I have.  There are
several different arguments by which nihilists seek to establish that
chairs (and other `ordinary things') do not exist; we will examine
some of these arguments below.  Having made these arguments, however,
the nihilists must reject our causal explanation of why we believe
that there are chairs.  Our explanation was that there are chairs and
we can see them.  But the nihilist denies that there are chairs, and
so should admit that, if we believe that there are chairs, there must
be a different explanation as to why we hold this belief.

\subsection{Paraphrasing beliefs}
\label{paraphrase}
Trenton Merricks denies that chairs exist, and claims that, if we
believe that chairs exist, we are mistaken.  His task will be to
explain why we form these false beliefs.  But not all nihilistic
philosophers deny that we are, in fact, mistaken.  They deny that
there are any chairs, but maintain that beliefs like the following
might still be true:

\begin{itemize}
  \item There are two chairs in the next room.
  \item I own some very nice 17th-century chairs.
  \item Some chairs are heavier than some tables.
\end{itemize}

Peter van Inwagen is one of these philosophers.  He denies the
existence of tables, chairs, apples, and all other inanimate composite
objects (van Inwagen's technical definition of `composite' will be
discussed below in sections \ref{scq} and \ref{tech}).  He allows that
the sort of propositions listed above may be true, but insists that
this does not mean that there are chairs (or tables):
\begin{squote}
I want to do what I can to disown a certain apparently almost
irresistible characterization of my view, or of that part of my view
that pertains to inanimate objects.  Many philosophers, in
conversation and correspondence, have insisted, despite repeated
protests on my part, on describing my position in words like these:
``Van Inwagen says that tables are not real''; ``\,\ldots not true
objects''; ``\,\ldots not actually {\em things}''; ``\,\ldots not
substances''; ``\,\ldots not unified wholes''; ``\,\ldots nothing more
than collections of particles.''  These are words that darken counsel.
They are, in fact, perfectly meaningless.  My position vis-\`{a}-vis
tables and other inanimate objects is simply that there {\em are}
none~(\citeyear[99]{inwagen1995}).
\end{squote}

Van Inwagen asserts, quite seriously, that ``there are no tables or
chairs or any other visible objects except living organisms''
(\citeyear[1]{inwagen1995}).  But van Inwagen cannot deny that we at
least {\em believe} that there are chairs.  He admits that many of us
hold beliefs that we would express as ``there are two chairs in the
next room'' or ``I bought a new chair today''.  Indeed, he admits that
such beliefs are often {\em true}: ``when people say things in the
ordinary business of life by uttering sentences that start `There are
chairs\,\ldots ' or `There are stars\,\ldots ', they very often say
things that are literally true'' (\citeyear[102]{inwagen1995}).

Van Inwagen, when denying that we have beliefs about chairs, appears
to maintain that the beliefs that we (erroneously) take to be about
chairs are not, in fact, beliefs about chairs.  If a belief expressed
as ``that is a fine chair'' was actually about a chair, then it could
only be true if there was at least one chair (a fine one).  But van
Inwagen denies that there is at least one chair, but nonetheless says
that such a belief might be true.  He accordingly recognizes the need
to explain what our beliefs really are about.  If he explains what the
{\em content} of our beliefs is, then he will also be able to explain
{\em why} we hold such beliefs.

\ifstandalone
\end{spacing}
\bibliography{everything}
\bibliographystyle{ChicagoReedweb}
\fi
\end{document}
