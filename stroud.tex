\documentclass[11pt]{article}
\usepackage{standalone} \newif\ifstandlone \standalonetrue
\usepackage[left=1.75in, right=1.75in, top=1.25in, bottom=1.25in]{geometry}
\geometry{letterpaper}
\usepackage{graphicx}
%\usepackage{tipa}
%\usepackage{exaccent}
%\usepackage{txfonts}
%\usepackage{pxfonts}
\usepackage{enumitem}
%\usepackage{amssymb}
\usepackage{amsmath}
\usepackage{epstopdf}
\usepackage{setspace}
\usepackage{natbib}
\setcitestyle{aysep={}}
\synctex=1

\DeclareSymbolFont{symbolsC}{U}{txsyc}{m}{n}
\DeclareMathSymbol{\strictif}{\mathrel}{symbolsC}{74}
\DeclareMathSymbol{\boxright}{\mathrel}{symbolsC}{128}

\newenvironment{squote}{\begin{quote}\begin{singlespace}}{\end{singlespace}\end{quote}}

\title{Denying the Ordinary}
\author{Alexander A. Dunn}
\begin{document}
\ifstandalone
\maketitle
\begin{spacing}{1.5}
\fi

%\begin{squote}
%The philosophical quest must start somewhere. It needs a set of beliefs about what the world is like. Without some attitudes, perceptions, beliefs, or theories to start with, it would have nothing to reflect on.~\citep[16]{stroud2000a}
%\end{squote}

\begin{squote}
\textbf{quine}, v. (1) To deny resolutely the existence or importance of something real or significant~\citep{dennett2008}.
\end{squote}

\section{How to defy common sense}
Every so often a philosopher will claim that some aspect of what we take to be our world is somehow illusory, bogus, or simply nonexistent. This sort of denial will take various forms. One might claim that a certain phenomenon does indeed find expression in the world, but that it is somehow subjective; without humans to experience it, there would be no such phenomena. Or one might deny that some ordinary object of experience is actually non-existent. Peter van Inwagen claims that tables do not exist. He recognizes, of course, that people talk and have beliefs about (what they take to be) tables, so he must find a way to explain our beliefs in tables in terms of those things that he does believe to exist. Finally, one might claim that some things are non-existent, and that we in fact {\em don't} really talk and have beliefs about them. This is the sort of denial we make of the existence of ghosts. Unlike the philosopher who denies the existence of tables, we who deny the existence of ghosts don't have to explain people's beliefs in ghosts---people simply don't {\em have} coherent beliefs about them, because they are entirely unreal. Few philosophers make denials of this sort about ``real or significant'' things, because such things (tables, chairs, people, custard) are the putative objects of many beliefs, and to say that these beliefs are utterly incoherent is to cross the line into nonsense.\footnote{A line Peter Unger has never been shy about crossing, as we will see below.}

In this section I will briefly look at each of these types of denial and how the philosophers making them manage to explain the beliefs that people have about the objects of the denials.

\subsection{The relegation}
Among the various phenomena we observe in the world, it can be tempting to draw a distinction between those that we somehow imprint upon the world and those that are independent of any human experience. The former are `subjective' while the latter are `objective' or absolute:
\begin{squote}
Whatever is due only to us and to our own ways of responding to and interacting with the world does not reflect or correspond to anything present in the world as it is independently of us. The aim of an ``absolute'' conception, then, is to form a description of the way the world is, not just independently of its being believed to be that way, but independently, too, of all the ways in which it happens to present itself to us human beings from our particular standpoint within it\,\ldots\,[So we] form some conception of that independent reality and come to understand parts or aspects of our original conception of the world as not representing it as it is. If we see them as products or reflections of something peculiar to human experience or to the human perspective on the universe, we assign them a merely ``subjective'' or dependent status and eliminate them from our conception of the world as it is independently of us~\citep[30--31]{stroud2000a}.
\end{squote}

This is how philosophers often argue when denying the reality of colors (Stroud himself rejects these arguments).%
%
%\footnote{Stroud himself does not deny the objective reality of colors 
%\footnote{The distinction between objective and subjective reality relies on ``a conception of the world or reality as being a certain way independently of the responses of any sentient beings; [regarding a claim about the `objective' world], it would have been that way whether there had been such responses or not''~\citep[12]{stroud2000a}.}
%
\ People do see colors in the world (we have color vision), but in some sense we are {\em putting} the colors there. The red color of a tomato, on this view, obtains only in our perception of the tomato; there is nothing {\em in} the tomato that is the redness (other animals, for instance, may not see color). The belief most commonly motivating this type of view, according to Stroud, is a belief that ``the world as it is independently of us'' is simply the world described by an ideal physics: ``physical science can describe every aspect of the figure or shape and the number and motions of the bodies that make up the world. We have words for what we think of as the colours, odours, and tastes of those objects as well, but those words stand for nothing that exists in reality''~(\citeyear[8]{stroud2000a}).

Someone who makes this argument does not thereby deny that we perceive colors, or that we believe that things are colored. To claim that we {\em actually don't} think things are colored---that we don't actually believe that tomatoes are red---would obviously be false:
\begin{squote}
It is a Moorean fact that there are colours rightly so-called. Deny it, and the most credible explanation of your denial is that you are in the grip of some philosophical (or scientific) error~\citep[333]{lewis1997}.
\end{squote}
We certainly do believe that tomatoes (at least most of them) are red; this is what makes the denial of color interesting. If the philosopher claimed that colors aren't objectively real {\em and that we don't believe them to be}, we ought to wonder why the philosopher is even bothering to make the argument.%
%
%\footnote{}
%
\ It would be like the claim that ghosts don't exist; this is not controversial or interesting, because we don't believe that ghosts exist.

So the philosopher who is denying the objective reality of color must ``recognize the presence in the world of perceptions of and beliefs about the colours of things''~\citep[199]{stroud2000a}. The challenge then is to explain why we do have these perceptions and beliefs. The philosopher who believes that only the world of physics is objectively real must explain the color phenomena in the vocabulary of the physical sciences. (And before this can be attempted, the question arises as to what this vocabulary is: ``Physical science changes. Physicists do not just change their minds as they learn more and more about the world; the very conception of what is to be included in physics changes''~\citep[53]{stroud2000a}.%
%
\footnote{Cf. Lewis' espousal of materialism: ``an adequate theory must be consistent with the truth and completeness of some theory in much the style of present-day physics\,\ldots\,Some fear that `materialism' conveys a commitment that this ultimate physics must be a physics of matter alone: no fields, no radiation, no causally active spacetime. Not so! Let us proclaim our solidarity with forebears who, like us, wanted their philosophy to agree with ultimate physics. Let us not chide and disown them for their less advanced ideas about what ultimate physics might say''~(\citeyear[332n2]{lewis1997}).}
%
\ So the philosopher relegating colors---or anything else---to subjective reality must have a clear idea of what is left in objective reality.)%

\subsection{The paraphrase}
The second sort of denial goes further in denying any kind of reality at all to the subject of inquiry. The philosopher above denied that colors were `objectively real', but not that they were `subjectively real'; she did not deny that we do at least perceive colors. But there are some things that are taken by some philosophers to be neither objectively or subjectively real; they simply do not exist. Peter van Inwagen, for example, denies the existence of tables, chairs, apples, and every other (inanimate) `composite' object.%
%
\footnote{Just what is meant by a composite objects will be examined later.}
%
\ He takes pains to make clear that his denial of these things is by no means limited to their objective reality, but to their subjective reality as well (if such a distinction there be):
\begin{squote}
I want to do what I can to disown a certain apparently almost irresistible characterization of my view, or of that part of my view that pertains to inanimate objects. Many philosophers, in conversation and correspondence, have insisted, despite repeated protests on my part, on describing my position in words like these: ``Van Inwagen says that tables are not real''; ``\ldots\,not true objects''; ``\ldots\,not actually {\em things}''; ``\ldots\,not substances''; ``\ldots\,not unified wholes''; ``\ldots\,nothing more than collections of particles.'' These are words that darken counsel. They are, in fact, perfectly meaningless. My position vis-\`{a}-vis tables and other inanimate objects is simply that there {\em are} none~(\citeyear[99]{inwagen1995}).
\end{squote}
Van Inwagen asserts, quite seriously, that ``the number of trees is 1 or more and that the number of apples is 0''~(\citeyear[711]{inwagen1993b}). This is a somewhat more bold claim than that of the philosopher skeptical of color. And just as she could not claim that we don't believe in colors, van Inwagen cannot deny that we {\em believe} there to be apples. We (apparently) do see, think about, and talk about apples. We eat them. So van Inwagen has a rather daunting task: he must explain our beliefs about tables in terms of whatever it is that he does believe to exist. As if this weren't hard enough, van Inwagen refuses to countenance anything but living organisms and the basic particles that make up the physical universe.%
%
\footnote{Van Inwagen assumes, without defense, ``that matter is ultimately particulate\,\ldots\,every material thing is composed of things that have no proper parts: `elementary particles' or `mereological atoms' or `metaphysical simples'\,''~(\citeyear[5]{inwagen1995}). Ted Sider takes him to task for this assumption~(\citeyear{sider1993}), claiming that the possibility of `gunk'---the possibility that the matter of the world is not fundamentally particulate but infinitely divisible---falsifies van Inwagen's thesis. I think it may be possible for van Inwagen to adapt to a gunky world (see Section~\ref{brute}, note~\ref{gunk}), but I think van Inwagen's thesis is false either way.}
%

Van Inwagen recognizes the necessity of giving such an explanation. Indeed, he admits that ``when people say things in the ordinary business of life by uttering sentences that start `There are chairs\,\ldots ' or `There are stars\,\ldots ', they very often say things that are literally true''~\cite[102]{inwagen1995}. Van Inwagen himself has argued that ``there is an $x$'', if true, entails the existence of at least one $x$~(\citeyear[237--241]{inwagen1998}). What van Inwagen has to say (I don't see what else he {\em could} say) is that {\em we}, when saying things like ``Some chairs are heavier than some tables'', {\em actually mean} this instead: ``There are $x$s [simple particles] that are arranged chairwise and there are $y$ that are arranged tablewise and the $x$s are heavier than the $y$s''~(\citeyear[109]{inwagen1995}). This seems to be a highly dubious hypothesis about the speech practices of other people, and van Inwagen expects resistance. He offers a quote of Kripke's that offers a representative sample of what form the resistance will take:
\begin{squote}
The philosopher advocates a view apparently in patent contradiction to common sense. Rather than repudiating common sense, he asserts that the conflict comes from a philosophical misinterpretation of common language---sometimes he adds that the misinterpretation is encouraged  by the `superficial form' of ordinary speech. He offers his own analysis of the relevant common assertions, one that shows that they do not really say what they seem to say\,\ldots\,I think such philosophical claims are almost invariably suspect. What the claimant calls a `misleading philosophical construal' of the ordinary statement is probably the natural and correct understanding. The real misconstrual comes when the claimant continues, ``All the ordinary man really means is\,\ldots '' and gives a sophisticated analysis compatible with his own philosophy~(\citeyear[65]{kripke1982}.
\end{squote}

I do not think van Inwagen's defense is ultimately successful (see Section~\ref{nihilism}), but it is precisely the sort of defense required when denying the existence of ordinary things like tables and chairs. He cannot claim, without courting absurdity, that we {\em don't} believe there to be tables and chairs in the world. We do believe so. Of course we might be wrong about the existence of tables and chairs; ``from the fact that we believe a certain thing it does not follow that it is true''~\citep[21]{stroud2000a}. Nevertheless it is true that we believe there to be such things; van Inwagen needs to explain the source of this belief. If there are no tables and chairs, then we must be able to understand the beliefs that we thought to be about the furniture to be about something else instead, and we need a story about what that something else is: ``you cannot hope to explain something unless you grant that there is such a thing and you have at least some idea of what it is''~\citep[97]{stroud2000a}.

\subsection{The resolute denial}
This is the sort of denial that ghosts make.

\ifstandalone
\end{spacing}
\bibliography{everything}
\bibliographystyle{ChicagoReedweb}
\fi
\end{document}
