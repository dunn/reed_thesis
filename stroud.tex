\documentclass[11pt]{article}
\usepackage{standalone} \newif\ifstandlone \standalonetrue
\usepackage[left=1.75in, right=1.75in, top=1.25in, bottom=1.25in]{geometry}
\geometry{letterpaper}
\usepackage{graphicx}
%\usepackage{tipa}
%\usepackage{exaccent}
%\usepackage{txfonts}
%\usepackage{pxfonts}
\usepackage{enumitem}
%\usepackage{amssymb}
\usepackage{amsmath}
\usepackage{epstopdf}
\usepackage{setspace}
\usepackage{natbib}
\setcitestyle{aysep={}}
\synctex=1

\DeclareSymbolFont{symbolsC}{U}{txsyc}{m}{n}
\DeclareMathSymbol{\strictif}{\mathrel}{symbolsC}{74}
\DeclareMathSymbol{\boxright}{\mathrel}{symbolsC}{128}

\newenvironment{squote}{\begin{quote}\begin{singlespace}}{\end{singlespace}\end{quote}}

\newenvironment{inq}{\vspace{0pt}%
	\begin{list}{}%
	{\setlength\labelwidth{0pt}%
	\setlength\leftmargin{2.5\oddsidemargin}%
	\setlength\rightmargin{\leftmargin}}
	\begin{spacing}{1}
	\item[]%
	}{
	\end{spacing}
	\end{list}
	\vspace{10pt}
	%\noindent%
	}

\title{Denying the Ordinary}
\author{Alexander A. Dunn}
\begin{document}
\ifstandalone
\maketitle
\begin{spacing}{1.5}
\fi

%\begin{inq}
%The philosophical quest must start somewhere. It needs a set of beliefs about what the world is like. Without some attitudes, perceptions, beliefs, or theories to start with, it would have nothing to reflect on.~\citep[16]{stroud2000a}
%\end{inq}

%	\begin{inq}\textbf{quine}, v. To deny resolutely the existence or importance of something real or significant.
%	\footnote{This is of course from the {\em Philosophical Lexicon}~\citep{dennett2008}.}%}
%	\end{inq}%

%\section{How to defy common sense}
%\label{denials}
\noindent Every so often a philosopher will claim that some aspect of what we take to be our world is somehow illusory, bogus, or simply nonexistent. This sort of denial will take various forms. One might claim that a certain phenomenon does indeed find expression in the world, but that it is somehow subjective; without humans to experience it, there would be no such phenomena. Or one might deny that some ordinary object of experience is actually non-existent. Peter van Inwagen claims that tables do not exist. He recognizes, of course, that people talk and have beliefs about (what they take to be) tables, so he must find a way to explain our beliefs in tables in terms of those things that he does believe to exist. Finally, one might claim that some things are non-existent, and that we in fact {\em don't} really talk and have beliefs about them. This is the sort of denial we make of the existence of ghosts. Unlike the philosopher who denies the existence of tables, we who deny the existence of ghosts don't have to explain people's beliefs in ghosts---people simply don't {\em have} coherent beliefs about them, because they are entirely unreal. Few philosophers make denials of this sort about ``real or significant'' things, because such things (tables, chairs, people, custard) are the putative objects of many beliefs, and to say that these beliefs are utterly incoherent is to cross the line into nonsense.%\footnote{A line Peter Unger has never been shy about crossing, as we will see below.}

In this section I will briefly look at each of these types of denial and how the philosophers making them manage to explain the beliefs that people have about the objects of the denials.

\section{The relegation}
\label{relegate}
Among the various phenomena we observe in the world, it can be tempting to draw a distinction between those that we somehow imprint upon the world and those that are independent of any human experience. The former are `subjective' while the latter are `objective' or absolute:
\begin{squote}
Whatever is due only to us and to our own ways of responding to and interacting with the world does not reflect or correspond to anything present in the world as it is independently of us. The aim of an ``absolute'' conception, then, is to form a description of the way the world is, not just independently of its being believed to be that way, but independently, too, of all the ways in which it happens to present itself to us human beings from our particular standpoint within it\,\ldots\,[So we] form some conception of that independent reality and come to understand parts or aspects of our original conception of the world as not representing it as it is. If we see them as products or reflections of something peculiar to human experience or to the human perspective on the universe, we assign them a merely ``subjective'' or dependent status and eliminate them from our conception of the world as it is independently of us~\citep[30--31]{stroud2000a}.
\end{squote}

This is how philosophers often argue when denying the reality of colors (Stroud himself rejects these arguments).%
%
%\footnote{Stroud himself does not deny the objective reality of colors 
%\footnote{The distinction between objective and subjective reality relies on ``a conception of the world or reality as being a certain way independently of the responses of any sentient beings; [regarding a claim about the `objective' world], it would have been that way whether there had been such responses or not''~\citep[12]{stroud2000a}.}
%
\ People do see colors in the world (we have color vision), but in some sense we are {\em putting} the colors there. The red color of a tomato, on this view, obtains only in our perception of the tomato; there is nothing {\em in} the tomato that is the redness (other animals, for instance, may not see color). The belief most commonly motivating this type of view, according to Stroud, is a belief that ``the world as it is independently of us'' is simply the world described by an ideal physics: ``physical science can describe every aspect of the figure or shape and the number and motions of the bodies that make up the world. We have words for what we think of as the colours, odours, and tastes of those objects as well, but those words stand for nothing that exists in reality''~(\citeyear[8]{stroud2000a}).

Someone who makes this argument does not thereby deny that we perceive colors, or that we believe that things are colored. To claim that we {\em actually don't} think things are colored---that we don't actually believe that tomatoes are red---would obviously be false.%
%\begin{squote}
%It is a Moorean fact that there are colours rightly so-called. Deny it, and the most credible explanation of your denial is that you are in the grip of some philosophical (or scientific) error~\citep[333]{lewis1997}.
%\end{squote}
\ We certainly do believe that tomatoes (at least most of them) are red; this is what makes the denial of color interesting. If the philosopher claimed that colors aren't objectively real {\em and that we don't believe them to be}, we ought to wonder why the philosopher is even bothering to make the argument.%
%
%\footnote{}
%
\ It would be like the claim that ghosts don't exist; this is not controversial or interesting, because we don't believe that ghosts exist.

So the philosopher who is denying the objective reality of color must ``recognize the presence in the world of perceptions of and beliefs about the colours of things''~\citep[199]{stroud2000a}. The challenge then is to explain why we do have these perceptions and beliefs. The philosopher who believes that only the world of physics is objectively real must explain the color phenomena in the vocabulary of the physical sciences. (And before this can be attempted, the question arises as to what this vocabulary is: ``Physical science changes. Physicists do not just change their minds as they learn more and more about the world; the very conception of what is to be included in physics changes''~\citep[53]{stroud2000a}.%
%
%\footnote{Cf. Lewis' espousal of materialism: ``an adequate theory must be consistent with the truth and completeness of some theory in much the style of present-day physics\,\ldots\,Some fear that `materialism' conveys a commitment that this ultimate physics must be a physics of matter alone: no fields, no radiation, no causally active spacetime. Not so! Let us proclaim our solidarity with forebears who, like us, wanted their philosophy to agree with ultimate physics. Let us not chide and disown them for their less advanced ideas about what ultimate physics might say''~(\citeyear[332n2]{lewis1997}).}
%
\ So the philosopher relegating colors---or anything else---to subjective reality must have a clear idea of what is left in objective reality.)%

\section{The paraphrase}
\label{paraphrase}
The second sort of denial goes further in denying any kind of reality at all to the subject of inquiry. The philosopher above denied that colors were `objectively real', but not that they were `subjectively real'; she did not deny that we do at least perceive colors. But there are some things that are taken by some philosophers to be neither objectively or subjectively real; they simply do not exist. A philosopher in a cynical mood might deny that love exists. Depending on how recently she was jilted, however, she might not deny that an utterance of ``there is love in the world'' is true. What she would do is paraphrase it as ``there are people in the world who are in symmetrical or asymmetrical relations with other people that can be described as `loving relationships'\,''. If it seems plausible that the original speaker meant something along these lines, then our philosopher has performed `the paraphrase' on love.%
\ (The philosopher may or may not go on to claim that the loving relationship would not exist without humans to instantiate it.)
%\ (The original speaker might reject this paraphrase, of course. She might insist that $\exists x(x=Love)$. In this situation we might perform a resolute denial of the sort described in section~\ref{resolute} below and say that there is no existing object that is love.)

Peter van Inwagen attempts to perform the paraphrase on tables, chairs, apples, and every other (inanimate) `composite' object.%
%
\footnote{Just what is meant by a composite objects will be examined later.}
%
\ He takes pains to make clear that his denial of these things is by no means limited to their objective reality, but to their subjective reality as well (if such a distinction there be):
\begin{squote}
I want to do what I can to disown a certain apparently almost irresistible characterization of my view, or of that part of my view that pertains to inanimate objects. Many philosophers, in conversation and correspondence, have insisted, despite repeated protests on my part, on describing my position in words like these: ``Van Inwagen says that tables are not real''; ``\ldots\,not true objects''; ``\ldots\,not actually {\em things}''; ``\ldots\,not substances''; ``\ldots\,not unified wholes''; ``\ldots\,nothing more than collections of particles.'' These are words that darken counsel. They are, in fact, perfectly meaningless. My position vis-\`{a}-vis tables and other inanimate objects is simply that there {\em are} none~(\citeyear[99]{inwagen1995}).
\end{squote}
Van Inwagen asserts, quite seriously, that ``the number of trees is 1 or more and that the number of apples is 0''~(\citeyear[711]{inwagen1993b}). This is a somewhat more bold claim than that of the philosopher skeptical of color. And just as she could not claim that we don't believe in colors, van Inwagen cannot deny that we {\em believe} there to be apples. We (apparently) do see, think about, and talk about apples. We eat them. So van Inwagen has a rather daunting task: he must explain our beliefs about tables in terms of whatever it is that he does believe to exist. (As if this weren't hard enough, van Inwagen refuses to countenance anything but living organisms and the basic particles that make up the physical universe.)%
%
\footnote{Van Inwagen assumes, without defense, ``that matter is ultimately particulate\,\ldots\,every material thing is composed of things that have no proper parts: `elementary particles' or `mereological atoms' or `metaphysical simples'\,''~(\citeyear[5]{inwagen1995}). Ted Sider takes him to task for this assumption~(\citeyear{sider1993}), claiming that the possibility of `gunk'---the possibility that the matter of the world is not fundamentally particulate but infinitely divisible---falsifies van Inwagen's thesis. I think it may be possible for van Inwagen to adapt to a gunky world (see Section~\ref{brute}, note~\ref{gunk}), but I think van Inwagen's thesis is false either way.}
%

Van Inwagen recognizes the necessity of giving such an explanation. Indeed, he admits that ``when people say things in the ordinary business of life by uttering sentences that start `There are chairs\,\ldots ' or `There are stars\,\ldots ', they very often say things that are literally true''~(\citeyear[102]{inwagen1995}). Van Inwagen himself has argued that ``there is an $x$'', if true, entails the existence of at least one $x$~(\citeyear[237--241]{inwagen1998}). What van Inwagen has to say (I don't see what else he {\em could} say) is that {\em we}, when saying things like ``Some chairs are heavier than some tables'', {\em actually mean} this instead: ``There are $x$s [simple particles] that are arranged chairwise and there are $y$s that are arranged tablewise and the $x$s are heavier than the $y$s''~(\citeyear[109]{inwagen1995}). This seems to be a highly dubious hypothesis about the speech practices of other people, and van Inwagen expects resistance. He himself mentions a quote of Kripke's that indicates what form the resistance might take:
\begin{squote}
The philosopher advocates a view apparently in patent contradiction to common sense. Rather than repudiating common sense, he asserts that the conflict comes from a philosophical misinterpretation of common language---sometimes he adds that the misinterpretation is encouraged  by the `superficial form' of ordinary speech. He offers his own analysis of the relevant common assertions, one that shows that they do not really say what they seem to say\,\ldots\,I think such philosophical claims are almost invariably suspect. What the claimant calls a `misleading philosophical construal' of the ordinary statement is probably the natural and correct understanding. The real misconstrual comes when the claimant continues, ``All the ordinary man really means is\,\ldots '' and gives a sophisticated analysis compatible with his own philosophy~(\citeyear[65]{kripke1982}).
\end{squote}

I do not think van Inwagen's defense is ultimately successful (see Section~\ref{pigletwise}), but it is precisely the sort of defense required when denying the existence of ordinary things like tables and chairs. He cannot claim, without courting absurdity, that we {\em don't} believe there to be tables and chairs in the world. We do believe so. Of course we might be wrong about the existence of tables and chairs; ``from the fact that we believe a certain thing it does not follow that it is true''~\citep[21]{stroud2000a}. Nevertheless it is true that we believe there to be such things; van Inwagen needs to explain the source of this belief. If there are no tables and chairs, then we must be able to understand the beliefs that we thought to be about the furniture to be about something else instead, and we need a story about what that something else is.
% ``you cannot hope to explain something unless you grant that there is such a thing and you have at least some idea of what it is''~\citep[97]{stroud2000a}.
\ As I will explain below, I don't think van Inwagen offers a good explanation of what our beliefs are about, if not about tables and chairs.

\section{The resolute denial}
\label{resolute}
The third kind of denial is that which I likened to a denial of ghosts. When I deny the existence of ghosts, I don't feel the need to explain what people are really talking and thinking about when they appear to be talking and thinking about ghosts. I can simply say that there is {\em nothing} in particular that they are talking about. There are no ghosts, and never have been.

Of course I cannot deny that some people believe that ghosts exist. Some people do believe this. But my resolute denial of the existence of ghosts is compatible with my giving an explanation of why someone thinks there are ghosts. Someone, for example, might tell me that they saw a ghost on the landing. I walk out and see a light from a high window flickering strangely on the wall. (If I squint, the pattern of the light looks almost humanoid.) So I tell her that what she thought was a ghost was really just a curious play of the light. I am not here conceding that she had a belief {\em about} a ghost. On the contrary, I have tried to show her that her belief was about anything but a ghost. There was in fact no ghost that she could have formed a belief about; ``if we show that what a frightened person saw in the attic on a particular occasion was a rippling reflection of the moon through the window, we implicitly deny the presence of a ghost in giving the explanation of the person's belief and fear''~\citep[76]{stroud2000a}. We do not deny that people have beliefs about what they take to be ghosts; what we deny is that they are correct in taking their beliefs to be about ghosts.

What {\em are} their beliefs about? It might be said that, in Stroud's example, the frightened person is frightened of the reflection of the moon. This is similar to how we might talk about a child's night terrors: ``she was afraid of the chair in her room (she thought it was a monster).'' Looking at things this way, the person's beliefs is {\em about} something, but it is something very different from what they took it to be about. I think, however, that this is not a fully accurate characterization of the object of the person's belief. It may be true that the reflection of the moon {\em caused} her belief (which caused her fright), but it would be at least a little misleading to say that Mrs --------- is frightened of the reflection of the moon. This is misleading because, if we succeed in showing her that there {\em is} no ghost in the attic, only a reflection of the moon, she will not still be afraid. When she understands that her belief in the ghost was mistaken, she will see that there is nothing to be afraid of. The belief that caused her fright, that there was a ghost in the attic, was in fact a belief {\em about} nothing at all.

\subsection{Unger's nihilism}
\label{unger}
Just as resolutely as we denied the existence of ghosts, so Peter Unger has denied the existence of such things as ``tables and chairs and spears\,\ldots\,swizzle sticks and sousaphones\,\ldots\,stones and rocks and twigs, and also tumbleweeds and fingernails''~(\citeyear[117]{unger1979}). He does not consider them merely `subjectively real' as opposed to objectively so---like van Inwagen, he claims that they simply do not exist. He comes to this conclusion from a different direction, however. As we will see, van Inwagen's denial of the existence of `ordinary things' is a consequence of his theory of composition (under what conditions some things compose another thing). Unger, on the other hand, draws his conclusion from an application of the sorites paradox:
\begin{squote}
Consider a stone, consisting of a certain finite number of atoms. If we or some physical process should remove one atom, without replacement, then there is left that number minus one, presumably constituting a stone still\,\ldots\,after another atom is removed, there is that original number minus two; so far, so good. But after that certain number has been removed, in similar stepwise fashion, there are no atoms at all in the situation, while we must still be supposing that there is a stone there. But as we have already agreed, if there is a stone present, then there must
be some atoms\,\ldots\,I suggest that any adequate response to this contradiction must include\,\ldots\,the denial of the existence of even a single stone.~\citep[121--122]{unger1979}
\end{squote}

Having made this denial, Unger must either explain how our beliefs about stones should be understood (van Inwagen has his paraphrasing strategy) or he must deny that we really {\em do} have any beliefs about stones. It appears that he selects the latter option: Unger seems to claim that, like the person who thought they had a belief about a ghost, we are wrong to think that we have any coherent beliefs about stones or any other ordinary things. Unger says that, like `ghost', our ``terms for ordinary things are incoherent [and] cannot apply to anything real''~\citep[147]{unger1979}. A consequence of this is that our language and thought concerning all such things is directed toward {\em nothing at all}: ``it may well be that I have never {\em thought of} any stones at all, or tables, or even human hands. If that is so, then it would seem that {\em a fortiori} I do not {\em know} anything {\em about these entities}, however commonly I might otherwise suppose''~(\citeyear[458]{unger1980a}).

This all seems very strange. Concerning ghosts, ``it is difficult even to find a fully coherent belief that might be exposed as false; we discover, at best, obscurity or perhaps confusion\,\ldots\,do we really understand what sort of thing a ghost is supposed to be''~\citep[76]{stroud2000a}? If someone tries to tell me about the ghost that visited him the previous night, it does not seem unjust to say that he really doesn't know what he is talking about. But can this be extended to some of the most common objects of experience?

When we denied the existence of ghosts, we denied also others' beliefs in them. We did not, however, deny that people have beliefs which they take to be about ghosts. But we were able to show that these beliefs were not {\em about} ghosts; in most cases they were about nothing at all. Likewise, Unger cannot deny that we have beliefs that we take to be about tables, chairs, and all the other things that he denies exist. If our beliefs about tables and chairs are really beliefs about nothing at all, there are two questions that must be answered: first, what causes us to form these beliefs? and second, if the utterances containing these empty terms are about nothing at all, how do we manage to communicate so effectively using them?

\subsubsection{Causes of belief}
\label{unger-cause}
People who believe in ghosts probably do so because they have unreflectively embraced the superstitions of their culture. They may initially come to believe that ghosts exist on the testimony of other people---older siblings, perhaps---or by reading too many ghost stories. Having formed the belief that there are ghosts, they may go on to attribute strange phenomena, such as a curious play of the light, to the presence of ghosts. People who know that ghosts do not exist would be more likely to recognize these phenomena as tricks of the light. If we were to see something that was definitely {\em not} a trick of the light, we would sooner attribute it to an hallucination than countenance the possibility of ghosts:
%
\begin{squote}
She rose, not as if she had heard me, but with an indescribable grand melancholy of indifference and detachment, and, within a dozen feet of me, stood there as my vile predecessor. Dishonored and tragic, she was all before me; but even as I fixed, and, for memory, secured it, the awful image passed away~\citep[58]{james1991}
\end{squote}
%
Most of us would still, even when presented with such a vision, {\em refuse to believe in ghosts}. This is because we know that the probability of their being such spirits is far less than the probability of us experiencing cracks in our sanity. Shaking my confidence would require, for example, my friend and I both seeing the {\em same} apparent ghost at the same time---and knowing that we were each experiencing the same vision. (Even then, we would want further confirmed sightings to convince us that we weren't, in fact, crazy.)

If this is an accurate characterization of our beliefs concerning ghosts, it is a very different characterization from one we might give of how we learn about and come to believe in chairs. Chairs are not something that children learn about from stories. A child learns what a `chair' is as an answer to the question, ``What is {\em that?}\,'' Let us suppose that the child is pointing at a chair in the center of a well-lit room containing no other furniture. The chair is clearly visible. If someone were to believe they were pointing at a ghost in a similar situation, we could safely assume that they would be experiencing a hallucination. Hallucinations are not shared experiences; if one person is hallucinating a cat, nobody else can see that cat, not even if they were {\em also} hallucinating a cat. Such a ghost sightings, therefore, would necessarily be experienced by a single person. A chair, on the other hand, can be sighted by multiple people simultaneously. This is what allows a parent to answer the child's question: ``That's a chair.'' If we supposed that the parent was making demonstrative reference to a figment of their own imagination, we should have to find it miraculous that the child understands what the parent is talking about. As we will see, Unger's thesis cannot account for the phenomenon of communication.

\subsubsection{Communication and incoherence}
\label{unger-comm}
boink

\ifstandalone
\end{spacing}
\bibliography{everything}
\bibliographystyle{ChicagoReedweb}
\fi
\end{document}
