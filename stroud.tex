\documentclass[11pt]{standalone}
\usepackage{standalone} \newif\ifstandlone \standalonetrue
\usepackage[left=1.75in, right=1.75in, top=1.25in, bottom=1.25in]{geometry}
\geometry{letterpaper}
\usepackage{graphicx}
%\usepackage{tipa}
%\usepackage{exaccent}
%\usepackage{txfonts}
%\usepackage{pxfonts}
\usepackage{enumitem}
%\usepackage{amssymb}
\usepackage{amsmath}
\usepackage{epstopdf}
\usepackage{setspace}
\usepackage{natbib}
\setcitestyle{aysep={}}
\usepackage%[colorlinks=true, citecolor=blue, linkcolor=black]%
{hyperref}

\synctex=1

\DeclareSymbolFont{symbolsC}{U}{txsyc}{m}{n}
\DeclareMathSymbol{\strictif}{\mathrel}{symbolsC}{74}
\DeclareMathSymbol{\boxright}{\mathrel}{symbolsC}{128}

\newcommand{\stager}[4]%
{%
	\begin{spacing}{1}%
	\vspace{0pt}
		\begin{description}[style=nextline, noitemsep, parsep=0pt, topsep=0pt, leftmargin=15mm, itemindent=-10mm, font=\mdseries]
			\item[\textsc{#1} \emph{#2}] #3
			\item[]%
			\begin{flushright}#4\end{flushright}
		\end{description}%
	\end{spacing}%
}

\newcommand{\stage}[3]%
{%
	\begin{spacing}{1}%
	\vspace{0pt}
		\begin{description}[style=nextline, parsep=0pt, leftmargin=15mm, itemindent=-10mm, font=\mdseries]
			\item[\textsc{#1} \emph{#2}] #3
		\end{description}%
	\end{spacing}%
}

\newenvironment{squote}{%
	\begin{spacing}{1}
	\begin{list}{}{%
	\setlength{\labelwidth}{0pt}%
	\rightmargin\leftmargin%
	}
	%\begin{singlespace}%
	\item\relax
	}{%
	%\end{singlespace}%
	\end{list}%
	\end{spacing}
	}

\newenvironment{inq}{\vspace{0pt}%
	\begin{list}{}%
	{\setlength\labelwidth{0pt}%
	\setlength\leftmargin{2.5\oddsidemargin}%
	\setlength\rightmargin{\leftmargin}}
	\begin{spacing}{1}
	\item[]%
	}{
	\end{spacing}
	\end{list}
	\vspace{10pt}
	%\noindent%
	}

\title{Denying the Ordinary}
\author{Alexander A. Dunn}
\begin{document}
\ifstandalone
\maketitle
\tableofcontents
\begin{spacing}{1.5}
\fi



%\begin{inq}
%The philosophical quest must start somewhere. It needs a set of beliefs about what the world is like. Without some attitudes, perceptions, beliefs, or theories to start with, it would have nothing to reflect on.~\citep[16]{stroud2000a}
%\end{inq}

%	\begin{inq}\textbf{quine}, v. To deny resolutely the existence or importance of something real or significant.
%	\footnote{This is of course from the {\em Philosophical Lexicon}~\citep{dennett2008}.}%}
%	\end{inq}%

\section{Introduction}
\label{intro-deny}
\noindent Every so often a philosopher will claim that some aspect of
what we take to be our world is somehow illusory, bogus, or simply
nonexistent.  This sort of denial will take various forms.  One might
claim that a certain phenomenon does indeed find expression in the
world, but that it is somehow subjective; without humans to experience
it, there would be no such phenomena.  Or one might claim that some
ordinary object of experience is actually non-existent.  Peter van
Inwagen claims that tables do not exist.  He recognizes, of course,
that most of us believe that there are tables and think that we do
talk and have beliefs about them; he must therefore explain why we
have these beliefs and what they are really about (if they are about
anything at all).  Of course, when giving this explanation he must
make reference only to those things that he \emph{does} take to
exist.  We will look at this strategy more closely in
sections~\ref{paraphrase} and~\ref{inwagen}.  Finally, one might claim
that some things are non-existent, and that we in fact {\em don't}
really talk and have beliefs about them.  This is the sort of denial I
would make regarding the existence of ghosts.  Unlike the philosopher
who denies the existence of tables, we who deny the existence of
ghosts don't have to explain people's beliefs in ghosts---people
simply don't {\em have} coherent beliefs about them, because they are
entirely unreal.  Few philosophers make denials of this sort about
``real or significant'' things, because the beliefs we have about such
things (tables, chairs, people, custard) are deeply integrated into
our daily lives.  Nonetheless Peter Unger does attempt this sort of
denial; we address his arguments in sections~\ref{resolute} and~\ref{unger}.

\section{Kinds of denials}
In this section I will briefly look at these three types of argument
and discuss how the philosophers making them manage to explain the
beliefs that people have about the objects of the denials.

\subsection{The relegation}
\label{relegate}
Among the various phenomena we observe in the world, it can be
tempting to draw a distinction between those that we somehow imprint
upon the world and those that are independent of any human experience.
The former are `subjective' while the latter are `objective' or
absolute:
\begin{squote}
Whatever is due only to us and to our own ways of responding to and
interacting with the world does not reflect or correspond to anything
present in the world as it is independently of us.  The aim of an
``absolute'' conception, then, is to form a description of the way the
world is, not just independently of its being believed to be that way,
but independently, too, of all the ways in which it happens to present
itself to us human beings from our particular standpoint within
it\,\ldots\,[So we] form some conception of that independent reality
and come to understand parts or aspects of our original conception of
the world as not representing it as it is.  If we see them as products
or reflections of something peculiar to human experience or to the
human perspective on the universe, we assign them a merely
``subjective'' or dependent status and eliminate them from our
conception of the world as it is independently of
us~\citep[30--31]{stroud2000a}.
\end{squote}

A philosopher who adheres to this distinction might claim that our
conception of the world as colored does not represent the world as it
is independently of us.  Colors, she would claim, are not objectively
real.  She allows that they are subjectively real, of course.  People
do see colors; we have color vision while some species do not.
Because of our color vision, we come to believe that the things we see
are colored.  A philosopher who is skeptical of the objective reality
of color ``cannot deny that we perceive many different colours or that
we believe physical objects to be coloured''~\citep[145]{stroud2000a}.
What the skeptic has to claim is something to the effect that, while
we see things {\em as} colored, things are not {\em themselves}
colored.  The red color of a tomato, on this view, obtains only in our
perception of the tomato; there is nothing {\em in} the tomato that
is the redness (other species may not see the redness when they see
the tomato).
%
% The belief most commonly motivating this type of view, according to
% Stroud, is a belief that ``the world as it is independently of us''
% is simply the world described by an ideal physics: ``physical
% science can describe every aspect of the figure or shape and the
% number and motions of the bodies that make up the world.  We have
% words for what we think of as the colours, odours, and tastes of
% those objects as well, but those words stand for nothing that exists
% in reality''~(\citeyear[8]{stroud2000a}).

Someone who makes this argument does not, therefore, deny that we
perceive colors, or that we believe that things are colored.  To claim
that we {\em actually don't} think things are colored---that we don't
actually believe that tomatoes are red---would obviously be false.  We
certainly do believe that tomatoes (at least most of them) are red;
this is what makes the denial of color interesting.  If the
philosopher claimed that colors aren't objectively real {\em and that
  we don't believe them to be}, we ought to wonder why the philosopher
is even bothering to make the argument.
%
%\footnote{}
%
\ It would be like the claim that ghosts don't exist; this is not
controversial or interesting, because we don't believe that ghosts
exist.

So the philosopher who is denying the objective reality of color must
``recognize the presence in the world of perceptions of and beliefs
about the colours of things''~\citep[199]{stroud2000a}.  The challenge
then is to explain why we do have these perceptions and beliefs.  For
example, a philosopher who believes that only the world of physics is
objectively real must explain the color phenomena in the vocabulary of
the physical sciences.  (And before this can be attempted, the
question arises as to what this vocabulary is: ``Physical science
changes.  Physicists do not just change their minds as they learn more
and more about the world; the very conception of what is to be
included in physics changes''~\citep[53]{stroud2000a}.  So the
philosopher relegating colors---or anything else---to subjective
reality must have a clear idea of what is left in objective reality.)

\subsection{The paraphrase}
\label{paraphrase}
The second sort of denial goes further in denying any kind of reality
at all to the subject of inquiry.  The philosopher above denied that
colors were `objectively real', but not that they were `subjectively
real'; she did not deny that we do at least perceive colors.  But
there are some things that are taken by some philosophers to be
neither objectively or subjectively real; they simply do not exist.
% A philosopher in a cynical mood might deny that love exists.
% Depending on how recently she was jilted, however, she might not
% deny that an utterance of ``there is love in the world'' is true.
% What she would do is paraphrase it as ``there are people in the
% world who are in symmetrical or asymmetrical relations with other
% people that can be described as `loving relationships'\,'' (assuming
% that the cynical philosopher doesn't deny the existence of people as
% well).  If it can be determined that the original speaker meant
% something along these lines, then our philosopher has performed `the
% paraphrase' on love.  (The philosopher may or may not go on to claim
% that the loving relationship is `subjective'; that it would not
% exist without humans to instantiate it.)  \ (The original speaker
% might reject this paraphrase, of course.  She might insist that
% $\exists x(x=Love)$.  In this situation we might perform a resolute
% denial of the sort described in section~\ref{resolute} below and say
% that there is no existing object that is love.)

\ Peter van Inwagen denies the existence of tables, chairs, apples,
and all other inanimate composite objects.
%
\footnote{The notion of `composite' will be discussed below in
  section~\ref{scq}.}
%
\ He takes pains to make clear that his denial of these things is not
a relegation of tables and chairs to `subjective reality'.  He wants
to claim that such things do not exist in any way, subjective or
objective:
\begin{squote}
I want to do what I can to disown a certain apparently almost
irresistible characterization of my view, or of that part of my view
that pertains to inanimate objects.  Many philosophers, in
conversation and correspondence, have insisted, despite repeated
protests on my part, on describing my position in words like these:
``Van Inwagen says that tables are not real''; ``\ldots\,not true
objects''; ``\ldots\,not actually {\em things}''; ``\ldots\,not
substances''; ``\ldots\,not unified wholes''; ``\ldots\,nothing more
than collections of particles.''  These are words that darken counsel.
They are, in fact, perfectly meaningless.  My position vis-\`{a}-vis
tables and other inanimate objects is simply that there {\em are}
none~(\citeyear[99]{inwagen1995}).
\end{squote}
Van Inwagen asserts, quite seriously, that ``there are no tables or
chairs or any other visible objects except living
organisms''~(\citeyear[1]{inwagen1995}).  This is a somewhat more bold
claim than that of the philosopher skeptical of color.  She at least
granted that we do see colors, even if we don't actually see things
that are (objectively) colored.  If, as van Inwagen claims, the only
{\em visible} objects are living organisms, then we certainly can't
{\em see} tables at all.  But just as our color skeptic could not
claim that we don't believe in colors, van Inwagen cannot deny that we
at least {\em believe} there to be apples.  We have many beliefs about
what we take to be apples.  We believe that they grow on trees, and go
well with many types of cheese (which we also believe to exist).  Van
Inwagen has, therefore, a rather daunting task: he must explain our
beliefs about apples (and about cheese, and tables, and chairs, and
about everything else he denies) in terms of whatever it is that he
does claim to exist (living organisms and the basic particles that
make up the physical universe).

Van Inwagen does not attempt to deny that we have beliefs about what
we take to be apples, cheeses, \&c.  Indeed, he admits that ``when
people say things in the ordinary business of life by uttering
sentences that start `There are chairs\,\ldots ' or `There are
stars\,\ldots ', they very often say things that are literally
true''~(\citeyear[102]{inwagen1995}).  By conceding this, he distances
himself from a skeptical philosopher (see section~\ref{unger} below)
whose denial of apples et.\ al.\ is on par with a denial of ghosts.
When someone has a belief about what they took to be a ghost, we their
belief is not about whatever actually caused their fright; it is most
charitable to say that their belief is really about nothing at all (I
explain why in section~\ref{resolute}).  Van Inwagen, when denying
that we have beliefs about apples, appears to maintain that the
beliefs that we erroneously take to be about apples are not beliefs
about {\em nothing}.  They are not empty; they are rather beliefs
about something {\em else}, something other than what we took them to
be about (apples).  Van Inwagen accordingly recognizes the need to
explain what our beliefs really are about.

What van Inwagen says is that, when saying things like ``Some chairs
are heavier than some tables'', if we are talking about anything at
all, we are talking about simple particles ``that are arranged
chairwise and\,\ldots\,that are arranged
tablewise''~(\citeyear[109]{inwagen1995}).  He develops an elaborate
`paraphrasing strategy' that is an attempt to show that many (if not
all) propositions expressed by ``There are chairs\,\ldots '' and
related sentences do not actually entail the existence of chairs.

I do not think van Inwagen's defense is ultimately successful (see
section~\ref{inwagen}), but it is precisely the sort of defense
required when denying the existence of ordinary things like tables and
chairs.  He cannot claim, without courting absurdity, that we {\em
  don't} believe there to be tables and chairs in the world.  We do
believe so.  We might, of course, be {\em wrong} about the existence
of tables and chairs; ``from the fact that we believe a certain thing
it does not follow that it is true''~\citep[21]{stroud2000a}.
Nevertheless it is true that we believe there to be such things, and
van Inwagen needs to explain the source of this belief.  If there are
no tables and chairs, then we must be able to understand the beliefs
that we thought to be about the furniture to be about something else
instead, and we need a story about what that something else is.
% ``you cannot hope to explain something unless you grant that there
% is such a thing and you have at least some idea of what it
% is''~\citep[97]{stroud2000a}.
\ As I said, I don't think van Inwagen's explanation is a good one.
But before discussing why, we should consider the third and most
radical sort of denial.

\subsection{The resolute denial}
\label{resolute}
The third kind of denial is that which I likened to a denial of
ghosts.  When I deny the existence of ghosts, I also deny that people
talk about and have beliefs about ghosts.  Even if someone says ``let
me tell you about the ghost I saw last night!'', I can maintain that
there is {\em nothing} in particular that they are talking about.
They are certainly not talking about a certain ghost; there are no
ghosts, and never have been.

Of course I cannot deny that some people believe that ghosts exist.
Some people do believe this.  But my resolute denial of the existence
of ghosts does not prevent me from explaining someone's belief that
there are ghosts.  This someone, for example, might tell me that she
saw a ghost on the landing.  I walk out and see a light from a high
window flickering strangely on the wall.  If I squint, the pattern of
the light looks almost humanoid.  So I tell her that what she thought
was a ghost was really just a curious play of the light.  I am not
here conceding that she had a belief {\em about} a ghost.  On the
contrary, I have tried to show her that her belief was about anything
but a ghost.  There was in fact no ghost that she could have formed a
belief about; ``if we show that what a frightened person saw in the
attic on a particular occasion was a rippling reflection of the moon
through the window, we implicitly deny the presence of a ghost in
giving the explanation of the person's belief and
fear''~\citep[76]{stroud2000a}.  We do not deny that people have
beliefs about what they take to be ghosts; what we deny is that they
are correct in taking their beliefs to be about ghosts.

Likewise, suppose she and I are looking in on an empty, well-lit room.
Suddenly she points and cries, ``Look, a ghost!''  In this case there
is nothing in the room that I can assume to have caused this belief.
There are no reflections of the moon or curious plays of the light; as
far as I can tell, she is pointing at nothing.

\stage{Me}{}{What on earth do you mean?}

\stage{Her}{(pointing)}{That ghost, there! See?}

If this continues, the most probable explanation is that she is having
an hallucination.  She has what she takes to be a belief about a ghost
in the room.  There is no ghost in the room, so what is her belief
about? Likewise, if someone says she saw a ghost in the attic, what
should we say her belief is really about?

In Stroud's example above, we might say that the person is frightened
of the reflection of the moon.  This is similar to how we might talk
about a child's night terrors: ``she was afraid of the chair in her
room (she thought it was a monster).''  Looking at things this way, the
person's beliefs is {\em about} something, but it is something very
different from what they took it to be about.  I think, however, that
this is not a fully accurate characterization of the object of the
person's belief.  It may be true that the reflection of the moon {\em
  caused} her belief (which caused her fright), but it would be at
least a little misleading to say that Mrs --------- is frightened of
the reflection of the moon.  This is misleading because, if we succeed
in showing her that there {\em is} no ghost in the attic (only a
reflection of the moon), she will not still be afraid.  When she
understands that her belief in the ghost was mistaken, she will see
that there is {\em nothing} to be afraid of.  The belief that caused
her fright, that there was a ghost in the attic, was in fact a belief
about nothing at all.  (The same goes for the person who sees the play
of light on the wall and believes that there is a ghost.)

If someone claims to see a ghost in an empty, well-lit room, is her
belief actually about the {\em hallucination}\,?  Again, charity
demands that we not say this.  If she is convinced that nobody else
sees a ghost, she may recognize that she was hallucinating.  Were she
to come to this conclusion, she would no longer be afraid of what she
saw (though she will no doubt be afraid that she is going mad).  The
belief that she took to be about a ghost was in fact about nothing at
all.

With these considerations in mind, we can examine several attempts by
philosophers to deny the existence of certain ordinary things.

\ifstandalone
\end{spacing}
\bibliography{everything}
\bibliographystyle{ChicagoReedweb}
\fi
\end{document}
