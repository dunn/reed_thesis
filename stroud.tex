\documentclass[11pt]{article}
\usepackage{standalone} \newif\ifstandlone \standalonetrue
\usepackage[left=1.75in, right=1.75in, top=1.25in, bottom=1.25in]{geometry}
\geometry{letterpaper}
\usepackage{graphicx}
\usepackage{enumitem}
%\usepackage{amssymb}
\usepackage{amsmath}
\usepackage{verbatim}
\usepackage{epstopdf}
\usepackage{setspace}
\usepackage{natbib}
\setcitestyle{aysep={}}
\usepackage%[colorlinks=true, citecolor=blue, linkcolor=black]%
{hyperref}

\synctex=1

\DeclareSymbolFont{symbolsC}{U}{txsyc}{m}{n}
\DeclareMathSymbol{\strictif}{\mathrel}{symbolsC}{74}
\DeclareMathSymbol{\boxright}{\mathrel}{symbolsC}{128}

\newcommand{\stager}[4]%
{%
	\begin{spacing}{1}%
	\vspace{0pt}
		\begin{description}[style=nextline, noitemsep,
                    parsep=0pt, topsep=0pt, leftmargin=15mm,
                    itemindent=-10mm, font=\mdseries]
			\item[\textsc{#1} \emph{#2}] #3
			\item[]%
			\begin{flushright}#4\end{flushright}
		\end{description}%
	\end{spacing}%
}

\newcommand{\stage}[3]%
{%
	\begin{spacing}{1}%
	\vspace{0pt}
		\begin{description}[style=nextline, parsep=0pt,
                    leftmargin=15mm, itemindent=-10mm, font=\mdseries]
			\item[\textsc{#1} \emph{#2}] #3
		\end{description}%
	\end{spacing}%
}

\newenvironment{squote}{%
	\begin{spacing}{1}
	\begin{list}{}{%
	\setlength{\labelwidth}{0pt}%
	\rightmargin\leftmargin%
	}
	%\begin{singlespace}%
	\item\relax
	}{%
	%\end{singlespace}%
	\end{list}%
	\end{spacing}
	}

\newenvironment{inq}{\vspace{0pt}%
	\begin{list}{}%
	{\setlength\labelwidth{0pt}%
	\setlength\leftmargin{2.5\oddsidemargin}%
	\setlength\rightmargin{\leftmargin}}
	\begin{spacing}{1}
	\item[]%
	}{
	\end{spacing}
	\end{list}
	\vspace{10pt}
	%\noindent%
	}

\title{Why do you think that?}
\author{Alexander A. Dunn}
\begin{document}
\ifstandalone
\maketitle
\begin{spacing}{1.5}
\fi
\label{stroud}

% \begin{inq}
% The philosophical quest must start somewhere. It needs a set of
% beliefs about what the world is like. Without some attitudes,
% perceptions, beliefs, or theories to start with, it would have
% nothing to reflect on.~\citep[16]{stroud2000a}
%\end{inq}

\noindent Section \ref{intro-beliefs} will motivate my claim that a
nihilistic metaphysical thesis should be accompanied by an explanation
of why people nonetheless believe that there are chairs and other
ordinary things.  I will then look at the specific theses of Peter van
Inwagen and Trenton Merricks.  After assessing the ability of each to
explain our beliefs, I will myself try to explain why they think that
it is not obviously true that there are chairs.  Van Inwagen and
Merricks claim that it is not obviously true because they overestimate
what is required for ``there are chairs'' to be true.

\section{Explaining the beliefs of others}
\label{intro-beliefs}
\noindent Many people have false beliefs.  They believe things that
misrepresent (in some sense) how the world is.  For example, some
people believe that ghosts exist.  These people each hold a false
belief, for it is not true that ghosts exist.  There are no ghosts in
the world.  Despite this fact---that there are no ghosts---some people
believe that there are.  Why?  What explanation can we give as to why
someone believes a falsehood like this?

In explaining why someone holds a belief, we appeal to {\em reasons}.
Even people who hold beliefs that we may consider irrational (like the
belief that there are ghosts) have reasons for holding these beliefs.
They may not be good reasons; someone might believe that there are
ghosts because her older sister told her that there are ghosts, or
read ghost stories as a child and took them seriously.  Someone who
believes in ghosts might even think that she has {\em seen} a ghost.
This too would be a false belief; there are no ghosts, so nobody can
have seen one.  But here too there will be a reason why she holds this
false belief.  Perhaps she saw a strange play of light on a distant
wall, or the reflection of the moon filtered through an attic window.
What she actually saw was perhaps one of these things, but she somehow
took what she saw to be a ghost.  Probably she already believed that
there were ghosts, and so, when confronted with a deceptive or
confusing sight, was predisposed to form the mistaken belief that she
was seeing a ghost.

Here and in what follows, when I say that there is a reason why
someone believes something, I mean that there is some {\em cause} that
produced the belief.  Above, I told a causal story about why the
person who believes that she saw a ghost holds that belief.  She had
been told that there were ghosts by a person who she thought
trustworthy, so she came to believe that there are ghosts.  Holding
that believe caused her to be predisposed to interpret unusual
phenomena as ghosts.  This disposition caused her to believe that she
was seeing a ghost when she saw a reflection of the moon.

My use of `reason', therefore, should be taken in this causal sense.
There are other ways that people use the word `reason'.  If someone
asks ``What reason do you have to believe that $((P \rightarrow Q )
\wedge P) \rightarrow Q$?''  I might reply that it is a theorem of
first-order logic.  Here I am not telling a causal story.  I am rather
{\em justifying} my belief that $((P \rightarrow Q ) \wedge P)
\rightarrow Q$.  But in this case it is perfectly correct to say that
I am giving a reason as to why I hold a belief.  It is just not a {\em
  causal} reason.  A causal reason would be something like the
following: $((P \rightarrow Q ) \wedge P) \rightarrow Q$ is true, and
I have learned the rules of logic, and so I can prove that $((P
\rightarrow Q ) \wedge P) \rightarrow Q$.

(Another example: suppose someone falsely believes that $((P
\rightarrow Q ) \wedge Q) \rightarrow P$ is a theorem of first-order
logic.  There will be some (causal) reason why they hold this belief;
probably they attempted to deduce it from no premises and therefore
believe that they succeeded.  There will, in turn, be a reason why
they hold {\em this} false belief; maybe they were not concentrating
on the proof steps, or they forgot certain rules of deduction.)

An example involving an apparently obviously true belief might help
clarify the distinction between causal reason and justifying reasons.
If someone were to ask me why I believe that the sky is blue during
the day, my immediate answer would probably be ``well, because it
is!''  There's not much else I can say to {\em justify} my belief.
But this not a {\em causal} explanation.  The fact that something is
true (the sky {\em is} blue) does not cause me to believe it.
Otherwise I would believe every truth, and I do not.  There are
doubtless many truths that I do not believe.  There must therefore be
another (causal) reason why I believe that the sky is blue, other than
the fact that the sky is blue.

I believe that the sky is blue because, first, it is blue, and second,
I have {\em seen} that it is blue.  My vision is generally reliable
(or at least seems to be), so the fact that my eyes `tell' me
something is good reason to believe it.  The same is true of my other
senses: they are generally reliable, so the fact that they `tell' me
something is a good reason to believe it.  It does not follow that it
is {\em true}, however (though no doubt we believe that it is true);
our eyes can be deceived.

A skeptic might claim that we cannot rule out the possibility that we
are {\em constantly} deceived.  They attempt to undermine the
reliability of our senses.  I will not be addressing such arguments.
Rather, in what follows I will examine arguments that deny (or appear
to deny) that many of our beliefs about `ordinary things' are true.
The philosophers making these denials do not claim that our eyes are
unreliable sources of information.  Their arguments are metaphysical
rather than epistemic; they deny that certain objects are {\em
  possible}.  

For example, Peter Unger claims that chairs do not exist.  He relies
on a number of metaphysical arguments to motivate this claim.  If he
is right, however, then it seems to follow from this that beliefs like
the following are necessarily false:

\begin{itemize}
  \item Some chairs are made of wood.
  \item There have existed many chairs which no longer exist.
  \item There are chairs.
\end{itemize}

I, however, believe that all of these propositions are true.  Even if
Unger is right, and they are all false, it still seems to be the case
that there are reasons why I believe these propositions.

If someone were to ask {\em me} why I believe that there are chairs, I
would probably answer ``because there are, and I have seen them (and
sat upon them)!''  It seems obviously true, just like the fact that
the sky is blue.  I have seen lots of chairs, and I can't have been
confused or deceived {\em every} time.

Nonetheless, Peter Unger and other philosophers (who we will call
`nihilists' or `eliminativists') say that I am mistaken.  They claim
that I have not in fact seen lots of chairs, though I may believe that
I have.  There are several different arguments by which nihilists seek
to establish that chairs (and other `ordinary things') do not exist;
we will examine some of these arguments below.  Having made these
arguments, however, the nihilists must reject our causal explanation
of why we believe that there are chairs.  Our explanation was that
there are chairs and we can see them.  But the nihilist denies that
there are chairs, and so should admit that, if we believe that there
are chairs, there must be a different explanation as to why we hold
this belief.

\subsection{Why bother?}
A metaphysical thesis that involves denying the existence of ordinary
things like chairs entails that the simplest explanation of why we
believe that there are chairs is incorrect.  I believe that such a
thesis should therefore be supplemented with a new explanation.  This
new explanation would identify the reasons why we would believe that
there are chairs if there are in fact none.  But why should I demand
this of a metaphysical theory?  Is it a reasonable request?

As an analogy, consider color.  Most people believe that things are
colored.  A simple causal story about why people believe that things
are colored might go like this:  things are colored, and people see
that things are colored.  

But imagine a philosopher who holds some version of {\em physicalism}
and claims that the world as described by physics is all that there
is.  This view is often seen to have the consequence that things
aren't actually colored.  In the `vocabulary of physics', things might
be described in such a way that the things color gets somehow left
out.  We may be unable to determine from the `physical description'
what color the object is.  The colors of objects are not included in
this philosopher's description of the world.

If the philosopher admits that people believe that things are colored,
she cannot explain this using the same story that I used above.  I
said that people believe that things are colored and that they see
that things are colored.  But the physicalist maintains that things
are not colored.  {\em If} she admits that people believe that things
are colored, then she needs a different explanation as to why people
believe that things are colored.

She might, however, deny that people believe that things are colored.
(This would be a rather bold claim.)  She could say that the notion of
color is entirely illusory.  If we believe that we see colors, she may
tell us we are wrong.  When we think that something is colored, we are
mistaken.  If we think that an apple is red, we have a false belief.
She might claim that color does not pose a difficulty for her view,
because humans do not experience `color'.

This, as I said, is a rather bold claim.  It seems simply true that we
see colors and that the apple looks red.  If a philosopher were to
deny these things, I would have difficulty understanding what she
meant.  This is not to say she is {\em wrong}; I have no argument
proving that her thesis is false.  But the claim that humans do not
experience color seems bizarre and unmotivated.  Fortunately I do not
know of anyone who actually holds this view.

Our imagined philosopher might make a less bold claim.  She might
instead claim that color is one of those things that are `subjective'
rather than `objective' or `absolute' features of the world.  A
subjective feature of the world is a feature that is present only
because we (or some other being) exists to experience it:

\begin{squote}
Whatever is due only to us and to our own ways of responding to and
interacting with the world does not reflect or correspond to anything
present in the world as it is independently of us.  The aim of an
``absolute'' conception, then, is to form a description of the way the
world is, not just independently of its being believed to be that way,
but independently, too, of all the ways in which it happens to present
itself to us human beings from our particular standpoint within
it\,\ldots\,[So we] form some conception of that independent reality
and come to understand parts or aspects of our original conception of
the world as not representing it as it is.  If we see them as products
or reflections of something peculiar to human experience or to the
human perspective on the universe, we assign them a merely
``subjective'' or dependent status and eliminate them from our
conception of the world as it is independently of
us~\citep[30--31]{stroud2000a}.
\end{squote}

A philosopher who adheres to this distinction might claim that our
conception of the world as colored does not represent the world as it
is independently of us.  Colors, she would claim, are not objectively
real.  She allows, however, that they are subjectively real.  She
admits that people do see colors.  Because of our color vision, we
come to believe that the things we see are colored.  A philosopher who
denies the objective reality of color does not thereby ``deny that we
perceive many different colours or that we believe physical objects to
be coloured'' \citep[145]{stroud2000a}.  What this philosopher claims
is something to the effect that, while we see things {\em as} colored,
things are not {\em themselves} colored.  The red color of a tomato,
on this view, obtains only in our perception of the tomato; there is
nothing {\em in} the tomato that is the redness (other species may not
see the redness when they see the tomato).

The philosopher who is denying the objective reality of color does
``recognize the presence in the world of perceptions of and beliefs
about the colours of things'' \citep[199]{stroud2000a}.  The challenge
then is for her to explain why we do have these perceptions and
beliefs.  If she believes that only the world of physics is
objectively real, she must explain why we hold these beliefs, and she
must give this explanation in a manner that commits her only to the
existence of physical things.  If she claims that the world as
described by physics is the only world there is, then she must explain
why, in a world that contains only physical things, we come to believe
that there are colors and colored objects.

Again: if our physicalist philosopher admits that people believe that
they experience color, and admits that people believe that things are
colored, {\em then} she commits herself to explaining why we form
beliefs that are, according to her, false.  Here is the analogy with
metaphysicians like Peter Unger: {\em if} they admit that many of us
believe that there are chairs and other ordinary objects, then they
commit themselves to explaining why we form these false beliefs.  For
as we have seen, even false beliefs are generally held for a reason.

\subsection{Paraphrasing beliefs}
\label{paraphrase}
Peter Unger denies that chairs exist, and claims that, if we believe
that chairs exist, we are mistaken.  His task will be to explain why
we form these false beliefs.  But not all nihilistic philosophers deny
that we are, in fact, mistaken.  They deny that there are any chairs,
but maintain that beliefs like the following might still be true:

\begin{itemize}
  \item There are two chairs in the next room.
  \item I own some very nice 17th-century chairs.
  \item Some chairs are heavier than some tables.
\end{itemize}

Peter van Inwagen is one of these philosophers.  He denies the
existence of tables, chairs, apples, and all other inanimate composite
objects (the definition of `composite' will be discussed below in
section~\ref{scq}).  He takes pains to make clear that his denial of
these things is not a relegation of tables and chairs to `subjective
reality'.  He wants to claim that such things do not exist in any way,
subjective or objective:
\begin{squote}
I want to do what I can to disown a certain apparently almost
irresistible characterization of my view, or of that part of my view
that pertains to inanimate objects.  Many philosophers, in
conversation and correspondence, have insisted, despite repeated
protests on my part, on describing my position in words like these:
``Van Inwagen says that tables are not real''; ``\ldots\,not true
objects''; ``\ldots\,not actually {\em things}''; ``\ldots\,not
substances''; ``\ldots\,not unified wholes''; ``\ldots\,nothing more
than collections of particles.''  These are words that darken counsel.
They are, in fact, perfectly meaningless.  My position vis-\`{a}-vis
tables and other inanimate objects is simply that there {\em are}
none~(\citeyear[99]{inwagen1995}).
\end{squote}

Van Inwagen asserts, quite seriously, that ``there are no tables or
chairs or any other visible objects except living organisms''
(\citeyear[1]{inwagen1995}).  This is a somewhat more bold claim than
that of the physicalist's with regard to color.  She at least granted
that we do see colors, even if we don't actually see things that are
(objectively) colored.  If, as van Inwagen claims, the only {\em
  visible} objects are living organisms, then we certainly can't {\em
  see} chairs at all.  But just as our physicalist could not claim
that we don't believe that there are colors, van Inwagen cannot deny
that we at least {\em believe} that there are chairs.

Van Inwagen does not attempt to deny that we hold beliefs like those
listed above.  He admits that many of us hold beliefs that we would
express as ``there are two chairs in the next room'' or ``I bought a
new chair today''.  Indeed, he admits that such beliefs are often {\em
  true}: ``when people say things in the ordinary business of life by
uttering sentences that start `There are chairs\,\ldots ' or `There
are stars\,\ldots ', they very often say things that are literally
true''~(\citeyear[102]{inwagen1995}).  Van Inwagen, when denying that
we have beliefs about chairs, appears to maintain that the beliefs
that we (erroneously) take to be about chairs are not, in fact,
beliefs about chairs.  If a belief expressed as ``that is a fine
chair'' was actually about a chair, then it could only be true if
there was at least one chair (a fine one).  But van Inwagen denies
that there is at least one chair, but nonetheless says that such a
belief might be true.  He accordingly recognizes the need to explain
what our beliefs really are about.  If he explains what the {\em
  content} of our beliefs is, then he will also be able to explain
{\em why} we hold such beliefs.

\ifstandalone
\end{spacing}
\bibliography{everything}
\bibliographystyle{ChicagoReedweb}
\fi
\end{document}
