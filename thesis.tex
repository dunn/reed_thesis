% This is the Reed College LaTeX thesis template. Most of the work 
% for the document class was done by Sam Noble (SN), as well as this
% template. Later comments etc. by Ben Salzberg (BTS). Additional
% restructuring and APA support by Jess Youngberg (JY).
% Your comments and suggestions are more than welcome; please email
% them to cus@reed.edu

\documentclass[12pt,twoside]{reedfancy}
\usepackage{standalone}
\usepackage{anyfontsize}
\usepackage{graphicx,latexsym} 
\usepackage{amssymb,amsthm,amsmath}
\usepackage{longtable,booktabs,setspace} 
\usepackage{verbatim}
\usepackage{url}
\usepackage{natbib}
\usepackage{enumitem}
\usepackage{hyperref}
\setcitestyle{aysep={}}
\synctex=1

\DeclareSymbolFont{symbolsC}{U}{txsyc}{m}{n}
\DeclareMathSymbol{\strictif}{\mathrel}{symbolsC}{74}
\DeclareMathSymbol{\boxright}{\mathrel}{symbolsC}{128}

\newenvironment{squote}{%
	\begin{spacing}{1}
	\begin{list}{}{%
	\setlength{\labelwidth}{0pt}%
	\rightmargin\leftmargin%
	}
	\item\relax
	}{%
	\end{list}%
	\end{spacing}
	}

\newcommand{\stager}[4]%
{%
	\begin{spacing}{1}%
	\vspace{0pt}
		\begin{description}[style=nextline, noitemsep, parsep=0pt, topsep=0pt, leftmargin=15mm, itemindent=-10mm, font=\mdseries]
			\item[\textsc{#1} \emph{#2}] #3
			\item[]%
			\begin{flushright}#4\end{flushright}
		\end{description}%
	\end{spacing}%
}

\newcommand{\stage}[3]%
{%
	\begin{spacing}{1}%
	\vspace{0pt}
		\begin{description}[style=nextline, parsep=0pt, leftmargin=15mm, itemindent=-10mm, font=\mdseries]
			\item[\textsc{#1} \emph{#2}] #3
		\end{description}%
	\end{spacing}%
}

\newenvironment{inq}{\vspace{0pt}%
	\begin{list}{}{%
	\setlength{\labelwidth}{0pt}%
	\setlength{\leftmargin}{2.5\oddsidemargin}%
	\setlength{\rightmargin}{\leftmargin}}
	\begin{spacing}{1}
	\item[]%
	}{
	\end{spacing}
	\end{list}
	\vspace{10pt}
	%\noindent%
	}
	
\newenvironment{epigram}{%
	\begin{minipage}[c]{0.75\textwidth}
	\vspace{2.5in}
	\begin{spacing}{1}
	\begin{list}{}{%
	\setlength{\labelwidth}{0pt}
	\setlength{\leftmargin}{1.4in}
	\setlength{\rightmargin}{.25in}}
	\item[]
	}{%
	\end{list}
	\end{spacing}
	\end{minipage}
	\newline
	}

% moved to reedfancy.cls:
%\def\thetitle{oink}
%\renewcommand{\firstmark}{\thetitle}
%\newcommand{\chapterpig}[1]{\def\thetitle{#1}}

\title{Real Things}
\author{Alexander A. Dunn}
% The month and year that you submit your FINAL draft TO THE LIBRARY
% (May or December)
\date{May 2012}
\division{Philosophy and Other Things}
\advisor{Paul Hovda}
\department{Philosophy}

\setlength{\parskip}{0pt}

%%%%%%%%%%%%%%%%%%%%%%%%%%%%%%%%%%%%%%%%%%%%

\begin{document}

  \maketitle
  \frontmatter % this stuff will be roman-numbered
  \pagestyle{empty} % this removes page numbers from the frontmatter

%% \begin{epigram}
%% When I was a child, I spake as a child, I understood as a child, I
%% thought as a child: but when I became a man, I put away childish
%% things. \\ \\ \textsc{1 Corinthians 13:11}
%% \end{epigram}

\begin{spacing}{1.25}

% Acknowledgements (Acceptable American spelling) are optional
% So are Acknowledgments (proper English spelling)
    \chapter*{Acknowledgements}
	I want to thank Peter Unger.

% The preface is optional
% To remove it, comment it out or delete it.
%    \chapter*{Preface}
%	This is an example of a thesis setup to use the reed thesis document class.

    \tableofcontents
% if you want a list of tables, optional
   % \listoftables
% if you want a list of figures, also optional
   % \listoffigures

% If your abstract is longer than a page, there may be a formatting issue.
\chapter*{Abstract}
Chairs exist.

	\mainmatter % here the regular arabic numbering starts
	\pagestyle{fancyplain} % turns page numbering back on
  
\chapter{Introduction}
\chapterpig{Introduction}
\documentclass[11pt]{article}
\usepackage{standalone} \newif\ifstandlone \standalonetrue
\usepackage[left=1.75in, right=1.75in, top=1.25in, bottom=1.25in]{geometry}
\geometry{letterpaper}
\usepackage{graphicx}
\usepackage{enumitem}
%\usepackage{amssymb}
\usepackage{amsmath}
\usepackage{verbatim}
\usepackage{epstopdf}
\usepackage{setspace}
\usepackage{natbib}
\setcitestyle{aysep={}}
\usepackage%[colorlinks=true, citecolor=blue, linkcolor=black]%
{hyperref}

\synctex=1

\DeclareSymbolFont{symbolsC}{U}{txsyc}{m}{n}
\DeclareMathSymbol{\strictif}{\mathrel}{symbolsC}{74}
\DeclareMathSymbol{\boxright}{\mathrel}{symbolsC}{128}

\newcommand{\stager}[4]%
{%
	\begin{spacing}{1}%
	\vspace{0pt}
		\begin{description}[style=nextline, noitemsep,
                    parsep=0pt, topsep=0pt, leftmargin=15mm,
                    itemindent=-10mm, font=\mdseries]
			\item[\textsc{#1} \emph{#2}] #3
			\item[]%
			\begin{flushright}#4\end{flushright}
		\end{description}%
	\end{spacing}%
}

\newcommand{\stage}[3]%
{%
	\begin{spacing}{1}%
	\vspace{0pt}
		\begin{description}[style=nextline, parsep=0pt,
                    leftmargin=15mm, itemindent=-10mm, font=\mdseries]
			\item[\textsc{#1} \emph{#2}] #3
		\end{description}%
	\end{spacing}%
}

\newenvironment{squote}{%
	\begin{spacing}{1}
	\begin{list}{}{%
	\setlength{\labelwidth}{0pt}%
	\rightmargin\leftmargin%
	}
	%\begin{singlespace}%
	\item\relax
	}{%
	%\end{singlespace}%
	\end{list}%
	\end{spacing}
	}

\newenvironment{inq}{\vspace{0pt}%
	\begin{list}{}%
	{\setlength\labelwidth{0pt}%
	\setlength\leftmargin{2.5\oddsidemargin}%
	\setlength\rightmargin{\leftmargin}}
	\begin{spacing}{1}
	\item[]%
	}{
	\end{spacing}
	\end{list}
	\vspace{10pt}
	%\noindent%
	}

\title{Why do you think that?}
\author{Alexander A. Dunn}
\begin{document}
\ifstandalone
\maketitle
\begin{spacing}{1.5}
\fi
\label{stroud}

% \begin{inq}
% The philosophical quest must start somewhere. It needs a set of
% beliefs about what the world is like. Without some attitudes,
% perceptions, beliefs, or theories to start with, it would have
% nothing to reflect on.~\citep[16]{stroud2000a}
%\end{inq}

\noindent Section \ref{intro-beliefs} will motivate my claim that a
nihilistic metaphysical thesis should be accompanied by an explanation
of why people nonetheless believe that there are chairs and other
ordinary things.  I will then look at the specific theses of Peter van
Inwagen and Trenton Merricks.  After assessing the ability of each to
explain our beliefs, I will myself try to explain why they think that
it is not obviously true that there are chairs.  Van Inwagen and
Merricks claim that it is not obviously true because they overestimate
what is required for ``there are chairs'' to be true.

\section{Explaining the beliefs of others}
\label{intro-beliefs}
\noindent Many people have false beliefs.  They believe things that
misrepresent (in some sense) how the world is.  For example, some
people believe that ghosts exist.  These people each hold a false
belief, for it is not true that ghosts exist.  There are no ghosts in
the world.  Despite this fact---that there are no ghosts---some people
believe that there are.  Why?  What explanation can we give as to why
someone believes a falsehood like this?

In explaining why someone holds a belief, we appeal to {\em reasons}.
Even people who hold beliefs that we may consider irrational (like the
belief that there are ghosts) have reasons for holding these beliefs.
They may not be good reasons; someone might believe that there are
ghosts because her older sister told her that there are ghosts, or
read ghost stories as a child and took them seriously.  Someone who
believes in ghosts might even think that she has {\em seen} a ghost.
This too would be a false belief; there are no ghosts, so nobody can
have seen one.  But here too there will be a reason why she holds this
false belief.  Perhaps she saw a strange play of light on a distant
wall, or the reflection of the moon filtered through an attic window.
What she actually saw was perhaps one of these things, but she somehow
took what she saw to be a ghost.  Probably she already believed that
there were ghosts, and so, when confronted with a deceptive or
confusing sight, was predisposed to form the mistaken belief that she
was seeing a ghost.

Here and in what follows, when I say that there is a reason why
someone believes something, I mean that there is some {\em cause} that
produced the belief.  Above, I told a causal story about why the
person who believes that she saw a ghost holds that belief.  She had
been told that there were ghosts by a person who she thought
trustworthy, so she came to believe that there are ghosts.  Holding
that believe caused her to be predisposed to interpret unusual
phenomena as ghosts.  This disposition caused her to believe that she
was seeing a ghost when she saw a reflection of the moon.

My use of `reason', therefore, should be taken in this causal sense.
There are other ways that people use the word `reason'.  If someone
asks ``What reason do you have to believe that $((P \rightarrow Q )
\wedge P) \rightarrow Q$?''  I might reply that it is a theorem of
first-order logic.  Here I am not telling a causal story.  I am rather
{\em justifying} my belief that $((P \rightarrow Q ) \wedge P)
\rightarrow Q$.  But in this case it is perfectly correct to say that
I am giving a reason as to why I hold a belief.  It is just not a {\em
  causal} reason.  A causal reason would be something like the
following: $((P \rightarrow Q ) \wedge P) \rightarrow Q$ is true, and
I have learned the rules of logic, and so I can prove that $((P
\rightarrow Q ) \wedge P) \rightarrow Q$.

(Another example: suppose someone falsely believes that $((P
\rightarrow Q ) \wedge Q) \rightarrow P$ is a theorem of first-order
logic.  There will be some (causal) reason why they hold this belief;
probably they attempted to deduce it from no premises and therefore
believe that they succeeded.  There will, in turn, be a reason why
they hold {\em this} false belief; maybe they were not concentrating
on the proof steps, or they forgot certain rules of deduction.)

An example involving an apparently obviously true belief might help
clarify the distinction between causal reason and justifying reasons.
If someone were to ask me why I believe that the sky is blue during
the day, my immediate answer would probably be ``well, because it
is!''  There's not much else I can say to {\em justify} my belief.
But this not a {\em causal} explanation.  The fact that something is
true (the sky {\em is} blue) does not cause me to believe it.
Otherwise I would believe every truth, and I do not.  There are
doubtless many truths that I do not believe.  There must therefore be
another (causal) reason why I believe that the sky is blue, other than
the fact that the sky is blue.

I believe that the sky is blue because, first, it is blue, and second,
I have {\em seen} that it is blue.  My vision is generally reliable
(or at least seems to be), so the fact that my eyes `tell' me
something is good reason to believe it.  The same is true of my other
senses: they are generally reliable, so the fact that they `tell' me
something is a good reason to believe it.  It does not follow that it
is {\em true}, however (though no doubt we believe that it is true);
our eyes can be deceived.

A skeptic might claim that we cannot rule out the possibility that we
are {\em constantly} deceived.  They attempt to undermine the
reliability of our senses.  I will not be addressing such arguments.
Rather, in what follows I will examine arguments that deny (or appear
to deny) that many of our beliefs about `ordinary things' are true.
The philosophers making these denials do not claim that our eyes are
unreliable sources of information.  Their arguments are metaphysical
rather than epistemic; they deny that certain objects are {\em
  possible}.  

For example, Peter Unger claims that chairs do not exist.  He relies
on a number of metaphysical arguments to motivate this claim.  If he
is right, however, then it seems to follow from this that beliefs like
the following are necessarily false:

\begin{itemize}
  \item Some chairs are made of wood.
  \item There have existed many chairs which no longer exist.
  \item There are chairs.
\end{itemize}

I, however, believe that all of these propositions are true.  Even if
Unger is right, and they are all false, it still seems to be the case
that there are reasons why I believe these propositions.

If someone were to ask {\em me} why I believe that there are chairs, I
would probably answer ``because there are, and I have seen them (and
sat upon them)!''  It seems obviously true, just like the fact that
the sky is blue.  I have seen lots of chairs, and I can't have been
confused or deceived {\em every} time.

Nonetheless, Peter Unger and other philosophers (who we will call
`nihilists' or `eliminativists') say that I am mistaken.  They claim
that I have not in fact seen lots of chairs, though I may believe that
I have.  There are several different arguments by which nihilists seek
to establish that chairs (and other `ordinary things') do not exist;
we will examine some of these arguments below.  Having made these
arguments, however, the nihilists must reject our causal explanation
of why we believe that there are chairs.  Our explanation was that
there are chairs and we can see them.  But the nihilist denies that
there are chairs, and so should admit that, if we believe that there
are chairs, there must be a different explanation as to why we hold
this belief.

\subsection{Why bother?}
A metaphysical thesis that involves denying the existence of ordinary
things like chairs entails that the simplest explanation of why we
believe that there are chairs is incorrect.  I believe that such a
thesis should therefore be supplemented with a new explanation.  This
new explanation would identify the reasons why we would believe that
there are chairs if there are in fact none.  But why should I demand
this of a metaphysical theory?  Is it a reasonable request?

As an analogy, consider color.  Most people believe that things are
colored.  A simple causal story about why people believe that things
are colored might go like this:  things are colored, and people see
that things are colored.  

But imagine a philosopher who holds some version of {\em physicalism}
and claims that the world as described by physics is all that there
is.  This view is often thought to have the consequence that things
aren't actually colored.  In the `vocabulary of physics', things might
be described in such a way that the things color gets somehow left
out.  We may be unable to determine from the `physical description'
what color the object is.  The colors of objects are not included in
this philosopher's description of the world.

If the philosopher admits that people believe that things are colored,
she cannot explain this using the same story that I used above.  I
said that people believe that things are colored and that they see
that things are colored.  But the physicalist maintains that things
are not colored.  {\em If} she admits that people believe that things
are colored, then she needs a different explanation as to why people
believe that things are colored.

She might, however, deny that people believe that things are colored.
(This would be a rather bold claim.)  She could say that the notion of
color is entirely illusory.  If we believe that we see colors, she may
tell us we are wrong.  When we think that something is colored, we are
mistaken.  If we think that an apple is red, we have a false belief.
She might claim that color does not pose a difficulty for her view,
because humans do not experience `color'.

This, as I said, is a rather bold claim.  It seems simply true that we
see colors and that the apple looks red.  If a philosopher were to
deny these things, I would have difficulty understanding what she
meant.  This is not to say she is {\em wrong}; I have no argument
proving that her thesis is false.  But the claim that humans do not
experience color seems bizarre and unmotivated.  Fortunately I do not
know of anyone who actually holds this view.

Our imagined philosopher might make a less bold claim.  She might
instead claim that color is one of those things that are `subjective'
rather than `objective' or `absolute' features of the world.  A
subjective feature of the world is a feature that is present only
because we (or some other being) exists to experience it:

\begin{squote}
Whatever is due only to us and to our own ways of responding to and
interacting with the world does not reflect or correspond to anything
present in the world as it is independently of us.  The aim of an
``absolute'' conception, then, is to form a description of the way the
world is, not just independently of its being believed to be that way,
but independently, too, of all the ways in which it happens to present
itself to us human beings from our particular standpoint within
it\,\ldots\,[So we] form some conception of that independent reality
and come to understand parts or aspects of our original conception of
the world as not representing it as it is.  If we see them as products
or reflections of something peculiar to human experience or to the
human perspective on the universe, we assign them a merely
``subjective'' or dependent status and eliminate them from our
conception of the world as it is independently of
us~\citep[30--31]{stroud2000a}.
\end{squote}

A philosopher who adheres to this distinction might claim that our
conception of the world as colored does not represent the world as it
is independently of us.  Colors, she would claim, are not objectively
real.  She allows, however, that they are subjectively real.  She
admits that people do see colors.  Because of our color vision, we
come to believe that the things we see are colored.  A philosopher who
denies the objective reality of color does not thereby ``deny that we
perceive many different colours or that we believe physical objects to
be coloured'' \citep[145]{stroud2000a}.  What this philosopher claims
is something to the effect that, while we see things {\em as} colored,
things are not {\em themselves} colored.  The red color of a tomato,
on this view, obtains only in our perception of the tomato; there is
nothing {\em in} the tomato that is the redness (other species may not
see the redness when they see the tomato).

The philosopher who is denying the objective reality of color does
``recognize the presence in the world of perceptions of and beliefs
about the colours of things'' \citep[199]{stroud2000a}.  The challenge
then is for her to explain why we do have these perceptions and
beliefs.  If she believes that only the world of physics is
objectively real, she must explain why we hold these beliefs, and she
must give this explanation in such a way that commits her only to the
existence of physical things.  If she claims that the world as
described by physics is the only world there is, then she must explain
why, in a world that contains only physical things, we come to believe
that there are colors and colored objects.

Again: if our physicalist philosopher admits that people believe that
they experience color, and admits that people believe that things are
colored, {\em then} she commits herself to explaining why we form
beliefs that are, according to her, false.  Here is the analogy with
metaphysicians like Peter Unger: {\em if} they admit that many of us
believe that there are chairs and other ordinary objects, then they
commit themselves to explaining why we form these false beliefs.  For
as we have seen, even false beliefs are generally held for a reason.

\subsection{Paraphrasing beliefs}
\label{paraphrase}
Peter Unger denies that chairs exist, and claims that, if we believe
that chairs exist, we are mistaken.  His task will be to explain why
we form these false beliefs.  But not all nihilistic philosophers deny
that we are, in fact, mistaken.  They deny that there are any chairs,
but maintain that beliefs like the following might still be true:

\begin{itemize}
  \item There are two chairs in the next room.
  \item I own some very nice 17th-century chairs.
  \item Some chairs are heavier than some tables.
\end{itemize}

Peter van Inwagen is one of these philosophers.  He denies the
existence of tables, chairs, apples, and all other inanimate composite
objects (van Inwagen's technical definition of `composite' will be
discussed below in section~\ref{scq}).  He takes pains to make clear
that his denial of these things is not a relegation of tables and
chairs to `subjective reality'.  He wants to claim that such things do
not exist in any way, subjective or objective:
\begin{squote}
I want to do what I can to disown a certain apparently almost
irresistible characterization of my view, or of that part of my view
that pertains to inanimate objects.  Many philosophers, in
conversation and correspondence, have insisted, despite repeated
protests on my part, on describing my position in words like these:
``Van Inwagen says that tables are not real''; ``\ldots\,not true
objects''; ``\ldots\,not actually {\em things}''; ``\ldots\,not
substances''; ``\ldots\,not unified wholes''; ``\ldots\,nothing more
than collections of particles.''  These are words that darken counsel.
They are, in fact, perfectly meaningless.  My position vis-\`{a}-vis
tables and other inanimate objects is simply that there {\em are}
none~(\citeyear[99]{inwagen1995}).
\end{squote}

Van Inwagen asserts, quite seriously, that ``there are no tables or
chairs or any other visible objects except living organisms''
(\citeyear[1]{inwagen1995}).  This is a somewhat more bold claim than
that of the physicalist's with regard to color.  She at least granted
that we do see colors, even if we don't actually see things that are
(objectively) colored.  If, as van Inwagen claims, the only {\em
  visible} objects are living organisms, then we certainly can't {\em
  see} chairs at all.  But just as our physicalist could not claim
that we don't believe that there are colors, van Inwagen cannot deny
that we at least {\em believe} that there are chairs.

Van Inwagen does not attempt to deny that we hold beliefs like those
listed above.  He admits that many of us hold beliefs that we would
express as ``there are two chairs in the next room'' or ``I bought a
new chair today''.  Indeed, he admits that such beliefs are often {\em
  true}: ``when people say things in the ordinary business of life by
uttering sentences that start `There are chairs\,\ldots ' or `There
are stars\,\ldots ', they very often say things that are literally
true''~(\citeyear[102]{inwagen1995}).  Van Inwagen, when denying that
we have beliefs about chairs, appears to maintain that the beliefs
that we (erroneously) take to be about chairs are not, in fact,
beliefs about chairs.  If a belief expressed as ``that is a fine
chair'' was actually about a chair, then it could only be true if
there was at least one chair (a fine one).  But van Inwagen denies
that there is at least one chair, but nonetheless says that such a
belief might be true.  He accordingly recognizes the need to explain
what our beliefs really are about.  If he explains what the {\em
  content} of our beliefs is, then he will also be able to explain
{\em why} we hold such beliefs.

\ifstandalone
\end{spacing}
\bibliography{everything}
\bibliographystyle{ChicagoReedweb}
\fi
\end{document}


%\addcontentsline{toc}{chapter}{Introduction}
%\chaptermark{Introduction}
%\markboth{Introduction}{Introduction}

% The three lines above are to make sure that the headers are right,
% that the intro gets included in the table of contents, and that it
% doesn't get numbered 1 so that chapter one is 1.

\chapter{Belief}
\chapterpig{Belief}
\documentclass[11pt]{article}
\usepackage{standalone}
\usepackage[left=1.75in, right=1.75in, top=1.25in, bottom=1.25in]{geometry}
\geometry{letterpaper}
\usepackage{graphicx}
%\usepackage{xyling}
%\usepackage{tipa}
%\usepackage{exaccent}
%\usepackage{txfonts}
%\usepackage{pxfonts}
\usepackage{enumitem}
%\usepackage{amssymb}
\usepackage{amsmath}
\usepackage{epstopdf}
\usepackage{setspace}
\usepackage{natbib}
\setcitestyle{aysep={}}
\usepackage{hyperref}
\usepackage{url}
\synctex=1

\DeclareSymbolFont{symbolsC}{U}{txsyc}{m}{n}
\DeclareMathSymbol{\strictif}{\mathrel}{symbolsC}{74}
\DeclareMathSymbol{\boxright}{\mathrel}{symbolsC}{128}		

\newcommand{\stager}[4]%
{%
	\begin{spacing}{1}%
	\vspace{0pt}
		\begin{description}[style=nextline, noitemsep, parsep=0pt, topsep=0pt, leftmargin=15mm, itemindent=-10mm, font=\mdseries]
			\item[\textsc{#1} \emph{#2}] #3
			\item[]%
			\begin{flushright}#4\end{flushright}
		\end{description}%
	\end{spacing}%
}

\newcommand{\stage}[3]%
{%
	\begin{spacing}{1}%
	\vspace{0pt}
		\begin{description}[style=nextline, parsep=0pt, leftmargin=15mm, itemindent=-10mm, font=\mdseries]
			\item[\textsc{#1} \emph{#2}] #3
		\end{description}%
	\end{spacing}%
}

\newenvironment{squote}{%
       \begin{spacing}{1}
       \begin{list}{}{%
       \setlength{\labelwidth}{0pt}%
       \rightmargin\leftmargin%
       }
       \item\relax
       }{%
       \end{list}%
       \end{spacing}
}

\title{Things}
\author{Alexander A. Dunn}
\begin{document}
\maketitle
\tableofcontents
\begin{spacing}{1.25}

\documentclass[11pt]{article}
\usepackage{standalone} \newif\ifstandlone \standalonetrue
\usepackage[left=1.75in, right=1.75in, top=1.25in, bottom=1.25in]{geometry}
\geometry{letterpaper}
\usepackage{graphicx}
\usepackage{enumitem}
%\usepackage{amssymb}
\usepackage{amsmath}
\usepackage{verbatim}
\usepackage{epstopdf}
\usepackage{setspace}
\usepackage{natbib}
\setcitestyle{aysep={}}
\usepackage%[colorlinks=true, citecolor=blue, linkcolor=black]%
{hyperref}

\synctex=1

\DeclareSymbolFont{symbolsC}{U}{txsyc}{m}{n}
\DeclareMathSymbol{\strictif}{\mathrel}{symbolsC}{74}
\DeclareMathSymbol{\boxright}{\mathrel}{symbolsC}{128}

\newcommand{\stager}[4]%
{%
	\begin{spacing}{1}%
	\vspace{0pt}
		\begin{description}[style=nextline, noitemsep,
                    parsep=0pt, topsep=0pt, leftmargin=15mm,
                    itemindent=-10mm, font=\mdseries]
			\item[\textsc{#1} \emph{#2}] #3
			\item[]%
			\begin{flushright}#4\end{flushright}
		\end{description}%
	\end{spacing}%
}

\newcommand{\stage}[3]%
{%
	\begin{spacing}{1}%
	\vspace{0pt}
		\begin{description}[style=nextline, parsep=0pt,
                    leftmargin=15mm, itemindent=-10mm, font=\mdseries]
			\item[\textsc{#1} \emph{#2}] #3
		\end{description}%
	\end{spacing}%
}

\newenvironment{squote}{%
	\begin{spacing}{1}
	\begin{list}{}{%
	\setlength{\labelwidth}{0pt}%
	\rightmargin\leftmargin%
	}
	%\begin{singlespace}%
	\item\relax
	}{%
	%\end{singlespace}%
	\end{list}%
	\end{spacing}
	}

\newenvironment{inq}{\vspace{0pt}%
	\begin{list}{}%
	{\setlength\labelwidth{0pt}%
	\setlength\leftmargin{2.5\oddsidemargin}%
	\setlength\rightmargin{\leftmargin}}
	\begin{spacing}{1}
	\item[]%
	}{
	\end{spacing}
	\end{list}
	\vspace{10pt}
	%\noindent%
	}

\title{Why do you think that?}
\author{Alexander A. Dunn}
\begin{document}
\ifstandalone
\maketitle
\begin{spacing}{1.5}
\fi
\label{stroud}

% \begin{inq}
% The philosophical quest must start somewhere. It needs a set of
% beliefs about what the world is like. Without some attitudes,
% perceptions, beliefs, or theories to start with, it would have
% nothing to reflect on.~\citep[16]{stroud2000a}
%\end{inq}

\noindent Section \ref{intro-beliefs} will motivate my claim that a
nihilistic metaphysical thesis should be accompanied by an explanation
of why people nonetheless believe that there are chairs and other
ordinary things.  I will then look at the specific theses of Peter van
Inwagen and Trenton Merricks.  After assessing the ability of each to
explain our beliefs, I will myself try to explain why they think that
it is not obviously true that there are chairs.  Van Inwagen and
Merricks claim that it is not obviously true because they overestimate
what is required for ``there are chairs'' to be true.

\section{Explaining the beliefs of others}
\label{intro-beliefs}
\noindent Many people have false beliefs.  They believe things that
misrepresent (in some sense) how the world is.  For example, some
people believe that ghosts exist.  These people each hold a false
belief, for it is not true that ghosts exist.  There are no ghosts in
the world.  Despite this fact---that there are no ghosts---some people
believe that there are.  Why?  What explanation can we give as to why
someone believes a falsehood like this?

In explaining why someone holds a belief, we appeal to {\em reasons}.
Even people who hold beliefs that we may consider irrational (like the
belief that there are ghosts) have reasons for holding these beliefs.
They may not be good reasons; someone might believe that there are
ghosts because her older sister told her that there are ghosts, or
read ghost stories as a child and took them seriously.  Someone who
believes in ghosts might even think that she has {\em seen} a ghost.
This too would be a false belief; there are no ghosts, so nobody can
have seen one.  But here too there will be a reason why she holds this
false belief.  Perhaps she saw a strange play of light on a distant
wall, or the reflection of the moon filtered through an attic window.
What she actually saw was perhaps one of these things, but she somehow
took what she saw to be a ghost.  Probably she already believed that
there were ghosts, and so, when confronted with a deceptive or
confusing sight, was predisposed to form the mistaken belief that she
was seeing a ghost.

Here and in what follows, when I say that there is a reason why
someone believes something, I mean that there is some {\em cause} that
produced the belief.  Above, I told a causal story about why the
person who believes that she saw a ghost holds that belief.  She had
been told that there were ghosts by a person who she thought
trustworthy, so she came to believe that there are ghosts.  Holding
that believe caused her to be predisposed to interpret unusual
phenomena as ghosts.  This disposition caused her to believe that she
was seeing a ghost when she saw a reflection of the moon.

My use of `reason', therefore, should be taken in this causal sense.
There are other ways that people use the word `reason'.  If someone
asks ``What reason do you have to believe that $((P \rightarrow Q )
\wedge P) \rightarrow Q$?''  I might reply that it is a theorem of
first-order logic.  Here I am not telling a causal story.  I am rather
{\em justifying} my belief that $((P \rightarrow Q ) \wedge P)
\rightarrow Q$.  But in this case it is perfectly correct to say that
I am giving a reason as to why I hold a belief.  It is just not a {\em
  causal} reason.  A causal reason would be something like the
following: $((P \rightarrow Q ) \wedge P) \rightarrow Q$ is true, and
I have learned the rules of logic, and so I can prove that $((P
\rightarrow Q ) \wedge P) \rightarrow Q$.

(Another example: suppose someone falsely believes that $((P
\rightarrow Q ) \wedge Q) \rightarrow P$ is a theorem of first-order
logic.  There will be some (causal) reason why they hold this belief;
probably they attempted to deduce it from no premises and therefore
believe that they succeeded.  There will, in turn, be a reason why
they hold {\em this} false belief; maybe they were not concentrating
on the proof steps, or they forgot certain rules of deduction.)

An example involving an apparently obviously true belief might help
clarify the distinction between causal reason and justifying reasons.
If someone were to ask me why I believe that the sky is blue during
the day, my immediate answer would probably be ``well, because it
is!''  There's not much else I can say to {\em justify} my belief.
But this not a {\em causal} explanation.  The fact that something is
true (the sky {\em is} blue) does not cause me to believe it.
Otherwise I would believe every truth, and I do not.  There are
doubtless many truths that I do not believe.  There must therefore be
another (causal) reason why I believe that the sky is blue, other than
the fact that the sky is blue.

I believe that the sky is blue because, first, it is blue, and second,
I have {\em seen} that it is blue.  My vision is generally reliable
(or at least seems to be), so the fact that my eyes `tell' me
something is good reason to believe it.  The same is true of my other
senses: they are generally reliable, so the fact that they `tell' me
something is a good reason to believe it.  It does not follow that it
is {\em true}, however (though no doubt we believe that it is true);
our eyes can be deceived.

A skeptic might claim that we cannot rule out the possibility that we
are {\em constantly} deceived.  They attempt to undermine the
reliability of our senses.  I will not be addressing such arguments.
Rather, in what follows I will examine arguments that deny (or appear
to deny) that many of our beliefs about `ordinary things' are true.
The philosophers making these denials do not claim that our eyes are
unreliable sources of information.  Their arguments are metaphysical
rather than epistemic; they deny that certain objects are {\em
  possible}.  

For example, Peter Unger claims that chairs do not exist.  He relies
on a number of metaphysical arguments to motivate this claim.  If he
is right, however, then it seems to follow from this that beliefs like
the following are necessarily false:

\begin{itemize}
  \item Some chairs are made of wood.
  \item There have existed many chairs which no longer exist.
  \item There are chairs.
\end{itemize}

I, however, believe that all of these propositions are true.  Even if
Unger is right, and they are all false, it still seems to be the case
that there are reasons why I believe these propositions.

If someone were to ask {\em me} why I believe that there are chairs, I
would probably answer ``because there are, and I have seen them (and
sat upon them)!''  It seems obviously true, just like the fact that
the sky is blue.  I have seen lots of chairs, and I can't have been
confused or deceived {\em every} time.

Nonetheless, Peter Unger and other philosophers (who we will call
`nihilists' or `eliminativists') say that I am mistaken.  They claim
that I have not in fact seen lots of chairs, though I may believe that
I have.  There are several different arguments by which nihilists seek
to establish that chairs (and other `ordinary things') do not exist;
we will examine some of these arguments below.  Having made these
arguments, however, the nihilists must reject our causal explanation
of why we believe that there are chairs.  Our explanation was that
there are chairs and we can see them.  But the nihilist denies that
there are chairs, and so should admit that, if we believe that there
are chairs, there must be a different explanation as to why we hold
this belief.

\subsection{Why bother?}
A metaphysical thesis that involves denying the existence of ordinary
things like chairs entails that the simplest explanation of why we
believe that there are chairs is incorrect.  I believe that such a
thesis should therefore be supplemented with a new explanation.  This
new explanation would identify the reasons why we would believe that
there are chairs if there are in fact none.  But why should I demand
this of a metaphysical theory?  Is it a reasonable request?

As an analogy, consider color.  Most people believe that things are
colored.  A simple causal story about why people believe that things
are colored might go like this:  things are colored, and people see
that things are colored.  

But imagine a philosopher who holds some version of {\em physicalism}
and claims that the world as described by physics is all that there
is.  This view is often thought to have the consequence that things
aren't actually colored.  In the `vocabulary of physics', things might
be described in such a way that the things color gets somehow left
out.  We may be unable to determine from the `physical description'
what color the object is.  The colors of objects are not included in
this philosopher's description of the world.

If the philosopher admits that people believe that things are colored,
she cannot explain this using the same story that I used above.  I
said that people believe that things are colored and that they see
that things are colored.  But the physicalist maintains that things
are not colored.  {\em If} she admits that people believe that things
are colored, then she needs a different explanation as to why people
believe that things are colored.

She might, however, deny that people believe that things are colored.
(This would be a rather bold claim.)  She could say that the notion of
color is entirely illusory.  If we believe that we see colors, she may
tell us we are wrong.  When we think that something is colored, we are
mistaken.  If we think that an apple is red, we have a false belief.
She might claim that color does not pose a difficulty for her view,
because humans do not experience `color'.

This, as I said, is a rather bold claim.  It seems simply true that we
see colors and that the apple looks red.  If a philosopher were to
deny these things, I would have difficulty understanding what she
meant.  This is not to say she is {\em wrong}; I have no argument
proving that her thesis is false.  But the claim that humans do not
experience color seems bizarre and unmotivated.  Fortunately I do not
know of anyone who actually holds this view.

Our imagined philosopher might make a less bold claim.  She might
instead claim that color is one of those things that are `subjective'
rather than `objective' or `absolute' features of the world.  A
subjective feature of the world is a feature that is present only
because we (or some other being) exists to experience it:

\begin{squote}
Whatever is due only to us and to our own ways of responding to and
interacting with the world does not reflect or correspond to anything
present in the world as it is independently of us.  The aim of an
``absolute'' conception, then, is to form a description of the way the
world is, not just independently of its being believed to be that way,
but independently, too, of all the ways in which it happens to present
itself to us human beings from our particular standpoint within
it\,\ldots\,[So we] form some conception of that independent reality
and come to understand parts or aspects of our original conception of
the world as not representing it as it is.  If we see them as products
or reflections of something peculiar to human experience or to the
human perspective on the universe, we assign them a merely
``subjective'' or dependent status and eliminate them from our
conception of the world as it is independently of
us~\citep[30--31]{stroud2000a}.
\end{squote}

A philosopher who adheres to this distinction might claim that our
conception of the world as colored does not represent the world as it
is independently of us.  Colors, she would claim, are not objectively
real.  She allows, however, that they are subjectively real.  She
admits that people do see colors.  Because of our color vision, we
come to believe that the things we see are colored.  A philosopher who
denies the objective reality of color does not thereby ``deny that we
perceive many different colours or that we believe physical objects to
be coloured'' \citep[145]{stroud2000a}.  What this philosopher claims
is something to the effect that, while we see things {\em as} colored,
things are not {\em themselves} colored.  The red color of a tomato,
on this view, obtains only in our perception of the tomato; there is
nothing {\em in} the tomato that is the redness (other species may not
see the redness when they see the tomato).

The philosopher who is denying the objective reality of color does
``recognize the presence in the world of perceptions of and beliefs
about the colours of things'' \citep[199]{stroud2000a}.  The challenge
then is for her to explain why we do have these perceptions and
beliefs.  If she believes that only the world of physics is
objectively real, she must explain why we hold these beliefs, and she
must give this explanation in such a way that commits her only to the
existence of physical things.  If she claims that the world as
described by physics is the only world there is, then she must explain
why, in a world that contains only physical things, we come to believe
that there are colors and colored objects.

Again: if our physicalist philosopher admits that people believe that
they experience color, and admits that people believe that things are
colored, {\em then} she commits herself to explaining why we form
beliefs that are, according to her, false.  Here is the analogy with
metaphysicians like Peter Unger: {\em if} they admit that many of us
believe that there are chairs and other ordinary objects, then they
commit themselves to explaining why we form these false beliefs.  For
as we have seen, even false beliefs are generally held for a reason.

\subsection{Paraphrasing beliefs}
\label{paraphrase}
Peter Unger denies that chairs exist, and claims that, if we believe
that chairs exist, we are mistaken.  His task will be to explain why
we form these false beliefs.  But not all nihilistic philosophers deny
that we are, in fact, mistaken.  They deny that there are any chairs,
but maintain that beliefs like the following might still be true:

\begin{itemize}
  \item There are two chairs in the next room.
  \item I own some very nice 17th-century chairs.
  \item Some chairs are heavier than some tables.
\end{itemize}

Peter van Inwagen is one of these philosophers.  He denies the
existence of tables, chairs, apples, and all other inanimate composite
objects (van Inwagen's technical definition of `composite' will be
discussed below in section~\ref{scq}).  He takes pains to make clear
that his denial of these things is not a relegation of tables and
chairs to `subjective reality'.  He wants to claim that such things do
not exist in any way, subjective or objective:
\begin{squote}
I want to do what I can to disown a certain apparently almost
irresistible characterization of my view, or of that part of my view
that pertains to inanimate objects.  Many philosophers, in
conversation and correspondence, have insisted, despite repeated
protests on my part, on describing my position in words like these:
``Van Inwagen says that tables are not real''; ``\ldots\,not true
objects''; ``\ldots\,not actually {\em things}''; ``\ldots\,not
substances''; ``\ldots\,not unified wholes''; ``\ldots\,nothing more
than collections of particles.''  These are words that darken counsel.
They are, in fact, perfectly meaningless.  My position vis-\`{a}-vis
tables and other inanimate objects is simply that there {\em are}
none~(\citeyear[99]{inwagen1995}).
\end{squote}

Van Inwagen asserts, quite seriously, that ``there are no tables or
chairs or any other visible objects except living organisms''
(\citeyear[1]{inwagen1995}).  This is a somewhat more bold claim than
that of the physicalist's with regard to color.  She at least granted
that we do see colors, even if we don't actually see things that are
(objectively) colored.  If, as van Inwagen claims, the only {\em
  visible} objects are living organisms, then we certainly can't {\em
  see} chairs at all.  But just as our physicalist could not claim
that we don't believe that there are colors, van Inwagen cannot deny
that we at least {\em believe} that there are chairs.

Van Inwagen does not attempt to deny that we hold beliefs like those
listed above.  He admits that many of us hold beliefs that we would
express as ``there are two chairs in the next room'' or ``I bought a
new chair today''.  Indeed, he admits that such beliefs are often {\em
  true}: ``when people say things in the ordinary business of life by
uttering sentences that start `There are chairs\,\ldots ' or `There
are stars\,\ldots ', they very often say things that are literally
true''~(\citeyear[102]{inwagen1995}).  Van Inwagen, when denying that
we have beliefs about chairs, appears to maintain that the beliefs
that we (erroneously) take to be about chairs are not, in fact,
beliefs about chairs.  If a belief expressed as ``that is a fine
chair'' was actually about a chair, then it could only be true if
there was at least one chair (a fine one).  But van Inwagen denies
that there is at least one chair, but nonetheless says that such a
belief might be true.  He accordingly recognizes the need to explain
what our beliefs really are about.  If he explains what the {\em
  content} of our beliefs is, then he will also be able to explain
{\em why} we hold such beliefs.

\ifstandalone
\end{spacing}
\bibliography{everything}
\bibliographystyle{ChicagoReedweb}
\fi
\end{document}


\documentclass[11pt]{article}
\usepackage{standalone} \newif\ifstandlone \standalonetrue
\usepackage[left=1.75in, right=1.75in, top=1.25in, bottom=1.25in]{geometry}
\geometry{letterpaper}
\usepackage{graphicx}
\usepackage{enumitem}
%\usepackage{amssymb}
\usepackage{amsmath}
\usepackage{epstopdf}
\usepackage{setspace}
\usepackage{natbib}
\setcitestyle{aysep={}}
\usepackage{hyperref}
		
\synctex=1

\DeclareSymbolFont{symbolsC}{U}{txsyc}{m}{n}
\DeclareMathSymbol{\strictif}{\mathrel}{symbolsC}{74}
\DeclareMathSymbol{\boxright}{\mathrel}{symbolsC}{128}

\newenvironment{squote}{%
	\begin{quote}\begin{singlespace}%
	}{%
	\end{singlespace}\end{quote}}

\newcommand{\stager}[4]%
{%
	\begin{spacing}{1}%
	\vspace{0pt}
		\begin{description}[style=nextline, noitemsep,
                    parsep=0pt, topsep=0pt, leftmargin=15mm,
                    itemindent=-10mm, font=\mdseries]
			\item[\textsc{#1} \emph{#2}] #3
			\item[]%
			\begin{flushright}#4\end{flushright}
		\end{description}%
	\end{spacing}%
}

\newcommand{\stage}[3]%
{%
	\begin{spacing}{1}%
	\vspace{0pt}
		\begin{description}[style=nextline, parsep=0pt,
                    leftmargin=15mm, itemindent=-10mm, font=\mdseries]
			\item[\textsc{#1} \emph{#2}] #3
		\end{description}%
	\end{spacing}%
}

\newenvironment{inq}{\vspace{0pt}%
	\begin{list}{}%
	{\setlength\labelwidth{0pt}%
	\setlength\leftmargin{2.5\oddsidemargin}%
	\setlength\rightmargin{\leftmargin}}
	\begin{spacing}{1}
	\item[]%
	}{
	\end{spacing}
	\end{list}
	\vspace{10pt}
	%\noindent%
	}

\title{Unger's arguments}
\author{Alex Dunn}
\begin{document}
\ifstandalone
\maketitle
\begin{spacing}{1.5}
\fi

Peter Unger has presented several arguments that threaten the kind of
universalism I sketched in section \ref{universalism}.  The versions
of the sorites paradox that he presents make trouble for all vague
concepts, including those that Merricks relies on for his version of
nihilism.  Unger's `problem of the many', on the other hand, poses a
problem specifically for versions of universalism, including my own.

\section{Incoherence and pluralities}
\label{unger}
Like Peter van Inwagen and Trenton Merricks, Peter Unger has denied
the existence of all `ordinary things'---such things as ``tables and
chairs and spears\,\ldots swizzle sticks and
sousaphones\,\ldots\,stones and rocks and twigs, and also tumbleweeds
and fingernails'' (\citeyear[117]{unger1979}).  Merricks has a
different motivation for his nihilism than does van Inwagen, and Unger
has a different motivation again.  In fact, he has two different
motivations; one is the sorites paradox and the other is the problem
of the many.  

Unger claims that the sorites paradox shows that terms for ordinary
things, like `chair', are {\em incoherent}.  He claims that incoherent
terms cannot apply to anything in the world; therefore he concludes
that there are no chairs (or any other ordinary thing).

Unger's presentation of the problem of the many is aimed to trouble
the concept of `chair' in a different way.  The conclusion of that
argument is not that `chair' is necessarily incoherent.  Rather, the
conclusion is a disjunction: {\em either} there are no chairs, {\em
  or} there are a plurality (possibly an infinity) of chairs where we
would normally take there to be only one.

We will examine these two arguments in turn.

\section{Sorites paradoxes}
\label{sorites}
A typical instance of the sorites paradox begins by having us imagine
some ordinary object; let us use a heap of sand.  Now suppose we
remove a single grain of sand.  If we were inclined to believe that
the initial quantity of sand did in fact constitute a heap, then after
the removal of a single grain, we should presumably still have a heap
(albeit a slightly smaller one).  It seems very implausible to think
that one grain of sand more or less could {\em ever} make a difference
as to whether something is or is not a heap.

But having conceded (a) that there is a heap and (b) that the removal
of a single grain cannot make the difference as to whether a quantity
of sand is a heap, we have unwittingly put our foot in it.  For if the
removal of a single grain {\em never} transforms a heap into a
non-heap, then by repeatedly removing one grain after another, we will
eventually find ourselves with a heap that consists of no sand at
all.  But it seems absurd to suppose that there could be a heap of
sand that is composed of no sand---indeed, of nothing whatsoever.

This is the sorites paradox.  While a heap is a useful example,
because it is so ill-defined, similar problems appear to afflict all
ordinary things.  Unger illustrates the difficulty for stones:

\begin{squote}
Consider a stone, consisting of a certain finite number of atoms.  If
we or some physical process should remove one atom, without
replacement, then there is left that number minus one, presumably
constituting a stone still\,\ldots after another atom is removed,
there is that original number minus two; so far, so good.  But after
that certain number has been removed, in similar stepwise fashion,
there are no atoms at all in the situation, while we must still be
supposing that there is a stone there.  But as we have already agreed,
if there is a stone present, then there must be some atoms\,\ldots I
suggest that any adequate response to this contradiction must
include\,\ldots the denial of the existence of even a single
stone.~\citep[121--122]{unger1979}
\end{squote}
Unger understands this dilemma to apply across the board, and
correspondingly argues that we should deny the existence of even a
single ordinary thing.

\subsection{The sorites paradox in relation to Merricks' nihilism}
\label{sorites-m}
The purpose of Unger's argument is to show that terms that are
susceptible to the sorites paradox are incoherent and cannot apply to
anything in the world.  Unger claims that terms like `chair' therefore
cannot apply to anything in the world---from which it follows that
there are no chairs.

Trenton Merricks' version of nihilism (section \ref{merricks}) denies
that there are chairs.  In this, Merricks is in agreement with Unger.
But unlike Unger, Merricks maintains that beliefs like ``there are
chairs'' are {\em justified} and {\em nearly as good as true}.  He
claims that such beliefs are nearly as good as true if they are caused
by simples arranged chairwise.  Merricks does not believe that there
are chairs, so ``there are chairs'' is, strictly speaking, false.  But
Merricks believes that there are simples arranged chairwise, and the
presence of such arrangements cause and justify false beliefs such as
``there are chairs''.

We have seen, however, that the sorites paradox threatens the
coherency of concepts like `chair'.  If we suppose that a given
collection of atoms composes a chair, we can remove them one by one
and at no point feel justified in saying that the chair suddenly
ceases to exist.  Now suppose that a given arrangement of simples is
arranged chairwise.  Remove one.  Is the arrangement still arranged
chairwise?  Just as in the case of the chair, it seems bizarre to
think that a single simple (or atom) can make a difference as to
whether `chairwise' applies.  But now remove another atom\,\ldots

If Unger's argument shows that the concept of `chair' is incoherent,
then it seems that the same argument shows the concept of `chairwise'
to be incoherent.  If so, then `chairwise' can apply to nothing in the
world.  There can be no chairwise arrangements of simples.  If there
are no chairwise arrangements of simples, then our belief that there
are chairs cannot be nearly as good as true.  If someone believes that
there is a ghost, that belief, according to Merricks, is not nearly as
good as true because there are no ghostwise arrangements of simples to
cause or justify the belief that there is a ghost.  If `chairwise' is
incoherent, then there are no chairwise arrangements of simples to
cause or justify the belief that there are chairs.  The belief that
there are chairs is (if Unger is right) simply false, just as is the
belief that there are ghosts.

Merricks cannot allow this conclusion.  But the sorites paradox can be
used to show that {\em any} vague term is incoherent.  Much of our
language is vague, but we are not therefore tempted to conclude that
our speech is rarely (if ever) coherent.  The problem of the sorites
paradox is a very general problem that requires a general theory of
vagueness.  Most metaphysical theories are threatened by the sorites
paradox; Merricks is not a special case.

With that in mind, it should not be surprising that Merricks does not
have full answer.  A full answer to the sorites paradox would be a
theory of vagueness.  Merricks is not trying to establish a theory of
vagueness, but it attempting to motivate a metaphysical thesis about
ordinary things.  That said, he does have a {\em partial} answer.  He
points out that while the sorites paradox does threaten the concept of
`chairwise', it does so in a less troubling way than the way in which
it threatens `chair'.  

How is the sorites paradox ``less troubling'' for chairwise
arrangements than for chairs?  Very roughly, it is because accepting
the vagueness of `chair' can lead us to {\em metaphysical} vagueness,
while accepting the vagueness of `chairwise' can only lead us to {\em
  linguistic} vagueness.  And linguistic vagueness is generally
considered to be less troubling than metaphysical vagueness.

We can get a sense of how this is so by imagining, as Merricks does,
the sorites paradox being played out as a series of questions.  Let us
suppose that there is a chair.  We then ask ourselves (Merricks asks
God), ``is `there is a chair' true?''  Presumably we will answer
``yes''.  Then we remove one atom (or simple, or other thing) from the
chair.  ``Is `there is a chair' true?''

If we agree that there is no single atom whose removal would destroy
the chair, then we must accept that at some point it becomes
indeterminate whether ``there is a chair'' is true.  If it is
indeterminate whether ``there is a chair'' is true, then it is
indeterminate whether there is a chair.  The idea that it could be
indeterminate whether something exists is taken by many to be
problematic (\textbf{CITE}).

We can compare this case with that of the things arranged chairwise.
Let us suppose that there are things arranged chairwise.  We ask ``is
`there things arranged chairwise' true?'' and answer ``yes''.  Then we
remove a thing and ask ``is `there things arranged chairwise' true?''
As in the chair case, if we are unwilling to allow that the removal of
a single thing could take us from ``yes'' to ``no'', then we must
admit that at some point it is indeterminate whether or not ``there
are things arranged chairwise'' is true.  It would then be
indeterminate whether there are things arranged chairwise.  {\em But
  this does have the result that it is indeterminate whether something
  exists.}  It may be perfectly determinate that there are the things
there are; all that is indeterminate is whether the things (which
there determinately are) are arranged in a certain way.  This kind of
indeterminacy seems less troubling than indeterminacy as to whether
something (a chair or anything else) exists.

Merricks therefore sees the problem posed by the sorites paradox as
less threatening to his chairwise arrangements than to chairs.  He
does not attempt to provide a solution, for that would require solving
the problem of linguistic vagueness.  But he finds the problem of
linguistic vagueness less troubling than the problem of metaphysical
vagueness, and he shows that the latter does not threaten his version
of nihilism.

\subsection{So what's the problem?}
\label{sorites-3}
We find ourselves wanting to hold three theses, which appear mutually
inconsistent:

\begin{enumerate}
  \item There is at least one chair (stone, cloud).
  \item If a chair (stone, cloud) exists, it must be made up of
    matter.
  \item If a chair (stone, cloud), exists, the removal of a single
    molecule (or otherwise insignificant quantity of matter) from it
    cannot destroy it or cause it to cease to exist.
\end{enumerate}

We seem to be clearly caught in a paradox; the only question is where
we have gone wrong.

But have we, in fact, gone wrong?  Peter Unger thinks that we are
right on target:

\begin{squote}
While Eubulides' contribution has often been labeled `the sorites
paradox', there is nothing here which is a paradox in any
philosophically important sense\,\ldots Accepting our negative
conclusions here does not mean important logical trouble for us; we
only think we have troubles while we refuse to admit their validity
(\citeyear[145]{unger1979}).
\end{squote}

Our situation is only paradoxical, says Unger, while we unreflectingly
cling to the first thesis.  If, however, we come to see that there are
no chairs (stones, clouds), then we happily escape paradox: if there
are no chairs (stones, clouds) to begin with, we do not have to worry
about what the addition or removal of small amounts of matter would do
to them; nor do we need concern ourselves with what they would be made
of.

But things are not quite so simple.  First, adding to the
implausibility of Unger's view, he must deny that our use of ordinary
terms like `chair' (`stone', `cloud') follow any sort of pattern or
display any competence at all.  Second, even if we manage to swallow
that consequence, Unger has no explanation as to why we believe that
there are chairs (stones, clouds).

\subsection{Competence and correctness}
\label{correct}
Setting aside whether or not expressions of propositions like ``that's
a chair'' are ever \emph{true}, it seems right to say that there are
at least correct and incorrect uses of the terms.  For a word like
`chair' (`stone', `cloud') we generally do not say that a child has
learned how to use it until she is capable of deploying it in certain
ways.  We admit that she understands what `chair' (`stone', `cloud')
means or what a chair (stone, cloud) is when she displays a certain
competence with the term.  If instead of using `chair' to refer to
chairs she used it to refer to dogs or people, we would say that she
is confused and attempt to correct her use.

But Unger maintains that this is all an illusion, and that there is no
such thing as the correct or incorrect use of a term like `chair'
(`stone', `cloud'):

\begin{squote}
Concerning words and kinds, now, we might say this.  First, we might
say that it is in connection with \emph{semantics} that our reasonings have
what are their most obvious implications and, second, that their most
obvious semantic implications concern certain \emph{sortal nouns}, namely,
those which purport to denote ordinary things.  Thus, it appears quite
obvious to us now that there will be no application to things for such
nouns as `stone' and `rock', `twig' and `log', `planet' and `sun',
`mountain' and `lake', `sweater' and `cardigan', `telescope' and
`microscope', and so on, and so forth.  Simple positive sentences
containing these terms will never, given their current meanings,
express anything true, correct, accurate, etc., or even anything which
is anywhere close to being any of those things
(\citeyear[148]{unger1979}).
\end{squote}

This seems simply bizarre.  On what grounds, then, do parents correct
their children with respect to their use of ordinary terms?  Are they
compelled by some irrational force to consider certain utterances
correct and others incorrect?  One may question whether or not we use
ordinary term entirely consistently, but it seems simply false to say that,
necessarily, we {\em never} use (or have used, or will use) ordinary
terms in correct, as opposed to incorrect, ways.  

The fact that Unger's position means that terms like `chair' are never
used correctly gives us reason to think his argument goes awry
somewhere.  However, attempting to identify the issue with his
argument and proposing a solution would be tantamount to attempting to
solve the problem of vagueness.  That is not something I will attempt,
at least not until we have examined the problem of the many.

\section{The problem of the many}
\label{many}
The `problem of the many', as Unger terms this second difficulty for
ordinary things, follows a similar line of reasoning as that of the
sorites paradox.  If we consider an ordinary thing---a cloud, for
instance---it is natural to think that it is made up of molecules.
There is probably then a set of molecules, the members of which make
up the cloud.  Call that set $S$.  Now consider $S_1$.  This is a set
of molecules that includes all of the members of $S$ as well as one
additional molecule.  Do the members of $S_1$ make up a cloud?  Surely
they are just as well suited to do so.  Now consider $S_2$\,\ldots

Because these numerous 'candidates' are equally (or nearly equally)
well suited to be clouds, we seem forced to conclude that there are
either many clouds where we supposed there to be one, or rather no
clouds at all:

\begin{squote}
No matter where we start, the complex first chosen has nothing
objectively in its favor to make it a better candidate for cloudhood
than so many of its overlappers are.  Putting the matter somewhat
personally, each one's claim to be a cloud is just as good, no better
and no worse, than each of the many others.  And, by all odds, each
complex has \emph{at least} as good a claim as any still further real
entity in the situation.  So, either \emph{all} of \emph{them} make it
or else \emph{nothing} does; in this real situation, either there are
many clouds or else there really are no clouds at all
\citep[415]{unger1980a}.
\end{squote}

The problem of the many can also arise by considering the {\em
  boundary} of a given cloud.  It is natural to suppose that a cloud
has a determinate boundary.  But if we look at the edge of the cloud,
where we suppose the boundary to be, ``we may find, side by side, or
themselves overlapping, a great many potential boundaries for
clouds\,\ldots if our alleged typical item {[}the cloud{]} is indeed
a typical cloud, then many of these candidates, millions at least, do
not fail to be clouds altogether but are clouds of some
sort'' \citep[420--421]{unger1980a}.

The pattern of argumentation is the same for both approaches to the
problem of the many.  For a given cloud, a certain set of members or a
certain boundary is supposed, and it is argued that a set or boundary
that differs minimally from the original must also make up or bound a
cloud.  The new set or boundary does not appear to differ from the
original in any relevant way; there seems no principled reason to deny
that if the first set's members make up a cloud, the second set's
members do too.  And since there are a great deal of very similar sets
and boundaries, we find ourselves threatened with a plurality of
clouds.

And of course, Unger does not rest content with applying the problem
of the many to clouds.  All ordinary objects get the same treatment;
he concludes that either there are a great many of them, or there are
none at all.  He claims, predictably, that the latter disjunct is
preferable.

\subsection{Is the problem of the many a problem for Merricks?}
\label{many-merricks}
It might seem that the problem of the many, if it makes things
difficult for chairs and other ordinary things, also causes trouble
for the chairwise arrangements upon which Merricks relies so heavily.
But there is an important disanalogy.  The problem of the many is only
a problem as long as we are unwilling to accept one of Unger's
disjuncts: that there are no chairs or that there are a plurality
where we took there to be one.  Whether or not it is part of the
meaning of `chair' that there is not an overlapping plurality of
chairs, it is simply unacceptable that this be the case.  (It is
likewise unacceptable that there be no chairs.)  But it {\em is}
acceptable, at least initially, that there be a plurality of chairwise
arrangements.  The idea that there is a plurality of different sets,
the members of which overlap and of which all are arranged chairwise,
is not particularly bizarre.  A potential difficulty for Merricks
would be in regard to his criterion of `arranged chairwise' (or
statuewise, as the case may be):

\begin{squote}
Atoms are \emph{arranged statuewise} if and only if they both have the
properties and also stand in the relations to microscopica upon which,
if statues existed, those atoms' \emph{composing a statue} would
non-trivially supervene (\citeyear[4]{merricks2001a}).
\end{squote}

If Merricks allows that there may be a plurality of statuewise
arrangements, then he is committed to the proposition that, if statues
existed, there may be pluralities of statues.  But Merricks may simply
take this as more evidence that his counterpossible conditional (``if
there were statues\,\ldots '') really is impossible.

There may be, however, a problem for Merricks with regard to the
notion of `singular thought'.  If Merricks says to me, ``those things
arranged chairwise are arranged very comfortably'', how can I know
which things he is talking about?  If there are numerous different
sets of things arranged chairwise, the chance that I am thinking of
the same things as Merricks is very low.  Are we really communicating,
then?  (Moreover, is it true that I am thinking of {\em any}
determinate set of things?  I certainly couldn't specify which
particular things I am thinking of.)

If the problem of the many is a problem for Merricks, then so much the
worse for his nihilism.  However, the problem is a problem for us as
well.

\section{Beliefs in things}
\label{u-belief}
When we examined previous versions of nihilism, we asked that their
proponents explained why, if there are no chairs, we nonetheless
believe that there are chairs.  Van Inwagen and Merricks both made a
claim to the effect that our beliefs are caused and (in some sense)
justified by arrangements of simples.  Although Merricks denies that
beliefs like ``there is a fine chair'' are strictly true, he agrees
with van Inwagen that they `get something right' in a way that beliefs
like ``there is a dancing chair'' do not.  Their explanation for our
belief that there are chairs is that they are caused (and justified)
by a `nearby' or somehow related fact---that there are simples
arranged chairwise.

It should therefore seem reasonable to demand a similar explanation
from Unger.  This explanation would be expected to take a different
form, depending on whether it accompanies the sorites paradox or the
problem of the many.  However, Unger offers no explanations.  

\subsection{Explaining our beliefs given the sorites paradox}
\label{expl-sorites}
Unger claims that our belief that there are chairs, like our beliefs
that there are other ordinary things, are not justified, `nearly as
good as true', or even coherent.  Unger says that ``terms for ordinary
things are incoherent [and] cannot apply to anything real''
\citep[147]{unger1979}.

Unger should not deny that believe that there are ordinary things.  If
our beliefs about tables and chairs are invariably false (even
incoherent), then what causes us to form these beliefs?  Why do we
believe in ordinary things to begin with?

Unlike van Inwagen and Merricks, Unger does not offer an explanation.
Having denied the existence of all ordinary things, he makes no
attempt to explain why we have so many false beliefs or what gives the
impression of coherency to our use of them in communication.  He seems
almost to revel in the strangeness of his position:

\begin{squote}
Now, it must of course be admitted that these arguments undermine the
possibility of any endeavor I should try to propose, or even the
putative thought that I should propose anything, just as all of my
putative essay is undermined.  But even so, I shall (incoherently)
propose that what we have now to do is invent new expressions which
are not inconsistent ones, and by means of which we may, to some
significant extent, think coherently about concrete reality
(\citeyear[544]{unger1980b}).
\end{squote}

I do not have an argument against the proposition that nearly all of
our language is hopelessly incoherent.  I do have a very hard time
believing that this is true, however; I'm not sure Unger believes it
himself.

\subsection{Explaining beliefs given the problem of the many}
\label{expl-many}
When presenting the problem of the many, Unger declares that he
prefers to maintain that there are no chairs, rather than that there
are pluralities of chairs.  Having made this nihilistic claim,
however, he does not offer an explanation of why we do in fact believe
that there are chairs.  However, he does not deny (at least not
explicitly) that there are things arranged chairwise, so Merricks'
explanation (see section \ref{connection}) might serve for Unger too.
We might propose, on Unger's behalf, that we believe that there are
chairs because there are things arranged chairwise.

In this case, however, I am tempted to repeat my arguments against
Merricks (section \ref{dogbush}).  I claimed that if there are things
arranged chairwise, {\em then there are chairs}.  If our belief that
there are chairs is caused by things arranged chairwise, then it is a
true, not a false, belief.

If we reject Unger's conclusion that there are no chairs, we are still
faced with a problem.  For we seem to run ourselves into the other
disjunct of the problem of the many.  If there are chairs, then there
are pluralities of chairs where we expect there to be only one.

This is an unacceptable conclusion, but not due to any explanatory
deficiency.  If we supposed that there was a plurality of chairs, then
{\em that} would explain why we believe that there are chairs.

\section{Referring to the many}
\label{refer}
Above (section \ref{many-merricks}) I mentioned that notions of
singular thought are threatened by the problem of the many.  More
should be said on this.

\ifstandalone
\bibliography{everything}
\bibliographystyle{ChicagoReedweb}
\end{spacing}
\fi
\end{document}


\documentclass[11pt]{standalone}
\usepackage{standalone} \newif\ifstandlone \standalonetrue
\usepackage[left=1.75in, right=1.75in, top=1.25in, bottom=1.25in]{geometry}
\geometry{letterpaper}
\usepackage{verbatim}
\usepackage{graphicx}
\usepackage{enumitem}
%\usepackage{amssymb}
\usepackage{amsmath}
\usepackage{epstopdf}
\usepackage{setspace}
\usepackage{natbib}
\setcitestyle{aysep={}}
\usepackage{hyperref}
\usepackage{url}
\synctex=1

\DeclareSymbolFont{symbolsC}{U}{txsyc}{m}{n}
\DeclareMathSymbol{\strictif}{\mathrel}{symbolsC}{74}
\DeclareMathSymbol{\boxright}{\mathrel}{symbolsC}{128}

\newenvironment{squote}{%
\begin{spacing}{1}
       	\begin{list}{}{%
\setlength{\labelwidth}{0pt}%
\rightmargin\leftmargin%
}
\item\relax
}{%
\end{list}%
\end{spacing}
}

\title{1.25}
\author{Alexander A. Dunn}
\begin{document}
\ifstandalone
\maketitle
\begin{spacing}{1.5}
\fi

\section{Paraphrases}
I have proposed that any attempt to deny the existence of ordinary
things such as tables and chairs must be supplemented by an
explanation as to why we believe in the existence of ordinary things.
As we will see, Peter Unger's claim that there are no ordinary things
and his claim that propositions like ``that is a chair'' are uniformly
false leaves it quite mysterious why we take there to be chairs in the
first place.  Some nihilistic philosophers, therefore, have attempted
to maintain Unger's first thesis---that there are no ordinary
things---while rejecting the second---that ordinary thing discourse is
invariably false.  Such a philosopher will claim that such discourse
is {\em compatible} with the nonexistence of chairs.  This may involve
the claim that while we take ourselves to have beliefs about chairs
and other ordinary things, our beliefs do not actually concern such
(non-existent) entities.  Rather, our thought and talk is (or should
be seen as) relating to such things as do exist.  Strategies that
follow this pattern can be called paraphrasing strategies, and Peter
van Inwagen has presented a well-known version.

Like Unger, van Inwagen claims that (necessarily) there are simply no
such things as tables and chairs in the world.  But unlike Unger, he
does not claim that when we take ourselves to be thinking and talking
about such things, we are thinking and talking about nothing at all.
At least, ``when people say things in the ordinary business of life by
uttering sentences that start `There are chairs\,\ldots ' or `There
are stars\,\ldots ', they very often say things that are literally
true'' \citep[102]{inwagen1995}.  

One might assume that if such statements are true, then it follows
that there are chairs and stars.  But van Inwagen denies that chairs
and stars exist.  How can he claim, then, that what was said was true?
What van Inwagen does is attempt to show that the statements in
question can be {\em paraphrased}---they can be reformulated to show
that they have no `ontological commitments'.  According to van
Inwagen, one can assert that there is a chair without being committed
to the existence of chairs.

Section~\ref{comp} will summarize the motivation for van Inwagen's
denial.  Section~\ref{inwagen} will introduce and criticize van
Inwagen's paraphrasing strategy.

\subsection{Composition}
\label{comp}
Van Inwagen's conclusion that there are no chairs is a consequence of
his views on {\em composition} (or `constitution').  Some things are
said to compose another thing if the former are {\em parts} of the
latter; the latter is `made up of' the former.  Van Inwagen believes
that ``the metaphysically puzzling features of material objects are
connected in deep and essential ways with metaphysically puzzling
features of the constitution of material objects by their
parts''~\citep[18]{inwagen1995}.  An example of such a puzzle is the
Ship of Theseus.  The Ship of Theseus is (presumably) an object
composed of many parts, including planks of wood.  As the planks (and
other parts of the ship) wear out, they are replaced.  These
replacements happen each by themselves; the entire ship (or even a
large section) is not replaced all at once.  But eventually no part of
the original ship remains; it is build of entirely different planks,
nails, rigging, etc.  And yet we would commonly say that it is still
the same ship.  But why should we think that the present ship is
identical with a past ship with which it shares no parts?

\subsection{The Special Composition Question}
\label{scq}
Answering the question `why is this ship identical with that past
ship?' requires first figuring out why (and how) these planks and
rigging and sails (et.\ al.) compose a ship in the first place.  Van
Inwagen asks ``in what circumstances do planks\footnote{For
  simplicity's sake, van Inwagen ignores the rigging and sails.}
compose (add up to, form) something?'' (\citeyear[21]{inwagen1995}) 
For some $x$s, then, van Inwagen asks us to consider when
\begin{equation}
\exists y\ \text{the}\ x\text{s compose}\ y
\end{equation}
is true.%
\footnote{Van Inwagen explains in some detail how plural referring
  expressions (like ``the planks'') can be given a logical
  formalization (\citeyear[23--28]{inwagen1995}), but suffice to say
  they work just as one would expect.}
%
\ Less formally, van Inwagen asks: ``suppose one had certain
(nonoverlapping) objects, the $x$s, at one's disposal; what would one
have to do---what {\em could} one do---to get the $x$s to compose
something?'' (\citeyear[31]{inwagen1995})  This is the Special
Composition Question.

(`Composition' is used in a technical sense with regard to the Special
Composition Question.  Van Inwagen defines it thus: ``the $x$s compose
$y$'' means that ``the $x$s are all parts of $y$ and no two of the
$x$s overlap and every part of $y$ overlaps at least one of the
$x$s\,\ldots\,a thing {\em overlaps} a thing---or: they overlap---if
they have a common part'' (\citeyear[29]{inwagen1995}).  For van
Inwagen, everything is a part of itself; some $x$ is a {\em proper}
part of some $y$ only if $x \neq y$.)

\subsection{The usual answers}
\label{scq-ans}
There are several prominent answers to the Special Composition
Question, including the following (These formulations are from
\citet{markosian1998a}):
\begin{description}
	\item[Nihilism] Necessarily, for any $x$s, there is an object
          composed of the $x$s iff there is only one of the $x$s,
          i.e., the only objects that exist are
          simples (\citeyear[219]{markosian1998a}).
	%
	\footnote{\label{flip} Note that this may not be Unger's view.
          He denies that people, apples, cheese, tables, chairs, and
          other ``ordinary things'' are nonexistence but he does not,
          as far as I know, take a stand on whether anything at all
          exists.  His view can be (flippantly) summarized thus: ``if
          we have a word for it, it doesn't exist.''}
	%\footnote{\label{gunk} Of course, it may be that the world is
	%not fundamentally particulate, and is filled not with simples
	%but with `gunk'; see \citet{schaffer2003}.  Nihilism (and van
	%Inwagen's second condition below) can be formulated to take
	%this possibility into account: ``for any quantity of gunk,
	%there is nothing composed of it.''}%
	%
	\item[Universalism] Necessarily, for any $x$s, there is an
          object composed of the $x$s iff no two of the $x$s
          overlap (\citeyear[227]{markosian1998a}).
	\item[Van Inwagenism] Necessarily, for any $x$s, there is an
          object composed of the $x$s iff either (i) the activity of
          the $x$s constitutes a life or (ii) there is only one of the
          $x$s (\citeyear[221]{markosian1998a}).
\end{description}

We will discuss Unger's version of nihilism in section \ref{unger}.
As I will argue, any version of nihilism that does not explain our
beliefs in the existence of ordinary objects is problematic.

Universalism raises a number of problems in relation to Peter Unger's
`problem of the many' (see section \ref{many}).  I will therefore
postpone discussion of this view until later.

Van Inwagen examines and rejects universalism and the version of
nihilism given above.  He also rejects a number of other answers to
the Special Composition Question.  Some are too strong: `some $x$s
compose a $y$ iff the $x$s are in contact' would entail that two
people shaking hands will result in a new object coming into being.
Others are too strong in some ways and too weak in others: `some $x$s
compose a $y$ iff the $x$s are fastened together' would entail that
two people being glued together would result in a new object; and it
would deny that an object can be composed without fastening its parts
together (such as when building a house of cards).  The only answer
van Inwagen finds consistent is what we have dubbed {\em van
  Inwagenism}, which entails that tables and chairs do not exist.

Because of this consequence, van Inwagenism should include an
explanation why we nonetheless believe that there are tables and
chairs.  Happily, van Inwagen recognizes this and is prepared with a
{\em paraphrasing strategy}.  This strategy aims to show that the
beliefs that we take to be about tables and chairs are really about
something else, and are not true beliefs.  If such beliefs are true,
then it should be relatively easy to explain why hold them: they are
true, and we learn of them through some reliable means (like our
eyes).

Unfortunately, van Inwagen's paraphrasing strategy does not work.

\section{Van Inwagen's paraphrasing strategy}
\label{inwagen}
Peter Unger maintains that terms like `chair' are incoherent; if this
is so, a statement involving the phrase `There is a chair\,\ldots '
could surely not be true.  Van Inwagen, on the other hand, admits that
``when people say things in the ordinary business of life by uttering
sentences that start `There are chairs\,\ldots ' or `There are
stars\,\ldots ', they very often say things that are literally true''
\cite[102]{inwagen1995}.  It does not seem unreasonable to assume that
if what people say with ``There are chairs\,\ldots '' and the like are
true, then chairs exist.  But van Inwagen denies this entailment.

How can van Inwagen maintain this?  Someone can say, truly, ``There
are simples arranged chairwise\,\ldots '' without committing oneself
to the existence of chairs.  Van Inwagen might then claim that when
someone says ``There is a chair\,\ldots '' she {\em means} `There are
simples arranged chairwise.'  This is, of course, a bold hypothesis
about the speech practices of ordinary speakers.  Certainly very few
speakers would, if asked, affirm that what they meant to say had
anything to do with simples; they would say that when they said that
there was a chair, they meant just that.  Van Inwagen recognizes that
this is not a viable position: ``The only thing I have to say about
what the ordinary man really means by `There are two valuable chairs
in the next room' is that he really means that there are two valuable
chairs in the next room'' (\citeyear[106]{inwagen1995}).

One might then assume that van Inwagen is thinking in analogy with
Russell.  He could attempt to claim that, despite the surface
appearance of language (`There is a chair\,\ldots '), the underlying
logical form does not make any mention of chairs (or tables); the
offending concept is analyzed away, leaving `There are simples
arranged chairwise\,\ldots '.  Van Inwagen notes that his ``suggested
technique of paraphrasing enables us to escape some of the more
embarrassing consequences of this position.  When someone says `Some
tables are heavier than some chairs,' there is obviously something
right about what he says.  Our technique of paraphrasis enables us to
capture what it is that is right about what he says''
(\citeyear[111]{inwagen1995}).  However, on the very next page he
admits that the ordinary language proposition and his paraphrased
version are different propositions: ``When the ordinary man utters the
sentence `Some chairs are heavier than some tables' (in an appropriate
context, and so on and so on), he expresses a certain proposition, and
one that is almost certainly true.  But I do not claim that this
proposition {\em is} the proposition that, for some $x$s, those $x$s
are arranged chairwise and for some $y$s, those $y$s are arranged
tablewise, and the $x$s are heavier than the $y$s''
(\citeyear[112]{inwagen1995}).  So van Inwagen is not making an appeal
to some notion of `logical form'.  But then what is the purpose of the
paraphrasing project?

Van Inwagen attempts to justify his method of paraphrasis by asserting
the following parallels between the original and paraphrased
propositions:
\begin{enumerate}[label=(\Alph*)]
	\item The paraphrase describes the same fact as the
          original.  \label{para-a}
	\item The paraphrase, unlike the original, does not even
          appear to imply that there are any objects that occupy
          chair-receptacles.  \label{para-b}
	\item The paraphrase is neutral with respect to competing
          metaphysical theories, {\em viz}.  the ``received'' theory,
          that there are objects that occupy chair-receptacles, and
          the theory I have proposed, according to which there are no
          such objects.  \label{para-c}
	\item The original, though it doubtless does not express the
          same proposition as the paraphrase, has the feature ascribed
          to the paraphrase in \ref{para-c}: It is neutral with
          respect to the question whether there are objects that fit
          exactly into
          chair-receptacles~(\citeyear[113]{inwagen1995}).  \label{para-d}
\end{enumerate}

I am rather dubious as to the truth of \ref{para-a}, but I am quite
sure that \ref{para-d} is false, and van Inwagen's thesis appears to
depend on it.  He admits in \ref{para-b} that the original sentence
(`There are chairs\,\ldots ') {\em implies} that there are chairs, but
claims in \ref{para-d} that it does not {\em entail} this.  But why
wouldn't it?

\subsection{Propositions and ontological commitment}
Let us review the situation.  First, van Inwagen agrees that when
someone says things like ``There is a chair\,\ldots '' they mean just
that.  Second, he admits that his `paraphrases' of such propositions
are not so faithful to the original that they can be called the same
proposition; the original and the paraphrase are two different
propositions.  Third, he claims nonetheless that {\em neither} the
original nor the paraphrase entail the existence of chairs.

This seems obviously untrue.  How can he claim that when someone says
``There is a chair\,\ldots '' and means just that, that the
proposition they express does not entail the existence of chairs?  To
defend his claim, van Inwagen appeals to his `Copernican analogy':

\begin{squote}
I accept the Copernican Hypothesis.  One day you hear me say, ``It was
cooler in the garden after the sun had moved behind the elms.''  You
say, ``You see, you can't consistently maintain your Copernicanism
outside the astronomer's study.  You say that the sun moved behind the
elms; yet, according to your official theory, the sun does not move.''
I reply that the proposition I expressed by saying ``It was cooler in
the garden after the sun had moved behind the elms'' is consistent
with the Copernican Hypothesis (\citeyear[101]{inwagen1995}).
\end{squote}
That is, van Inwagen claims that the proposition he expressed with
``It was cooler in the garden after the sun had moved behind the
elms'' does not entail that the sun actually moved.  And he argues
that this is analogous to our talk of chairs: most propositions
expressed with ``There is a chair\,\ldots '' do not entail that chairs
actually exist.

First, does the proposition van Inwagen expresses with ``The sun moved
behind the elms'' entail that the sun moved? I am inclined to say that
it does.  If I were to say simply ``The sun moved'' (meaning just
that), I think I would have committed myself to the movement of the
sun.  Why should we think that the addition of `behind the elms'
defeats this entailment?  Without some explanation of what the
difference is, I see no reason to think that saying ``The sun moved
behind the elms'' (and meaning it) does not entail the movement of the
sun.  But van Inwagen may be forced to say here that neither
proposition entails that the sun moved.  For he certainly won't allow
that either entails that the sun {\em exists}.

There is an analogy here, though perhaps not the one van Inwagen had
in mind.  He claims that a proposition expressed by ``There are two
very valuable chairs in the next room'' does not necessarily entail
the existence of chairs.  If this proposition does not entail that
chairs exist, then what about `There are two valuable chairs left in
the world' or `There are at least two chairs in the world' or `There
are at least two chairs' or simply `There are chairs'?  Van Inwagen
appears committed to the claim that the proposition I would express
with ``There are chairs'' does not entail that there are chairs.

Why on earth should this be?  Does not the proposition expressed by my
saying ``There are simples arranged chairwise\,\ldots '' entail the
existence of simples?  If van Inwagen says that there are simples
arranged chairwise, and means just that, then it would appear to
follow that there are simples.  Indeed, van Inwagen's argument relies
rather heavily on the assumption that simples exist.
%
%% \footnote{Ted Sider takes him to task for this
%%   assumption~(\citeyear{sider1993}), claiming that the possibility of
%%   `gunk'---the possibility that the matter of the world is not
%%   fundamentally particulate but infinitely divisible---falsifies van
%%   Inwagen's thesis.  I think it may be possible for van Inwagen to
%%   adapt to a gunky world (he might be able to claim that nothing
%%   exists but organisms, who are composed of other organisms and/or
%%   gunk), but I think van Inwagen's thesis is false either way.}
%
\ But if `There are chairs' does not entail that there are chairs and
if `The sun moved behind the trees' entails neither that the sun moved
nor that the sun exists, then how can van Inwagen maintain that `There
are simples arranged chairwise' entails that there are simples, or
that they are arranged chairwise?  He has given us no reason to
believe one and not the other.

\subsection{Loose truth}
\label{loose-v}
Van Inwagen's attempt to maintain that there are no chairs and that
``there are chairs'' is true does not appear promising.  But he may
instead claim that ``there are chairs'' is not true, but {\em loosely
  true}.  He admits this position as a possibility:

\begin{squote}
I can say this [that `There are chairs\,\ldots ' can be true yet not
  entail that there are chairs] because I accept certain theses in the
philosophy of language.  I can say this because I accept certain
theses in the philosophy of language.  Some people, I suppose, would
reject these theses.  These people would say that when I
said\,\ldots\ The sun moved behind the elms,' I said something
false\,\ldots If someone maintains that `The sun moved behind the
elms' expresses a falsehood, he must still have some way to
distinguish between this sentence and those sentences (like `The sun
exploded' and `The sun turned green') that the vulgar would regard as
the sentences that expressed falsehoods about the sun\,\ldots [if I
  took this position,] I should not be willing to say that people who
uttered things like `There are two valuable chairs in the next room'
very often said what was true.  I should be willing to say only that
they very often say what might be treated as a truth for all practical
purposes (\citeyear[102--103]{inwagen1995}).
\end{squote}

Van Inwagen admits that `There are two very valuable chairs in the
next room', ``when it is successfully used to report a fact, does
report a fact about the existence of {\em something}''
(\citeyear[102]{inwagen1995}).  Presumably van Inwagen thinks that
`something' is the chairwise arrangements of simples.  Van Inwagen
must therefore explain how a chairwise arrangement of simples can
cause people to believe that there are chairs.  Van Inwagen attempts
to bolster his case by drawing an analogy with the imaginary bliger.
The bliger, according to van Inwagen, is what happens when a monkey,
four owls, and a tiger attach themselves together temporarily.  The
conglomeration appears to the untrained observer to be a single
animal, and gullible farmers dubbed it a `bliger'.  Van Inwagen's
point is that there are no bligers, but that a farmer saying ``there's
a bliger'' when pointing at such a conglomeration would be saying
something loosely true.  The fact being reported by ``there's a
bliger'' is the fact that a monkey, four owls, and a tiger are there.
People believe that there are bligers because they mistake the group
of animals for a single thing, which has been dubbed `bliger'.

Likewise, van Inwagen maintains that people mistake chairwise
arrangements of simples for chairs.  When someone says ``there's a
chair'' what she says is loosely true because there is a chairwise
arrangement of simples there.  People believe that there are chairs
because they mistake the group of animals for a single thing, which
has been dubbed `chair'.

I agree with van Inwagen that these cases are analogous.  However,
where van Inwagen takes this analogy to show that there are no chairs,
I take it to show that there {\em are} bligers in van Inwagen's
imaginary scenario.  When it is discovered that bligers are built up
from six other creatures, we are learning something about bligers:

\begin{squote}
\ldots {\em of course} there are bligers in [van Inwagen's] story.
Bligers are what the story is about.  The zoologists do not report
that there are no bligers.  Rather they tell us what a bliger is.
They explain that a bliger is not a single large carnivorous animal
but a transient symbiotic union of six animals
\citep[704]{rosenberg1993}.
\end{squote}

In short, van Inwagen's analogy does not provide us with an
explanation of why we would believe in chairs even if there were none.
Just as the Ungerian treatment of loose truth failed to explain our
beliefs in chairs, van Inwagen's appeal to loose truth does not
explain why we believe that there are chairs, rather than chairwise
arrangements of simples.

\ifstandalone
\end{spacing}
\fi
\end{document}


\documentclass[11pt]{standalone}
\usepackage{standalone} \newif\ifstandlone \standalonetrue
\usepackage[left=1.75in, right=1.75in, top=1.25in, bottom=1.25in]{geometry}
\geometry{letterpaper}
\usepackage{graphicx}
%\usepackage{tipa}
%\usepackage{exaccent}
%\usepackage{txfonts}
%\usepackage{pxfonts}
\usepackage{enumitem}
%\usepackage{amssymb}
\usepackage{amsmath}
\usepackage{epstopdf}
\usepackage{setspace}
\usepackage{natbib}
\setcitestyle{aysep={}}
\usepackage{hyperref}
\usepackage{url}
\synctex=1

\DeclareSymbolFont{symbolsC}{U}{txsyc}{m}{n}
\DeclareMathSymbol{\strictif}{\mathrel}{symbolsC}{74}
\DeclareMathSymbol{\boxright}{\mathrel}{symbolsC}{128}

                \newenvironment{squote}{%
\begin{spacing}{1}
       	\begin{list}{}{%
\setlength{\labelwidth}{0pt}%
\rightmargin\leftmargin%
}
\item\relax
}{%
\end{list}%
\end{spacing}
}

\title{Vague terms and some other stuff}
\author{Alexander A. Dunn}
\begin{document}
\ifstandalone
\maketitle
\begin{spacing}{1.5}
\fi

\section{The tolerance of ordinary terms}
There is a feature of ordinary terms that is exploited by the sorites
paradoxes and by the problem of the many.  This feature is what
Crispin Wright calls ``tolerance''.  A term is tolerance if, given
that it applies to a certain situation, it would also apply to a
situation minutely (or indiscriminably) different from the first
situation.  `Stone' is tolerant because, for any given object to which
`stone' applies, we can remove a single atom (or even a larger speck)
from that object and remain justified in applying the term to the
slightly smaller object.

Why should we think that `stone' applies to the slightly smaller
object?  If we could apply the term to the first object but withhold
it from the second (slightly smaller) object, we would not be so
easily ensnared by the sorites paradox.

What compels us to apply the term to both objects is the pressure of
consistency.  Because there is no relevant difference between the two
objects that would justify our application of `stone' to one and not
the other, we feel that the term must therefore apply to both.  We
imagine that our language-use must be consistent and {\em regular}:

\begin{squote}
We suppose our use of language to be fundamentally {\em regular}; we
picture the learning of language as the acquisition [or] grasp of a
set of rules for the combination and application of expressions
\citep[326]{wright1975}.
\end{squote}

This is the first of two assumptions that are required by
sorites-style arguments.  The second is that we can discover these
(consistent, regular) rules of language-use by examining our language
``from within'':

\begin{squote}
The question now arises, what means are legitimate in the attempt to
discover features of the \emph{substantial} rules for expressions in
our language, the rules which determine specifically the senses of
such expressions?  The view of the matter with which we are centrally
concerned in this paper is that we may legitimately approach our use
of language from within, that is, reflectively as self-conscious
masters of it, rather than externally, equipped only with behavioural
notions.  We may appeal to our conception of what justifies the
application of a particular expression; we may appeal to our
conception of what we should count as an adequate explanation of the
sense of a particular expression; to the limitations imposed by our
senses and memories on the kind of instruction which we can actually
implement; and to the kind of consequence which we attach to the
application of a given predicate, to what we conceive as the point of
the classification which the predicate effects.  The notion that forms
the primary concern of this paper---henceforward referred to as the
\emph{governing view}---is that we can derive from such considerations
a reflective awareness of how expressions in our language are
understood, and so of the nature of the rules which determine their
correct use.  The governing view, then, is a conjunction of two
theses: that our use of language is properly seen, like a game, as an
activity in which the allowability of a move is determined by rule,
and that properties of the rules may be discovered by means of the
sorts of consideration just described. \citep[327]{wright1975}.
\end{squote}

Wright uses the term `red' to illustrate how this second assumption
functions to generate sorites paradoxes.  If we reflect on how the
term `red' is taught, learned and used, we see that it is defined
\emph{ostensively}.  We learn what `red' means by being shown red
things, and we cannot be said to understand the term unless we can,
for example, point out the red tiles on a quilt of several colors.  It
therefore appears that the criteria for applying the term `red' are
observational: if something looks red then, as long as one is not
deceived by strange lighting, `red' applies to that thing.  But
because my vision is far from perfect, I am unable to distinguish very
subtle differences in color.  Two patches of red might look identical
to me, but be slightly different shades.  But because \emph{I} cannot
tell them apart, and because the application-criteria for `red' are
observational, `red' will presumably apply to both patches of color,
if it applies to either.

But now suppose there is a long series of color patches on a wall; the
leftmost one is definitely red and the rightmost definitely orange.
There are enough patches in between that the difference in color
between any two adjacent patches is indiscernible to a human observer.
Now if `red' applies to the leftmost patch, then it applies to the
patch immediately to the right; the two patches are indistinguishable
in terms of their color.  But if `red' applies to this second patch,
then it applies to the third, because \emph{those} are
indistinguishable.  Eventually we will find ourselves applying `red'
to the rightmost patch, which, by stipulation, is \emph{not} red but
orange.  The sorites paradox is established.

The arguments involving stones and other things can be understood as
involving these two assumptions as well.  `Stone', `chair' and other
ordinary terms are also defined ostensively; the application-criteria
for these terms are observational.  `Observational' may over-emphasize
the role of vision; the other senses play a role too.  For example,
here Austin lists some application-criteria for `telephone':

\begin{squote}
\ldots\,you tell me there's a telephone in the next room, and,
(feeling mistrustful) I decide to verify this\,\ldots\,I go into the
next room, and certainly there's something there that looks exactly
like a telephone.  But is it a case perhaps of \emph{trompe l'oeil}
painting?  I can soon settle that.  Is it just a dummy perhaps, not
connected up and with no proper works?  Well, I can take it to pieces
a bit and find out, or actually use it to ring somebody up---and
perhaps get them to ring me up too, just to make sure.  And of course,
if I do all these things, I \emph{do} make sure; what more could
possibly be required?  This object has already stood up to amply
enough tests to establish that it really is a telephone; and it isn't
just that, for everyday or practical or ordinary purposes, enough is
\emph{as good as} a telephone; what meets all these tests just
\emph{is} a telephone, no doubt about it \citep[118--119]{austin1964}.
\end{squote}

Of course, if we slowly remove bits of matter from the telephone,
there will probably be a point at which it suddenly stops working;
there may be a sharp cut-off between a working telephone and a broken
one.  But we could theoretically construct a case in which this is not
so; perhaps the volume is decreased by an indiscriminable amount
before finally cutting out completely.  By the end of this process the
phone is obviously not functional, but when should we say that it
stops working?  A sorites paradox can, it seems, be constructed for
most (if not all) terms that rely on observational criteria of
application.

(The two theses of the governing view are behind the success of the
problem of the many as well.  Because we cannot distinguish between
the various sets and various boundaries of objects, we must (if we are
to be consistent) apply `cloud' and the other terms to each of the
candidates.)

\section{Getting out}
It appears that to escape the sorites paradoxes as well as the
problems of the many, we will have to reject or revise one or both of
the theses of the governing view.  Much of what follows will be
examinations of various strategies.

(The solution to the sorites paradox and the problem of the many is
not to deny that there are, in fact, chairs.  Instead we must look at
the two assumptions of the governing view, and decide what is to be
done with them.)

\subsection{Working within the constraints of the governing view}
I think certain brands of supervaluationism might be attempting to
solve the sorites paradox from within the constraints of the governing
view.  These versions of supervaluationism rely on our ability to make
vague terms precise.  `Red', for instance, might, on a certain
precisification, be defined as a hue within a certain range of patches
on a color-chart.  Then the application-criteria for `red' will no
longer be observational.  Whether or not `red' applies in a given
situation will not depend on what something looks like, but where it
falls on the chart.

This might make our use of color-terms less convenient.  If we really
want to know whether or not something is red, we can't just trust our
senses; we have to consult the chart.  But this is similar to what
goes on with our use of terms for length.  We may ultimately have to
appeal to a ruler to determine whether or not `more than 1 foot'
applies to a given situation, but that does not mean that we cannot
often tell without a ruler.  It is only with regard to the borderline
cases that we must pull out our rulers (and color-charts).

But

\begin{squote}
the possibility of our dispensing with paradigms [rulers and
  color-charts] for most practical purposes depends upon our capacity
e.g. to distinguish between cases where we could tell whether or not
`red' applied just by looking and cases where we could not, where we
should have recourse to a chart.  If we are able to make such a
distinction, what objection can there be to introducing a predicate to
express it?  But then, it seems, the semantics of this predicate will
have to be observational \citep[359]{wright1975}.
\end{squote}

A predicate like ``looks as if it were red'' cannot be made precise.
It is specifically introduced to rely on observational
criteria---whether or not something {\em looks} red.  Now this
predicate is susceptible to the sorites paradox.  Recall the series of
color-patches, moving indiscriminably from red to orange.  The
leftmost (red) patch certainly looks as if it were red, and the
rightmost (orange) patch certainly does not look as if it were red.
But there is no point in between at which we can say that the patches
cease to look as if they were red.  So if we continue to hold both
theses of the governing view, we are forced to conclude that this
predicate is incoherent.  It seems that we must, after all, reject one
of the theses of the governing view.

\subsection{The second thesis}
Our options:
\begin{enumerate}
  \item Epistemicism seems to involve a rejection of the second thesis
    of the governing view.  It involves the claim that the meanings of
    vague terms like `heap' and `red' {\em are} precise, but that we
    do not (indeed, cannot) know exactly where their applicability
    ends.  So there is no question of deducing their
    application-criteria from the sort of internal examination that is
    called for by the second thesis.  This is an interesting view, but
    does seem pretty implausible that our (probably inconsistent) use
    of vague terms determines their meaning with such precision.
    Unger judgement as to what is and is not absurd may not be the
    most reliable, but I think he might be right here:

    \begin{squote}
      Does anyone imagine that our concept of a swizzle stick
      discriminates at the required atomic level?  Surely, this is
      quite absurd.  But, then, it is just as absurd in the case of
      tables, and of stones (\citeyear[126]{unger1979}).
    \end{squote}

  \item One might attempt to apply Markosian's theory of `brutal
    composition' to semantics.  On his theory of mereology
    (\citeyear{markosian1998a}), whether or not certain objects
    compose another object is a brute, unexplained fact---there is no
    finite, non-trivial answer to van Inwagen's Special Composition
    Question.  Correspondingly one might claim that whether or not a
    term (like `red') applies to a given situation is also a brute
    fact.  There would be no underlying principle from which one could
    deduce the application-criteria for such a term---and certainly no
    principle discoverable by means of the internal examination called
    for by the second thesis.

    This might work for ordinary object terms, but it less appropriate
    for color terms.  As Wright points out, we may in fact apply color
    terms inconsistently, but what `ontological import' does this
    have?  If ``\,`x is red' is true iff x is red'' is true, then if
    we allow any inconsistency in our (correct) application of ``r is
    red'', then we may end up committed to some color being red which
    is closer on the color-spectrum to orange than some color which is
    not red but orange.  This doesn't make much sense (including this
    explanation).
\end{enumerate}

\subsection{The first thesis}
What would be involved in rejecting the notion that our language-use
is governed by consistent rules?  Earlier we criticized Unger for
denying that ordinary terms could ever have correct or accurate uses,
but if there are no rules governing them, on what grounds can we say
that a given use is correct?
\ifstandalone
\end{spacing}
\bibliography{everything}
\bibliographystyle{ChicagoReedweb}
\fi
\end{document}


\end{spacing}
\bibliography{everything}
\bibliographystyle{ChicagoReedweb}
\end{document}


%% \chapter{Paraphrases}
%% \chapterpig{Paraphrases}
%% \documentclass[11pt]{standalone}
\usepackage{standalone} \newif\ifstandlone \standalonetrue
\usepackage[left=1.75in, right=1.75in, top=1.25in, bottom=1.25in]{geometry}
\geometry{letterpaper}
\usepackage{verbatim}
\usepackage{graphicx}
\usepackage{enumitem}
%\usepackage{amssymb}
\usepackage{amsmath}
\usepackage{epstopdf}
\usepackage{setspace}
\usepackage{natbib}
\setcitestyle{aysep={}}
\usepackage{hyperref}
\usepackage{url}
\synctex=1

\DeclareSymbolFont{symbolsC}{U}{txsyc}{m}{n}
\DeclareMathSymbol{\strictif}{\mathrel}{symbolsC}{74}
\DeclareMathSymbol{\boxright}{\mathrel}{symbolsC}{128}

\newenvironment{squote}{%
\begin{spacing}{1}
       	\begin{list}{}{%
\setlength{\labelwidth}{0pt}%
\rightmargin\leftmargin%
}
\item\relax
}{%
\end{list}%
\end{spacing}
}

\title{1.25}
\author{Alexander A. Dunn}
\begin{document}
\ifstandalone
\maketitle
\begin{spacing}{1.5}
\fi

\section{Paraphrases}
I have proposed that any attempt to deny the existence of ordinary
things such as tables and chairs must be supplemented by an
explanation as to why we believe in the existence of ordinary things.
As we will see, Peter Unger's claim that there are no ordinary things
and his claim that propositions like ``that is a chair'' are uniformly
false leaves it quite mysterious why we take there to be chairs in the
first place.  Some nihilistic philosophers, therefore, have attempted
to maintain Unger's first thesis---that there are no ordinary
things---while rejecting the second---that ordinary thing discourse is
invariably false.  Such a philosopher will claim that such discourse
is {\em compatible} with the nonexistence of chairs.  This may involve
the claim that while we take ourselves to have beliefs about chairs
and other ordinary things, our beliefs do not actually concern such
(non-existent) entities.  Rather, our thought and talk is (or should
be seen as) relating to such things as do exist.  Strategies that
follow this pattern can be called paraphrasing strategies, and Peter
van Inwagen has presented a well-known version.

Like Unger, van Inwagen claims that (necessarily) there are simply no
such things as tables and chairs in the world.  But unlike Unger, he
does not claim that when we take ourselves to be thinking and talking
about such things, we are thinking and talking about nothing at all.
At least, ``when people say things in the ordinary business of life by
uttering sentences that start `There are chairs\,\ldots ' or `There
are stars\,\ldots ', they very often say things that are literally
true'' \citep[102]{inwagen1995}.  

One might assume that if such statements are true, then it follows
that there are chairs and stars.  But van Inwagen denies that chairs
and stars exist.  How can he claim, then, that what was said was true?
What van Inwagen does is attempt to show that the statements in
question can be {\em paraphrased}---they can be reformulated to show
that they have no `ontological commitments'.  According to van
Inwagen, one can assert that there is a chair without being committed
to the existence of chairs.

Section~\ref{comp} will summarize the motivation for van Inwagen's
denial.  Section~\ref{inwagen} will introduce and criticize van
Inwagen's paraphrasing strategy.

\subsection{Composition}
\label{comp}
Van Inwagen's conclusion that there are no chairs is a consequence of
his views on {\em composition} (or `constitution').  Some things are
said to compose another thing if the former are {\em parts} of the
latter; the latter is `made up of' the former.  Van Inwagen believes
that ``the metaphysically puzzling features of material objects are
connected in deep and essential ways with metaphysically puzzling
features of the constitution of material objects by their
parts''~\citep[18]{inwagen1995}.  An example of such a puzzle is the
Ship of Theseus.  The Ship of Theseus is (presumably) an object
composed of many parts, including planks of wood.  As the planks (and
other parts of the ship) wear out, they are replaced.  These
replacements happen each by themselves; the entire ship (or even a
large section) is not replaced all at once.  But eventually no part of
the original ship remains; it is build of entirely different planks,
nails, rigging, etc.  And yet we would commonly say that it is still
the same ship.  But why should we think that the present ship is
identical with a past ship with which it shares no parts?

\subsection{The Special Composition Question}
\label{scq}
Answering the question `why is this ship identical with that past
ship?' requires first figuring out why (and how) these planks and
rigging and sails (et.\ al.) compose a ship in the first place.  Van
Inwagen asks ``in what circumstances do planks\footnote{For
  simplicity's sake, van Inwagen ignores the rigging and sails.}
compose (add up to, form) something?'' (\citeyear[21]{inwagen1995}) 
For some $x$s, then, van Inwagen asks us to consider when
\begin{equation}
\exists y\ \text{the}\ x\text{s compose}\ y
\end{equation}
is true.%
\footnote{Van Inwagen explains in some detail how plural referring
  expressions (like ``the planks'') can be given a logical
  formalization (\citeyear[23--28]{inwagen1995}), but suffice to say
  they work just as one would expect.}
%
\ Less formally, van Inwagen asks: ``suppose one had certain
(nonoverlapping) objects, the $x$s, at one's disposal; what would one
have to do---what {\em could} one do---to get the $x$s to compose
something?'' (\citeyear[31]{inwagen1995})  This is the Special
Composition Question.

(`Composition' is used in a technical sense with regard to the Special
Composition Question.  Van Inwagen defines it thus: ``the $x$s compose
$y$'' means that ``the $x$s are all parts of $y$ and no two of the
$x$s overlap and every part of $y$ overlaps at least one of the
$x$s\,\ldots\,a thing {\em overlaps} a thing---or: they overlap---if
they have a common part'' (\citeyear[29]{inwagen1995}).  For van
Inwagen, everything is a part of itself; some $x$ is a {\em proper}
part of some $y$ only if $x \neq y$.)

\subsection{The usual answers}
\label{scq-ans}
There are several prominent answers to the Special Composition
Question, including the following (These formulations are from
\citet{markosian1998a}):
\begin{description}
	\item[Nihilism] Necessarily, for any $x$s, there is an object
          composed of the $x$s iff there is only one of the $x$s,
          i.e., the only objects that exist are
          simples (\citeyear[219]{markosian1998a}).
	%
	\footnote{\label{flip} Note that this may not be Unger's view.
          He denies that people, apples, cheese, tables, chairs, and
          other ``ordinary things'' are nonexistence but he does not,
          as far as I know, take a stand on whether anything at all
          exists.  His view can be (flippantly) summarized thus: ``if
          we have a word for it, it doesn't exist.''}
	%\footnote{\label{gunk} Of course, it may be that the world is
	%not fundamentally particulate, and is filled not with simples
	%but with `gunk'; see \citet{schaffer2003}.  Nihilism (and van
	%Inwagen's second condition below) can be formulated to take
	%this possibility into account: ``for any quantity of gunk,
	%there is nothing composed of it.''}%
	%
	\item[Universalism] Necessarily, for any $x$s, there is an
          object composed of the $x$s iff no two of the $x$s
          overlap (\citeyear[227]{markosian1998a}).
	\item[Van Inwagenism] Necessarily, for any $x$s, there is an
          object composed of the $x$s iff either (i) the activity of
          the $x$s constitutes a life or (ii) there is only one of the
          $x$s (\citeyear[221]{markosian1998a}).
\end{description}

We will discuss Unger's version of nihilism in section \ref{unger}.
As I will argue, any version of nihilism that does not explain our
beliefs in the existence of ordinary objects is problematic.

Universalism raises a number of problems in relation to Peter Unger's
`problem of the many' (see section \ref{many}).  I will therefore
postpone discussion of this view until later.

Van Inwagen examines and rejects universalism and the version of
nihilism given above.  He also rejects a number of other answers to
the Special Composition Question.  Some are too strong: `some $x$s
compose a $y$ iff the $x$s are in contact' would entail that two
people shaking hands will result in a new object coming into being.
Others are too strong in some ways and too weak in others: `some $x$s
compose a $y$ iff the $x$s are fastened together' would entail that
two people being glued together would result in a new object; and it
would deny that an object can be composed without fastening its parts
together (such as when building a house of cards).  The only answer
van Inwagen finds consistent is what we have dubbed {\em van
  Inwagenism}, which entails that tables and chairs do not exist.

Because of this consequence, van Inwagenism should include an
explanation why we nonetheless believe that there are tables and
chairs.  Happily, van Inwagen recognizes this and is prepared with a
{\em paraphrasing strategy}.  This strategy aims to show that the
beliefs that we take to be about tables and chairs are really about
something else, and are not true beliefs.  If such beliefs are true,
then it should be relatively easy to explain why hold them: they are
true, and we learn of them through some reliable means (like our
eyes).

Unfortunately, van Inwagen's paraphrasing strategy does not work.

\section{Van Inwagen's paraphrasing strategy}
\label{inwagen}
Peter Unger maintains that terms like `chair' are incoherent; if this
is so, a statement involving the phrase `There is a chair\,\ldots '
could surely not be true.  Van Inwagen, on the other hand, admits that
``when people say things in the ordinary business of life by uttering
sentences that start `There are chairs\,\ldots ' or `There are
stars\,\ldots ', they very often say things that are literally true''
\cite[102]{inwagen1995}.  It does not seem unreasonable to assume that
if what people say with ``There are chairs\,\ldots '' and the like are
true, then chairs exist.  But van Inwagen denies this entailment.

How can van Inwagen maintain this?  Someone can say, truly, ``There
are simples arranged chairwise\,\ldots '' without committing oneself
to the existence of chairs.  Van Inwagen might then claim that when
someone says ``There is a chair\,\ldots '' she {\em means} `There are
simples arranged chairwise.'  This is, of course, a bold hypothesis
about the speech practices of ordinary speakers.  Certainly very few
speakers would, if asked, affirm that what they meant to say had
anything to do with simples; they would say that when they said that
there was a chair, they meant just that.  Van Inwagen recognizes that
this is not a viable position: ``The only thing I have to say about
what the ordinary man really means by `There are two valuable chairs
in the next room' is that he really means that there are two valuable
chairs in the next room'' (\citeyear[106]{inwagen1995}).

One might then assume that van Inwagen is thinking in analogy with
Russell.  He could attempt to claim that, despite the surface
appearance of language (`There is a chair\,\ldots '), the underlying
logical form does not make any mention of chairs (or tables); the
offending concept is analyzed away, leaving `There are simples
arranged chairwise\,\ldots '.  Van Inwagen notes that his ``suggested
technique of paraphrasing enables us to escape some of the more
embarrassing consequences of this position.  When someone says `Some
tables are heavier than some chairs,' there is obviously something
right about what he says.  Our technique of paraphrasis enables us to
capture what it is that is right about what he says''
(\citeyear[111]{inwagen1995}).  However, on the very next page he
admits that the ordinary language proposition and his paraphrased
version are different propositions: ``When the ordinary man utters the
sentence `Some chairs are heavier than some tables' (in an appropriate
context, and so on and so on), he expresses a certain proposition, and
one that is almost certainly true.  But I do not claim that this
proposition {\em is} the proposition that, for some $x$s, those $x$s
are arranged chairwise and for some $y$s, those $y$s are arranged
tablewise, and the $x$s are heavier than the $y$s''
(\citeyear[112]{inwagen1995}).  So van Inwagen is not making an appeal
to some notion of `logical form'.  But then what is the purpose of the
paraphrasing project?

Van Inwagen attempts to justify his method of paraphrasis by asserting
the following parallels between the original and paraphrased
propositions:
\begin{enumerate}[label=(\Alph*)]
	\item The paraphrase describes the same fact as the
          original.  \label{para-a}
	\item The paraphrase, unlike the original, does not even
          appear to imply that there are any objects that occupy
          chair-receptacles.  \label{para-b}
	\item The paraphrase is neutral with respect to competing
          metaphysical theories, {\em viz}.  the ``received'' theory,
          that there are objects that occupy chair-receptacles, and
          the theory I have proposed, according to which there are no
          such objects.  \label{para-c}
	\item The original, though it doubtless does not express the
          same proposition as the paraphrase, has the feature ascribed
          to the paraphrase in \ref{para-c}: It is neutral with
          respect to the question whether there are objects that fit
          exactly into
          chair-receptacles~(\citeyear[113]{inwagen1995}).  \label{para-d}
\end{enumerate}

I am rather dubious as to the truth of \ref{para-a}, but I am quite
sure that \ref{para-d} is false, and van Inwagen's thesis appears to
depend on it.  He admits in \ref{para-b} that the original sentence
(`There are chairs\,\ldots ') {\em implies} that there are chairs, but
claims in \ref{para-d} that it does not {\em entail} this.  But why
wouldn't it?

\subsection{Propositions and ontological commitment}
Let us review the situation.  First, van Inwagen agrees that when
someone says things like ``There is a chair\,\ldots '' they mean just
that.  Second, he admits that his `paraphrases' of such propositions
are not so faithful to the original that they can be called the same
proposition; the original and the paraphrase are two different
propositions.  Third, he claims nonetheless that {\em neither} the
original nor the paraphrase entail the existence of chairs.

This seems obviously untrue.  How can he claim that when someone says
``There is a chair\,\ldots '' and means just that, that the
proposition they express does not entail the existence of chairs?  To
defend his claim, van Inwagen appeals to his `Copernican analogy':

\begin{squote}
I accept the Copernican Hypothesis.  One day you hear me say, ``It was
cooler in the garden after the sun had moved behind the elms.''  You
say, ``You see, you can't consistently maintain your Copernicanism
outside the astronomer's study.  You say that the sun moved behind the
elms; yet, according to your official theory, the sun does not move.''
I reply that the proposition I expressed by saying ``It was cooler in
the garden after the sun had moved behind the elms'' is consistent
with the Copernican Hypothesis (\citeyear[101]{inwagen1995}).
\end{squote}
That is, van Inwagen claims that the proposition he expressed with
``It was cooler in the garden after the sun had moved behind the
elms'' does not entail that the sun actually moved.  And he argues
that this is analogous to our talk of chairs: most propositions
expressed with ``There is a chair\,\ldots '' do not entail that chairs
actually exist.

First, does the proposition van Inwagen expresses with ``The sun moved
behind the elms'' entail that the sun moved? I am inclined to say that
it does.  If I were to say simply ``The sun moved'' (meaning just
that), I think I would have committed myself to the movement of the
sun.  Why should we think that the addition of `behind the elms'
defeats this entailment?  Without some explanation of what the
difference is, I see no reason to think that saying ``The sun moved
behind the elms'' (and meaning it) does not entail the movement of the
sun.  But van Inwagen may be forced to say here that neither
proposition entails that the sun moved.  For he certainly won't allow
that either entails that the sun {\em exists}.

There is an analogy here, though perhaps not the one van Inwagen had
in mind.  He claims that a proposition expressed by ``There are two
very valuable chairs in the next room'' does not necessarily entail
the existence of chairs.  If this proposition does not entail that
chairs exist, then what about `There are two valuable chairs left in
the world' or `There are at least two chairs in the world' or `There
are at least two chairs' or simply `There are chairs'?  Van Inwagen
appears committed to the claim that the proposition I would express
with ``There are chairs'' does not entail that there are chairs.

Why on earth should this be?  Does not the proposition expressed by my
saying ``There are simples arranged chairwise\,\ldots '' entail the
existence of simples?  If van Inwagen says that there are simples
arranged chairwise, and means just that, then it would appear to
follow that there are simples.  Indeed, van Inwagen's argument relies
rather heavily on the assumption that simples exist.
%
%% \footnote{Ted Sider takes him to task for this
%%   assumption~(\citeyear{sider1993}), claiming that the possibility of
%%   `gunk'---the possibility that the matter of the world is not
%%   fundamentally particulate but infinitely divisible---falsifies van
%%   Inwagen's thesis.  I think it may be possible for van Inwagen to
%%   adapt to a gunky world (he might be able to claim that nothing
%%   exists but organisms, who are composed of other organisms and/or
%%   gunk), but I think van Inwagen's thesis is false either way.}
%
\ But if `There are chairs' does not entail that there are chairs and
if `The sun moved behind the trees' entails neither that the sun moved
nor that the sun exists, then how can van Inwagen maintain that `There
are simples arranged chairwise' entails that there are simples, or
that they are arranged chairwise?  He has given us no reason to
believe one and not the other.

\subsection{Loose truth}
\label{loose-v}
Van Inwagen's attempt to maintain that there are no chairs and that
``there are chairs'' is true does not appear promising.  But he may
instead claim that ``there are chairs'' is not true, but {\em loosely
  true}.  He admits this position as a possibility:

\begin{squote}
I can say this [that `There are chairs\,\ldots ' can be true yet not
  entail that there are chairs] because I accept certain theses in the
philosophy of language.  I can say this because I accept certain
theses in the philosophy of language.  Some people, I suppose, would
reject these theses.  These people would say that when I
said\,\ldots\ The sun moved behind the elms,' I said something
false\,\ldots If someone maintains that `The sun moved behind the
elms' expresses a falsehood, he must still have some way to
distinguish between this sentence and those sentences (like `The sun
exploded' and `The sun turned green') that the vulgar would regard as
the sentences that expressed falsehoods about the sun\,\ldots [if I
  took this position,] I should not be willing to say that people who
uttered things like `There are two valuable chairs in the next room'
very often said what was true.  I should be willing to say only that
they very often say what might be treated as a truth for all practical
purposes (\citeyear[102--103]{inwagen1995}).
\end{squote}

Van Inwagen admits that `There are two very valuable chairs in the
next room', ``when it is successfully used to report a fact, does
report a fact about the existence of {\em something}''
(\citeyear[102]{inwagen1995}).  Presumably van Inwagen thinks that
`something' is the chairwise arrangements of simples.  Van Inwagen
must therefore explain how a chairwise arrangement of simples can
cause people to believe that there are chairs.  Van Inwagen attempts
to bolster his case by drawing an analogy with the imaginary bliger.
The bliger, according to van Inwagen, is what happens when a monkey,
four owls, and a tiger attach themselves together temporarily.  The
conglomeration appears to the untrained observer to be a single
animal, and gullible farmers dubbed it a `bliger'.  Van Inwagen's
point is that there are no bligers, but that a farmer saying ``there's
a bliger'' when pointing at such a conglomeration would be saying
something loosely true.  The fact being reported by ``there's a
bliger'' is the fact that a monkey, four owls, and a tiger are there.
People believe that there are bligers because they mistake the group
of animals for a single thing, which has been dubbed `bliger'.

Likewise, van Inwagen maintains that people mistake chairwise
arrangements of simples for chairs.  When someone says ``there's a
chair'' what she says is loosely true because there is a chairwise
arrangement of simples there.  People believe that there are chairs
because they mistake the group of animals for a single thing, which
has been dubbed `chair'.

I agree with van Inwagen that these cases are analogous.  However,
where van Inwagen takes this analogy to show that there are no chairs,
I take it to show that there {\em are} bligers in van Inwagen's
imaginary scenario.  When it is discovered that bligers are built up
from six other creatures, we are learning something about bligers:

\begin{squote}
\ldots {\em of course} there are bligers in [van Inwagen's] story.
Bligers are what the story is about.  The zoologists do not report
that there are no bligers.  Rather they tell us what a bliger is.
They explain that a bliger is not a single large carnivorous animal
but a transient symbiotic union of six animals
\citep[704]{rosenberg1993}.
\end{squote}

In short, van Inwagen's analogy does not provide us with an
explanation of why we would believe in chairs even if there were none.
Just as the Ungerian treatment of loose truth failed to explain our
beliefs in chairs, van Inwagen's appeal to loose truth does not
explain why we believe that there are chairs, rather than chairwise
arrangements of simples.

\ifstandalone
\end{spacing}
\fi
\end{document}

	
%% \chapter{``Nearly as good as true''}
%% \chapterpig{``Nearly as good as true''}
%% \documentclass[11pt]{article}
\usepackage{standalone} \newif\ifstandlone \standalonetrue
\usepackage[left=1.75in, right=1.75in, top=1.25in, bottom=1.25in]{geometry}
\geometry{letterpaper}
\usepackage{graphicx}
\usepackage{enumitem}
\usepackage{amssymb}
\usepackage{amsmath}
\usepackage{tipa}
\usepackage{epstopdf}
\usepackage{verbatim}
\usepackage{setspace}
\usepackage{natbib}
\setcitestyle{aysep={}}
\usepackage{url}
\synctex=1
\usepackage{hyperref}

\newenvironment{squote}{%
\begin{spacing}{1}
       	\begin{list}{}{%
\setlength{\labelwidth}{0pt}%
\rightmargin\leftmargin%
}
\item\relax
}{%
\end{list}%
\end{spacing}
}

\title{``Nearly as good as true''}
\author{Alexander A. Dunn}
\begin{document}
\ifstandalone
\maketitle
\begin{spacing}{1.25}
\fi

\section{How does Merricks explain what we believe?}
\label{universe}
\label{merricks}
Trenton Merricks comes to the same metaphysical conclusions as does
van Inwagen.  That is, he claims that there are no physical objects
other than human beings.  However, he comes to this conclusion through
a different path of reasoning.  He claims, roughly, that positing
ordinary things (excluding people) is causally redundant; everything
that ordinary things are said to do can be described in terms of their
parts. (The details are unimportant; what matters is how Merricks
explains why we nonetheless believe that there are ordinary things.)

Despite the fact that Merricks has a different motivation for his
nihilism, we can pose the same question to him as we posed to van
Inwagen.  Why, if there are no chairs, do we believe that there are
chairs?  Happily, Merricks addresses our concern.  Even more happily,
he has a better explanation than van Inwagen.  He explains why, if
nihilism is true, we might nonetheless believe that there are chairs.
But when giving his explanation, he presupposes that {\em
  universalism} is false (see sections \ref{scq-ans} and
\ref{universalism}).  Universalism, like nihilism, seems to contradict
certain of our beliefs, but Merricks' explanation can also explain
why, if universalism is true, we nonetheless hold these certain
beliefs.  Merricks' explanation does not therefore provide nihilism
any advantage over universalism, and the latter is far more plausible.

\subsection{Nearly as good as true}
\label{near}
Merricks claims that `folk' beliefs, such as the belief that there are
chairs, are false, but nonetheless are {\em nearly as good as true}.
What does this mean?

\begin{squote}
People who believe in unicorns [or ghosts] are few and far between.
And those few are generally unjustified.  On the other hand, people
who believe in statues are legion.  And they are generally justified
in so believing.  Given the truth of eliminativism [what I have been
  calling nihilism], we might ask {\em why} the belief in statues is
more common, and more commonly justified, than the belief in unicorns.

The answer is that statue beliefs are nearly as good as true.  For, so
I claim here, {\em atoms arranged statuewise} often play a key role in
producing, and grounding the justification of, the belief that statues
exist.  In general, a false belief's being nearly as good as true
explains how {\em reasonable} people come to hold it.  And, relatedly,
its being nearly as good as true can ground its justification.
Because the belief that unicorns exist is not nearly as good as true
(i.e.\ because there are no things arranged unicornwise), there is no
similar explanation of its production or similar reason to think it is
justified (\citeyear[171--172]{merricks2001a}).
\end{squote}

To say that something is ``nearly as good as true'' seems to be
equivalent to saying that it is `loosely true', or `true for practical
purposes'.  In each case, the proposition in question is false, but it
is somehow close enough to the truth for a given purpose or situation.
For example, suppose we have decided to buy a fake holiday tree for
the holidays this year.  We are looking at a number of different fake
trees.  I point to one and say ``that is a nice tree''.  What I have
said is false; that is not a tree.  It is a fake tree.  But what I
mean---and what my audience recognizes me to mean---is that it is a
nice {\em fake} tree.  We both know that we are looking at fake trees;
there is no point qualifying every use of `tree' with `fake'.  When I
say ``that is a nice tree'', therefore, what I say is quite sufficient
to allow for successful communication, despite being false.  Merricks
claims that propositions expressed by things like ``there are chairs''
are also loosely true.  They are false, but are nonetheless good
enough for certain purposes.

Initially, this seems like a bizarre claim.  After all, Merricks is
claiming that chairs {\em necessarily} do not exist.  According to
Merricks, ``chairs exist'', given its current meaning, could {\em
  never} be true.  If the proposition expressed by ``chairs exist'' is
necessarily false, how could it nonetheless be ``nearly as good as
true''?

\subsection{The conceptual connection}
\label{connection}
Merricks' argument relies on a very close conceptual connection
between ``chair'' and ``chairwise'' (and likewise for all ordinary
terms).  Despite claiming that chairs are impossible, Merricks admits
that we understand perfectly what chairs {\em would} be, if they
existed.  Because we understand the concept of `chair', we can
recognize {\em actually existing} things that are arranged
`chairwise':

\begin{squote}
The folk concept of \emph{statue} plays a role in determining which
atomic arrangements are statuewise. I would even go so far as to say
that if \emph{being arranged statuewise} were not derivative upon
folk-ontological concepts\,\ldots something would be amiss
(\citeyear[8]{merricks2001a}).
\end{squote}

For Merricks, to know what things are actually arranged statue- or
chairwise requires knowing what things would compose a statue or a
chair, if such things were possible:

\begin{squote}
Atoms are \emph{arranged statuewise} if and only if they both have the
properties and also stand in the relations to microscopica upon which,
if statues existed, those atoms' \emph{composing a statue} would
non-trivially supervene (\citeyear[4]{merricks2001a}).
\end{squote}

Merricks' explanation of why we believe that there are chairs relies
on this conceptual connection.  It also is structurally similar to the
explanation we gave in section \ref{intro-beliefs}.  Recall that our
explanation of why we believe that there are chairs (or statues) is
that, first, there are chairs, and, second, we see that there are
chairs (or learn that there are chairs through a similarly reliable
mechanism).

Merricks' definition of `nearly as good as true' allows us to produce
a parallel explanation.  His definition is this:

\begin{squote}
Any folk-ontological claim of the form `F exists' is \emph{nearly as
  good as true} if and only if (i) `F exists' is false and (ii) there
are things arranged F-wise. So, for example, `the statue \emph{David}
exists' is nearly as good as true because (it is false and) there are
some things arranged Davidwise (\citeyear[171]{merricks2001a}).
\end{squote}

We may now say on behalf of Merricks that we believe that there are
chairs (and statues) because, first, there are things arranged
chairwise and, second, we see that there are things arranged
chairwise.

The structure of the two explanations is analogous, but there is an
apparent disanalogy in the content of the two.  The disanalogy does
not favor Merricks.  For it is easy enough to understand why there
being chairs, and us seeing that there are chairs, would cause us to
believe that there are chairs.  But it is less obvious why there being
things arranged chairwise, and us seeing that there are things
arranged chairwise, would cause us to believe {\em not} that there are
things arranged chairwise, but that there are {\em chairs}.

(While it is certainly true that we believe that there are chairs, I
am not sure if all or even most of us {\em also} believe that there
are things arranged chairwise.  Let us suppose for now that we do.)

The close conceptual connection between `chair' and `chairwise' is
very important for Merricks.  It is this {\em connection} that is
doing the explanatory work.  The only thing that can explain why there
being things arranged chairwise would cause us to believe that there
are chairs is this connection between the concepts.  The existence of
things arranged chairwise, and the belief that there are things
arranged chairwise, is supposed to cause the {\em additional} belief
that there are chairs.  How does this happen?

Merricks' answer appears to go something like this: certain
arrangements of things---chairwise arrangements, statuewise
arrangements, and all ordinary arrangements---play important roles in
our lives.  These arrangements of things are of interest to us, so we
have developed words that allow us to refer to them.  For whatever
reason---historical, psychological, or otherwise---we think of each
arrangement as a single thing, rather than as things.  Words like
`chair' and `statue', being singular, reflect this (incorrect) view of
the world.  We are, in a sense, fooled by grammar.

This is more than Merricks says himself.  I have not found a passage
in which he explicitly describes the nature of the conceptual
connection between concepts like `chair' and `chairwise', and explains
why, from our belief that there are things arranged chairwise, we
invariably infer that there are chairs.  But I think he would endorse
something like this.  In the first chapter of his book, he claims that
whether there is a statue or merely things arranged statuewise is not
an empirical question.  He claims that were there not a statue and
merely things arranged chairwise, our ``visual evidence'' would be the
same.  He supports this claim with an analogy:

\begin{squote}
{[}Consider{]} the claim that the atoms arranged
my-neighbour's-dogwise and the-top-half-of-the-tree-in-my-backyardwise
compose an object\ldots it won't do to defend this claim with nothing
more than `I can \emph{just see} the object composed of the atoms
arranged dog-and-treetopwise'. Part of why this won't do, presumably,
is that one's visual evidence would be the same \emph{whether or not}
those atoms composed something (\citeyear[8--9]{merricks2001a}).
\end{squote}

He assumes, of course, that we do not believe that there is a thing
composed of a dog and some of a tree.  Later he suggests that it is
arbitrary to claim that there are statues but not dog-tree things:
``we ought to see that the only difference between arbitrary sums and
statues is a matter of conventional wisdom and local custom''
\citeyearpar[75]{merricks2001a}.  He seems sympathetic to the idea
that the reason we believe that there are statues, and not dog-tree
composites, is due to our conventional speech practices: ``it is at
least somewhat plausible that atoms arranged statuewise are united not
by composing something but, instead and in part, by how we speak and
think'' \citeyearpar[121]{merricks2001a}.

On this picture, whether we see an arrangement of things as composing
an object or not depends more on our own interests than features of
the things themselves.  We have words for chairs and statues because
things arranged chairwise and statuewise interest us.  We don't have a
word for things arranged ``my-neighbor's-dogwise and
the-top-half-of-the-tree-in-my-backyardwise'' because such an
arrangement does not hold much interest for us.  But each of these
arrangements exist, and it seems arbitrary to say that the chairwise
and statuewise arrangements compose chairs and statues while the other
arrangement composes nothing.

Merricks might explain why we believe there are things arranged
my-neighbor's-dogwise and the-top-half-of-the-tree-in-my-backyardwise
thus: there are things arranged my-neighbor's-dogwise and
the-top-half-of-the-tree-in-my-backyardwise, and we see that there are
things so arranged.  This is exactly the same explanation that I would
give.

Now Merricks explains why we believe that there are chairs thus: there
are things arranged chairwise, and we see that there are things
arranged chairwise.  {\em And incidentally, due to our own human
  peculiarities, we have found it convenient to refer to and think
  about things arranged chairwise as if they were `chairs'---single
  unified objects}.

\section{Strange objects}
\label{dogbush}
This is a somewhat plausible explanation of why we would believe that
there are chairs if there were not.  It is certainly much better than
van Inwagen's.  But I think that it fails.  I think that when we look
closer at Merricks' attempts to motivate nihilism, we will see that
they do not support nihilism at all.  If anything they support a
version of {\em universalism}.

Merricks observes that one might object to nihilism simply by saying,
``I just {\em see} the chair!''  He claims that if this objection
moves us, we should think about an analogous objection, which he finds
much less moving:

\begin{squote}
Whether atoms arranged statuewise compose a statue is analogous to
whether atoms arranged my-neighbour's-dogwise and
the-top-half-of-the-tree-in-my-backyardwise compose an object\,\ldots
it would not do to support an affirmative answer to the latter
question simply by saying `I can just see that object'
\citeyearpar[73]{merricks2001a}.
\end{squote}

It does indeed seem initially plausible to say that the top half of a
tree and my neighbor's dog do not compose anything.  But I think this
is ultimately incorrect.

Recall the bliger story that van Inwagen used to motivate his version
of nihilism (section \ref{prop-ont}).  A bliger was supposed to be
four monkeys, an owl, and a sloth, who arrange themselves into a
temporary symbiotic configuration.  Van Inwagen thought we would agree
that bligers did not exist.  He claimed that it is not true that ``six
animals arranged in bliger fashion compose anything, and that is what
I mean to deny when I say that there are no bligers''
\citeyearpar[104]{inwagen1995}.

But as we saw, it is simply false that there are no bligers:

\begin{squote}
\ldots {\em of course} there are bligers in [van Inwagen's] story.
Bligers are what the story is about.  The zoologists do not report
that there are no bligers.  Rather they tell us what a bliger is.
They explain that a bliger is not a single large carnivorous animal
but a transient symbiotic union of six animals
\citep[704]{rosenberg1993}.
\end{squote}

One only reason we might be tempted to say that there are no bligers
is that van Inwagen presents the question in an unintuitive way.  He
asks us if there is some thing, some object, that is composed of the
other six animals.  This gives one the impression that, were there to
be such a thing, it would perhaps be another animal (a seventh); were
there such a thing, it should somehow pop out at us.  But all we see
when we picture the scene are the six animals together, so we feel
that van Inwagen might be right.  There is no {\em other} thing.  But
if we phrase the question differently, things become clearer.  Rather
than ask if there is some thing composed of such and such other
things, we simply ask, ``are there bligers?''  And of course there
are.  Van Inwagen's use of the word `composition' led our intuitions
astray.

Merricks makes the same mistake in his passage above.  Imagine if he
had said, ``Consider five discontinuous islands.  One cannot argue
that they compose some further thing by simply saying `I just see
it!'\,'' If these five islands are an archipelago, then one {\em can}
say ``I just see the archipelago!''  {\em Of course} there are
archipelagos.  They are, as one might put it, `scattered objects'.
The archipelago is made up of a number of separate islands, but it is
nonetheless a thing.  It is an archipelago.  Now let us suppose there
is an archipelago in the Mediterranean Sea (this example is adapted
from \citet{hawthorne2008}).  This archipelago is called the Roman
Archipelago, due to the fact that there are a number of Roman ruins on
one of its islands.  There are several research camps on the islands,
where archaeologists dig for artifacts.  Their researches result in a
surprising discovery: one of the islands {\em is} a Roman ruin.  What
was thought to a rocky and curiously shaped island is in fact a
massive collapsed temple.  Further investigation reveals that another
island is made up of the bones of an extinct sea monster, and another
island is a crashed UFO.

Despite these extraordinary circumstances, it is nonetheless true that
the Roman Archipelago exists.  It just happens to be composed of
several islands, a Roman ruin, a pile of old bones, and an alien
spacecraft.  To say the Roman Archipelago does not exist would entail
that these things are {\em not} sitting in the Mediterranean Sea.  (Of
course I made this story up, so the Roman Archipelago in fact doesn't
exist; but it does in the story.)

If Merricks or someone else asks us ``could scattered islands, Roman
ruins, old bones and alien spacecraft ever compose anything?'' we
should reply ``{\em of course}''.  Now take this example:

\begin{quote}
Pranksters break into a museum to install joke pieces of art.  One one
wall they put up a bathroom mirror and towel ring (complete with
towel).  Under the mirror they put a little sign reading ``Wash your
hands''.  The installation is accepted as art by the gullible curator,
who gets an equally gullible journalist to write about it.  {\em Wash
  Your Hands} quickly becomes a valuable piece of art---valuable
enough that art thieves target it.  They break into the museum in
order to steal {\em Wash Your Hands}, but trip an alarm and are forced
to flee.  All they get away with is the towel.  In the morning the
guards tell the curator that part of {\em Wash Your Hands} is missing.
The curator orders them to remove the rest of the piece and informs
crestfallen visitors that {\em Wash Your Hands} is no longer in the
museum's collection.
\end{quote}

Here, the only point at which is it true to say that {\em Wash Your
  Hands} is not in the museum is when it is finally removed.  Someone
who claimed that it was {\em never} in the museum because it doesn't
exist would be saying something quite clearly false.  Thus if Merricks
asks us ``do mirrors and towels ever compose anything?'' we should say
``{\em of course!}\,''

In these two examples, it is clear that the things in question really
do exist.  Nobody will deny that there are archipelagos and works of
art without having first been moved by a philosophical argument.  But
it may be that people {\em will} deny that there are things composed
of the tops of trees and dogs, even before hearing an argument.

Call the things composed of dogs and treetops `dogbushes'.  For
example, in a park that contains one tree and one dog, there is also
one dogbush.  Is it {\em obvious} that there are dogbushes?  Is it
just as obvious as that there are archipelagos and chairs and pieces
of crappy modern art like the {\em Wash Your Hands}?  If not, why?
What is the difference between things like archipelagos and things
like dogbushes?

One obvious difference is that things like archipelagos interest us.
I argued above that Merricks motivates his nihilism by drawing our
attention to the role of tradition and convention in our talk.  We
have a word for archipelagos because they {\em matter} to us.  We
don't have a word for dogbushes because they {\em don't} matter.
Merricks argued, in effect, that since we are not inclined to say that
there are dogbushes, and since there is no metaphysical difference
between dogbushes and archipelagos, we should not be inclined to say
that there are archipelagos.

But we can reverse Merricks' argument.  Since we {\em are} inclined to
say that there are archipelagos, and since there is no metaphysical
difference between archipelagos and dogbushes, we should not be
inclined to deny that there are dogbushes.

\section{Universalism}
\label{universalism}
I claimed in section \ref{connection} that Merricks' explanation of
why we believe that there are chairs is something like this: there are
things arranged chairwise, and we see that there are things arranged
chairwise.  {\em And incidentally, due to our own human peculiarities,
  we have found it convenient to refer to and think about things
  arranged chairwise as if they were `chairs'---single unified
  objects}.  I attributed to Merricks the idea that just because
things arranged chairwise interest us, we should not therefore suppose
that there are chairs.  What interests us should not be a guide to
what exists.  But now there is an obvious counter against this move.
Just because dogbushes do {\em not} interest us, we should not
therefore suppose that there are not dogbushes.  In this spirit,
Judith Thomson suggests that we ``think of Reality as like an
over-crowded attic, some of its contents interesting, and most merely
junk.  There is no need to deny the junk; we can simply leave it to
gather dust'' \citep[167]{thomson1998a}.  This is the intuition behind
{\em universalism}, one of the answers to the Special Composition
Question (section \ref{scq}):

\begin{description}
\item[Universalism] Necessarily, for any $x$s, there is an object
  composed of the $x$s if and only if no two of the $x$s overlap
  \citep[227]{markosian1998a}.
\end{description}

The following argument suggests itself:

\begin{enumerate}[ref=(\arabic*)]
  \item Chairs exist. \label{u-1}
  \item Things that do not differ from chairs (or archipelagos, or
    works of art) in metaphysically significant ways also exist.
  \item Dogbushes do not differ from chairs in metaphysically
    significant ways.
  \item {\em Therefore}, dogbushes exist. \label{u-4}
\end{enumerate}

I imagine that Merricks would deny the conclusion \ref{u-4} and so, by
{\em modus tollens}, deny one or more premises (and we have seen that
he denies \ref{u-1}).  But I affirm the premises and so, by {\em modus
  ponens}, affirm the conclusion.

This argument helps us see what is wrong with Ned Markosian's response
to the Special Composition Question.  Markosian defends what he calls
`brutal composition'.  The thesis of brutal composition is that, while
there is indeed no ``no true, non-trivial, and finitely long answer to
[the Special Composition Question]''
\citeyearpar[214]{markosian1998a}, this is not because we should refer
questions of composition to the empirical sciences.  Rather, whether
or not some things compose another is simply a {\em brute fact}.

This is a clever reply, but whether true or not I do not think it does
the work that Markosian expects it to.  He presents his theory as
``consistent with standard, pre-philosophical intuitions about the
universe's composite objects'' \citeyearpar[211]{markosian1998a}.  But
his theory will only be consistent with such intuitions if, first, it
is a brute fact that all of the things we ordinarily take to exist
(tables, chairs, etc.) do in fact exist, and, second, that it is a
brute fact that the things that we don't take to exist don't in fact
exist.  But why should we expect there to be a {\em metaphysical}
difference between things that interest us and things that don't?  The
chance that the brute facts of composition happen to line up with our
(or Markosian's) intuitions seems to be incredibly low.

But accepting the above argument for universalism has some strange
consequences that are not immediately apparent.  Ned Markosian brings
out such a consequence in this passage:

\begin{squote}
There is what seems to me a fatal objection to Universalism:
Universalism entails that there are far more composite objects than
common sense intuitions can allow.  To give just one example,
Universalism entails that the following sentence is true:\,\ldots
There is an object composed of (i) London Bridge, (ii) a certain
sub-atomic particle located far beneath the surface of the moon, and
(iii) Cal Ripken, Jr.  My intuitions tell me that there is no such
object, and I suspect that the intuitions of the man on the street
would agree with mine on this point \citeyearpar[228]{markosian1998a}.
\end{squote}

If this is a compelling objection, it is because such an object (call
it `Lumpkin') does not interest us in the least.  As Merricks observed
(see section \ref{connection}), the things that we believe to exist
are largely the things that interest us.  We believe that there are
archipelagos; van Inwagen's imaginary farmers believe that there are
bligers.  If we do not believe that there are dogbushes or Lumpkins,
this may be because they do not interest us.

Suppose Markosian wrote this instead:

\begin{squote}
Universalism entails that the following sentence is true:\,\ldots
There is an object composed of (i) an island, (ii) a Roman ruin, and
(iii) the bones of a sea monster.
\end{squote}

But this is just the Roman Archipelago I mentioned in section
\ref{dogbush}.  We are (or should be) happy to admit that it exists.
If the Roman Archipelago exists, and if it does not differ from the
Lumpkin in any metaphysically significant ways, why shouldn't we admit
that the Lumpkin exists?  Of course we don't {\em care} about the
Lumpkin.  We have no need to refer to it; it doesn't matter to our
lives.  But why should we expect---as Markosian seems to---that our
intuitions should perfectly track what exists?

\section{Lessons, part 2}
\label{lessons-m}
What we have learned from examining Merricks' arguments is not that
there are no chairs.  What we have learned is that since there {\em
  are} chairs, and since dogbushes do not differ from chairs in
metaphysically significant ways, there are therefore also dogbushes.

If we agree that there are chairs and archipelagos and dogbushes and
the Lumpkin, however, new questions arise. For example: What are the
parts of a chair?  How do they compose the chair?  Do the parts of the
chair change over time?  How?  We will address these questions in the
next section.  

\ifstandalone
\end{spacing}
\bibliography{everything}
\bibliographystyle{ChicagoReedweb}
\fi
\end{document}


%% \chapter{Incoherence}
%% \chapterpig{Incoherence}
%% \documentclass[11pt]{article}
\usepackage{standalone} \newif\ifstandlone \standalonetrue
\usepackage[left=1.75in, right=1.75in, top=1.25in, bottom=1.25in]{geometry}
\geometry{letterpaper}
\usepackage{graphicx}
\usepackage{enumitem}
%\usepackage{amssymb}
\usepackage{amsmath}
\usepackage{epstopdf}
\usepackage{setspace}
\usepackage{natbib}
\setcitestyle{aysep={}}
\usepackage{hyperref}
		
\synctex=1

\DeclareSymbolFont{symbolsC}{U}{txsyc}{m}{n}
\DeclareMathSymbol{\strictif}{\mathrel}{symbolsC}{74}
\DeclareMathSymbol{\boxright}{\mathrel}{symbolsC}{128}

\newenvironment{squote}{%
	\begin{quote}\begin{singlespace}%
	}{%
	\end{singlespace}\end{quote}}

\newcommand{\stager}[4]%
{%
	\begin{spacing}{1}%
	\vspace{0pt}
		\begin{description}[style=nextline, noitemsep,
                    parsep=0pt, topsep=0pt, leftmargin=15mm,
                    itemindent=-10mm, font=\mdseries]
			\item[\textsc{#1} \emph{#2}] #3
			\item[]%
			\begin{flushright}#4\end{flushright}
		\end{description}%
	\end{spacing}%
}

\newcommand{\stage}[3]%
{%
	\begin{spacing}{1}%
	\vspace{0pt}
		\begin{description}[style=nextline, parsep=0pt,
                    leftmargin=15mm, itemindent=-10mm, font=\mdseries]
			\item[\textsc{#1} \emph{#2}] #3
		\end{description}%
	\end{spacing}%
}

\newenvironment{inq}{\vspace{0pt}%
	\begin{list}{}%
	{\setlength\labelwidth{0pt}%
	\setlength\leftmargin{2.5\oddsidemargin}%
	\setlength\rightmargin{\leftmargin}}
	\begin{spacing}{1}
	\item[]%
	}{
	\end{spacing}
	\end{list}
	\vspace{10pt}
	%\noindent%
	}

\title{Unger's arguments}
\author{Alex Dunn}
\begin{document}
\ifstandalone
\maketitle
\begin{spacing}{1.5}
\fi

Peter Unger has presented several arguments that threaten the kind of
universalism I sketched in section \ref{universalism}.  The versions
of the sorites paradox that he presents make trouble for all vague
concepts, including those that Merricks relies on for his version of
nihilism.  Unger's `problem of the many', on the other hand, poses a
problem specifically for versions of universalism, including my own.

\section{Incoherence and pluralities}
\label{unger}
Like Peter van Inwagen and Trenton Merricks, Peter Unger has denied
the existence of all `ordinary things'---such things as ``tables and
chairs and spears\,\ldots swizzle sticks and
sousaphones\,\ldots\,stones and rocks and twigs, and also tumbleweeds
and fingernails'' (\citeyear[117]{unger1979}).  Merricks has a
different motivation for his nihilism than does van Inwagen, and Unger
has a different motivation again.  In fact, he has two different
motivations; one is the sorites paradox and the other is the problem
of the many.  

Unger claims that the sorites paradox shows that terms for ordinary
things, like `chair', are {\em incoherent}.  He claims that incoherent
terms cannot apply to anything in the world; therefore he concludes
that there are no chairs (or any other ordinary thing).

Unger's presentation of the problem of the many is aimed to trouble
the concept of `chair' in a different way.  The conclusion of that
argument is not that `chair' is necessarily incoherent.  Rather, the
conclusion is a disjunction: {\em either} there are no chairs, {\em
  or} there are a plurality (possibly an infinity) of chairs where we
would normally take there to be only one.

We will examine these two arguments in turn.

\section{Sorites paradoxes}
\label{sorites}
A typical instance of the sorites paradox begins by having us imagine
some ordinary object; let us use a heap of sand.  Now suppose we
remove a single grain of sand.  If we were inclined to believe that
the initial quantity of sand did in fact constitute a heap, then after
the removal of a single grain, we should presumably still have a heap
(albeit a slightly smaller one).  It seems very implausible to think
that one grain of sand more or less could {\em ever} make a difference
as to whether something is or is not a heap.

But having conceded (a) that there is a heap and (b) that the removal
of a single grain cannot make the difference as to whether a quantity
of sand is a heap, we have unwittingly put our foot in it.  For if the
removal of a single grain {\em never} transforms a heap into a
non-heap, then by repeatedly removing one grain after another, we will
eventually find ourselves with a heap that consists of no sand at
all.  But it seems absurd to suppose that there could be a heap of
sand that is composed of no sand---indeed, of nothing whatsoever.

This is the sorites paradox.  While a heap is a useful example,
because it is so ill-defined, similar problems appear to afflict all
ordinary things.  Unger illustrates the difficulty for stones:

\begin{squote}
Consider a stone, consisting of a certain finite number of atoms.  If
we or some physical process should remove one atom, without
replacement, then there is left that number minus one, presumably
constituting a stone still\,\ldots after another atom is removed,
there is that original number minus two; so far, so good.  But after
that certain number has been removed, in similar stepwise fashion,
there are no atoms at all in the situation, while we must still be
supposing that there is a stone there.  But as we have already agreed,
if there is a stone present, then there must be some atoms\,\ldots I
suggest that any adequate response to this contradiction must
include\,\ldots the denial of the existence of even a single
stone.~\citep[121--122]{unger1979}
\end{squote}
Unger understands this dilemma to apply across the board, and
correspondingly argues that we should deny the existence of even a
single ordinary thing.

\subsection{The sorites paradox in relation to Merricks' nihilism}
\label{sorites-m}
The purpose of Unger's argument is to show that terms that are
susceptible to the sorites paradox are incoherent and cannot apply to
anything in the world.  Unger claims that terms like `chair' therefore
cannot apply to anything in the world---from which it follows that
there are no chairs.

Trenton Merricks' version of nihilism (section \ref{merricks}) denies
that there are chairs.  In this, Merricks is in agreement with Unger.
But unlike Unger, Merricks maintains that beliefs like ``there are
chairs'' are {\em justified} and {\em nearly as good as true}.  He
claims that such beliefs are nearly as good as true if they are caused
by simples arranged chairwise.  Merricks does not believe that there
are chairs, so ``there are chairs'' is, strictly speaking, false.  But
Merricks believes that there are simples arranged chairwise, and the
presence of such arrangements cause and justify false beliefs such as
``there are chairs''.

We have seen, however, that the sorites paradox threatens the
coherency of concepts like `chair'.  If we suppose that a given
collection of atoms composes a chair, we can remove them one by one
and at no point feel justified in saying that the chair suddenly
ceases to exist.  Now suppose that a given arrangement of simples is
arranged chairwise.  Remove one.  Is the arrangement still arranged
chairwise?  Just as in the case of the chair, it seems bizarre to
think that a single simple (or atom) can make a difference as to
whether `chairwise' applies.  But now remove another atom\,\ldots

If Unger's argument shows that the concept of `chair' is incoherent,
then it seems that the same argument shows the concept of `chairwise'
to be incoherent.  If so, then `chairwise' can apply to nothing in the
world.  There can be no chairwise arrangements of simples.  If there
are no chairwise arrangements of simples, then our belief that there
are chairs cannot be nearly as good as true.  If someone believes that
there is a ghost, that belief, according to Merricks, is not nearly as
good as true because there are no ghostwise arrangements of simples to
cause or justify the belief that there is a ghost.  If `chairwise' is
incoherent, then there are no chairwise arrangements of simples to
cause or justify the belief that there are chairs.  The belief that
there are chairs is (if Unger is right) simply false, just as is the
belief that there are ghosts.

Merricks cannot allow this conclusion.  But the sorites paradox can be
used to show that {\em any} vague term is incoherent.  Much of our
language is vague, but we are not therefore tempted to conclude that
our speech is rarely (if ever) coherent.  The problem of the sorites
paradox is a very general problem that requires a general theory of
vagueness.  Most metaphysical theories are threatened by the sorites
paradox; Merricks is not a special case.

With that in mind, it should not be surprising that Merricks does not
have full answer.  A full answer to the sorites paradox would be a
theory of vagueness.  Merricks is not trying to establish a theory of
vagueness, but it attempting to motivate a metaphysical thesis about
ordinary things.  That said, he does have a {\em partial} answer.  He
points out that while the sorites paradox does threaten the concept of
`chairwise', it does so in a less troubling way than the way in which
it threatens `chair'.  

How is the sorites paradox ``less troubling'' for chairwise
arrangements than for chairs?  Very roughly, it is because accepting
the vagueness of `chair' can lead us to {\em metaphysical} vagueness,
while accepting the vagueness of `chairwise' can only lead us to {\em
  linguistic} vagueness.  And linguistic vagueness is generally
considered to be less troubling than metaphysical vagueness.

We can get a sense of how this is so by imagining, as Merricks does,
the sorites paradox being played out as a series of questions.  Let us
suppose that there is a chair.  We then ask ourselves (Merricks asks
God), ``is `there is a chair' true?''  Presumably we will answer
``yes''.  Then we remove one atom (or simple, or other thing) from the
chair.  ``Is `there is a chair' true?''

If we agree that there is no single atom whose removal would destroy
the chair, then we must accept that at some point it becomes
indeterminate whether ``there is a chair'' is true.  If it is
indeterminate whether ``there is a chair'' is true, then it is
indeterminate whether there is a chair.  The idea that it could be
indeterminate whether something exists is taken by many to be
problematic (\textbf{CITE}).

We can compare this case with that of the things arranged chairwise.
Let us suppose that there are things arranged chairwise.  We ask ``is
`there things arranged chairwise' true?'' and answer ``yes''.  Then we
remove a thing and ask ``is `there things arranged chairwise' true?''
As in the chair case, if we are unwilling to allow that the removal of
a single thing could take us from ``yes'' to ``no'', then we must
admit that at some point it is indeterminate whether or not ``there
are things arranged chairwise'' is true.  It would then be
indeterminate whether there are things arranged chairwise.  {\em But
  this does have the result that it is indeterminate whether something
  exists.}  It may be perfectly determinate that there are the things
there are; all that is indeterminate is whether the things (which
there determinately are) are arranged in a certain way.  This kind of
indeterminacy seems less troubling than indeterminacy as to whether
something (a chair or anything else) exists.

Merricks therefore sees the problem posed by the sorites paradox as
less threatening to his chairwise arrangements than to chairs.  He
does not attempt to provide a solution, for that would require solving
the problem of linguistic vagueness.  But he finds the problem of
linguistic vagueness less troubling than the problem of metaphysical
vagueness, and he shows that the latter does not threaten his version
of nihilism.

\subsection{So what's the problem?}
\label{sorites-3}
We find ourselves wanting to hold three theses, which appear mutually
inconsistent:

\begin{enumerate}
  \item There is at least one chair (stone, cloud).
  \item If a chair (stone, cloud) exists, it must be made up of
    matter.
  \item If a chair (stone, cloud), exists, the removal of a single
    molecule (or otherwise insignificant quantity of matter) from it
    cannot destroy it or cause it to cease to exist.
\end{enumerate}

We seem to be clearly caught in a paradox; the only question is where
we have gone wrong.

But have we, in fact, gone wrong?  Peter Unger thinks that we are
right on target:

\begin{squote}
While Eubulides' contribution has often been labeled `the sorites
paradox', there is nothing here which is a paradox in any
philosophically important sense\,\ldots Accepting our negative
conclusions here does not mean important logical trouble for us; we
only think we have troubles while we refuse to admit their validity
(\citeyear[145]{unger1979}).
\end{squote}

Our situation is only paradoxical, says Unger, while we unreflectingly
cling to the first thesis.  If, however, we come to see that there are
no chairs (stones, clouds), then we happily escape paradox: if there
are no chairs (stones, clouds) to begin with, we do not have to worry
about what the addition or removal of small amounts of matter would do
to them; nor do we need concern ourselves with what they would be made
of.

But things are not quite so simple.  First, adding to the
implausibility of Unger's view, he must deny that our use of ordinary
terms like `chair' (`stone', `cloud') follow any sort of pattern or
display any competence at all.  Second, even if we manage to swallow
that consequence, Unger has no explanation as to why we believe that
there are chairs (stones, clouds).

\subsection{Competence and correctness}
\label{correct}
Setting aside whether or not expressions of propositions like ``that's
a chair'' are ever \emph{true}, it seems right to say that there are
at least correct and incorrect uses of the terms.  For a word like
`chair' (`stone', `cloud') we generally do not say that a child has
learned how to use it until she is capable of deploying it in certain
ways.  We admit that she understands what `chair' (`stone', `cloud')
means or what a chair (stone, cloud) is when she displays a certain
competence with the term.  If instead of using `chair' to refer to
chairs she used it to refer to dogs or people, we would say that she
is confused and attempt to correct her use.

But Unger maintains that this is all an illusion, and that there is no
such thing as the correct or incorrect use of a term like `chair'
(`stone', `cloud'):

\begin{squote}
Concerning words and kinds, now, we might say this.  First, we might
say that it is in connection with \emph{semantics} that our reasonings have
what are their most obvious implications and, second, that their most
obvious semantic implications concern certain \emph{sortal nouns}, namely,
those which purport to denote ordinary things.  Thus, it appears quite
obvious to us now that there will be no application to things for such
nouns as `stone' and `rock', `twig' and `log', `planet' and `sun',
`mountain' and `lake', `sweater' and `cardigan', `telescope' and
`microscope', and so on, and so forth.  Simple positive sentences
containing these terms will never, given their current meanings,
express anything true, correct, accurate, etc., or even anything which
is anywhere close to being any of those things
(\citeyear[148]{unger1979}).
\end{squote}

This seems simply bizarre.  On what grounds, then, do parents correct
their children with respect to their use of ordinary terms?  Are they
compelled by some irrational force to consider certain utterances
correct and others incorrect?  One may question whether or not we use
ordinary term entirely consistently, but it seems simply false to say that,
necessarily, we {\em never} use (or have used, or will use) ordinary
terms in correct, as opposed to incorrect, ways.  

The fact that Unger's position means that terms like `chair' are never
used correctly gives us reason to think his argument goes awry
somewhere.  However, attempting to identify the issue with his
argument and proposing a solution would be tantamount to attempting to
solve the problem of vagueness.  That is not something I will attempt,
at least not until we have examined the problem of the many.

\section{The problem of the many}
\label{many}
The `problem of the many', as Unger terms this second difficulty for
ordinary things, follows a similar line of reasoning as that of the
sorites paradox.  If we consider an ordinary thing---a cloud, for
instance---it is natural to think that it is made up of molecules.
There is probably then a set of molecules, the members of which make
up the cloud.  Call that set $S$.  Now consider $S_1$.  This is a set
of molecules that includes all of the members of $S$ as well as one
additional molecule.  Do the members of $S_1$ make up a cloud?  Surely
they are just as well suited to do so.  Now consider $S_2$\,\ldots

Because these numerous 'candidates' are equally (or nearly equally)
well suited to be clouds, we seem forced to conclude that there are
either many clouds where we supposed there to be one, or rather no
clouds at all:

\begin{squote}
No matter where we start, the complex first chosen has nothing
objectively in its favor to make it a better candidate for cloudhood
than so many of its overlappers are.  Putting the matter somewhat
personally, each one's claim to be a cloud is just as good, no better
and no worse, than each of the many others.  And, by all odds, each
complex has \emph{at least} as good a claim as any still further real
entity in the situation.  So, either \emph{all} of \emph{them} make it
or else \emph{nothing} does; in this real situation, either there are
many clouds or else there really are no clouds at all
\citep[415]{unger1980a}.
\end{squote}

The problem of the many can also arise by considering the {\em
  boundary} of a given cloud.  It is natural to suppose that a cloud
has a determinate boundary.  But if we look at the edge of the cloud,
where we suppose the boundary to be, ``we may find, side by side, or
themselves overlapping, a great many potential boundaries for
clouds\,\ldots if our alleged typical item {[}the cloud{]} is indeed
a typical cloud, then many of these candidates, millions at least, do
not fail to be clouds altogether but are clouds of some
sort'' \citep[420--421]{unger1980a}.

The pattern of argumentation is the same for both approaches to the
problem of the many.  For a given cloud, a certain set of members or a
certain boundary is supposed, and it is argued that a set or boundary
that differs minimally from the original must also make up or bound a
cloud.  The new set or boundary does not appear to differ from the
original in any relevant way; there seems no principled reason to deny
that if the first set's members make up a cloud, the second set's
members do too.  And since there are a great deal of very similar sets
and boundaries, we find ourselves threatened with a plurality of
clouds.

And of course, Unger does not rest content with applying the problem
of the many to clouds.  All ordinary objects get the same treatment;
he concludes that either there are a great many of them, or there are
none at all.  He claims, predictably, that the latter disjunct is
preferable.

\subsection{Is the problem of the many a problem for Merricks?}
\label{many-merricks}
It might seem that the problem of the many, if it makes things
difficult for chairs and other ordinary things, also causes trouble
for the chairwise arrangements upon which Merricks relies so heavily.
But there is an important disanalogy.  The problem of the many is only
a problem as long as we are unwilling to accept one of Unger's
disjuncts: that there are no chairs or that there are a plurality
where we took there to be one.  Whether or not it is part of the
meaning of `chair' that there is not an overlapping plurality of
chairs, it is simply unacceptable that this be the case.  (It is
likewise unacceptable that there be no chairs.)  But it {\em is}
acceptable, at least initially, that there be a plurality of chairwise
arrangements.  The idea that there is a plurality of different sets,
the members of which overlap and of which all are arranged chairwise,
is not particularly bizarre.  A potential difficulty for Merricks
would be in regard to his criterion of `arranged chairwise' (or
statuewise, as the case may be):

\begin{squote}
Atoms are \emph{arranged statuewise} if and only if they both have the
properties and also stand in the relations to microscopica upon which,
if statues existed, those atoms' \emph{composing a statue} would
non-trivially supervene (\citeyear[4]{merricks2001a}).
\end{squote}

If Merricks allows that there may be a plurality of statuewise
arrangements, then he is committed to the proposition that, if statues
existed, there may be pluralities of statues.  But Merricks may simply
take this as more evidence that his counterpossible conditional (``if
there were statues\,\ldots '') really is impossible.

There may be, however, a problem for Merricks with regard to the
notion of `singular thought'.  If Merricks says to me, ``those things
arranged chairwise are arranged very comfortably'', how can I know
which things he is talking about?  If there are numerous different
sets of things arranged chairwise, the chance that I am thinking of
the same things as Merricks is very low.  Are we really communicating,
then?  (Moreover, is it true that I am thinking of {\em any}
determinate set of things?  I certainly couldn't specify which
particular things I am thinking of.)

If the problem of the many is a problem for Merricks, then so much the
worse for his nihilism.  However, the problem is a problem for us as
well.

\section{Beliefs in things}
\label{u-belief}
When we examined previous versions of nihilism, we asked that their
proponents explained why, if there are no chairs, we nonetheless
believe that there are chairs.  Van Inwagen and Merricks both made a
claim to the effect that our beliefs are caused and (in some sense)
justified by arrangements of simples.  Although Merricks denies that
beliefs like ``there is a fine chair'' are strictly true, he agrees
with van Inwagen that they `get something right' in a way that beliefs
like ``there is a dancing chair'' do not.  Their explanation for our
belief that there are chairs is that they are caused (and justified)
by a `nearby' or somehow related fact---that there are simples
arranged chairwise.

It should therefore seem reasonable to demand a similar explanation
from Unger.  This explanation would be expected to take a different
form, depending on whether it accompanies the sorites paradox or the
problem of the many.  However, Unger offers no explanations.  

\subsection{Explaining our beliefs given the sorites paradox}
\label{expl-sorites}
Unger claims that our belief that there are chairs, like our beliefs
that there are other ordinary things, are not justified, `nearly as
good as true', or even coherent.  Unger says that ``terms for ordinary
things are incoherent [and] cannot apply to anything real''
\citep[147]{unger1979}.

Unger should not deny that believe that there are ordinary things.  If
our beliefs about tables and chairs are invariably false (even
incoherent), then what causes us to form these beliefs?  Why do we
believe in ordinary things to begin with?

Unlike van Inwagen and Merricks, Unger does not offer an explanation.
Having denied the existence of all ordinary things, he makes no
attempt to explain why we have so many false beliefs or what gives the
impression of coherency to our use of them in communication.  He seems
almost to revel in the strangeness of his position:

\begin{squote}
Now, it must of course be admitted that these arguments undermine the
possibility of any endeavor I should try to propose, or even the
putative thought that I should propose anything, just as all of my
putative essay is undermined.  But even so, I shall (incoherently)
propose that what we have now to do is invent new expressions which
are not inconsistent ones, and by means of which we may, to some
significant extent, think coherently about concrete reality
(\citeyear[544]{unger1980b}).
\end{squote}

I do not have an argument against the proposition that nearly all of
our language is hopelessly incoherent.  I do have a very hard time
believing that this is true, however; I'm not sure Unger believes it
himself.

\subsection{Explaining beliefs given the problem of the many}
\label{expl-many}
When presenting the problem of the many, Unger declares that he
prefers to maintain that there are no chairs, rather than that there
are pluralities of chairs.  Having made this nihilistic claim,
however, he does not offer an explanation of why we do in fact believe
that there are chairs.  However, he does not deny (at least not
explicitly) that there are things arranged chairwise, so Merricks'
explanation (see section \ref{connection}) might serve for Unger too.
We might propose, on Unger's behalf, that we believe that there are
chairs because there are things arranged chairwise.

In this case, however, I am tempted to repeat my arguments against
Merricks (section \ref{dogbush}).  I claimed that if there are things
arranged chairwise, {\em then there are chairs}.  If our belief that
there are chairs is caused by things arranged chairwise, then it is a
true, not a false, belief.

If we reject Unger's conclusion that there are no chairs, we are still
faced with a problem.  For we seem to run ourselves into the other
disjunct of the problem of the many.  If there are chairs, then there
are pluralities of chairs where we expect there to be only one.

This is an unacceptable conclusion, but not due to any explanatory
deficiency.  If we supposed that there was a plurality of chairs, then
{\em that} would explain why we believe that there are chairs.

\section{Referring to the many}
\label{refer}
Above (section \ref{many-merricks}) I mentioned that notions of
singular thought are threatened by the problem of the many.  More
should be said on this.

\ifstandalone
\bibliography{everything}
\bibliographystyle{ChicagoReedweb}
\end{spacing}
\fi
\end{document}


\chapter{Conclusion}
\label{concl}
\chapterpig{Conclusion}
%         \addcontentsline{toc}{chapter}{Conclusion}
%	\chaptermark{Conclusion}
%	\markboth{Conclusion}{Conclusion}
%	\setcounter{chapter}{4}
%	\setcounter{section}{0}
	
Here's a conclusion, demonstrating the use of all that manual
incrementing and table of contents adding that has to happen if you
use the starred form of the chapter command.  The deal is, the chapter
command in \LaTeX\ does a lot of things: it increments the chapter
counter, it resets the section counter to zero, it puts the name of
the chapter into the table of contents and the running headers, and
probably some other stuff.

So, if you remove all that stuff because you don't like it to say
``Chapter 4: Conclusion'', then you have to manually add all the
things \LaTeX\ would normally do for you.  Maybe someday we'll write a
new chapter macro that doesn't add ``Chapter X'' to the beginning of
every chapter title.

\section{More info}
And here's some other random info: the first paragraph after a chapter
title or section head \emph{shouldn't be} indented, because indents
are to tell the reader that you're starting a new paragraph.  Since
that's obvious after a chapter or section title, proper typesetting
doesn't add an indent there.


%If you feel it necessary to include an appendix, it goes here.
%    \appendix
%      \chapter{The First Appendix}
%      \chapter{The Second Appendix, for Fun}


%This is where endnotes are supposed to go, if you have them.

  \backmatter % backmatter makes the index and bibliography appear
              % properly in the t.o.c...

% Make my bibliography be called "Bibliography" and not "References" (or "Works Cited" or...):
% \renewcommand{\bibname}{Works Cited}
    \bibliographystyle{ChicagoReedweb} % there are a variety of styles available; 
% replace ``plainnat'' with the style of choice.  You can refer to files in the bsts or APA 
% subfolder, e.g.  
% \bibliographystyle{APA/apa-good}  % or
% \bibliographystyle{bsts/mla-good} 

% if you're using bibtex, the next line forces every entry in the bibtex file to be included
% in your bibliography, regardless of whether or not you've cited it in the thesis.
    %\nocite{*}
    \bibliography{everything}

\end{spacing}
% Finally, an index would go here\,\ldots\,but it is also optional.
\end{document}
