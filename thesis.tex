% This is the Reed College LaTeX thesis template. Most of the work 
% for the document class was done by Sam Noble (SN), as well as this
% template. Later comments etc. by Ben Salzberg (BTS). Additional
% restructuring and APA support by Jess Youngberg (JY).
% Your comments and suggestions are more than welcome; please email
% them to cus@reed.edu

\documentclass[12pt,twoside]{reedfancy}
\usepackage{standalone}
\usepackage{anyfontsize}
\usepackage{graphicx,latexsym} 
\usepackage{amssymb,amsthm,amsmath}
\usepackage{tipa}
\usepackage{longtable,booktabs,setspace} 
\usepackage{verbatim}
\usepackage{url}
\usepackage{natbib}
\usepackage{enumitem}
\usepackage{url}
\usepackage{hyperref}
\setcitestyle{aysep={}}
\synctex=1

\DeclareSymbolFont{symbolsC}{U}{txsyc}{m}{n}
\DeclareMathSymbol{\strictif}{\mathrel}{symbolsC}{74}
\DeclareMathSymbol{\boxright}{\mathrel}{symbolsC}{128}

\newenvironment{squote}{%
	\begin{spacing}{1}
	\begin{list}{}{%
	\setlength{\labelwidth}{0pt}%
	\rightmargin\leftmargin%
	}
	\item\relax
	}{%
	\end{list}%
	\end{spacing}
	}

\newcommand{\stager}[4]%
{%
	\begin{spacing}{1}%
	\vspace{0pt}
		\begin{description}[style=nextline, noitemsep,
                    parsep=0pt, topsep=0pt, leftmargin=15mm,
                    itemindent=-10mm, font=\mdseries]
			\item[\textsc{#1} \emph{#2}] #3
			\item[]%
			\begin{flushright}#4\end{flushright}
		\end{description}%
	\end{spacing}%
}

\newcommand{\stage}[3]%
{%
	\begin{spacing}{1}%
	\vspace{0pt}
		\begin{description}[style=nextline, parsep=0pt,
                    leftmargin=15mm, itemindent=-10mm, font=\mdseries]
			\item[\textsc{#1} \emph{#2}] #3
		\end{description}%
	\end{spacing}%
}

\newenvironment{inq}{\vspace{0pt}%
	\begin{list}{}{%
	\setlength{\labelwidth}{0pt}%
	\setlength{\leftmargin}{2.5\oddsidemargin}%
	\setlength{\rightmargin}{\leftmargin}}
	\begin{spacing}{1}
	\item[]%
	}{
	\end{spacing}
	\end{list}
	\vspace{10pt}
	%\noindent%
	}
	
\newenvironment{epigram}{%
	\begin{minipage}[c]{0.75\textwidth}
	\vspace{2.5in}
	\begin{spacing}{1}
	\begin{list}{}{%
	\setlength{\labelwidth}{0pt}
	\setlength{\leftmargin}{1.4in}
	\setlength{\rightmargin}{.25in}}
	\item[]
	}{%
	\end{list}
	\end{spacing}
	\end{minipage}
	\newline
	}

% moved to reedfancy.cls:
%\def\thetitle{oink}
%\renewcommand{\firstmark}{\thetitle}
%\newcommand{\chapterpig}[1]{\def\thetitle{#1}}

\title{Why are there chairs?}
\author{Alexander A. Dunn}
% The month and year that you submit your FINAL draft TO THE LIBRARY
% (May or December)
\date{May 2012}
\division{Philosophy, Religion, Psychology, and Linguistics}
\advisor{Paul Hovda}
\department{Philosophy}

\setlength{\parskip}{0pt}

%%%%%%%%%%%%%%%%%%%%%%%%%%%%%%%%%%%%%%%%%%%%

\begin{document}

  \maketitle
  \frontmatter % this stuff will be roman-numbered
  \pagestyle{empty} % this removes page numbers from the frontmatter

%% \begin{epigram}
%% A curious thing about the ontological problem is its simplicity.  It
%% can be put in three Anglo-Saxon monosyllables: ``What is there?''  It
%% can be answered, moreover, in a word---``Everything''---and everyone
%% will accept this answer as true. \\ \\ \textsc{W.\,V.\,O.\ Quine
%%   \citeyearpar{quine1948}}
%% \end{epigram}

\begin{spacing}{1.25}

% Acknowledgements (Acceptable American spelling) are optional
% So are Acknowledgments (proper English spelling)
%% \chapter*{Acknowledgements}
%% \textsc{Thank you people.}

% The preface is optional
% To remove it, comment it out or delete it.
%    \chapter*{Preface}
%	This is an example of a thesis setup to use the reed thesis document class.

    \tableofcontents
% if you want a list of tables, optional
   % \listoftables
% if you want a list of figures, also optional
   % \listoffigures

% If your abstract is longer than a page, there may be a formatting issue.
\chapter*{Abstract}
Theories of metaphysical nihilism claim that there are no (or nearly
no) objects with parts.  Such theories claim that there are no chairs,
houses, mountains; some versions claim that there are no people.  The
philosophers who make these claims have trouble explaining why we
nonetheless believe that there are chairs.  The only successful
explanation is compatible both with nihilism and with metaphysical
universalism.  Universalism is that claim that for every set of
things, there is something else made up of those things; this thesis
is intuitively more plausible than nihilism.  But if we assume that
universalism is true, and if we do not presuppose four-dimensionalism,
we have to choose between two different versions of universalism: one
that posits a plurality of co-located (entirely overlapping) objects,
or one that denies that things (including chairs) can change their
parts over time.

\mainmatter % here the regular arabic numbering starts
\pagestyle{fancyplain} % turns page numbering back on
  
\chapter*{Introduction}
\chapterpig{Introduction}
\addcontentsline{toc}{chapter}{Introduction}
\chaptermark{Introduction}
\markboth{Introduction}{Introduction}
\documentclass[11pt]{article}
\usepackage{standalone} \newif\ifstandlone \standalonetrue
\usepackage[left=1.75in, right=1.75in, top=1.25in, bottom=1.25in]{geometry}
\geometry{letterpaper}
\usepackage{graphicx}
\usepackage{enumitem}
%\usepackage{amssymb}
\usepackage{amsmath}
\usepackage{epstopdf}
\usepackage{verbatim}
\usepackage{setspace}
\usepackage{natbib}
\setcitestyle{aysep={}}
\usepackage{url}
\usepackage{hyperref}
\synctex=1

\DeclareSymbolFont{symbolsC}{U}{txsyc}{m}{n}
\DeclareMathSymbol{\strictif}{\mathrel}{symbolsC}{74}
\DeclareMathSymbol{\boxright}{\mathrel}{symbolsC}{128}

\newenvironment{squote}{%
\begin{spacing}{1}
\begin{list}{}{%
\setlength{\labelwidth}{0pt}%
\rightmargin\leftmargin%
}
\item\relax
}{%
\end{list}%
\end{spacing}
}

\title{Introduction}
\author{Alexander A. Dunn}
\begin{document}
\ifstandalone
\maketitle
\begin{spacing}{1.5}
\fi

%\section{How to deny what is true}
It is true that there are chairs.  This, as far as I'm concerned, is
obvious.  If someone says that it is not true that there are
chairs---that there are not chairs---then it is clear to me that
somehow they have gone astray.  If they have an argument for this
conclusion, there must be something wrong with the argument.  There
must be something wrong because it is true---obviously true---that
there are chairs.

In many parts of what follows, I will often say things such as ``I
believe that there are chairs''.  This should not be taken as an
unwillingness to assert a stronger claim---that there are chairs.  The
stronger claim is, I have said, obviously true.  But I will use the
weaker claim---```I believe that there are chairs''---because while
the philosophers I am criticizing deny that there are chairs, they do
not deny that I believe that there are chairs.  And I am going to
argue that this fact alone---that I believe that there are
chairs---causes some trouble for their views, and gives us reason to
doubt their extraordinary conclusions.

So I believe that there are chairs.  I also believe that there are
desks, and desk lamps, and doors, and doorways, and houses, and
gardens, and plants, and people.  Such things, and many others, are
commonly termed `ordinary things'.  This term is extremely vague in
its application, but is taken to refer to macroscopic objects, such as
those listed above, that are parts of our everyday lives.

Many philosophers have denied that ordinary things exist.  Until
recently, such a denial was generally a consequence of the
philosopher's views on other matters.  If a philosopher claimed that
there was no external world, or that the world was not at all like it
appears, then they might deny that there were any physical things, or
any things that exist outside the mind, or anything at all.  It would
follow from such a claim that there would be no ordinary things, like
chairs.  But the philosopher would not be specifically interesting in
denying that chairs exist.  They were interested in denying that {\em
  anything} exists; the denial of chairs was a minor consequence.

In the past 30 years, however, philosophers have begun to construct
arguments specifically aiming to show that there are no ordinary
things.  (Peter Unger was one of the first, with the aptly titled
paper, ``There are no ordinary things''.)  These philosophers do not
deny that there is an external world, or that it contains many
physical things; these propositions are readily granted to be true.
But the philosophers are unwilling to admit that such a world
does---or even possibly could---contain chairs.

Most philosophers making this sort of claim admit that it is strange
and unintuitive.  But they believe that the benefits of denying the
existence of ordinary things outweighs the costs.  Different
philosophers cite different benefits: consistency with regard to our
notion of composition, theoretical simplicity, or greater coherency
among our beliefs.  

The benefits do not out-weight the cost.  Moreover, I am unable to
imagine that any argument could convince me that there are no ordinary
things.  I believe that any argument that has the nonexistence of
chairs as a consequence is flawed.  Whether or not we can immediately
identify the flaw in the argument, the fact that it entails a
falsehood shows that something has gone amiss.

It will be objected that this is merely a fact about myself; other
philosophers are perfectly willing to deny that there are chairs.  (It
is another fact about me that I doubt that they really believe that
there are no chairs.)  It may be argued that the fact that I consider
``there are chairs'' to be true regardless of arguments against its
truth shows that I consider it to be, in some sense, a conceptual
truth.  It may be further argued that, since there are philosophers
willing to deny that ``there are chairs'' is true, what I mean by
``there are chairs'' is something different than what these
philosophers mean by ``there are chairs''.  We may be thought to be
using our words in different ways.

In section \ref{verbal} I will argue that we are {\em not} using our
words in different ways.  When I say ``there are chairs'' and someone
else says ``there are not chairs'', we are having a real disagreement.
Moreover, we are disagreeing in English; there is no special
`ontological language' in which we do metaphysical philosophy.

In section \ref{stroud} I will argue that any philosopher who attempts
to deny that there are chairs should be able to explain why we
nonetheless believe that there are chairs.  This is, I think, a
reasonable request, but it is surprisingly hard to satisfy.  The
difficulties that `nihilistic' philosophers have in explaining why we
believe that there are chairs should give us reason to suspect their
conclusions.

But even if we show that there are problems with the arguments of
philosophers who deny that ordinary things exist, we have not thereby
proved that they {\em do} exist.  The philosophers who say that there
are no chairs are motivated by a number of questions about the nature
of ordinary things.  For example, why are there chairs and tables, but
not chair-tables (single objects composed of an adjacent table and
chair)?

In section \ref{universe}, however, I will argue that some of the
considerations that philosophers take to be good reasons to deny that
chairs exist are not good reasons at all.  In effect, these
philosopher take the apparent non-existence of chair-tables to tell
against the existence of chairs and tables.  On the contrary, as we
will see, the {\em obvious} existence of chairs and tables tells for
the existence of chair-tables.  Additionally, I will argue that teams,
families, crews and other `groups' exist and have parts, just like
`material objects' like chairs.  

In section \ref{parts} I will argue (following Kit Fine) that things
can be `parts' of other things in many different ways.  The way that
something is part of a chair differs from the way a thing is part of a
group.  However, I will argue that the way in which a thing is part of
a group is the way that things are parts of sets---I will argue that
groups can be identified with sets.  Since sets cannot change their
membership over time, the set referred to by a term for a group (`the
Supreme Court') will change over time.  The most interesting
consequence of this is that the {\em identity conditions} for a group
over time---the conditions in which a certain set is identical with
the group---are wholly conventional.  Even more interesting is that
this seems to be the case with ordinary things like chairs as well;
whether and how a chair persists over time is largely up to us.

Throughout this thesis, there are certain things I will {\em not}
presuppose.  First, I will take no stand on whether or not things have
`temporal parts'.  I am not sure I fully understand the doctrine of
temporal parts, but it is often summarized thus: if a thing has
temporal parts, then for each time at which it exist, there exists at
that time (and only at that time) another thing---a temporal part or
`slice' of the larger object.  The (temporally) larger object is
somehow `built up' from these temporally smaller part.  If a thing
does not have temporal parts, then it is not divided into temporal
`slices'---it is ``wholly present'' at every moment of its existence.
Whatever this debate comes to, I will try to avoid relying on the
truth or falsity of the doctrine of temporal parts.

Second, I will not presuppose {\em eternalism}.  Eternalism is,
roughly, the view that past and future time are just as `real' as the
present.  An analogy is often drawn with space; what's behind me and
in front of me is just as real as what is under me.  There is nothing
special about `here' rather than `there'.  Likewise the eternalist
claims that `now' is no more special that `then'.  Eternalism is
generally opposed to {\em presentism}, which is the view that only the
present is real.  The presentist and the eternalist both agree that
there {\em were} dinosaurs, but for the eternalist there is a sense in
which there {\em are} dinosaurs (they are just not `now').  Again, I
will attempt to avoid committing myself to either of these positions.

\ifstandalone
\end{spacing}
\bibliography{everything}
\bibliographystyle{ChicagoReedweb}
\fi
\end{document}


% The three lines above are to make sure that the headers are right,
% that the intro gets included in the table of contents, and that it
% doesn't get numbered 1 so that chapter one is 1.

\chapter{What do I mean when I say there are chairs?}
\chapterpig{What do I mean when I say there are chairs?}
\documentclass[11pt]{article}
\usepackage{standalone} \newif\ifstandlone \standalonetrue
\usepackage[left=1.75in, right=1.75in, top=1.25in, bottom=1.25in]{geometry}
\geometry{letterpaper}
\usepackage{graphicx}
\usepackage{enumitem}
%\usepackage{amssymb}
\usepackage{amsmath}
\usepackage{epstopdf}
\usepackage{verbatim}
\usepackage{setspace}
\usepackage{natbib}
\setcitestyle{aysep={}}
\usepackage{hyperref}
\usepackage{url}
\synctex=1

\DeclareSymbolFont{symbolsC}{U}{txsyc}{m}{n}
\DeclareMathSymbol{\strictif}{\mathrel}{symbolsC}{74}
\DeclareMathSymbol{\boxright}{\mathrel}{symbolsC}{128}

\newenvironment{squote}{%
\begin{spacing}{1}
       	\begin{list}{}{%
\setlength{\labelwidth}{0pt}%
\rightmargin\leftmargin%
}
\item\relax
}{%
\end{list}%
\end{spacing}
}

\title{What do I mean when I say there are chairs?}
\author{Alexander A. Dunn}
\begin{document}
\ifstandalone
\maketitle
\begin{spacing}{1.25}
\fi

\label{verbal}
Philosophers such as Eli Hirsch have argued that metaphysical
disputes---such as whether there are chairs---are verbal disputes that
have no real import.  Hirsch claims that ``there are chairs'' is
obviously true in English.  If a philosopher says ``there are no
chairs'' on the grounds that there is nothing in the world but
partless atoms, Hirsch will {\em interpret} them as meaning something
like ``there are no partless atoms that are also chairs''.  This
interpretation allows Hirsch to maintain that both philosophers are
saying true things, and are not really disagreeing at all.  Ted Sider
and others have replied by attempting to invent a new language,
Ontologese, in which it is not obviously that ``there are chairs'' is
true.  In this section I will argue that Sider's `Ontologese gambit'
cannot work, but that Hirsch's argument is flawed as well.

\section{Verbal disputes}
\label{hirsch}
% Is Sider's Ontologese just physics?
Some philosophers maintain that the recent disputes over the existence
of ordinary things are {\em merely verbal disputes}.  Suppose one
philosopher claims that nothing is part of something else.  This
philosopher will say things like ``it is not true that there are
chairs'' Suppose another philosopher rejects this view.  This
philosopher will says ``there are chairs''.  It seems obvious that
these philosophers are disagreeing.  But according to some, they are
not disagreeing at all.  

Eli Hirsch claims that our two philosophers are engaged in a {\em
  verbal dispute}.  A verbal dispute is one that is somehow not
substantive; Hirsch's paradigm case of a verbal dispute is over
whether glasses are cups:

\begin{squote}
  I know someone, whom I'll call $A$, who claimed that a standard
  drinking glass is a cup.  ``Just as a cat is a kind of animal,'' she
  said,``a glass is a kind of cup.''  Everyone else whom I've asked
  about this agrees with me that a glass is not a cup.  Clearly, this
  dispute is, in some sense, merely about
  language~(\citeyear[69]{hirsch2005}).
\end{squote}

To see that this dispute is verbal, Hirsch instructs us to do the
following:

\begin{enumerate}
  \item Take what each disputant says.
  \item Postulate a community that agrees with that disputant.
  \item Interpret each community's language so that the relevant
    utterances come out true.
  \item Interpret each disputant as speaking the language of their
    community.
\end{enumerate}

For example, when $A$ says ``There is a cup on the table'', Hirsch
would say instead ``There is a cup or there is a glass on the table''.
Postulating a community that agrees with $A$ (the $A$-community) and
imagining a community that agrees with Hirsch (the $H$-community), we
may assign the truth-conditions this way:
\begin{enumerate}[itemindent=25pt, label=(T)]
    \item ``There is a cup on the table'' is true in $A$-English iff
    ``There is a cup or glass on the table'' is true in $H$-English
\end{enumerate}
($A$ is the dialect of English spoken in the $A$-community and $H$ is
the dialect spoken in the $H$-community.)

One might object that we have not shown that $A$ and $H$ are in fact
speaking different languages.  ``All you have shown,'' $H$ might say,
``is that we can imagine them speaking a language in which what they
say is true.  But you have not shown that they {\em are} speaking such
a language.  What $A$ says sounds like normal English to me, and in
English, what she says is simply false.''

What justifies us in postulating $A$- and $H$-English is simply that
what $A$ {\em means} by `cup' is not what $H$ means by `cup'.  $A$
uses `cup' to refer to all the things that $H$ uses `cup' to refer to,
but she also uses `cup' to refer to those things that $H$ refers to
exclusively by `glass'.  What $A$ means by `cup' is what $H$ means by
`cup or glass'.

It is this difference in {\em meaning}, as well as in
truth-conditions, that allows us to postulate `$A$-English' and
`$H$-English' and to conclude that the dispute between $A$ and Hirsch
is verbal. If we understand $A$ to mean by ``There is a cup on the
table'' what we mean by ``There is a cup or a glass on the table'',
then $A$ is not saying something false.  We thought that they were
disagreeing over what was on the table, but they simply meant
different things by their words.

Hirsch does not explicitly claim that meaning is reducible to
truth-conditions, but he is clearly relying on a close connection
between the two:

\begin{squote}
When I speak throughout this paper about interpreting a language this
is always to be understood in the narrow sense of assigning truth
conditions.  I leave it open what there is to understand a language
beyond knowing the truth conditions of its sentences, but, whatever
this additional element may be, it will have a bearing on my argument
only insofar as it might affect the plausibility of certain
truth-condition assignments \citeyearpar[72]{hirsch2005}.
\end{squote}

Having given this warning, Hirsch speaks freely of meaning instead of
mere truth-conditions.  When imaging himself as David Lewis
interpreting Roderick Chisholm, he suggests that we ``reject the
assumption that the RC-speakers [Roderick Chisholm's `community
  language'] mean what we [speakers of the David Lewis language]
mean'' \citeyearpar[76]{hirsch2005} and advocates ``{\em semantically
  restricted quantifiers}'' \citeyearpar[76, his
  emphasis]{hirsch2005}.  When discussing these `RC' quantifiers, he
goes on to say this:

\begin{squote}
The RC-speakers will, of course, make the platitudinous disquotational
assertion, ``If something exists it is referred to by the word
`something'.''  Given what they {\em mean} by ``something'' this
sentence is trivially true \citeyearpar[77, my emphasis]{hirsch2005}.
\end{squote}

Without committing Hirsch to exactly the following thesis, I think he
would accept some claim along the lines of ``if it is necessary that
propositions $p$ and $q$ have the same truth-value (either true or
false), then $p$ and $q$ mean the same thing.''  Hirsch seems at least
sympathetic to some modification of this.  We can express it more
formally as \ref{v}:
\begin{enumerate}[itemindent=25pt, label=(V)]
    \item If $\square$($p$ is true iff $q$ is true), then $p$ and $q$ mean
    the same thing. \label{v}
\end{enumerate}

This is a stronger thesis than Hirsch needs to accept.  Moreover, it
is probably not true; it seems to entail that there is only one
necessary proposition.  But {\em something like this} underlies
Hirsch's argument.

\subsection{Charity}
\label{charity}
But Hirsch's conclusion that $A$ means by `cup' what he means by `cup
or glass' does not follow from~\ref{v} alone.  To see why this is so,
recall the dispute between $A$ and $H$ over whether a glass is a cup.
Suppose that $H$ accepts~\ref{v}.  He might say, ``We are both
speaking English.  In English, `cup' does not mean the same as `cup or
glass'. Therefore, by {\em modus ponens}, it is not necessarily true
that `There is a cup on the table' is true iff `There is a cup or
glass on the table' is true.  For when there is a glass on the table,
the latter proposition is true and the former false.  And yet $A$
insists on treating these as somehow identical.  She affirms one if
and only if she affirms the other.  Evidently, she is deeply
confused.''

$H$ could accuse $A$ of making fundamental mistakes about language or
perception, and $A$ could level the same accusation at $H$.  But
Hirsch thinks that this is a poor way of understanding the
debate.  Instead of supposing that ``the other has some incurably
irrational tendency to make a priori mistakes about what they perceive
in front of their faces''~\citep[78]{hirsch2005}, we should pursue a
policy of {\em interpretive charity}:

\begin{squote}
Why is it plausible to suppose that in the $A$-language the word
`cup' doesn't mean what it means in our language, so that the sentence
`A glass is a cup'' is true in that language?  The basic answer to
this question comes out of a widely accepted principle of linguistic
interpretation that has often been called the principle of charity.''
This principle, put very roughly, says that, other things being equal,
an interpretation is plausible to the extent that its effect is to
make many of the community's shared assertions come out true or at
least reasonable~(\citeyear[71]{hirsch2005}).
\end{squote}

We can see the correctness of this principle by imagining a resolution
to the dispute between $A$ and $H$.  Any neutral arbitrator should
sit them down and explain things thus:  ``Now $A$, you said that just
as a cat is a kind of animal, a glass is a kind of cup.  $H$, you
probably disagree; you think cups and glasses are like cats and
dogs.  But given that $A$ thinks of cups and glasses like she does, you
should remember when she says `cup', that she just means anything that
you'd call either a cup or a glass.  And $A$, when $H$ talks about
cups, remember that he means only the cups that aren't glasses.''

Unless $A$ and $H$ are simply looking for something to bicker about,
they will agree that they each mean these different things by `cup';
having recognized this, the argument dissolves.  The only question
remaining is which meaning is shared by the majority of English
speakers~\citep[70]{hirsch2005}.

Hirsch diagnoses verbal disputes by applying his principle of
interpretive charity alongside a version of~\ref{v}.  If he can
interpret the propositions of two disputants so as to make all come
out true, and if these equivalences in truth-conditions correspond
with equivalent meanings, then Hirsch has shown a dispute to be
verbal.  Unfortunately, while this method works well for his test case
involving $A$ and $H$, it does not appear to succeed when applied to
the metaphysical disputes that are his primary subjects.  He does
manage to interpret the apparently conflicting propositions of the
competing metaphysicians so that neither contradicts the other;
however, his truth-conditional interpretations fail to preserve
meaning.

Consider Hirsch's analysis of the dispute between a
four-dimensionalist and a mereological essentialist.  Hirsch uses
David Lewis and Roderick Chisholm as mascots for these respective
positions.  We are to suppose that Chisholm (hereafter referred to as
`$RC$') and Lewis (`$DL$') are sitting at a table.  Upon the table is
a pencil. $DL$ claims that objects have temporal parts, and that any
set of objects and/or temporal parts has a `fusion' (in other words,
for any set of objects and/or temporal parts, there is another object
composed of the things in that set).  $RC$, on the other hand, claims
that objects do not have temporal parts (there are no such things);
the only physical objects are masses of matter.

$DL$ and $RC$ obviously have different things to say about the pencil
on the table.  $DL$ claims that a temporal part of the eraser from
$t_{1}$ fuses with a temporal part of the wood from $t_{2}$; thus $DL$
says that ``There is something on the table that is pink, then brown.
$RC$ denies this asserting that ``There is nothing on the table that
is pink, then brown''.  Both, however, agree that ``There is something
that is pink, then something that is brown''.

Hirsch imagines himself as a four-dimensionalist trying to interpret
$RC$, and then as a mereological essentialist interpreting $DL$.  He
claims that from the point of view of $DL$, the quantifiers in
$RC$-English are {\em semantically restricted}; ``the rough idea seems
to be that the range of the $RC$-quantifiers excludes any physical
object that is composed of matter but is not itself a mass of
matter''~\citep[76]{hirsch2005}.

Hirsch then adopts the perspective of $RC$.  He finds that speakers of
$DL$-English consider the sentence ``There is first something that is
$F$ and later there is something that is $G$'' to be `` (a priori
necessarily) equivalent'' to ``There is something that is first $F$
and later $G$''.  He says that ``we should make the charitable
assumption that in $DL$-English these sentences really are
equivalent''~(\citeyear[78]{hirsch2005}).

Given the mereological axioms that $DL$ has adopted, it is
uncontroversially true that whenever there is something that is pink,
then something that is brown, they fuse to create something that is
first pink and then brown.  Given $RC$'s doctrines, it is also
true that everything (every physical thing) he claims to exist is a
mass of matter.

Having completing his `charitable' interpretation, Hirsch applies his
version of~\ref{v} and claims that these truth-equivalent propositions
{\em mean} the same thing.  He claims that $DL$ means the same thing
by ``There is first something that is $F$ and later there is something
that is $G$'' and by ``There is something that is first $F$ and later
$G$.''  He also claims that when $RC$ says ``something'' he means
``something that is either a mass of matter or is not composed of
matter''~(\citeyear[76]{hirsch2005}).

Hirsch concludes that when $DL$ says ``There is something here that is
first pink and then brown'', he should be taken to mean that there is
something that is pink and then something that is brown.  And $RC$ can
agree with that.  Hirsch also claims that when Chisholm says that
``there is nothing here that is first pink and then brown'' he means
that there is no mass of matter that is first pink and then
brown.  $DL$ will not deny that.  So Hirsch concludes that $DL$ and
$RC$ are engaged in a verbal dispute; they are simply talking past
each other.

Hirsch's analysis of $DL$ is dubious.  $DL$ will of course admit that
these sentences are truth-conditionally equivalent, but we can imagine
him saying ``Do they mean the same thing?  Well, no.  The second
proposition---`there is something that is first $F$ and later
$G$'---entails that there is one thing that is pink then brown; the
first---`there is first something that is $F$ and later there is
something that is $G$'---does not (in fact, it implies that they are
not the same thing).''

If that seems dubious, Hirsch's analysis of $RC$ seems downright
false.  ``Does `something' {\em mean} `something that is a mass of
matter or not composed of matter'?''  $RC$ might ask.  ``Of course
not!  If I meant that, then by ``there is not something that is not a
mass of matter'' all I would mean would be ``there is not something
that is a mass of matter that is not a mass of matter''.  That's
trivially true, and thoroughly uninteresting.  But I'm not speaking in
tautologies; I'm expounding a controversial metaphysical thesis;
namely, that {\em everything that exists is either a mass of matter or
  is not composed of matter}.  Only after having done some rigorous
metaphysics can we affirm that `Something exists' is true iff `Some
mass of matter or immaterial object exists'.  That claim reports a
discovery about the world, not about what I mean by my words.''

\subsection{Hostile interpretations}
\label{hostile}
Hirsch's claim, that ontological disputes like the above are merely
verbal, relies on a controversial theory of meaning.  If a
truth-conditional theory of meaning is correct (or largely so), then
Hirsch's interpretations of the disputing metaphysicians would also be
correct.  But there would still be a sense in which the verbal
disputes as to whether there are chairs differs from the disputes as
to whether glasses are cups.  Above I said that two people arguing
over whether glasses are cups will agree that they do not mean the
same thing by `cup'.  They will agree that they are engaged in a
verbal dispute.

The metaphysicians are not so cooperative.  Even after Hirsch has
diagnosed their dispute as verbal, the disputants maintain that they
are {\em not} engaged in a verbal dispute.  Hirsch's interpretations
are therefore \emph{hostile}.  They diverge substantially from the
expectations of the speakers.  Even if our ontologists tell Hirsch
``we're \emph{not} engaged in a verbal dispute'', Hirsch will be
unmoved:

\begin{squote}
The presumption of charity is supposed to be an a priori principle
that is partially constitutive of linguistic meaning.  Insofar as the
disputing ontologists assert the sentence, ``We are not engaged in a
verbal dispute,'' this sentence will figure, together with all the
other asserted sentences, in arriving at the most charitable
interpretation.  I would suspect that meta-level, quasi-technical
(self-aggrandizing) assertions probably have low priority as
supplicants for charity.  In any case, it can't be seriously suggested
that the charitable presumption in favor of the correctness of this
one assertion threatens to trump the presumption in favor of all of
the other assertions made by the ontologists (\citeyear{hirsch2008}).
\end{squote}

As long as Hirsch can produce truth-condition assignments that make
the relevant assertions of both sides true, there seems to be nothing
they can do to convince him that they are having a real argument
(other than convince him that his truth-conditional theory of meaning
is mistaken).

\section{Ontologese}
\label{ontologese}
Or isn't there?  Sider's strategy for dealing with Hirsch is to
stipulate---in concert with other metaphysicians---that they use
`$\exists$' and `$\forall$' in a special sense:

\begin{squote}
{[}The philosophers{]} should stipulate that their quantifiers are to be
understood as theoretical terms (and so are not subject to the same
level of metasemantic pressure from charity that governs terms like
`sofa' and `game') that stand for whatever joint-carving notion is in
the vicinity (\citeyear[9]{sider2011b}).
\end{squote}

By explicitly \emph{intending} to mean by `$\exists$' and `$\forall$'
whatever `joint-carving' notions are `in the vicinity', Sider hopes to
evade Hirsch's argument from interpretive charity.  The idea is that
charity can be put on hold.  Sider hopes that when {\em he} says
``there are no chairs'', it will be true (if it is true) only because
a ``joint-carving'' quantifier does not range over chairs.

\subsection{Sider's retreat}
\label{retreat}
The first thing to notice about Sider's argument is that it involves
conceding that {\em Hirsch's theory of meaning is correct}.  Sider
implicitly grants that some sort of truth-conditional theory of
meaning is true.  There is no other reason why Sider should feel the
need to build a language with the specific purpose of avoiding
interpretive charity.  As I pointed out above, Hirsch's
conclusion---that metaphysical disputes are verbal---requires not only
a principle of interpretive charity, but a truth-conditional theory of
meaning---something like \ref{v}.

A principle of charity alone cannot secure Hirsch's conclusion.  When
Hirsch charitably interprets the metaphysicians, it is because his
`charitable' interpretations involve only the assignment of
truth-conditions that he can claim that the metaphysicians mean
different things.  We say above that Hirsch recommends David Lewis and
other universalists to interpret Roderick Chisholm, when he says
`something' to mean `something that is a mass of matter or not
composed of matter'.  This is a bizarre and uncharitable
interpretation, {\em unless} we assume a truth-conditional theory of
meaning.

I am not sure that a truth-conditional theory of meaning is correct.
There are a number of powerful arguments against such theories, and I
do not know how to argue against them.  Sider makes an unnecessary
concession when he allows Hirsch to build his thesis on a
truth-conditional theory of meaning.

That being said, I simply do not understand how `Ontologese' is
supposed to work.

\subsection{Intention and reference}
Sider places a heavy emphasis on his notion of `joint-carving'.  He
seems to think that one can, when one uses a word, {\em intend} to use
it to `pick out' a `joint-carving' meaning.  His example involving
simultaneity was, I think, supposed to show this.  He says elsewhere
that

\begin{squote}
a highly joint-carving interpretation can ``trump'' the superior
charity of rival interpretations.  Now, this prediction is in some
cases correct, especially for ``theoretical'' terms---terms that are,
intuitively, {\em intended} to stand for joint-carving meanings.  When
a term like `mass' is introduced in physics, it's intended to stand
for a fundamental physical magnitude, and so if there's a
joint-carving property in the vicinity then that property is meant by
`mass', even if it doesn't quite fit the physicists' theory of `mass'
(\citeyear[32]{sider2011d}).
\end{squote}

It's not clear to me exactly what it means to intend to have one's
words interpreted in a `highly joint-carving' way.  If one intends to
be understood in a certain way, then generally one has some idea as to
what this amounts to.

I can say ``I intend to use `Harriet' to designate Harriet Tubman.''
If I go on to say things like ``Harriet was a very brave woman'', my
audience (if they know who Harriet Tubman is) recognize my intention
to refer to Harriet Tubman; communication is thereby secured.  But
suppose instead that I say ``I intend to use `Heerriet' to designate
whichever person named `Harriet' is nearest to me.''  I can't thereby
talk about Heerriet.  I can certainly say things like ``I wonder if
Heerriet remembered to call her mother on her birthday'', but neither
I nor my audience have any idea who I'm talking about (unless we both
happen to know who Heerriet currently is).  Likewise, I simply don't
understand how Sider expects to know what he's talking about when he
uses his `new' quantifier phrases.

Suppose this exchange were to occur:

\stage{Sophists}{(in unison)}{We do not mean `love' such that it is
  obviously true that two people can fall in love.  Rather, we intend
  to use `love' such that it is an open question whether or not two
  people can fall in love.  Let our uses of the word pick out whatever
  `joint-carving' notion is in the vicinity.}

\stage{Sophist 1}{}{Two people can fall in love!}

\stage{Sophist 2}{}{Two people cannot fall in love!}

\stage{Sophist 3}{}{We're not engaged in a verbal dispute!}

I'm inclined to say that our sophists simply do not know what they are
talking about.  Unless they have {\em some} idea of what they (and
each other) mean by `love', this is not even a verbal dispute---it is
nonsense.  (Of course, if we continue to listen we may discover that
it {\em is} just a verbal dispute.  One of the sophists might declare that
true love can only occur in stories, or something.)

I'm dubious, therefore, that one can simply {\em stipulate} that they
mean ``the most `joint-carving' meaning in the area, whatever it
happens to be''.  One must have {\em some} idea of what one is talking
about, at the risk of talking nonsense.  This brings us back to the
ontologists:

\stage{Ontologists}{(in unison)}{We do not mean `$\exists$' and
  `$\forall$' such that it is obviously true that there exist chairs.
  Rather, we intend to use `$\exists$' and `$\forall$' such that it is
  an open question whether or not chairs exist.  Let our uses of the
  quantifiers pick out whatever `joint-carving' notion is in the
  vicinity.}

\stage{Ontologist 1}{}{There exist chairs!}

\stage{Ontologist 2}{}{There do not exist chairs!}

\stage{Ontologist 3}{}{We're not engaged in a verbal dispute!}

If these ontologists are to be taken as saying anything at all, they
must have some idea what they mean by `$\exists$' and `$\forall$'.
They cannot simply appeal to whatever `joint-carving' notion is in the
vicinity.  That would be analogous to me stipulating I will use `John'
to refer to the physically closest person named `John'.  I simply
cannot do this, unless I have some idea who that is.  Likewise, the
ontologists must have some idea of what they mean by their
quantifiers.  And according to Hirsch, if the ontologists are using
the quantifiers with some intelligible meaning, figuring out what that
meaning is requires interpretive charity.

Suppose that, having made their resolution to speak Ontologese, the
ontologists go on much as before, taking sides over whether or not
there are chairs and people.  Despite their protestations, Hirsch will
conclude that what they mean by their quantifiers has not changed:

\begin{squote}
If one philosopher talks like a typical organicist and another talks
like a typical common sense ontologist then, despite their
protestations that they are both speaking the philosophically best
language [Ontologese], it's probably more plausible to hold that the
first is speaking O$\ast$-English and the second is speaking
C$\ast$-English (\citeyear[12]{hirsch2008}).
\end{squote}

Again, because Hirsch can continue to make charitable truth-condition
assignments, he will conclude that, unlike the scientists' debate over
simultaneity, the ontological disputes remain verbal.

If the ontologists, having announced their intention to speak
Ontologese, no longer sound like organicists and nihilists but rather
use the quantifiers in wholly unexpected ways, then Hirsch may simply
decide that they no longer know what they are talking about.  Like my
attempted reference to the closest `John', the ontologists have no
clear idea of what they mean.  They are no longer making sense.

\subsection{An English ontology}
I have argued that Sider makes two mistakes in his reply to Hirsch.
The first is to concede to Hirsch a controversial theory of meaning.
The second is to claim that we can intend to refer to things---like
Heerriet and `joint-carving' quantifiers---that we know nothing about.
 When Sider attempts to use the `Ontologese' quantifiers, he does not
 actually know what he is talking about.

I do not subscribe to a truth-conditional theory of meaning, so I
reject Hirsch's motivation for interpreting metaphysical utterances as
being spoken in different languages.  I think it is most charitable
for the philosophers to interpret each other as speaking English.  I
have (or so I claim) been writing in English this whole time.  When I
said ``there are chairs'', that was part of an English sentence.  If
``there are chairs'' is true in English, then there are chairs.

Sider was attempting to use his quantifiers in a wholly mysterious
way.  I, on the other hand, am using them with their ordinary English
meaning.  When I use quantifier phrases like ``there are'' in ``there
are chairs'', I know perfectly well what {\em I} mean.  Regarding
chairs, there are some (many) of them.

%% I'm not sure that I fully understand what Sider is trying to do here.
%% When at the beginning of this Introduction I said ``there are
%% chairs'', I was writing in English.  The only language with which I
%% have any proficiency is English; I don't know how to make that claim
%% in any other language.  If what I said was true---if ``there are
%% chairs'' is true in English---then there are chairs.  That is the
%% claim I want to defend.  If Sider is willing to give it to me, then my
%% job is done.

\section{But aren't English quantifiers restricted?}
\label{eng-quant}
I have been arguing that Sider's attempt to postulate a `fundamental'
language with `fundamental' quantifier phrases is neither necessary
nor successful.  But one might object to what has been said above by
claiming that English quantifiers like `there is' and `everything' are
somehow `restricted' or `nonfundamental' \citep[127]{sider2011d}.  The
`fundamental quantifiers' that Sider refers to are, according to this
objection, simply {\em un}restricted or fundamental.  The idea of
unrestricted quantifiers is perfectly sensible, so I cannot claim that
Sider is using his quantifiers in a ``wholly mysterious'' way.

This objection fails because English quantifiers are neither
`restricted' nor `nonfundamental'.

\subsection{Restricted quantifiers}
\label{restrict}
There is some temptation to think that ordinary uses of quantifier
phrases like `there is' are somehow restricted.  Suppose I am having a
party and you say ``there is no beer''.  One might think that here
`there is' is restricted to my house; you are quantifying only over
objects in the building.  The conclusion is then drawn that English
quantifiers are therefore not `fundamental', and that there is a need
to postulate a {\em totally unrestricted quantifier} that ranges over
{\em absolutely everything whatsoever}.

But as a counterexample to this thought, consider the following
exchange:

\stage{You}{}{There is no beer.}

\stage{Me}{}{I'll go get more.}

\stage{You}{}{No, aren't you listening?   There is no beer.}

\stage{Me}{}{Anywhere?}

\stage{You}{}{{\em There is no beer}.}

\stage{Me}{}{Oh my.  I thought you just meant that there is no beer in
  the house.}

The philosopher claiming that English quantifiers are restricted would
have us believe that you are actually {\em saying different things}
each time you say `there is no beer'.  But that is obviously not true.
What you {\em say} is the same.  What I take you to {\em mean} is
different.  The philosopher who claims that English quantifiers are
restricted is confusing saying and meaning.  If I say that there is no
beer, and there is beer (say, at the corner store), then {\em what I
  say} is false.  But in {\em most} cases, {\em what I mean} is that
there is no beer in the house.

What is said does not change relative to a context.  If I say ``there
is no beer'', what I have said is that there is no beer.  But what I
mean may certainly change relative to a context:

\stage{Me}{}{There is no beer.}

\stage{You}{}{Then pick some up!}

\stage{Me}{}{\textsc{Ok!}}

\stage{Lauren}{(enters)}{Prohibitionists have destroyed all the beer
  everywhere!}

\stage{Me}{(horror-struck)}{But then\,\ldots there is no beer!}

The first time I say that there is no beer, what I mean is that there
is no beer in the house.  The second time I say that there is no beer,
what I mean is that there is no beer.  What I say does not change, but
what I mean does.

%% We can explain this change in meaning in terms of {\em implicatures}.
%% An implicature is something that is implied; for instance, if I say
%% ``some of the students were drunk'', I {\em imply} that not {\em all}
%% the students were drunk, although I do not actually say as much.  I
%% {\em generate an implicature} to that effect.

%% Although saying ``some of the students were drunk'' implies that not
%% all were drunk, it does not {\em entail} it.  What I say is compatible
%% with it being the case that all the employees were drunk.  A reliable
%% way to test whether something is implied or entailed is to see if it
%% can be {\em cancelled}.  Implicatures can be cancelled, but
%% entailments cannot.  For example, when I say ``some of the employees
%% were drunk'', if I go on to say ``in fact, they were all drunk'', I
%% have cancelled the implicature.  But I have not contradicted myself.
%% If, on the other hand, I say ``some of the employees were drunk; in
%% fact, none of the employees were drunk'', I {\em have} contradicted
%% myself, because I have tried to cancel an entailment.

%% Recall the example above:

%% \stage{You}{}{There is no beer.}

%% \stage{Me}{}{I'll go get more.}

%% When I here say ``there is no beer'', I imply that there is beer
%% elsewhere (I tell you so that you can go get more).  But this
%% implication can be cancelled.  After saying ``there is no beer'', I
%% can continue, ``and not just here, there's no beer anywhere''.  In
%% saying this, I do not contradict myself.

\subsection{`Nonfundamental' quantifiers}
\label{fun-quant}
Sider and Hirsch both occasionally claim that quantified phrases like
``there is a table'' might actually {\em mean} something other than
that there is a table.  This is a mistake.  Examples from Hirsch are
documented in section \ref{charity}; here is an example from Sider:

\begin{squote}
The best metaphysical semantics of an ordinary sentence like `There is
a table' might not be a strict semantics that interprets it as making
the false claim that there exists, in the fundamental sense, a table,
but rather a tolerant semantics, which interprets it as making the
true claim that there are subatomic particles suitably arranged
\citeyearpar[171]{sider2011d}.
\end{squote}

But unless Sider subscribes to a truth-conditional theory of meaning
(see section \ref{retreat}), according to which `there is a table' is
synonymous with `there exist things arranged tablewise' if and only if
(`there is a table' is true if and only if `there exist things
arranged tablewise' is true), then it is just not true that ordinary
uses of `there is a table' mean anything other than that there is a
table.

\section{Lessons}
Henceforth I will assume that the debate over whether there are chairs
is conducted in English.  But even some philosophers who agree to this
will deny that ``there are chairs'' is a conceptual truth.  Nor will
they admit that it is obviously true (they are denying that it is
true, after all).  Such philosophers will object that so far, the only
objection I have raised against the view that there are no chairs is
that I cannot bring myself to believe it.

However, there is another reason to resist their conclusions, one that
is independent of my inability to believe that there are no chairs.
As we will see in Section \ref{stroud}, philosophers who deny that
there are chairs have a difficult time explaining why we believe that
there are chairs.  To the extent that they cannot explain why we hold
this belief (and others concerning ordinary things), we have reason to
suspect that their denials might be unfounded.

Moreover, just as the philosophers cannot deny that I believe that
there are chairs, they cannot (they should not) deny that it {\em
  seems} obviously true that there are chairs.  It is `intuitively
true' that there are chairs.  The philosophers accordingly have ways
of undermining this intuition.  They try to show that although the
unreflective individual might think that it is obvious that there are
chairs, careful consideration will show her that it is far from
obvious.  After I try to show that these philosophers cannot explain
why we believe that there are chairs, I will then explain why we
should resist their attempts to undermine the obviousness of the fact
that there are chairs.

\ifstandalone
\end{spacing}
\bibliography{everything}
\bibliographystyle{ChicagoReedweb}
\fi
\end{document}


\chapter{Why do I believe that there are chairs?}
\chapterpig{Why do I believe that there are chairs?}
\documentclass[11pt]{article}
\usepackage{standalone} \newif\ifstandlone \standalonetrue
\usepackage[left=1.75in, right=1.75in, top=1.25in, bottom=1.25in]{geometry}
\geometry{letterpaper}
\usepackage{graphicx}
%\usepackage{tipa}
%\usepackage{exaccent}
%\usepackage{txfonts}
%\usepackage{pxfonts}
\usepackage{enumitem}
%\usepackage{amssymb}
\usepackage{amsmath}
\usepackage{epstopdf}
\usepackage{setspace}
\usepackage{natbib}
\setcitestyle{aysep={}}
\synctex=1

\DeclareSymbolFont{symbolsC}{U}{txsyc}{m}{n}
\DeclareMathSymbol{\strictif}{\mathrel}{symbolsC}{74}
\DeclareMathSymbol{\boxright}{\mathrel}{symbolsC}{128}

\newcommand{\stager}[4]%
{%
	\begin{spacing}{1}%
	\vspace{0pt}
		\begin{description}[style=nextline, noitemsep, parsep=0pt, topsep=0pt, leftmargin=15mm, itemindent=-10mm, font=\mdseries]
			\item[\textsc{#1} \emph{#2}] #3
			\item[]%
			\begin{flushright}#4\end{flushright}
		\end{description}%
	\end{spacing}%
}

\newcommand{\stage}[3]%
{%
	\begin{spacing}{1}%
	\vspace{0pt}
		\begin{description}[style=nextline, parsep=0pt, leftmargin=15mm, itemindent=-10mm, font=\mdseries]
			\item[\textsc{#1} \emph{#2}] #3
		\end{description}%
	\end{spacing}%
}

\newenvironment{squote}{%
	\begin{spacing}{1}
	\begin{list}{}{%
	\setlength{\labelwidth}{0pt}%
	\rightmargin\leftmargin%
	}
	%\begin{singlespace}%
	\item\relax
	}{%
	%\end{singlespace}%
	\end{list}%
	\end{spacing}
	}

\newenvironment{inq}{\vspace{0pt}%
	\begin{list}{}%
	{\setlength\labelwidth{0pt}%
	\setlength\leftmargin{2.5\oddsidemargin}%
	\setlength\rightmargin{\leftmargin}}
	\begin{spacing}{1}
	\item[]%
	}{
	\end{spacing}
	\end{list}
	\vspace{10pt}
	%\noindent%
	}

\title{Denying the Ordinary}
\author{Alexander A. Dunn}
\begin{document}
\ifstandalone
\maketitle
\begin{spacing}{1.5}
\fi

%\begin{inq}
%The philosophical quest must start somewhere. It needs a set of beliefs about what the world is like. Without some attitudes, perceptions, beliefs, or theories to start with, it would have nothing to reflect on.~\citep[16]{stroud2000a}
%\end{inq}

%	\begin{inq}\textbf{quine}, v. To deny resolutely the existence or importance of something real or significant.
%	\footnote{This is of course from the {\em Philosophical Lexicon}~\citep{dennett2008}.}%}
%	\end{inq}%

%\section{How to defy common sense}
%\label{denials}
\noindent Every so often a philosopher will claim that some aspect of what we take to be our world is somehow illusory, bogus, or simply nonexistent. This sort of denial will take various forms. One might claim that a certain phenomenon does indeed find expression in the world, but that it is somehow subjective; without humans to experience it, there would be no such phenomena. Or one might deny that some ordinary object of experience is actually non-existent. Peter van Inwagen claims that tables do not exist. He recognizes, of course, that people talk and have beliefs about (what they take to be) tables, so he must find a way to explain our beliefs in tables in terms of those things that he does believe to exist. Finally, one might claim that some things are non-existent, and that we in fact {\em don't} really talk and have beliefs about them. This is the sort of denial we make of the existence of ghosts. Unlike the philosopher who denies the existence of tables, we who deny the existence of ghosts don't have to explain people's beliefs in ghosts---people simply don't {\em have} coherent beliefs about them, because they are entirely unreal. Few philosophers make denials of this sort about ``real or significant'' things, because the beliefs we have about such things (tables, chairs, people, custard) are deeply integrated into our daily lives, and to say that these beliefs are utterly incoherent is to risk crossing the line into nonsense.

\section{Kinds of denials}
In this section I will briefly look at each of these types of denial and how the philosophers making them manage to explain the beliefs that people have about the objects of the denials.

\subsection{The relegation}
\label{relegate}
Among the various phenomena we observe in the world, it can be tempting to draw a distinction between those that we somehow imprint upon the world and those that are independent of any human experience. The former are `subjective' while the latter are `objective' or absolute:
\begin{squote}
Whatever is due only to us and to our own ways of responding to and interacting with the world does not reflect or correspond to anything present in the world as it is independently of us. The aim of an ``absolute'' conception, then, is to form a description of the way the world is, not just independently of its being believed to be that way, but independently, too, of all the ways in which it happens to present itself to us human beings from our particular standpoint within it\,\ldots\,[So we] form some conception of that independent reality and come to understand parts or aspects of our original conception of the world as not representing it as it is. If we see them as products or reflections of something peculiar to human experience or to the human perspective on the universe, we assign them a merely ``subjective'' or dependent status and eliminate them from our conception of the world as it is independently of us~\citep[30--31]{stroud2000a}.
\end{squote}

A philosopher who adheres to this distinction might claim that our conception of the world as colored does not represent the world as it is independently of us. Colors, she would claim, are not objectively real. She allows that they are subjectively real, of course. People {\em do} see colors; we have color vision while some species do not. Because of our color vision, we come to believe that the things we see are colored. A philosopher who is skeptical of the objective reality of color ``cannot deny that we perceive many different colours or that we believe physical objects to be coloured''~\citep[145]{stroud2000a}. What the skeptic has to claim is something to the effect that, while we see things {\em as} colored, things are not {\em themselves} colored. The red color of a tomato, on this view, obtains only in our perception of the tomato; there is nothing {\em in} the tomato that is the redness (other species may not see the redness when they see the tomato).% The belief most commonly motivating this type of view, according to Stroud, is a belief that ``the world as it is independently of us'' is simply the world described by an ideal physics: ``physical science can describe every aspect of the figure or shape and the number and motions of the bodies that make up the world. We have words for what we think of as the colours, odours, and tastes of those objects as well, but those words stand for nothing that exists in reality''~(\citeyear[8]{stroud2000a}).

Someone who makes this argument does not, therefore, deny that we perceive colors, or that we believe that things are colored. To claim that we {\em actually don't} think things are colored---that we don't actually believe that tomatoes are red---would obviously be false. We certainly do believe that tomatoes (at least most of them) are red; this is what makes the denial of color interesting. If the philosopher claimed that colors aren't objectively real {\em and that we don't believe them to be}, we ought to wonder why the philosopher is even bothering to make the argument.%
%
%\footnote{}
%
\ It would be like the claim that ghosts don't exist; this is not controversial or interesting, because we don't believe that ghosts exist.

So the philosopher who is denying the objective reality of color must ``recognize the presence in the world of perceptions of and beliefs about the colours of things''~\citep[199]{stroud2000a}. The challenge then is to explain why we do have these perceptions and beliefs. For example, a philosopher who believes that only the world of physics is objectively real must explain the color phenomena in the vocabulary of the physical sciences. (And before this can be attempted, the question arises as to what this vocabulary is: ``Physical science changes. Physicists do not just change their minds as they learn more and more about the world; the very conception of what is to be included in physics changes''~\citep[53]{stroud2000a}. So the philosopher relegating colors---or anything else---to subjective reality must have a clear idea of what is left in objective reality.)%

\subsection{The paraphrase}
\label{paraphrase}
The second sort of denial goes further in denying any kind of reality at all to the subject of inquiry. The philosopher above denied that colors were `objectively real', but not that they were `subjectively real'; she did not deny that we do at least perceive colors. But there are some things that are taken by some philosophers to be neither objectively or subjectively real; they simply do not exist.% A philosopher in a cynical mood might deny that love exists. Depending on how recently she was jilted, however, she might not deny that an utterance of ``there is love in the world'' is true. What she would do is paraphrase it as ``there are people in the world who are in symmetrical or asymmetrical relations with other people that can be described as `loving relationships'\,'' (assuming that the cynical philosopher doesn't deny the existence of people as well). If it can be determined that the original speaker meant something along these lines, then our philosopher has performed `the paraphrase' on love. (The philosopher may or may not go on to claim that the loving relationship is `subjective'; that it would not exist without humans to instantiate it.)
%\ (The original speaker might reject this paraphrase, of course. She might insist that $\exists x(x=Love)$. In this situation we might perform a resolute denial of the sort described in section~\ref{resolute} below and say that there is no existing object that is love.)

\ Peter van Inwagen denies the existence of tables, chairs, apples, and every other inanimate composite object.%
%
\footnote{The notion of `composite' will be discussed below in section~\ref{scq}.}
%
\ He takes pains to make clear that his denial of these things is not a relegation of tables and chairs to `subjective reality'. He wants to claim that such things do not exist in any way, subjective or objective:
\begin{squote}
I want to do what I can to disown a certain apparently almost irresistible characterization of my view, or of that part of my view that pertains to inanimate objects. Many philosophers, in conversation and correspondence, have insisted, despite repeated protests on my part, on describing my position in words like these: ``Van Inwagen says that tables are not real''; ``\ldots\,not true objects''; ``\ldots\,not actually {\em things}''; ``\ldots\,not substances''; ``\ldots\,not unified wholes''; ``\ldots\,nothing more than collections of particles.'' These are words that darken counsel. They are, in fact, perfectly meaningless. My position vis-\`{a}-vis tables and other inanimate objects is simply that there {\em are} none~(\citeyear[99]{inwagen1995}).
\end{squote}
Van Inwagen asserts, quite seriously, that ``there are no tables or chairs or any other visible objects except living organisms''~(\citeyear[1]{inwagen1995}). This is a somewhat more bold claim than that of the philosopher skeptical of color. She at least granted that we do see colors, even if we don't actually see things that are (objectively) colored. If, as van Inwagen claims, the only {\em visible} objects are living organisms, then we certainly can't {\em see} tables at all. But just as our color skeptic could not claim that we don't believe in colors, van Inwagen cannot deny that we at least {\em believe} there to be apples. We have many beliefs about what we take to be apples. We believe that they grow on trees, and go well with many types of cheese (which we also believe to exist). Van Inwagen has, therefore, a rather daunting task: he must explain our beliefs about apples (and about cheese, and tables, and chairs, and about everything else he denies) in terms of whatever it is that he does claim to exist (living organisms and the basic particles that make up the physical universe).
%
%\footnote{Van Inwagen assumes, without defense, ``that matter is ultimately particulate\,\ldots\,every material thing is composed of things that have no proper parts: `elementary particles' or `mereological atoms' or `metaphysical simples'\,''~(\citeyear[5]{inwagen1995}). Ted Sider takes him to task for this assumption~(\citeyear{sider1993}), claiming that the possibility of `gunk'---the possibility that the matter of the world is not fundamentally particulate but infinitely divisible---falsifies van Inwagen's thesis. I think it may be possible for van Inwagen to adapt to a gunky world (see Section~\ref{brute}, note~\ref{gunk}), but I think van Inwagen's thesis is false either way.}
%

Van Inwagen does not attempt to deny that we have beliefs about what we take to be apples, cheeses, \&c. Indeed, he admits that ``when people say things in the ordinary business of life by uttering sentences that start `There are chairs\,\ldots ' or `There are stars\,\ldots ', they very often say things that are literally true''~(\citeyear[102]{inwagen1995}). By conceding this, he distances himself from a skeptical philosopher (see section~\ref{unger} below) whose denial of apples et.\ al.\ is on par with a denial of ghosts. When someone has a belief about what they took to be a ghost, we their belief is not about whatever actually caused their fright; it is most charitable to say that their belief is really about nothing at all (I explain why in section~\ref{resolute}). Van Inwagen, when denying that we have beliefs about apples, appears to maintain that the beliefs that we erroneously take to be about apples are not beliefs about {\em nothing}. They are not empty; they are rather beliefs about something {\em else}, something other than what we took them to be about (apples). Van Inwagen accordingly recognizes the need to explain what our beliefs really are about.

What van Inwagen says is that, when saying things like ``Some chairs are heavier than some tables'', if we are talking about anything at all, we are talking about simple particles ``that are arranged chairwise and\,\ldots\,that are arranged tablewise''~(\citeyear[109]{inwagen1995}). He develops an elaborate `paraphrasing strategy' that is an attempt to show that many (if not all) propositions expressed by ``There are chairs\,\ldots '' and related sentences do not actually entail the existence of chairs.

I do not think van Inwagen's defense is ultimately successful (see section~\ref{pigletwise}), but it is precisely the sort of defense required when denying the existence of ordinary things like tables and chairs. He cannot claim, without courting absurdity, that we {\em don't} believe there to be tables and chairs in the world. We do believe so. We might, of course, be {\em wrong} about the existence of tables and chairs; ``from the fact that we believe a certain thing it does not follow that it is true''~\citep[21]{stroud2000a}. Nevertheless it is true that we believe there to be such things, and van Inwagen needs to explain the source of this belief. If there are no tables and chairs, then we must be able to understand the beliefs that we thought to be about the furniture to be about something else instead, and we need a story about what that something else is.
% ``you cannot hope to explain something unless you grant that there is such a thing and you have at least some idea of what it is''~\citep[97]{stroud2000a}.
\ As I said, I don't think van Inwagen's explanation is a good one. But before discussing why, we should consider the third and most radical sort of denial.

\subsection{The resolute denial}
\label{resolute}
The third kind of denial is that which I likened to a denial of ghosts. When I deny the existence of ghosts, I also deny that people talk about and have beliefs about ghosts. Even if someone says ``let me tell you about the ghost I saw last night!'', I can maintain that there is {\em nothing} in particular that they are talking about. They are certainly not talking about a certain ghost; there are no ghosts, and never have been.

Of course I cannot deny that some people believe that ghosts exist. Some people do believe this. But my resolute denial of the existence of ghosts does not prevent me from explaining someone's belief that there are ghosts. This someone, for example, might tell me that she saw a ghost on the landing. I walk out and see a light from a high window flickering strangely on the wall. (If I squint, the pattern of the light looks almost humanoid.) So I tell her that what she thought was a ghost was really just a curious play of the light. I am not here conceding that she had a belief {\em about} a ghost. On the contrary, I have tried to show her that her belief was about anything but a ghost. There was in fact no ghost that she could have formed a belief about; ``if we show that what a frightened person saw in the attic on a particular occasion was a rippling reflection of the moon through the window, we implicitly deny the presence of a ghost in giving the explanation of the person's belief and fear''~\citep[76]{stroud2000a}. We do not deny that people have beliefs about what they take to be ghosts; what we deny is that they are correct in taking their beliefs to be about ghosts.

Likewise, suppose she and I are looking in on an empty, well-lit room. Suddenly she points and cries, ``Look, a ghost!'' In this case there is nothing in the room that I can assume to have caused this belief. There are no reflections of the moon or curious plays of the light; as far as I can tell, she is pointing at nothing.

\stage{Me}{}{What on earth do you mean?}

\stage{Her}{(pointing)}{That ghost, there! See?}

If this continues, the most probable explanation is that she is having an hallucination. She has what she takes to be a belief about a ghost in the room. There is no ghost in the room, so what is her belief about? Likewise, if someone says she saw a ghost in the attic, what should we say her belief is really about?

In Stroud's example above, we might say that the person is frightened of the reflection of the moon. This is similar to how we might talk about a child's night terrors: ``she was afraid of the chair in her room (she thought it was a monster).'' Looking at things this way, the person's beliefs is {\em about} something, but it is something very different from what they took it to be about. I think, however, that this is not a fully accurate characterization of the object of the person's belief. It may be true that the reflection of the moon {\em caused} her belief (which caused her fright), but it would be at least a little misleading to say that Mrs --------- is frightened of the reflection of the moon. This is misleading because, if we succeed in showing her that there {\em is} no ghost in the attic (only a reflection of the moon), she will not still be afraid. When she understands that her belief in the ghost was mistaken, she will see that there is {\em nothing} to be afraid of. The belief that caused her fright, that there was a ghost in the attic, was in fact a belief about nothing at all. (The same goes for the person who sees the play of light on the wall and believes that there is a ghost.)

If someone claims to see a ghost in an empty, well-lit room, is her belief actually about the {\em hallucination}\,? Again, charity demands that we not say this. If she is convinced that nobody else sees a ghost, she may recognize that she was hallucinating. Were she to come to this conclusion, she would no longer be afraid of what she saw (though she will no doubt be afraid that she is going mad). The belief that she took to be about a ghost was in fact about nothing at all.

\section{Unger's nihilism}
\label{unger}
Just as resolutely as we denied the existence of ghosts, so Peter Unger has denied the existence of such things as ``tables and chairs and spears\,\ldots\,swizzle sticks and sousaphones\,\ldots\,stones and rocks and twigs, and also tumbleweeds and fingernails''~(\citeyear[117]{unger1979}). He does not consider them merely `subjectively real' as opposed to objectively so---like van Inwagen, he claims that they simply do not exist. He comes to this conclusion from a different direction, however. As we will see, van Inwagen's denial of the existence of `ordinary things' is a consequence of his theory of composition (under what conditions some things compose another thing). Unger, on the other hand, draws his conclusion from an application of the sorites paradox:%
%
\footnote{Unger also motivates his nihilism by way of `the problem of the many'. We will examine this problem in section~\ref{many}.}
%
\begin{squote}
Consider a stone, consisting of a certain finite number of atoms. If we or some physical process should remove one atom, without replacement, then there is left that number minus one, presumably constituting a stone still\,\ldots\,after another atom is removed, there is that original number minus two; so far, so good. But after that certain number has been removed, in similar stepwise fashion, there are no atoms at all in the situation, while we must still be supposing that there is a stone there. But as we have already agreed, if there is a stone present, then there must
be some atoms\,\ldots\,I suggest that any adequate response to this contradiction must include\,\ldots\,the denial of the existence of even a single stone.~\citep[121--122]{unger1979}
\end{squote}

Having made this denial, Unger must either explain how our beliefs about stones should be understood (van Inwagen has his paraphrasing strategy) or he must deny that we really {\em do} have any beliefs about stones. It appears that he selects the latter option: Unger seems to claim that, like the person who thought they had a belief about a ghost, we are wrong to think that we have any coherent beliefs about stones or any other ordinary things. Unger says that, like `ghost', our ``terms for ordinary things are incoherent [and] cannot apply to anything real''~\citep[147]{unger1979}. A consequence of this is that our language and thought concerning all such things is directed toward {\em nothing at all}: ``it may well be that I have never {\em thought of} any stones at all, or tables, or even human hands. If that is so, then it would seem that {\em a fortiori} I do not {\em know} anything {\em about these entities}, however commonly I might otherwise suppose''~(\citeyear[458]{unger1980a}).

This all seems very strange. Concerning ghosts, ``it is difficult even to find a fully coherent belief that might be exposed as false; we discover, at best, obscurity or perhaps confusion\,\ldots\,do we really understand what sort of thing a ghost is supposed to be''~\citep[76]{stroud2000a}? If someone tries to tell me about the ghost that visited him the previous night, it does not seem unjust to say that he doesn't really know what he is talking about. But can this be extended to some of the most common objects of experience?

When we denied the existence of ghosts, we denied also others' beliefs in them. We did not, however, deny that people have beliefs which they take to be about ghosts. But we were able to show that these beliefs were not {\em about} ghosts; in most cases they were about nothing at all. Likewise, Unger cannot deny that we have beliefs that we take to be about tables, chairs, and all the other things that he denies exist. If our beliefs about tables and chairs are really beliefs about nothing at all, there are two questions that must be answered: first, what causes us to form these beliefs?\ and second, if the utterances containing these empty terms are about nothing at all, how do we manage to communicate so effectively using them?

\subsection{Causes of belief}
\label{unger-cause}
People who believe in ghosts probably do so because they have unreflectively embraced the superstitions of their culture. They may initially come to believe that ghosts exist on the testimony of other people---older siblings, perhaps---or by reading too many ghost stories. Much as Catherine in Jane Austen's {\em Northanger Abbey} jumps to the most macabre conclusions as a result of having absorbed too many gothic novels, so might our gullible reader of ghost stories interpret such innocent phenomena as reflections of the moon as ghostly assailants. Those of us who have not taken our cues from fiction would be more likely to recognize such phenomena as tricks of the light. Even if we were to see something that was definitely {\em not} a trick of the light, we would sooner attribute it to an hallucination than countenance the possibility of ghosts. Suppose {\em you} saw what you took to be a ghost in an empty, well-lit room.%
%
%\begin{squote}
%She rose, not as if she had heard me, but with an indescribable grand melancholy of indifference and detachment, and, within a dozen feet of me, stood there as my vile predecessor. Dishonored and tragic, she was all before me; but even as I fixed, and, for memory, secured it, the awful image passed away~\citep[58]{james1991}.
%\end{squote}
%
\ Most of us would still, even if presented with such a vision, {\em refuse to believe in ghosts}. This is because we know that the probability of there being such spirits is far less than the probability of us experiencing cracks in our sanity. Undermining my belief that ghosts don't exist would require a great deal---for example, my friend and I both seeing the {\em same} apparent ghost at the same time, and knowing that we were each experiencing the same vision. (Even then, we would want further confirmed sightings to convince us that we weren't, in fact, crazy.)

If this is an accurate characterization of our beliefs concerning ghosts, it is a very different characterization than one we might give of how we learn about and come to believe in chairs. Chairs are not something that children learn about from stories. A child learns what a chair is as an answer to the question, ``What is {\em that?}\,'' Let us suppose that the child is pointing at a chair in the center of a well-lit room containing no other furniture. The chair is clearly visible. If someone were to believe they were pointing at a ghost in a similarly well-lit situation, we could safely assume that they would be experiencing a hallucination. Hallucinations are not shared experiences; if one person is hallucinating a ghost, nobody else can see that ghost, not even if they were {\em also} hallucinating a ghost (they can't both hallucinate the {\em same} ghost). Such a ghost sighting, therefore, would necessarily be experienced by a single person. A chair, on the other hand, can be sighted by multiple people simultaneously. The parent sees the same chair that the child sees. This is what allows the parent to answer the child's question (``That's a chair'') and this is how the child learns about chairs. We learn about chairs by coming upon them in the world, and being {\em told} what it is we have come upon. This brings us to communication.

\subsection{Communication and incoherence}
\label{unger-comm}
A consequence of Peter Unger's thesis is that most of our communication is empty of content, if not entirely incoherent. For me to warn my friend of a low-hanging tree branch, I must refer her to it. For her to heed my warning, she must recognize my intention to refer to the branch in question, and, as a result of that intention, herself come to think of the branch (and then duck). But according to Unger, ``when we are under the impression that we are thinking about an object in the world\,\ldots\,our impression is mistaken''~(\citeyear[149]{unger1979}). Unger is denying that I can ever possibly warn my friend about a low-hanging branch.

This seems plainly false. For, after all, my friend ducked. She heard my warning and avoided the low-hanging branch. Unless we are to suppose that it was sheer luck that she moved her head in time to dodge the branch, we must conclude that I did, in fact, warn my friend. If I did, in fact, warn my friend, then we both thought of the branch. %(It doesn't really matter how Unger argues for the impossibility of such communication; it {\em does} succeed.)% (I want to set it down as an axiom that \textsc{communication occurs}.)

That ought to be enough to prove Unger wrong. Just to be thorough, however, let's return to our well-lit room. There is a chair in the center of it. A child and her parent enter.

\stage{Child}{(pointing at the chair)}{What's that?}

Now, if the parent is going to be in a position to understand the question, she must recognize the intention of the child to refer to the chair; recognizing this intention, she comes herself to think of the chair. Only if this process occurs can the parent know what the child means by ``that''. Having performed this feat, the parent can then tell the child ``That's a chair''. If the child was hallucinating the chair (if the room was really empty), then the parent could not recognize the referential intention of the child. She could not recognize a referential intention, because that would require her to think of the object that the child is thinking of (and intending to refer to). But as I have argued, the child, if hallucinating a chair, has a belief about {\em nothing} in particular. The parent cannot come to think {\em of} the child's hallucination, at least not as a result of recognizing the child's referential intention (she may come to think {\em that} her child is hallucinating, but that's a different matter). Rather, she will look at an empty room and say

\stage{Parent}{}{What's what?}

But if the room is not empty, and the child is not hallucinating, then the parent {\em will} recognize the child's referential intention. If the child is referring to a chair, then the parent will say ``That's a chair''.

%And yet somehow Unger maintains that the kind of object picked out by `chair' is ``never instanced''~(\citeyear[147]{unger1979}). That is, the word ``chair'' necessarily has no application in the world. But suppose that this happens instead:
%
%\stage{Child}{}{I love the color of this chair!}
%
%\stage{Parent}{}{I painted it just for you.}

%\noindent %
How is Unger supposed to explain their communication? The child, according to Unger, has no determinate thought; she is certainly not thinking of a {\em chair}. Nor is the parent thinking of a chair. Unger might say that, just as people see things (reflections of the moon) that they mistakenly take to be of a ghost, so the child and parent are seeing something that each mistakenly takes to be a chair.

But now what could they be seeing? This is a well-lit, sparsely furnished room. There is no chance of curious plays of the light or visual tricks that might deceive the child and her parent. If the child claimed to be seeing a ghost in such a room, then (as above) we should have to say that she is hallucinating. Is the child then hallucinating a chair? If this were the case, then (as above) we should not expect the parent to understand the referential intention. Yet she apparently does. Are {\em both} of them hallucinating? That would allow us to say that communication appears to succeed when in fact it does not. But for the parent and child to have such similar hallucinations is incredibly unlikely. It may happen on rare occasions, but to say that such a coincidence is actually a daily occurrence is absurd. There is clearly {\em something} that the parent and child are communicating about. The default assumption is that it is a chair.

One sympathetic to Unger's thesis might admit that they are communicating about something, but deny that the subject of their communication is a chair. This philosopher would take refuge in the notion of `loose truth'. She would maintain that it is strictly false that there is a chair in the room, but that it is loosely true; it is close enough to the truth for practical purposes. These practical purposes include the communication we have observed above. She will appeal to such examples as this:

\stage{Countess}{}{Where on earth am I going to find someone to invest in my eel farm?}

\stage{Count}{(pointing)}{There's a millionaire for you.}

\stage{Countess}{(incredulous)}{Henry? A millionaire? He hasn't got above nine hundred ninety-five thousand pounds.}

\stage{Count}{}{Oh, it's close enough.}

We are supposing that there is no millionaire in the room; strictly speaking, the count said something false with ``There's a millionaire''. Nonetheless, communication occurred because the term `millionaire' made the count's referential intention clear: he intended to refer to the person who was {\em almost} a millionaire. (The term is regularly used to refer to non-millionaires who have relatively great wealth.) The Ungerian is claiming that this is analogous to the case of the parent and child. Strictly speaking, what the parent said (``That's a chair'') was false, but it allowed for communication by making the parent's referential intention clear.

Is this a coherent objection? Without concerning ourselves too much with the nature of loose truth, I think it is fair to claim that, just as a (strict) truth has a `truthmaker', so a loose truth must have a `loose-truthmaker'. In the example above, the truthmaker for ``There's a millionaire'' would have been the fact that the count was referring to a millionaire. This fact did not obtain, so the statement is, strictly speaking, false. The loose-truthmaker is evidently the fact that the count is referring to someone who is {\em almost} a millionaire. (What counts as `almost' will no doubt vary between contexts, but in this context I am supposing it is true that Henry is almost a millionaire.)

The truthmaker for ``That's a chair'' must obviously be the fact that the parent is referring to a chair. According to Unger, there are no chairs, so nobody can refer to them. The parent's statement would therefore be, strictly speaking, false. Now what is the loose-truthmaker for the parent's use of ``That's a chair''? It cannot be the fact that there is {\em almost} a chair (a partially built chair?), at least not if that entails that there could ever be a chair. Unger maintains that the kind of object picked out by `chair' is ``never instanced''~(\citeyear[147]{unger1979}). Is there, perhaps, something closely resembling a chair in the room, and the parent is referring to {\em that} thing instead? This raises two objections of its own. First, what is there in the room that ``closely resembles'' a chair, other than the chair itself? Second, if we cannot ever have coherent thoughts about chairs (and therefore cannot know anything about chairs), how are we supposed to know what resembles a chair?

I do not think there are satisfactory answers to these questions. Moreover, I do not think Unger ever espoused a `loose-truth' nihilism, so we are not slighting him by moving on.

%\stage{Peter Unger}{}{I never said the room was {\em empty}. It isn't empty. But that doesn't mean that there is a {\em chair} in the room. There's something else entirely.}
%
%\noindent Now we may ask, ``What {\em is} in the room, if not a chair?'' Unger might walk over to the chair, saying ``There seems to be a concentration of solid matter in this vicinity which, when placed appropriately, may be sat on.'' And now we say, ``Oh, you mean a {\em chair!} That's what it's called, you see.''

\subsection{The moral}
There are limits to what one can resolutely deny the existence of. We can deny that certain things, like ghosts, exist {\em and} deny that people have beliefs in them. We can do this because in each situation where a person has a belief about what they take to be a ghost, we can show that their belief is really about nothing at all. If someone sees a reflection of the moon or experiences a hallucination, and so thinks she is seeing a ghost, we can say that she is afraid of nothing at all. Under no circumstances must we say that her belief is really about a ghost. Moreover, we can explain how people come to believe in ghosts---they read too many ghost stories, or believe the lies of others.

This is not something we can do with tables, chairs, and other ``ordinary things'', let alone people. For one, Unger has no explanation of how we come to form our beliefs in these things, if not by {\em seeing them}. Secondly, to deny that our thought and talk about such things are really about nothing at all is to deny that communication regularly occurs. Bizarrely, this is a consequence Unger appears willing to accept:
\begin{squote}
Now, it must of course be admitted that these arguments [for his strain of nihilism] undermine the possibility of any endeavor I should try to propose, or even the putative thought that I should propose anything, just as all of my putative essay is undermined. But even so, I shall (incoherently) propose that what we have now to do is invent new expressions which are not inconsistent ones, and by means of which we may, to some significant extent, think coherently about concrete reality~(\citeyear[544]{unger1980b}).
\end{squote}
If Unger seriously believes this, then he could not expect us even to understand his essay ({\em why would he write it?}). But I think it is safe to say that Unger does {\em not} actually believe that there are no people or ordinary things. In a book on ethics, Unger has unambiguously expressed his belief in people:

\begin{squote}
Each year millions of children die from easy to beat disease, from malnutrition, and from bad drinking water\,\ldots\,As UNICEF has made clear to millions of us well-off American adults at one time or another, with a packet of oral rehydration salts that costs 15 cents, a child can be saved from dying soon~(\citeyear[3]{unger1996}).
\end{squote}
There are only two possibilities: either Unger does believe that people (at least children and Americans) do exist, or he takes himself to be flat-out lying in the quoted passage.

As far as other ordinary things go, Unger claims to ``often now believe that there really are no tables or rocks, and never so firmly believe that there are such things as I once did''~(\citeyear[543]{unger1980b}). All I can say is that I don't believe him. (To show that he does believe in these things, we would need to spend some time with him, observing his behavior. We could invite him for a walk along a trail with lots of low-hanging branches, then warn him about them.)

\section{The Problem of the Many}
\label{many}
In section~\ref{unger} we looked at a version of metaphysical nihilism. Peter Unger attempted to deny that any of the `ordinary things' in the world (tables, chairs, apples, people, \&c.) actually exist. His motivation for this claim was drawn from an apparent paradox involving the terms for ordinary things. If we have a stone, then removing one atom of matter will not destroy the stone. Nor will removing another atom. But if we remove enough atoms, there will not be a stone. One solution to this puzzle is to deny that there ever is a stone. But this, we have seen, is not workable.

Another solution is to claim that there are {\em many} stones where we once thought there was only one. The motivation for this claim sometimes comes from what Unger calls ``the problem of the many''. There are a number of different formulations of this problem. Van Inwagen nicely summarizes one:
\begin{squote}
Assume I exist. Then certain simples compose me. Call them `M'. Now, a single simple is a negligible item indeed. Let $x$ be one of these negligible parts of me---one that is somewhere in my right arm, say. Now consider the simples that compose me {\em other than} $x$ (`M -- $x$'). Since $x$ is so very negligible, M -- $x$ {\em could} [my emphasis] compose a human being just as well as M could. We may say that M and M -- $x$ are ``equally well suited'' to compose human beings. And, of course, for {\em any} simple $y$, ``M -- $y$ will be as well suited to compose a human being as M are. Moreover, it would be surprising indeed if there were not a simple $z$ such that ``M + $z$'' were as well suited to compose a human being as M are. It would, in fact (if I may once more use this phrase), be intolerably arbitrary to say that M composed a human being although M -- $x$ {\em didn't} [my emphasis] and M -- $y$ {\em didn't} [my emphasis] and M + $z$ {\em didn't} [my emphasis]. Suppose, therefore, that M -- $x$ et al.\ {\em do} [my emphasis] compose human beings~(\citeyear[215]{inwagen1995}).
\end{squote}

I think this formulation is problematic. We are supposing that M does compose a human being. But it does not immediately follow from this that M -- $x$ also composes a human being. As I have pointed out with italics, there is a slide from the claim that M -- $x$ {\em could} compose a human being to the claim that M -- $x$ {\em does} compose a human being. As an analogy, take a house of blocks. Suppose that the blocks do compose the house. Is there also something composed by the blocks minus one? There {\em could} be; but intuitively, there would be only if that one block were removed. Then we would have a house with a missing roof. But we do not obviously have that second thing already, without having removed the block.

Van Inwagen understands there to be a number of additional premises required for this argument. He formulates them thus:
\begin{enumerate}
	\item In every situation of which we should ordinarily say that it contained just one man, there are many sets of simples whose members are as suitably arranged to compose men as any simples could be. \label{many1}
	\item The members of each of these sets compose something. \label{many2}
	\item Each of these ``somethings'' is a man, provided there are any men at all. \label{many3}
	\item If I exist, there is a man~(\citeyear[216]{inwagen1995}). \label{many4}
\end{enumerate}
I think only~\ref{many4} is uncontroversially true. There are many difficult questions raised by~\ref{many2}, and I'm not sure how to answer them. I'm quite sure, however, that the first and third premises are false. These two are closely related, so we'll examine them together.

\subsection{Problems with the first and third premises}
\label{many13p}
Unger begins his presentation of the problem of the many by directing our attention to a cloud; looking at the edge of the cloud, ``all that is there to be seen is a {\em gradual transition} from the more dense [concentration of water molecules] to the less so\,\ldots\,there is no natural break, or boundary, or stopping place, for any would-be cloud to have''~(\citeyear[415]{unger1980a}). Therefore any boundary we choose for the cloud will be somewhat arbitrary. A minutely larger or smaller boundary would be just as `suitable' for bounding a cloud; ``if our alleged typical item [the cloud] is indeed a typical cloud, then many of these candidates, millions at least, do not fail to be clouds altogether but are clouds of some sort''~(\citeyear[421]{unger1980a}). Unger draws the conclusion that either there are millions of such things there, or none. This argument can be generalized to chairs, stones, and people. Regarding chairs, at least, we have already shown that the conclusion that there are none is false; perhaps then, there are millions of chairs in our well-lit room. But there are two problems with this argument.

\paragraph{An objection regarding communication.}
First, the idea of there being millions of objects where we supposed there was only one poses a problem for referential communication. In fact, it is quite the same problem that Unger's nihilism faced above: how is referential communication possible under this hypothesis? When we supposed that there was nothing being referred to by `chair', we were at a loss to explain how people nonetheless managed to communicate using that term. Now we are supposing there to be millions of chairs, each eligible to be referred to by ``that chair''. What needs explaining now is how, if there are millions of chairs that might be referred to, how two speakers can be sure that they are talking about the {\em same} chair. When the child points and says ``What's that?'', the parent knows that the child is pointing at {\em a} chair, but how is she to determine {\em which} chair the child is referring to? The child did {\em not} say ``What are those?'' because she took herself to be referring to {\em one} chair, and expressed her referential intention accordingly.

Even if the parent and child got lucky and happened to think of the same chair, how would they know it? If there really were millions of chairs in the center of the room, it seems implausible to suggest that anyone could distinguish between each of them. There would be nothing the parent or child could do to make it clear to the other which of the millions of chairs they meant to refer to. Indeed, they could not make it clear to themselves. If I am holding a rock in my hand, ``there are millions of ``overlapping stones'' before me\,\ldots\,how am I to think of a single one of them, while not then equally thinking of so many others''~\citep[456]{unger1980a}? Unger's brand of universalism, like his nihilism, precludes successful communication, and so must be rejected.

\paragraph{An objection regarding boundaries.}
Second, it seems to be simply false that all boundaries drawn about a cloud are equally arbitrary. If a cloud is a concentration of water molecules in the air, then the boundary of the cloud is the edge of the concentration, beyond which the level of water in the air is normal. There are not millions of clouds with millions of different boundaries; any stipulated boundary that falls inside or outside the actual boundary is arbitrary because it does not track the concentration of water molecules. For simples to be `suitably arranged' so as to compose a cloud, their boundary must be the edge of the concentration of water molecules. I'm not sure, but it may be possible to extend this argument to cover most `ordinary things'---tables and chairs (and even people) have molecular boundaries.

Premise~\ref{many1} is therefore false. For clouds, at least, it is not true that there are ``many sets of simples whose members are as suitably arranged'' to compose them. These closely related sets do not have boundaries that track the concentration of water molecules in the air, so they are not suitably arranged to compose clouds. If the boundary of the cloud shrunk, then a smaller set of molecules {\em would} be suitably arranged to compose the cloud, and so it would. But just as the blocks minus one {\em would} compose a house (but don't), so this smaller set {\em would} compose a cloud (but doesn't). So premise~\ref{many3} seems false too; at least, if these sets compose anything, they compose something other than a cloud.

\section{Paraphrasing and composition}
In section~\ref{unger} we found that an attempt to deny the existence of ordinary things such as tables and chairs cannot succeed unless an explanation is given for our successful communication using the terms `table' and `chair'. Unger's nihilistic thesis failed to explain what we are really thinking and talking about when we take ourselves to be thinking and talking about chairs. We found no reason to suppose that we are not, in fact, thinking and talking about tables and chairs. If we are able to think and talk about tables and chairs, then it seems to follow that tables and chairs exist. (Remember that the fact that ghosts do not exist gives us reason to believe that we do not think and talk about ghosts.)

Peter van Inwagen has developed a more sophisticated thesis denying the existence of tables and chairs. Like Unger, he claims that (necessarily) there are simply no such things as tables and chairs in the world. But unlike Unger, he does not claim that when we take ourselves to be thinking and talking about such things, we are thinking and talking about nothing at all. Or at least ``when people say things in the ordinary business of life by uttering sentences that start `There are chairs\,\ldots\,' or `There are stars\,\ldots\,', they very often say things that are literally true''~\citep[102]{inwagen1995}. One would generally assume that if such statements are true, then it follows that chairs and stars exist. But van Inwagen denies that chairs and stars exist. How can he claim, then, that what was said was true? As we mentioned in section~\ref{paraphrase}, van Inwagen attempts to show that the statements in question can be {\em paraphrased}---they can be reformulated to show that they have no ``ontological commitments''. According to van Inwagen, one can assert that there is a chair without being committed to the existence of chairs.

Section~\ref{comp} will summarize the motivation for van Inwagen's denial. Section~\ref{pigletwise} will introduce and criticize van Inwagen's paraphrasing strategy.

\subsection{Composition}
\label{comp}
The Special Composition Question was given a precise formulation by van Inwagen, who finds that ``the metaphysically puzzling features of material objects are connected in deep and essential ways with metaphysically puzzling features of the constitution of material objects by their parts''~\citep[18]{inwagen1995}. A ready example is the Ship of Theseus: the planks and rigging and sails (and every part of the ship) are replaced as they individually wear out. These replacements happen each by themselves; it's not the case that the entire ship (or even a large section) is swapped out all at once. But eventually no part of the original ship remains. And yet we would commonly say that it is still the same ship. But why should we think that the present ship is identical with a past ship with which it shares no parts?

\subsection{The question}
\label{scq}
Answering the question ``why is this ship identical with that past ship?'' requires first figuring out why (and how) these planks and rigging and sails (et.\ al.) compose a ship in the first place. Van Inwagen asks ``in what circumstances do planks\footnote{For simplicity's sake, van Inwagen ignores the rigging and sails.} compose (add up to, form) something?''~(\citeyear[21]{inwagen1995}) For some $x$s, then, van Inwagen asks us to consider when
\begin{equation}
\exists y\ \text{the}\ x\text{s compose}\ y
\end{equation}
is true.%
\footnote{Van Inwagen explains in some detail how plural referring expressions (like ``the planks'') can be given a logical formalization (\citeyear[23--28]{inwagen1995}), but suffice to say they work just as one would expect.}%
%
\ Less formally, van Inwagen asks: ``suppose one had certain (nonoverlapping) objects, the $x$s, at one's disposal; what would one have to do---what {\em could} one do---to get the $x$s to compose something?''~(\citeyear[31]{inwagen1995})

({\em Composition} is a technical term for van Inwagen. He understands it thus: ``the $x$s compose $y$'' means that ``the $x$s are all parts of $y$ and no two of the $x$s overlap and every part of $y$ overlaps at least one of the $x$s\,\ldots\,a thing {\em overlaps} a thing---or: they overlap---if they have a common part''~(\citeyear[29]{inwagen1995}). For van Inwagen, everything is a part of itself; some $x$ is a {\em proper} part of some $y$ only if $x \neq y$.)

\subsection{The usual answers}
There are several prominent answers to the Special Composition Question, including the following:\footnote{These formulations are from~\citet{markosian1998a}.}
\begin{description}
	\item[Nihilism] Necessarily, for any $x$s, there is an object composed of the $x$s iff there is only one of the $x$s, i.e., the only objects that exist are simples~(\citeyear[219]{markosian1998a}).%
	%
	\footnote{\label{flip} Note that this may not be Unger's view. He denies that people, apples, cheese, tables, chairs, and other ``ordinary things'' are nonexistence but he does not, as far as I know, take a stand on whether anything at all exists. His view can be (flippantly) summarized thus: ``if we have a word for it, it doesn't exist.''}
	%\footnote{\label{gunk} Of course, it may be that the world is not fundamentally particulate, and is filled not with simples but with `gunk'; see \citet{schaffer2003}. Nihilism (and van Inwagen's second condition below) can be formulated to take this possibility into account: ``for any quantity of gunk, there is nothing composed of it.''}%
	%
	\item[Universalism] Necessarily, for any $x$s, there is an object composed of the $x$s iff no two of the $x$s overlap~(\citeyear[227]{markosian1998a}).
	\item[Van Inwagenism] Necessarily, for any $x$s, there is an object composed of the $x$s iff either (i) the activity of the $x$s constitutes a life or (ii) there is only one of the $x$s~(\citeyear[221]{markosian1998a}).
\end{description}

As we have seen, any version of nihilism that does explain our putative communication about tables and chairs is false.

Whether or not universalism is false is a more difficult question. As we saw in section~\ref{many13p}, any version of universalism that entails the existence of millions of chairs where we assumed there to be only one is false. This is because such a thesis would entail that we are unable to successfully communicate about ordinary things; we would not be able to know that were were talking about the {\em same thing} (not things) as our audience. It may be that a version of universalism without this consequence is true; but we will leave that possibility aside for now.

Van Inwagen examines and rejects a number of answers to the Special Composition Question that would entail the existence of tables and chairs. Some are too strong: `some $x$s compose a $y$ iff the $x$s are in contact' would entail that two people shaking hands will result in a new object coming into being. Others are too strong in some ways and too weak in others: `some $x$s compose a $y$ iff the $x$s are fastened together' would entail that two people being glued together would result in a new object; and it would deny that an object can be composed without fastening its parts together (such as when building a house of cards). The only answer van Inwagen finds consistent is what we have dubbed {\em van Inwagenism}, which entails that tables and chairs do not exist.

However, without a good explanation of what our beliefs about tables and chairs are really about, the fact that van Inwagenism entails the nonexistence of tables and chairs only shows that van Inwagenism is false. Happily, though, van Inwagen recognizes this and is prepared with a paraphrasing strategy aimed to show that the beliefs that we take to be about tables and chairs are really about something else.

\section{Van Inwagen's paraphrasing strategy}
\label{pigletwise}
Van Inwagen distances himself from the kind of resolute denial we saw Unger attempting in section~\ref{unger}. Unger maintained that terms like `chair' are incoherent; were this so, a statement involving the phrase ``There is a chair\,\ldots '' could surely not be true. Van Inwagen, on the other hand, admits that ``when people say things in the ordinary business of life by uttering sentences that start `There are chairs\,\ldots\,' or `There are stars\,\ldots\,', they very often say things that are literally true''~\cite[102]{inwagen1995}. One would generally assume that if what people say with ``There are chairs\,\ldots '' and the like are true, then chairs exist. But van Inwagen denies this entailment.

How can van Inwagen maintain this? He claims that one can also say, truly, ``There are simples arranged chairwise\,\ldots '' without committing oneself to the existence of chairs. He could, therefore, claim that when someone says ``There is a chair\,\ldots '' she {\em means} ``There are simples arranged chairwise.'' This is, of course, a bold hypothesis about the speech practices of ordinary speakers. Certainly very few speakers would, if asked, affirm that what they meant to say had anything to do with simples; they would say that when they said that there was a chair, they meant just that. Van Inwagen recognizes that this is not a viable position: ``The only thing I have to say about what the ordinary man really means by `There are two valuable chairs in the next room' is that he really means that there are two valuable chairs in the next room''~(\citeyear[106]{inwagen1995}).

One might then assume that van Inwagen is thinking in analogy with Russell. He could attempt to claim that, despite the surface appearance of language (``There is a chair\,\ldots ''), the underlying logical form does not make any mention of chairs (or tables); the offending concept is analyzed away, leaving ``There are simples arranged chairwise\,\ldots ''. Van Inwagen notes that his ``suggested technique of paraphrasing enables us to escape some of the more embarrassing consequences of this position. When someone says ``Some tables are heavier than some chairs,'' there is obviously something right about what he says. Our technique of paraphrasis enables us to capture what it is that is right about what he says''~(\citeyear[111]{inwagen1995}). However, on the very next page he admits that the ordinary language proposition and his paraphrased version are different propositions: ``When the ordinary man utters the sentence `Some chairs are heavier than some tables' (in an appropriate context, and so on and so on), he expresses a certain proposition, and one that is almost certainly true. But I do not claim that this proposition {\em is} the proposition that, for some $x$s, those $x$s are arranged chairwise and for some $y$s, those $y$s are arranged tablewise, and the $x$s are heavier than the $y$s''~(\citeyear[112]{inwagen1995}). So van Inwagen is not making an appeal to some notion of `logical form'. But then what is the purpose of the paraphrasing project?

Van Inwagen attempts to justify his method of paraphrasis by asserting the following parallels between the original and paraphrased propositions:
\begin{enumerate}[label=(\Alph*)]
	\item The paraphrase describes the same fact as the original. \label{para-a}
	\item The paraphrase, unlike the original, does not even appear to imply that there are any objects that occupy chair-receptacles. \label{para-b}
	\item The paraphrase is neutral with respect to competing metaphysical theories, {\em viz}. the ``received'' theory, that there are objects that occupy chair-receptacles, and the theory I have proposed, according to which there are no such objects. \label{para-c}
	\item The original, though it doubtless does not express the same proposition as the paraphrase, has the feature ascribed to the paraphrase in \ref{para-c}: It is neutral with respect to the question whether there are objects that fit exactly into chair-receptacles~(\citeyear[113]{inwagen1995}). \label{para-d}
\end{enumerate}
I am rather dubious as to the truth of \ref{para-a}, but I am quite sure that \ref{para-d} is false, and van Inwagen's thesis appears to depend on it. He admits in~\ref{para-b} that the original sentence (e.g., ``There are chairs\,\ldots '') {\em implies} that there are chairs, but claims in~\ref{para-d} that it does not {\em entail} this. But why wouldn't it?

\subsection{Propositions and ontological commitment}
Let us review the situation so as to appreciate the mess van Inwagen has gotten himself into. First, he agrees that when someone says thinks like ``There is a chair\,\ldots '' they mean just that. Second, he admits that his `paraphrases' of such propositions are not so faithful to the original that they can be called the same proposition; the original and the paraphrase are two different propositions. Third, he claims nonetheless that {\em neither} the original nor the paraphrase entail the existence of chairs.

This may strike one as obviously untrue. How can he claim that when someone says ``There is a chair\,\ldots '' and means just that, that the proposition they express does not entail the existence of chairs? To defend his claim, van Inwagen appeals to his `Copernican analogy':
\begin{squote}
I accept the Copernican Hypothesis. One day you hear me say, ``It was cooler in the garden after the sun had moved behind the elms.'' You say, ``You see, you can't consistently maintain your Copernicanism outside the astronomer's study. You say that the sun moved behind the elms; yet, according to your official theory, the sun does not move.'' I reply that the proposition I expressed by saying ``It was cooler in the garden after the sun had moved behind the elms'' is consistent with the Copernican Hypothesis~(\citeyear[101]{inwagen1995}).
\end{squote}
That is, van Inwagen claims that the proposition he expressed with ``It was cooler in the garden after the sun had moved behind the elms'' does not entail that the sun actually moved. And he argues that this is analogous to our talk of chairs: most propositions expressed with ``There is a chair\,\ldots '' do not entail that chairs actually exist.

First, does the proposition van Inwagen expresses with ``The sun moved behind the elms'' entail that the sun moved? I am inclined to say that it does. If I were to say simply ``The sun moved'' (meaning just that), I think I would have committed myself to the movement of the sun. Why should we think that the addition of ``behind the elms'' defeats this entailment? Without some explanation of what the difference is, I see no reason to think that saying ``The sun moved behind the elms'' (and meaning it) does not entail the movement of the sun. But van Inwagen may be forced to say here that neither proposition entails that the sun moved. For he certainly won't allow that either entails that the sun {\em exists.}

There is an analogy here, though perhaps not the one van Inwagen had in mind. He claims that a proposition expressed by ``There are two very valuable chairs in the next room'' does not necessarily entail the existence of chairs. If this proposition does not entail that chairs exist, then what about ``There are two valuable chairs left in the world'' or ``There are at least two chairs in the world'' or ``There are at least two chairs'' or simply ``There are chairs''? Van Inwagen appears committed to the claim that the proposition I would express with ``There are chairs'' does not entail that there are chairs.

Why on earth should this be? Does not the proposition expressed by my saying ``There are simples arranged chairwise\,\ldots '' entail the existence of simples? If van Inwagen says that there are simples arranged chairwise, and means just that, then it would appear to follow that there are simples. Van Inwagen's argument relies rather heavily on the assumption that simples exist.%
%
\footnote{Ted Sider takes him to task for this assumption~(\citeyear{sider1993}), claiming that the possibility of `gunk'---the possibility that the matter of the world is not fundamentally particulate but infinitely divisible---falsifies van Inwagen's thesis. I think it may be possible for van Inwagen to adapt to a gunky world (he might be able to claim that nothing exists but organisms, who are composed of other organisms and/or gunk), but I think van Inwagen's thesis is false either way.}
%
\ But if ``There are chairs'' does not entail that there are chairs and if ``The sun moved behind the trees'' entails neither that the sun moved nor that the sun exists, then how can van Inwagen maintain that ``There are simples arranged chairwise'' entails that there are simples, or that they are arranged chairwise? He has given us no reason to believe one and not the other.

\subsection{Loose truth, again}
When criticizing Peter Unger's nihilism, we imagined a defense of his thesis based on the notion of `loose truth'. Our Unger partisan claimed that while such claims as ``There is a chair\,\ldots '' are invariably false (because incoherent), they may be loosely true. Unfortunately for Unger, his defender was unable to give a loose-truthmaker for these supposed loose truths.

Might van Inwagen also appeal to loose truths? He admits it as a last-ditch possibility:
\begin{squote}
I can say this [that ``There are chairs\,\ldots '' can be true yet not entail that there are chairs] because I accept certain theses in the philosophy of language. I can say this because I accept certain theses in the philosophy of language. Some people, I suppose, would reject these theses. These people would say that when I said\,\ldots\,`The sun moved behind the elms,' I said something false\,\ldots\,If someone maintains that `The sun moved behind the elms' expresses a falsehood, he must still have some way to distinguish between this sentence and those sentences (like `The sun exploded' and `The sun turned green') that the vulgar would regard as the sentences that expressed falsehoods about the sun\,\ldots\,[if I took this line,] I should not be willing to say that people who uttered things like `There are two valuable chairs in the next room' very often said what was true. I should be willing to say only that they very often say what might be treated as a truth for all practical purposes~(\citeyear[102--103]{inwagen1995}).
\end{squote}
We can distinguish the given propositions based on the fact that only ``The sun moved behind the elms'' has a loose-truthmaker:
\begin{squote}
Owing to a change in the relative positions and orientations of the earth and the sun, it came to pass that a straight line drawn between the sun and this point (which is on the surface of the earth) would have passed through the elms~\citep[112--113]{inwagen1995}.
\end{squote}

But now what is the loose-truthmaker for ``There are chairs''? Presumably van Inwagen will say that it is the fact that there are simples arranged chairwise. Therefore the beliefs that we take to be about tables and chairs are, if they are about anything at all, about the arrangements of simples. For this to be a legitimate move, however, van Inwagen needs to give a non-circular definition of `chairwise'. The loose-truthmaker for the sun's movement contained no appeal to movement; the movement in question was defined in other terms. Likewise, the loose-truthmaker for the chair's existence must not make a covert appeal to chairs.

\subsection{Chair(wise)}
We are demanding that van Inwagen give us definitions of `chairwise' and `chair' that are not definitions in terms of each other. Van Inwagen claims to be able to do this; responding to criticism by Jay Rosenberg, he says that ``it is easy to see how to define `chairwise' in terms of `chair' without supposing that there are any chairs. Let a ``chair'' be defined as an object that has the properties $C_{1}, C_{2},\,\dots\,C_{n}$''~(\citeyear[719]{inwagen1993b}). In order for the definitions to be non-circular, these properties must not include things like ``is a chair'', ``is shaped chairwise'', etc.

I am dubious that van Inwagen can provide us with necessary and sufficient property-list definitions of chair and chairwise that are not circular. And I am not alone in my skepticism:
\begin{squote}
When one says chair, one thinks vaguely of an average chair. But collect individual instances, think of arm-chairs and reading chairs, and dining-room chairs and kitchen chairs, chairs that pass into benches, chairs that cross the boundary and become settees, dentists' chairs, thrones, opera stalls, seats of all sorts, those miraculous fungoid growths that cumber the floor of the Arts and Crafts Exhibition, and you will perceive what a lax bundle in fact is this simple straightforward term. In co-operation with an intelligent joiner I would undertake to defeat any definition of chair or chairishness that you gave me.% Chairs just as much as individual organisms, just as much as mineral and rock specimens, are unique things---if you know them well enough you will find an individual difference even in a set of machine-made chairs---and it is only because we do not possess minds of unlimited capacity, because our brain has only a limited number of pigeon-holes for our correspondences with an unlimited universe of objective uniques, that we have to delude ourselves into the belief that there is a chairishness in this species common to and distinctive of all chairs
~\citep[384--385]{wells1904}.
\end{squote}

Without a proposal as to how we can capture all these types of chairs under an exclusive definition (leaving none out and bringing nothing else in), I suggest that `chair' has no necessary and sufficient property-list definition; it is a `family resemblance'-type concept. We can say well enough whether a given object is a chair or not, but this is not because it has all and only those properties that belong to chairs. [It is because\,\ldots ?]

\section{Now what?}
We have come to the following conclusions:
\begin{enumerate}
	\item Ordinary things (tables, chairs, people, \&c.) exist, and we communicate intelligibly about them. Therefore, most versions of nihilism are false.
	\item Because we communicate intelligibly about ordinary things, we reject versions of universalism that entail a profusion of tables where we assume a single table to be.
	\item Van Inwagen's paraphrasing strategy could not coherently show that a literally true proposition expressed by ``There are chairs\,\ldots '' does not entail that there are chairs. He cannot appeal to loose truth until he has given non-circular definitions of `chair' and `chairwise'.
	\item No such definition seems to be forthcoming.
\end{enumerate}

Although the theories presented by Unger and van Inwagen are unsatisfactory, the problems they are meant to solve do require some answer. Despite the fact that we communicate perfectly well with terms like `chair' and `person', it is difficult to spell out exactly what it is that we use these terms to refer to. A chair may lose some of its matter (we might sand it down) or gain more matter (we might paint it), yet we generally assume that the same chair endures the change. We shed cells constantly, but it would be overhasty to conclude that we do not persist through time, because we are not continuously constituted out the very same matter. When we refer to these chairs and people, then, it seems that the things we are referring to are not simply the particles that make them up. But then what are they?

\ifstandalone
\end{spacing}
\bibliography{everything}
\bibliographystyle{ChicagoReedweb}
\fi
\end{document}
\documentclass[11pt]{article}
\usepackage{standalone} \newif\ifstandlone \standalonetrue
\usepackage[left=1.75in, right=1.75in, top=1.25in, bottom=1.25in]{geometry}
\geometry{letterpaper}
\usepackage{verbatim}
\usepackage{graphicx}
\usepackage{enumitem}
%\usepackage{amssymb}
\usepackage{amsmath}
\usepackage{epstopdf}
\usepackage{setspace}
\usepackage{natbib}
\setcitestyle{aysep={}}
\usepackage{hyperref}
\usepackage{url}
\synctex=1

\DeclareSymbolFont{symbolsC}{U}{txsyc}{m}{n}
\DeclareMathSymbol{\strictif}{\mathrel}{symbolsC}{74}
\DeclareMathSymbol{\boxright}{\mathrel}{symbolsC}{128}

\newenvironment{squote}{%
\begin{spacing}{1}
\begin{list}{}{%
    \setlength{\labelwidth}{0pt}%
    \rightmargin\leftmargin%
  }
\item\relax
}{%
\end{list}%
\end{spacing}
}

\title{Paraphrases}
\author{Alexander A. Dunn}
\begin{document}
\ifstandalone
\maketitle
\begin{spacing}{1.5}
\fi

\section{Paraphrases}
\label{van-paraphrase}
I have proposed that any attempt to deny the existence of ordinary
things such as tables and chairs must be supplemented by an
explanation as to why we believe in the existence of ordinary things.
As we will see, Peter Unger's claim that there are no ordinary things
and his claim that propositions like ``that is a chair'' are uniformly
false leaves it quite mysterious why we take there to be chairs in the
first place.  Some nihilistic philosophers, therefore, have attempted
to maintain Unger's first thesis---that there are no ordinary
things---while rejecting the second---that ordinary thing discourse is
invariably false.  Such a philosopher will claim that such discourse
is {\em compatible} with the nonexistence of chairs.  This may involve
the claim that while we take ourselves to have beliefs about chairs
and other ordinary things, our beliefs do not actually concern such
(non-existent) entities.  Rather, our thought and talk is (or should
be seen as) relating to such things as do exist.  Strategies that
follow this pattern can be called paraphrasing strategies, and Peter
van Inwagen has presented a well-known version.

Like Unger, van Inwagen claims that (necessarily) there are simply no
such things as tables and chairs in the world.  But unlike Unger, he
does not claim that when we take ourselves to be thinking and talking
about such things, we are thinking and talking about nothing at all.
At least, ``when people say things in the ordinary business of life by
uttering sentences that start `There are chairs\,\ldots ' or `There
are stars\,\ldots ', they very often say things that are literally
true'' \citep[102]{inwagen1995}.  

One might assume that if such statements are true, then it follows
that there are chairs and stars.  But van Inwagen denies that chairs
and stars exist.  How can he claim, then, that what was said was true?
What van Inwagen does is attempt to show that the statements in
question can be {\em paraphrased}---they can be reformulated to show
that they have no `ontological commitments'.  According to van
Inwagen, one can assert that there is a chair without being committed
to the existence of chairs.

Section \ref{comp} will summarize the motivation for van Inwagen's
denial.  Section \ref{i-para} will introduce and criticize van
Inwagen's paraphrasing strategy.

\subsection{Composition}
\label{comp}
Van Inwagen's conclusion that there are no chairs is a consequence of
his views on {\em composition} (or `constitution').  Some things are
said to compose another thing if the former are {\em parts} of the
latter; the latter is `made up of' the former.  Van Inwagen believes
that ``the metaphysically puzzling features of material objects are
connected in deep and essential ways with metaphysically puzzling
features of the constitution of material objects by their
parts'' \citep[18]{inwagen1995}.  An example of such a puzzle is the
Ship of Theseus.  The Ship of Theseus is (presumably) an object
composed of many parts, including planks of wood.  As the planks (and
other parts of the ship) wear out, they are replaced.  These
replacements happen each by themselves; the entire ship (or even a
large section) is not replaced all at once.  But eventually no part of
the original ship remains; it is build of entirely different planks,
nails, rigging, etc.  And yet we would commonly say that it is still
the same ship.  But why should we think that the present ship is
identical with a past ship with which it shares no parts?

\subsection{The Special Composition Question}
\label{scq}
Answering the question `why is this ship identical with that past
ship?' requires first figuring out why (and how) these planks and
rigging and sails (et.\ al.) compose a ship in the first place.  Van
Inwagen asks ``in what circumstances do planks\footnote{For
  simplicity's sake, van Inwagen ignores the rigging and sails.}
compose (add up to, form) something?'' (\citeyear[21]{inwagen1995}) 
For some $x$s, then, van Inwagen asks us to consider when
\begin{displaymath}
\exists y\ \text{the}\ x\text{s compose}\ y
\end{displaymath}
is true.%
\footnote{Van Inwagen explains in some detail how plural referring
  expressions (like ``the planks'') can be given a logical
  formalization (\citeyear[23--28]{inwagen1995}), but suffice to say
  they work just as one would expect.}
%
\ Less formally, van Inwagen asks: ``suppose one had certain
(nonoverlapping) objects, the $x$s, at one's disposal; what would one
have to do---what {\em could} one do---to get the $x$s to compose
something?'' (\citeyear[31]{inwagen1995})  This is the Special
Composition Question.

(`Composition' is used in a technical sense with regard to the Special
Composition Question.  Van Inwagen defines it thus: ``the $x$s compose
$y$'' means that ``the $x$s are all parts of $y$ and no two of the
$x$s overlap and every part of $y$ overlaps at least one of the
$x$s\,\ldots\,a thing {\em overlaps} a thing---or: they overlap---if
they have a common part'' (\citeyear[29]{inwagen1995}).  For van
Inwagen, everything is a part of itself; some $x$ is a {\em proper}
part of some $y$ only if $x \neq y$.)

\subsection{The usual answers}
\label{scq-ans}
There are several prominent answers to the Special Composition
Question, including the following (these formulations are from
\citet{markosian1998a}):
\begin{description}
	\item[Nihilism] Necessarily, for any $x$s, there is an object
          composed of the $x$s iff there is only one of the $x$s,
          i.e., the only objects that exist are
          simples (\citeyear[219]{markosian1998a}).%
\footnote{\label{flip} Note that this may not be Unger's view.  He
  denies that people, apples, cheese, tables, chairs, and other
  ``ordinary things'' are nonexistence but he does not, as far as I
  know, take a stand on whether anything at all exists.  His view can
  be (flippantly) summarized thus: ``if we have a word for it, it
  doesn't exist.''}
	\item[Universalism] Necessarily, for any $x$s, there is an
          object composed of the $x$s iff no two of the $x$s
          overlap (\citeyear[227]{markosian1998a}).
	\item[Van Inwagenism] Necessarily, for any $x$s, there is an
          object composed of the $x$s iff either (i) the activity of
          the $x$s constitutes a life or (ii) there is only one of the
          $x$s (\citeyear[221]{markosian1998a}).
\end{description}

We will discuss Unger's version of nihilism in section \ref{unger}.
As I will argue, any version of nihilism that does not explain our
beliefs in the existence of ordinary objects is problematic.

Universalism raises a number of issues, some in connection with
Trenton Merricks' explanation of our beliefs and others in relation to
Peter Unger's arguments for nihilism.  I will therefore postpone
discussion of this view until later.  (See sections \ref{universalism}
and \ref{many}).

Van Inwagen examines and rejects universalism and the version of
nihilism given above.  He also rejects a number of other answers to
the Special Composition Question.  Some are too strong: `some $x$s
compose a $y$ iff the $x$s are in contact' would entail that two
people shaking hands will result in a new object coming into being.
Others are too strong in some ways and too weak in others: `some $x$s
compose a $y$ iff the $x$s are fastened together' would entail that
two people being glued together would result in a new object; and it
would deny that an object can be composed without fastening its parts
together (such as when building a house of cards).  The only answer
van Inwagen finds consistent is what we have dubbed {\em van
  Inwagenism}, which entails that tables and chairs do not exist.

Because of this consequence, van Inwagenism should include an
explanation why we nonetheless believe that there are tables and
chairs.  Happily, van Inwagen recognizes this and is prepared with a
{\em paraphrasing strategy}.  This strategy aims to show that the
beliefs that we take to be about tables and chairs are really about
something else, and are not true beliefs.  If such beliefs are true,
then it should be relatively easy to explain why hold them: they are
true, and we learn of them through some reliable means (like our
eyes).

Unfortunately, van Inwagen's paraphrasing strategy does not work.

\section{Van Inwagen's paraphrasing strategy}
\label{i-para}
Peter Unger, as we will see, claims that terms like `chair' are
incoherent; if this is so, a statement involving the phrase `There is
a chair\,\ldots ' could surely not be true.  Van Inwagen, on the other
hand, admits that ``when people say things in the ordinary business of
life by uttering sentences that start `There are chairs\,\ldots ' or
`There are stars\,\ldots ', they very often say things that are
literally true'' \cite[102]{inwagen1995}.  It does not seem
unreasonable to assume that if what people say with ``There are
chairs\,\ldots '' and the like are true, then chairs exist.  But van
Inwagen denies this entailment.

How can van Inwagen maintain this?  We may first observe that someone
can say, truly, ``There are simples arranged chairwise\,\ldots ''
without committing oneself to the existence of chairs.  Van Inwagen
might then claim that when someone says ``There is a chair\,\ldots ''
she {\em means} `There are simples arranged chairwise.'  This is, of
course, a bold hypothesis about the speech practices of ordinary
speakers.  Certainly very few speakers would, if asked, affirm that
what they meant to say had anything to do with simples; they would say
that when they said that there was a chair, they meant just that.  Van
Inwagen recognizes that this is not a viable position: ``The only
thing I have to say about what the ordinary man really means by `There
are two valuable chairs in the next room' is that he really means that
there are two valuable chairs in the next room''
(\citeyear[106]{inwagen1995}).

One might then assume that van Inwagen is thinking in analogy with
Russell.  He could attempt to claim that, despite the surface
appearance of language (`There is a chair\,\ldots '), the underlying
logical form does not make any mention of chairs (or tables); the
offending concept is analyzed away, leaving `There are simples
arranged chairwise\,\ldots '.  Van Inwagen notes that his ``suggested
technique of paraphrasing enables us to escape some of the more
embarrassing consequences of this position.  When someone says `Some
tables are heavier than some chairs,' there is obviously something
right about what he says.  Our technique of paraphrasis enables us to
capture what it is that is right about what he says''
(\citeyear[111]{inwagen1995}).  However, on the very next page he
admits that the ordinary language proposition and his paraphrased
version are different propositions: ``When the ordinary man utters the
sentence `Some chairs are heavier than some tables' (in an appropriate
context, and so on and so on), he expresses a certain proposition, and
one that is almost certainly true.  But I do not claim that this
proposition {\em is} the proposition that, for some $x$s, those $x$s
are arranged chairwise and for some $y$s, those $y$s are arranged
tablewise, and the $x$s are heavier than the $y$s''
(\citeyear[112]{inwagen1995}).  So van Inwagen is not making an appeal
to some notion of `logical form'.  But then what is the purpose of the
paraphrasing project?

Van Inwagen attempts to justify his method of paraphrasis by asserting
the following parallels between the original and paraphrased
propositions:
\begin{enumerate}[label=(\Alph*)]
	\item The paraphrase describes the same fact as the
          original.  \label{para-a}
	\item The paraphrase, unlike the original, does not even
          appear to imply that there are any objects that occupy
          chair-receptacles.  \label{para-b}
	\item The paraphrase is neutral with respect to competing
          metaphysical theories, {\em viz}.  the ``received'' theory,
          that there are objects that occupy chair-receptacles, and
          the theory I have proposed, according to which there are no
          such objects.  \label{para-c}
	\item The original, though it doubtless does not express the
          same proposition as the paraphrase, has the feature ascribed
          to the paraphrase in \ref{para-c}: It is neutral with
          respect to the question whether there are objects that fit
          exactly into
          chair-receptacles~(\citeyear[113]{inwagen1995}).  \label{para-d}
\end{enumerate}

I am willing to grant that \ref{para-a}--\ref{para-c} are true, but I
am quite sure that \ref{para-d} is false, and van Inwagen's thesis
appears to depend on it.  He admits in \ref{para-b} that the original
sentence (`There are chairs\,\ldots ') {\em implies} that there are
chairs, but claims in \ref{para-d} that it does not {\em entail} this.
But why wouldn't it?

\subsection{Propositions and ontological commitment}
\label{prop-ont}
Let us review the situation.  First, van Inwagen agrees that when
someone says things like ``There is a chair\,\ldots '' they mean just
that.  Second, he admits that his `paraphrases' of such propositions
are not so faithful to the original that they can be called the same
proposition; the original and the paraphrase are two different
propositions.  Third, he claims nonetheless that {\em neither} the
original nor the paraphrase entail the existence of chairs.

This seems obviously untrue.  How can he claim that when someone says
``There is a chair\,\ldots '' and means just that, that the
proposition they express does not entail the existence of chairs?  To
defend his claim, van Inwagen appeals to his `Copernican analogy':

\begin{squote}
I accept the Copernican Hypothesis.  One day you hear me say, ``It was
cooler in the garden after the sun had moved behind the elms.''  You
say, ``You see, you can't consistently maintain your Copernicanism
outside the astronomer's study.  You say that the sun moved behind the
elms; yet, according to your official theory, the sun does not move.''
I reply that the proposition I expressed by saying ``It was cooler in
the garden after the sun had moved behind the elms'' is consistent
with the Copernican Hypothesis (\citeyear[101]{inwagen1995}).
\end{squote}
That is, van Inwagen claims that the proposition he expressed with
``It was cooler in the garden after the sun had moved behind the
elms'' does not entail that the sun actually moved.  And he argues
that this is analogous to our talk of chairs: most propositions
expressed with ``There is a chair\,\ldots '' do not entail that chairs
actually exist.

Does the proposition van Inwagen expresses with ``The sun moved behind
the elms'' entail that the sun moved? I am inclined to say that it
does.  If I were to say simply ``The sun moved'' (meaning just that),
I think I would have committed myself to the movement of the sun.  Why
should we think that the addition of `behind the elms' defeats this
entailment?  Without some explanation of what the difference is, I see
no reason to think that saying ``The sun moved behind the elms'' (and
meaning it) does not entail the movement of the sun.  Likewise, if
``there are chairs in the next room'' does not entail that there are
chairs, then it would appear that ``there are chairs'' does not entail
that there are chairs.

Before we dismiss van Inwagen's paraphrasing strategy, we should
examine another, perhaps more plausible, analogy.  This analogy
involves an imaginary planet called Pluralia where there is a
``creature'' known as a bliger.  The bliger, according to van Inwagen,
is what happens when four monkeys, an owl, and a sloth attach
themselves together temporarily.  The conglomeration appears to the
untrained observer to be a single animal.  Gullible farmers have
dubbed this conglomeration a `bliger'.  Van Inwagen's point is that
there are no bligers, but that a farmer saying ``there's a bliger''
when pointing at such a conglomeration would be saying something true.
Even though there are no bligers (according to van Inwagen), someone
saying ``there's a bliger'' says something true because she reports a
fact.  The fact being reported by ``there's a bliger'' is the fact
that a monkey, four owls, and a sloth are there.  If she has instead
said ``that bliger just exploded'', what she said would be false,
because there is no fact that her proposition reports.

People believe that there are bligers because they mistake the group
of animals for a single thing, which has been dubbed `bliger'.
Likewise, van Inwagen maintains that people mistake chairwise
arrangements of simples for chairs.  When someone says ``there's a
chair'' what she says is true because it reports a fact.  The fact
being reported is that there is a chairwise arrangement of simples
there.  People believe that there are chairs because they mistake the
things arranged chairwise for a single thing, which has been dubbed
`chair'.

I agree with van Inwagen that these cases are analogous.  However,
where van Inwagen takes this analogy to show that there are no chairs,
I take it to show that there {\em are} bligers in van Inwagen's
imaginary scenario.  When it is discovered that bligers are built up
from six other creatures, we are learning something about bligers:

\begin{squote}
\ldots {\em of course} there are bligers in [van Inwagen's] story.
Bligers are what the story is about.  The zoologists do not report
that there are no bligers.  Rather they tell us what a bliger is.
They explain that a bliger is not a single large carnivorous animal
but a transient symbiotic union of six animals
\citep[704]{rosenberg1993}.
\end{squote}

In short, van Inwagen's analogy does not provide us with an
explanation of why we would believe in chairs even if there were
none.  All it should be taken to show is that just as we believe that
there are chairs because there are chairs, so we would believe there
were bligers if there were bligers.

\subsection{Backfire}
\label{backfire}
Van Inwagen should be thankful that his analogy does not succeed.  If
he showed that ``there are chairs'' does not entail that there are
chairs, then the whole notion of ontological commitment would be
undermined.  If ``there are chairs'' did not entail that there are
chairs, then why should any proposition of the form ``there are $x$s''
entail that there are $x$s?

It is surely true that van Inwagen would affirm ``there are simples
arranged chairwise''.  And no doubt he would affirm that it follows
from the truth of that proposition that there are simples arranged
chairwise.  But how can he affirm this, if he denies that ``there are
chairs'' entails that there are chairs?

If ``there are chairs in the next room'' does not entail that there
are chairs and if ``the sun moved behind the trees'' does not entail
sun moved (nor that it exists), then how can van Inwagen maintain that
``there are simples arranged chairwise'' entails that there are
simples, or that they are arranged chairwise?  He has given us no
reason to believe one and not the other.

\section{Lessons}
\label{lessons-v}
Van Inwagen has not succeeded in explaining why we believe that there
are chairs when (according to him) there are none.  This gives us
reason be suspicious of his conclusion that there are no ordinary
things except for organisms.  This explanatory deficiency should give
us hope that what seems obviously true---that there are
chairs---really is so, and that van Inwagen's argument against that
truth is faulty.

If van Inwagen's conclusion is false, there must be something wrong
with his argument.  One possibility is that he has overlooked a better
answer to the Special Composition Question.  But it is also possible
that the Special Composition Question is itself the wrong question to
be asking.  Jay Rosenberg brings out this worry nicely.  Van Inwagen's
informal version of the question is this: ``suppose one had certain
(nonoverlapping) objects, the $x$s, at one's disposal; what would one
have to do---what {\em could} one do---to get the $x$s to compose
something?''  (\citeyear[31]{inwagen1995}) This is how Rosenberg
replies:

\begin{quote}
To me it just seems obvious that the answer to such a question will
always depend on what sorts of things one has at one's disposal and
what sort of thing one is trying to get them to compose.  If the $x$s
are, for example, ``a lot of wooden blocks that one may do with as one
wills'', then to get them to compose, for example, a wall, it may be
sufficient to stack them up in the manner we call ``building a wall.''
To get them to compose a wooden raft, on the other hand, one would
surely need to fasten them together more securely, e.g., by gluing
them to one another.  And there's nothing at all one could do with
them to get them to add up to a fish or a clock or a sports car
(\citeyear[705]{rosenberg1993}).
\end{quote}

So it may be that, rather than asking ``what is required for
composition'', we should be asking ``what is required to compose a
chair/boat/house/etc.?''  There may be no answer to the Special
Composition Question as it is formulated.  There may not be any single
composition `operation' that applies to every composite material
object.  Rather, different composite objects are composed in different
ways.  Moreover, explaining how and why different objects are composed
in the ways they are will draw upon different fields of study:
``Microphysics explains how protons, neutrons, and electrons compose
different species of atoms, and physical chemistry, how atoms of
various species compose different sorts of molecules''
\citep[706]{rosenberg1993}.

\subsection{A pluralist `answer'}
\label{pluralist}
Ned Markosian objects to answers like this.  He claims that it is a
mistake to look to the empirical sciences to answer a metaphysical
question:

\begin{squote}
Rosenberg seems to want to suggest that sciences such as microphysics,
physical chemistry, and biology can help us to answer SCQ [the Special
  Composition Question].  This is puzzling to me because those
sciences are, after all, empirical sciences, where as a correct answer
to SCQ would have to express a proposition that is necessarily true.
For this reason, it is hard for me to see how exploring any of the
sciences mentioned by Rosenberg could help us to find a correct answer
to SCQ \citeyearpar[229]{markosian1998a}.
\end{squote}

But I think Rosenberg is not claiming that {\em the answer} to the
Special Composition Question is a conjunction of claims about
microphysics, physical chemistry, biology, and the other empirical
sciences.  Rather, Rosenberg is claiming that there is {\em no
  answer}---or at best a trivial answer like ``the $x$s compose $y$
iff the $x$s compose $y$; see the empirical sciences for more''.

\subsection{A brute answer}
Markosian himself claims that, while there is indeed no ``no true,
non-trivial, and finitely long answer to [the Special Composition
  Question]'' \citeyearpar[214]{markosian1998a}, this is not because
we should refer questions of composition to the empirical sciences.
Rather, he claims that whether or not some things compose another is
simply a {\em brute fact}.

[More to come, but] this is a clever answer, but whether true or not I
do not think it serves the purpose that Markosian expects it to.  He
proposes his theory of `brutal composition' to be ``consistent with
standard, pre-philosophical intuitions about the universe's composite
objects'' \citeyearpar[211]{markosian1998a}.  But his theory will only
be consistent with such intuitions if, first, it is a brute fact that
all (or nearly all) of the things we ordinarily take to exist (chairs
etc.) do in fact exist, and, second, that it is a brute fact that the
things that we don't take to exist (or that Markosian takes not to
exist) don't in fact exist.  The chance that the brute facts of
composition happen to line up with our (or Markosian's) intuitions
seems to be incredibly low.

I think that if Markosian is right that there is no interesting answer
to the Special Composition Question, it is not because whether things
compose other things is a brute fact.

\subsection{The allure of the Special Composition Question}
There is, however, something compelling about the Special Composition
Question.  Certainly van Inwagen is not the only metaphysician to
think that it is the right question to be asking (\textbf{Sider, Dorr
  citations}).  I think Rosenberg is probably right when he claims
that the Special Composition Question does not have an answer (or at
least an interesting answer), but it is relatively easy to think
otherwise.  I think the emphasis that van Inwagen places on the
Special Composition Question is what helps him undermine the
obviousness of the fact that there are chairs.

Van Inwagen claims, rather explicitly, that it is not obviously true
that there are chairs: ``Common sense tells us to taste our food
before we salt it and to cut the cards.  It does not tell us that
there are chairs'' \citeyearpar[103]{inwagen1995}.  One reason van
Inwagen might think that it is not obviously true is that he thinks
the existence of any `composite' object---like a chair---must follow
from the answer to the Special Composition Question.  But if we reject
the Special Composition Question, there is no reason to think that
chairs must be `composed' in just the same way as people are.  The
fact that van Inwagen's answer to the Special Composition Question
cannot explain why chairs exist gives us no reason to think that
``there are chairs'' is not obviously true.  The lesson of van
Inwagen's nihilism is not that there are no chairs, but that what it
takes to be a chair may not be what it takes to be an organism.

Van Inwagen's approach to nihilism is not the only one, however.
Trenton Merricks has proposed a very similar thesis---that the only
composite objects are human beings---for very different reasons.  Like
van Inwagen, he tries to undermine the obviousness of the fact that
there are chairs, and he attempts to explain why we believe that there
are chairs at all.  Like van Inwagen's, I think his arguments do not
succeed.  But like van Inwagen's, they offer us lessons about what it
takes to be a chair.

\ifstandalone
\end{spacing}
\bibliography{everything}
\bibliographystyle{ChicagoReedweb}
\fi

\end{document}

\documentclass[11pt]{article}
\usepackage{standalone} \newif\ifstandlone \standalonetrue
\usepackage[left=1.75in, right=1.75in, top=1.25in, bottom=1.25in]{geometry}
\geometry{letterpaper}
\usepackage{graphicx}
\usepackage{enumitem}
%\usepackage{amssymb}
\usepackage{amsmath}
\usepackage{epstopdf}
\usepackage{verbatim}
\usepackage{setspace}
\usepackage{natbib}
\setcitestyle{aysep={}}
\usepackage{url}
\synctex=1
\usepackage{hyperref}

%% \newcounter{dcount}
%% \newcommand{\ditem}[1]{%
%%   \item[#1] \refstepcounter{dcount}\label{#1}
%% }
%% \newcommand{\dref}[1]{\hyperref[#1]{#1}}

\DeclareSymbolFont{symbolsC}{U}{txsyc}{m}{n}
\DeclareMathSymbol{\strictif}{\mathrel}{symbolsC}{74}
\DeclareMathSymbol{\boxright}{\mathrel}{symbolsC}{128}

\newenvironment{squote}{%
\begin{spacing}{1}
       	\begin{list}{}{%
\setlength{\labelwidth}{0pt}%
\rightmargin\leftmargin%
}
\item\relax
}{%
\end{list}%
\end{spacing}
}

\title{``Nearly as good as true''}
\author{Alexander A. Dunn}
\begin{document}
\ifstandalone
\maketitle
\begin{spacing}{1.25}
\fi

\noindent Trenton Merricks' explanation of why we believe that there
are chairs is much more plausible that van Inwagen's.  However, the
plausibility of Merricks' explanation relies on his ability to
undermine our reasons to think that it is obviously true that there
are chairs.  I will argue that his attempt to show that ``there are
chairs'' is not obviously true fails; consequently, his explanation of
why we believe that there are chairs fails too.  

The lesson of van Inwagen's nihilism was that the Special Composition
Question is probably a poor question.  The lesson of Merricks'
nihilism is that `composition' is itself a tricky notion.  It is
useful, when formally describing a thing and its parts, to describe
the thing as `composed' of its parts.  But talking about when things
do or do not `compose' another things is an unintuitive way to
talking, which tends to make nihilism seem more plausible than it is.

\section{Merricks and the indispensability of ordinary concepts}
\label{merricks}
Trenton Merricks comes to the same metaphysical conclusions as does
van Inwagen.  That is, he claims that there are no physical objects
other than human beings.  However, he comes to this conclusion through
a different path of reasoning.  (I have not studied these arguments
yet, so I will skip that part for now.)

Despite the fact that Merricks has a different motivation for his
nihilism, we can pose the same question to him as we posed to van
Inwagen.  Why, if there are no chairs, do we believe that there are
chairs?  Happily, Merricks addresses our concern.  Even more happily,
he has a better explanation than van Inwagen.  But sadly, even if his
explanation is good, it will probably be undermined by Unger's
arguments for nihilism.

\subsection{Nearly as good as true}
\label{near}
Merricks claims that `folk' beliefs, such as the belief that there are
chairs, are false, but nonetheless are {\em nearly as good as true}.
What does this mean?

\begin{squote}
People who believe in unicorns [or ghosts] are few and far between.
And those few are generally unjustified.  On the other hand, people
who believe in statues are legion.  And they are generally justified
in so believing.  Given the truth of eliminativism [what I have been
  calling nihilism], we might ask {\em why} the belief in statues is
more common, and more commonly justified, than the belief in unicorns.

The answer is that statue beliefs are nearly as good as true.  For, so
I claim here, {\em atoms arranged statuewise} often play a key role in
producing, and grounding the justification of, the belief that statues
exist.  In general, a false belief's being nearly as good as true
explains how {\em reasonable} people come to hold it.  And, relatedly,
its being nearly as good as true can ground its justification.
Because the belief that unicorns exist is not nearly as good as true
(i.e.\ because there are no things arranged unicornwise), there is no
similar explanation of its production or similar reason to think it is
justified (\citeyear[171--172]{merricks2001a}).
\end{squote}

To say that something is ``nearly as good as true'' seems to be
equivalent to saying that it is `loosely true', or `true for practical
purposes'.  In each case, the proposition in question is false, but it
is somehow close enough to the truth for a given purpose or situation.
For example, suppose we have decided to buy a fake holiday tree for
the holidays this year.  We are looking at a number of different fake
trees.  I point to one and say ``that is a nice tree''.  What I have
said is false; that is not a tree.  It is a fake tree.  But what I
mean---and what my audience recognizes me to mean---is that it is a
nice {\em fake} tree.  We both know that we are looking at fake trees;
there is no point qualifying every use of `tree' with `fake'.  When I
say ``that is a nice tree'', therefore, what I say is quite sufficient
to allow for successful communication. despite being false.  Merricks
claims that propositions expressed by things like ``there are chairs''
are also loosely true.  They are false, but are nonetheless good
enough for certain purposes.

Initially, this seems like a bizarre claim.  After all, Merricks is
claiming that chairs {\em necessarily} do not exist.  According to
Merricks, ``chairs exist'', given its current meaning, could {\em
  never} be true.  If the proposition expressed by ``chairs exist'' is
necessarily false, how could it nonetheless be ``nearly as good as
true''?

\subsection{The necessary connection}
\label{connection}
Merricks' argument relies on a very close conceptual connection
between ``chair'' and ``chairwise'' (and likewise for all ordinary
terms).  Despite claiming that chairs are impossible, Merricks admits
that we understand perfectly what chairs {\em would} be, if they
existed.  Because we understand the concept of `chair', we can
recognize {\em actually existing} things that are arranged
`chairwise':

\begin{squote}
The folk concept of \emph{statue} plays a role in determining which
atomic arrangements are statuewise. I would even go so far as to say
that if \emph{being arranged statuewise} were not derivative upon
folk-ontological concepts\,\ldots something would be amiss
(\citeyear[8]{merricks2001a}).
\end{squote}

For Merricks, to know what things are actually arranged statue- or
chairwise requires knowing what things would compose a chair, if
chairs were possible:

\begin{squote}
Atoms are \emph{arranged statuewise} if and only if they both have the
properties and also stand in the relations to microscopica upon which,
if statues existed, those atoms' \emph{composing a statue} would
non-trivially supervene (\citeyear[4]{merricks2001a}).
\end{squote}

When we look at Peter Unger's arguments for nihilism, we will see that
this close conceptual connection between terms like `chairwise' and
`chair' will cause trouble for Merricks.  But is Merricks' position
plausible anyway?  Does he have a good explanation of why we believe
in chairs?

Merricks' explanation is, at least, plausible.  One reason is the
structure of the explanation.  Recall that our explanation of why we
believe that there are chairs (or statues) is that, first, there are
chairs, and, second, we see that there are chairs (or learn that there
are chairs through a similarly reliable mechanism).  

Merricks' definition of `nearly as good as true' allows us to produce
a parallel explanation:

\begin{squote}
Any folk-ontological claim of the form `F exists' is \emph{nearly as
  good as true} if and only if (i) `F exists' is false and (ii) there
are things arranged F-wise. So, for example, `the statue \emph{David}
exist' is nearly as good as true because (it is false and) there are
some things arranged Davidwise (\citeyear[171]{merricks2001a}).
\end{squote}

We may now say on behalf of Merricks that we believe that there are
chairs (and statues) because, first, there are things arranged
chairwise and, second, we see that there are things arranged
chairwise.

The structure of the two explanations is analogous, but there is an
apparent disanalogy in the content of the two.  The disanalogy does
not favor Merricks.  For it is easy enough to see why there being
chairs, and us seeing that there are chairs, would cause us to believe
that there are chairs.  But it is less obvious why there being things
arranged chairwise, and us seeing that there are things arranged
chairwise, would cause us to believe {\em not} that there are things
arranged chairwise, but that there are {\em chairs}.

(While it is certainly true that we believe that there are chairs, I
am not sure if all or even most of us {\em also} believe that there
are things arranged chairwise.  Let us suppose for now that we do.)

The close conceptual connection between `chair' and `chairwise' is
very important for Merricks.  It is this {\em connection} that is
doing the explanatory work.  The only thing that can explain why there
being things arranged chairwise would cause us to believe that there
are chairs is this connection between the concepts.  The existence of
things arranged chairwise, and the belief that there are things
arranged chairwise, is supposed to cause the {\em additional} belief
that there are chairs.  How does this happen?

I believe that Merricks' answer would go something like this: certain
arrangements of things---chairwise arrangements, statuewise
arrangements, and all ordinary arrangements---play important roles in
our lives.  These arrangements of things are of interest to us, so we
have developed words that allow us to refer to them.  For whatever
reason---sociological, psychological, or otherwise---we think of each
arrangement as a single thing, rather than as things.  Words like
`chair' and `statue', being singular, reflect this (incorrect) view of
the world.  We are, in a sense, fooled by grammar.

This is more than Merricks says himself.  I have not found a passage
in which he explicitly describes the nature of the conceptual
connection between concepts like `chair' and `chairwise', and explains
why, from our belief that there are things arranged chairwise, we
invariably infer that there are chairs.  But I think he would endorse
something like this.  In the first chapter of his book, he claims that
whether there is a statue or merely things arranged statuewise is not
an empirical question.  He claims that were there not a statue and
merely things arranged chairwise, our ``visual evidence'' would be the
same.  He supports this claim with an analogy:

\begin{squote}
{[}Consider{]} the claim that the atoms arranged my-neighbour's-dogwise
and the-top-half-of-the-tree-in-my-backyardwise compose an
object\ldots{}it won't do to defend this claim with nothing more than `I
can \emph{just see} the object composed of the atoms arranged
dog-and-treetopwise'. Part of why this won't do, presumably, is that
one's visual evidence would be the same \emph{whether or not} those
atoms composed something (\citeyear[8--9]{merricks2001a}).
\end{squote}

He relies here on his assumption that we do not believe that there is
a thing composed of a dog and some of a tree (I will challenge this
assumption below).  Later he says that it is (at least possibly)
arbitrary to claim that there are statues but not dog-tree things:
``we ought to see that the only difference between arbitrary sums and
statues is a matter of conventional wisdom and local custom''
\citeyearpar[75]{merricks2001a}.  He seems sympathetic to the idea
that the reason we believe that there are statues, and not dog-tree
composites, is due to our conventional speech practices: ``it is at
least somewhat plausible that atoms arranged statuewise are united not
by composing something but, instead and in part, by how we speak and
think'' \citeyearpar[121]{merricks2001a}.

On this picture, whether we see an arrangement of things as composing
an object or not depends more on our own interests than features of
the things themselves.  We have words for chairs and statues because
things arranged chairwise and statuewise interest us.  We don't have a
word for things arranged ``my-neighbor's-dogwise and
the-top-half-of-the-tree-in-my-backyardwise'' because such an
arrangement does not hold much interest for us.  But each of these
arrangements exist, and it seems arbitrary to say that the chairwise
and statuewise arrangements compose chairs and statues while the other
arrangement composes nothing.

Merricks might explain why we believe that there are things arranged
my-neighbor's-dogwise and the-top-half-of-the-tree-in-my-backyardwise
thus: there are things arranged my-neighbor's-dogwise and
the-top-half-of-the-tree-in-my-backyardwise, and we see that there are
things so arranged.  This is exactly the same explanation that I would
give.

Now Merricks explains why we believe that there are chairs thus: there
are things arranged chairwise, and we see that there are things
arranged chairwise.  {\em And incidentally, due to our own human
  peculiarities, we have found it convenient to refer to and think
  about things arranged chairwise as if they were single objects}.

\section{Dogbushes}
\label{dogbush}
This is a somewhat plausible explanation of why we would belief that
there are chairs if there were not.  It is certainly much better than
van Inwagen's.  But I think that it fails.  I think that when we look
closer at Merricks' attempts to motivate nihilism, we will see that
they do not support nihilism at all.  If anything they support a
version of {\em universalism}.

Merricks observes that one might object to nihilism simply by saying,
``I just {\em see} the chair!''  He claims that if this objection
moves us, we should think about an analogous objection, which he finds
much less moving:

\begin{squote}
Whether atoms arranged statuewise compose a statue is analogous to
whether atoms arranged my-neighbour's-dogwise and
the-top-half-of-the-tree-in-my-backyardwise compose an object\,\ldots
it would not do to support an affirmative answer to the latter
question simply by saying `I can just see that object'
\citeyearpar[73]{merricks2001a}.
\end{squote}

It does indeed seem plausible to say that the top half of a tree and
my neighbor's dog do not `compose' anything.  But I think this is
incorrect, and I think we find it plausible only because of the word
`compose'.

Recall the bliger story that van Inwagen used to motivate his version
of nihilism (section \ref{prop-ont}).  A bliger was supposed to be
four monkeys, an owl, and a sloth, who arrange themselves into a
temporary symbiotic configuration.  Van Inwagen thought we would agree
that bligers did not exist.  Like Merricks, he brought in the language
of `composition' to make his claim more plausible.  Van Inwagen says
that it is not true that ``six animals arranged in bliger fashion
compose anything, and that is what I mean to deny when I say that
there are no bligers'' \citeyearpar[104]{inwagen1995}.

But as we saw, it is simply false that there are no bligers:

\begin{squote}
\ldots {\em of course} there are bligers in [van Inwagen's] story.
Bligers are what the story is about.  The zoologists do not report
that there are no bligers.  Rather they tell us what a bliger is.
They explain that a bliger is not a single large carnivorous animal
but a transient symbiotic union of six animals
\citep[704]{rosenberg1993}.
\end{squote}

The only reason we might be tempted to say that there are no bligers
is that van Inwagen presents the question in an unintuitive way.  He
asks us if there is some thing, some object, that is composed of the
other six animals.  This gives one the impression that, were there to
be such a thing, it would be another animal (a seventh); were there
such a thing, it should somehow pop out at us.  But all we see when we
picture the scene are the six animals together, so we feel that van
Inwagen might be right.  There is no {\em other} thing.  But if we ask
the question in a more intuitive manner, things become clearer.
Rather than ask if there is some thing composed of such and such other
things, we simply ask, ``are there bligers?''  And of course there
are.  Van Inwagen relied too heavily on the term `composition', and as
a result concluded that ``there are bligers'' was not only less than
obviously true, but actually false.

Merricks makes the same mistake in his passage above.  Following van
Inwagen, let's give our composite object a name.  For simplicity's
sake, I'm also going to change the example slightly.  I hereby
introduce `dogbush' as a term designating trees and dogs that are
within 3 meters of each other.  This term applies to situations like
these:

\begin{itemize}
  \item At the park, there are six trees, each with a (single) dog
    sleeping under it.  There are six dogbushes in the park.
  \item In the above example, if one dog walked away from its tree,
    there would be one less dogbush.
  \item If that dog then walked under another tree (where another dog
    was sleeping), we would again have six dogbushes.
\end{itemize}

Van Inwagen and Merricks might now ask us if there is object composed
of a tree and a dog when they are within 3 meters of each other.  But
let's set aside the language of `composition' and simply ask, ``are
there dogbushes?''

{\em Of course} there are dogbushes.  There are trees, and there are
dogs within 3 meters of those trees.  That's all it takes to make
``there are dogbushes'' true.  How could ``there are dogbushes'' be
false; how could there be no dogbushes?  Are there {\em no} dogs
within 3 meters of a tree?  There obviously are, and so it is
obviously true that there are dogbushes.  It is correct to say that a
dogbush is a thing composed of a dog and a tree, but to rely on that
formulation invites misinterpretation.

\subsection{Visual evidence}
\label{visual}
Merricks claims that our ``visual evidence'' would be the same whether
or not there were dogbushes, but this is simply false: if there were
no dogbushes, that would mean that there were no dogs within 3 meters
of trees.  Imagine if van Inwagen had claimed that the visual evidence
of his imagined farmers would be the same whether or not there were
bligers.  This would be nonsense, for if there were no bligers then
that would mean that owls, monkeys, and sloths never formed their
symbiotic unions.  And, {\em by hypothesis}, they do form such
unions.  Farmers see these unions.  Farmers would not see these unions
if there were no bligers.  Their visual evidence would certainly not
be the same.

The same point applies to dogbushes.  I believe there are dogbushes
because I {\em see} them.  In making this claim I am, apparently,
alone:

\begin{squote}
There are many philosophers who believe in arbitrary sums like the
`dog-and-treetop', but none of them---not one---defends the existence
of such things on merely perceptual grounds. No one says we should
believe that such an object exists simply because we can see it or
simply because we can hear it (gnawing on a bone while rustling its
leaves) \citep[74]{merricks2001a}.
\end{squote}

Merricks claims rather that, if we believe in such things, our belief
should be defended on {\em philosophical} grounds.  ``Merely
perceptual grounds'' are not good enough.  The point of this, of
course, is to undermine our belief that there dogbushes.  Once
Merricks has convinced us that there are no dogbushes, the trap is
sprung.  For what, really, is the difference between dogbushes and
ordinary things like statues?  Merricks thinks ``we ought to see that
the only difference between arbitrary sums and statues is a matter of
conventional wisdom and local custom.  Once this is pointed out, one
is no longer justified in believing that statues exist merely because
one can supposedly see them'' \citeyearpar[75]{merricks2001a}.  On the
contrary, I say that once this is pointed out, one is no longer
justified in believing that dogbushes {\em don't} exist.  For our
visual evidence would {\em not} be the same if there were none.

Likewise, if there were no chairs, our visual evidence would certainly
not be the same.  If there were no chairs, there would not be any
things arranged chairwise either.  And Merricks agrees that there are
things arranged chairwise.  Now how could ``there are chair'' be
false; how could there be no chairs?  Are there {\em no} things
arranged chairwise?  There obviously are, and so it is obviously true
that there are chairs.

\subsection{The meaning and truth-conditions of `chair'}
\label{meaning}
I think I know how Merricks would object.  He might take me to be
accusing him of `linguistic contradiction'.  He might think that I
think that `there are chairs' means `there are things arranged
chairwise'; to affirm the one and deny the other would be absurd.  His
response to this sort of objection is as follows:

\begin{squote}
`There are married bachelors' is no explicitly formally contradictory,
  but it is contradictory in some quite straightforward sense.  And
  one might object that `there are atoms arranged statuewise but no
  statues' is contradictory in the same way.  For as `bachelor' means
  someone who is, among other things, unmarried, so---the objector
  insists---`there are [composite] statues' {\em just means} that
  there are some things arranged statuewise.  Because of its
  contradictory nature, we should not take seriously an ontology
  according to which there are married bachelors.  Likewise, this
  objection concludes, we should not take seriously the
  eliminativist's ontology, with its atoms arranged statuewise but no
  statues \citeyearpar[13]{merricks2001a}.
\end{squote}

Merricks points out that ``\,`There are statues' does {\em not} mean
only that there are some things arranged statuewise\,\ldots this is
simply not a plausible claim about ordinary meaning''
\citeyearpar[13]{merricks2001a}.  And this seems quite true.  But I do
not need to claim that ``there are statues'' (or chairs) {\em means}
that there are things arranged statuewise (or chairwise).  All I need
to claim is that the following is true:

\begin{squote}
`There are chairs' is true if and only if `there are things arranged
  chairwise' is true.
\end{squote}

(If, as some claim, meaning is reducible to truth-conditions, then I
would be committed to the additional claim that `there are chairs' and
`there are things arranged chairwise' mean the same thing.  But if
Merricks does not hold this thesis about meaning, then he has no
grounds to resist the distinction between two propositions meaning the
same thing and their being truth-conditionally equivalent.  It is only
the latter that I claim.)

The claim above can be generalized:

\begin{description}
  \item[The $F$ existence principle] `There is an $F$'' is true if
    and only if ``There are things arranged $F$-wise'' is
    true. \label{fwise}
\end{description}

Nihilists like Trenton Merricks have assumed that what is required for
``there are chairs'' to be true is something more than what is
required for ``there are things chairwise'' to be true.  But this is a
mistake.  Ted Sider makes this mistake in a recent paper on parthood:

\begin{squote}
Consider Nihilo, god and creator of a world comprised solely of
subatomic particles.  On the first day Nihilo creates some particles
and arranges them in beautiful but lifeless patterns.  But he becomes
lonely, so on the second day he creates some minions (or rather,
particles arranged minion-wise).  On the third day he tries to teach
his minions to speak.  But this goes badly.  The minions aren't very
bright, and are slow to catch on to Nihilo's talk of subatomic
particles and their physical states.  So on the fourth day he teaches
them an easier way to speak.  Whenever an electron is bonded (in a
certain way) to a proton, he teaches them to say ``there is a hydrogen
atom''; whenever some subatomic particles are arranged chairwise he
teaches them to say ``there is a chair'', and so on.  (Pretend that
electrons and protons have no proper parts.)

When the minions utter sentences like ``there is a hydrogen atom'', do
they speak falsely?  They do if their language is the same as the
language I used to describe the example, since I described Nihilo as
having created a world comprised solely of subatomic particles
\citeyearpar[7]{sider2011c}.
\end{squote}

Sider seems to be thinking that he can stipulate that the
`metaphysical laws' of his imagined world make it impossible for there
to be minions.  But he has clearly not succeeded, for there are
minions right there in his story!  Nihilo created minions.  In
Nihilo's world, ``there are minions'' is true.  So is ``there are
hydrogen atoms''.  When the minions utter such sentences, what they
say is true.  It is true even if their language is the same language
as Sider used to describe the example (the language is English; recall
section \ref{english} above).  For there to be minions, nothing is
required over and above there being things arranged minionwise.
``There are minions'' is true if and only if ``there are things
arranged minionwise'' is true.

At one point Merricks attempts to motivate his nihilism by claiming
that ``whether atoms arranged statuewise compose a statue is not
straightforwardly empirical'' \citeyearpar[9]{merricks2001a}.  After
reminding us that our visual evidence would be the same whether or not
things arranged statuewise composed a statue, he writes:

\begin{squote}
The fundamental question is not so much whether some particular
alleged statue exists.  That question might---sceptical scenarios
aside---seem to be a matter of just looking and seeing.  The issue is
rather whether, in general, atoms arranged statuewise compose a
statue.  But it seems that this question of metaphysical necessity
cannot be decided, one way or another, simply by a trip to the museum
or a ride down Monument Avenue.  It must be decided on philosophical
grounds \citeyearpar[9]{merricks2001a}.
\end{squote}

I disagree.  The question ``whether things arranged statuewise compose
a statue'' is equivalent to the question whether there are statues.
If the answer to one is ``yes'', the answer to the other is also
``yes''.  The answer to the latter is ``yes''.  So the answer to the
former is ``yes''.  The answer to both is ``yes'', and this just {\em
  is} an empirical question that can be settled ``by a trip to the
museum''.

\section{Universalism}
\label{universalism}
Above I claimed that Merricks' explanation of why we believe that
there are chairs is something like this: there are things arranged
chairwise, and we see that there are things arranged chairwise.  {\em
  And incidentally, due to our own human peculiarities, we have found
  it convenient to refer to and think about things arranged chairwise
  as if they were single objects}.  I attributed to Merricks the idea
that just because things arranged chairwise interest us, we should not
therefore suppose that there are chairs.  What interests us should not
be a guide to what exists.  But now there is an obvious counter
against this move.  Just because dogbushes do {\em not} interest us,
we should not therefore suppose that there are not dogbushes.  (What
interests us should not be a guide to what exists.)

I said in section \ref{scq-ans} that I would postpone discussion of
universalism until later.  It it time now to discuss it.  For I seem
to be committing myself to some version of universalism.  The
formulation I cited above was this:

\begin{description}
\item[Universalism] Necessarily, for any $x$s, there is an object
  composed of the $x$s iff no two of the $x$s overlap
  \citep[227]{markosian1998a}.
\end{description}

Much of the resistance to this thesis (at least, much of my early
resistance) probably stems from its wording; as I argued above, the
unintuitive language of `composition' can lead us to deny obvious
truths, such as that there are chairs and dogbushes.  Ned Markosian's
objection to universalism provides another example of how the term
`composition' obscures the plausibility of universalism:

\begin{squote}
There is what seems to me a fatal objection to Universalism:
Universalism entails that there are far more composite objects than
common sense intuitions can allow.  To give just one example,
Universalism entails that the following sentence is true:\,\ldots
There is an object composed of (i) London Bridge, (ii) a certain
sub-atomic particle located far beneath the surface of the moon, and
(iii) Cal Ripken, Jr.  My intuitions tell me that there is no such
object, and I suspect that the intuitions of the man on the street
would agree with mine on this point \citeyearpar[228]{markosian1998a}.
\end{squote}

I suspect that Markosian's intuitions are led astray because he relies
so heavily on the term `composition'.  As in the case of the dog and
the tree, it does seem unintuitive that the tree and dog `compose'
something else.  But it is obvious that there are dogbushes.  Let us
therefore drop `composition' from Markosian's example and see if that
makes things clearer.

Let `Lumpkin Junior' designate the London Bridge, a particle in the
moon, and Cal Ripkin, Jr.  Does Lumpkin Junior exist?

Let us introduce another term so as to bring the analogy closer.  A
`lumpkin' is the London Bridge, a sub-atomic particle in the moon, and
a retired major league baseball player.  Now we can ask ``are there
lumpkins?''  And, as in the case of the dogbush, {\em of course there
  are}.  If there were no lumpkins, there would have to be either no
former ball-players, no London Bridge, or no particles in the moon.
Given that there are all these things, there are obviously lumpkins.
Lumpkin Junior is one of many lumpkins.

I cannot resist including another example, partly for its silliness
and partly in the hope that it will bring out further the way in which
terms like `composition' are misleading.  The Blues Brothers have a
song called ``Rubber Biscuit''.  In it, they refer to a `wish
sandwich'.  ``A wish sandwich,'' they say, ``is the kind of a sandwich
where you have two slices of bread and {\em wish} you had some meat.''
This term having been introduced, one can truthfully say, ``I had a
wish sandwich the other day.''  All that is required for this to be
true is that she had two slices of bread and wished she had some meat.
There was then a wish sandwich.  Should we say that there was some
`composite' thing, `composed of' two slices of bread and a wish?  We
can talk this way, I suppose, but it serves little purpose but to
confuse us.

%% Rather than risk being mislead by unintuitive formulations of
%% universalism, I will rely on and defend the $F$ existence principle
%% from \ref{fwise}: ``There is an $F$'' is true if and only if
%% ``There are things arranged $F$-wise'' is true.

\subsection{Begging the question}
\label{beg}
The most straightforward objection to the above is that I am begging
the question.  For I claim that it is {\em true} (if misleading) to
say that Lumpkin Junior is ``that object composed of the London
Bridge, a particle of the moon, and Cal Ripkin, Jr.''  So when
Markosian denies that Lumpkin Junior exists, my `counterargument' is
just that it does exist!

Above, I attempted to make it plausible that Lumpkin Junior exists.  I
claimed that the wording used by Markosian and others is unintuitive
and causes us to come to incorrect conclusions.  I think that if we
formulate the question more naturally (``are there lumpkins?''), it is
intuitively true that there {\em are} such things.  I think this is an
appropriate way to proceed, given that Markosian's argument {\em
  against} lumpkins is an appeal to intuitions as well.  He says is
that ``[his] intuitions tell [him] that there is no such object, and
[he] suspect[s] that the intuitions of the man on the street would
agree with [his] on this point'' \citeyearpar[228]{markosian1998a}.  I
suggest that his intuitions are led astray by an awkward wording.

I think Markosian and I are going through the following argument in
different directions:

\begin{enumerate}
  \item There are dogbushes
  \item If there are dogbushes, then they are things composed of dogs
    and trees.
  \item There are things composed of dogs and trees.
\end{enumerate}

Markosian begins at the bottom.  He finds it implausible to say that
there are things composed of dogs and trees, and so, by {\em modus
  tollens}, denies that there are dogbushes.  I start from the top.  I
find it perfectly plausible to say that there are dogbushes.  By {\em
  modus ponens}, I claim that there are things composed of dogs and
trees.

I think it is more appropriate to begin from the top.  As I have said,
the phrasing ``there are dogbushes'' is more natural; therefore our
intuitive judgments regarding it should be more accurate.  When we
arrive at technical formulations like ``there is some thing such that
it is composed of a dog and a tree'', our intuitions will be less
reliable.  Even if that proposition seems intuitively false, if it
follows from one that seems clearly true, then we should not reject
it unless we have good independent reasons for doing so.

Another example: the Reed College women's rugby team.  There is,
obviously, such a team (whether or not it has College funding).  The
rugby team exists.  Having established this, there are various
consequences.  For instance, each player is part of the team.  The
team is made up of the players and the coach.  Expressing this
formally, we might say that there is some thing (the team) composed of
the $x$s (the players and coach).  If this formal treatment is
equivalent to the informal ``there is a team'', and if the informal
phrasing is uncontroversial, the formal phrasing should not be
controversial either.

Merricks has a related example:

\begin{squote}
Consider whether `the Crew of the USS {\em Enterprise}' is a plural
referring expression---akin to `Locke, Berkeley, and Hume'---or,
instead, the name of a single large object with each crew member as a
proper part.  Note, in fact, that there are two questions here.
First, there is the semantic question of what `the Crew of the USS
{\em Enterprise}' is supposed to mean.  Second, there is the
metaphysical question of whether there really is a big physical object
that has all and only the crew members as its parts (at one level of
decomposition), a scattered object that weighs as much as the sum of
the weights of those people taken individually.

I am not sure how to answer the first question.  But, I say, the
answer to the second question is `no'.  Some philosophers would
disagree.  No matter.  The point here---in this section of this
chapter---is not to settle either the metaphysical or the semantic
dispute surrounding `the Crew of the USS {\em Enterprise}'.  It is,
rather, that such disputes are neither here nor there with respect to
everyday uses of `the Crew of the USS {\em Enterprise}'.  `The Crew of
the USS {\em Enterprise}' will continue to perform its ordinary duties
regardless of how or whether the semantic and metaphysical disputes
get settled \citeyearpar[10]{merricks2001a}.
\end{squote}

I will discuss plural referring expressions in a moment.  But first I
want to point out that the last sentence of this quote by Merricks
does not seem to be true.  Suppose I believed that the crew of the
{\em Enterprise}, were it to exist, would be a thing composed of the
crewmembers, {\em and on those grounds} I denied that the crew exists.
If the crew does not exist, if there is no crew, then that could only
mean that the ship was unmanned.  If there are crewmembers, there is a
crew.  If there is a crew, then there are crewmembers.  If Merricks or
anyone denies that there is a crew, what would it mean for them to say
that `the crew' ``will continue to perform its ordinary duties''?

\subsection{Plural referring expressions}
In the quoted material above, Merricks suggests that terms like `the
crew of the USS {\em Enterprise}' may not be singular terms, but
plural referring expressions.  For example, `the Dunns' refers
plurally to me and the other members of my family.  I say things like
``the Dunns are fine people''; the term obviously does not function as
a singular term.  The suggestion is that `the crew' behaves similarly.
When I say that the crew exists, it would then {\em not} follow that
there is a {\em thing} composed of the crewmembers.  Saying ``the crew
exists'' would instead be equivalent to saying ``the crewmembers all
exist''.

I do not think that this a plausible claim.  Recall the analogy I
tried to draw between `the crew' and `the Dunns'.  On closer
inspection, this analogy appears weak.  A stronger analogy would be
between a term like `the crew' and a term like `the Dunn family'.
`The Dunn family' is {\em not} a plural referring expression.  It is
used to refer to a {\em thing}.  If I talk about `the Dunn family', I
would say things like, ``The Dunn family is waning'', or ``The Dunn
family must regain its political power''.  The term `the Dunn family'
is a singular term that designates a thing---the family.

`The crew' appears to behave like `the Dunn family' and not like `the
Dunns'.  We say things like ``there is a skeleton crew on board'', or
``the crew is small for such a large ship''.  Were we to say things
like ``the crew are abandoning the ship'', this would most likely be a
case of non-literal speech; `the crew' is being used non-literally to
refer to the crewmembers.

The phenomenon of non-literal talk can lead us astray when thinking
about terms like `team' as well.  `The Reed College women's rugby
team' is a singular term, for it behaves in the same ways as do `the
crew' and `the Dunn family'.  We say things like ``The Reed College
women's rugby team is going to win'', or ``The Reed College women's
rugby team is in Seattle this weekend''.  However, non-literal speech
is more commonly used with teams than with terms like `family' or
`crew'.  This is often due to pluralized team names.  The Reed College
women's rugby team is called ``The Badass Sparkle Princesses''.  This
leads us to say things like ``The Badass Sparkle Princesses are on a
losing streak''.  Here we are led by the plural construction
to---perhaps unconsciously---use the term non-literally, referring not
to the team itself but to the players.  The Badass Sparkle Princesses
{\em is} a rugby team, but it is far more natural (yet not literally
true) to say that the Badass Sparkle Princesses are rugby players.

I will henceforth assume that terms like `crew', `family', and `team'
are not plural referring expressions, but rather singular terms that
designate things---crews, families, and teams.

\section{Parthood and the language of composition}
Things like teams, crews, and families are indeed {\em things}.  They
are not disguised references to plurals, nor are they ``mere
collections'' \citep[29]{inwagen2009}.  Moreover, things like teams
are things with {\em parts}.  The rugby players are each {\em part} of
the Reed College women's rugby team.  The team is made up of---it is
composed of---the players.

When I say that the players are part of the team, or that the
crewmembers are part of the crew, or that I am part of my family, is
that use of `part' the same as when I say that the tree is part of the
dogbush, or that the seat is part of the chair?  Are {\em any} of
these uses of `part' the same?

Technical notions of composition are often defined in terms of
parthood.  How I am using the term `part' will therefore influence how
I construct formal equivalences for propositions like ``there are
chairs'', ``there are dogbushes'', and ``there are teams''.  Is the
appropriate formalization for each of these the same?  Can chairs,
dogbushes, and teams each `fit in' to the schema ``there is a thing
such that it is composed of the $y$s''?  Or is `composition' one of
many {\em operations} that `produce' things?

\subsection{Van Inwagen's notion of parthood}
Van Inwagen defines his technical notion of composition (see section
\ref{scq}) in terms of a largely intuitive notion of parthood.  Van
Inwagen's interest, however, is restricted to `material' objects
(objects made exclusively of quarks and protons, or whatever the basic
atoms of the physical world turn out to be).  While he goes on to use
`part' only in reference to material objects, he recognizes that the
term has much wider application:

\begin{squote}
Parthood will occupy a central place in the present study of material
objects.  It is therefore worth noting that the word `part' is applied
to many things besides material objects.  We have already noted that
submicroscopic objects like quarks and protons are at least not clear
cases of material objects; nevertheless, every material object would
seem pretty clearly to have quarks and protons as \emph{parts}, and,
it would seem, in exactly the same sense of \emph{part} as that in
which a paradigmatic material object might have another paradigmatic
material object as a part.  A ``part,'' therefore, need not be a thing
that is clearly a material object. Moreover, the word `part' is
applied to things that are clearly \emph{not} material objects---or at
least it is on the assumption that these things really exist and that
apparent reference to them is not a mere manner of speaking.  A stanza
is a part of a poem; Botvinnik was in trouble for part of the game;
the part of the curve that lies below the x-axis contains two minima;
parts of his story are hard to believe\,\ldots\,such examples can be
multiplied indefinitely.  Does this word `part' mean the same thing
when we speak of parts of cats, parts of poems, parts of games, parts
of curves, and parts of stories \citeyearpar[18--19]{inwagen1995}?
\end{squote} 

Van Inwagen suggests that `part' does have a number of different
meanings.  Later he says that ``there is one relation called
`parthood' whose field comprises material objects\,\ldots\,another
relation called `parthood' defined on events, another still defined on
stories, yet another defined on curves, and so on''
\citeyearpar[19]{inwagen1995}.

One reason why we might resist this conclusion is that it appears to
rule out the silly example of the wish sandwich.  A wish sandwich,
recall, is the kind of sandwich where you have two slices of bread and
wish you had some meat.  The slices of bread and the wish for meat are
all parts of the wish sandwich.  But if they are parts of the sandwich
in different ways, then in what sense did I use `part' in the
preceding sentence?  If it was part$_b$---the parthood relation for
foodstuffs---then the sentence was false, because a wish does not
partake in that sort of relation.  If the relation was that of
parthood$_w$---parthood for wishes---then the sentence would again be
false, because {\em that} relation does not govern foodstuffs.  I
could instead say ``the slices of bread are part$_b$ of the sandwich
and the wish is part$_w$ of the sandwich'', but it still seems to me
that the original sentence is {\em true}.  Might this suggest that
there is really just one parthood relation that both foodstuffs and
wishes partake in?

Moreover, our modified sentence still faces a difficulty.  For the
notion of composition is generally defined in terms of parthood.
Since van Inwagen's technical definition of composition is given in
terms of his notion of parthood, `composition' for van Inwagen can
only apply to things whose parts are all material things.  Van
Inwagen's notion of composition cannot make sense of the wish
sandwich, or any thing with both material and non-material parts.
(Van Inwagen does not see this as a disadvantage; he finds mysterious
the idea that there could be something composed of ``you and I and the
number two'' \citeyearpar[20]{inwagen1995}.)

We could, perhaps, define `composition' in terms of not just one
parthood relation but all of them.  Composition would take into
account all possible ways there are of being a part.  Kit Fine has
proposed a theory of parthood that takes seriously the possibility
that there are a plurality of different parthood relations.

\subsection{Fine's theory of part}
Fine agrees with van Inwagen that the notion of parthood should not be
reserved only for material things:

\begin{squote}
Philosophers have often supposed the notion of part only has proper
application to material things or the like and that its application
to abstract objects such as sets or properties is somehow improper
and not sanctioned by ordinary use.  But I suspect that this is
something of a philosopher’s myth.  We happily talk of a sentence
being composed of words and of the words being composed of
letters---and not just the sentence and work tokens, mind, but also
the types.  And similarly, a symphony (and not just its performance)
will be composed of movements, a play of acts, a proof of steps.  I
wonder how many of these philosophers have said such things as ``this
paper is in three parts.''  When they have, then I very much doubt
that they would have any inclination, as ordinary speakers of the
language, to add ``but not, of course, in a strict or literal sense of
the term''; and the intended reference here is not primarily---or
perhaps not at all---to the tokens of the paper but to the type of
which they are the tokens.  The evidence concerning our ordinary talk
of part is mixed and complicated, but it does not seem especially to
favor taking material things to be the only true relata of the
relation \citeyearpar[561]{fine2010}.
\end{squote}

Having motivated the idea that there are things other than `material
things' that have parts, Fine gives a number of reasons to think that
there might be different senses of `part'---or as he puts it,
different ways of being a part:

\begin{squote}
Now, on the face of it, there would appear to be a wide variety of
basic ways in which one object can be a part of another.  The letter
`n' would appear to be a part of the expression `no', for example, and
a particular pint of milk part of a particular quart; and if these two
relations of part are not themselves basic (perhaps through being
restricted to expressions or quantities), there would appear to be
basic relations of part that hold between `n' and `no' or the pint and
the quart.  It is also plausible that the way in which `n' is a part
of `no' is different from the way in which the pint is a part of the
quart.  For if the two ways were the same, then how could it be that
two pints were only capable of composing a single quart, while the two
letters `n' and `o' were capable of composing two expressions, `no'
and `on' \citeyearpar[562]{fine2010}?
\end{squote}

\subsection{Deflationary metaphysics}
\label{deflate}
Kathrin Koslicki has an interesting objection to universalist theses
such as the one I appear committed to.  Her objection amounts to this:
if every `collection' of objects (such as the London Bridge, a
particle in the moon, and Cal Ripkin, Jr.) is a thing in its own
right, then metaphysics becomes uninteresting.  There is no longer any
debate about whether chairs or dogbushes are more `real' or have a
stronger claim to existence.  They both (obviously) exist, and the
difference between chairs and dogbushes is not ontological but
conceptual: `chair' is more embedded in our talk, and so chairs have
greater importance to {\em us}.  But metaphysically, or ontologically,
chairs and dogbushes are on the same level.  There is no sense in
which chairs exist and dogbushes do not.

In the quoted material below, Koslicki is criticizing a version of
four-dimensionalism that Sider has previously defended.  Sider's
position was that any collection of objects-at-times is a thing in its
own right.  Sider calls these things `fusions'.  For example, a chair
is a fusion of a large number of {\em temporal part} of things (wood
molecules, or atoms, or simples).  Each thing (wood molecule, atom, or
simple) is a fusion of {\em its} temporal parts.  Each temporal part
of the chair is also a thing (a fusion).

I take no stance on whether objects have temporal parts or rather
`endure' through time.  But Koslicki's comments are relevant
nonetheless:

\begin{squote}
There is room, in Sider's theory, for {\em some} genuine ontological
disagreements: for example, the universalist, the nihilist and the
holder of the intermediary position genuinely disagree over how many
and which fusions that exist.  But the only genuine ontological
disagreements for which there is room, in Sider's world, are ones that
concern disagreements over `bare' fusions, so to speak.  What has
happened to the houses, trees, people, and cars, the familiar concrete
objects of common-sense, whose persistence this account set out to
analyze?  There are no `deep' ontological facts as to whether a given
fusion should count as a house or not\,\ldots

[By claiming that there can be genuine ontological disputes,] Sider is
guilty of a bit of false advertising: his account is really a way of
saying that, at the end of the day, there is no interesting {\em
  ontological} story to be told about the persistence of our familiar
concrete objects of common-sense; whatever there is to say about the
persistence of houses, trees, people and cars concerns the
organization of our conceptual household
\citeyearpar[124--125]{koslicki2003}.
\end{squote}

Koslicki seems to think that we ought to be able to find some
ontological difference between ``the familiar concrete objects of
common-sense'' and things like dogbushes or chairs-at-times.  But as I
remarked above, why should what interests us (familiar objects like
chairs) be a guide to what exists?  The conclusion that ``the
persistence [and other properties] of houses, trees, people and cars
concerns the organization of our conceptual household'' seems to be a
most welcome one.

In section \ref{lessons-v} I considered Jay Rosenberg's claim that
the Special Composition Question is the wrong question to be asking.
Rosenberg's position seems to be that there is {\em no} answer to the
Special Composition Question.  Rather, he thinks what it takes to
`compose' something depends on what that something is---making a chair
is not like making a pie.

I agree that the Special Composition Question is, in a sense, the
wrong question.  But this is not because I think there is no answer to
is; rather I think there is no {\em interesting} answer.  Once we set
aside misleading terms like `composition', we see that an answer much
like universalism is obviously correct.  There may be interesting
questions in the vicinity---When are we willing to call something a
chair?  What conditions must be fulfilled?---but these are, as
Koslicki observes, not ontological questions anymore.  They are
questions about our ``conceptual household.''

\section{Lessons}
What we have learned from examining Merricks' arguments is not that
there are no chairs.  What we have learned is that the language of
`composition' can fool us into thinking that there are no dogbushes
(or bligers).  Worse, it can undermine our confidence that ``there are
chairs'' is obviously true.  It may be that talk of things `composing'
other things gives the impression that `composition' is something that
things {\em do}---as if they were gathering or attaching themselves
together.  If we avoid terms like `composition', we can see that there
is no motivation for Merricks' nihilism, other than his thesis of
causal over-determination.  And given that his conclusion---that there
are no chairs---is obviously untrue, we should suspect that causal
efficacy is not required for ``there are chairs'' to be true.  Rather,
all that is required for ``there are chairs'' to be true is that
``there are things arranged chairwise'' be true.

In this section I have tentatively supported a version of
universalism.  However, in the next section we will examine a number
of powerful objections to all versions of universalism.

\ifstandalone
\end{spacing}
\bibliography{everything}
\bibliographystyle{ChicagoReedweb}
\fi
\end{document}


\chapter{Can a chair change its parts?}
\chapterpig{Can a chair change its parts?}
\documentclass[11pt]{article}
\usepackage{standalone} \newif\ifstandlone \standalonetrue
\usepackage[left=1.75in, right=1.75in, top=1.25in, bottom=1.25in]{geometry}
\geometry{letterpaper}
\usepackage{graphicx}
\usepackage{enumitem}
\usepackage{amssymb}
\usepackage{amsmath}
\usepackage{epstopdf}
\usepackage{verbatim}
\usepackage{setspace}
\usepackage{natbib}
\setcitestyle{aysep={}}
\usepackage{url}
\usepackage{hyperref}
\synctex=1

\DeclareSymbolFont{symbolsC}{U}{txsyc}{m}{n}
\DeclareMathSymbol{\strictif}{\mathrel}{symbolsC}{74}
\DeclareMathSymbol{\boxright}{\mathrel}{symbolsC}{128}

\newenvironment{squote}{%
\begin{spacing}{1}
\begin{list}{}{%
\setlength{\labelwidth}{0pt}%
\rightmargin\leftmargin%
}
\item\relax
}{%
\end{list}%
\end{spacing}
}

\title{How do chairs persist over time?}
\author{Alexander A. Dunn}
\begin{document}
\ifstandalone
\maketitle
\begin{spacing}{1.5}
\fi

\label{parts}

In the previous section I argued that not only do `material things'
like chairs exist, but other kinds of things---like teams and
families---exist as well.  In this section I will adopt Kit Fine's
theory of parthood, which claims that there are many different ways of
being a part of something.  The legs of a chair are part of the chair
in one way, and the Supreme Court justices are parts of the Supreme
Court in another way.  Fine's theory shows that this is because a
chair and the Supreme Court are different kinds of things.

Fine's theory by itself does not tell us what kind of thing a chair
is, or what kind of thing the Supreme Court is.  A chair might be a
{\em mereological sum}, or it might be something else.  The Supreme
Court might be a {\em set}, or it might be something else.  Fine's
theory does not tell us what these things are, but it does give us the
resources to start thinking up possible answers.

I will argue in this section that there are only two possible
answers.  If chairs are {\em not} identical with mereological sums,
then the things that are parts of a chair (in one way) will be the
same things that are part of a mereological sum (in another way).  In
other words, we will have {\em co-located objects}.  And once we have
two objects located in the same place, there will be pressure to
recognize more.  We may find ourselves with a {\em plurality} of
co-located objects.  (There is an analogous problem if the Supreme
Court is not identical with a set.  If there is some other thing that
has the same parts (but has them in a different way), we will find
that it is possible that there be a large number of different kinds of
things, all with the same parts.)

The second answer is to affirm that chairs are identical with
mereological sums.  This will allow us to deny that there are a
plurality of co-located objects where my chair is; we will be able to
say that there is just the one thing---the chair.  But this answer has
unintuitive consequences as well.  It is generally thought that
mereological sums cannot change their parts, but it is generally
thought that chairs can change their parts.  If we defend this second
answer, then, we will have to interpret talk about `the same chair'
over time as being talk about {\em different} sums over time.  The
appearance of persistence over time (and through change) is a product
of {\em conventions} that govern terms like `chair'.  What
mereological sum is the referent of `the chair' from one day to the
next depends not upon the actual identity of the two referents, but
upon convention.

\section{Parthood and the language of composition}
\label{parthood}
I argued in section \ref{plural-ref} that things like teams, crews,
and families are indeed {\em things}.  Terms like `team', `crew', and
`family' are not disguised references to plurals.  Moreover, things
like teams are things with {\em parts}.  The rugby players are each
{\em part} of the Reed College women's rugby team.  The team is made
up of---it is composed of---the players.

When I say that the players are part of the team, or that the
crewmembers are part of the crew, or that I am part of my family, is
that use of `part' the same as when I say that the tree is part of the
dogbush, or that the seat is part of the chair?  Are {\em any} of
these uses of `part' the same?

Technical notions of composition are often defined in terms of
parthood.  How I am using the term `part' will therefore influence how
I construct formal equivalences for propositions like ``there are
chairs'', ``there are dogbushes'', and ``there are teams''.  Is the
appropriate formalization for each of these the same?  Can chairs,
dogbushes, and teams each `fit in' to the schema ``there is an $x$
such that it is composed of the $y$s''?  Or is `composition' one of
many {\em operations} that `produce' things?

\subsection{Van Inwagen's notion of parthood}
\label{van-part}
Van Inwagen defines his technical notion of composition (see section
\ref{scq}) in terms of a largely intuitive notion of parthood.  Van
Inwagen's interest, however, is restricted to `material' objects
(objects made exclusively of quarks and protons, or whatever the basic
atoms of the physical world turn out to be).  While he goes on to use
`part' only in reference to material objects, he recognizes that the
term has much wider application:

\begin{squote}
Parthood will occupy a central place in the present study of material
objects.  It is therefore worth noting that the word `part' is applied
to many things besides material objects.  We have already noted that
submicroscopic objects like quarks and protons are at least not clear
cases of material objects; nevertheless, every material object would
seem pretty clearly to have quarks and protons as \emph{parts}, and,
it would seem, in exactly the same sense of \emph{part} as that in
which a paradigmatic material object might have another paradigmatic
material object as a part.  A ``part,'' therefore, need not be a thing
that is clearly a material object.  Moreover, the word `part' is
applied to things that are clearly \emph{not} material objects---or at
least it is on the assumption that these things really exist and that
apparent reference to them is not a mere manner of speaking.  A stanza
is a part of a poem; Botvinnik was in trouble for part of the game;
the part of the curve that lies below the x-axis contains two minima;
parts of his story are hard to believe\,\ldots\,such examples can be
multiplied indefinitely.  Does this word `part' mean the same thing
when we speak of parts of cats, parts of poems, parts of games, parts
of curves, and parts of stories \citeyearpar[18--19]{inwagen1995}?
\end{squote} 

Van Inwagen suggests that `part' does have a number of different
meanings.  Later he says that ``there is one relation called
`parthood' whose field comprises material objects\,\ldots\,another
relation called `parthood' defined on events, another still defined on
stories, yet another defined on curves, and so on''
\citeyearpar[19]{inwagen1995}.

This may very well be, but what is the similarity between these
relations?  The parthood relation in classical mereology is
well-defined, as is the membership relation in set theory.  But there
is no equally well-defined relation for event parthood, or poem parthood.

However, Kit Fine has proposed a theory of parthood that takes
seriously the possibility that there are a plurality of different
parthood relations, and has the resources to define numerous parthood
relations.  The parthood relation for poems may be given as rigorous a
treatment as the parthood relation for sets.

\section{Fine's theory of part}
\label{fine}
Fine agrees with van Inwagen that the notion of parthood should not be
reserved only for material things:

\begin{squote}
Philosophers have often supposed the notion of part only has proper
application to material things or the like and that its application
to abstract objects such as sets or properties is somehow improper
and not sanctioned by ordinary use.  But I suspect that this is
something of a philosopher’s myth.  We happily talk of a sentence
being composed of words and of the words being composed of
letters---and not just the sentence and work tokens, mind, but also
the types.  And similarly, a symphony (and not just its performance)
will be composed of movements, a play of acts, a proof of steps.  I
wonder how many of these philosophers have said such things as ``this
paper is in three parts.''  When they have, then I very much doubt
that they would have any inclination, as ordinary speakers of the
language, to add ``but not, of course, in a strict or literal sense of
the term''; and the intended reference here is not primarily---or
perhaps not at all---to the tokens of the paper but to the type of
which they are the tokens.  The evidence concerning our ordinary talk
of part is mixed and complicated, but it does not seem especially to
favor taking material things to be the only true relata of the
relation \citeyearpar[561]{fine2010}.
\end{squote}

But even if one accepts the idea that there are things other than
`material things' that have parts, one might object that they are
still all parts in the same sense.  This might be the parthood
relation of classical mereology, or it might be some other, general
relation.  Moreover, one who maintained the univocality of parthood
can still concede that there are different ways of being a part.  But
for the believer in the univocality of parthood---the `monist'---these
are only {\em derivative} kinds of parthood.  For example, for any
given mereological sum, there are bigger and smaller parts of it. But
these are bigger and smaller parts of the same {\em kind}.  The
pluralist goes further and claims that there are parts of different
kinds. These different kinds are not derivative but \emph{basic}; they
are ``not definable in terms of other ways of being a part''
\citep[561]{fine2010}.  Fine gives a number of reasons to think that
there might be different basic parthood relations:

\begin{squote}
Now, on the face of it, there would appear to be a wide variety of
basic ways in which one object can be a part of another.  The letter
`n' would appear to be a part of the expression `no', for example, and
a particular pint of milk part of a particular quart; and if these two
relations of part are not themselves basic (perhaps through being
restricted to expressions or quantities), there would appear to be
basic relations of part that hold between `n' and `no' or the pint and
the quart.  It is also plausible that the way in which `n' is a part
of `no' is different from the way in which the pint is a part of the
quart.  For if the two ways were the same, then how could it be that
two pints were only capable of composing a single quart, while the two
letters `n' and `o' were capable of composing two expressions, `no'
and `on' \citeyearpar[562]{fine2010}?
\end{squote}

The parthood relation for sets is again different.  The set containing
the only the letters `n' and `o' has the letters as parts.  When the
letters are parts of a set, their order is irrelevant, but when the
letters are parts of a word, order matters; hence `no' and `on'.  The
parthood relation for sets is also different from the parthood
relation for quantities (of milk):

\begin{squote}
If four quarts compose a gallon the pints which compose the quarts
will compose the gallon in the same way in which they compose the
quarts, whereas, if four sets compose a further set the members of the
sets will not compose the further set in the same way in which they
compose the component sets.  Thus we would now appear to have three
different basic ways in which one object can be a part of another
(pint/gallon, letter/word, and member/set); and once these cases have
been granted, it is plausible that there will be many more
\citeyearpar[562]{fine2010}.
\end{squote}

One might, of course, refuse to grant these cases.  But one would have
to refuse them {\em all}; for if it can be established that there are
even two different (basic) ways of being a part, then the pluralist
position is established.  Once it is established that there are at
least two ways of being a part, it becomes much more plausible that
there might be three ways, or more.  Fine therefore attempts to
motivate the idea that the members of a set are, quite literally,
parts of the set.

\subsection{Parts of sets}
\label{sets}
The first objection is that while parthood is supposed to be
transitive, the membership relation of sets is not.  The letter `n' is
a member of the set \{`n',\{`n',`o'\}\}, but `o' is not.  The
objection claims that sets have {\em members}, not parts, and that
Fine has confused the two.

But while it is true that the membership relation is not the parthood
relation, this is no reason to think that sets do not have parts.  A
given set will have certain members---the $x$s---and certain
parts---the $y$s---and only sometimes will the $x$s and the $y$s be
the very same things.  The set \{`n',\{`n',`o'\}\} has two members
but three parts.  The parthood relation for sets can even be defined
in set-theoretic terms:

\begin{squote}
It may well be thought that the way in which a member is a part of a
set is given, not by the membership relation itself, but by the
ancestral of the membership relation, where this is the relation that
holds between $x$ and $y$ when $x$ is a member of $y$ or a member of a
member of $y$ or a member of a member of a member of $y$, and so on
\citep[563]{fine2010}.
\end{squote}

A second objection is that talk of parthood in connection with things
like sets is somehow metaphorical or non-literal.  We saw above that
van Inwagen admits that many different things are said to have parts.
However, he qualifies this in two ways.  First, he seems to have
doubts (or at least is sympathetic with those who have doubts) as to
whether the non-material things that are said to have parts really
exist:

\begin{squote}
The word `part' is applied to things that are clearly \emph{not}
material objects---or at least it is on the assumption that these
things really exist and that apparent reference to them is not a mere
manner of speaking \citep[19]{inwagen1995}.
\end{squote}

If there are no such things as tennis matches or poems or papers, then
of course they do not have parts.  But I think it is obviously true
that there are such things.  This being so, what does it mean to say
that they have parts?  This is where van Inwagen's second
qualification comes in.  For he suggests not only that the `parts' of
tennis matches and poems are parts in a different way than are the
parts of a table, but that these different relations of parthood are
only tenuously connected.  Van Inwagen says that the various relations
of parthood (if such there be) are connected only by the ``unity of
analogy'' \citeyearpar[19]{inwagen1995}.  If the only similarity
between the parthood relation for poems and the parthood relation for
chairs is that they share the `analogy' of parthood, then is there
anything important or interesting about `parts' of poems?  Is the
parthood relation for sets likewise only interesting because of the
analogy with the parthood relation for chairs?

At least in the case of parthood for sets, the notion does not appear
to be wholly metaphorical:

\begin{squote}
In the case of set-membership, there would appear to be nothing that
might plausibly be taken to indicate that the talk of part-whole is
not to be taken literally. A set is indeed composed of or built up
from its members, and we should add that we may meaningfully
talk---and in the intended way---of \emph{replacing} one member of a
set with another.  Thus Aristotle in the set \{Plato, Aristotle\} may
be replaced with Socrates to obtain the set \{Plato, Socrates\}, with
the given set becoming a different set from what it was. In the case
of sets, our conception of members as parts seems to extend all the
way \citep[564]{fine2010}.
\end{squote}

But the second worry raised by van Inwagen remains.  Why should we
think that there is any {\em real} similarity between these different
parthood relations, other than the fact that we call them all
`parthood'?

\subsection{Operationalism}
\label{operation}
Fine's theory of {\em operationalism} helps answer this worry.
Various {\em operations} produce different things---mereological
summation produces mereological sums or fusions, the set-builder
produces sets, and so forth.  Parts are therefore {\em things} that
have been `combined', through one or more such operations, into a
single {\em thing}.  What is common to all parthood relations is that
from each set of parts is produced a {\em whole} by means of a
composition operator.  From parts (letters, atoms) are made something
else (a word, a set, a chair).  What ties together all the ways of
being a part is that they are involved in a composition operation that
produces a single thing from a number of things:

\begin{squote}
In formulating the principles of mereology, it has been usual to take
the relation of part-whole or some associated relation (such as
overlap) as primitive.  But I believe that, in formulating a more
general theory, it is important to take the operation of composition
as primitive rather than the more familiar relation of part-whole.  In
the case of classical mereology, the operation of composition will
take some objects into the sum or fusion of those objects, while, in
the set-theoretic case, it will take some objects into the set of
those objects; and, in general, the operation of composition will be
the characteristic means (summation, set-builder, and so on) by which
a given kind of whole is formed from its parts \citep[565]{fine2010}.
\end{squote}

Each way of being a part can then be defined in terms of the related
composition operation:

\begin{squote}
Once given a compositional operation, a corresponding relation of part
may be defined in two steps.  We say first that $x$ is a component of
$y$ if $y$ is the result of applying $\sum$ to $x$ or to $x$ and some
other objects.  In other words, $y$ should be of the form $\sum
(x_{1}, x_{2}, \mathellipsis )$, where at least one of $x_1$, $x_2,
\mathellipsis$ is $x$.  Thus when $\sum$ is mereological summation the
components of an object will be mere parts, and where $\sum$ is the
set-builder the components of an object will be its members.  We may
then define $x$ to be a part of $y$ if there is a sequence of objects
$x_1$, $x_2, \mathellipsis x_n$, $n$ \textgreater{} $0$, for which $x
= x_1$, $y = x_n$, and $x_i$ is a component of $x_{i+1}$ for $i = 1$,
$2, \mathellipsis, n-1$. The parts of an object are the object itself,
or its components, or the components of the components, and so on
\citep[567--568]{fine2010}.
\end{squote}

The parthood relation for mereological sums can therefore be seen to
exhibit reflexivity, transitivity and anti-symmetry:

\begin{description}
\item[Reflexivity] Each object is a part of itself.
\item[Transitivity] If $x$ is a part of $y$ and $y$ of $z$, then $x$
  is a part of $z$.
\item[Anti-symmetry] $x$ is a part of $y$ and $y$ of $x$ only when $x
  = y$ \citep[568]{fine2010}.
\end{description}

But not all definitions of parthood that issue from a composition
operator will exhibit these features:

\begin{squote}
When the underlying operation is summation, each object will be a part
of itself, since the unit sum of any object is the object itself, but
when the underlying operation is the set-builder, no object will be a
part of itself, since no object is ever an ancestral member of itself
\citep[569]{fine2010}.
\end{squote}

\subsection{Principles}
\label{principle}
Each composition operation will, according to Fine, be governed by
various principles:

\begin{squote}
I believe that the principles governing the basic forms of composition
will conform to a general template.  Variations in the principles for
the different forms of composition will then arise from variations in
how the template is to be filled in.  The template will comprise two
broad categories of principle---the {\em formal} and the {\em
  material} (though not quite in the sense of Husserl).  Among the
formal principles, we may distinguish between those that provide
conditions of application for the operation and those that provide
identity conditions; among the material principles, we may distinguish
between those that provide conditions for the presence of a whole (in
space and time or at a world) and those that specify the descriptive
character of the whole.  The presence conditions, in their turn, may
concern either the existence of the whole or its extension
\citeyearpar[569--570]{fine2010}.
\end{squote}

\paragraph{Formal principles}
\label{formal}
The formal principles govern when composition occurs and when two
products of a composition operation are identical:

\begin{description}
  \item[Application] The application conditions are ``the conditions
    under which there are wholes of a given sort---which, on the
    operational approach, is a matter of stating when the result of
    applying the compositional operation to various objects will be
    defined'' \citep[570]{fine2010}.  For the summation operation, the
    application conditions are very lax: for any $n$ physical objects
    (where $n$ \textgreater{} 1), there is a thing composed of those
    objects.  The application conditions for the set-builder will be
    more limited; and other operations will be more restricted
    still. \label{fine-app}
  \item[Identity] Setting out identity conditions on Fine's
    operational approach ``is a matter of stating when a whole formed
    in one way by means of the compositional operation is the same as
    a given object or a whole that has been formed in some other way''
    \citeyearpar[570]{fine2010}.  For example, suppose $y$ is the
    result of applying the summation operator to $x$, and suppose
    $y^\prime$ is the result of applying the set-builder to $x$.  If
    $y \neq y^\prime$, it will be because of some difference in the
    respective operations that produced these composites.  As we will
    see below, the summation operator is defined such that its
    application to a single thing ($x$) produces that very thing.  The
    set-builder, however, produces the singleton of $x$, which is not
    $x$.  And whereas applying the set-builder to nothing produces the
    null set, the result of applying the summation operator to nothing
    might be undefined---nothing would result.
\end{description}

\paragraph{Material principles}
\label{material}
As above, there are two subcategories:

\begin{description}
  \item[Presence] Fine claims that ``there are two fundamentally
    different ways in which an object might be present in space or
    time; it may \emph{exist} in space or time, or it may be
    \emph{extended} (or \emph{located}) in space or time. Thus a
    material thing will exist in time but be extended in space while
    an event will be extended in both space and time''
    \citeyearpar[570]{fine2010}.  Whether or not this is actually true
    seems to depend on whether a `three-dimensional' theory of
    persistence is correct.  However, that question will remain unasked.
  \item[Character] ``The character conditions will tend to have a much
    more ad hoc character than the other conditions that we have
    considered. The color of a house, for example, is the color of its
    siding; the color of an egg, the color of its shell; the color of
    a pencil, the color of its lead. In the case of the `intrinsic'
    character of a thing---such as its mass or color---the character
    of the whole will be some sort of function of the character of the
    parts. But the function in question will vary from case to case''
    \citep[571]{fine2010}.  What we decide is the color of a house
    will turn on our concept `house', rather than anything about the
    house itself (anything beyond its having that color to {\em some}
    extent).
\end{description}

\subsection{Fine's pluralist account of classical mereology}
\label{classical}
Of the principles sketched above, Fine gives most attention to the
identity conditions for composition operations.  The composition
operation used as a paradigm is the summation operation of classical
mereology.  Fine's exposition of identity conditions for sums relies
on the notion of `regularity':

\begin{squote}
Call an identity condition $s = t$ {\em regular} if the variables
appearing in $s$ and in $t$ are the same.  Thus $\sum (x, y) = \sum
(y, x)$ is regular while $\sum (x, y) = x$ is not
\citeyearpar[572]{fine2010}.
\end{squote}

With this notion in hand, Fine proposes this condition for identity of
sums:

\begin{description}
  \item[Summative Identity] $s = t$ whenever `$s = t$' is a regular
    identity \citeyearpar[572]{fine2010}.
\end{description}

One particularly interesting aspect of this condition is that it
entails four more principles of the summation operation:

\begin{description}
  \item[Absorption] $\sum (\mathellipsis, x, x, \mathellipsis,
    \mathellipsis, y, y, \mathellipsis, \mathellipsis = \sum (
    \mathellipsis, x, \mathellipsis, y, \mathellipsis )$;
\item[Collapse] $\sum (x) = x$;
\item[Leveling] $\sum (\mathellipsis, \sum (x, y, z, \mathellipsis ),
  \mathellipsis, \sum (u, v, w, \mathellipsis ), \mathellipsis ) \\ =
  \sum (\mathellipsis, x, y, z, \mathellipsis, \mathellipsis, u, v, w,
  \mathellipsis, \mathellipsis )$;
\item[Permutation] $\sum (x, y, z, \mathellipsis ) = \sum (y, z, x,
  \mathellipsis )$ (and similarly for all other permutations)
  \citep[573]{fine2010}.
\end{description}

We can define other compositional identity criteria (e.g., sequences)
in terms of which of these principles apply to their compositional
operation.  But we may also devise new principles by which we may then
define new types of composition:

\begin{squote}
We should note that there would appear to be no good reason to require
that the defining principles for the various operations should be
limited to the particular principles (C [collapse], L [leveling], A
[absorption], and P [permutation]) that we used in characterizing
sums; for any set of regular identities would appear to be equally
well suited to defining a basic form of composition, so long as they
conform to Anti-cyclicity.  Indeed, I would conjecture that any such
set of principles in fact will correspond to a form of composition and
a corresponding form of whole.  How the resulting forms of composition
and whole might be organized is an interesting question, but it should
be apparent that the approach will lead to an infinitude of forms of
composition, each differing from one another in how exactly the
identity of the resulting wholes is to be
determined. \citep[575--576]{fine2010}.
\end{squote}

It is at this point that the importance of Fine's theory becomes
obvious.  Above I stressed that things like teams and families are
really {\em things}; moreover I made this claim as part of an attempt
to motivate a sort of universalistic outlook on metaphysics.  I argued
that the term `composition' was potentially misleading, but that it
was nevertheless correct to say that things like dogbushes, wish
sandwiches, and teams are composed of their parts.  But now it is
apparent that `composition' will mean something different when applied
to each of these things.  Each thing will be the product of a
different composition operation.

Fine's theory reveals new {\em kinds} of universalism.  One might be
committed to the existence of dogbushes---and so to unrestricted
mereological composition---but deny the existence of teams, groups,
crews, and families.  Or one might defend unrestricted composition of
groups while claiming a restriction on mereological composition.

Below I will look at how a definition of the composition operator for
groups might be formulated.  But I will first return to Fine's theory.

\subsection{Hybrid parts}
\label{hybrid}
Above, we saw that one objection to the idea of sets having parts was
that parthood is transitive and set-membership is not.  Moreover it
was supposed (rather plausibly) that the only reason we think that
sets have parts is {\em because} they have members; it is the members
of a set that we are tempted to call parts.  But, the objection goes,
it is a mistake to think of members as part.  I am the only member of
my singleton (the singleton of $x$ is the set resulting from applying
the set-builder to $x$ alone).  My hand, for instance, is not a member
of my singleton.  But my hand is a part of me.  If I was a part of my
singleton, then---because parthood is transitive---my hand would be a
part of my singleton.  And if that means that my hand is a {\em
  member} of my singleton, that is clearly wrong.

Fine points out, of course, that the objection makes the mistake of
supposing that something (me, my hand) can be a part in only one way
(in this case, through set-membership).  Once we recognize that there
are a plurality of ways of being a part, it becomes clear that my hand
is part of the set in one way, but not in another:

\begin{squote}
Given the specific relations of part, we may derive various {\em
  hybrid} relations of part.  Suppose, for example, that we are given
the relations of set-theoretic and mereological part---which we may
designate as \textepsilon -part and $m$-part. We may then take one
object to be an \textepsilon ,$m$-part of another if it is an
\textepsilon -part or an $m$-part or an $m$-part of an \textepsilon
-part or an \textepsilon -part of an $m$-part, or an $m$-part of an
\textepsilon -part of an $m$-part, and so on. More generally, if $K$
is a family of specific ways of being a part, we may take an object to
be a {\em K-part} of another if $x$ and $y$ can be linked by
relationships of $k$-part for $k$ in $K$ \citep[579]{fine2010}.
\end{squote}

My hand is a \textepsilon ,$m$-part of my singleton, but not a
\textepsilon -part.

By conjoining every way of being a part, we arrive at the most general
notion of part:

\begin{squote}
Among the hybrid relations of part, of special interest is the
relation of $K$-part where $K$ is the family of {\em all} the specific
ways of being a part.  This is the relation of $K$-part that holds
between two objects when they may be linked by relationships of
$k$-part without restriction on $k$.  We might call it the {\em
  general} relation of part, and it is a relation that holds between
$x$ and $y$ whenever $x$ is in any way whatever a part of $y$
\citep[580]{fine2010}.
\end{squote}

\subsection{Generating kinds}
\label{generate}
On this theory, what kind a thing is depends on what operation
produced it.  If a chair or a dogbush is a mereological sum, then this
is because they are produced by the summation operation.  The Dunn
family is `produced' by the family operation.  Groups are produced by
the group operation (see section \ref{group}).

But there is a difficulty to be avoided here.  As we saw in section
\ref{classical}, the mereological sum of a single thing $x$ is just
$x$.  Therefore there is a sense in which every physical thing,
including every simple, is a mereological sum, for the application of
the summation operation would just produce that thing.  To avoid this
consequence Fine introduces the notion of a {\em generative}
application of an operation:

\begin{squote}
We might say that the application $y = \Gamma (x_1, x_2, x_3,
\mathellipsis )$ of an operation $\Gamma$ is {\em generative} if there
is an explanation of the identity of $y$ as $\Gamma (x_1, x_2, x_3,
\mathellipsis )$; and we might say that the operation $\Gamma$ is
itself {\em generative} if it permits a generative application. Thus
both the set-builder and the operation of predication will be
generative in this sense \citeyearpar[582]{fine2010}.
\end{squote}

Whether or not the summation operation is generative depends on the
things it is being applied to.  When summing a dog and a tree, it is
generative; when summing a dog by itself, it is not.

For any operation, there will be things it applies to that it cannot
produce.  The summation operator fuses simples, but cannot produce
them; the set-builder combines many things that it cannot produce
(like letters).  For any given operation, there is a `level 0'
consisting of the things that the operator itself cannot produce:

\begin{squote}
We suppose that certain objects are simply given.  These are the
objects whose identity does not require an explanation in terms of
$\Gamma$.  Thus, when $\Gamma$ is the set-builder, they are the
objects that are not sets and, when $\Gamma$ is summation, they are
the objects that are not sums or, rather, the objects that do not need
to be seen as sums.

We now `generate' objects in stages.  At stage 0 are the givens; at
stage 1, we add the objects that result from a single application of
the generative operation $\Gamma$ to the givens \citep[583]{fine2010}.
\end{squote}

An application can now be identified as generative in a strong or a
weak sense:

\begin{description}
  \item[Strong generative application] Also called `strict' by Fine, a
    ``[strong] generative application of $\Gamma$ to the objects $x_1,
    x_2, \mathellipsis$ can now be defined as one in which $y = \Gamma
    (x_1, x_2, \mathellipsis )$ is of a higher level than each of
    $x_1, x_2, \mathellipsis$'' \citeyearpar[584]{fine2010}.  For
    example, summing the simples $x$ and $y$ to produce the fusion $z$
    would be a strong generative application of the summation
    operator; the simples are level 0 and $z$ is level 1.  Summing two
    composites, or a composite and a simple, would not be strongly
    generative; one or both of the parts would be the same level (1)
    as the product.
  \item[Weak generative application] To illuminate this notion Fine
    introduces another, that of a {\em putative generative
      application}: ``Let us say, in the first place, that $y = \Gamma
    (x_1, x_2, \mathellipsis )$ is a putative generative application
    of $\Gamma$ if $y$ is of a higher or of the same level as each of
    $x_1, x_2, \mathellipsis$.  This gives us the notions of a
    putative prior component and of a putative prior in the usual way.
    We now say that the application $y = \Gamma (x_1, x_2,
    \mathellipsis )$ of $\Gamma$ is a {\em weak} generative
    application if it is the putative generative application and if
    $y$ is not putatively prior to any of $x1, x2, \mathellipsis$.  We
    can get from $x_1, x_2, \mathellipsis$ to $y$ without an ascent in
    level but not from $y$ to any of $x_1, x_2, \mathellipsis$''
    \citeyearpar[584]{fine2010}.
\end{description}

Applying the summation operator to a simple is neither strongly nor
weakly generative.  It is not strongly generative because the result
is a simple, which is at level 0---the same level as its part
(itself).  It is not weakly generative because the result of the
operation is putatively prior to its parts.

\section{Groups and sets}
\label{group}
Fine's theory gives us the resources to define new composition
operators, and so new {\em kinds} of things.  But there is no point in
defining a composition operator for a kind of thing that does appear
in the world.  For example, suppose that unicorns are a special kind
of thing.  We could, presumably, define their existence conditions and
define a `unicorn-builder'---a composition operator for unicorns.  But
there are no unicorns, so why should we.  If we are to define a kind
of thing, we should first determine that there are such things.  One
kind of thing that does seem to have instances, unlike {\em unicorn},
is {\em group}.

Why should we think that there are groups, in addition to sets?  One
important reason is the apparent fact that groups can change their
parts, while sets cannot.  If groups are distinct from sets, we should
be able to define a group-builder that is distinct from a set-builder.
However, one new composition operator---the `group-builder'---will not
be enough.  We will see that, if we wish to introduce groups, we will
have to introduce many {\em kinds} of groups.  Fine's theory of parts
allows us to do this, but it is not clear that we ought to.  For it
may be that every group will be itself a {\em unique} kind of thing.

If we wish to avoid this conclusion, we may have to reject parts of
Fine's theory and place a limit on the kinds of things there are.  But
there are costs on both approaches.

\subsection{Motivating groups}
The set-builder, on Fine's theory, takes things (such as jurists) and
produces a set composed of them.  There is a set $S$ composed of the
2004 Supreme Court justices:

\begin{squote}
$S = \sum _{\in}$ (Rehnquist, Stevens, O'Connor, Scalia, Kennedy,
  Souter, Thomas, \\ Ginsburg, Breyer) $ = $ \{Rehnquist, Stevens,
  O'Connor, Scalia, Kennedy, \\ Souter, Thomas, Ginsburg, Breyer\}
\end{squote}

But some claim that, in addition to sets, there are also {\em groups}.
There may, in addition to the set \{Rehnquist, Stevens, O'Connor,
Scalia, Kennedy, Souter, Thomas, Ginsburg, Breyer\}, a group
containing the very same people.

One might ask why this is necessary.  Groups, it might be objected,
are really no different than sets.  When we speak of a group of
people, we are actually referring to the set of which they are
members.

But there are some reasons why it seems incorrect to identify groups
with sets.  Take the Supreme Court.  It seems that any attempt to
identify the Supreme Court with the set of the Supreme Court justices
will not succeed.  This is because the membership of the Supreme Court
changes over time, while the members of a set do not.  The set
containing the 1990 justices is a {\em different} set from the set
containing the 2012 justices, but the 2012 Supreme Court is not a
different entity than the 1990 Court.  (We may of course say things
like ``it's a different court now'', but by that we mean only that it
is composed of different people, and so may rule differently---note
that we do {\em not} say ``it's a different Court now''.)

If one grants that groups such as the Supreme Court are not sets, it
may be objected that they are therefore simply mereological sums.  But

\begin{squote}
membership in the Supreme Court is very different from
the part-whole relation on material objects.  The part-whole relation
on material objects is a transitive relation.  Thus if one identified
the Supreme Court with a material object and Justice Breyer with a
part of it, then one would be forced to conclude that Justice Breyer's
arm must be a part of the Supreme Court as well.  Yet, it is plain
that Justice Breyer's arm is neither a part nor a member of the
Supreme Court \citep[136--137]{uzquiano2004a}.
\end{squote}

If the Supreme Court were a mereological sum, it would behave very
strangely.  What its parts would be on a given occasion would depend
on the appointment decisions of the President.  (If we accept a
`four-dimensional' version of universalism, then objects have {\em
  temporal} as well as spatial parts.  There would then be a
mereological sum of the parts of the justices that existed during
their appointments.  This would be an object whose existence would not
depend on the President.)

There is at least some motivation to posit a new {\em kind} of thing
that is not a set or a sum.  This new kind is the group.
Unfortunately, once we try to spell out the identity conditions of
groups, we will see that groups are not one kind of thing: there are
many different kinds of things (possibly an infinite number of kinds
of things) that we lump under the term `group'.  

%% If we are to maintain that there are groups (that are not sets), we
%% will be committed to there being a plurality of kinds of things,
%% united under a family resemblance.

\subsection{The place of the group in Fine's template}
\label{group-temp}
In section \ref{classical} above, Kit Fine showed how the summation
operator relates to four different properties: Collapse, Leveling,
Absorption, and Permutation.

\begin{description}
  \item[Absorption] $\sum (\mathellipsis, x, x, \mathellipsis,
    \mathellipsis, y, y, \mathellipsis, \mathellipsis = \sum (
    \mathellipsis, x, \mathellipsis, y, \mathellipsis )$;
\item[Collapse] $\sum (x) = x$;
\item[Leveling] $\sum (\mathellipsis, \sum (x, y, z, \mathellipsis ),
  \mathellipsis, \sum (u, v, w, \mathellipsis ), \mathellipsis ) \\ =
  \sum (\mathellipsis, x, y, z, \mathellipsis, \mathellipsis, u, v, w,
  \mathellipsis, \mathellipsis )$;
\item[Permutation] $\sum (x, y, z, \mathellipsis ) = \sum (y, z, x,
  \mathellipsis )$ (and similarly for all other permutations)
  \citep[573]{fine2010}.
\end{description}

Sums have all four properties, while sets have only Permutation and
Absorption.  We can begin to define our group operator by thinking
about which of these properties it has.

I tentatively suggest that groups mimic sets with regard to these
four properties.  Groups possess Absorption due to the fact that one
cannot be twice a member of the same group.  Groups possess
Permutation, since there is no `order' with regard to membership of a
group (there may be {\em temporal} order---I joined the group
first!---but that is not the same thing).  Groups do not possess
Collapse, since (I am inclined to think) a group can have a single
member without thereby {\em being} that member.  If, for example, a
task force is created and only one individual assigned to it, the
findings of the task force will be of the {\em task force} and not of
the individual.  Groups do not possess Leveling either, since there
can be groups made up of groups.

%% So far we have seen that groups are quite similar to sets.  But there
%% are some differences.  As we have seen, groups can change their
%% membership, while sets cannot.  Additionally, I suggest that the
%% application conditions (see section \ref{fine-app}) for groups is
%% different from that of sets.  While sets can have more or less
%% anything as members (letters, people, other sets), I propose that {\em
%%   groups may be composed only of living things and other groups}.

%% This claim is made on intuitive grounds.  It simply seems odd to talk
%% about a group of rocks, or a group of sets.  Talk of groups implies
%% some sort of activity, and so it is more natural to use `group' to
%% refer to people and other animals.  Even a `group' of trees is not
%% wholly bizarre.  And a group---for example, the Special Committee on
%% Judicial Ethics---might be part of another group---the Committee of
%% Ethics Committees \citep[145]{uzquiano2004a}.

%% Another point on application conditions: I think that group
%% composition is {\em unrestricted} among living things and groups.
%% That is, for any set of living things and/or groups, there is a
%% group of them.  Any restriction seems arbitrary.

\subsection{How do groups change their members?}
\label{group-time}
The most important apparent difference between sets and groups is that
while the members of a set are necessarily so, the members of a group
may change over time.  How does this work?

We could think of the group composition operator (the group-builder)
as operating {\em not} on things (living things and groups) but on
things-at-times.  The group-builder for the Supreme Court takes the
various justices during the times of their service and produces the
group---the Supreme Court---from those people-at-times.

One problem with this proposal is that it appears to presuppose {\em
  temporal parts}.  For if the group-builder works in similar fashion
to the set-builder and summation operator, then it operates on {\em
  things}.  If our group-builder is going to operate on
people-at-times, then we seem to commit ourselves to the claim that
people-at-times are {\em things}.  And what things could they be but
temporal parts of other things?

Thinking of groups as being composed of things-at-times rather than
things is unintuitive in any case.  Sandra Day O'Connor \emph{was} a
member of the Supreme Court.  Taking temporal parts seriously would
require us saying instead that her 1981--2006 part \emph{is} a member
of the Supreme Court.  But Sandra Day O'Connor is no longer a member
{\em at all}.

If we don't want to presuppose temporal parts, the group operator has
to be somehow \emph{dynamic}. It can't just take things, compose them
and be done---it has to \emph{add and remove things over time}.

Making sense of a dynamic group operator might allow us to avoid
presupposing {\em eternalism} as well (ultimately it will not).  If
the group-builder made the Supreme Court `in one go', then future
justices would have to already exist in some sense.  How else could
the group-builder operate on them?

The most readily apparent way of making sense of a dynamic operator is
by relativizing the group-builder to times.  We can think of the
operator as taking a set at a time and producing a group: $G = \sum
_{t} (S)$.  (There are two interpretations of $\sum _{t}$: we might
say that the composition operator as (re-)producing a group at a
number of different times $t$, or we might say that there is a {\em
  different} composition operator at each time $t$.  I am not sure how
much depends on a choice here, but I will suppose that the former is
correct.)

Following \citet{uzquiano2004a}, we can say that set $S$
composes group $G$ at time $t$ if and only if:

\begin{enumerate}[label=(\arabic*)]
  \item $\forall x\ (x \in S \leftrightarrow x\ \text{is a member of}\ G
  \  @\ t)$
  \item $\exists x {[} x\ \text{is a member
      of}\ G\ @\ t\ \wedge\ \square ( x \in S )\ \wedge
    \\ \diamond\ \exists t^{\prime} ( G\ \text{exists}\ @
    \ t^{\prime}\ \wedge\ \neg ( x\ \text{is a member
      of}\ G\ @\ t^{\prime} )) {]}\ \vee \\ \exists x^{\prime} {[}
    \neg ( x^{\prime}\ \text{is a member of}\ G\ @\ t)\ \wedge \\ \neg
    \diamond (x^{\prime} \in S ) \wedge \ \diamond \exists t^{\prime
      \prime} (x^{\prime} \text{is a member of}\ G\ @\ t^{\prime
      \prime}) {]}$\ \citeyearpar[150]{uzquiano2004a}
\end{enumerate}

I will assume that group composition is {\em unrestricted}.  That is,
for any people and/or groups at any time $t$, there is a group
composed of them at that time.

\subsection{Identity conditions}
\label{group-id}
Given that groups can change their parts over time, there will be
cases in which $G = \sum _{t_1} ( S )$ and $G^{\prime} = \sum _{t_2}
( S^{\prime} )$ and $G = G^{\prime}$ and $t_1 \neq t_2$ and $S \neq
S^{\prime}$.  When will this occur?  Under what conditions does $G =
G^{\prime}$?

For example, $t_1$ might be 2004, $t_2$ might be 2012, $S$ might be
\{Rehnquist, Stevens, O'Connor, Scalia, Kennedy, Souter, Thomas,
Ginsburg, Breyer\} and $S^{\prime}$ might be \{Roberts, Stevens,
O'Connor, Scalia, Kennedy, Souter, Thomas, Ginsburg, Breyer\}.  If we
suppose that $G$ is the Supreme Court in 2004 and that $G^{\prime}$
is the Supreme Court, then we want to be able to say that $G =
G^{\prime}$.  

It may appear that we have hit a snag, however, For as Uzquiano points
out, a set of people can compose more than one group at a time.
Suppose that all and only the members of the Supreme Court in 2004 are
part of the Special Committee on Judicial Ethics.  In this case ``the
Supreme Court share[s] all of its members with the Special Committee
on Judicial Ethics as of a certain time [2004]''
\citep[151]{uzquiano2004a}.  It seems false to say that, in 2004, the
Supreme Court was identical with the Special Committee.  But if the
Supreme Court, $G$, is $\sum _{t} ( S )$ and the Special Committee,
$C$, is also $\sum _{t} ( S )$, then how can we deny that $G = C$?

%% One way is to allow that the group-builder can produce numerically
%% distinct groups from repeated applications of the same operation.

One proposal would be to look at the past and future histories of $G$
and $C$.  The Supreme Court is composed of $S^{\prime}$ in 2012, while
the Special Committee has been dissolved.  But I do not see how we can
appeal to identity across time without assuming eternalism.  If we
assume eternalism, we can say that $G = \sum ( \mathellipsis , \sum
_{2004}(S), \sum _{2012}(S^{\prime}), \mathellipsis )$.  But if we do
not assume eternalism, we will have to use temporal operators like
\textsc{was} and \textsc{will}.  We can therefore say only that
\textsc{was}($G = \sum (S)$), $G = \sum (S^{\prime})$ and
\textsc{will}($G = \sum (S^{\prime \prime})$).  We are still left with
its being presently the case that $G = C$.

What we {\em can} do is look at {\em possible} past and future
histories.  That is, we can appeal to the modal properties of the two
groups.  Recall Fine's discussion of the {\em character} conditions of
composition operators (section \ref{material}).  The characteristics,
or properties, of a thing will vary depending on the composition
operator used to produce it.  These properties may include modal
properties.  Different composition operators will produce things with
different modal properties.  Whatever members a set has, it has
necessarily; but a group may possibly have different members.  This is
because groups and sets are produced by means of different operators.

If we want the Supreme Court and the Special Committee to be distinct,
then we can assign them different modal properties.  But there is no
reason to restrict ourselves to modal properties.  The Supreme Court
and the Special Committee differ in many ways: the Supreme Court has
the power to interpret the Constitution, while the Special Committee
can issue reports on judicial ethics.  These different properties
correspond to differences in the character conditions of the two
groups.

But in order to have different character conditions, the Supreme Court
and the Special Committee must be produced by means of different
operators.  The Supreme Court will be the product of some operator
$\sum _{sc}$ and the Special Committee of $\sum _{sp}$.  And since
these two things are the products of different operators, they need
not be identical.

But just as sets and sums are different kinds of things, so the
Supreme Court and the Special Committee must now be recognized as not
merely different groups, but as different {\em kinds}.  There may be a
greater resemblance between their two kinds than there is between
things like sums and sets, but ultimately they have been estranged.
Calling both `groups' is simply categorizing both kinds under a common
label.

\subsection{The explosion of reality}
Is there anything wrong with this conclusion?  It does seem bizarre in
some ways.  For it is clear enough that one person (or group) may be a
member of an infinite number of groups; each of these `groups' will
therefore be a product of a different composition operator.  And each
will, strictly speaking, be a different kind of thing.

Previously we had a relatively tidy ontology.  There were sums, and
sets, and other well-known kinds; but now each task force or
subcommittee is potentially a kind unto itself.  Fine recognizes that
his approach ``will lead to an infinitude of forms of
composition\,\ldots a vast mereological firmament''
\citeyearpar[576]{fine2010}.  But he does not consider this to be a
drawback.

Perhaps my resistance is unwarranted, but I want to maintain that the
Supreme Court, the Special Committee, and everything else we call a
group is of one kind.  This cannot be maintained in Fine's
mereological firmament.  So I wish to re-examine the thesis that
groups---all of them---really are identical with sets.  This will lead
to some strange consequences, but it may be that they are {\em less}
strange than the `explosion of reality' that we otherwise face.

\subsection{Re-examining the set identity thesis}
\label{set-id}
The primary motivation cited above for positing groups was the fact
that the Supreme Court changes its members over time.  For example,
both of the following sentences are true:

\begin{enumerate}[label=(\arabic*)]
  \item The Supreme Court ruled on Roe vs.\ Wade in 1973. \label{roe1}

  \item The set of justices now serving as Supreme Court Justices did
    not rule on Roe vs.\ Wade in 1973
    \citep[135]{uzquiano2004a}. \label{roe2}
\end{enumerate}

One way to accommodate these facts is to ``insist that the Supreme
Court is a set, but to abandon the assumption that there is a single
set to which the phrase `the Supreme Court' refers in sentences
\ref{roe1} and \ref{roe2}'' \citep[138]{uzquiano2004a}.  To
successfully use the term `the Supreme Court' to refer to a set of
justices, there must be an implicit or explicit temporal reference.
If an utterance of \ref{roe1} is true it will be true because it the
speaker intends her audience to recognize her intention to refer to
the set of justices that was the Supreme Court in 1973.  If her
audience, for whatever reason, takes her to be referring to the
current Court, then they will evaluate \ref{roe1} as false.

Considered in this light, `the Supreme Court' is used to express a
relation between sets and times; ``$x$ is the Supreme Court at $t$''
\citep[140]{uzquiano2004a}.  There is some precedent for this sort of
interpretation:

\begin{squote}
Our use of the phrase `the Supreme Court' to express a relation a set
of justices bears to a time is much like our use of the phrase `the
president of the United States' to express a relation an individual
bears to a time.  Different persons may be the president of the United
States at different times, but there is at most one person that bears
that relation to each time \citep[138]{uzquiano2004a}.
\end{squote}

``But,'' it will be objected, ``there is an important difference here.
We use both terms---`the Supreme Court' and `the president'---to refer
to a past, present or future set that `is' the thing, but we also use
`the Supreme Court' to refer to {\em the Supreme Court}, which has
changed its membership over time.  If I say, `the Supreme Court has
become more conservative over the past century', there is no one set I
am referring to.  I must be referring to something else; the obvious
candidate is the {\em group} that is the Court.''

One reply here is to claim that all that what ``the Supreme Court has
become more conservative over the past century'' actually means is
that the members of the sets that have been the Supreme Court have
become more conservative.  Another, similar reply is that someone who
utters ``the Supreme Court has become more conservative over the past
century'' is saying something literally false (either because there is
no unique set that is being referred to, or because there is a unique
set referred to, but one that does not make the proposition true), but
can generally be understood to mean something else; namely, that the
members of the sets that have been the Supreme Court have become more
conservative.

Neither reply is {\em very} unintuitive; indeed, there is something
attractive about a thesis that reserves application of adjectives like
`conservative' for people, rather than other things like groups.

But there is a more pressing worry for the set identity thesis.
Recall that the set that is the Supreme Court at a given time might
also be the Special Committee on Judicial Ethics.  We must admit that
the Supreme Court in 2004 is the set \{Rehnquist, Stevens, O'Connor,
Scalia, Kennedy, Souter, Thomas, Ginsburg, Breyer\}, and the Special
Committee in 2004 is that very same set.  But now we are committed to
this argument:

\begin{enumerate}[ref=(\arabic*)]
  \item The Special Committee on Judicial Ethics is one of the
    committees assembled by the Senate.

  \item The Special Committee on Judicial Ethics is identical with the
    Supreme Court.

  \item {\em Therefore} the Supreme Court is one of the committees
    assembled by the
    Senate. \citep[144]{uzquiano2004a} \label{sup-com}
\end{enumerate}

And \ref{sup-com} seems false.

But it may be possible to argue that \ref{sup-com} is not false but
only {\em misleading} (indeed, very misleading).  For it
(conversationally) implies that future sets referred to by `the
Supreme Court' will be identical to future sets referred to by `the
Special Committee'.  And it is {\em this} that is certainly false.

\subsection{Set membership and literal speech}
\label{implicate}
Suppose we arrive at a meeting of the Special Committee on Judicial
Ethics.  Rehnquist, Stevens, O'Connor, Scalia, Kennedy, Souter,
Thomas, Ginsburg, and Breyer are sitting around a center table.  As we
take our seats you turn to me and say, ``they look rather familiar,
don't they?''  I say ``that's also the Supreme Court.''

What am I referring to with the demonstrative expression ``that''?  If
one thinks that I am referring to a {\em group}---the Special
Committee---that is distinct from the Supreme Court, my utterance will
have to be interpreted as non-literal.  I will have to be understood
to mean that the {\em members} of the Special Committee are also the
members of the Supreme Court.  (On the other hand, if I am referring
to the {\em set} of justices, what I said is literally true.)

Suppose instead that you ask me who the members of the Special
Committee are.  I say ``Rehnquist, Stevens, O'Connor, Scalia, Kennedy,
Souter, Thomas, Ginsburg, and Breyer.  The Special Committee is just
the Supreme Court.''  Here again one could argue that I am speaking
non-literally; what I mean is that the members of the Special
Committee are just the members of the Supreme Court.  (But if the
Supreme Court and the Special Committe are just sets---the same
set---I have again said something literally true.)

Now suppose that the Special Committee is dissolved in 2004.  In 2005,
we see the members of the Supreme Court (still Rehnquist, Stevens,
O'Connor, Scalia, Kennedy, Souter, Thomas, Ginsburg, and Breyer) out
to lunch together.  I point and say ``that was the Special Committee
on Judicial Ethics.''  Now what is the referent of ``that''?  It
cannot be the Special Committee, for that has ceased to be.  It must
either be the Supreme Court or the set \{Rehnquist, Stevens, O'Connor,
Scalia, Kennedy, Souter, Thomas, Ginsburg, and Breyer\}.  Either way,
the proponent of groups will have to interpret this utterance as
non-literal.  (The set-identity theorist can interpret this utterance
as literally true, however; that set was the Special Committe before
the dissolution.)

Now suppose that the Special Committee is dissolved in 2004 and
Rehnquist retired before dying in 2005 (let's pretend he retired in
May).  Now in August we see Rehnquist, Stevens, O'Connor, Scalia,
Kennedy, Souter, Thomas, Ginsburg, and Breyer out to lunch together.
I point and say ``that was the Supreme Court {\em and} the Special
Committee on Judicial Ethics.''  I can only be referring to the set of
justices.  Why not suppose that I have only {\em ever} been referring
to the set of justices?  If I am in fact referring to the set
\{Rehnquist, Stevens, O'Connor, Scalia, Kennedy, Souter, Thomas,
Ginsburg, Breyer\}, then when I say ``that was the Supreme Court {\em
  and} the Special Committee'', I say something literally true.

I also say something literally true when I say ``the Supreme Court is
one of the committees assembled by the Senate'' or ``the Supreme Court
is the Special Committee on Judicial Ethics''.  But it is very
misleading to say either.  By saying ``the Supreme Court is the
Special Committee'' I imply that future referents of `the Supreme
Court' will be identical to future referents of `the Special
Committee'.  It is less misleading to say ``the current Supreme Court
is the Special Committee on Judicial Ethics''.  (It is even less
misleading to say ``the current Supreme Court is also the Special
Committee''.)

We use `the Supreme Court' sometimes to refer (whether literally or
non-literally) to more than one set.  When I say ``the Supreme Court
has become more diverse'' I mean that the members of the sets that
have been the Supreme Court have become more diverse.  It may be due
to this fact that we so easily misinterpret uttered propositions like
``the Supreme Court is the Special Committee''.  A listener might take
this to mean that the members of the sets that have been the Supreme
Court are identical with the members of the sets that have been the
Special Committee.  They would therefore evaluate the utterance as
false.

I have tried to show here that by identifying groups with sets, we can
interpret some talk as literal rather than non-literal.  I think this
is a good thing.  But it will be objected that the set identity thesis
forces us to interpret a great deal of {\em other} talk as
non-literal, talk which is commonly thought of as literal.  Take, for
example, ``the Supreme Court has become more conservative''.  If we
claim that the Supreme Court is a set, we cannot interpret this
utterance as literally true; {\em sets} do not have political
leanings.  Someone who utters this, according to the set identity
thesis, must be taken to mean that the members of the sets that have
been the referents of ``the Supreme Court'' have become more
conservative.

But can the advocate of groups interpret this literally?  Can groups
{\em literally} have political leanings?  Or must the speaker be
interpreted as meaning that the members of the group have become more
conservative?  I think it is plausible that ``the Supreme Court has
become more conservative'' must be interpreted as non-literal, whether
the Supreme Court is a group or a set.

What about an utterance such as ``the Supreme Court ruled against the
defendant''?  If the Supreme Court is a set, this utterance will have
to be interpreted as non-literal.  Sets don't {\em do} things; we will
have to interpret the speaker as meaning that the Supreme Court
justices ruled against the defendant.  But what if the Court was
divided over the ruling?  If several justices wrote dissenting
opinions, it seems that {\em they} didn't rule against the defendant.
Rather, we want to say that the {\em group} ruled against the
defendant.  The proponent of groups may be in a slightly stronger
position here.  But if we identify the Supreme Court with a set, we
can still say that the {\em majority} of the Supreme Court justices
ruled against the defendant.  And that is more or less what we mean
when we say ``the Supreme Court ruled against the defendant''.

Whether we identify the Supreme Court with a group or a set, we will
have to interpret certain talk as non-literal.  This should not worry
us.  It is likely that much of our talk {\em is} non-literal (see
\citet{bach1987}).  For example, we may have to interpret ``the chair
is mine'' as non-literal, because it might be that ``the chair is
mine'' entails that there is only one chair in the world.  Even
propositions involving proper names might be non-literal.  If `Alex'
designates every person named `Alex', then ``Alex is lying down''
might be literally false because it entails either that there is only
one `Alex' or that every `Alex' is lying down.

The lesson is this: the fact that the set identity thesis forces us to
interpret some apparently literal talk as non-literal should not lead
us to reject the set identity thesis.

%% This is because it is mutually assumed that the set referred to by
%% `the Supreme Court' will change; the current set of justices will
%% not always be the Supreme Court.  It is a contingent fact that one
%% set---\{Rehnquist, Stevens, O'Connor, Scalia, Kennedy, Souter,
%% Thomas, Ginsburg, and Breyer\}---is both the Supreme Court and the
%% Special Committee in 2004, But since the form of the sentence ``the
%% Supreme Court is the Special Committee'' is that of an identity
%% statement, and sincethe audience will understandably

\subsection{Conventional identity conditions}
\label{set-convention}
Even supposing everything above is right, there is still more to be
said.  What are the `identity conditions' for the Supreme Court over
time?  Although the Supreme Court is a set, we use `the Supreme Court'
to refer to different sets at different times.  What governs this
shifting reference?  What makes it true that one set is the Supreme
Court in 2004 and a different set is in 2012?

What makes it true that a given set is the Supreme Court at a given
time is simply our legal conventions.  The Constitution authorizes the
recognition of a set of justices as `the Supreme Court'.  Which set is
recognized as the Supreme Court is decided by the legislative and
executive branches.  The president nominates a set (the sitting
justices and the nominated justice) and the legislative branch votes.
The outcome of the vote make it true or false that a given set is the
Supreme Court.

(Now what do we mean when we say ``the Supreme Court was established
in 1789''?  Perhaps that the convention of referring to a set of
justices as `the Supreme Court'---and the granting of legal powers to
them---began in 1789.)

This is analogous for all `groups'.  Teams, bands, militias---what
makes it true that a certain set is a team, band or militia is just
the conventions governing the group.  If I desert my militia and the
other members of the militia recognize my absence as a desertion, then
it is understood that I am no longer part of the militia; for that
reason it is then true that the set containing me is no longer the
militia.  A smaller set, not containing me, is now the militia.

It seems, then, that {\em `identity' conditions over time for groups
  are wholly conventional}.  This is plausible; groups are social
entities, and it makes sense that their composition should be a matter
of social convention.  But if this is true, it suggests something more
radical: that identity conditions over time for physical objects like
chairs are conventional as well.

\section{The conventions of ordinary things}
Recall how Fine defined the summation operator of classical
mereology (section \ref{classical}:

\begin{displaymath}
\sum _m (\mathellipsis , x, y, z, \mathellipsis )
\end{displaymath}

The identity conditions for sums is {\em regular}: sums $A$ and $B$
are identical if and only if the summation operation that produced $A$
took the same things---{\em other than sums}---as variables
($\mathellipsis , x, y, z, \mathellipsis $) as did the operation that
produced $B$.  In other words, if mereological sums are ultimately
composed of partless atoms (`simples'), then two sums are identical if
and only if they share all the same simples.

There is an obvious parallel between the current conception of sums
and that of sets.  Neither sets nor mereological sums, so conceived,
can change their parts.

Ordinary things like statues, however, {\em can} apparently change
their parts.  If I break the nose of my statue, I can replace it.  Do
I thereby have a new statue?  It seems I do not; rather, I have fixed
my old statue.

At this point one might say that statues are therefore not
mereological sums, and posit a type of entity that includes statues
and {\em can} change its parts over time.  Perhaps a statue would then
be composed by a sum at a time.  We could relativize the composition
operator for statues to times.  This would be parallel to our
treatment of groups above (section \ref{group-time}).  The
`object-builder' would then be an operator that takes sums at times
and produces an object---in this case, a statue: $O = \sum _{t} ( S
)$.

But having divorced the statue from the sum that composes it, trouble
arises.  For seems plausible that a sum (at a time) might compose more
than one object.  Just as the set \{Rehnquist, Stevens, O'Connor,
Scalia, Kennedy, Souter, Thomas, Ginsburg, Breyer\} composed both the
Supreme Court and the Special Committee on Judicial Ethics in 2004, so
the sum $S$ might compose the statue and the lump of clay.  Just as
the Supreme Court has different powers and a different history than
the Special Committee, so the statue and the lump have (apparently)
different properties and histories.  The statue cannot survive a
squashing; the lump can.  The lump may have been formed on Monday; the
statue might not have come into being until Tuesday.

We might assume that the lump is also produced by means of the
object-builder: $O^{\prime} = \sum _{t} ( S )$.  But now we face the
same problems that plagued the time-relative group operator (section
\ref{group-id}).  Above, we produced the statue by means of the same
object-builder operation: $O = \sum _{t} ( S )$.  How can we deny that
$O = O^{\prime}$, that the statue is identical with the lump?

Just as with the Supreme Court and the Special Committee, the obvious
way to distinguish the statue and the lump is by reference to their
different properties (modal and otherwise).  Just as we produced the
Supreme Court and the Special Committee by means of different
operators, so we can introduce separate composition operators to build
the statue and the lump.  The statue will be the product of the
statue-builder $\sum _{st}$ and the lump the lump-builder $\sum
_{lump}$.

But the same problems arise.  Having the Supreme Court and the Special
Committee be the products of different operators means that they are
different {\em kinds} of things---to say that both are `groups' is
simply to bring two heterogeneous kinds under one label.  Likewise, the
statue and the lump are different kinds of things---to say that both
are `physical objects' is simply to bring two heterogeneous kinds
under one label.

And just as we faced an `explosion of reality' after allowing the
proliferation of `groups', so again we face an explosion of `physical
things' with different properties.  Since we have allowed that the
statue and the lump may be different things composed of the same sums,
why stop there?  We can introduce more composition operators that
produce distinct objects.  Where we see a statue and a lump, why not
suppose that there is an infinity of objects, each of a different kind
and with slightly different properties?

This is not a particularly attractive position, but it is not
indefensible (see \citet{bennett2004}).  Since we have already allowed
a plurality of scattered objects like archipelagos and dogbushes, why
not allow a plurality of co-located objects?

I do not think I can rule out this possibility, but I think there is
another option that has some plausibility.  Just as we found a way to
identify groups with sets, and so shrink the ``mereological
firmament'', so can we identify ordinary things with sums.

\subsection{Referring to chairs}
\label{chair-ref}
Just as we identified groups like the Supreme Court with different
sets at different times, so we can identify things like chairs with
different sums at different times.  Terms like `chair' will be used to
express a relation between a sum and a time.

This will of course have the consequence that, at $t_1$, the statue is
identical with the lump of clay.  This will be true, since they are
both the set $S$.  The statue {\em is} the lump of clay.  If we feel
tempted to say that the statue is not identical with the lump, this is
because future (and {\em possible}) utterances of `statue' may not be
used to refer to the same sum as future utterances of `lump'.  (Thus
we can account for their apparent modal differences as well.)

The identification of statues (and lumps) with sums allows us to
explain some sorts of talk that would be otherwise problematic.  For
example, suppose I make a statue out of a lump of clay.  You come
along and squish the statue (thereby destroying it):

\stage{Alex}{}{That was my statue!}

If we thought that the statue was a distinct thing from the sum (and
from the lump), we would have to interpret what I say non-literally.
For if the statue was a distinct thing that has been destroyed, then
when I use a demonstrative like ``that'' I cannot be referring to the
non-existence statue.  My audience may interpret me as referring to
the lump, and meaning that there used to be a statue co-located with
the lump.  But if we suppose that the statue was not a distinct thing
from the sum (and from the lump), then what I said is literally true.
For ``that''---that sum---was a statue, but is no longer.  It is no
longer a statue because it no longer satisfies our conventional
criteria for what counts as a statue.  (You could dispute these
criteria; after I say ``that was my statue'', you could say ``it still
is your statue; it's just a flatter statue than it was''.)

So like the identity of groups over time, the identity over time of a
given thing---like my chair---will be conventional.

But this does not mean that {\em I} have the final say over how my
chair persists through time.  The identity over time of the Supreme
Court is conventional, but it is not up to me.  There are entrenched
legal conventions governing the persistence of the Supreme Court.
Likewise, for different things, different conventions will govern
their persistence.  Recall Rosenberg's discussion of the problem with
van Inwagen's Special Composition Question (section \ref{lessons-v}).
He rejected the idea that there is just one way in which `material
objects' are composed:

\begin{squote}
Microphysics explains how protons, neutrons, and electrons compose
different species of atoms, and physical chemistry, how atoms of
various species compose different sorts of molecules
\citep[706]{rosenberg1993}.
\end{squote}

I think it is plausible to claim that atoms, molecules, animals,
chairs, statues and lumps are, in fact, identical with sums.  If this
is correct, then there is really just {\em one} way in which these
things are composed.  But they {\em persist through time} in very
different ways; which sum is identical with a given animal, for
example, depends on the `conventions' of the biological sciences.

\subsection{Do we need Fine's theory at all?}
I have argued that we can identify ordinary things like chairs as
mereological sums, and we can identify things like groups as sets.  It
is therefore not necessary to use Fine's theory of parthood to
describe these things.  Do we need Fine's theory at all?

I think the theory is still useful.  For it shows that sums and sets
both have parts, but in different ways.  There are also sequences,
strings, and other things produced by various composition operators.
And there are other things that exist, like words, poems, events, and
quantities, that have parts in other ways.  Fine's theory may help us
define composition operators for these things, if they cannot be
described in terms of sets and sums (which I suspect they cannot be).

Fine's theory also suggests a novel way of describing temporal parts.
He claims that not all operators are compositional; some are {\em
  decompositional}.  He introduces the segmentation operation that
takes partless simples and produces their `parts'---the top and
bottom, left and right half, etc.  These `parts' are derived from the
wholes, rather than the wholes being built up from the parts:

\begin{squote}
Let us suppose that the universe consists of physical atoms which are
physically indivisible but of finite volume.  We might then
distinguish between the upper and lower parts of the atom (relative to
its orientation at a given time); and it is plausible that the atoms
are to be taken as givens, there being no explanation of their
identity in more basic terms, while the identity of the upper and
lower parts of an atom is to be explained in terms of their {\em
  being} the upper and lower parts of the atom.  Thus the account of
the part is in terms of the whole rather than the other way around
\citep[585]{fine2010}.
\end{squote}

Likewise, if we maintain that objects are not {\em composed} of
temporal parts, but are `temporally indivisible', we can nonetheless
have a temporal-segmentation operator that generates temporal `parts'
from temporally `simple' wholes.  These parts would have theoretical
utility; we could quantify over them without committing ourselves to
their existence (in a `basic' sense).  This sort of derived temporal
part might allow for a form of four-dimensionalism (the conjunction of
universalism and the doctrine of temporal parts) that avoids the
`spinning disc' problem, among others.

So whether or not we accept the ``vast mereological firmament'' that
Fine envisions, his theory of part is useful.

%% \subsection{Are there lumpkins after all?}
%% \label{after}

\section{Lessons}
\label{lessons-p}
We have seen two different theories that account for the existence of
chairs.  They are very different, and neither is entirely
satisfactory.  However, I am not sure what the alternatives are.

The first theory is an application of Fine's theory of part.  On this
theory, there are chairs and statues and lumps, but they all are
different kinds of things.  If they are all `material things', this is
only because we group together the individual kinds that they belong
to.  This is not obviously wrong, but once we have admitted that the
same matter might compose a statue and a lump, we have trouble
resisting the idea that there might be {\em more} things, of other
kinds, composed of that same matter.  Where we might take there to be
one thing (or maybe two), we now seem committed to there being a huge
number of co-located things.

The second theory identifies chairs and other ordinary things with
mereological sums.  As I understand the concept, a sum is like a set
in that it does not change its parts.  Therefore when refer to `my
chair', I am referring to a sum.  If I replace the leg on my chair, I
henceforth use `my chair' to refer to a different sum.  Which sum I
refer to with `my chair' is governed by convention.

I prefer the second theory because it does not commit us to an
`explosion of reality'.  That is not a decisive objection, of course.
It may well be that such an explosion is more plausible than certain
consequences of the second theory.  But I think one of the two must be
right.

\subsection{Deflationary metaphysics}
\label{deflate}
Kathrin Koslicki has an interesting objection to `universalist' theses
such as the one I appear committed to.  Her objection amounts to this:
if every `collection' of objects (such as the London Bridge, a
particle in the moon, and Cal Ripkin, Jr.) is a thing in its own right
(a sum), then metaphysics becomes uninteresting.  There is no longer
any debate about whether chairs or dogbushes are more `real' or have a
stronger claim to existence.  They both (obviously) exist, and the
difference between chairs and lumpkins is not ontological but
conceptual: `chair' is more embedded in our talk, and so chairs have
greater importance to {\em us}.  But metaphysically, or ontologically,
chairs and dogbushes are on the same level.  There is no sense in
which chairs exist and lumpkins do not.

In the quoted material below, Koslicki is criticizing a version of
four-dimensionalism that Sider has previously defended.  Sider's
position was that any collection of objects-at-times is a thing in its
own right.  Sider calls these things `fusions'.  For example, a chair
is a fusion of a large number of {\em temporal part} of things (wood
molecules, or atoms, or simples).  Each thing (wood molecule, atom, or
simple) is a fusion of {\em its} temporal parts.  Each temporal part
of the chair is also a thing (a fusion).

I take no stance on whether objects have temporal parts or rather
`endure' through time.  Moreover, if we accept Fine's theory of parts
then we reject the idea that there is just one composition operation;
the operation that produces fusions is one among many.  But Koslicki's
comments are relevant nonetheless:

\begin{squote}
There is room, in Sider's theory, for {\em some} genuine ontological
disagreements: for example, the universalist, the nihilist and the
holder of the intermediary position genuinely disagree over how many
and which fusions that exist.  But the only genuine ontological
disagreements for which there is room, in Sider's world, are ones that
concern disagreements over `bare' fusions, so to speak.  What has
happened to the houses, trees, people, and cars, the familiar concrete
objects of common-sense, whose persistence this account set out to
analyze?  There are no `deep' ontological facts as to whether a given
fusion should count as a house or not\,\ldots

[By claiming that there can be genuine ontological disputes,] Sider is
guilty of a bit of false advertising: his account is really a way of
saying that, at the end of the day, there is no interesting {\em
  ontological} story to be told about the persistence of our familiar
concrete objects of common-sense; whatever there is to say about the
persistence of houses, trees, people and cars concerns the
organization of our conceptual household
\citeyearpar[124--125]{koslicki2003}.
\end{squote}

Koslicki seems to think that we ought to be able to find some
ontological difference between ``the familiar concrete objects of
common-sense'' and things like lumpkins or chairs-at-times.  But as I
remarked above (section \ref{universalism}), why should what interests
us (familiar objects like chairs) be a guide to what exists?  The
conclusion that ``the persistence [and other properties] of houses,
trees, people and cars concerns the organization of our conceptual
household'' seems to be correct.

However, there {\em is} an ontological difference between some things,
if not between chairs and dogbushes.  One lesson of Kit Fine's theory
of parts is that mereological sums are not the only kind of composite
thing.  There are sets as well, and strings, and sequences, and
perhaps infinitely many other types of thing.  The difference between
a set and a sum is an ontological difference.  Within each type,
however, we must rely on our own conceptual `scheme' to organize
things.

In section \ref{lessons-v} I considered Jay Rosenberg's claim that
the Special Composition Question is the wrong question to be asking.
Rosenberg's position seems to be that there is {\em no} answer to the
Special Composition Question.  Rather, he thinks what it takes to
`compose' something depends on what that something is---making a chair
is not like making a pie.

This insight of Rosenberg's can be connected with the insights of
Fine's theory.  If we understand `composition' in the Special
Composition Question to mean {\em mereological composition}, then
Rosenberg was wrong if he held that there is no correct answer to the
Special Composition Question.  It seems intuitively true that
mereological composition is unrestricted.  But if take `composition'
in the Special Composition Question to be $K$-composition---any
composition operator at all---then Rosenberg was {\em right} that
there is no answer.  What the application conditions are for a
composition operator depends on {\em which} composition operation is
being applied.  

Moreover, determining the application conditions for these composition
operators, and determining their identity conditions, and determining
the other properties of these various operators is a task for
metaphysics.  The field is not then so barren as Koslicki seems to
have feared.  But it is true that many interesting questions---When
are we willing to call something a chair, and why?  What conditions
must be fulfilled?---are not ontological questions anymore.  They are
questions about our ``conceptual household.''

\ifstandalone
\end{spacing}
\bibliography{everything}
\bibliographystyle{ChicagoReedweb}
\fi
\end{document}


\chapter{How does a chair persist over time?}
\chapterpig{How does a chair persist over time?}
\documentclass[11pt]{article}
\usepackage{standalone} \newif\ifstandlone \standalonetrue
\usepackage[left=1.75in, right=1.75in, top=1.25in, bottom=1.25in]{geometry}
\geometry{letterpaper}
\usepackage{graphicx}
\usepackage{enumitem}
\usepackage{amssymb}
\usepackage{amsmath}
\usepackage{epstopdf}
\usepackage{verbatim}
\usepackage{setspace}
\usepackage{natbib}
\setcitestyle{aysep={}}
\usepackage{hyperref}
\usepackage{url}
\synctex=1

\DeclareSymbolFont{symbolsC}{U}{txsyc}{m}{n}
\DeclareMathSymbol{\strictif}{\mathrel}{symbolsC}{74}
\DeclareMathSymbol{\boxright}{\mathrel}{symbolsC}{128}

\newenvironment{squote}{%
\begin{spacing}{1}
\begin{list}{}{%
\setlength{\labelwidth}{0pt}%
\rightmargin\leftmargin%
}
\item\relax
}{%
\end{list}%
\end{spacing}
}

\title{Essentialism}
\author{Alexander A. Dunn}
\begin{document}
\ifstandalone
\maketitle
\begin{spacing}{1.5}
\fi

\label{essential}

In section \ref{parts} I presented three different theories that
modified classical mereology.  These modification were made to
explain, among other things, how objects change their parts over time
and how co-located objects (like the statue and the lump) have
different properties.  But each of these theories require us to posit
an extraordinary plurality of (if only temporarily) co-located
objects.  Such theories are, if not false, at least very strange.

In this section, therefore, I will attempt to sketch an {\em
  essentialist} theory of things.  This theory will allow us to reject
the `plurality thesis'---that there are pluralities of co-located
objects---but it will have problems of its own.  The most glaring is
the consequence that, strictly speaking, things don't change their
parts.

Before we assess the merits of that controversial thesis, there is
another possibility that should be addressed.  If we assume the theory
of {\em four-dimensionalism}, many of our problems appear to go away.
Unfortunately, new ones arise.

\section{What if we assume four-dimensionalism?}
\label{4d}
So far I have been supposing that four-dimensionalism is false.  That
is, I have assumed neither that things are composed of temporal as
well as spatial parts, nor that the past and future exist.  But if we
{\em do} assume four-dimensionalism, we have access to new solutions
to the problems relating to ordinary things.

What is four-dimensionalism?  I am supposing that
`four-dimensionalism' refers to the conjunction of two theories.  The
first is that things have {\em temporal parts}.  The second is {\em
  eternalism}.

Ted Sider presents a relatively clear picture of the first part of the
theory of four-dimensionalism, that of temporal parts:

\begin{squote}
Think of your life as a long story.  Let the story be a rather
narcissistic story: cut out all details about everything else except
you.  So the story begins with an infant (or perhaps a fetus).  It
describes the infant developing into a child and then an adolescent.
The adolescent passes into young adulthood, then adulthood, middle
age, and finally old age and death.  Like all stories, this story has
parts.  We can distinguish the part of the story concerning childhood
from the part concerning adulthood.  Given enough details, there will
be parts concerning individual days, minutes, or even instants.

According to the `four-dimensionalist' conception of persons (all all
other objects that persist over time), persons are a lot like their
stories.  Just as my story has a part for my childhood, so {\em I}
have a part consisting just of my childhood.  Just as my story has a
part describing just this instant, so I have a part that is
me-at-this-very-instant \citeyearpar[1]{sider2001}.
\end{squote}

The claim that we have these {\em temporal
  parts}---me-at-this-instant, or me-as-a-child---relies on a close
analogy between space and time.  It is relatively uncontroversial to
claim that we have {\em spatial} parts.  My foot is a part of me, for
instance, but it is not {\em all} of me (it is a proper part, to use
mereological terms).  The philosopher who claims that we have temporal
parts is saying that, likewise, my adulthood is a part of me, but it
is not all of me.  My childhood is---or was, if we do not assume
eternalism---another part of me.  My infancy, childhood, adulthood,
etc. together {\em compose} me.

This theory of temporal parts is often conjoined with a theory about
time.  This theory is commonly referred to as {\em eternalism}.
According to eternalism, ``time is like space.  There is nothing
special about the things here; things at other places are just as
real; no place is metaphysically distinguished.  Similarly, for the
eternalist, there is nothing special about the present; things at
other times are just as real; no time is metaphysically
distinguished'' \citep[122]{hinchliff1996}.  For the eternalist, there
is a sense in which ``there are dinosaurs'' is true.  Everyone agrees
that there are no dinosaurs {\em now}; the question is whether the
dinosaurs of the past still exist {\em in the past}.  

I have no firm intuition on whether either conjunct of the
four-dimensionalist theory is correct.  I do not know whether things
have temporal parts, and I do not know if the past (and future) exist.
Nonetheless I will assume in this section that four-dimensionalism is
true; {\em if} this assumption is correct, we can explain the
existence of ordinary things in new and interesting ways.

\subsection{Four-dimensional essentialism}
\label{4de}
According to four-dimensionalism, ordinary things like chairs and
statues are {\em four-dimensional spacetime worms}.  They are composed
of temporal parts or {\em slices}; a chair might be made up of
`chair-slices' at $t_{1}$, $t_{2}$, $t_{3}$\,\ldots\,, etc.  Depending
on who you ask, these `slices' have a very small temporal duration or
none at all.  If the latter, they are {\em extended} in only three
dimensions; their temporal extension is point-sized (this is what I
will assume).

Four-dimensionalism is very commonly conjoined with universalism---the
theory, defended in section \ref{universe}, that for any things, there
is something composed of them.  If we assume universalism, then
four-dimensionalism entails that for every set of temporal slices,
there is something composed of them.  There is an object composed of
the first ten years of my life, the Kremlin from 1970--1990, and one
second of a puppy's existence in 2020.  This thing is not, of course,
a person; nor is something composed of the first 10 years of my life
and the last ten years of someone else's.  Certain causal or
psychological connections must hold between the temporal parts of a
thing in order for it to be a person.

The objects composed of these temporal slices are mereological sums in
the classical sense.  Let us use `Krupkin' to designate the object
made of the first ten years of my life, the Kremlin from 1970--1990,
and one second of a puppy's existence in 2020.  Because the past and
future exist (we're assuming eternalism), Krupkin always has the same
parts.  Strictly speaking, it doesn't ever change its parts.  In 1991
it is true to say ``the Kremlin is not {\em now} part of Krupkin'',
but it is not true to say ``the Kremlin is not part of Krupkin.

If we assume universalism in addition to four-dimensionalism, then not
only does Krupkin not change its parts, it {\em cannot} change its
parts.  It cannot change its parts for the reason given in section
\ref{change}.  Let us use `Alkin' to designate the object composed of
the first 10 years of my life and the Kremlin from 1970--1990.  Now if
Krupkin could change its parts, it could lose a part.  Suppose it lost
its puppy part.  Then, if it still exists, it would be the object
composed of the first 10 years of my life and the Kremlin from
1970--1990.  But {\em that} object is Alkin; Krupkin would therefore
become identical with Alkin.  Alkin and Krupkin are not identical,
however, because Krupkin has a property that Alkin does not: the
property of having had a puppy as a part.  So Krupkin cannot, in fact,
lose a part; otherwise we would have a contradiction.

Technically, therefore, four-dimensional universalism is a version of
{\em essentialism}---the thesis that things cannot change their parts.
Four-dimensionalists explain change by relativizing things to times:
for a thing to change its color is just for it to have the relation of
being green at one time and red at another, or having a part at one
time and not at another.

I am somewhat sympathetic to this view.  In section \ref{essential} I
will sketch an essentialist theory of things, but one that presupposes
neither temporal parts nor eternalism.  But here I will briefly
examine how a four-dimensionalist essentialism addresses the issues
related to ordinary things that we have been concerned with.

Four-dimensionalism has two advantages and two disadvantages, when
compared with the three theories above.  The first advantage is that
four-dimensionalism does not posits a plurality of {\em kinds} of
things.  The second is that it does not posit co-located objects.  The
material objects that a four-dimensionalist recognizes are all
mereological sums in the classical sense.  The first disadvantage is
that four-dimensionalism, when conjoined with universalism, produces a
plurality of objects, just as the three theories above do.  The second
disadvantage is that four-dimensionalism has difficulty distinguishing
objects that are co-located for the entirety of their existence.

\subsection{Four-dimensional solutions}
\label{4ds}
The first advantage of four-dimensionalism---that it does not have to posit
a plurality of kinds of things---is primarily an advantage relative to
Fine's theory of composition operators (section \ref{fine-c}).  That
theory produced an incredible plurality, not only of things in
general, but of different kinds of things.  Four-dimensional things
are simply mereological sums, in the classical sense.

The second advantage of four-dimensionalism is that, unlike all three
theories presented above, it does not posit co-located objects.  The
theories in section \ref{parts}, in order to distinguish objects like
the statue and the lump---objects that (currently) share all their
parts---had do posit co-located objects.  But on the four-dimensional
picture, this is unnecessary.  Suppose that the lump is formed on
Monday, and the statue on Tuesday.  The lump therefore has temporal
parts that are `earlier' than any of the statue's parts.  They do not
share all their parts, and so are not co-located.  It is true that
they share all their Tuesday parts; the temporal slices that compose
the lump on Tuesday are the same that compose the statue on Tuesday.
But they share parts only at certain times.  They do not share all
their parts at all times.  (This leads into a problem for
four-dimensionalism, however; it does not appear to let us
differentiate a statue and a lump that {\em always} share their parts.
See section \ref{4dp}).

Four-dimensionalism can also readily account for the apparent change
in membership of the Supreme Court.  Since people (including the
justices) are composed of temporal slices, the Supreme Court can be
identified with the sum of person-slices that correspond to the
various justices' terms.  For example, the 25 September 1981--31
January 2006 temporal parts of Sandra Day O'Connor are part of the
set.  The phenomenon of the Supreme Court `changing' its members
(e.g., Day O'Connor retiring) is simply it being 1 February 2006 and
that temporal part of Day O'Connor not being part of the set.

\subsection{Problems for four-dimensionalism}
\label{4dp}
But there are two disadvantages to four-dimensional universalism.  The
first is that while four-dimensionalism does not posit a plurality of
kinds of things or a plurality of co-located objects, there is still a
sense in which it is a `plurality thesis'.  Any given temporal slice
is part of a plurality of things.  When I point at my chair, I am also
pointing at a thing composed of my chair and a black bear from the
1800s, as well as a thing composed of my chair and the head of Thomas
Aquinas.  All those things (and {\em many} more) are currently located
in the very same place.

This is certainly bizarre, but it no more {\em disproves}
four-dimensionalism than does it disprove the three theories
previously examined.  Unfortunately there is another disadvantage to
four-dimensionalism, one that does threaten it as a theory.

The second disadvantage of four-dimensionalism is that it has trouble
distinguishing between objects that are co-located for the entirety of
their existence.  Suppose that I have two lumps of clay; I form one
into the top half of a figure and I shape the other into the bottom
half.  Having done this, I stick the two pieces of clay together,
forming a statue.  When I do this I also form a new, larger lump of
clay.  I admire the statue and the lump for a little while, then smash
them with a hammer.

Let $S$ be the thing composed of all the statue-slices.  Let $L$ be
the thing composed of all the larger-lump-slices.  $S$ and $L$ are
mereological sums composed of the very same parts; $S = L$.  But if
the statue had been squashed instead of smashed, $L$ would have
survived; but $S$ would not have survived being squashed.  $L$ has a
property that $S$ does not---the property `could survive being
squashed'---and therefore $S \neq L$.  This is a problem.

The four-dimensionalist could say that the lump would not have
survived being squashed, or that the statue would have survived.
Since there is only one thing under investigation (since $S = L$),
that thing must have a consistent set of properties.  It can't be such
that it would both survive and not survive a squashing.  So the
four-dimensionalist will have to say that one of our two intuitions is
wrong.

But there is another, related, difficulty.  In the case just
presented, the statue is the lump ($S = L$).  But suppose there is a
situation exactly like the one presented, but in which the statue and
lump are first squashed, then smashed.  In this case, we are inclined
to say that the lump $L^{\prime}$ continues to exist after the
squashing.  Its parts include temporal slices of the clay after it has
been squashed.  In the case of the statue $S^{\prime}$, however, we
are inclined to say that the statue does not have any temporal parts
after the squashing.  The statue is destroyed when it is squashed.
Since $L^{\prime}$ and $S^{\prime}$ have different parts, they are not
the same thing; $L^{\prime} \neq S^{\prime}$.  The four-dimensionalist
is committed to the claim that whether there is one thing (a statue
that is also a lump) or two things (a statue and a lump) on the table,
and what that thing's (or those things') modal properties are depends
upon whether I squash or smash it.  This seems highly implausible.  By
choosing to squash the statue rather than smash it, do I thereby {\em
  make it the case} that there were two things, rather than one?

The four-dimensionalist will object that, since the future already
exists, it was {\em already} true that there were two things (it has
always been true).  But claiming that it is already the case that I
will squash the statue seems to commit the four-dimensionalist to some
version of {\em determinism}---the thesis that, roughly, the events of
the future are determined, or fixed, to occur.  This may well be true,
but it is largely an empirical hypothesis; to rely on it here would be
unwise.  (If the four-dimensionalist does not assume determinism, and
instead assumes indeterminism, they will presumably have to say that,
since it is indeterminate whether or not I will squash the statue, the
number of things on the table is therefore also indeterminate.  This
seems even worse.)

\subsection{Moving along}
\label{4dc}
Four-dimensionalism allows for the resolution of a number of puzzles
related to ordinary things.  It does not resolve everything, however,
and it introduces a few problems of its own.  Moreover, it requires a
number of controversial assumptions: the theory of temporal parts,
eternalism, and possibly determinism.  I will therefore set aside
four-dimensionalism, and suppose henceforth that the past and future
do not exist, and that things do not have temporal parts.

\section{Re-examining the set identity thesis}
\label{set-id}
The primary motivation cited in section \ref{group} for positing
groups was the fact that the Supreme Court changes its members over
time.  For example, both of the following sentences are true:

\begin{enumerate}[label=(\arabic*)]
  \item The Supreme Court ruled on Roe vs.\ Wade in 1973. \label{roe1}

  \item The set of justices now serving as Supreme Court Justices did
    not rule on Roe vs.\ Wade in 1973
    \citep[135]{uzquiano2004a}. \label{roe2}
\end{enumerate}

One way to accommodate these facts is to ``insist that the Supreme
Court is a set, but to abandon the assumption that there is a single
set to which the phrase `the Supreme Court' refers in sentences
\ref{roe1} and \ref{roe2}'' \citep[138]{uzquiano2004a}.  To
successfully use the term `the Supreme Court' to refer to a set of
justices, there must be an implicit or explicit temporal reference.
If an utterance of \ref{roe1} is true it will be true because it the
speaker intends her audience to recognize her intention to refer to
the set of justices that was the Supreme Court in 1973.  If her
audience, for whatever reason, takes her to be referring to the
current Court, then they will evaluate \ref{roe1} as false.

Considered in this light, `the Supreme Court' is used to express a
relation between sets and times; ``$x$ is the Supreme Court at $t$''
\citep[140]{uzquiano2004a}.  There is some precedent for this sort of
interpretation:

\begin{squote}
Our use of the phrase `the Supreme Court' to express a relation a set
of justices bears to a time is much like our use of the phrase `the
president of the United States' to express a relation an individual
bears to a time.  Different persons may be the president of the United
States at different times, but there is at most one person that bears
that relation to each time \citep[138]{uzquiano2004a}.
\end{squote}

``But,'' it will be objected, ``there is an important difference here.
We use both terms---`the Supreme Court' and `the president'---to refer
to a past, present or future set that `is' the thing, but we also use
`the Supreme Court' to refer to {\em the Supreme Court}, which has
changed its membership over time.  If I say, `the Supreme Court has
become more conservative over the past century', there is no one set I
am referring to.  I must be referring to something else; the obvious
candidate is the {\em group} that is the Court.''

One reply here is to claim that all that what ``the Supreme Court has
become more conservative over the past century'' actually means is
that the members of the sets that have been the Supreme Court have
become more conservative.  Another, similar reply is that someone who
utters ``the Supreme Court has become more conservative over the past
century'' is saying something literally false (either because there is
no unique set that is being referred to, or because there is a unique
set referred to, but one that does not make the proposition true), but
can generally be understood to mean something else; namely, that the
members of the sets that have been the Supreme Court have become more
conservative.

Neither reply is {\em very} unintuitive; indeed, there is something
attractive about a thesis that reserves application of adjectives like
`conservative' for people, rather than other things like groups.

But there is a more pressing worry for the set identity thesis.
Recall that the set that is the Supreme Court at a given time might
also be the Special Committee on Judicial Ethics.  We must admit that
the Supreme Court in 2004 is the set \{Rehnquist, Stevens, O'Connor,
Scalia, Kennedy, Souter, Thomas, Ginsburg, Breyer\}, and the Special
Committee in 2004 is that very same set.  But now we are committed to
this argument:

\begin{enumerate}[ref=(\arabic*)]
  \item The Special Committee on Judicial Ethics is one of the
    committees assembled by the Senate.

  \item The Special Committee on Judicial Ethics is identical with the
    Supreme Court.

  \item {\em Therefore} the Supreme Court is one of the committees
    assembled by the
    Senate. \citep[144]{uzquiano2004a} \label{sup-com}
\end{enumerate}

And \ref{sup-com} seems false.

But it may be possible to argue that \ref{sup-com} is not false but
only {\em misleading} (indeed, very misleading).  For it
(conversationally) implies that future sets referred to by `the
Supreme Court' will be identical to future sets referred to by `the
Special Committee'.  And it is {\em this} that is certainly false.

\subsection{Set membership and literal speech}
\label{implicate}
I argued in section \ref{eng-quant} that ordinary uses of `there is'
are often false.  For example, if I say ``there is no beer'', what I
say is almost certainly false---there is beer {\em somewhere}---but
what I mean is that there is no beer in the house.

It is very likely that much of our ordinary talk is similarly
non-literal (see \citet{bach1987}).  For example, we should interpret
``the chair is mine'' as non-literal, because ``the chair is mine''
entails that there is only one chair in the world.  Even propositions
involving proper names might be non-literal.  If `Alex' designates
every person named `Alex', then ``Alex is lying down'' is literally
false, since it entails either that there is only one `Alex' or that
every `Alex' is lying down.

Therefore, if a theory predicts that some of our talk about groups is
non-literal, we should not necessarily be worried.  But not {\em all}
of our talk about groups is non-literal, and when a theory can
preserve the intuition that certain thinks are literally true, that
should be taken as an advantage.  At least for a certain class of
examples, the set-identity thesis preserves more of our intuitive
judgments about literal speech than does the theory that posits
groups as distinct from sets.

\begin{enumerate}
  \item Suppose we arrive at a meeting of the Special Committee on
    Judicial Ethics.  Rehnquist, Stevens, O'Connor, Scalia, Kennedy,
    Souter, Thomas, Ginsburg, and Breyer are sitting around a center
    table.  As we take our seats you turn to me and say, ``they look
    rather familiar, don't they?''  I say ``that's also the Supreme
    Court.''

    What am I referring to with the demonstrative expression ``that''?
    \begin{itemize}
      \item If one thinks that I am referring to a {\em group}---the
        Special Committee---that is distinct from the Supreme Court,
        my utterance will have to be interpreted as non-literal.  I
        will have to be understood to mean that the {\em members} of
        the Special Committee are also the members of the Supreme
        Court.
      \item On the other hand, if I am referring to the {\em set}
        of justices, what I said is literally true.
     \end{itemize}

  \item Suppose instead that you ask me who the members of the Special
    Committee are.  I say ``Rehnquist, Stevens, O'Connor, Scalia,
    Kennedy, Souter, Thomas, Ginsburg, and Breyer.  The Special
    Committee is just the Supreme Court.''  
    \begin{itemize}
      \item Here again one could argue that I am speaking
        non-literally; what I mean is that the members of the Special
        Committee are just the members of the Supreme Court.  
      \item But if the Supreme Court and the Special Committee are
        just sets---the same set---I have again said something
        literally true.
    \end{itemize}

  \item Now suppose that the Special Committee is dissolved in 2004.
    In 2005, we see the members of the Supreme Court (still Rehnquist,
    Stevens, O'Connor, Scalia, Kennedy, Souter, Thomas, Ginsburg, and
    Breyer) out to lunch together.  I point and say ``that was the
    Special Committee on Judicial Ethics.''  Now what is the referent
    of ``that''?  It cannot be the Special Committee, for that has
    ceased to be.  It must either be the Supreme Court or the set
    \{Rehnquist, Stevens, O'Connor, Scalia, Kennedy, Souter, Thomas,
    Ginsburg, and Breyer\}.  
    \begin{itemize}
      \item Either way, the proponent of groups will
    have to interpret this utterance as non-literal.  
      \item The set-identity theorist can interpret this utterance as
        literally true, however; that set was the Special Committee
        before the dissolution.
    \end{itemize}

  \item Now suppose that the Special Committee is dissolved in 2004
    and Rehnquist retired before dying in 2005 (let's pretend he
    retired in May).  Now in August we see Rehnquist, Stevens,
    O'Connor, Scalia, Kennedy, Souter, Thomas, Ginsburg, and Breyer
    out to lunch together.  I point and say ``that was the Supreme
    Court {\em and} the Special Committee on Judicial Ethics.''  I can
    only be referring to the set of justices.  Why not suppose that I
    have only {\em ever} been referring to the set of justices?  If I
    am in fact referring to the set \{Rehnquist, Stevens, O'Connor,
    Scalia, Kennedy, Souter, Thomas, Ginsburg, Breyer\}, then when I
    say ``that was the Supreme Court {\em and} the Special
    Committee'', I say something literally true.
\end{enumerate}

These examples provide some support for the set identity thesis.  At
the very least they show that identifying groups with sets does not
mean that all our talk about groups must be interpreted as
non-literal.  However, the set identity thesis also predicts that some
propositions will be literally true, when intuitively we may believe
that they are not.  For example, according to the set identity thesis,
I say something literally true when I say ``the Supreme Court is one
of the committees assembled by the Senate'' or ``the Supreme Court is
the Special Committee on Judicial Ethics''.  But it is very misleading
to say either.  By saying ``the Supreme Court is the Special
Committee'' I imply that future referents of `the Supreme Court' will
be identical to future referents of `the Special Committee'.  It is
less misleading to say ``the current Supreme Court is the Special
Committee on Judicial Ethics''.  (It is even less misleading to say
``the current Supreme Court is also the Special Committee''.)

One may object that, while this is all well and good, the set identity
thesis fails the most important test.  The set identity thesis
predicts that propositions about the Supreme Court evolution over
time, such as ``the Supreme Court has become more diverse'' are
literally false.

This is a drawback for the set identity theorist, but I do not think
it is a great one.  As a parallel case, take the proposition ``the
temperature is dropping''.  For this to be literally true, there would
have to be some thing---the temperature---that is, in some sense,
dropping.  But it is plausible to interpret a speaker who says ``the
temperature is dropping'' as meaning that soon, the number that is the
referent of `the temperature' will be lower than the number currently
referred to by `the temperature'.  Likewise, when I say ``the Supreme
Court has become more diverse'' I mean that the members of the sets
that have been the Supreme Court have become more diverse.

It may be due to this fact that we so easily misinterpret uttered
propositions like ``the Supreme Court is the Special Committee''.  A
listener might take this to mean that the members of the sets that
have been (and will be) the Supreme Court are identical with the
members of the sets that have been (and will be) the Special
Committee.  They would therefore evaluate the utterance as false.

Another example that the set identity thesis predicts as non-literal
is ``the Supreme Court has become more conservative''.  If we claim
that the Supreme Court is a set, we cannot interpret this utterance as
literally true; {\em sets} do not have political leanings.  Someone
who utters this, according to the set identity thesis, must be taken
to mean that the members of the sets that have been the referents of
``the Supreme Court'' have become more conservative.

But can a philosopher who distinguishes groups from sets interpret
this literally?  Can groups {\em literally} have political leanings?
Or must the speaker be interpreted as meaning that the members of the
group have become more conservative?  I think it is plausible that
``the Supreme Court has become more conservative'' must be interpreted
as non-literal, whether the Supreme Court is a group or a set.

What about an utterance such as ``the Supreme Court ruled against the
defendant''?  If the Supreme Court is a set, this utterance will have
to be interpreted as non-literal.  Sets don't {\em do} things; we will
have to interpret the speaker as meaning that the Supreme Court
justices ruled against the defendant.  But what if the Court was
divided over the ruling?  If several justices wrote dissenting
opinions, it seems that {\em they} didn't rule against the defendant.
Rather, we want to say that the {\em group} ruled against the
defendant.  The proponent of groups may be in a slightly stronger
position here.  But if we identify the Supreme Court with a set, we
can still say that the {\em majority} of the Supreme Court justices
ruled against the defendant.  And that is more or less what we mean
when we say ``the Supreme Court ruled against the defendant''.

Whether or not we identify the Supreme Court and other groups with
sets, we will have to interpret some apparently literal speech as
non-literal.  But the set identity thesis, at least with regard to
this slate of examples, treats talk about groups more consistently
than does the theory that groups are distinct from sets.

\subsection{Conventional identity conditions}
\label{set-convention}
Even supposing everything above is right, there is still more to be
said.  What are the `identity conditions' for the Supreme Court over
time?  Although the Supreme Court is a set, we use `the Supreme Court'
to refer to different sets at different times.  What governs this
shifting reference?  What makes it true that one set is the Supreme
Court in 2004 and a different set is in 2012?

What makes it true that a given set is the Supreme Court at a given
time is simply our legal conventions.  The Constitution authorizes the
recognition of a set of justices as `the Supreme Court'.  Which set is
recognized as the Supreme Court is decided by the legislative and
executive branches.  The president nominates a set (the sitting
justices and the nominated justice) and the legislative branch votes.
The outcome of the vote make it true or false that a given set is the
Supreme Court.

(Now what do we mean when we say ``the Supreme Court was established
in 1789''?  Perhaps that the convention of referring to a set of
justices as `the Supreme Court'---and the granting of legal powers to
them---began in 1789.)

This is analogous for all `groups'.  Teams, bands, militias---what
makes it true that a certain set is a team, band or militia is just
the conventions governing the group.  If I desert my militia and the
other members of the militia recognize my absence as a desertion, then
it is understood that I am no longer part of the militia; for that
reason it is then true that the set containing me is no longer the
militia.  A smaller set, not containing me, is now the militia.

It seems, then, that {\em `identity' conditions over time for groups
  are wholly conventional}.  This is plausible; groups are social
entities, and it makes sense that their composition should be a matter
of social convention.  But if this is true, it suggests something more
radical: that identity conditions over time for physical objects like
chairs are conventional as well.

\section{The conventions of ordinary things}
\label{chair-ref}
Just as we identified groups like the Supreme Court with different
sets at different times, so we can identify things like chairs with
different sums at different times.

This will have the consequence that, at $t_1$, the statue is identical
with the lump of clay.  This will be true, since they are both the set
$S$.  The statue {\em is} the lump of clay.  This seems correct.  If
someone were to ask ``I see the statue, but where did the lump of clay
go?'' we would reply, ``the statue {\em is} the lump''.

If we feel tempted to say that the statue is not identical with the
lump, this is because future (and {\em possible}) utterances of
`statue' may not be used to refer to the same sum as future utterances
of `lump'.  (Thus we can account for their apparent modal differences
as well.)

The identification of statues (and lumps) with sums allows us to
explain some sorts of talk that would be otherwise problematic.  For
example, suppose I make a statue out of a lump of clay.  You come
along and squish the statue (thereby destroying it):

\stage{Alex}{}{That was my statue!}

If we thought that the statue was a distinct thing from the sum (and
from the lump), we would have to interpret what I say non-literally.
For if the statue was a distinct thing that has been destroyed, then
when I use a demonstrative like ``that'' I cannot be referring to the
non-existence statue.  My audience may interpret me as referring to
the lump, and meaning that there used to be a statue co-located with
the lump.  But if we suppose that the statue was not a distinct thing
from the sum (and from the lump), then what I said is literally true.
For ``that''---that sum---was a statue, but is no longer.  It is no
longer a statue because it no longer satisfies our conventional
criteria for what counts as a statue.  (You could dispute these
criteria; after I say ``that was my statue'', you could say ``it still
is your statue; it's just a flatter statue than it was''.)

One objection that may arise here is that, {\em strictly speaking},
the statue still exists.  Worse, if the statue is just the sum, then
the statue existed even before it was sculpted!  How can that be?

Suppose I have a lump of clay on Monday:

\stage{Alex}{}{This will be a statue!}

On Tuesday I make a statue out of the clay:

\stage{Alex}{}{Yesterday this was nothing more than a lump of clay!
  Now look at it!}

On Wednesday you squish the statue:

\stage{Alex}{}{Well, it's not a statue anymore.}

With these examples, I am trying to motivate the idea that we refer to
the sum when we use terms like `it' and `this'.  The sum referred to
on Monday is the same (or nearly the same) sum referred to on Tuesday
and on Wednesday.  When I say ``this will be a statue'', therefore, I
mean that this sum will meet the criteria for being referred to as a
statue.  When I say ``this was nothing more than a lump of clay'', I
mean that this sum previously met only the criteria for being referred
to as a lump of clay.  When I say ``it's not a statue anymore'', I
mean that it no longer meets the criteria for being referred to as a
statue.

But what if I say ``that statue doesn't exist anymore''?  I suggest
that this should be interpreted in almost all cases as meaning that
the sum that is understood to have been the referent of `that statue'
no longer meets the criteria for being the referent of `statue'.  If
``that statue doesn't exist'' is taken in a more literal sense, then
it is false.

Peter van Inwagen has a similar objection to the idea that sums cannot
change their parts (he assumes that ordinary things are sums).
Suppose that sums cannot change change their parts:

\begin{squote}
Call the bricks that were piled in the yard last Tuesday the ``Tuesday
bricks.''  Between last Tuesday and today, the Wise Pig has built a
house---the ``Brick House''---out of the Tuesday bricks (using them
all and using no other materials).  The Brick House did not exist last
Tuesday (that is, it was not then a pile of bricks, a thing that was
not yet a house but would become a house).  The Brick House is not,
therefore, a mereological sum; for if it were, it would have been (it
would have ``existed as'') a pile of bricks last Tuesday
\citeyearpar[616]{inwagen2006}.
\end{squote}

But since the Brick House {\em is} a mereological sum, van Inwagen
concludes that our supposition that sums can't change their parts is
false; he claims that mereological sums {\em can} change their parts.

However, I suggest that, strictly speaking, the Brick House {\em did}
exist last Tuesday.  But last Tuesday it (the sum) did not meet the
criteria for being referred to as a house.  Today I point to the Brick
House and say ``last Tuesday that was just a pile of bricks.  Now it's
a house!''  By `that' I mean the Brick House---the sum---which was a
pile of bricks on Tuesday.  If I say ``the Brick House did not exist
last Tuesday'' I should be taken to mean just that the Brick House did
not meet the criteria for being referred to as `the Brick House' last
Tuesday.

\subsection{Criteria and convention}
\label{criteria}
Like the `identity' of groups over time, the `identity' over time of a
given thing---like my chair---will be conventional.

But this does not mean that {\em I} have the final say over how my
chair persists through time.  The identity over time of the Supreme
Court is conventional, but it is not up to me.  There are entrenched
legal conventions governing the persistence of the Supreme Court.
Likewise, for different things, different conventions will govern
their persistence.  Recall Rosenberg's discussion of the problem with
van Inwagen's Special Composition Question (section \ref{lessons-v}).
He rejected the idea that there is just one way in which `material
objects' are composed:

\begin{squote}
Microphysics explains how protons, neutrons, and electrons compose
different species of atoms, and physical chemistry, how atoms of
various species compose different sorts of molecules
\citep[706]{rosenberg1993}.
\end{squote}

I think it is plausible to claim that atoms, molecules, animals,
chairs, statues and lumps are, in fact, identical with sums.  If this
is correct, then there is really just {\em one} way in which these
things are composed.  But they {\em persist through time} in very
different ways; which sum is identical with a given animal, for
example, depends on the `conventions' of the biological sciences.

\\

The theory I have been illustrating is a version of essentialism---the
thesis that things do not change their parts.  This is a controversial
thesis, and I will address the objections to it in section
\ref{essentialism}.  The primary objection is that much of our talk
involving numerical sameness---for example, ``that is the same chair
as yesterday''---is literally false.  I have claimed that such
expressions should be interpreted non-literally.

Adopting Ted Sider's theory of `temporal counterparts' would enable me
to maintain that utterances like ``that is the same chair as
yesterday'' are literally true.  But Sider's theory relies on dubious
assumptions about meaning.

%% There are, however, theories of ordinary speech that are compatible
%% with essentialism and preserve the literal truth of utterances like
%% ``that is the same chair as yesterday''.  The first is Roderick
%% Chisholm's theory of `entia successiva'.  The second is Ted Sider's
%% `stage view'.  Neither is satisfactory.

% chisholm does NOT preserve literal truth.

%% \subsection{What is an `ens successivum'?}
%% \label{ens}

\subsection{Ordinary speech and temporal counterpart theory}
\label{counterpart}
On my version of essentialism, it is often literally false to point to
a chair and say ``that chair was a chair two hours ago''.  This is
because the referent of `that chair' is a mereological sum that did
not satisfy the criteria for being the referent of `chair' two hours
ago (a different sum was `that chair').  This is obviously an
unintuitive conclusion.  Ted Sider's temporal counterpart theory
offers a possible way to avoid this conclusion.

Sider's counterpart theory is part of the theory of
four-dimensionalism he once promoted \citeyearpar{sider2001}.  Unlike
most four-dimensionalists who claim that we use terms like `chair' to
refer to `spacetime worms' or `aggregates of chair-stages', Sider
argued that we use such terms to refer to instantaneous stages, not
`continuant' worms or aggregates.  What this means is that in ordinary
talk we never refer to the same thing twice; the chair I refer to at
$t_1$ is one temporal part (chair-at-$t_1$) and the chair I refer to
at $t_2$ is another.  When I use `Ted' to refer to Ted Sider, I am not
referring to the temporally extended object that includes a childhood;
I am referring to something that lasts only for an instant.

Nonetheless Sider claims that when I say ``Ted was once a boy'', I say
something literally true.  How can this be?  The object I am referring
to was never a boy.  It is here that Sider introduces temporal
counterparts:

\begin{squote}
According to my temporal counterpart theory, the truth condition of an
utterance like `Ted was once a boy' is this: there exists some person
stage $x$ prior to the time of the utterance, such that $x$ was a boy,
and $x$ bears the temporal counterpart relation to Ted.  Since there
is such a stage, the claim is true. \citeyearpar[193]{sider2001}.
\end{squote}

I think Sider must be assuming that what `Ted was once a boy' {\em
  means} is `there exists some person stage $x$ prior to the time of
the utterance, such that $x$ was a boy, and $x$ bears the temporal
counterpart relation to Ted'.  If they are not synonymous, then there
is no reason why the truth-condition of the former would be the
latter.  It seems obvious that the truth-condition of `Ted was once a
boy' is that Ted (the stage) was once a boy.  If this is not the
truth-condition, then it must be because `Ted was once a boy' does not
actually mean that Ted was once a boy, but instead means that there
exists some person stage $x$ prior to the time of the utterance, such
that $x$ was a boy, and $x$ bears the temporal counterpart relation to
Ted.

In order to maintain that `Ted was once a boy' is literally true,
Sider must claim that it means something other than that Ted was once
a boy.  This is a highly implausible and unmotivated claim; the only
reason I can think of as to why Sider might make such a claim would be
because he holds a truth-conditional theory of meaning and believes
that `Ted was once a boy' is true if and only if `there exists some
person stage $x$\,\ldots ' is true.  But a truth-conditional theory of
meaning (see section \ref{hirsch}) is controversial and susceptible to
numerous counter-examples.  It seems far more reasonable to admit that
`Ted was once a boy' means that Ted was once a boy, and is literally
false.

\section{Essentialism}
\label{essentialism}
The theory I have been promoting is a version of {\em essentialism}.
Essentialism is the thesis that, strictly speaking, things don't
change their parts.  One can endorse or oppose essentialism in various
domains.  For example, everyone is a set essentialist; nobody (as far
as I know) claims that sets can change their parts.  But not everyone
is a {\em mereological} essentialist.

People who deny mereological essentialism are, I think, making one of
two claims:

\begin{enumerate}
  \item They may be claiming that ordinary things like chairs are not
    mereological sums; chairs can change their parts, so essentialism
    {\em with regard to chairs} is false.  A philosopher who makes
    this claim might allow that mereological sums, if there are such
    things, cannot change their parts.
  \item They may be claiming that mereological sums, whether or not
    they are identical with ordinary things like chairs, can change
    their parts.
\end{enumerate}

I argued against the second claim in section \ref{change}.  But a
philosopher who makes either claim will reject the theory I have been
building.  They will say that my theory flies in the face of common
sense (and I made so much of common sense in earlier sections!).  They
will say things like this:

\begin{squote}
According to [the essentialist], it is never literally correct to say
that a thing survives a change in parts.  This is a point of massive
departure from ordinary belief \citep[184]{sider2001}.
\end{squote}

This is more or less the argument against essentialism.  You point at
a chair and say ``I'm supposed to believe that if that chair loses
{\em one atom}, it's literally a different chair?''

It is interesting to note that in the past, many philosophers were
more than willing to affirm this.  Roderick Chisholm points out that

\begin{squote}
Abelard held that `no thing has more or less parts at one time than at
another'\,\ldots [and] Leibniz said `we cannot say, speaking according
to the great truth of things, that the same whole is preserved when a
part is lost'\,'' \citeyearpar[145]{chisholm1979}.
\end{squote}

Joseph Butler also held that ``when a man swears to the same tree, as
having stood fifty years in the same place, he means only the same as
to all the purposes of property and uses of common life, and not that
the tree has been all that time the same in the strict philosophical
sense of the word'' \citeyearpar[100]{butler1975a}.

One might object that these philosophers were simply failing to
distinguish {\em descriptive} and {\em numerical} sameness.  When I
say I have the same guitar as you, all I mean is that it is
descriptively the same, not that I have your guitar.  Likewise perhaps
Abelard, Leibniz, and Butler observed that a sapling is descriptively
different from the mature tree that it grows into, and then drew the
unwarranted conclusion that the sapling and mature tree are not
therefore the same.

This seems a bit uncharitable, but in any case there are arguments
supporting the same conclusion.  The first comes from Chisholm:

\begin{squote}
Let us picture to ourselves a very simple table, improvised from a
stump and a board.  Now one might have constructed a very similar
table by using the same stump and a different board, or by using the
same board and a different stump.  But the only way of constructing
precisely {\em that} table is to use that particular stump and that
particular board.  It would seem, therefore, that that particular
table is {\em necessarily} made up of that particular stump and that
particular board \citeyearpar[146]{chisholm1979}.
\end{squote}

It may be objected that, {\em once the table is built}, it is possible
to change its parts without thereby destroying one table and
constructing another.  Once I have built a table, it seems true that I
could take it apart and reassemble the very same table.  It even seems
that I could take it apart and reassemble the very same table with a
slight modification; for example, I could put it back together with
one new leg.  It may be, then, that all Chisholm's argument shows is
that this particular table necessarily {\em began} its existence with
a particular stump and board.  But there is nothing in the argument
that shows that it necessarily cannot go on to change its parts while
remaining numerically identical.

But if a chair can remain numerically identical after changing a part,
it is difficult to say {\em how large} a part the chair can lose while
remaining the (numerically) same chair.  If most of the chair is
blasted away, then we may very well say that the chair is no more.
But {\em how much} must be blasted away?  Or suppose we have a portion
of gold.  How many atoms of gold can be stripped off before it is no
longer the same portion?  Thomson claims that ordinary uses of
`portion' are context-dependent:

\begin{squote}
The ordinary use of the term ``portion'' is heavily context-dependent.
If an atom drifts away from your portion of gold, do you still have
the same portion of gold?  You will say no if you are a scientist
engaged in an experiment for which every atom matters. You will say
yes if you are a jeweler about to make a ring.  Similarly, in fact,
for clay.  If you have just bought a load of clay, and a handful falls
off while you are on your way home, is the portion you have when you
get home the same as the portion you bought?  You will say no if you
had carefully measured and bought exactly as much as you need.  You
will say yes if loss of a handful makes no difference to you
\citeyearpar[163]{thomson1998a}.
\end{squote}

When we say that a use of a term is `context-dependent', that can mean
one of two things.  First, it may mean that whether an utterance
involving a use of the term is {\em correct}, or {\em appropriate},
depends on the context.  It would not be appropriate for the scientist
to say that she has the same portion after the loss of several atoms,
because those atoms matter for the experiment.  Second, to say that
the use of a term is `context-dependent' may mean that whether an
utterance involving a use of the term is {\em true} depends on the
context.  In the quoted passage above, do the scientist and jeweler
both say true things?  If they do, then the truth-conditions of
`portion' are context-dependent.  This would mean that whether an
utterance involving `portion' is true depends on the context of the
utterance.  This would also suggest that the {\em meaning} of
`portion' depends on the context, for the truth-value of a sentence is
generally thought to be a function of the meaning of its constituent
elements, including words.

But just as I do not think there are different senses of `there is'
(see section \ref{eng-quant}), so I do not think that there are
multiple senses of `portion'.  I find it far more plausible to think
that only the scientist says something that is, {\em strictly
  speaking}, true.  The jeweler, when she affirms that she has the
same portion of gold, may say something correct or appropriate, given
the context, but it is not {\em true}.  Strictly speaking, a portion
cannot change its parts; why should we assume that a chair can?

But even if ``that's the same chair as yesterday'' is literally false,
the sum that yesterday satisfies the criteria for being the referent
of ``chair'' no longer satisfies those criteria; a different sum that
has many of the same parts now satisfies those criteria, and so
qualifies as `the same chair'.  (How similar the two sums has to be
is, again, a conventional matter.)

Maybe I am wrong and it is true that (strictly speaking) the chair can
lose its parts and yet remain the (numerically) same chair.  If so,
then we must accept the mereological firmament that comes with the
theories of section \ref{parts}.

\section{Am I a mereological sum?}
\label{i-sum}
I have proposed that ordinary things like chairs and statues are
mereological sums.  Their apparent persistence through change is a
result of certain conventions---a chair at $t_1$ is the `same' chair
at $t_2$ if, first, there is a sum at each time that meets our
criteria for being a referent of `chair' and, second, either the sum
that is the referent of `chair' at $t_1$ is the same sum that is the
referent of `chair' at $t_2$, or the sum that was the referent of
`chair' at $t_1$ does not meet the criteria for being the referent
of `chair' at $t_2$ and a different sum does meet these criteria at
$t_2$ and is sufficiently related to the first sum.  What is
`sufficiently related' is again a conventional matter, but will
presumably involve causal and spatiotemporal continuity.

If ordinary things are sums, then are other things sums as well?  I
will suppose that sums are `material things' as opposed to `abstract
things' (whatever that distinction comes to), but are {\em all}
material things sums?  If we are material things, are we therefore
sums?

\subsection{All material things are sums}
\label{material-sum}
If we think that ordinary things are sums, and that ordinary things
are material things, I think it is extremely plausible to conclude
that all material things are sums.  For what else would they be?

What is included under the label `material thing'?  I would include
things like chairs, and desks, and desk lamps, and doors, and
doorways, and houses, and gardens, and plants.  I would also include
minuscule objects like molecules and massive objects like planets and
galaxies.  What would these things be, if not sums?

I proposed that ordinary things are sums so as to avoid the conclusion
that there is a plurality of different kinds of ordinary things
(statues and lumps only scratch the surface) all overlapping each
other.  This essentialist proposal was made so as to avoid positing
many different kinds of things.  So anyone who accepts the
essentialist theory should be sympathetic to the idea that all
material things are sums.

I don't have a powerful argument for this conclusion, but I don't see
the {\em point} of supposing that all and only ordinary things are
sums, but other material things are some different kind of object.

\subsection{We are material things}
\label{material-beings}
Even if the idea that all material things are sums is (or should be)
relatively uncontroversial, the idea that {\em we} are material beings
will not be unanimously accepted.  For it does have some unintuitive
consequences.  

First, it rules out identifying us with our mental states.  Suppose
all my psychological characteristics---memory, personality---is
somehow transferred to another body.  The brain in that body is
`wiped' before my psychology is transferred, and after the operation
my old brain is similarly `wiped'.  There is a temptation to say that
I exist in the new body.  But saying this commits us to the claim that
I am not a material thing, because I `left' my old material body and
came to `inhabit' a new one:

\begin{squote}
 If I am identical with the thinking substance in which I am thus
 placed, then I cannot be transferred {\em from} that substance to
 another substance \citep[107]{chisholm1979}.
\end{squote}

Claiming that we are material things entails that psychological
continuity is not a criterion of identity.  The body into which my
psychology is transferred is not me, according to the materialist
claim.  Psychological continuity is often taken to be {\em the}
criterion of identity, so one might take this consequence as a
refutation of the claim that we are material things.

But if we are not material things, what are we?  The only alternative
I see is to claim that we are immaterial minds or souls.  These
positions seem, to me, to be more implausible than the claim that we
are material things.  (Much, of course, can be said in defense of such
a position.)

Claiming that we are material things, however, gives rise to another
question: what material things are we?  Are we identical with our
brains, or with our bodies?

I suggest, though somewhat tentatively, that we are identical with our
bodies.  I agree with Peter van Inwagen on this much:

\begin{squote}
I suppose that [the objects of mental predicates]---Descartes, you,
I---are material objects, in the sense that they are ultimately
composed entirely of quarks and electrons.  They are, moreover, a very
special sort of material object.  They are not brains or cerebral
hemispheres.  They are living animals; being {\em human} animals, they
are things shaped roughly like statues of human beings
\citeyearpar[6]{inwagen1995}.
\end{squote}

Eric Olson has a very plausible argument for the same conclusion:

\begin{enumerate}
  \item There is a human animal sitting in your chair.
  \item The human animal sitting in your chair is thinking. (If you
    like, every human animal sitting there is thinking.)
  \item You are the thinking being sitting in your chair. The one and
    only thinking being sitting in your chair is none other than
    you. Hence, you are that animal \citeyearpar[354]{olson2003a}.
\end{enumerate}

One apparent consequence of the claim that we are material human
animals is that if my brain is removed from my body and put into
another body, that new person is not me.  Claiming that we are
material things required denying that psychological continuity is a
criterion of identity; claiming that we are material human animals
requires denying that even brain continuity is a criterion of
identity.

This may seem to be a troubling consequence, but it is much less
troubling if we accept the essentialist theory.  If material objects
are sums, and if we are material objects, then we are sums.  And if
sums do not, strictly speaking, change their parts over time, then,
like the `identity' conditions for chairs and other ordinary things,
the `identity' conditions over time for {\em us} is conventional.

Another difficulty with identifying us with human animals disappears
if we accept an essentialist theory.  Dean Zimmerman has objected to
Olson's argument by claiming that `human animal' can be replaced with
`human body' without making the argument invalid
\citeyearpar[24]{zimmerman2008a}.  The problem, however, is that it
seems true that we cease to exist when we die.  So Zimmerman concludes
that we are not bodies or animals.

If we accept an essentialist theory, however, the problem disappears.
If, strictly speaking, I can't change my parts over time, then I am
not (strictly speaking) the same person that will be the referent of
`Alex' a month from now (or even a week).  I will certainly not be
identical with a dead body further down the road.

\subsection{How do I `persist' over time?}
\label{person-persist}
The idea that, strictly speaking, I don't change my parts over time
seems crazy.  And maybe it is.  But I don't think it is obviously
false.

Someone who thinks that I do, strictly speaking, persist over time
might say that it is obvious that I persist.  After all, I engage in
activities that take long periods of time, I remember things from long
ago, and I bear unique attitudes toward my past and future selves.  I
feel pride or regret at past actions, and anticipation or apprehension
at future ones.  How could these past and future selves not be me?

One reply begins by pointing out that, whether or not we persist in a
strict sense, the world will look the same.  I will still engage in
activities that take time; but it will not be I who completes them.  I
will still remember things from long ago; but it will not be I who
experienced them.  I will bear attitudes towards past and future
people, but those people will not, strictly speaking, be me.  But it
will {\em seem} as if they are me, and they will meet the conventional
criteria for being the referent of `me'.  As in the case of tables and
chairs, there are conventional `identity' conditions for people over
time.  Like tables and chairs, these criteria will involve causal and
spatiotemporal continuity.  What person is the referent of `Alex' a
week from now will depend on a causal chain connected to me.

Psychological continuity may also play a role.  For example, if by
some miracle I am vaporized and (coincidentally) an qualitatively
identical person is summoned into existence nearby, that person will
not, strictly speaking be me.  But it may be agreed that the person
meets the criteria for being the referent of `Alex'.  Then again, it
may not.  If this person meets these criteria, however, it will be on
account of the apparent psychological continuity between us.

The criteria for the `identity' over time of people is not fully
precise, as shown by our indecision over whether a spontaneous
duplicate of me ought to be referred to as `Alex'.  Another, more
realistic, situation in which this indecision manifests itself is in
death.  Suppose I die, and a wake is held for my body.  It is
perfectly correct for someone to point and say, ``that was Alex''.
But it is equally correct to say ``that's Alex''.  (The latter may be
more appropriate if it is necessary to identify my body.)  Is the
mereological sum that is the (deceased) body really me, or not?  If we
accept the essentialist theory, it is (strictly speaking) not, but it
may be correct or appropriate to refer to the body as `Alex'.

\section{Can the essentialist theory explain what we believe?}
\label{explain-e}
In section \ref{explain-p} I assessed whether any of the three
`plurality' theories could explain why we held beliefs that conflicted
with certain consequences of the theories.  The same assessment may
be conducted with regard to the essentialist theory I have sketched in
this section.  If the essentialist thesis is right, why do we believe
that chairs are literally identical over time?  If we want to defend
the essentialist theory, we should try to explain why we generally
seem to think that things like chairs literally persist over time, and
can change their parts over time as well.

One reply is simply to claim that we {\em don't} believe that things
literally persist over time.  When asked ``is it {\em literally} the
same chair without its leg?'' some of us may waver, and perhaps
concede that we don't think it is really the same chair.  But I doubt
this reply will convince any philosopher who has already made up her
mind about essentialism.

Another reply is that we are fooled by the great similarity between
`successive' chairs, both with regard to appearance and with regard to
their spatiotemporal location.  If we see a certain chair in the
sitting room, and while we are away it is replaced by a different
chair (someone carries one out and places another in the room), then
we will likewise be fooled by the similarities between the two, and
mistake them for one and the same.

%% Second, we are fooled by the {\em causal} relations between successive
%% chairs.  For example, if I hit the chair in the sitting room with a
%% poker and a bit of wood flies away, it is a different chair in the
%% sitting room.  By striking the one chair and sending part of it
%% flying, I cause {\em that} sum to no longer meet the criteria for
%% being the referent of `chair' (chairs are not nearly so scattered).  I
%% have also therefore caused a different sum to meet the criteria for
%% being the referent of `chair'.  I have caused us to shift our
%% reference from one sum to the other; 

\section{Lessons}
\label{lessons-e}
In section \ref{parts} I examined three different versions of the
`plurality thesis'; the view that there are pluralities of co-located
objects.  In this section I offered an alternative.  I am not sure
whether my theory or one of the plurality theses is correct, but I
suspect that it is one or the other.  My conclusion is largely the
same as that of Karen Bennett:

\begin{squote}
The only live options, then, are to be either a one-thinger or a
bazillion-thinger.  We must either think that there is only one thing per
spatio-temporal location, or else that there are lots and \emph{lots} of
spatio-temporally coincident things \citeyearpar[358]{bennett2004}.
\end{squote}

I would prefer to be a `one-thinger' because it does not commit me to
a `bazillion' things all in the same place.  That is not a decisive
objection, of course.  It may well be that such an explosion is more
plausible than certain consequences of the `one-thinger' theory.  But
I think one of the two theories must be right.

\subsection{What can we learn from Fine's theory?}
\label{need-fine}
I have argued that we can identify ordinary things like chairs as
mereological sums, and we can identify things like groups as sets.  It
is therefore not necessary to use Fine's theory of operators (section
\ref{fine-c}) to describe these things.  Is there anything we can take
away from Fine's theory?

At the very least, Fine's theory is valuable for its insight that
there are different ways of being a part.  It shows that sums and sets
both have parts, but in different ways.  It suggests that there are
also sequences, strings, words, poems, events, and quantities, each
perhaps having its parts in different ways.

Some philosophers who adhere to a more or less classical mereology
believe that physical or material things are the only things that
exist (van Inwagen is one).  For such philosophers, there is only one
way of being a part, and anything that has parts (which is everything)
is a mereological sum.  I do not share this view; I think there are
also sets, and probably other kinds of things.  I do not think,
therefore, that anything that has parts is a mereological sum.  Sets
have parts, and sets are not sums.  My theory of essentialism must
therefore operate with a definition of mereology that does not entail
that everything that has parts is a sum.  One way to ensure this is to
say that mereological sums are all and only physical things.
Appropriate qualifications may easily be added to the definitions in
section \ref{tech}.

But if everything is not a sum, if there are sets and probably other
kinds of things as well, how {\em many} kinds of things are there?  If
the essentialist theory in this section was meant to avoid many
different kinds of overlapping things, how can I allow that, in
addition to sums and sets, there might also be strings, and sequences,
and words, and poems, and an unknown number of other things?

One, perhaps minor, advantage of my theory is that it allows us to
retain at least a semblance of our pre-reflective categorization of
ordinary things.  According to Fine, chairs, statues, lumps, boats,
and kittens are all different kinds of things, occupying different
ontological categories.  According to the essentialist, they are all
the same kind of thing---they are all physical objects.  The
essentialist theory may recognize different kinds of things, but it
does not multiply kinds beyond necessity.

\subsection{Deflationary metaphysics}
\label{deflate}
Kathrin Koslicki has an interesting objection to universalist theses
such as the one I appear committed to.  Her objection amounts to this:
if every `collection' of objects (such as the London Bridge, a
particle in the moon, and Cal Ripkin, Jr.) is a thing in its own right
(a sum), then metaphysics becomes uninteresting.  There is no longer
any debate about whether chairs or dogbushes are more `real' or have a
stronger claim to existence.  They both (obviously) exist, and the
difference between chairs and lumpkins is not ontological but
conceptual: `chair' is more embedded in our talk, and so chairs have
greater importance to {\em us}.  But metaphysically, or ontologically,
chairs and dogbushes are on the same level.  There is no sense in
which chairs exist and lumpkins do not.

In the quote below, Koslicki is criticizing a version of
four-dimensionalism that Sider has previously defended.  Sider's
position was that any collection of objects-at-times is a thing in its
own right.  Sider calls these things `fusions'.  For example, a chair
is a fusion of a large number of {\em temporal part} of things (wood
molecules, or atoms, or simples).  Each thing (wood molecule, atom, or
simple) is a fusion of {\em its} temporal parts.  Each temporal part
of the chair is also a thing (a fusion).

I take no stance on whether objects have temporal parts or rather
`endure' through time.  Moreover, if we accept Fine's theory of parts
then we reject the idea that there is just one composition operation;
the operation that produces fusions is one among many.  But Koslicki's
comments are relevant nonetheless:

\begin{squote}
There is room, in Sider's theory, for {\em some} genuine ontological
disagreements: for example, the universalist, the nihilist and the
holder of the intermediary position genuinely disagree over how many
and which fusions that exist.  But the only genuine ontological
disagreements for which there is room, in Sider's world, are ones that
concern disagreements over `bare' fusions, so to speak.  What has
happened to the houses, trees, people, and cars, the familiar concrete
objects of common-sense, whose persistence this account set out to
analyze?  There are no `deep' ontological facts as to whether a given
fusion should count as a house or not\,\ldots

[By claiming that there can be genuine ontological disputes,] Sider is
guilty of a bit of false advertising: his account is really a way of
saying that, at the end of the day, there is no interesting {\em
  ontological} story to be told about the persistence of our familiar
concrete objects of common-sense; whatever there is to say about the
persistence of houses, trees, people and cars concerns the
organization of our conceptual household
\citeyearpar[124--125]{koslicki2003}.
\end{squote}

Koslicki seems to think that we ought to be able to find some
ontological difference between ``the familiar concrete objects of
common-sense'' and things like lumpkins or chairs-at-times.  But as I
remarked above (section \ref{universalism}), why should what interests
us (familiar objects like chairs) be a guide to what exists?  The
conclusion that ``the persistence [and other properties] of houses,
trees, people and cars concerns the organization of our conceptual
household'' seems to be correct.

However, there {\em is} an ontological difference between some things,
if not between chairs and dogbushes.  One lesson of Kit Fine's theory
of parts is that mereological sums may not be the only kind of
composite thing.  There are apparently sets as well, and strings, and
sequences, and perhaps many other types of thing.  The difference
between a set and a sum is probably an ontological difference, and
identifying what distinguishes sets from sums (and from other kinds of
things) is an interesting question.  The field of metaphysics is not
then so barren, as Koslicki seems to have feared.  But it is true that
many interesting questions---When are we willing to call something a
chair, and why?  What conditions must be fulfilled?---are not
ontological questions anymore.  They are questions about our
``conceptual household.''

\ifstandalone
\end{spacing}
\bibliography{everything}
\bibliographystyle{ChicagoReedweb}
\fi
\end{document}


\chapter*{Conclusion}
\label{concl}
\chapterpig{Conclusion}
\addcontentsline{toc}{chapter}{Conclusion}
\chaptermark{Conclusion}
\markboth{Conclusion}{Conclusion}
\documentclass[11pt]{article}
\usepackage{standalone} \newif\ifstandlone \standalonetrue
\usepackage[left=1.75in, right=1.75in, top=1.25in, bottom=1.25in]{geometry}
\geometry{letterpaper}
\usepackage{graphicx}
\usepackage{enumitem}
%\usepackage{amssymb}
\usepackage{amsmath}
\usepackage{epstopdf}
\usepackage{verbatim}
\usepackage{setspace}
\usepackage{natbib}
\setcitestyle{aysep={}}
\usepackage{hyperref}
\usepackage{url}
\synctex=1

\DeclareSymbolFont{symbolsC}{U}{txsyc}{m}{n}
\DeclareMathSymbol{\strictif}{\mathrel}{symbolsC}{74}
\DeclareMathSymbol{\boxright}{\mathrel}{symbolsC}{128}

\newenvironment{squote}{%
\begin{spacing}{1}
\begin{list}{}{%
\setlength{\labelwidth}{0pt}%
\rightmargin\leftmargin%
}
\item\relax
}{%
\end{list}%
\end{spacing}
}

\title{The End: Oh Make it Stop}
\author{Alexander A. Dunn}
\begin{document}
\ifstandalone
\maketitle
\begin{spacing}{1.5}
\fi

I have argued for a number of claims in the preceding sections.

First, debates in metaphysics such as the one I have been engaged in
are conducted in English (or French, or German) and not in
``Ontologese'' or some other pseudo-language.  If it is ``really'' or
``fundamentally'' the case that there are no chairs, then ``there are
no chairs'' is true in English.

Second, philosophers who deny that there are chairs have some
difficulty explaining why we nonetheless believe that there are
chairs.  Van Inwagen's explanation fails outright.  Trenton Merricks
claims that because things arranged chairwise matter to us, we have
introduced the word `chair' to refer to them; we are fooled by the
singular nature of the word `chair' and come to think that there is
some single {\em thing} that we are referring to, when in fact there
is not.  This explanation, however, is equally compatible with
universalism: the claim that for every set of things, there is some
other thing they compose.  And universalism is a much more plausible
thesis than the nihilism of Merricks.

Third, if we assume that universalism is true then we have a choice to
make.  We can either adopt a ``plurality theory'' that posits a huge
number of things (and possibly different {\em kinds} of things), or we
can adopt a version of essentialism, maintaining that, strictly
speaking, things do not change their parts over time.  I have
suggested that the essentialist theory avoids some of the excesses of
co-location that plague the plurality theories while offering some
neat solutions to problems of personal identity over time.  But
neither route is obviously superior, and both are defensible.\\

In the Introduction and in Section \ref{stroud}, I emphasized that my
opposition to metaphysical nihilism was based, largely, on the fact
that it is {\em obviously true} that there are chairs.  I claimed to
be arguing for what is clearly so, and rejecting what is clearly not.

But surely, mereological essentialism is not {\em obviously} true.
Some would say it is obviously false.  In either case, I can no longer
claim to be arguing for what is clearly so.

But not everything is clear in metaphysics.  ({\em This} is obviously
true.)  There are a few things that are obviously true; many other
things are not, but they are no less true.  It is obvious that there
are chairs; given that, what are they like?  What sort of thing are
they?  Can they change their parts?  If these questions have answers,
they are not obvious.

Moreover we seem to be forced to choose between two possibilities,
both of which might be decried as obviously false.  If there are
chairs, and all the other things that universalism entails, then
either things change their parts or they do not.  If things change
their parts, then (again ignoring four-dimensionalism) there must be
very many co-located things.  Someone who takes this to be false may
be forced to conclude that things do not change their parts.

This thesis therefore ends with no fully-formed theory.  I have
offered a disjunction: either a plurality theory or an essentialist
theory is correct. I have no decisive intuitions here.  I appreciate
the minimalism of the essentialist solution, which dissolves problems
of persistence and identity over time.  But I recognize its
strangeness, and see also the strengths of theories that posit
pluralities of things.

\ifstandalone
\end{spacing}
\bibliography{everything}
\bibliographystyle{ChicagoReedweb}
\fi
\end{document}

%	\setcounter{chapter}{4}
%	\setcounter{section}{0}
	
%If you feel it necessary to include an appendix, it goes here.
%    \appendix
%      \chapter{The First Appendix}
%      \chapter{The Second Appendix, for Fun}


%This is where endnotes are supposed to go, if you have them.

  \backmatter % backmatter makes the index and bibliography appear
              % properly in the t.o.c...

\bibliographystyle{chicago}
\bibliography{everything}

\end{spacing}
% Finally, an index would go here\,\ldots\,but it is also optional.
\end{document}
