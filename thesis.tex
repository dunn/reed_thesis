% This is the Reed College LaTeX thesis template. Most of the work 
% for the document class was done by Sam Noble (SN), as well as this
% template. Later comments etc. by Ben Salzberg (BTS). Additional
% restructuring and APA support by Jess Youngberg (JY).
% Your comments and suggestions are more than welcome; please email
% them to cus@reed.edu

\documentclass[12pt,twoside]{reedfancy}
\usepackage{standalone}
\usepackage{anyfontsize}
\usepackage{graphicx,latexsym} 
\usepackage{amssymb,amsthm,amsmath}
\usepackage{tipa}
\usepackage{longtable,booktabs,setspace} 
\usepackage{verbatim}
\usepackage{url}
\usepackage{natbib}
\usepackage{enumitem}
\usepackage{url}
\usepackage{hyperref}
\setcitestyle{aysep={}}
\synctex=1

\DeclareSymbolFont{symbolsC}{U}{txsyc}{m}{n}
\DeclareMathSymbol{\strictif}{\mathrel}{symbolsC}{74}
\DeclareMathSymbol{\boxright}{\mathrel}{symbolsC}{128}

\newenvironment{squote}{%
	\begin{spacing}{1}
	\begin{list}{}{%
	\setlength{\labelwidth}{0pt}%
	\rightmargin\leftmargin%
	}
	\item\relax
	}{%
	\end{list}%
	\end{spacing}
	}

\newcommand{\stager}[4]%
{%
	\begin{spacing}{1}%
	\vspace{0pt}
		\begin{description}[style=nextline, noitemsep,
                    parsep=0pt, topsep=0pt, leftmargin=15mm,
                    itemindent=-10mm, font=\mdseries]
			\item[\textsc{#1} \emph{#2}] #3
			\item[]%
			\begin{flushright}#4\end{flushright}
		\end{description}%
	\end{spacing}%
}

\newcommand{\stage}[3]%
{%
	\begin{spacing}{1}%
	\vspace{0pt}
		\begin{description}[style=nextline, parsep=0pt,
                    leftmargin=15mm, itemindent=-10mm, font=\mdseries]
			\item[\textsc{#1} \emph{#2}] #3
		\end{description}%
	\end{spacing}%
}

\newenvironment{inq}{\vspace{0pt}%
	\begin{list}{}{%
	\setlength{\labelwidth}{0pt}%
	\setlength{\leftmargin}{2.5\oddsidemargin}%
	\setlength{\rightmargin}{\leftmargin}}
	\begin{spacing}{1}
	\item[]%
	}{
	\end{spacing}
	\end{list}
	\vspace{10pt}
	%\noindent%
	}
	
\newenvironment{epigram}{%
	\begin{minipage}[c]{0.75\textwidth}
	\vspace{2.5in}
	\begin{spacing}{1}
	\begin{list}{}{%
	\setlength{\labelwidth}{0pt}
	\setlength{\leftmargin}{1.4in}
	\setlength{\rightmargin}{.25in}}
	\item[]
	}{%
	\end{list}
	\end{spacing}
	\end{minipage}
	\newline
	}

% moved to reedfancy.cls:
%\def\thetitle{oink}
%\renewcommand{\firstmark}{\thetitle}
%\newcommand{\chapterpig}[1]{\def\thetitle{#1}}

\title{The Metaphysics of Ordinary Things}
\author{Alexander A. Dunn}
\date{May 2012}
\division{Philosophy, Religion, Psychology, and Linguistics}
\advisor{Paul Hovda}
\department{Philosophy}

\setlength{\parskip}{0pt}

%%%%%%%%%%%%%%%%%%%%%%%%%%%%%%%%%%%%%%%%%%%%

\begin{document}

  \maketitle
  \frontmatter % this stuff will be roman-numbered
  \pagestyle{empty} % this removes page numbers from the frontmatter

\begin{spacing}{1.25}

\chapter*{Acknowledgments}
Paul Hovda has patiently worked with me all year.  If there is
anything in this thesis that is true, it is thanks to him.  If the
rest is at least false, that too is due to his help.

Andrew Dunn and Stuart Sherwin gave me feedback on drafts, helping me
clarify arguments and eliminate errors.

A short conversation with Nat Tabris helped me see better the
structure of Section~\ref{parts}, which resulted in a much clearer
argument.

Lauren Cary and Joe Arriaga looked after me while I recovered from a
back injury; without them I may not have finished this thesis.

My parents, of course, are the ones who gave me this opportunity in
the first place.

%%%%%%%%

\tableofcontents

%%%%%%%%

\chapter*{Abstract}
Theories of metaphysical nihilism claim that there are no (or nearly
no) objects with parts: no chairs, houses, mountains, and perhaps even
no people.  The philosophers who make these claims have trouble
explaining why we nonetheless believe there are such things.  The only
successful explanation is compatible both with nihilism and with
metaphysical universalism.  Universalism claims that for every set of
things, there is something else made up of those things; this thesis
is intuitively more plausible than nihilism.  But if we assume that
universalism is true, and if we do not presuppose four-dimensionalism,
we have to choose between two unintuitive versions of universalism:
one that posits a plurality of co-located (entirely overlapping)
objects, or one that denies that things can change their parts.

\mainmatter % here the regular arabic numbering starts
\pagestyle{fancyplain} % turns page numbering back on

\fancyhead[CE]{\textit{\thetitle}}%
\fancyhead[CO]{\textit{\thetitle}}
\renewcommand{\headrulewidth}{0.0pt}

\chapter*{Introduction}
\chapterpig{Introduction}
\addcontentsline{toc}{chapter}{Introduction}
\chaptermark{Introduction}
\markboth{Introduction}{Introduction}
It is true that there are chairs.  This, as far as I'm concerned, is
obvious.  If someone denies that there are chairs, then it seem to me
that somehow they have gone astray.  If they have an argument for this
conclusion, there must be something wrong with the argument.  There
must be something wrong because it is true---obviously true---that
there are chairs.

In many parts of what follows, I will often say things such as ``I
believe that there are chairs''.  This is not due to an unwillingness
to assert the stronger claim---that there are chairs.  The stronger
claim is, I have said, obviously true.  But I will use the weaker
claim---``I believe that there are chairs''---because while the
philosophers I am criticizing deny that there are chairs, they do not
deny that I believe that there are chairs.  And this fact alone---that
I believe that there are chairs---causes some trouble for their views,
and gives us reason to doubt their extraordinary conclusions.

So I believe that there are chairs.  I also believe that there are
desks, and desk lamps, and doors, and doorways, and houses, and
gardens, and plants.  Such things, and many others, are commonly
referred to as ``ordinary things''.  This phrase is extremely vague in
its application, but may be taken to designate macroscopic objects,
such as those listed above, that are parts of our everyday lives.

Many philosophers have denied that ordinary things exist.  Until
recently, such a denial was generally a consequence of the
philosopher's views on other matters.  If a philosopher claims that
there is no external world, or that the world is not at all like it
appears, then she might deny that there are any physical things, or
any things that exist outside the mind, or anything at all.  It
follows from such a claim that there are no ordinary things like
chairs.  But such a philosopher is not specifically interested in
denying that chairs exist.  She is interested in denying that {\em
  anything} exists; the denial of chairs is a minor consequence.

In the past 30 years, however, certain philosophers who we will refer
to collectively as {\em nihilists} have constructed arguments
specifically designed to show that there are no ordinary things.
(Peter Unger was one of the first, with the aptly titled paper,
``There are no ordinary things''.)  These philosophers do not deny
that there is an external world, or that it contains many physical
things; these propositions are readily granted to be true.  But they
are unwilling to admit that such a world does---or even possibly
could---contain chairs.

Most philosophers making this sort of claim admit that it is strange
and unintuitive.  But they believe that the benefits of denying the
existence of ordinary things outweighs the costs.  Different
philosophers cite different benefits: consistency with regard to our
notion of composition, theoretical simplicity, or greater coherence
with our other beliefs.

The benefits do not outweigh the costs.  Moreover, I am unable to
imagine that any argument could convince me that there are no ordinary
things.  I believe that any argument that has the nonexistence of
chairs as a consequence is flawed.  Whether or not we can immediately
identify the flaw in the argument, the fact that it entails a
falsehood shows that something has gone amiss.

It will be objected that this is merely a fact about myself; other
philosophers are perfectly willing to deny that there are chairs.  It
may be argued that since I consider `There are chairs' to be true no
matter what, I must consider it to be some sort of conceptual truth.
It may be further argued that, since there are philosophers willing to
deny that `There are chairs' is true, what I mean by `There are
chairs' is something different than what these philosophers mean by
`There are chairs'.  We may be thought to be using our words in
different ways.

In Section \ref{verbal} I will argue that we are {\em not} using our
words in different ways.  When I say ``There are chairs'' and someone
else says ``There are not chairs'', we are having a real disagreement.
Moreover, we are disagreeing in English; there is no special
``ontological language'' in which we do metaphysical philosophy.

In Section \ref{stroud} I will argue that any philosopher who attempts
to deny that there are chairs should be able to explain why we
nonetheless believe that there are chairs.  This seems to be a
reasonable request, but it is surprisingly hard to satisfy.  The
difficulties that nihilistic philosophers have in explaining why we
believe that there are chairs should make us suspicious of their
conclusions.

But even if we show that there are problems with the arguments of
philosophers who deny that ordinary things exist, we have not thereby
proved that they {\em do} exist.  The nihilistic philosophers who deny
that there are chairs are motivated to do so by a number of puzzles
about the nature of ordinary things.  For example, why are there
chairs and tables, but not chair-tables (single objects composed of an
adjacent table and chair)?

In Section \ref{universe}, however, I will argue that some of the
considerations that philosophers take to be good reasons to deny that
chairs exist are not good reasons at all.  In effect, these
philosophers take the apparent non-existence of chair-tables to tell
against the existence of chairs and tables.  On the contrary, as we
will see, the obvious existence of chairs and tables tells {\em for}
the existence of chair-tables, dogbushes, and other strange things.

In Section \ref{parts} I will examine three theories that seek to make
sense of all these different objects.  I will argue that all have the
consequence that there are a {\em plurality of overlapping
  objects}---that where we might think there is just one thing (a lump
of clay), there are actually millions or more.  I will suggest that
this unwelcome consequence should encourage us to look for a different
sort of theory.

In Section \ref{essential} I will attempt to defend the claim that
ordinary things are {\em mereological sums} which cannot change their
parts.  What we take to be a chair with a new leg, for instance, is
really a new chair.  The most interesting consequence of this is that
the ``persistence conditions'' for things over time---the conditions
in which a certain thing is the referent of a term like `the
Washington Monument'---are wholly conventional.

Throughout this thesis, there are certain things I will {\em not}
presuppose.  First, I will take no stand on whether or not things have
{\em temporal parts}.  I am not sure I fully understand the doctrine
of temporal parts, but it is often summarized thus: if a thing has
temporal parts, then for each time at which it exists, there exists at
that time (and only at that time) another thing---a temporal part or
``slice'' of the larger object.  The (temporally) larger object is
somehow ``built up'' from these temporally smaller parts.  If a thing
does not have temporal parts, then it is not divided into temporal
slices---it is ``wholly present'' at every moment of its existence.
Whatever this debate comes to, I will try to avoid relying on the
truth or falsity of the doctrine of temporal parts.

Second, I will not presuppose {\em eternalism}.  Eternalism is,
roughly, the view that past and future time are just as ``real'' as
the present.  An analogy is often drawn with space; what's behind me
and in front of me is just as real as what is under me.  There is
nothing special about {\em here} rather than {\em there}.  Likewise
the eternalist claims that {\em now} is no more special that {\em
  then}.  Eternalism is generally opposed to {\em presentism}, which
is the view that only the present is real.  The presentist and the
eternalist both agree that there {\em were} dinosaurs, but for the
eternalist there is a sense in which there {\em are} dinosaurs (they
just don't exist now).  I will attempt to avoid committing myself to
the reality of anything but the present.


\chapter{What do I mean when I say there are chairs?}
\chapterpig{What do I mean when I say there are chairs?}
%% return to normal
\fancyhead[CE]{\textit{Section\ \thechapter}}%
\fancyhead[CO]{\textit{\thetitle}}
\renewcommand{\headrulewidth}{0.0pt}
\label{verbal}
Philosophers such as Eli Hirsch have argued that metaphysical
disputes---such as whether there are chairs---are verbal disputes that
have no real import.  Hirsch claims that `There are chairs' is
obviously true in English.  If a philosopher says ``There are no
chairs'' as part of her claim that there is nothing in the world but
partless atoms, Hirsch will {\em interpret} them as meaning something
like `there are no partless atoms that are also chairs'.  This
interpretation allows Hirsch to maintain that both philosophers are
saying true things, and are not really disagreeing at all.  Ted Sider
has replied by attempting to invent a new language, Ontologese, in
which it is not obvious that `There are chairs' is true.  In this
section I will argue that Hirsch's argument relies upon a
controversial theory of meaning, and that Sider's response, valid or
not, is unnecessary.

\section{Verbal disputes}
\label{hirsch}
Some philosophers maintain that the recent disputes over the existence
of ordinary things are {\em merely verbal disputes}.  Suppose one
philosopher claims that nothing is part of something else.  This
philosopher will say things like ``It is not true that there are
chairs'' Suppose another philosopher rejects this view.  This
philosopher will say ``There are chairs''.  It appears that these
philosophers are disagreeing.  But according to some, this appearance
is an illusion.

Eli Hirsch claims that our two philosophers are engaged in a {\em
  verbal dispute}.  A verbal dispute is one that is somehow not
substantive; Hirsch's paradigm case of a verbal dispute is over
whether glasses are cups:

\begin{squote}
  I know someone, whom I'll call $A$, who claimed that a standard
  drinking glass is a cup.  ``Just as a cat is a kind of animal,'' she
  said,``a glass is a kind of cup.''  Everyone else whom I've asked
  about this agrees with me that a glass is not a cup.  Clearly, this
  dispute is, in some sense, merely about
  language~(\citeyear[69]{hirsch2005}).
\end{squote}

To see that this dispute is verbal, Hirsch instructs us to do the
following:

\begin{enumerate}
  \item Take what each disputant says.
  \item Postulate a community that agrees with that disputant.
  \item Interpret each community's language so that the relevant
    utterances come out true.
  \item Interpret each disputant as speaking the language of their
    community.
\end{enumerate}

For example, when $A$ says ``There is a cup on the table'', Hirsch
would say instead ``There is a cup or there is a glass on the table''.
Postulating a community that agrees with $A$ (the $A$-community) and
imagining a community that agrees with Hirsch (the $H$-community), we
may assign the truth-conditions this way:
\begin{enumerate}[itemindent=25pt, label=(T)]
    \item ``There is a cup on the table'' is true in $A$-English if
      and only if ``There is a cup or glass on the table'' is true in
      $H$-English
\end{enumerate}
($A$-English is the dialect of English spoken in the $A$-community and
$H$-English is the dialect spoken in the $H$-community.)

One might object that we have not shown that $A$ and $H$ are in fact
speaking different languages.  ``All you have shown,'' $H$ might say,
``is that we can imagine $A$ speaking a language in which what she
says is true.  But you have not shown that she {\em is} speaking such
a language.  What $A$ says sounds like normal English to me, and in
English, what she says is simply false.''

What justifies us in postulating $A$- and $H$-English is simply that
what $A$ {\em means} by `cup' is not what $H$ means by `cup'.  $A$
uses `cup' to refer to all the things that $H$ uses `cup' to refer to,
but she also uses `cup' to refer to those things that $H$ refers to
exclusively by `glass'.  What $A$ means by `cup' is what $H$ means by
`cup or glass'.

It is this difference in {\em meaning}, as well as in
truth-conditions, that allows us to postulate $A$-English and
$H$-English and to conclude that the dispute between $A$ and Hirsch is
verbal. If we understand $A$ to mean by ``There is a cup on the
table'' what we mean by ``There is a cup or a glass on the table'',
then $A$ is not saying something false.  We thought that they were
disagreeing over what was on the table, but they simply meant
different things by their words.

Hirsch does not explicitly claim that meaning is reducible to
truth-conditions, but he is clearly relying on a close connection
between the two:

\begin{squote}
When I speak throughout this paper about interpreting a language this
is always to be understood in the narrow sense of assigning truth
conditions.  I leave it open what there is to understand a language
beyond knowing the truth conditions of its sentences, but, whatever
this additional element may be, it will have a bearing on my argument
only insofar as it might affect the plausibility of certain
truth-condition assignments \citeyearpar[72]{hirsch2005}.
\end{squote}

Having given this warning, Hirsch speaks freely of meaning instead of
mere truth-conditions.  When imaging himself as David Lewis
interpreting Roderick Chisholm, he suggests that we ``reject the
assumption that the RC-speakers [Roderick Chisholm's ``community
  language''] mean what we [speakers of the David Lewis language]
mean'' \citeyearpar[76]{hirsch2005} and advocates ``{\em semantically
  restricted quantifiers}'' \citeyearpar[76, his
  emphasis]{hirsch2005}.  When discussing these ``RC'' quantifiers, he
goes on to say this:

\begin{squote}
The RC-speakers will, of course, make the platitudinous disquotational
assertion, ``If something exists it is referred to by the word
`something'.''  Given what they {\em mean} by `something' this
sentence is trivially true \citeyearpar[77, my emphasis]{hirsch2005}.
\end{squote}

Without committing Hirsch to exactly the following thesis, I think he
would accept some claim along these lines: {\em if it is necessary
  that propositions $p$ and $q$ have the same truth-value (either true
  or false), then $p$ and $q$ mean the same thing}.  Hirsch seems at
least sympathetic to some modification of this.  We can express it
more formally as \ref{v}:
\begin{enumerate}[itemindent=25pt, label=(M)]
    \item If $\square$($p$ is true if and only if $q$ is true), then
      $p$ and $q$ mean the same thing. \label{v}
\end{enumerate}

This is a stronger thesis than Hirsch needs to accept.  Moreover, it
is probably not true; it seems to entail that there is only one
necessary proposition.  But {\em something like this} underlies
Hirsch's argument.

\subsection{Charity}
\label{charity}
But Hirsch's conclusion that $A$ means by `cup' what he means by `cup
or glass' does not follow from~\ref{v} alone.  To see why this is so,
recall the dispute between $A$ and $H$ over whether a glass is a cup.
Suppose that $H$ accepts~\ref{v}.  He might say, ``We are both
speaking English.  In English, `cup' does not mean the same as `cup or
glass'. Therefore, by {\em modus ponens}, it is not necessarily true
that `There is a cup on the table' is true if and only if `There is a
cup or glass on the table' is true.  For when there is a glass on the
table, the latter proposition is true and the former false.  And yet
$A$ insists on treating these as somehow identical.  She affirms one
if and only if she affirms the other.  Evidently, she is deeply
confused.''

$H$ could accuse $A$ of making fundamental mistakes about language or
perception, and $A$ could level the same accusation at $H$.  But
Hirsch thinks that this is a poor way of understanding the
debate.  Instead of supposing that ``the other has some incurably
irrational tendency to make a priori mistakes about what they perceive
in front of their faces'' \citep[78]{hirsch2005}, we should pursue a
policy of {\em interpretive charity}:

\begin{squote}
Why is it plausible to suppose that in the $A$-language the word `cup'
doesn't mean what it means in our language, so that the sentence `A
glass is a cup' is true in that language?  The basic answer to this
question comes out of a widely accepted principle of linguistic
interpretation that has often been called the ``principle of
charity''.  This principle, put very roughly, says that, other things
being equal, an interpretation is plausible to the extent that its
effect is to make many of the community's shared assertions come out
true or at least reasonable (\citeyear[71]{hirsch2005}).
\end{squote}

We can see the correctness of this principle by imagining a resolution
to the dispute between $A$ and $H$.  Any neutral arbitrator should sit
them down and explain things thus: ``Now $A$, you said that just as a
cat is a kind of animal, a glass is a kind of cup.  The set of glasses
is a {\em subset} of the set of cups.  $H$, you probably disagree; you
think cups and glasses are like cats and dogs---the set of one is {\em
  not} a subset of the other.  But given that $A$ thinks of cups and
glasses like she does, you should remember when she says `cup', that
she just means anything that you'd call either a cup or a glass.  And
$A$, when $H$ talks about cups, remember that he means only the cups
that aren't glasses.''

Unless $A$ and $H$ are simply looking for something to bicker about,
they will agree that they each mean these different things by `cup';
having recognized this, the argument dissolves.  The only question
remaining is which meaning is shared by the majority of English
speakers~\citep[70]{hirsch2005}.

Hirsch diagnoses verbal disputes by applying his principle of
interpretive charity alongside a version of~\ref{v}.  If he can
interpret the propositions of two disputants so as to make all come
out true, and if these equivalences in truth-conditions correspond
with equivalent meanings, then Hirsch has shown a dispute to be
verbal.  Unfortunately, while this method works well for his test case
involving $A$ and $H$, it does not appear to succeed when applied to
the metaphysical disputes that are his primary subjects.  He does
manage to interpret the apparently conflicting propositions of the
competing metaphysicians so that neither contradicts the other;
however, his truth-conditional interpretations fail to preserve
meaning.

Consider Hirsch's analysis of the dispute between a
four-dimensionalist and a mereological essentialist.  Hirsch uses
David Lewis and Roderick Chisholm as mascots for these respective
positions.  We are to suppose that Chisholm ($RC$) and Lewis ($DL$)
are sitting at a table.  Upon the table is a pencil. $DL$ claims that
objects have temporal parts, and that any set of objects and/or
temporal parts has a {\em fusion} (in other words, for any set of
objects and/or temporal parts, there is another object composed of the
things in that set).  $RC$, on the other hand, claims that objects do
not have temporal parts (there are no such things); the only physical
objects are masses of matter.

$DL$ and $RC$ obviously have different things to say about the pencil
on the table.  $DL$ claims that a temporal part of the eraser from
$t_{1}$ fuses with a temporal part of the wood from $t_{2}$; thus $DL$
says that ``There is something on the table that is pink, then brown.
$RC$ denies this asserting that ``There is nothing on the table that
is pink, then brown''.  Both, however, will say that ``There is
something that is pink, then there is something that is brown''.

Hirsch imagines himself as $DL$ trying to interpret $RC$, and then as
$RC$ interpreting $DL$.  He claims that from the point of view of
$DL$, the quantifiers in $RC$-English are {\em semantically
  restricted}; ``the rough idea seems to be that the range of the
$RC$-quantifiers excludes any physical object that is composed of
matter but is not itself a mass of matter'' \citep[76]{hirsch2005}.

Hirsch then adopts the perspective of $RC$.  He finds that speakers of
$DL$-English consider the sentence `There is first something that is
$F$ and later there is something that is $G$' to be ``(a priori
necessarily) equivalent'' to `There is something that is first $F$ and
later $G$'.  He says that ``we should make the charitable assumption
that in $DL$-English these sentences really are
equivalent'' (\citeyear[78]{hirsch2005}).

Given the mereological axioms that $DL$ has adopted, it is
uncontroversially true that whenever there is something that is pink,
then something that is brown, they fuse to create something that is
first pink and then brown.  Given $RC$'s doctrines, it is also
true that everything (every physical thing) he claims to exist is a
mass of matter.

Having completed his ``charitable'' interpretation, Hirsch applies his
version of~\ref{v} and claims that these truth-equivalent propositions
{\em mean} the same thing.  He claims that $DL$ means the same thing
by `There is first something that is $F$ and later there is something
that is $G$' and by `There is something that is first $F$ and later
$G$.'  He also claims that $RC$ uses `something' to mean `something
that is either a mass of matter or is not composed of matter'
(\citeyear[76]{hirsch2005}).

Hirsch concludes that when $DL$ says ``There is something here that is
first pink and then brown'', he should be taken to mean that there is
something that is pink and then something that is brown.  And $RC$ can
agree with that.  Hirsch also claims that when Chisholm says that
``There is nothing here that is first pink and then brown'' he means
that there is no mass of matter that is first pink and then
brown.  $DL$ will not deny that.  So Hirsch concludes that $DL$ and
$RC$ are engaged in a verbal dispute; they are simply talking past
each other.

Hirsch's analysis of $DL$ is dubious.  $DL$ will of course admit that
these sentences are truth-conditionally equivalent, but we can imagine
him saying ``Do they mean the same thing?  Well, no.  The second
proposition---`There is something that is first pink and later
brown'---entails that there is one thing that is pink then brown; the
first---`There is first something that is pink and later there is
something that is brown'---does not (in fact, it suggests that they
are not the same thing).''

If that seems dubious, Hirsch's analysis of $RC$ seems downright
false.  ``Does `something' {\em mean} `something that is a mass of
matter or not composed of matter'?''  $RC$ might ask.  ``Of course
not!  If I meant that, then by `There is not something that is not a
mass of matter' all I would mean would be `There is not something that
is a mass of matter that is not a mass of matter'.  That's trivially
true, and thoroughly uninteresting.  But I'm not speaking in
tautologies; I'm expounding a controversial metaphysical thesis;
namely, that {\em everything that exists is either a mass of matter or
  is not composed of matter}.  Only after having done some rigorous
metaphysics can we affirm that `Something exists' is true if and only
if `Some mass of matter or immaterial object exists'.  That claim
reports a discovery about the world, not about what I mean by my
words.''

\subsection{Hostile interpretations}
\label{hostile}
Hirsch's claim, that ontological disputes like the above are merely
verbal, relies on a controversial theory of meaning.  If a
truth-conditional theory of meaning is correct (or largely so), then
Hirsch's interpretations of the disputing metaphysicians would also be
correct.  But there would still be a sense in which the verbal
disputes as to whether there are chairs differs from the disputes as
to whether glasses are cups.  Above I said that two people arguing
over whether glasses are cups will agree that they do not mean the
same thing by `cup'.  They will agree that they are engaged in a
verbal dispute.

The metaphysicians are not so cooperative.  Even after Hirsch has
diagnosed their dispute as verbal, the disputants maintain that they
are {\em not} engaged in a verbal dispute.  Hirsch's interpretations
are therefore \emph{hostile}.  They diverge substantially from the
expectations of the speakers.  Even if our ontologists tell Hirsch
``we're \emph{not} engaged in a verbal dispute'', Hirsch will be
unmoved:

\begin{squote}
The presumption of charity is supposed to be an a priori principle
that is partially constitutive of linguistic meaning.  Insofar as the
disputing ontologists assert the sentence, ``We are not engaged in a
verbal dispute,'' this sentence will figure, together with all the
other asserted sentences, in arriving at the most charitable
interpretation.  I would suspect that meta-level, quasi-technical
(self-aggrandizing) assertions probably have low priority as
supplicants for charity.  In any case, it can't be seriously suggested
that the charitable presumption in favor of the correctness of this
one assertion threatens to trump the presumption in favor of all of
the other assertions made by the ontologists
\citeyearpar[515]{hirsch2008}.
\end{squote}

As long as Hirsch can produce truth-condition assignments that make
the relevant assertions of both sides true, there seems to be nothing
they can do to convince him that they are having a real argument
(other than convince him that his truth-conditional theory of meaning
is mistaken).

\section{Ontologese}
\label{ontologese}
Or isn't there?  Sider's strategy for dealing with Hirsch is to
stipulate---in concert with other metaphysicians---that they use
quantifier phrases like `there are' and `there exist' in a special
sense:

\begin{squote}
{[}The philosophers{]} should stipulate that their quantifiers are to be
understood as theoretical terms (and so are not subject to the same
level of metasemantic pressure from charity that governs terms like
`sofa' and `game') that stand for whatever joint-carving notion is in
the vicinity (\citeyear[9]{sider2011b}).
\end{squote}

By explicitly \emph{intending} to mean by their quantifiers whatever
``joint-carving'' notions are ``in the vicinity'', Sider hopes to
evade Hirsch's argument from interpretive charity.  The idea is that
charity can be put on hold.  Sider hopes that when {\em he} says
``There are no chairs'', it will be true (if it is true) only because
a ``joint-carving'' quantifier does not range over chairs.

Whether valid or not, what is curious about this response is that it
involves {\em conceding that Hirsch's theory of meaning is correct}.
Sider implicitly grants that some sort of truth-conditional theory of
meaning is true.  There is no other reason why Sider should feel the
need to build a language with the specific purpose of avoiding
interpretive charity.  As I pointed out above, Hirsch's
conclusion---that metaphysical disputes are verbal---requires not only
a principle of interpretive charity, but a truth-conditional theory of
meaning---something like \ref{v}.

A principle of charity alone cannot secure Hirsch's conclusion.  When
Hirsch charitably interprets the metaphysicians, it is because his
``charitable'' interpretations involve only the assignment of
truth-conditions that he can claim that the metaphysicians mean
different things.  We saw above that Hirsch recommends David Lewis and
other universalists to interpret Roderick Chisholm as using
`something' to mean `something that is a mass of matter or not
composed of matter'.  This is a bizarre and uncharitable
interpretation, {\em unless} we assume a truth-conditional theory of
meaning.

I do not subscribe to a truth-conditional theory of meaning.  There
are a number of powerful arguments against such theories, and I do not
know how to argue against them.  (For example, they entail that there
is only one necessary truth, and that `I have a cat' means `I have a
cat and $2 + 2 = 4$'.)  I therefore reject Hirsch's motivation for
interpreting metaphysical utterances as being spoken in different
languages.  I think it is most charitable for the philosophers to
interpret each other as speaking English.  I have (or so I claim) been
writing in English this whole time.  When I said ``There are chairs'',
that was part of an English sentence.  If `There are chairs' is true
in English, then there are chairs.  That is, the truth condition for
`There are chairs'---what makes `There are chairs' true, if it is
true---is that there are chairs.

This seems perfectly obvious, but it is something that both Hirsch and
Sider reject.  Without explicitly endorsing Hirsch's theory of
meaning, Sider sometimes tries to motivate the idea that English
quantifiers are ``nonfundamental'' by claiming that ``There are
chairs'' might not mean that there are chairs
\citeyearpar[171]{sider2011d}.  He suggests that instead ``There are
chairs'' might mean that there are things arranged chairwise.  He
thereby makes room for a notion of ``fundamental quantification'';
only by using ``fundamental terms'' like $\mathcal{F} ( \exists x Cx
)$ can we express the proposition that there are chairs (where
`$\mathcal{F}$' means `it is fundamentally the case that').

But this notion of fundamental quantification is unnecessary and even
unintelligible if we reject the idea that `There are chairs' doesn't
mean that there are chairs.  The only reason we would suppose in the
first place that `There are chairs' means anything other than that
there are chairs is if we held a truth-conditional or otherwise
nonstandard theory of meaning.  If we do not subscribe to such a
theory, then we should naturally suppose that `There are chairs' means
that there are chairs; any sort of appeal to a more fundamental or
``joint-carving'' kind of quantification becomes wholly mysterious.

\section{Aren't English quantifiers restricted?}
\label{eng-quant}
One might claim that there is another sort of precedent for what Sider
is doing.  It is sometimes supposed that English quantifier phrases
like `there is' and `all' are {\em semantically restricted}.  A
``fundamental quantifier'' would be {\em un}restricted.  The idea of
unrestricted quantifiers is perfectly sensible, so I cannot claim that
Sider's proposal is ``wholly mysterious''.

There is some temptation to think that ordinary uses of quantifier
phrases like `there is' are somehow restricted.  Suppose I am having a
party and you say ``There is no beer''.  One might think that here
`there is' is restricted to my house; you are quantifying only over
objects in the building.  The conclusion is then drawn that English
quantifiers are therefore not ``fundamental'', and that there is a
need to postulate a {\em totally unrestricted quantifier} that ranges
over {\em absolutely everything whatsoever}.

But as a counterexample to this thought, consider the following
exchange:

\stage{You}{}{There is no beer.}

\stage{Me}{}{I'll go get more.}

\stage{You}{}{Aren't you listening?  There is no beer.}

\stage{Me}{}{Anywhere?}

\stage{You}{}{{\em There is no beer}.}

\stage{Me}{}{Oh my.  I thought you just meant that there is no beer in
  the house.}

The philosopher claiming that English quantifiers are restricted would
have us believe that you are actually {\em saying different things}
each time you say ``There is no beer''.  But that does not seem to be
true.  What you {\em say} is the same.  What I take you to {\em mean}
is different.  The philosopher who claims that English quantifiers are
restricted is confusing saying and meaning.  If I say that there is no
beer, and there is beer (say, at the corner store), then {\em what I
  say} is false.  But in {\em most} cases, {\em what I mean} is that
there is no beer in the house.

The utterance ``There is no beer'' and the example below illustrate
what Kent Bach calls {\em expansion}:

\stage{Mother}{(treating her child's cut)}{Be quiet, you're not going
to die.}

What is said in both cases is a complete proposition, but the speaker
means something {\em more} than what she says; ``where expansion is
involved, what is meant is closely related to what is said\,\ldots but
is not identical to it'' \citep{bach1994}.  These are therefore cases
of nonliterality; what is meant is not what is said.  In Bach's
example,

\begin{squote}
the mother is using each of her words literally but is omitting an
additional phrase that could have made what she meant fully explicit.
If her son had replied, ``You mean I'm going to live forever, Mom?'',
it would not be because she was being obscure but because he was being
obtuse---he would be taking her utterance strictly and literally, not
as she meant it \citeyearpar{bach1994}.
\end{squote}

We can represent what is implicit in brackets like this:

\stage{Mother}{(treating her child's cut)}{Be quiet, you're not going
to die \{from that cut\}.}

\noindent What is implicit depends on the context, and may even change
within a short stretch of dialogue.  For example, the exchange between
the mother and child might continue thus:

\stage{Mother}{(treating her child's cut)}{Be quiet, you're not going
  to die \{from that cut\}.}

\stage{Scientist}{(runs in)}{I've done it!  I've discovered the secret
  of immortality!  This child will be the first to be cured of Death!}

\stage{Mother}{(to her child)}{You're not going to die \{ever\}!}

A parallel treatment may be given of the ``There is no beer'' example:

\stage{Me}{}{There is no beer \{here\}.}

\stage{Lauren}{(enters)}{Prohibitionists have destroyed all the beer
  everywhere!}

\stage{Me}{(horror-struck)}{But then\,\ldots there is no beer
  \{anywhere\}!}

The first time I say that there is no beer, what I mean is that there
is no beer in the house.  The second time I say that there is no beer,
what I mean is that there is no beer.  What I say does not change, but
what I mean does.

\section{Lessons}
\label{lessons-verb}
Henceforth I will assume that the debate over whether there are chairs
is conducted in English.  But even some philosophers who agree to this
will deny that ``There are chairs'' is a conceptual truth.  Nor will
they admit that it is obviously true (they are denying that it is
true, after all).  Such philosophers will object that so far, the only
objection I have raised against the view that there are no chairs is
that I cannot bring myself to believe it.

However, there is another reason to resist their conclusions, one that
is independent of my inability to believe that there are no chairs.
As we will see in Section \ref{stroud}, philosophers who deny that
there are chairs have a difficult time explaining why we believe that
there are chairs.  To the extent that they cannot explain why we hold
this belief (and others concerning ordinary things), we have reason to
suspect that their denials might be unfounded.


\chapter{Why do I believe that there are chairs?}
\chapterpig{Why do I believe that there are chairs?}
\label{stroud}

A nihilistic metaphysical thesis should be accompanied by an
explanation of why people nonetheless believe that there are chairs
and other ordinary things.  Peter van Inwagen and Trenton Merricks
each have the beginnings of such an explanation.  The explanation
offered by van Inwagen is flawed, but Merricks has a more promising
strategy.  I expand on what I take to be Merricks explanation of why
we believe that there are chairs, and conclude that it is successful,
given one other assumption.  That assumption is the denial of
metaphysical universalism.  However, universalism is independently
plausible, and {\em its} unintuitive consequences can be
satisfactorily explained using Merricks' own strategy.

\section{Explaining the beliefs of others}
\label{intro-beliefs}
\noindent Many people have false beliefs.  These beliefs misrepresent
how the world is.  For example, some people believe that ghosts exist.
These people each hold a false belief, for it is not true that ghosts
exist.  There are no ghosts in the world.  Despite this fact---that
there are no ghosts---some people believe that there are.  Why?  What
explanation can we give as to why someone believes a falsehood like
this?

In explaining why someone holds a belief, we appeal to {\em reasons}.
Even people who hold beliefs that we may consider irrational (like the
belief that there are ghosts) have reasons for holding these beliefs.
They may not be good reasons; someone might believe that there are
ghosts because her older sister told her that there are ghosts, or
because she read ghost stories as a child and took them seriously.
Someone who believes in ghosts might even think that she has {\em
  seen} a ghost.  This too would be a false belief; there are no
ghosts, so nobody can have seen one.  But here too there will be a
reason why she holds this false belief.  Perhaps she saw a strange
play of light on a distant wall, or the reflection of the moon
filtered through an attic window.  What she actually saw was perhaps
one of these things, but she somehow took what she saw to be a ghost.
Probably she already believed that there were ghosts, and so, when
confronted with a deceptive or confusing sight, was predisposed to
form the mistaken belief that she was seeing a ghost.

Here and in what follows, when I say that there is a reason why
someone believes something, I mean that there is some {\em cause} that
produced the belief.  Above, I told a causal story about why the
person who believes that she saw a ghost holds that belief.  She had
been told that there were ghosts by a person who she thought
trustworthy, so she came to believe that there are ghosts.  Holding
that belief caused her to be predisposed to interpret unusual
phenomena as ghosts.  This disposition caused her to believe that she
was seeing a ghost when she saw a reflection of the moon.

My use of the word `reason', therefore, should be taken in this causal
sense.  There are other ways that people use the word `reason'.  If
someone asks ``What reason do you have to believe that $((P
\rightarrow Q ) \wedge P) \rightarrow Q$?''  I might reply that it is
a theorem of first-order logic.  Here I am not telling a causal story.
I am rather {\em justifying} my belief that $((P \rightarrow Q )
\wedge P) \rightarrow Q$.  But in this case it is perfectly correct to
say that I am giving a reason as to why I hold a belief.  It is just
not a {\em causal} reason.  A causal reason would be something like
the following: $((P \rightarrow Q ) \wedge P) \rightarrow Q$ is true,
and I have done the proof.

(Another example: suppose someone falsely believes that $((P
\rightarrow Q ) \wedge Q) \rightarrow P$ is a theorem of first-order
logic.  There will be some (causal) reason why they hold this belief;
probably they attempted to deduce it from no premises and believe that
they succeeded.  There will, in turn, be a reason why they hold {\em
  this} false belief; maybe they were not concentrating on the proof
steps, or they forgot certain rules of deduction.)

An example involving an obviously true belief might help clarify the
distinction between causal reason and justifying reasons.  If someone
were to ask me why I believe that the sky is blue during the day, my
answer would be ``Because it is!''  There's not much else I can say to
{\em justify} my belief.  But this not a {\em causal} explanation.
The fact that something is true (the sky {\em is} blue) does not cause
me to believe it.  Otherwise I would believe every truth, and I do
not.  There are doubtless many truths that I do not believe.  There
must therefore be another (causal) reason why I believe that the sky
is blue, other than the fact that the sky is blue.

I believe that the sky is blue because, first, it is blue, and second,
I have {\em seen} that it is blue.  My vision is generally reliable
(or at least seems to be), so the fact that my eyes ``tell'' me
something is good reason to believe it.  The same is true of my other
senses: they are generally reliable, so the fact that they ``tell'' me
something is a good reason to believe it.  It does not follow that it
is {\em true}, however (though no doubt we believe that it is true);
our eyes can be deceived.

A skeptic might claim that we cannot rule out the possibility that we
are {\em constantly} deceived.  They attempt to undermine the
reliability of our senses.  I will not be addressing such arguments.
Rather, in what follows I will examine arguments that deny (or appear
to deny) that many of our beliefs about ordinary things are true.  The
philosophers making these denials do not claim that our eyes are
unreliable sources of information.  Their arguments are metaphysical
rather than epistemic; they deny that certain objects are {\em
  possible}.

For example, Trenton Merricks believes that chairs do not exist.  He
relies on a number of metaphysical arguments to motivate this claim.
If he is right, however, then it seems to follow from this that
beliefs like ``There are chairs'' are necessarily false.  I, however,
believe that there are chairs.  Even if Merricks is right, and my
belief is (necessarily) false, there are reasons why I believe this.

If someone were to ask me why I believe that there are chairs, I would
probably answer ``Because there are, and I have seen them (and sat
upon them)!''  It seems obviously true, just like the fact that the
sky is blue.  I have seen lots of chairs, and I can't have been
confused or deceived {\em every} time.

Nonetheless, Trenton Merricks, Peter van Inwagen and other
philosophers say that I am mistaken.  They claim that I have not in
fact seen lots of chairs, though I believe that I have.  There are
several different arguments by which nihilists seek to establish that
chairs (and other ``ordinary things'') do not exist; we will examine
some of these arguments below.  Having made these arguments, however,
the nihilists must reject our causal explanation of why we believe
that there are chairs.  Our explanation was that there are chairs and
we can see them.  But the nihilist denies that there are chairs, and
so should admit that, if we believe that there are chairs, there must
be a different explanation as to why we hold this belief.

\subsection{Paraphrasing beliefs}
\label{paraphrase}
Trenton Merricks denies that chairs exist, and claims that, if we
believe that chairs exist, we are mistaken.  His task will be to
explain why we form these false beliefs.  But not all nihilistic
philosophers deny that we are, in fact, mistaken.  They deny that
there are any chairs, but maintain that beliefs like the following
might still be true:

\begin{itemize}
  \item There are two chairs in the next room.
  \item I own some very nice 17th-century chairs.
  \item Some chairs are heavier than some tables.
\end{itemize}

Peter van Inwagen is one of these philosophers.  He denies the
existence of tables, chairs, apples, and all other inanimate composite
objects (van Inwagen's technical definition of `composite' will be
discussed below in Sections \ref{scq} and \ref{tech}).  He allows that
the sort of propositions listed above may be true, but insists that
this does not mean that there are chairs (or tables):
\begin{squote}
I want to do what I can to disown a certain apparently almost
irresistible characterization of my view, or of that part of my view
that pertains to inanimate objects.  Many philosophers, in
conversation and correspondence, have insisted, despite repeated
protests on my part, on describing my position in words like these:
``Van Inwagen says that tables are not real''; ``\,\ldots not true
objects''; ``\,\ldots not actually {\em things}''; ``\,\ldots not
substances''; ``\,\ldots not unified wholes''; ``\,\ldots nothing more
than collections of particles.''  These are words that darken counsel.
They are, in fact, perfectly meaningless.  My position vis-\`{a}-vis
tables and other inanimate objects is simply that there {\em are}
none~(\citeyear[99]{inwagen1995}).
\end{squote}

Van Inwagen asserts, quite seriously, that ``there are no tables or
chairs or any other visible objects except living organisms''
(\citeyear[1]{inwagen1995}).  But van Inwagen cannot deny that we at
least {\em believe} that there are chairs.  He admits that many of us
hold beliefs that we would express as ``There are two chairs in the
next room'' or ``I bought a new chair today''.  Indeed, he admits that
such beliefs are often {\em true}: ``when people say things in the
ordinary business of life by uttering sentences that start `There are
chairs\,\ldots ' or `There are stars\,\ldots ', they very often say
things that are literally true'' (\citeyear[102]{inwagen1995}).

Van Inwagen, when denying that we have beliefs about chairs, appears
to maintain that the beliefs that we (erroneously) take to be about
chairs are not, in fact, beliefs about chairs.  If a belief expressed
as ``That is a fine chair'' was actually about a chair, then it could
only be true if there was at least one chair (and a fine one).  But
van Inwagen denies that there is at least one chair, but nonetheless
says that such a belief might be true.  He accordingly recognizes the
need to explain what our beliefs really are about.  If he explains
what the {\em content} of our beliefs is, then he will also be able to
explain {\em why} we hold such beliefs.

\section{Paraphrases}
\label{van-paraphrase}
Van Inwagen attempts to maintain that there are no chairs while
rejecting the further claim that ``There are chairs'' is false.  He
claims that such discourse is {\em compatible} with the nonexistence
of chairs.  According to van Inwagen, ``when people say things in the
ordinary business of life by uttering sentences that start `There are
chairs\,\ldots ' or `There are stars\,\ldots ', they very often say
things that are literally true'' \citep[102]{inwagen1995}.

One might assume that if such statements are true, then it follows
that there are chairs and stars.  But van Inwagen denies that chairs
and stars exist.  How can he claim, then, that what was said was true?
What van Inwagen does is attempt to show that the statements in
question can be {\em paraphrased}---they can be reformulated to show
that they have no ``ontological commitments''.  According to van
Inwagen, one can assert that there is a chair without being committed
to the existence of chairs.

Section \ref{comp} will summarize the motivation for van Inwagen's
denial.  Section \ref{i-para} will introduce and criticize van
Inwagen's paraphrasing strategy.

\subsection{Composition}
\label{comp}
Van Inwagen's conclusion that there are no chairs is a consequence of
his views on {\em composition} (or ``constitution'').  Some things are
said to compose another thing if the former are {\em parts} of the
latter; the latter is ``made up of'' the former.  Van Inwagen believes
that ``the metaphysically puzzling features of material objects are
connected in deep and essential ways with metaphysically puzzling
features of the constitution of material objects by their parts''
\citep[18]{inwagen1995}.  One case often used to illustrate these
puzzling features is that of the Ship of Theseus.  The Ship of Theseus
is an object---a ship---composed of many parts, including planks of
wood.  As the planks (and other parts of the ship) wear out, they are
replaced.  These replacements happen each by themselves; the entire
ship (or even a large section) is not replaced all at once.  But
eventually no part of the original ship remains; it is built of
entirely different planks, nails, rigging, etc.  And yet we would no
doubt say that it is still the same ship.  But why should we think
that the present ship is identical with a past ship with which it
shares no parts?

\subsection{The Special Composition Question}
\label{scq}
Answering the question ``why is this ship identical with that past
ship?'' requires first figuring out how these planks (and rigging and
sails) compose a ship in the first place.  Van Inwagen asks ``in what
circumstances do planks compose (add up to, form) something''
(\citeyear[21]{inwagen1995})?  (For simplicity's sake, van Inwagen
ignores the rigging and sails.)  For some $x$s, van Inwagen is asking
us to consider when
\begin{displaymath}
\exists y\ \text{the}\ x\text{s compose}\ y
\end{displaymath}
is true.

Less formally, van Inwagen asks: ``suppose one had certain
(nonoverlapping) objects, the $x$s, at one's disposal; what would one
have to do---what {\em could} one do---to get the $x$s to compose
something'' (\citeyear[31]{inwagen1995})?  This is the Special
Composition Question.

(`Composition' is used in a technical sense with regard to the Special
Composition Question.  Van Inwagen defines it thus: ``the $x$s compose
$y$'' means that ``the $x$s are all parts of $y$ and no two of the
$x$s overlap and every part of $y$ overlaps at least one of the
$x$s\,\ldots a thing {\em overlaps} a thing---or: they overlap---if
they have a common part'' (\citeyear[29]{inwagen1995}).  For van
Inwagen, everything is a part of itself; some $x$ is a {\em proper}
part of some $y$ only if $x \neq y$.  I discuss the technical notion
of composition in more detail in Section \ref{tech}.)

\subsection{The usual answers}
\label{scq-ans}
There are several prominent answers to the Special Composition
Question, including the following (these formulations are from
\citet{markosian1998a}):
\begin{description}
	\item[Nihilism] Necessarily, for any $x$s, there is an object
          composed of the $x$s if and only if there is only one of the
          $x$s, i.e., the only objects that exist are simples
          (\citeyear[219]{markosian1998a}).
	\item[Universalism] Necessarily, for any $x$s, there is an
          object composed of the $x$s if and only if no two of the
          $x$s overlap (\citeyear[227]{markosian1998a}).
	\item[Van Inwagenism] Necessarily, for any $x$s, there is an
          object composed of the $x$s if and only if either (i) the
          activity of the $x$s constitutes a life or (ii) there is
          only one of the $x$s (\citeyear[221]{markosian1998a}).
\end{description}

Contemporary nihilists include Ted Sider and Cian Dorr---these
philosophers deny that {\em anything} has parts.  However, the
arguments I will be making against van Inwagen and Trenton
Merricks---who allow that people exist and have parts---will apply
equally to Sider and Dorr, for both factions deny that there are
chairs and other ordinary things.

Universalism raises a number of issues, some in connection with
Trenton Merricks' explanation of our beliefs.  I will therefore
postpone discussion of this view until later.  (See Section
\ref{universalism}).

Van Inwagen examines and rejects universalism and the version of
nihilism given above.  He also rejects a number of other answers to
the Special Composition Question.  Some are too strong: {\em some $x$s
  compose a $y$ if and only if the $x$s are in contact} would entail
that two people shaking hands will result in a new object coming into
being.  Others are too strong in some ways and too weak in others:
{\em some $x$s compose a $y$ if and only if the $x$s are fastened
  together} would entail that two people being glued together would
result in a new object; and it would deny that an object can be
composed without fastening its parts together (such as when building a
house of cards).  The only answer van Inwagen finds acceptable is what
we have dubbed {\em van Inwagenism}, which entails that tables and
chairs do not exist.

Because of this consequence, van Inwagenism should include an
explanation why we nonetheless believe that there are tables and
chairs.  Happily, van Inwagen recognizes this and is prepared with a
{\em paraphrasing strategy}.  This strategy aims to show that the
beliefs that we take to be about tables and chairs---such as ``There
are two fine chairs in the next room''---are actually very often true.
They are true not because there are chairs (in particular, two fine
ones in the next room), but because such beliefs are not actually {\em
  about} chairs to begin with.  They are about something else.

If such beliefs are true, then, once we know what they are about, it
should be relatively easy to explain why we hold them: they are true,
and we learn of them through some reliable means (like our eyes).

Unfortunately, van Inwagen's paraphrasing strategy does not work.

\section{Van Inwagen's paraphrasing strategy}
\label{i-para}
Van Inwagen admits that ``when people say things in the ordinary
business of life by uttering sentences that start `There are
chairs\,\ldots ' or `There are stars\,\ldots ', they very often say
things that are literally true'' \cite[102]{inwagen1995}.  It does not
seem unreasonable to assume that if what people say with ``There are
chairs\,\ldots '' and the like are true, then chairs exist.  But van
Inwagen denies this entailment.

How can van Inwagen maintain this?  We may first observe that someone
can say, truly, ``There are simples arranged chairwise\,\ldots ''
without committing oneself to the existence of chairs.  Van Inwagen
might then claim that when someone says ``There is a chair\,\ldots ''
she {\em means} ``There are simples arranged chairwise''.  This is, of
course, a bold hypothesis about the speech practices of ordinary
speakers.  Certainly very few speakers would, if asked, affirm that
what they meant to say had anything to do with simples; they would say
that when they said that there was a chair, they meant just that.  Van
Inwagen recognizes that this is not a viable position: ``The only
thing I have to say about what the ordinary man really means by `There
are two valuable chairs in the next room' is that he really means that
there are two valuable chairs in the next room''
(\citeyear[106]{inwagen1995}).

One might then assume that van Inwagen is thinking in analogy with
Russell.  He could attempt to claim that, despite the surface
appearance of language (`There is a chair\,\ldots '), the underlying
logical form does not make any mention of chairs (or tables); the
offending concept is analyzed away, leaving `There are simples
arranged chairwise\,\ldots '.  Van Inwagen notes that his ``suggested
technique of paraphrasing enables us to escape some of the more
embarrassing consequences of this position.  When someone says `Some
tables are heavier than some chairs,' there is obviously something
right about what he says.  Our technique of paraphrasis enables us to
capture what it is that is right about what he says''
(\citeyear[111]{inwagen1995}).  This approach is similar to that of
Ted Sider and Eli Hirsch (Sections \ref{hirsch} and \ref{ontologese});
both are sympathetic to the idea that `There are chairs' means (at
least in some circumstances) that there are things arranged chairwise,
{\em not} that there are chairs.  Such a position is, as I argued in
Section \ref{verbal}, correct only if one adopts a truth-conditional
theory of meaning.

However, van Inwagen does not defend this position.  He admits that
the original proposition and his paraphrased version are different:
``When the ordinary man utters the sentence `Some chairs are heavier
than some tables' (in an appropriate context, and so on and so on), he
expresses a certain proposition, and one that is almost certainly
true.  But I do not claim that this proposition {\em is} the
proposition that, for some $x$s, those $x$s are arranged chairwise and
for some $y$s, those $y$s are arranged tablewise, and the $x$s are
heavier than the $y$s'' (\citeyear[112]{inwagen1995}).  So van Inwagen
is not making an appeal to some notion of ``logical form'', and he is
not proposing, like Hirsch and Sider sometimes do, that `There is a
table' just means `There are things arranged tablewise'.  But then
what is the purpose of the paraphrasing project?

Van Inwagen attempts to justify his method of paraphrasis by asserting
the following parallels between the original and paraphrased
propositions:
\begin{enumerate}[ref=(\arabic*)]
	\item The paraphrase describes the same fact as the
          original.  \label{para-a}
	\item The paraphrase, unlike the original, does not even
          appear to imply that there are any objects that occupy
          chair-receptacles [a chair-receptacle is a region of space
            said to be occupied by a chair].  \label{para-b}
	\item The paraphrase is neutral with respect to competing
          metaphysical theories, {\em viz}.  the ``received'' theory,
          that there are objects that occupy chair-receptacles, and
          the theory I have proposed, according to which there are no
          such objects.  \label{para-c}
	\item The original, though it doubtless does not express the
          same proposition as the paraphrase, has the feature ascribed
          to the paraphrase in \ref{para-c}: It is neutral with
          respect to the question whether there are objects that fit
          exactly into
          chair-receptacles~(\citeyear[113]{inwagen1995}).  \label{para-d}
\end{enumerate}
I am willing to grant that \ref{para-a}--\ref{para-c} are true, but I
am quite sure that \ref{para-d} is false, and van Inwagen's thesis
appears to depend on it.  He admits in \ref{para-b} that the original
proposition (`There are chairs\,\ldots ') {\em implies} that there
are chairs, but claims in \ref{para-d} that it does not {\em entail}
this.  But why wouldn't it?

\subsection{Propositions and ontological commitment}
\label{prop-ont}
Let us review the situation.  First, van Inwagen agrees that when
someone says something like ``There is a chair\,\ldots '' they mean
just that.  Second, he admits that his ``paraphrases'' of such
propositions are not so faithful to the original that they can be
called the same proposition; the original and the paraphrase are two
different propositions.  Third, he claims nonetheless that {\em
  neither} the original nor the paraphrase entail the existence of
chairs.

This seems obviously untrue.  How can he claim that when someone says
``There is a chair\,\ldots '' and means just that, that the
proposition they express does not entail the existence of chairs?  To
defend his claim, van Inwagen appeals to his ``Copernican analogy'':

\begin{squote}
I accept the Copernican Hypothesis.  One day you hear me say, ``It was
cooler in the garden after the sun had moved behind the elms.''  You
say, ``You see, you can't consistently maintain your Copernicanism
outside the astronomer's study.  You say that the sun moved behind the
elms; yet, according to your official theory, the sun does not move.''
I reply that the proposition I expressed by saying ``It was cooler in
the garden after the sun had moved behind the elms'' is consistent
with the Copernican Hypothesis (\citeyear[101]{inwagen1995}).
\end{squote}
That is, van Inwagen claims that the proposition he expressed with `It
was cooler in the garden after the sun had moved behind the elms' does
not entail that the sun actually moved.  And he argues that this is
analogous to our talk of chairs: most propositions expressed with
`There is a chair\,\ldots ' do not entail that chairs actually exist.

Does the proposition van Inwagen expresses with `The sun moved behind
the elms' entail that the sun moved?  I am inclined to say that it
does.  If I were to say simply ``The sun moved'' (meaning just that),
I think I would have committed myself to the movement of the sun.  Why
should we think that the addition of ``behind the elms'' removes this
entailment?  Without some explanation of what the difference is, I see
no reason to think that saying ``The sun moved behind the elms'' (and
meaning it) does not entail the movement of the sun.  Likewise, if
`There are chairs in the next room' does not entail that there are
chairs, then it would appear that `There are chairs' does not entail
that there are chairs.

Before we dismiss van Inwagen's paraphrasing strategy, we should
examine another, perhaps more plausible, analogy.  This analogy
involves an imaginary planet called Pluralia where there is a
``creature'' known as a bliger.  The bliger, according to van Inwagen,
is what happens when four monkeys, an owl, and a sloth attach
themselves together temporarily.  The conglomeration appears to the
untrained observer to be a single animal.  Gullible farmers have
designated this type of conglomeration with the word `bliger'.  Van
Inwagen's point is that there are no bligers, but that a farmer saying
``There's a bliger'' when pointing at such a conglomeration would be
saying something true.  Even though there are no bligers (according to
van Inwagen), someone saying ``There's a bliger'' says something true
because she ``reports a fact''.  The fact being reported by `There's a
bliger' is the fact that a monkey, four owls, and a sloth are there.
If she has instead said ``That bliger just exploded'', what she said
would be false, because there is no fact that her proposition reports.

People believe that there are bligers because they mistake the group
of animals for a single thing, which has been dubbed `bliger'.
Likewise, van Inwagen maintains that people mistake chairwise
arrangements of simples for chairs.  When someone says ``There's a
chair'' what she says is true because it reports a fact.  The fact
being reported is that there is a chairwise arrangement of simples
there.  People believe that there are chairs because they mistake the
things arranged chairwise for a single thing, which has been dubbed
`chair'.

I agree with van Inwagen that these cases are analogous.  However,
where van Inwagen takes this analogy to show that there are no chairs,
I take it to show that there {\em are} bligers in van Inwagen's
imaginary scenario.  When it is discovered that bligers are built up
from six other creatures, we are learning something about bligers:

\begin{squote}
\ldots {\em of course} there are bligers in [van Inwagen's] story.
Bligers are what the story is about.  The zoologists do not report
that there are no bligers.  Rather they tell us what a bliger is.
They explain that a bliger is not a single large carnivorous animal
but a transient symbiotic union of six animals
\citep[704]{rosenberg1993}.
\end{squote}

In short, van Inwagen's analogy does not provide us with an
explanation of why we would believe in chairs even if there were
none.  All it should be taken to show is that just as we believe that
there are chairs because there are chairs, so we would believe there
were bligers if there were bligers.

\subsection{Unforeseen consequences}
\label{backfire}
Van Inwagen should be glad that his paraphrasing strategy does not
succeed.  If he showed that `There are chairs' does not entail that
there are chairs, then the whole notion of ontological commitment
would be undermined.  If `There are chairs' did not entail that there
are chairs, then why should any proposition of the form `There are
$x$s' entail that there are $x$s?

It is surely true that van Inwagen would affirm ``There are simples
arranged chairwise''.  And no doubt he thinks that it follows from the
truth of that proposition that there are simples arranged chairwise.
But how can he affirm this, if he denies that `There are chairs'
entails that there are chairs?

If `There are chairs in the next room' does not entail that there are
chairs and if `The sun moved behind the trees' does not entail that
the sun moved (nor that it exists), then how can van Inwagen maintain
that `There are simples arranged chairwise' entails that there are
simples, or that they are arranged chairwise?  He has given us no
reason to believe one and not the other.

(And when he says, of tables and chairs, that there are none, what
follows from that?  Surely that there are no tables or chairs.  But if
van Inwagen denies that `There are chairs' entails that there are
chairs, why should he think that `There are no chairs' entails that
there are no chairs?  The only reasonable conclusion here is to reject
van Inwagen's paraphrasing strategy in full.)

\section{Lessons, part 1}
\label{lessons-v}
Van Inwagen has not succeeded in explaining why we believe that there
are chairs when (according to him) there are none.  This explanatory
deficiency should give us hope that what seems obviously true---that
there are chairs---really is so, and that van Inwagen's argument
against that truth is faulty.

If van Inwagen's conclusion is false, there must be something wrong
with his argument.  One possibility is that he has overlooked a better
answer to the Special Composition Question (in Section
\ref{universalism} I will suggest that some version of universalism is
true).  But it is also possible that the Special Composition Question
is itself the wrong question to be asking.  Jay Rosenberg brings out
this worry nicely.  Van Inwagen's informal version of the question is
this: ``Suppose one had certain (nonoverlapping) objects, the $x$s, at
one's disposal; what would one have to do---what {\em could} one
do---to get the $x$s to compose something''
(\citeyear[31]{inwagen1995})?  This is how Rosenberg replies:

\begin{squote}
To me it just seems obvious that the answer to such a question will
always depend on what sorts of things one has at one's disposal and
what sort of thing one is trying to get them to compose.  If the $x$s
are, for example, ``a lot of wooden blocks that one may do with as one
wills'', then to get them to compose, for example, a wall, it may be
sufficient to stack them up in the manner we call ``building a wall.''
To get them to compose a wooden raft, on the other hand, one would
surely need to fasten them together more securely, e.g., by gluing
them to one another.  And there's nothing at all one could do with
them to get them to add up to a fish or a clock or a sports car
(\citeyear[705]{rosenberg1993}).
\end{squote}

Rosenberg suggests that, rather than asking ``What is required for
composition'', we should be asking ``What is required to compose a
chair, or a boat, or a house?''  He may be claiming that there is no
answer to the Special Composition Question as it is formulated;
different composite objects are composed in different ways.  Moreover,
explaining how and why different objects are composed in the ways they
are will draw upon different fields of study: ``Microphysics explains
how protons, neutrons, and electrons compose different species of
atoms, and physical chemistry, how atoms of various species compose
different sorts of molecules'' \citep[706]{rosenberg1993}.

I am sympathetic to this sort of worry.  Ultimately I will claim that
the Special Composition Question has an answer: any things whatever
compose an object.  But the sorts of questions that Rosenberg wants
answered---``What is required to compose a chair, or a boat, or a
house?''---are left unanswered.  The fact that any things whatever
compose some further thing does not tell us whether that further thing
is a chair, or a boat, or a house.  I will suggest in Section
\ref{deflate}, however, that these sort of questions are not
metaphysical questions at all; rather, they are questions about our
``conceptual household''.

\subsection{Another attempt to explain our beliefs}
\label{explain-merricks}
Van Inwagen's approach to nihilism is not the only one.  Trenton
Merricks has proposed a very similar thesis---that the only composite
objects are human beings---for very different reasons.  Like van
Inwagen, he tries to undermine the obviousness of the fact that there
are chairs, and he attempts to explain why we believe that there are
chairs at all.  Merricks' attempt succeeds only if universalism is
rejected.  But Merricks' method of explanation can also explain why we
hold beliefs that conflict with universalism.  And given that
universalism is intuitively more plausible than nihilism, the lesson
we will take away from Merricks is that universalism is probably true.

\section{How does Merricks explain what we believe?}
\label{universe}
\label{merricks}
Trenton Merricks, like van Inwagen, claims that there are no physical
objects other than human beings.  However, he comes to this conclusion
through a different path of reasoning.  He claims, roughly, that
positing ordinary things (excluding people) is causally redundant;
everything that ordinary things are said to do can be described in
terms of their parts. (The details are unimportant; what matters is
how Merricks explains why we nonetheless believe that there are
ordinary things.)

Despite the fact that Merricks has a different motivation for his
nihilism, we can pose the same question to him as we posed to van
Inwagen.  Why, if there are no chairs, do we believe that there are
chairs?  Happily, Merricks addresses our concern.  Even more happily,
he has a better explanation than van Inwagen.  He explains why, if
nihilism is true, we might nonetheless believe that there are chairs.
But his strategy presupposes that universalism is false (see Sections
\ref{scq-ans} and \ref{universalism}).  Universalism, like nihilism,
seems to contradict certain of our beliefs, but Merricks' strategy can
also explain why, if universalism is true, we nonetheless hold these
certain beliefs.  Merricks' strategy does not therefore provide
nihilism any advantage over universalism, and universalism is
intuitively more plausible than nihilism.

\subsection{Nearly as good as true}
\label{near}
Merricks claims that ``folk'' beliefs, such as the belief that there
are chairs, are false, but nonetheless are {\em nearly as good as
  true}.  What does this mean?

\begin{squote}
People who believe in unicorns [or ghosts] are few and far between.
And those few are generally unjustified.  On the other hand, people
who believe in statues are legion.  And they are generally justified
in so believing.  Given the truth of eliminativism [what I have been
  calling `nihilism'], we might ask {\em why} the belief in statues is
more common, and more commonly justified, than the belief in unicorns.

The answer is that statue beliefs are nearly as good as true.  For, so
I claim here, {\em atoms arranged statuewise} often play a key role in
producing, and grounding the justification of, the belief that statues
exist.  In general, a false belief's being nearly as good as true
explains how {\em reasonable} people come to hold it.  And, relatedly,
its being nearly as good as true can ground its justification.
Because the belief that unicorns exist is not nearly as good as true
(i.e., because there are no things arranged unicornwise), there is no
similar explanation of its production or similar reason to think it is
justified (\citeyear[171--172]{merricks2001a}).
\end{squote}

To say that a proposition is ``nearly as good as true'' seems to mean
that while it is false, it is nonetheless somehow close enough to the
truth for a given purpose or situation.  For example, suppose we have
decided to buy a fake holiday tree for the holidays this year.  We are
looking at a number of different fake trees.  I point to one and say
``That is a nice tree''.  What I have said is false; that is not a
tree.  It is a fake tree.  But what I mean---and what my audience
recognizes me to mean---is that it is a nice {\em fake} tree.  We both
know that we are looking at fake trees; there is no point in saying
``fake tree'' every time.  When I say ``That is a nice tree'',
therefore, what I say is quite sufficient to allow for successful
communication, despite being false.  Merricks claims that propositions
expressed by things like `There are chairs' are also loosely true.
They are false, but are nonetheless good enough for certain purposes.

Initially, this seems like a bizarre claim.  After all, Merricks is
claiming that chairs {\em necessarily} do not exist.  According to
Merricks, `There are chairs', given its current meaning, could {\em
  never} be true.  If the proposition expressed by `There are chairs'
is necessarily false, how could it nonetheless be ``nearly as good as
true''?

\subsection{The conceptual connection}
\label{connection}
Merricks' argument relies on a very close connection between the
concepts {\em chair} and {\em chairwise} (and likewise for all
``ordinary concepts'').  Despite claiming that chairs are impossible,
Merricks admits that we understand perfectly what chairs {\em would}
be, if they existed.  Because we understand the concept {\em chair},
we can recognize {\em actually existing} things that are arranged
chairwise:

\begin{squote}
The folk concept of \emph{statue} plays a role in determining which
atomic arrangements are statuewise. I would even go so far as to say
that if \emph{being arranged statuewise} were not derivative upon
folk-ontological concepts\,\ldots something would be amiss
(\citeyear[8]{merricks2001a}).
\end{squote}

For Merricks, to know what things are actually arranged statue- or
chairwise requires knowing what things would compose a statue or a
chair, if such things were possible:

\begin{squote}
Atoms are \emph{arranged statuewise} if and only if they both have the
properties and also stand in the relations to microscopica upon which,
if statues existed, those atoms' \emph{composing a statue} would
non-trivially supervene (\citeyear[4]{merricks2001a}).
\end{squote}

Merricks' explanation of why we believe that there are chairs relies
on this conceptual connection.  It also is structurally similar to the
explanation we gave in Section \ref{intro-beliefs}.  Recall that our
explanation of why we believe that there are chairs (or statues) is
that, first, there are chairs, and, second, we see that there are
chairs (or learn that there are chairs through a similarly reliable
mechanism).

Merricks' definition of `nearly as good as true' allows us to produce
a parallel explanation.  His definition is this:

\begin{squote}
Any folk-ontological claim of the form `F exists' is \emph{nearly as
  good as true} if and only if (i) `F exists' is false and (ii) there
are things arranged F-wise. So, for example, `the statue \emph{David}
exists' is nearly as good as true because (it is false and) there are
some things arranged Davidwise (\citeyear[171]{merricks2001a}).
\end{squote}

We may now say on behalf of Merricks that we believe that there are
chairs (and statues) because, first, there are things arranged
chairwise and, second, we see that there are things arranged
chairwise.

The structure of the two explanations is analogous, but there is an
apparent disanalogy in the content of the two.  The disanalogy does
not favor Merricks.  For it is easy enough to understand why there
being chairs, and us seeing that there are chairs, would cause us to
believe that there are chairs.  But it is less obvious why there being
things arranged chairwise, and us seeing that there are things
arranged chairwise, would cause us to believe {\em not} that there are
things arranged chairwise, but that there are {\em chairs}.

(While it is certainly true that we believe that there are chairs, I
am not sure if all or even most of us {\em also} believe that there
are things arranged chairwise.  Let us suppose for now that we do.)

The close conceptual connection between {\em chair} and {\em
  chairwise} is very important for Merricks.  It is this {\em
  connection} that is doing the explanatory work.  The only thing that
can explain why there being things arranged chairwise would cause us
to believe that there are chairs is this connection between the
concepts.  The existence of things arranged chairwise, and the belief
that there are things arranged chairwise, is supposed to cause the
{\em additional} belief that there are chairs.  How does this happen?

Merricks' answer appears to go something like this: chairwise
arrangements, statuewise arrangements, and other ordinary arrangements
of things play important roles in our lives.  These arrangements of
things are of interest to us, so we have developed words that allow us
to refer to them.  For whatever reason---historical, psychological, or
otherwise---we think of each arrangement as a single thing, rather
than as things.  Words like `chair' and `statue', being singular,
reflect this (incorrect) view of the world.  We are, in a sense,
fooled by grammar.

This is more than Merricks says himself.  I have not found a passage
in which he explicitly describes the nature of the conceptual
connection between concepts like {\em chair} and {\em chairwise}, and
explains why, from our belief that there are things arranged
chairwise, we invariably infer that there are chairs.  But I think he
would endorse something like this.  In the first chapter of his book,
he claims that whether there is a statue or merely things arranged
statuewise is not an empirical question.  He claims that were there
not a statue and merely things arranged chairwise, our ``visual
evidence'' would be the same.  He supports this claim with an analogy:

\begin{squote}
{[}Consider{]} the claim that the atoms arranged
my-neighbour's-dogwise and the-top-half-of-the-tree-in-my-backyardwise
compose an object\ldots it won't do to defend this claim with nothing
more than ``I can \emph{just see} the object composed of the atoms
arranged dog-and-treetopwise''. Part of why this won't do, presumably,
is that one's visual evidence would be the same \emph{whether or not}
those atoms composed something (\citeyear[8--9]{merricks2001a}).
\end{squote}

He assumes, of course, that we do not believe that there is a thing
composed of a dog and some of a tree.  Later he suggests that it is
arbitrary to claim that there are statues but not dog-tree things:
``we ought to see that the only difference between arbitrary sums and
statues is a matter of conventional wisdom and local custom''
\citeyearpar[75]{merricks2001a}.  He seems sympathetic to the idea
that the reason we believe that there are statues, and not dog-tree
composites, is due to our conventional speech practices: ``it is at
least somewhat plausible that atoms arranged statuewise are united not
by composing something but, instead and in part, by how we speak and
think'' \citeyearpar[121]{merricks2001a}.

On this picture, whether we see an arrangement of things as composing
an object or not depends more on our own interests than features of
the things themselves.  We have words for chairs and statues because
things arranged chairwise and statuewise interest us.  We don't have a
word for things arranged my-neighbor's-dogwise and
the-top-half-of-the-tree-in-my-backyardwise because such an
arrangement does not hold much interest for us.  But each of these
arrangements exist, and it seems arbitrary to say that the chairwise
and statuewise arrangements compose chairs and statues while the other
arrangement composes nothing.

Merricks might explain why we believe there are things arranged
my-neighbor's-dogwise and the-top-half-of-the-tree-in-my-backyardwise
thus: there are things arranged my-neighbor's-dogwise and
the-top-half-of-the-tree-in-my-backyardwise, and we see that there are
things so arranged.  This is exactly the same explanation that I would
give.

Now Merricks explains why we believe that there are chairs thus: there
are things arranged chairwise, and we see that there are things
arranged chairwise.  {\em And incidentally, due to our own human
  peculiarities, we have found it convenient to refer to and think
  about things arranged chairwise as if they were ``chairs''---single
  unified objects}.

\section{Strange objects}
\label{dogbush}
This is a somewhat plausible explanation of why we would believe that
there are chairs if there were not.  It is certainly much better than
van Inwagen's.  But I think that it fails.  I think that when we look
closer at Merricks' attempts to motivate nihilism, we will see that
they do not support nihilism at all.  If anything they support a
version of {\em universalism}.

Merricks observes that one might object to nihilism simply by saying,
``I just {\em see} the chair!''  He claims that if this objection
moves us, we should think about an analogous objection, which he finds
much less moving:

\begin{squote}
Whether atoms arranged statuewise compose a statue is analogous to
whether atoms arranged my-neighbour's-dogwise and
the-top-half-of-the-tree-in-my-backyardwise compose an object\,\ldots
it would not do to support an affirmative answer to the latter
question simply by saying ``I can just see that object''
\citeyearpar[73]{merricks2001a}.
\end{squote}

It does indeed seem initially plausible to say that the top half of a
tree and my neighbor's dog do not compose anything.  But I think this
is ultimately incorrect.

Recall the bliger story that van Inwagen used to motivate his version
of nihilism (Section \ref{prop-ont}).  A bliger was supposed to be
four monkeys, an owl, and a sloth, who arrange themselves into a
temporary symbiotic configuration.  Van Inwagen thought we would agree
that bligers did not exist.  He claimed that it is not true that ``six
animals arranged in bliger fashion compose anything, and that is what
I mean to deny when I say that there are no bligers''
\citeyearpar[104]{inwagen1995}.

But as we saw, it is simply false that there are no bligers:

\begin{squote}
\ldots {\em of course} there are bligers in [van Inwagen's] story.
Bligers are what the story is about.  The zoologists do not report
that there are no bligers.  Rather they tell us what a bliger is.
They explain that a bliger is not a single large carnivorous animal
but a transient symbiotic union of six animals
\citep[704]{rosenberg1993}.
\end{squote}

We might be tempted to say that there are no bligers because van
Inwagen presents the question in an unintuitive way.  He asks us if
there is some thing, some object, that is composed of the other six
animals.  This gives one the impression that, were there to be such a
thing, it would perhaps be another animal (a seventh); were there such
a thing, it should somehow pop out at us.  But all we see when we
picture the scene are the six animals together, so we feel that van
Inwagen might be right.  There is no {\em other} thing.  But if we
phrase the question differently, things become clearer.  Rather than
ask if there is some thing composed of such and such other things, we
simply ask, ``Are there bligers?''  And of course there are.  Van
Inwagen's use of the word `composition' led our intuitions astray.

Merricks makes the same mistake in his passage above.  Imagine if he
had said, ``Consider five discontinuous islands.  One cannot argue
that they compose some further thing by simply saying `I just see
it!'\,'' If these five islands are an archipelago, then one {\em can}
say ``I just see the archipelago!''  {\em Of course} there are
archipelagos.  They are, as one might put it, {\em scattered objects}.
The archipelago is made up of a number of separate islands, but it is
nonetheless a thing.  It is an archipelago.  Now let us suppose there
is an archipelago in the Mediterranean Sea (this example is adapted
from \citet{hawthorne2008}).  This archipelago is called the Roman
Archipelago, due to the fact that there are a number of Roman ruins on
one of its islands.  There are several research camps on the islands,
where archaeologists dig for artifacts.  Their researches result in a
surprising discovery: one of the islands {\em is} a Roman ruin.  What
was thought to a rocky and curiously shaped island is in fact a
massive collapsed temple.  Further investigation reveals that another
island is made up of the bones of an extinct sea monster, and another
island is a crashed \textsc{ufo}.

Despite these extraordinary circumstances, it is nonetheless true that
the Roman Archipelago exists.  It just happens to be composed of
several islands, a Roman ruin, a pile of old bones, and an alien
spacecraft.  To say the Roman Archipelago does not exist would entail
that these things are {\em not} sitting in the Mediterranean Sea.  (Of
course I made this story up, so the Roman Archipelago in fact doesn't
exist; but it does in the story.)

If Merricks or someone else asks us ``could scattered islands, Roman
ruins, old bones and alien spacecraft ever compose anything?'' we
should reply ``{\em of course}''.  Now take this example:

\begin{quote}
Pranksters break into a museum to install joke pieces of art.  One one
wall they put up a bathroom mirror and towel ring (complete with
towel).  Under the mirror they put a little sign reading ``Wash your
hands''.  The installation is accepted as art by the gullible curator,
who gets an equally gullible journalist to write about it.  {\em Wash
  Your Hands} quickly becomes a valuable piece of art---valuable
enough that art thieves target it.  They break into the museum in
order to steal {\em Wash Your Hands}, but trip an alarm and are forced
to flee.  All they get away with is the towel.  In the morning the
guards tell the curator that part of {\em Wash Your Hands} is missing.
The curator orders them to remove the rest of the piece and informs
crestfallen visitors that {\em Wash Your Hands} is no longer in the
museum's collection.
\end{quote}

Here, the only point at which is it true to say that {\em Wash Your
  Hands} is not in the museum is when it is finally removed.  Someone
who claimed that it was {\em never} in the museum because it doesn't
exist would be saying something quite clearly false.  Thus if Merricks
asks us ``do mirrors and towels ever compose anything?'' we should say
``{\em of course!}\,''

In these two examples, it is clear that the things in question really
do exist.  Nobody will deny that there are archipelagos and works of
art without having first been moved by a philosophical argument.  But
it may be that people {\em will} deny that there are things composed
of the tops of trees and dogs, even before hearing an argument.

Let us use `dogbush' to refer to things composed of dogs and treetops.
For example, in a park that contains one tree and one dog, there is
also one dogbush.  Is it {\em obvious} that there are dogbushes?  Is
it just as obvious as that there are archipelagos and chairs and the
{\em Wash Your Hands}?  If not, why?  What is the difference between
things like archipelagos and things like dogbushes?

One obvious difference is that things like archipelagos interest us.
I argued above that Merricks motivates his nihilism by drawing our
attention to the role of tradition and convention in our talk.  We
have a word for archipelagos because they {\em matter} to us.  We
don't have a word for dogbushes because they {\em don't} matter.
Merricks argued, in effect, that since we are not inclined to say that
there are dogbushes, and since there is no metaphysical difference
between dogbushes and archipelagos, we should not be inclined to say
that there are archipelagos.

But we can reverse Merricks' argument.  Since we {\em are} inclined to
say that there are archipelagos, and since there is no metaphysical
difference between archipelagos and dogbushes, we should not be
inclined to deny that there are dogbushes.

\section{Universalism}
\label{universalism}
I claimed in Section \ref{connection} that Merricks' explanation of
why we believe that there are chairs is something like this: there are
things arranged chairwise, and we see that there are things arranged
chairwise.  {\em And incidentally, due to our own human peculiarities,
  we have found it convenient to refer to and think about things
  arranged chairwise as if they were ``chairs''---single unified
  objects}.  I attributed to Merricks the idea that just because
things arranged chairwise interest us, we should not therefore suppose
that there are chairs.  What interests us should not be a guide to
what exists.  But now it is obvious how we should reply to Merricks.
Just because dogbushes do {\em not} interest us, we should not
therefore suppose that there are not dogbushes.  In this spirit,
Judith Thomson suggests that we ``think of Reality as like an
over-crowded attic, some of its contents interesting, and most merely
junk.  There is no need to deny the junk; we can simply leave it to
gather dust'' \citep[167]{thomson1998a}.  This is the intuition behind
universalism, one of the answers to the Special Composition Question
(Section \ref{scq}):

\begin{description}
\item[Universalism] Necessarily, for any $x$s, there is an object
  composed of the $x$s if and only if no two of the $x$s overlap
  \citep[227]{markosian1998a}.
\end{description}

The above considerations suggest an argument of this sort:

\begin{enumerate}[ref=(\arabic*)]
  \item Chairs exist. \label{u-1}
  \item Things that do not differ from chairs (or archipelagos, or
    works of art) in metaphysically significant ways also exist.
  \item Dogbushes do not differ from chairs in metaphysically
    significant ways.
  \item {\em Therefore}, dogbushes exist. \label{u-4}
\end{enumerate}

I imagine that Merricks would deny the conclusion \ref{u-4} and so, by
{\em modus tollens}, deny one or more premises (and we have seen that
he denies \ref{u-1}).  But I affirm the premises and so, by {\em modus
  ponens}, affirm the conclusion.

This argument helps us see what is wrong with Ned Markosian's response
to the Special Composition Question.  Markosian defends what he calls
``brutal composition''.  The thesis of brutal composition is that,
while there is indeed no ``no true, non-trivial, and finitely long
answer to [the Special Composition Question]''
\citeyearpar[214]{markosian1998a}, this is not because we should refer
questions of composition to the empirical sciences.  Rather, whether
or not some things compose another is simply a {\em brute fact}.

This is a clever reply, but whether true or not I do not think it does
the work that Markosian expects it to.  He presents his theory as
``consistent with standard, pre-philosophical intuitions about the
universe's composite objects'' \citeyearpar[211]{markosian1998a}.  But
his theory will only be consistent with such intuitions if, first, it
is a brute fact that all of the things we ordinarily take to exist
(tables, chairs, etc.) do in fact exist, and, second, that it is a
brute fact that the things that we don't take to exist don't in fact
exist.  But why should we expect there to be a {\em metaphysical}
difference between things that interest us and things that don't?  The
chance that the brute facts of composition happen to line up with our
(or Markosian's) intuitions seems to be incredibly low.

But accepting the above argument for universalism has some strange
consequences that are not immediately apparent.  Ned Markosian brings
out such a consequence in this passage:

\begin{squote}
There is what seems to me a fatal objection to Universalism:
Universalism entails that there are far more composite objects than
common sense intuitions can allow.  To give just one example,
Universalism entails that the following sentence is true:\,\ldots
There is an object composed of (i) London Bridge, (ii) a certain
sub-atomic particle located far beneath the surface of the moon, and
(iii) Cal Ripken, Jr.  My intuitions tell me that there is no such
object, and I suspect that the intuitions of the man on the street
would agree with mine on this point \citeyearpar[228]{markosian1998a}.
\end{squote}

If this is a compelling objection, it is because such an object (call
it `Lumpkin') does not interest us in the least.  As Merricks observed
(see Section \ref{connection}), the things that we believe to exist
are largely the things that interest us.  We believe that there are
archipelagos; van Inwagen's imaginary farmers believe that there are
bligers.  If we do not believe that there are dogbushes or Lumpkins,
this may be because they do not interest us.

Suppose Markosian wrote this instead:

\begin{squote}
Universalism entails that the following sentence is true:\,\ldots
There is an object composed of (i) an island, (ii) a Roman ruin, and
(iii) the bones of a sea monster.
\end{squote}

But this is just the Roman Archipelago I mentioned in Section
\ref{dogbush}.  We are (or should be) happy to admit that it exists.
If the Roman Archipelago exists, and if it does not differ from the
Lumpkin in any metaphysically significant ways, why shouldn't we admit
that the Lumpkin exists?  Of course we don't {\em care} about the
Lumpkin.  We have no need to refer to it; it doesn't matter to our
lives.  But why should we expect---as Markosian seems to---that our
intuitions should perfectly track what exists?

\section{Lessons, part 2}
\label{lessons-m}
What we have learned from examining Merricks' arguments is not that
there are no chairs.  What we have learned is that since there {\em
  are} chairs, and since dogbushes do not differ from chairs in
metaphysically significant ways, there are therefore also dogbushes.

If we agree that there are chairs and archipelagos and dogbushes and
the Lumpkin, however, new questions arise. For example: What are the
parts of a chair?  How do they compose the chair?  Do the parts of the
chair change over time?  How?  We will address these questions in the
next section.  


\chapter{How are chairs composed of their parts?}
\chapterpig{How are chairs composed of their parts?}
\label{parts}

In the previous section I argued that not only are there ordinary
things like chairs, but that there are more unusual things like
archipelagos, scattered works of art, and perhaps even dogbushes.  I
tried to make it at least plausible to assume that some version of
{\em universalism} is true---that for any material things, there is
some material object made up of them.

If this is the case, a new problem arises.  The problem is that
universalism, along with a few plausible assumptions, rules out the
possibility that things change their parts.  But this seems clearly
false; things appear to change their parts all the time.  The
philosopher who accepts universalism and claims that things change
their parts must deny one or more of the plausible assumptions that,
with universalism, rule out the possibility of things' changing their
parts.  One of these plausible assumptions is the assumption that no
two things completely overlap one another; that is, are {\em
  co-located}.  Denying this assumption leads to the possibility that
there are very many things in any given location; for example, there
might be a plurality of objects co-located with my chair at this very
instant.

There are reasons to be suspicious of a theory that entails such a
plurality of things.  But the alternative, which I will present in
Section \ref{essential}, is to deny that things really can change
their parts.

\section{Parthood and composition}
\label{part-comp}
This subsection will describe the classical notions of mereology.  The
technical formulae of classical mereology are often defined in terms
of parthood, which is itself left undefined (except for the
stipulation that everything is a part of itself).  `Part', moreover,
is understood to have a single, univocal, meaning; the consequence is
that everything that has a part (or parts) is a mereological sum.
Since everything is by definition a part of itself, classical
mereology entails that everything is a mereological sum.  We will see
in Section \ref{all-sum} that this consequence is problematic.

If we make several plausible assumptions, classical mereology also
entails that things cannot change their parts.  This consequence also
seems problematic, and the theories that I will present below are
motivated largely to avoid this consequence.

\subsection{Classical mereology}
\label{tech}
Peter van Inwagen provides the following definitions for classical
mereology:

\begin{enumerate}
  \item $x\ \text{is a part of}\ y =_{df} x\ \text{is a proper part
    of}\ y\ \text{or}\ x = y$
\end{enumerate}

It is assumed that everything is a part of itself.  A {\em proper
  part} of some $x$ is a part that is not $x$ itself.

\begin{enumerate}[start=2]
  \item $x\ \text{overlaps}\ y =_{df} \text{For some}\ z, z\ \text{is
    a part of}\ x\ \text{and}\ z\ \text{is a part of}\ y$
\end{enumerate}

Just as everything is a part of itself, everything overlaps itself
(when $x = y = z$).

\begin{enumerate}[start=3]
  \item $x$ is a mereological sum of the $y$s $=_{df}$ For all $z$ (if
    $z$ is one of the $y$s, $z$ is a part of $x$) and for all $z$ (if
    $z$ is a part of $x$, then for some $w$, ($w$ is one of the $y$s
    and $z$ overlaps $w$)) \citeyearpar[618--619]{inwagen2006}.
\end{enumerate}

The first part of this last definition---``if $z$ is one of the $y$s,
$z$ is a part of $x$''---specifies that all of the $y$s are part of
$x$.  We cannot say that {\em only} the $y$s are part of $x$, because
the sum of half of the $y$s is also part of $x$.  But we can say that
every part of $x$ overlaps one of the $y$s.  The second part of the
definition---``if $z$ is a part of $x$, then for some $w$, $w$ is one
of the $y$s and $z$ overlaps $w$''---secures this.

There are at least two limitations to this classical formulation of
mereology.  First, it entails that everything that has parts is a
mereological sum.  Second, it does not explain how, if at all,
mereological sums can change their parts.

\subsection{Is everything a mereological sum?}
\label{all-sum}
One problem with the classical formulations of mereology is
that---like van Inwagen's definitions above---they entail that
everything that has parts is a mereological sum.  And though it may be
an unreflective prejudice, I am inclined to believe that mereological
sums are {\em physical}, or material, things.  Material things
certainly have parts, but they are not the only things:

\begin{squote}
The word `part' is applied to many things besides material objects.
We have already noted that submicroscopic objects like quarks and
protons are at least not clear cases of material objects;
nevertheless, every material object would seem pretty clearly to have
quarks and protons as \emph{parts}, and, it would seem, in exactly the
same sense of \emph{part} as that in which a paradigmatic material
object might have another paradigmatic material object as a part.  A
``part,'' therefore, need not be a thing that is clearly a material
object.  Moreover, the word `part' is applied to things that are
clearly \emph{not} material objects---or at least it is on the
assumption that these things really exist and that apparent reference
to them is not a mere manner of speaking.  A stanza is a part of a
poem; Botvinnik was in trouble for part of the game; the part of the
curve that lies below the x-axis contains two minima; parts of his
story are hard to believe\,\ldots\,such examples can be multiplied
indefinitely \citeyearpar[18--19]{inwagen1995}.
\end{squote}

Under our current conception of a mereological sum, things like poems
seem to be included.  For recall van Inwagen's definition:

\begin{squote}
$x$ is a mereological sum of the $y$s $=_{df}$ For all $z$ (if $z$ is
  one of the $y$s, $z$ is a part of $x$) and for all $z$ (if $z$ is a
  part of $x$, then for some $w$, ($w$ is one of the $y$s and $z$
  overlaps $w$)) \citeyearpar[618--619]{inwagen2006}.
\end{squote}

Stanzas are parts of the poem, as are lines, words, and letters.  For
simplicity's sake, though, let us pretend that only words are parts of
poems.  Let the $y$s therefore be all the words in a poem $x$.

Suppose $z$ is the word `bear'.  The word `bear' is a word in the
poem, so it is one of the $y$s.  The first part of van Inwagen's
definition tells us that `bear' is therefore part of the poem.  Now
take the second part of the definition.  We have established that $z$
(`bear') is part of $x$ (the poem), so the antecedent of the
conditional (``if $z$ is a part of $x$'') is true.  If the poem is a
mereological sum, then the consequent must also be true.  There must
be some one of the $y$s that overlaps $z$.  Since $z$ is one of the
$y$s, and everything overlaps itself, the consequent is true.  We can
follow the same steps for every word in the poem.  It seems,
therefore, that poems are mereological sums.

The problem with this conclusion is that {\em order of composition}
matters with poems, but not with sums.  The mereological sum of $A, B,
C$ is identical with the sum of $A, C, B$ and also with the sum of $C,
B, A$ (and so on).  But the poem that begins ``The bear ate
him\,\ldots '' is not identical with the poem that begins ``Ate him
the bear\,\ldots ''.  It appears that words are parts of a poem in a
{\em different way} than things are parts of mereological sums.  And
once we recognize this case, others press in upon us:

\begin{squote}
Now, on the face of it, there would appear to be a wide variety of
basic ways in which one object can be a part of another.  The letter
`n' would appear to be a part of the expression `no', for example, and
a particular pint of milk part of a particular quart; and if these two
relations of part are not themselves basic (perhaps through being
restricted to expressions or quantities), there would appear to be
basic relations of part that hold between `n' and `no' or the pint and
the quart.  It is also plausible that the way in which `n' is a part
of `no' is different from the way in which the pint is a part of the
quart.  For if the two ways were the same, then how could it be that
two pints were only capable of composing a single quart, while the two
letters `n' and `o' were capable of composing two expressions, `no'
and `on' \citep[562]{fine2010}?
\end{squote}

The parthood relation for sets is similar to the parthood relation for
sums.  The set containing the only the letters `n' and `o' has the
letters as parts.  When the letters are parts of a set, their order is
irrelevant, but when the letters are parts of a word, order matters;
hence `no' and `on'.  The parthood relation for sets is also different
from the parthood relation for quantities (of milk):

\begin{squote}
If four quarts compose a gallon the pints which compose the quarts
will compose the gallon in the same way in which they compose the
quarts, whereas, if four sets compose a further set the members of the
sets will not compose the further set in the same way in which they
compose the component sets.  Thus we would now appear to have three
different basic ways in which one object can be a part of another
(pint/gallon, letter/word, and member/set); and once these cases have
been granted, it is plausible that there will be many more
\citep[562]{fine2010}.
\end{squote}

Classical mereology does not appear to be capable of handling the
parthood relation that applies to sets, or that which applies to
words.  But if these things (sets, words) have parts, and if they are
not sums, then classical mereology is flawed, for it entails that
everything that has parts is a sum.

\subsection{Can mereological sums change their parts?}
\label{change}
Many philosophers believe that mereological sums cannot change their
parts.  Since many also believe that chairs and other ordinary things
{\em can} change their parts, concerns arise about the utility of
mereological sums.  If chairs and other ordinary things change their
parts, then they are not sums; what then {\em are} sums?

But not all philosophers do believe that sums cannot change their
parts.  Peter van Inwagen is one.  His argument is very
straightforward:  just as it follows from the definition of
`mereological sum' that things like poems are sums, so it follows that
things like chairs are sums.  Things like chairs can change their
parts.  Therefore, sums can change their parts.

This simple argument requires some supplementation, for there is an
(almost) equally simple argument that purports to show that sums {\em
  cannot} change their parts:

\begin{squote}
Consider an object $\alpha$ that is the mereological sum of $A$, $B$,
and $C$ (that is $\alpha = A + B + C$).  We suppose that $A$, $B$, and
$C$ are simples (that they have no proper parts), and that none of
them overlaps either of the others.  And let us suppose that nothing
{\em else} exists---that nothing exists besides $A$, $B$, $C$, $A +
B$, $B + C$, $A + C$, and $A + B + C$.  Now suppose that a little time
has passed since we supposed this, and that, during this brief
interval, $C$ has been annihilated (and that nothing has been created
{\em ex nihilo}).  Can it be that $\alpha$ still exists?  Well, here
is a complete inventory of the things that now exist: $A$, $B$, and $A
+ B$.  And $\alpha$ is none of these things, for, before the
annihilation of $C$, they existed and $\alpha$ existed and $\alpha$
was was not identical with any of them (all three of them were then
proper parts of $\alpha$).  And nothing can become identical with
something else: $x \neq y \rightarrow \square\ x \neq y$; a thing and
another thing cannot become a thing and itself.  We do not, in fact,
have to appeal to any modal principle to establish this conclusion,
for if $\alpha$ were (now) identical with, say, $A + B$, that identity
would constitute a violation of Leibniz's Law, since the object that
is both $\alpha$ and $A + B$ would both have and lack the property
{\em once having had $C$ as a part} \citep[628]{inwagen2006}.
\end{squote}

Three assumptions are required for this argument to be valid.  First,
the thing that is the sum of $A$ and $B$ before $C$ is destroyed must
be the same thing that is the sum of $A$ and $B$ after $C$ is
destroyed.  That is,

\begin{squote}
If $A$ and $B$ had a unique mereological sum before the annihilation
of $C$, and if $A$ and $B$ had a unique mereological sum after the
annihilation of $C$, the object that was their sum before the
annihilation of $C$ and the object that was their sum after the
annihilation of $C$ are identical \citep[629]{inwagen2006}.
\end{squote}

This assumption seems very plausible.  If $C$ had not been destroyed,
we would have had little or no inclination to say that the sum of $A$
and $B$ at the earlier time is not identical with the sum of $A$ and
$B$ at the later time.  So I do not see why we should think that {\em
  if} $C$ is destroyed, then the sum of $A$ and $B$ at the earlier
time is not identical with the sum of $A$ and $B$ at the later time.

We can generalize this as the {\em existence assumption}:

\begin{enumerate}[ref=(\arabic*)]
  \item If there exists a sum $S$ of some things $A$ and $B$, then $S$
    exists when and only when $A$ and $B$ exist. \label{ass-ex}
\end{enumerate}

The second assumption is that composition is unrestricted (that is,
that universalism is true).  Van Inwagen escapes the conclusion that
mereological sums cannot change their parts by denying that
mereological composition is unrestricted.  He denies that for any
things, there is an object composed of them.  In the example above,
therefore, van Inwagen might deny that, before the annihilation of
$C$, there was a sum of $A + B$.  He would therefore be able to
maintain that $\alpha$ loses a part, going from $A + B + C$ to $A +
B$.  There would be no preexisting $A + B$ to compete with.  As I
argued in Sections \ref{universe}--\ref{universalism}, I think
unrestricted composition---universalism---is true:

\begin{enumerate}[start=2, ref=(\arabic*)]
  \item Universalism is true. \label{ass-uni}
\end{enumerate}

The third assumption is that that there are no {\em co-located}
(completely overlapping) objects.  This assumption, along with the
principle that a thing is located where its parts are located, entails
that for any things, there is at most {\em one} thing composed of
them.  This consequence is generally referred to as {\em uniqueness}.

\begin{enumerate}[start=3, ref=(\arabic*)]
  \item There are no co-located things, and wholes are located where
    their parts are located. \label{ass-co}
\end{enumerate}

Anyone wishing to maintain that things {\em do} change their parts
must reject one of these three assumptions.  Philosophers like van
Inwagen and Merricks reject universalism (and, for good measure, the
existence assumption).  Anyone who accepts universalism will, I think,
also accept that a whole exists whenever its parts do; I can see no
reason why a universalist would deny the existence assumption.  But if
a universalist claims that things can change their parts, she must
therefore reject the third assumption, that there are no co-located
objects.  Below I will examine three different theories that allow for
co-located objects.  If such co-location is unacceptable, we will have
to reject all three theories.

Section \ref{fine-h} will discuss Kit Fine's theory of embodiments.
Section \ref{fine-c} will discuss Fine's more recent theory of
composition operators.  Section \ref{hovda} will discuss Paul Hovda's
theory of temporal mereology.

Before we assess the merits of these theses, however, there is another
possibility that should be addressed.  If we adopt the theory of {\em
  four-dimensionalism}, many of our problems appear to go away.
Unfortunately, new ones arise.

\section{What if we assume four-dimensionalism?}
\label{4d}
I am using `four-dimensionalism' to refer to the conjunction of two
theories.  The first is that things have {\em temporal parts}.  The
second is {\em eternalism}.

Ted Sider presents a relatively clear picture of the doctrine of
temporal parts:

\begin{squote}
Think of your life as a long story.  Let the story be a rather
narcissistic story: cut out all details about everything else except
you.  So the story begins with an infant (or perhaps a fetus).  It
describes the infant developing into a child and then an adolescent.
The adolescent passes into young adulthood, then adulthood, middle
age, and finally old age and death.  Like all stories, this story has
parts.  We can distinguish the part of the story concerning childhood
from the part concerning adulthood.  Given enough details, there will
be parts concerning individual days, minutes, or even instants.

According to the four-dimensionalist conception of persons (and all
other objects that persist over time), persons are a lot like their
stories.  Just as my story has a part for my childhood, so {\em I}
have a part consisting just of my childhood.  Just as my story has a
part describing just this instant, so I have a part that is
me-at-this-very-instant \citeyearpar[1]{sider2001}.
\end{squote}

The claim that we have these {\em temporal
  parts}---me-at-this-instant, or me-as-a-child---relies on a close
analogy between space and time.  It is uncontroversial to claim that
we have {\em spatial} parts.  My foot is a part of me, for instance,
but it is not {\em all} of me (it is a proper part, in mereological
terms).  The proponent of temporal parts claims that, likewise, my
adulthood is a part of me, but it is not all of me.  My childhood
is---or was, if we do not assume eternalism---another part of me.  My
infancy, childhood, adulthood, etc. together {\em compose} me.

This theory of temporal parts is often conjoined with a theory about
time.  This theory is commonly referred to as {\em eternalism}.
According to eternalism, ``time is like space.  There is nothing
special about the things here; things at other places are just as
real; no place is metaphysically distinguished.  Similarly, for the
eternalist, there is nothing special about the present; things at
other times are just as real; no time is metaphysically
distinguished'' \citep[122]{hinchliff1996}.  For the eternalist, there
is a sense in which `there are dinosaurs' is true.  Everyone agrees
that there are no dinosaurs {\em now}; the question is whether the
dinosaurs of the past still exist {\em in the past}.

I have no firm intuition as to whether either conjunct of
four-dimensionalism is true.  I do not know whether things have
temporal parts, and I do not know if the past and future exist.  But
let us suppose for now that four-dimensionalism is true; {\em if} this
assumption is correct, we can explain the existence of ordinary things
in new and interesting ways.

\subsection{Four-dimensional essentialism}
\label{4de}
According to the standard versions of four-dimensionalism, ordinary
things like chairs and statues are {\em four-dimensional spacetime
  worms}.  They are composed of temporal parts or {\em slices}; a
chair might be made up of ``chair-slices'' at $t_{1}$, $t_{2}$,
$t_{3}$\,\ldots\,, etc.  These ``slices'' are generally supposed to
have no temporal duration.  They are {\em extended} in only three
dimensions; their temporal extension is point-sized.

Four-dimensionalism is very commonly conjoined with universalism---the
theory, defended in Sections \ref{universe}--\ref{universalism}, that
for any things, there is something composed of them.  If we assume
universalism, then four-dimensionalism entails that for every set of
temporal slices, there is something composed of them.  There is an
object composed of the first ten years of my life, the Kremlin from
1970--1990, and one second of a puppy's existence in 2020.  This thing
is not, of course, a person; nor is something composed of the first 10
years of my life and the last ten years of someone else's.  Certain
causal or psychological connections must hold between the temporal
parts of a thing in order for it to be a person.

The objects composed of these temporal slices are mereological sums in
the classical sense.  Let us use `Krupkin' to designate the object
made of the first ten years of my life, the Kremlin from 1970--1990,
and one second of a puppy's existence in 2020.  Because the past and
future exist (we're assuming eternalism), Krupkin always has the same
parts.  Strictly speaking, it doesn't ever change its parts.  In 1991
it is true to say ``The Kremlin is not {\em now} part of Krupkin'',
but it is not true to say ``The Kremlin is not part of Krupkin.

If we assume universalism in addition to four-dimensionalism, then not
only does Krupkin not change its parts, it {\em cannot} change its
parts.  It cannot change its parts for the reason given in Section
\ref{change}.  Let us use `Alkin' to designate the object composed of
the first 10 years of my life and the Kremlin from 1970--1990.  Now if
Krupkin could change its parts, it could lose a part.  Suppose it lost
its puppy part.  Then, if it still exists, it would be the object
composed of the first 10 years of my life and the Kremlin from
1970--1990.  But {\em that} object is Alkin; Krupkin would therefore
become identical with Alkin.  Alkin and Krupkin are not identical,
however, because Krupkin has a property that Alkin does not: the
property of having had a puppy as a part.  So Krupkin cannot, in fact,
lose a part; otherwise we would have a contradiction.

Technically, therefore, four-dimensional universalism is a version of
{\em essentialism}---the thesis that things cannot change their parts.
Saying that a thing ``changes'' a part just means that it has some but
not all of that part's temporal parts as parts.  For a chair to lose a
leg is for the chair to have the leg-at-$t_1$ as a part and not have
the leg-at-$t_2$ as a part.  The chair has one of the leg's temporal
parts (the leg-at-$t_1$) as a part, but not both (it does not have the
leg-at-$t_2$).

I am somewhat sympathetic to this view.  In Section \ref{essential} I
will sketch an essentialist theory of things, but one that presupposes
neither temporal parts nor eternalism.  But here I will briefly
examine how a four-dimensional essentialism addresses the issues
related to ordinary things that we have been concerned with.

Four-dimensionalism has two advantages and two disadvantages, when
compared with the three theories below (Sections
\ref{fine-h}--\ref{hovda}).  The first advantage is that
four-dimensionalism does not posits a plurality of {\em kinds} of
things.  The material objects that a four-dimensionalist recognizes
are all mereological sums in the classical sense.  The second
advantage is that four-dimensionalism does not posit co-located
objects.  The first disadvantage is that four-dimensionalism, when
conjoined with universalism, produces a plurality of objects, just as
the three other theories do.  The second disadvantage is that
four-dimensionalism has difficulty distinguishing objects that are
co-located for the entirety of their existence.

\subsection{Four-dimensional solutions}
\label{4ds}
The first advantage of four-dimensionalism---that it does not have to
posit a plurality of kinds of things---is primarily an advantage
relative to Fine's theory of composition operators (Section
\ref{fine-c}).  That theory, as we will see, produces an incredible
plurality, not only of things in general, but of different kinds of
things.  Four-dimensional things are simply mereological sums, in the
classical sense.

The second advantage of four-dimensionalism is that, unlike the three
theories presented below, it does not posit co-located objects.  The
theories of Fine and Hovda, in order to distinguish objects like the
statue and the lump---objects that (currently) share all their
parts---have to posit co-located objects.  But on the four-dimensional
picture, this is unnecessary.  Suppose that the lump is formed on
Monday, and the statue on Tuesday.  The lump therefore has temporal
parts that are ``earlier'' than any of the statue's parts.  They do
not share all their parts, and so are not co-located.  It is true that
they share all their Tuesday parts; the temporal slices that compose
the lump on Tuesday are the same that compose the statue on Tuesday.
But they share parts only at certain times.  They do not share all
their parts at all times.

This leads into a problem for four-dimensionalism, however; it does
not appear to let us differentiate a statue and a lump that {\em
  always} share their parts.

\subsection{Problems for four-dimensionalism}
\label{4dp}
There are two disadvantages to four-dimensional universalism.  The
first is that while four-dimensionalism does not posit a plurality of
kinds of things or a plurality of co-located objects, there is still a
sense in which it is a ``plurality theory''.  Any given temporal slice
is part of a plurality of things.  When I point at my chair, I am also
pointing at a thing composed of my chair and a black bear from the
1800s, as well as a thing composed of my chair and the head of Thomas
Aquinas.  All those things (and {\em many} more) are currently located
in the very same place.

This is certainly bizarre, and it makes four-dimensionalism somewhat
unpalatable, but it does not {\em disprove} the theory.  Unfortunately
there is another disadvantage to four-dimensionalism, one that does
threaten it as a theory.

The second disadvantage of four-dimensionalism is that it has trouble
distinguishing between objects that are co-located for the entirety of
their existence.  Suppose that I have two lumps of clay; I form one
into the top half of a figure and I shape the other into the bottom
half.  Having done this, I stick the two pieces of clay together,
forming a statue.  When I do this I also form a new, larger lump of
clay.  I admire the statue and the lump for a little while, then smash
them with a hammer.

Let $S$ be the thing composed of all the statue-slices.  Let $L$ be
the thing composed of all the larger-lump-slices.  $S$ and $L$ are
mereological sums composed of the very same parts; $S = L$.  But if
the statue had been squashed instead of smashed, $L$ would have
survived; but $S$ would not have survived being squashed.  $L$ has a
property that $S$ does not---the property {\em could survive being
  squashed}---and therefore $S \neq L$.  This is a problem.

The four-dimensionalist could say that the lump would not have
survived being squashed, or that the statue would have survived.
Since there is only one thing under investigation (since $S = L$),
that thing must have a consistent set of properties.  It can't be such
that it would both survive and not survive a squashing.  So the
four-dimensionalist will have to say that one of our two intuitions is
wrong.

But there is another, related, difficulty.  In the case just
presented, the statue is the lump ($S = L$).  But suppose there is a
situation exactly like the one just presented, but in which the statue
and lump are first squashed, then smashed.  In this case, we are
inclined to say that the lump $L^{\prime}$ continues to exist after
the squashing.  Its parts include temporal slices of the clay after it
has been squashed.  In the case of the statue $S^{\prime}$, however,
we are inclined to say that the statue does not have any temporal
parts after the squashing.  The statue is destroyed when it is
squashed.  Since $L^{\prime}$ and $S^{\prime}$ have different parts,
they are not the same thing; $L^{\prime} \neq S^{\prime}$.  The
four-dimensionalist is committed to the claim that whether there is
one thing (a statue that is also a lump) or two things (a statue and a
lump) on the table, and what that thing's (or those things') modal
properties are depends upon whether I squash or smash it.  This seems
highly implausible.  By choosing to squash the statue rather than
smash it, do I thereby {\em make it the case} that there were two
things, rather than one?

The four-dimensionalist will object that, since the future already
exists, it was {\em already} true that there were two things (it has
always been true).  But claiming that it is already the case that I
will squash the statue seems to commit the four-dimensionalist to some
version of {\em determinism}---the thesis that, roughly, the events of
the future are determined, or fixed, to occur.  This may well be true,
but it is largely an empirical hypothesis; to rely on it here would be
unwise.  (If the four-dimensionalist does not assume determinism, and
instead assumes that the future is undetermined, they will presumably
have to say that, since it is indeterminate whether or not I will
squash the statue, the number of things on the table is therefore also
indeterminate.  This seems even worse.)\\

Four-dimensionalism allows for the resolution of a number of puzzles
related to ordinary things.  It does not resolve everything, however,
and it introduces a few problems of its own.  Moreover, it requires a
number of controversial assumptions: the theory of temporal parts,
eternalism, and possibly determinism.  I will therefore set aside
four-dimensionalism, and suppose henceforth that the past and future
do not exist, and that things do not have temporal parts.

In the section that follows, I will examine the first of two theories
by Kit Fine that attempt to explain how things can change their parts.

\section{First theory: rigid and variable embodiments}
\label{fine-h}
Kit Fine's theory of embodiments is presented in his paper ``Things
and their parts'' \citeyearpar{fine1999}.  His objective in this paper
is to present a satisfactory account of how things can change their
parts over time.

In Section \ref{change} I explained why, given three assumptions,
mereological sums do not change their parts.  One of those assumptions
is that a mereological sum exists whenever the things that compose it
exist.  The mereological sum of $a, b, c$ exists whenever (and
wherever) $a, b, c$ exist.

Now, however, there are some things that do not seem to obey this
assumption.  Take a ham sandwich, for example.  It has two slices of
bread and a piece of ham as parts.  It seems to fit the definition of
a mereological sum.  But

\begin{squote}
the sum $a + b + c + ... $ will exist {\em whenever} any of
its components $a, b, c, ... $ exists (just as it is
located, at any time, {\em wherever} any of its components are
located).  It follows that, under the proposed analysis of the ham
sandwich, it will exist as soon as the piece of ham or either slice of
bread exists.  Yet surely this is not so.  Surely the ham sandwich
will not exist until the ham is actually placed between the two slices
of bread.  After all, one {\em makes} a ham sandwich; and to make
something is to bring into existence something that formerly did not
exist \citep[62]{fine1999}.
\end{squote}

If it is true that the sandwich comes into existence only when the
bread and meat are put together, then the sandwich cannot be a
mereological sum in the classical sense.  How, then, is it composed?

\subsection{Composition relations}
\label{rigid}
Fine's suggestion is that things like the sandwich be seen not merely
as the sum of the bread and meat, but as an object composed of the
bread and the meat {\em standing in the relation of betweenness}.
Likewise, a bunch of flowers is not merely the sum of the individual
flowers, but as an object composed of the flowers {\em in the relation
  of being bunched}:

\begin{squote}
Given objects $a, b, c, ... $ and given a relation $R$ that
may hold or fail to hold of those objects at any given time, we
suppose that there is a new object---what one may call `the objects
$a, b, c, ... $ in the relation $R$.'  So, for example,
given some flowers and given the relation of being bunched, there will
be a new object---the flowers in the relation of being bunched (what
might ordinarily be called a `bunch of flowers')
\citeyearpar[65]{fine1999}.
\end{squote}

Fine can be understood here to be modifying our existence
assumption---the assumption \ref{ass-ex} that a sum exists whenever
its parts do.  Instead, something composed of certain objects and a
relation---a composite object that Fine calls a {\em rigid
  embodiment}---exists whenever its parts stand in the given relation.

But rigid embodiments cannot change their parts.  The sandwich is {\em
  destroyed} when its parts fail to stand in the correct relation.
Fine must introduce another kind of thing---a {\em variable
  embodiment}---in order to make it possible for things to change
their parts.  In doing so, he allows for (very many) co-located
objects.

\subsection{How things change their parts}
\label{h-part}
Given certain assumptions, classical mereological sums cannot change
their parts.  Given the same assumptions, rigid embodiments cannot
change their parts either.

Fine stipulates that a thing $x$ composed of $a, b, c$ in relation $R$
exists at a time $t$ if and only if $R$ holds of $a, b, c$ at $t$.  If
$x$ exists at $t_1$, it is because $a, b, c$ are in $R$ at that time.
If at $t_2$, $a, b, c$ are not in $R$---say that only $b, c$ are in
that relation---then $x$ does not exist.  This is the analogue of our
assumption \ref{ass-ex} that a sum exists whenever its parts do.

If our assumption \ref{ass-uni} of universalism holds here, then for
any things ($z$s) in a relation $R$, there is an object composed of
the $z$s in that relation.  Suppose, as above, that there is an object
$x$ composed of $a, b, c$ in relation $R$.  If $a, b$ alone also stand
in $R$, then, {\em if} composition is unrestricted, there is also an
object $y$ composed of $a, b$ in relation $R$.  Objects $x$ and $y$
have different parts and are therefore different things.

If our assumption \ref{ass-co} holds, then there is at most one thing
composed of $a, b$ in relation $R$.

Now suppose $c$ is destroyed or somehow no longer stands in $R$ with
$a, b$.  If we assume that composition is unrestricted and that the
object composed of $a, b$ in $R$ before $c$ is destroyed is identical
with the object composed of $a, b$ in $R$ after $c$ is destroyed, then
we cannot say that $x$ has lost a part and is now composed of $a, b$
in $R$.  There is already an object composed of $a, b$ in $R$---the
object $y$.  If we said that $x$ has lost a part, we would be
committed to the claim that $x = y$, even though previously $x \neq
y$.  If $y$ exists, then we must say that $x$ ceases to exist when it
loses a part ($c$).

(Here I am assuming that relations like $R$ are {\em not} fixed
polyadic relations.  That is, there is not one relation $R$ that can
apply to three things---the schema being $Rxyz$---and a different
relation $R^{\prime}$ that can apply to two things---$Rxy$.  Rather, I
am assuming that relations like $R$ have a single variable ``slot''
that can accommodate {\em plural variables}.  The schema is something
like $Rx$s, where $x$s is a plural variable that can designate any
number of things.  Therefore it is the {\em same} relation $R$ that
applies to $a$, $b$ and to $a$, $b$, $c$.)

Therefore Fine has a separate proposal for objects that can change
their parts.  These things Fine calls {\em variable embodiments}.
Variable embodiments have, at different times, different {\em rigid}
embodiments as parts.  What part a variable embodiment has at a given
time is determined by a function that assigns rigid embodiments to
times.  Fine illustrates this with the water of a river.  There is the
quantity of water that currently composes the river, but there is also
the ``variable'' water, that consists of different quantities of water
at different times:

\begin{squote}
I take it that the water in the river in the second sense---what we may
call the variable water---is now constituted by one quantity of water
and now by another. But what is the variable water?\,\ldots

I would like to take the bold step of supposing that there is here a
hitherto unrecognized method by which wholes may be formed from parts.
In the case of the variable water, there is a function, or
``principle,'' that determines which quantity of water constitutes the
variable water at any given time \citeyearpar[68]{fine1999}.
\end{squote}

In effect, {\em the water} of the river---the thing that is the
variable embodiment---is composed of other things---rigid embodiments
that are in turn composed of water molecules.  The water molecules are
not directly part of {\em the water}, but they are parts of its parts.

\subsection{Problems with the first theory}
\label{problems1}
There are two problems with this theory.  First, it has the
consequence that relations (like that of being bunched) are actually
{\em parts} of things (the relation of being bunched is part of the
bunch of flowers).  Second, it produces a plurality of co-located
objects.

It is certainly not true that a relation is part of a bunch of flowers
in the same way that the flowers are part of the bunch.  Fine
recognizes this; it constitutes one of his objections to a possible
extension of classical mereology.  He observes that one could claim
that mereological sums are made up of things like bread and meat as
well as {\em tropes}, or relations.  But

\begin{squote}
even if we grant that the trope is a part of the sandwich, it is hard
to believe that it is a part in the same way as the standard
ingredients.  Thus we should not regard the sandwich as a
straightforward mereological sum of $s_1$, $s_2$, $h$, and $r$, but in
some other way that has yet to be made clear \citep[64]{fine1999}.
\end{squote}

Fine's theory of embodiments recognizes relations as parts of things,
but in a different way than things like slices of bread are parts of
things.  This is suggested by his notation for a rigid embodiment of
$a$, $b$, and $c$ in relation $R$: $a, b, c / R$.  But this does not
explain in {\em what} way relations are parts of things.  Moreover, it
just seems false that the relation of being bunched {\em is} a part of
the bunch of flowers in any way.  The relation {\em holds} of the
flowers, and it explains why the flowers are a bunch, but that does
not convince me that the relation is in fact part of the bunch.  In
Section \ref{all-sum} we saw examples of many different kinds of
things that have many different kinds of parts.  Tennis matches have
sets, sets have members, poems have stanzas, stanzas have lines, lines
have words.  But {\em relations} were not included in this catalog of
parts.  A theory that has the consequence that relations are parts is,
at least, unintuitive.  (Fine's theory can be modified to avoid this
consequence, as we will see in Section \ref{fine-c}.)

The second problem with Fine's theory is that which affects all three
theories: it posits a plurality of co-located objects, violating our
assumption \ref{ass-co} of non-co-location.  Every variable embodiment
is composed at different times of different rigid embodiments.  At any
given time, therefore, a variable embodiment and the rigid embodiment
that composes it at that time occupy the very same location.

(Because Fine claims that rigid and variable embodiments are different
{\em kinds} of things, he is not required to deny {\em uniqueness}.
The parts of the rigid embodiments (including itself) are parts of it
in a different {\em way} than the parts of the variable embodiment are
part of it.  It remains true that for any parts, there is one whole
composed of them, but Fine considers this statement ambiguous: there
are at least two different {\em senses} of `part' and `compose' that
might be meant here, corresponding to the rigid and variable notions.)

Fine illustrates how his theory entails that even people are
co-located with many other things:

\begin{squote}
An especially important class of cases are those in which the
principle of embodiment is a property $P$ rather than a polyadic
relation $R$.  The rigid embodiment is then of the form `$a/P$' and
may be read as `$a$ qua $P$' or as `$a$ under the description $P$.'
An airline passenger, for example, is not the same as the person who
is the passenger since, in counting the passengers who pass through an
airport on a given weekend, we may legitimately count the same person
several times.  This therefore suggests that we should take an airline
passenger to be someone under the description of being flown on such
and such a flight.  And similarly for mayors and judges and other
``personages'' of this sort \citeyearpar[67--68]{fine1999}.
\end{squote}

One might take this to be an unacceptable consequence of Fine's
theory.  For persons can think, and airline passengers can think as
well.  Are we therefore being asked to accept that there are at least
{\em two} thinking things in every seat on the airplane?

This objection, however, comes from confusing rigid and variable
embodiments.  Rigid embodiments, like the person-as-passenger, cannot
change their parts.  As soon as the person-as-passenger loses {\em
  any} of its parts, it ceases to exist.  I think Fine would say that
rigid embodiments, because they cannot undergo change, cannot properly
be said to think.  Things that {\em do} think are variable
embodiments; for example, the human person that at one time is
composed of some of the same parts as the person-as-passenger (but not
all of the same parts, for the person-as-passenger has a relation as a
part).  If it is only variable embodiments that can think, then a
variable embodiment overlapped by one or more rigid embodiments cannot
result in co-located thinkers.

Moreover, if functions are identified by their assignments of things
to times---that is, extensionally---then there may not be {\em always}
co-located variable embodiments.  If there are no two functions that
assign the very same things to the very same times, then there can be
no co-located thinkers.  (But pluralities of variable embodiments will
overlap at any given time.)

However, even if there cannot be co-located thinkers---thinkers who
completely overlap---why can't there be partially overlapping
thinkers?  For example, Fine's theory may well predict the existence
of a variable embodiment that is composed of the various rigid
embodiments of Alex-as-passenger during a particular flight.  (That
is, since I change some of my parts during a flight, there are a
number of different rigid embodiments that may be called
Alex-as-passenger.  Then the question is whether there is a variable
embodiment composed of each of these rigid embodiments in turn.)  If
there is such a variable embodiment, why shouldn't we expect {\em it}
to think?

I think Fine will have to simply deny that such a thing could think.
There are a number of reasons that may be appealed to: the thing does
not have the right sort of history (it is at best a ``restriction'' of
me---the real thinking thing), or there is a better candidate (me) for
being the one and only thinking thing in that location.  (But I think
as a result of the functioning of my brain; my brain is also part of
the passenger.  How can only one of us think?  I will raise this
objection against Hovda as well in Section \ref{problems3}.)

In any case, it seems simply bizarre that by boarding an airplane I
thereby cause a new thing to come into existence.  If I become a
judge, then according to Fine, a new {\em thing} has come into
existence.  Why not just say that a description is true of me that was
once not true of me?  For

\begin{squote}
suppose that Mary got married at noon.  Her marrying did not make a
wife come into existence: it merely made her become a wife.  Your
reaching the age of 20 did not make a teenager go out of existence; it
merely made you cease to be a teenager.  And so on
\citep[151]{thomson1998a}.
\end{squote}

The consequence of Fine's theory that passengers are things distinct
from people, coupled with the ``explosion of reality'' that occurs
simultaneously, is cause for concern.  I have in fact understated the
size of the explosion, for in addition to the pluralities of
co-located rigid embodiments, there is likely also a plurality of
variable embodiments, each corresponding to a possible function.

But these consequences are not limited to our first theory.  The
second theory, as we will see, results in a similar explosion.

\section{Second theory: composition operators}
\label{fine-c}
Recently Kit Fine has proposed a new analysis of things.  In ``Toward
a theory of part'' \citeyearpar{fine2010}, he suggests that not only
are there a plurality of mereological sums, but that there is a
plurality of {\em kinds of things}; sums are only one kind in a vast
``mereological firmament''.  Fine's theory is extremely interesting,
but ultimately it faces a particularly acute version of the problem of
co-location that faces the other two theories.  For while Fine's
theory of embodiments and Hovda's theory of tensed mereology (Section
\ref{hovda}) predict a plurality of overlapping things, Fine's theory
of composition operators predicts, in addition, a plurality of {\em
  kinds} of things.

Fine's new theory has a number of connections with his theory of {\em
  rigid embodiments} (see Section \ref{rigid}).  That theory posited
{\em relations} as parts of things, but as parts in a different {\em
  way}.  The relation of being bunched was supposed to be part of the
bunch of flowers, but in a different way than the flowers themselves
are part of the bunch.  But it was not explained {\em how} something
can have different parts in different ways.  

One might suppose that if we reject Fine's theory of embodiments, we
can reject this pluralist conception of parthood.  But as we saw in
Section \ref{all-sum}, there are independent grounds for thinking that
there are different ways of being a part.  The way that letters are
parts of words is different from the way members are parts of sets,
and both are different from the way things are parts of sums.  Fine's
new theory begins by defending this pluralist claim about parthood.

\subsection{Problems for pluralists}
\label{sets}
There are a number of objections to Fine's pluralism about parthood.
The first objection is that while parthood is supposed to be
transitive, the membership relation of sets is not.  The letter `n' is
a member of the set \{`n',\{`n',`o'\}\}, but `o' is not.  The
objection claims that sets have {\em members}, not parts, and that
Fine has confused the two.

But while it is true that the membership relation is not the parthood
relation, this is no reason to think that sets do not have parts.  A
given set will have certain members---the $x$s---and certain
parts---the $y$s---and only sometimes will the $x$s and the $y$s be
the very same things.  The set \{`n',\{`n',`o'\}\} has two members
but three parts.  The parthood relation for sets can even be defined
in set-theoretic terms:

\begin{squote}
It may well be thought that the way in which a member is a part of a
set is given, not by the membership relation itself, but by the
ancestral of the membership relation, where this is the relation that
holds between $x$ and $y$ when $x$ is a member of $y$ or a member of a
member of $y$ or a member of a member of a member of $y$, and so on
\citep[563]{fine2010}.
\end{squote}

A second objection is that talk of parthood in connection with things
like sets is somehow metaphorical or non-literal.  We saw above that
van Inwagen admits that many different things are said to have parts.
However, he qualifies this in two ways.  First, he seems to have
doubts (or at least is sympathetic with those who have doubts) as to
whether the non-material things that are said to have parts really
exist:

\begin{squote}
The word `part' is applied to things that are clearly \emph{not}
material objects---or at least it is on the assumption that these
things really exist and that apparent reference to them is not a mere
manner of speaking \citep[19]{inwagen1995}.
\end{squote}

If there are no such things as tennis matches or poems or papers, then
of course they do not have parts.  But I think it is obviously true
that there are such things.  This being so, what does it mean to say
that they have parts?  This is where van Inwagen's second
qualification comes in.  For he suggests not only that the parts of
tennis matches and poems are parts in a different way than are the
parts of a table, but that these different relations of parthood are
only tenuously connected.  Van Inwagen says that the various relations
of parthood (if such there be) are connected only by the ``unity of
analogy'' \citeyearpar[19]{inwagen1995}.  If the only similarity
between the parthood relation for poems and the parthood relation for
chairs is that they share the ``analogy'' of parthood, then is there
anything important or interesting about ``parts'' of poems?  Is the
parthood relation for sets likewise only interesting because of the
analogy with the parthood relation for chairs?

At least in the case of parthood for sets, the notion does not appear
to be wholly metaphorical:

\begin{squote}
In the case of set-membership, there would appear to be nothing that
might plausibly be taken to indicate that the talk of part-whole is
not to be taken literally. A set is indeed composed of or built up
from its members, and we should add that we may meaningfully
talk---and in the intended way---of \emph{replacing} one member of a
set with another.  Thus Aristotle in the set \{Plato, Aristotle\} may
be replaced with Socrates to obtain the set \{Plato, Socrates\}, with
the given set becoming a different set from what it was. In the case
of sets, our conception of members as parts seems to extend all the
way \citep[564]{fine2010}.
\end{squote}

But the second worry raised by van Inwagen remains.  Why should we
think that there is any {\em real} similarity between these different
parthood relations, other than the fact that we call them all
``parthood''?

\subsection{Operationalism}
\label{operation}
Fine's doctrine of {\em operationalism} helps answer this worry.
Various {\em operations} produce different things---mereological
summation produces mereological sums or fusions, the set-builder
produces sets, and so forth.  Parts are therefore {\em things} that
have been ``combined'', through one or more such operations, into a
single {\em thing}.  What is common to all parthood relations is that
from each set of parts is produced a {\em whole} by means of a
composition operator.  From parts (letters, atoms) are made something
else (a word, a set, a chair).  What ties together all the ways of
being a part is that they are involved in a composition operation that
produces a single thing from a number of things:

\begin{squote}
In formulating the principles of mereology, it has been usual to take
the relation of part-whole or some associated relation (such as
overlap) as primitive.  But I believe that, in formulating a more
general theory, it is important to take the operation of composition
as primitive rather than the more familiar relation of part-whole.  In
the case of classical mereology, the operation of composition will
take some objects into the sum or fusion of those objects, while, in
the set-theoretic case, it will take some objects into the set of
those objects; and, in general, the operation of composition will be
the characteristic means (summation, set-builder, and so on) by which
a given kind of whole is formed from its parts \citep[565]{fine2010}.
\end{squote}

Each way of being a part can then be defined in terms of the related
composition operation:

\begin{squote}
Once given a compositional operation, a corresponding relation of part
may be defined in two steps.  We say first that $x$ is a {\em
  component} of $y$ if $y$ is the result of applying $\sum$ to $x$ or
to $x$ and some other objects.  In other words, $y$ should be of the
form $\sum (x_{1}, x_{2}, ... )$, where at least one of
$x_1$, $x_2, ...$ is $x$.  Thus when $\sum$ is mereological
summation the components of an object will be mere parts, and where
$\sum$ is the set-builder the components of an object will be its
members.  We may then define $x$ to be a part of $y$ if there is a
sequence of objects $x_1$, $x_2, ... x_n$, $n$
\textgreater{} $0$, for which $x = x_1$, $y = x_n$, and $x_i$ is a
component of $x_{i+1}$ for $i = 1$, $2, ..., n-1$. The parts
of an object are the object itself, or its components, or the
components of the components, and so on \citep[567--568]{fine2010}.
\end{squote}

The parthood relation for mereological sums can therefore be shown to
exhibit reflexivity, transitivity and anti-symmetry
\citep[568]{fine2010}:

\begin{description}
\item[Reflexivity] Each object is a part of itself.
\item[Transitivity] If $x$ is a part of $y$ and $y$ of $z$, then $x$
  is a part of $z$.
\item[Anti-symmetry] $x$ is a part of $y$ and $y$ of $x$ only when $x
  = y$.
\end{description}

But not all definitions of parthood that issue from a composition
operator will exhibit these features:

\begin{squote}
When the underlying operation is summation, each object will be a part
of itself, since the unit sum of any object is the object itself, but
when the underlying operation is the set-builder, no object will be a
part of itself, since no object is ever an ancestral member of itself
\citep[569]{fine2010}.
\end{squote}

In every case, how some thing is part of a whole (if it is) will
depend on the composition operation that produced the whole.  Other
properties, both of a whole and its parts, will be determined by the
nature of the composition operator that produced it.  Each composition
operation will, according to Fine, be governed by various principles.
The ``formal principles'' govern when composition occurs and when two
products of a composition operation are identical.  The ``material
principles'' govern both how the object ``sits'' in space and
time---whether it has spatial and/or temporal parts (see Section
\ref{4d}) or not---and the specific characteristics of the object
(such as its color and weight).

\subsection{Fine's pluralist account of classical mereology}
\label{classical}
Of the principles sketched above, Fine gives most attention to the
identity conditions for composition operations.  The composition
operation used as a paradigm is the summation operation of classical
mereology.  Fine's exposition of identity conditions for sums relies
on the notion of {\em regularity}:

\begin{squote}
Call an identity condition $s = t$ {\em regular} if the variables
appearing in $s$ and in $t$ are the same.  Thus $\sum (x, y) = \sum
(y, x)$ is regular while $\sum (x, y) = x$ is not
\citeyearpar[572]{fine2010}.
\end{squote}

With this notion in hand, Fine proposes this condition for identity of
sums:

\begin{description}
  \item[Summative Identity] $s = t$ whenever `$s = t$' is a regular
    identity \citeyearpar[572]{fine2010}.
\end{description}

One particularly interesting aspect of this condition is that it
entails four more principles of the summation operation:

\begin{description}
  \item[Absorption] $\sum (..., x, x, ...,
    ..., y, y, ..., ... = \sum (
    ..., x, ..., y, ... )$;
\item[Collapse] $\sum (x) = x$;
\item[Leveling] $\sum (..., \sum (x, y, z, ... ), ..., \sum (u, v, w, ... ), ... ) = \sum (..., x, y, z, ..., ..., u, v, w, ..., ... )$;
\item[Permutation] $\sum (x, y, z, ... ) = \sum (y, z, x,
  ... )$ (and similarly for all other permutations)
  \citep[573]{fine2010}.
\end{description}

We can define other compositional identity criteria (e.g., sequences)
in terms of which of these principles apply to their compositional
operation.  But we may also devise new principles by which we may then
define new types of composition:

\begin{squote}
We should note that there would appear to be no good reason to require
that the defining principles for the various operations should be
limited to the particular principles (C [collapse], L [leveling], A
[absorption], and P [permutation]) that we used in characterizing
sums; for any set of regular identities would appear to be equally
well suited to defining a basic form of composition, so long as they
conform to Anti-cyclicity.  Indeed, I would conjecture that any such
set of principles in fact will correspond to a form of composition and
a corresponding form of whole.  How the resulting forms of composition
and whole might be organized is an interesting question, but it should
be apparent that the approach will lead to an infinitude of forms of
composition, each differing from one another in how exactly the
identity of the resulting wholes is to be
determined. \citep[575--576]{fine2010}.
\end{squote}

It is at this point that the importance of Fine's theory becomes
obvious.  In Section \ref{universalism} I argued for universalism, but
according to Fine's new theory, there are many different {\em kinds}
of universalism.  One might be committed to the existence of
dogbushes, and so to unrestricted {\em mereological} composition, but
deny the existence of some other kind of thing (for example,
groups---see Section \ref{groups}).  Or one might defend unrestricted
composition of other kinds of things while claiming a restriction on
mereological composition.

\subsection{Composition operators and time}
\label{c-change}
In Fine's theory of embodiments (Section \ref{fine-h}) he recognizes
at least two kinds of things: rigid and variable embodiments.  Rigid
embodiments have their parts timelessly.  They exist when and only
when their parts exist, and at all times during which they exist, they
have the same parts.  Rigid embodiments, therefore, cannot change
their parts.  Variable embodiments {\em can} change their parts,
however; what rigid embodiment a given variable embodiment is composed
of at a given time is determined by a function.

Fine's account of composition operators explains how the create things
that, like rigid embodiments, do not (and presumably cannot) change
their parts.  He does not address how composition operators might
produce things that, like variable embodiments, {\em can} change their
parts over time; he opens his paper on composition operators by saying
that ``it is not [his] aim to discuss either the notion of relative
part or its connection with the absolute notion''
\citeyearpar[559]{fine2010}.  However, I think we can imagine a few
ways in which Fine's theory of composition operators might be adapted
to relative or temporary parthood.

One way to adapt Fine's new theory so as to allow things to change
their parts would be to think of the mereological sum operator (the
sum-builder) as operating {\em not} on ordinary things but on
things-at-times.  By ``things-at-times'' I mean {\em temporal slices}
of things.  For example, a temporal slice of a chair is an object that
resembles a chair but has no temporal duration.  The sum-builder for a
chair would take such temporal slices and compose from them a chair.
The chair would have temporal duration and would be capable of
changing its parts.  (As in the four-dimensional picture presented in
Section \ref{4de}, for a thing to gain or lose some part $x$ would be
analyzed as: having some but not all of $x$'s temporal parts as
parts.)

One problem with this proposal is that it requires temporal parts (see
Section \ref{4d}).  For it seems that composition operators like the
sum-builder operate on {\em things}.  If sum-builder can operate on
things-at-times, then we commit ourselves to the claim that
things-at-times are {\em things}.  And what things could they be but
temporal parts of other things?

If we don't want to presuppose temporal parts, the sum-builder has to
be somehow \emph{dynamic}. It can't just take things, compose them and
be done---it has to \emph{add and remove things over time}.

Making sense of a dynamic operator might allow us to avoid
presupposing {\em eternalism} as well.  If the sum-builder composes a
chair ``in one go'' out of different temporal slices, then the future
slices would have to already exist in some sense.  How else could the
sum-builder operate on them?

One way to make sense of a dynamic operator is by relativizing the
sum-builder to times.  We can think of the operator as taking some
things at a time and producing a sum: $G = \sum _{t} (S)$.  (There are
two interpretations of $\sum _{t}$: we might say that the composition
operator (re-)produces a sum at a number of different times $t$, or we
might say that there is a {\em different} composition operator at each
time $t$.  I will suppose that the former is correct.)

A second way to make sense of a dynamic operator is the way that Fine
makes sense of variable embodiments.  Variable embodiments were
composed of different things at different times according to a
function.  Likewise, a composition operator that produces a thing that
has different parts might do so by means of a function.  Rather than
operating directly on some things, the operator could apply to a
function.  Instead of

\begin{displaymath}
\sum (a, b, c, ... )
\end{displaymath}
we would have something like this:

\begin{displaymath}
\sum ( f )
\end{displaymath}

On this understanding of a ``dynamic operator'', the only {\em
  component} (see Section \ref{operation}) of the object is the
function, but at any given time it has as parts (in some sense)
whatever objects the function assigns to that time.

On either understanding of the ``dynamic operator'', things can change
their parts, but things can (and will) also be co-located.

\subsection{Problems with temporally relativized operators}
\label{problems2a}
Suppose we take the first suggestion and relativize the sum-builder to
a time.  The primary problem with this is that it leads to a great
plurality of co-located objects.  What is particularly objectionable
in this case is that the objects are all of different {\em kinds}.

To see why this is so, let us suppose that ordinary things like chairs
and statues are produced by means of the temporally relativized
sum-builder $\sum_{s_t}$.  A statue might then be produced thus:

\begin{displaymath}
ST = \sum_{s_t} (a, b, c, ... )
\end{displaymath}

Now since we are assuming that universalism---assumption
\ref{ass-uni}---is true, there is an object composed of all the parts
of the statue except for the left hand:

\begin{displaymath}
LF = \sum_{s_t} (a, b, ... )
\end{displaymath}

At $t$, these are obviously different things.  Since $ST$ and $LF$
have different parts, $ST \neq LF$.  But now suppose the statue
changes its parts---by losing its left hand---while the lump of clay
$LF$ remains the same.  We will have these two objects:

\begin{displaymath}
ST = \sum_{s_t} (a, b, c, ... )
\end{displaymath}

\begin{displaymath}
LF = \sum_{s_t} (a, b, ... )
\end{displaymath}

But now we are committed to it being the case that $ST = LF$.
Identity is not a temporary or contingent relation.  If any two things
are actually the same thing, they are necessarily so.  That is, $ST =
LF \rightarrow \square ST = LF$.  It cannot ever be the case that $ST
\neq LF$.  But at $t$, this was apparently so.

The way to avoid this contradiction is to deny that the statue and the
lump are produced by the same composition operator.  The statue must
be seen to be the product of the ``statue-builder''---$\sum
_{st_t}$---and the lump the ``lump-builder''---$\sum _{lump_t}$.

But the products of different operators are of different {\em kinds}.
The statue and the lump, therefore, are different kinds of things---to
say that both are ``physical objects'' or ``ordinary things'' is
simply to bring two heterogeneous kinds under one label.

Thus, in the same fashion as Fine's theory of embodiments, we find
ourselves rejecting non-co-location---assumption
\ref{ass-co}---without denying uniqueness.  Because things that are
co-located (like the statue and lump) are different kinds of things,
they have their parts in different ways.  When we say ``These things
are parts of the statue'' and ``These things are parts of the lump'',
we are using `parts' differently in each case.

But just as with Fine's previous theory, the theory of composition
operators creates an ``explosion of reality'', with the additional
strangeness of a plurality of {\em kinds} of things.  Since we have
allowed that the statue and the lump may be different things composed
of the same sums, why stop there?  We can introduce more composition
operators that produce distinct objects.  Where we see a statue and a
lump, why not suppose that there is a great plurality of objects, each
of a different kind and with slightly different properties?

This is not a particularly attractive position, but it is not
indefensible (see \citet[Section 4]{bennett2004}).  Since we have
already allowed a plurality of scattered objects like archipelagos and
dogbushes, why not allow a plurality of co-located objects?

One additional difficulty for this theory is that it is unclear how
Fine would avoid there being co-located thinkers.  When discussing
Fine's theory of embodiments (Section \ref{fine-h}) we saw that he has
to claim that ``qua-objects'' like airline passengers
(people-as-passengers) don't actually think, but that it is
nonetheless correct to say that passengers think.

Fine's theory of operators may have to include a similar clause.  If
the theory is correct, then there will no doubt be many things
composed of the same atoms that compose me, but none of them will
think.  Only I will be thinking.  Fine needs an explanation both of
why there can only be one thinking thing composed of any given
parts---why only one ``thinker-builder'' can apply to some
things---and of what the ``thinker-builder'' is.  What builds me?

\subsection{Problems with functional operators}
\label{problems2b}
The second way that I suggested we make sense of a ``dynamic
operator'' was to understand it as applying not to things but to a
single function:

\begin{displaymath}
G = \sum ( f )
\end{displaymath}

The function assigns certain things to certain times, so to determine
what is part of $G$ at a given time, we appeal to the function: what
thing or things does the function assign to that time?  In Fine's
theory of embodiments (Section \ref{fine-h}), the things that composed
a variable embodiment $V$ at a given time was the rigid embodiment
that is determined by $V$'s function.  Non-dynamic operators (Section
\ref{operation}) produce things that do not change their parts, much
as rigid embodiments do not change their parts.  The function of a
dynamic operator might therefore assign products of non-dynamic
operations to times, just as the functions of variable embodiments
assign rigid embodiments to times.

The major advantage of this kind of dynamic operator is that it does
not result in a plurality of different kinds of things.  The statue
and the lump can now be built from the same composition operator.
That operator, in producing them, will of course be operating on
different functions; the ``statue function'' will assign different
pieces of clay to different times than will the ``lump function''.
Since the operator $\sum$ will be operating on two different functions
(rather than on the same objects), it will produce two different
things.

This version of the dynamic operator also blocks the entailment from
co-location to uniqueness.  The statue and the lump may be in the same
place at the same time, but they do not have all the same parts.  The
statue has the statue-function as a part, and the lump has the
lump-function as a part.  (If these functions were not parts, there
would be no way to distinguish the statue and the lump.)  This version
of the dynamic operator therefore treats functions as parts of things,
in much the same way as Fine's theory of embodiments treats relations
as parts of things.  Just as it seems false to say that relations are
parts of things (in any sense), so it seems false that functions are
parts of things.

Moreover, this version of the dynamic operator still leaves us with a
great plurality of objects.  Indeed, there seems no principled limit
to the functions that might, through an application of a composition
operator, give rise to new things.  If there is a ``statue function''
that assigns pieces of clay to times for every moment at which the
statue in question exists, then it seems arbitrary to say that there
isn't a function that is identical but for its beginning 10 minutes
later.  If we apply the same operator to {\em that} function, do we
get another thing that overlaps the statue for most of its existence?
What about a function that leaves off the first 10.1 minutes?  What
about a function that assigns Tacitus to 100 AD and my cat to today?
Is there a thing that existed momentarily last year, then exists for
all of today, then ceases to be?

Another difficulty with relying on functions is that it may also
assume eternalism.  If the function seeks to assign objects to past or
future times, we may be thereby committed to the existence of those
past and future objects and times.\\

Whether we construe the dynamic operator as relative to time or
operating on functions, we get an absurd number of co-located things
of different kinds.  Of course, in addition to this plurality of
``dynamic'' objects, there is {\em also} a plurality of ``static''
objects.  This is analogous to the result of Fine's theory of
embodiments, with pluralities of both rigid and variable embodiments.

There are enough problems with Fine's theory to encourage us to look
for something better.  The third theory I will examine has a number of
similarities to those of Fine (especially his theory of embodiments),
but it has some advantages as well.  Unlike previous theories, Hovda's
theory of tensed mereology does not have the consequence that
relations or functions are parts of things.  Nor does not posit a
large number of different kinds of things.  But it does nonetheless
posit what I take to be an objectionable number of overlapping things.

\section{Third theory: tensed mereology}
\label{hovda}
Paul Hovda has proposed an amended version of classical mereology that
presupposes neither eternalism nor presentism, and that allows for
mereological sums to change their parts over time.  (In his paper
``Tensed mereology'' \citeyearpar{hovda2011} he in fact offers three
versions of his theory; I will focus on the first formulation.)

{\em Tensed mereology} is similar to Fine's theory of embodiments in
at least one important way.  According to both theories, some relation
or property is required to specify when and where a sum of some things
exist.  Fine's example was of a bunch of flowers; the flowers compose
the bunch when and only when the relation of being bunched holds of
them.

Hovda uses `condition' to cover both relations and properties:

\begin{squote}
We will want a ``condition'' to be an open sentence that may have more
than one free variable, \emph{together with a specification of a
  target variable}. For example, we will want to consider
``conditions'' like `$y$ loves $x$', with `$x$' as target.  This is
because we want to consider, in effect, for each object that might be
a value of the variable `$y$', the property of being a thing loved by
that object.  The point of this may be brought out by an example.  We
want as an instance of the plenitude principle, roughly this: that for
every $y$, if $y$ loves at least one thing, then there is a thing $b$
such that $b$ is a fusion of the condition (with respect to $x$) `$y$
loves $x$' (i.e., a fusion of the condition of being loved by $y$)
\citeyearpar[sec. 1.1n2]{hovda2011}.
\end{squote}

With this notion in mind, Hovda replaces classical mereological sums
with {\em diachronic fusions}:

\begin{description}
  \item[Diachronic fusion] An object $b$ is a ``diachronic fusion'' of
    a condition if and only if it is always the case that (1) every
    $x$ that meets the condition is part of $b$; and (2) every part of
    $b$ overlaps something that meets the condition
    \citeyearpar[sec. 1.1]{hovda2011}.
\end{description}

Like Fine, Hovda takes composition to be unrestricted: ``every
suitable condition has a diachronic fusion (where a condition is
suitable iff it is not always empty; i.e., it is suitable iff at some
time, at least one thing satisfies it)'' \citeyearpar[sec.
  3.1]{hovda2011}.  Not only does every suitable condition have a
diachronic fusion, but ``it is always the case that every suitable
condition has a diachronic fusion''
\citeyearpar[sec. 3.1.1]{hovda2011}.  In other words,

\begin{itemize}
  \item For every condition $K$,
  \item if it is ever the case that something satisfies $K$, then
  \item there is exactly one thing $b$ such that at at any time $t$
    during which anything satisfies $K$, all the things that satisfy
    $K$ at $t$ are parts of $b$ at $t$ and all the parts of $b$ at $t$
    overlap at least one of the things that satisfies $K$ at $t$.
\end{itemize}

This has the welcome consequence that there are no two things that are
{\em always} co-located.  However, it does mean that there will be
very many things that are co-located at some time or other, and this
may cause problems.  Things being co-located at a time will cause
problems if we make two plausible assumptions about how parts work.
These two assumptions are {\em strong supplementation} and {\em
  anti-symmetry}.

Strong supplementation is a common assumption in mereology to the
effect that if everything that overlaps one thing overlaps another
thing, then the first thing is part of the second.  It may be
formalized as:

\begin{displaymath}
\forall x \forall y ( \forall z ( z \circ x \rightarrow z \circ y )
\rightarrow x \leq y )
\end{displaymath}
(Here `$x \circ y$' means `$x$ overlaps $y$' and `$x \leq y$' means
`$x$ is part of $y$'.)

Anti-symmetry is the assumption that if two things are parts of each
other, it follows that they are really the {\em same thing}:

\begin{displaymath}
\forall x \forall y ( ( x \leq y \wedge y \leq x ) \rightarrow x = y )
\end{displaymath}

At this point, a problem arises:

\begin{squote}
Consider a (diachronic) fusion of the condition on $x$ that `($x$ is
Socrates and Socrates is sitting) or ($x$ is Plato and Socrates is not
sitting).'  Suppose $\beta$ is such a fusion.  Then, when Socrates and
Plato are sitting at dinner, $\beta$ exists and it should hold (then)
that everything that overlaps Socrates overlaps $\beta$ and
vice-versa.  By strong supplementation, Socrates and $\beta$ then bear
$\leq$ to one another.  By anti-symmetry, they are then identical.
But later, when Socrates stands, $\beta$ will then (by similar
reasoning) be identical with Plato, yet Socrates won't be identical
with Plato, so Socrates and $\beta$ are then non-identical.  I take
this result to be unacceptable: once identical, always identical,
certainly if ``both'' exist \citeyearpar[sec. 3.1.2]{hovda2011}.
\end{squote}

One of our assumptions---strong supplementation or
anti-symmetry---must be withdrawn.  Hovda chooses to deny
anti-symmetry:

\begin{squote}
Instead of saying that it is always true that any mutual parts are
identical, [we] will say, roughly, that any things that are always
mutual parts are identical \citep[sec. 3.1.2]{hovda2011}.
\end{squote}

The rejection of anti-symmetry helps to show why Hovda's theory will
result in there being, at particular times, many co-located objects.
(Fine's theory of embodiments did not have to reject anti-symmetry
because his co-located objects did not share all the same parts---they
had different relations as parts.  Fine's theory of operators did not
have to reject anti-symmetry because his ``co-located'' things had the
same parts, but in different ways.)

\subsection{Problems with the third theory}
\label{problems3}
Hovda's tensed mereology avoids the conclusion that conditions or
relations or functions are parts of things.  The theory does this,
however, only by rejecting our assumption \ref{ass-co} of both
non-co-location {\em and} uniqueness.  When Tibbles is sitting, the
fusion of the condition of being Tibbles and the fusion of the
condition of being Tibbles sitting have all the same parts (in the
same sense of `part').  The two things are distinguished by the fact
that it {\em will be} or {\em was} the case that they do not share all
the same parts.

Like Fine's theory of embodiments, Hovda's theory produces a huge
number of things, most (perhaps all) of which are temporarily
co-located with other things.  For instance, when Tibbles the cat is
sitting, there are also an unknown number of other objects co-located
with Tibbles: the fusion of the condition of being Tibbles sitting,
the fusion of the condition of being Tibbles while less than 3 years
old, the fusion of the condition of being Tibbles with a full stomach,
etc.  We should therefore pose to Hovda the same objection, by
Thomson, that we posed to Fine's theory of embodiments in Section
\ref{problems1}: is it really true that when a cat sits, it thereby
comes to pass that a new thing (a sitting-cat) comes into existence?
When (if) I graduate college, does a college-graduate pop into being?

One might object further that Hovda's theory results in temporarily
co-located thinkers, which would be a grave objection indeed.  We
should all agree that the object that fuses the condition of being
Tibbles surely thinks.  How, then can the object that fuses the
condition of being a sitting cat {\em not} think?  Like Fine in
Section \ref{problems1}, Hovda must reply that the fusion of the
condition of being Tibbles sitting is just not the kind of thing that
can think.

It is not obviously false that the fusion of being a sitting cat
cannot think, but what about the fusion of being a {\em thinking} cat?
That is, can the fusion of being Tibbles while thinking itself think?
I am inclined to think so.

Hovda could object that this thing is identical with the fusion of the
condition of being Tibbles.  We saw above that any ``two'' diachronic
fusions that always share the same parts are really the very same
thing.  But is it true that Tibbles is {\em always} thinking?  Tibbles
probably does not think when unconscious (and not dreaming).
Moreover, depending on when the life of Tibbles begins, there may be a
period in which Tibbles {\em can't} think; for example, while a fetus.

If one wants to quibble about whether or how cats think, we can run
the same argument with people.  There is a fusion of the condition of
being Alex; this fusion is me, a thinking thing.  But there is also
the condition of being Alex thinking.

It might seem that {\em by definition} this thing thinks; after all,
the condition that is fused makes direct reference to thinking.  But
it is important to keep in mind that ``the formal theories behind the
idea of a fusion make no mention of change, or its absence, or of the
essential natures of [fusions]'' \citep[sec. 1]{hovda2011}.  The
condition being fused merely picks out the objects that satisfy it.
Moreover, any two fusions which always have the same parts are
identical; if it was true that we exist when and only when we think,
the fusion of the condition of being Alex thinking would {\em be} the
fusion of the condition of being Alex.

Nonetheless there is reason to think that the fusion of the condition
of being Alex thinking {\em does} think.  At any time during which
this fusion exists, it has all the same parts as I---the fusion of the
condition of being Alex---have.  If I think, it seems that this is as
a result of the functioning of my brain; and my brain is also part of
the fusion of the condition of being me thinking.  How can only one of
us think?  If we are not to have co-located thinkers, we must deny
that if a thing is thinking at a given time, it is in virtue of the
structure or functioning of its parts at that time.  We must instead
say that it is in virtue of its history (only the fusion of the
condition of being Alex is a {\em human}) or other properties.

There may be a satisfactory reply here, one that allows Hovda to avoid
co-located thinkers.  But his theory still entails a curious number of
overlapping things.  A theory that gave us fewer would, I think, be
better.

\section{Does the Supreme Court exist?}
\label{groups}
These three theories are bizarre---they predict a huge number of
co-located things always popping into and out of existence.  In the
case of Fine's theory of composition operators (Section \ref{fine-c}),
there may be a swarm of different kinds of things as well.  

But bizarre as this is, there are some reasons to adopt such a theory.
Not only does it appear to follow from other theoretical assumptions
(universalism, change of parts, the negation of four-dimensionalism),
but it allows us to account for certain aspects of ordinary belief.
For example, the three theories we have examined can help us describe
the existence of {\em groups} better than can the theory of classical
mereology alone.

\subsection{What's a group?}
\label{what-g}
Groups, as I will understand them, are things that have other things
as members.  Families, sports teams, support groups, and committees
are all examples of groups.

But why should we suppose that groups are {\em things}?  For example,
one might claim that we use `the Dunns' to refer {\em plurally} to me
and the other members of my family.  I say things like ``The Dunns are
fine people''; the term obviously does not function as a singular
term.  The suggestion is that `the crew' behaves similarly.  When I
say that the crew exists, it would then {\em not} follow that there is
a {\em thing} composed of the crewmembers.  Saying ``The crew exists''
would instead be equivalent to saying ``The crewmembers all exist''.

I do not think that this a plausible claim.  Recall the analogy I
tried to draw between `the crew' and `the Dunns'.  On closer
inspection, this analogy appears weak.  A stronger analogy would be
between a term like `the crew' and a term like `the Dunn family'.
`The Dunn family' is {\em not} a plural referring expression.  It is
used to refer to a {\em thing}.  If I talk about the Dunn family, I
will say things like ``The Dunn family is waning'', or ``The Dunn
family must regain its political power''.  The term `the Dunn family'
is a singular term that designates a thing---the family.

`The crew' appears to behave like `the Dunn family' and not like `the
Dunns'.  We say things like ``There is a skeleton crew on board'', or
``The crew is small for such a large ship'', and ``The crew is
abandoning the ship''.  We so also say things like ``The crew {\em
  are} abandoning the ship'', but this may be a case of {\em
  non-literal} speech; `the crew' is being used non-literally to refer
to the crewmembers.  If this is not non-literal speech, then it seems
most likely that `the crew' is {\em ambiguous}: it can be used to
refer either to {\em the crew} or to the crew{\em members}.  In the
former case, `the crew' is used as a singular term.

Similar considerations apply to terms like `team' as well.  `The Reed
College women's rugby team' is a singular term, for it behaves in the
same ways as do `the crew' and `the Dunn family'.  We say things like
``The Reed College women's rugby team is going to win'', or ``The Reed
College women's rugby team is in Seattle this weekend''.  However,
team names are often used (whether non-literally or not) to refer to
the team-members, rather than to the team itself.  This is often due
to pluralized team names.  The Reed College women's rugby team is
called `The Badass Sparkle Princesses'.  This leads us to say things
like ``The Badass Sparkle Princesses are on a losing streak''.  Here
we are led by the plural construction to---perhaps unconsciously---use
the term to refer not to the team itself but to the players.  The
Badass Sparkle Princesses {\em is} a rugby team, but it is far more
natural to say that the Badass Sparkle Princesses are rugby players.

Yet even if it is agreed that groups are things, why shouldn't we
think they are just sets, or sums?  Why suppose that there is another
kind of thing?

\subsection{Groups and sets}
\label{group-set}
I have suggested that families, crews, and other groups are, strictly
speaking, things.  But one might object that there is no need to
suppose that groups are some special {\em kind} of thing; we can
identify families, crews, courts, etc.\ with {\em sets}, and avoid the
``ontological clutter'' that would result from the introduction of
groups.  Groups, it may be said, are really just sets.  When we speak
of a group of people, we are actually referring to the set of which
they are members.

But there are some reasons why it seems incorrect to identify groups
with sets.  Take the Supreme Court.  It seems that any attempt to
identify the Supreme Court with the set of the Supreme Court justices
will not succeed.  This is because the membership of the Supreme Court
changes over time, while the members of a set do not.  The set
containing the 1990 justices is a {\em different} set from the set
containing the 2012 justices, but the 2012 Supreme Court is not a
different entity than the 1990 Court.  (We may of course say things
like ``it's a different court now'', but by that we mean only that it
is composed of different people, and so may rule differently---note
that we do {\em not} say ``it's a different Court now''.)

In Section \ref{set-id} I will re-examine these arguments against
identifying groups with sets.  But let us suppose for now that they
are correct.

\subsection{Groups and sums}
\label{group-sum}
Even if it is granted that groups such as the Supreme Court are not
sets, it may be objected that groups are therefore mereological sums
(see Section \ref{tech}), like chairs and people.  But it seems that

\begin{squote}
membership in the Supreme Court is very different from
the part-whole relation on material objects.  The part-whole relation
on material objects is a transitive relation.  Thus if one identified
the Supreme Court with a material object and Justice Breyer with a
part of it, then one would be forced to conclude that Justice Breyer's
arm must be a part of the Supreme Court as well.  Yet, it is plain
that Justice Breyer's arm is neither a part nor a member of the
Supreme Court \citep[136--137]{uzquiano2004a}.
\end{squote}

If we are going to attempt to account for groups with Fine's theory of
embodiments (Section \ref{fine-h}) or Hovda's theory of tensed
mereology (Section \ref{hovda}), we must accept this strange result,
and identify groups with sums.  But if we adopt Fine's theory of
composition operators (Section \ref{fine-c}), we do not need to
identify groups with either sets or sums.

According to Fine's theory of embodiments, when we say ``The Supreme
Court has become more diverse over time'' we are referring to a
variable embodiment that is composed of different rigid embodiments at
different times.  These rigid embodiments are things composed of
justices (for example, Rehnquist, Stevens, O'Connor, Scalia, Kennedy,
Souter, Thomas, Ginsburg, and Breyer) in a certain relation (that of
being part of the Supreme Court).  The rigid embodiment $S =$
(Rehnquist, Stevens, O'Connor, Scalia, Kennedy, Souter, Thomas,
Ginsburg, Breyer)/$R$ exists when and only when those justices stand
in that relation; when Rehnquist died, $S$ ceased to exist.  But the
variable embodiment that is the Supreme Court did not cease to exist;
it was simply no longer composed of {\em that} rigid embodiment.

According to Hovda's mereology, the Supreme Court is the diachronic
fusion of the condition of being the Supreme Court.  It is a scattered
object that overlaps the individual justices as well as other things
at various times (like the fusion of the condition of being Rehnquist,
Stevens, O'Connor, Scalia, Kennedy, Souter, Thomas, Ginsburg, or
Breyer, with which it is temporarily co-located).

The treatment of groups from within the framework of Fine's theory of
composition operators is somewhat more involved.  Fine's theory has
the advantage of treating groups as different than sums or sets, but
it also requires treating different groups {\em themselves} as
different kinds of things.

\subsection{Composition operators and groups}
\label{group-fine}
To see why the theory of composition operators must treat different
groups as different kinds, we must first recognize that a set of
people can compose more than one group at a time.  Suppose that all
and only the members of the Supreme Court in 2004 are part of the
Special Committee on Judicial Ethics.  In this case ``The Supreme
Court share[s] all of its members with the Special Committee on
Judicial Ethics as of a certain time'' \citep[151]{uzquiano2004a}.  It
seems false to say that, in 2004, the Supreme Court was identical with
the Special Committee.  But if the Supreme Court, $G$, is $\sum _{t} (
S )$ and the Special Committee, $C$, is also $\sum _{t} ( S )$, then
how can we deny that $G = C$?

Just as the co-located statue and lump were produced by means of
different operators, so the Supreme Court and the Special Committee
must be produced by means of different operators.  The Supreme Court
will be the product of some operator $\sum _{sc}$ and the Special
Committee of $\sum _{sp}$.  Since these two things are the products of
different operators, they are not identical.

But just as the statue and the lump are therefore different kinds of
things, so the Supreme Court and the Special Committee must now be
recognized as not merely different groups, but as different {\em
  kinds}.  There may be a greater resemblance between their two kinds
than there is between things like sums and sets, but ultimately they
have been estranged.  Using `group' to refer to both is simply
categorizing two kinds under a common label.

Is there anything wrong with this conclusion?  It does seem bizarre in
some ways.  For it is clear enough that one person (or group) may be a
member of an indefinite number of groups; each of these ``groups''
will therefore be a product of a different composition operator.  And
each will, strictly speaking, be a different kind of thing.

Previously we had a relatively tidy ontology.  There were sums, and
sets, and other well-known kinds; but now each task force or
subcommittee is potentially a kind unto itself.  Fine recognizes that
his approach ``will lead to an infinitude of forms of
composition\,\ldots a vast mereological firmament''
\citeyearpar[576]{fine2010}.  But he does not consider this to be a
drawback.

\section{Lessons}
\label{lessons-p}
The objections I have raised in this section by no means show that any
of the three theories presented are false.  But I would prefer a
solution that does not postulate such bizarre pluralities.  In Section
\ref{set-id}, therefore, I set aside the theories of this section and
look at a new possibility.  I will re-examine the thesis that
groups---all of them---really are identical with sets, and that
ordinary things are identical with sums.  This will lead to some
strange consequences, but it may be that they are {\em less} strange
than the ``explosion of reality'' that we otherwise face.

First, however, there is a question that arises for each theory: can
it explain (or be supplemented with an explanation) of why we believe
things that the theory denies?

\subsection{Can the plurality theories explain what we believe?}
\label{explain-p}
In Section \ref{stroud} I claimed that a theory that denies that there
are chairs should be supplemented with an explanation of why we
nonetheless believe that there are chairs.

None of the theories I have proposed deny that there are chairs, but
they do make other unexpected claims that conflict with certain of our
beliefs.  Therefore, the theories should be supplemented with
explanations of why we hold these beliefs.

If the plurality thesis is right, why do we believe that there are
chairs when we {\em don't} believe that there are millions of other
objects?  Why, when there is a ten-pound chair in an otherwise empty
room, are we inclined to say that there is just {\em one} thing that
weighs ten pounds?  What is so special about the chair that promotes
it to our attention out of the many objects in the room?

It may be, in fact, that we {\em do} believe that there are co-located
things.  Many philosophers believe that there are co-located statues
and lumps of clay; do any normal people hold this belief too?
If so, then the question become: why do people believe in only {\em
  some} co-located things?

One answer might be simply that things like chairs, statues, and lumps
matter to us more than the other things.  This is similar to Trenton
Merricks' explanation of why we believe that there are chairs (Section
\ref{universe}).  Merricks denies that there are chairs, but claimed
that we believe there are chairs because {\em things arranged
  chairwise} matter to us.  Because they matter, we have terms to
refer to them; for the sake of convenience (or for some other reason)
we use singular terms to refer to things arranged chairwise, and so we
are fooled by the grammar into thinking that there are chairs.

Likewise, a philosopher like Fine or Hovda could claim that chairs
(which do exist) matter to us more than the plurality of co-located
objects that share parts with the chair.  We introduce terms to pick
out one object from among the plurality (how this happens is a
difficult question) and ignore the others.  The things that do not
matter of course remain; ``we just pay most of them no attention''
\citep[356]{bennett2004}.

\subsection{Is there an alternative to the plurality?}
We have examined three different theories that account for the
existence of chairs.  Each has the difficult consequence that there
are pluralities of co-located objects.  Once we have admitted that,
for instance, the same matter might compose a statue and a lump, we
have trouble resisting the idea that there might be {\em more} things,
of other kinds, composed of that same matter.  (Some theories that I
have not discussed, like Thomson's \citeyearpar{thomson1998a}, are
also forced to posit co-located kinds.)  Where we might take there to
be one thing (or maybe two), we now seem committed to there being a
huge number of co-located things.  I believe that this consequence is
a reason to reject each theory and look for an alternative.

In Section \ref{essential} I will sketch an alternative theory.  This
theory identifies chairs and other ordinary things with mereological
sums in the classical sense.  Groups like the Supreme Court will be
identified with sets.  As I have argued, a sum is like a set in that
it does not change its parts.  Therefore the thing I refer to with `my
chair' is a sum.  If I replace the leg on my chair, I will use `my
chair' to refer to a different sum.  Which sum I refer to with `my
chair' will be governed by convention.


\chapter{Can chairs change their parts?}
\chapterpig{Can chairs change their parts?}
\label{essential}

In Section \ref{parts} I presented three different theories that
modified classical mereology.  These modification were made to explain
how objects change their parts over time.  But each of these theories
require us to posit an extraordinary plurality of (if only
temporarily) co-located objects.  Such theories are, if not false, at
least very strange.

In this section, therefore, I will attempt to sketch an {\em
  essentialist} theory of things.  This theory will allow us to reject
the ``plurality thesis''---that there are pluralities of co-located
objects---but it will have problems of its own.  The most glaring is
the consequence that, strictly speaking, things don't change their
parts.

In Section \ref{groups} I claimed that three theories---Fine's theory
of embodiments, his theory of composition operators, and Hovda's
theory of tensed mereology---could each explain the existence of {\em
  groups} like the Supreme Court.  These theories are able to do so
because they each have a way of accounting for how things, including
groups, change their parts over time.  In this section, however, I am
suggesting that perhaps things cannot change their parts.  Must I
therefore deny the existence of groups?

Instead of denying that there are groups, instead I will argue that it
is possible to identify groups with {\em sets}; strictly speaking,
therefore, groups cannot change their parts.

\section{Re-examining the set identity thesis}
\label{set-id}
The primary motivation cited in Section \ref{group-set} for positing
groups was the fact that the Supreme Court appears to change its
members over time.  For example, both of the following sentences seem
to be true:

\begin{enumerate}[ref=(\arabic*)]
  \item The Supreme Court ruled on Roe vs.\ Wade in 1973. \label{roe1}

  \item The set of justices now serving as Supreme Court Justices did
    not rule on Roe vs.\ Wade in 1973
    \citep[135]{uzquiano2004a}. \label{roe2}
\end{enumerate}

One way to accommodate these facts is to ``insist that the Supreme
Court is a set, but to abandon the assumption that there is a single
set to which the phrase `the Supreme Court' refers in sentences
\ref{roe1} and \ref{roe2}'' \citep[138]{uzquiano2004a}.  To
successfully use `the Supreme Court' to refer to a set of justices,
there must be an implicit or explicit temporal reference.  If an
utterance of \ref{roe1} is true it will be true because it the speaker
intends her audience to recognize her intention to refer to the set of
justices that was the Supreme Court in 1973.  If her audience, for
whatever reason, takes her to be referring to the current Court, then
they will evaluate \ref{roe1} as false.

Considered in this light, `the Supreme Court' is used to express a
relation between sets and times; `$x$ is the Supreme Court at $t$'
\citep[140]{uzquiano2004a}.  There is some precedent for this sort of
interpretation:

\begin{squote}
Our use of the phrase `the Supreme Court' to express a relation a set
of justices bears to a time is much like our use of the phrase `the
president of the United States' to express a relation an individual
bears to a time.  Different persons may be the president of the United
States at different times, but there is at most one person that bears
that relation to each time \citep[138]{uzquiano2004a}.
\end{squote}

``But,'' it will be objected, ``there is an important difference here.
We use both phrases---`the Supreme Court' and `the president'---to
refer to a past, present or future set that `is' the thing, but we
also use `the Supreme Court' to refer to {\em the Supreme Court},
which has changed its membership over time.  If I say, `The Supreme
Court has become more conservative over the past century', there is no
one set I am referring to.  I must be referring to something else; the
obvious candidate is the {\em group} that is the Court.''

One reply here is to claim that all that what ``The Supreme Court has
become more conservative over the past century'' actually means is
that the members of the sets that have been the Supreme Court have
become more conservative.  Another, similar reply is that someone who
utters ``The Supreme Court has become more conservative over the past
century'' is saying something literally false (either because there is
no unique set that is being referred to, or because there is a unique
set referred to, but one that does not make the proposition true), but
can generally be understood to mean something else; namely, that the
members of the sets that have been the Supreme Court have become more
conservative.

Neither reply is {\em very} unintuitive; indeed, there is something
attractive about a thesis that reserves application of adjectives like
`conservative' for people, rather than other things like groups.

But there is a more pressing worry for the set identity thesis.
Recall that the set that is the Supreme Court at a given time might
also be the Special Committee on Judicial Ethics.  We must admit that
the Supreme Court in 2004 is the set \{Rehnquist, Stevens, O'Connor,
Scalia, Kennedy, Souter, Thomas, Ginsburg, Breyer\}, and the Special
Committee in 2004 is that very same set.  But now we are committed to
this argument:

\begin{enumerate}[ref=(\arabic*)]
  \item The Special Committee on Judicial Ethics is one of the
    committees assembled by the Senate.

  \item The Special Committee on Judicial Ethics is identical with the
    Supreme Court.

  \item {\em Therefore} the Supreme Court is one of the committees
    assembled by the
    Senate. \citep[144]{uzquiano2004a} \label{sup-com}
\end{enumerate}

And \ref{sup-com} seems false.

But it may be possible to argue that \ref{sup-com} is not false but
only {\em misleading} (indeed, very misleading).  For it
(conversationally) implies that future sets referred to by `the
Supreme Court' will be identical to future sets referred to by `the
Special Committee'.  And it is {\em this} that is certainly false.

This possibility raises another: that ordinary things like chairs are
identical with {\em sums} in the classical sense.  That is, just as it
may be that, strictly speaking, the Supreme Court cannot change its
parts (its members), so a chair cannot, strictly speaking, change its
parts.  Just as we identified groups like the Supreme Court with
different sets at different times, so we can identify things like
chairs with different sums at different times.

This is a bizarre possibility, and one apparently at odds with common
sense.  Isn't it {\em obvious} that things change their parts?  They
certainly seem to, and it may be argued that much of our talk
presupposes this.  But not all of our talk does, and some stretches of
discourse can actually be {\em better} interpreted on the assumption
that groups are sets, or that ordinary things are sums.

\section{Sets, sums, and literal speech}
\label{talk}
I argued in Section \ref{eng-quant} that ordinary uses of `there is'
are often false.  For example, if I say ``There is no beer'', what I
say is almost certainly false---there is beer {\em somewhere}---but
what I mean is that there is no beer in the house.

It is very likely that much of our ordinary talk is similarly
non-literal (see \citet{bach1987}).  For example, we should interpret
all uses of `The chair is mine' as non-literal, because saying ``The
chair is mine'' entails that there is only one chair in the world.
Even propositions involving proper names might be non-literal.  If
`Alex' designates every person named `Alex', then ``Alex is lying
down'' is literally false, since it entails either that there is only
one `Alex' or that every `Alex' is lying down.

Therefore, if a theory predicts that some of our talk is non-literal,
we should not necessarily be worried.  But not {\em all} of our talk
is non-literal, and when a theory can preserve the intuition that
certain things are literally true, that should be taken as an
advantage.

\subsection{Talking about sets}
\label{sets-talk}
At least for a certain class of examples, the set-identity thesis
preserves more of our intuitive judgments about literal speech than
does the theory that posits groups as distinct from sets.

\begin{enumerate}
  \item Suppose we arrive at a meeting of the Special Committee on
    Judicial Ethics.  Rehnquist, Stevens, O'Connor, Scalia, Kennedy,
    Souter, Thomas, Ginsburg, and Breyer are sitting around a center
    table.  As we take our seats you turn to me and say, ``They look
    rather familiar, don't they?''  I say ``That's also the Supreme
    Court.''

    What am I referring to with the demonstrative expression `that'?
    If one thinks that I am referring to a {\em group}---the Special
    Committee---that is distinct from the Supreme Court, my utterance
    will have to be interpreted as non-literal.  I will have to be
    understood to mean that the {\em members} of the Special Committee
    are also the members of the Supreme Court.  On the other hand, if
    I am referring to the {\em set} of justices, what I said is
    literally true.

  \item Suppose instead that you ask me who the members of the Special
    Committee are.  I say ``Rehnquist, Stevens, O'Connor, Scalia,
    Kennedy, Souter, Thomas, Ginsburg, and Breyer.  The Special
    Committee is just the Supreme Court.''  

    Here again one could argue that I am speaking non-literally; what
    I mean is that the members of the Special Committee are just the
    members of the Supreme Court.  But if the Supreme Court and the
    Special Committee are just sets---the same set---I have again said
    something literally true.

  \item Now suppose that the Special Committee is dissolved in 2004.
    In 2005, we see the members of the Supreme Court (still Rehnquist,
    Stevens, O'Connor, Scalia, Kennedy, Souter, Thomas, Ginsburg, and
    Breyer) out to lunch together.  I point and say ``That was the
    Special Committee on Judicial Ethics.''  Now what is `that' used
    to refer to?  It cannot be the Special Committee, for that has
    ceased to be.  It must either be the Supreme Court or the set
    \{Rehnquist, Stevens, O'Connor, Scalia, Kennedy, Souter, Thomas,
    Ginsburg, and Breyer\}.

    Either way, the proponent of groups will have to interpret this
    utterance as non-literal.  The set-identity theorist can interpret
    this utterance as literally true, however; the set in question was
    the Special Committee before the dissolution.

  \item Now suppose that the Special Committee is dissolved in 2004
    and Rehnquist retired before dying in 2005 (let's pretend he
    retired in May).  Now in August we see Rehnquist, Stevens,
    O'Connor, Scalia, Kennedy, Souter, Thomas, Ginsburg, and Breyer
    out to lunch together.  I point and say ``That was the Supreme
    Court {\em and} the Special Committee on Judicial Ethics.''  I can
    only be referring to the set of justices.  Why not suppose that I
    have only {\em ever} been referring to the set of justices?  If I
    am in fact referring to the set \{Rehnquist, Stevens, O'Connor,
    Scalia, Kennedy, Souter, Thomas, Ginsburg, Breyer\}, then when I
    say ``That was the Supreme Court {\em and} the Special
    Committee'', I say something literally true.
\end{enumerate}

These examples provide some support for the set identity thesis.  At
the very least they show that identifying groups with sets does not
mean that all our talk about groups must be interpreted as
non-literal.  However, the set identity thesis also predicts that some
propositions will be literally true, when intuitively we may believe
that they are not.  For example, according to the set identity thesis,
I say something literally true when I say ``The Supreme Court is one
of the committees assembled by the Senate'' or ``The Supreme Court is
the Special Committee on Judicial Ethics''.  But it is very misleading
to say either.  By saying ``The Supreme Court is the Special
Committee'' I imply that future things designated by `the Supreme
Court' will be identical to future things designated by `the Special
Committee'.  It is less misleading to say ``The current Supreme Court
is the Special Committee on Judicial Ethics''.  (It is even less
misleading to say ``The current Supreme Court is also the Special
Committee''.)

\subsection{Talking about sums}
\label{sums-talk}
The identification of statues (and lumps) with sums allows us to again
explain some sorts of talk that would be otherwise problematic:

\begin{enumerate}
  \item Suppose you help me carry a lump of clay into my workshop.  In
    the afternoon you drop by and see that I am sculpting a statue.
    You say, ``That looks familiar''.  I reply, ``It's the lump of
    clay from this morning''.

    What are you referring to with `that', and what am I referring to
    with `it'?  If we thought that the statue and the lump were two
    distinct things, we might assume that we are both referring to the
    statue.  We would therefore have to interpret what I say as
    somehow non-literal.  But if we are both referring to a single
    sum, then what I say is literally true.

  \item Suppose you're a little dense.  You see the statue and ask,
    ``What a great big statue!  Where did it come from, and where did
    the lump of clay disappear to?''  I reply, ``The statue {\em is}
    the lump''.

    If the statue and the lump are distinct things and my use of `is'
    is that of identity, then what I say is false.  It must be
    interpreted non-literally, as meaning that the statue is composed
    of the same matter as the lump.  But if the statue and the lump
    are the very same sum, then what I say is literally true.
  \item Suppose you come along and squish the statue, thereby
    destroying it.  I cry, ``That was my statue!''

    If we are imagining that the statue was a distinct thing from the
    sum (and from the lump), we would have to interpret what I say
    non-literally.  For if the statue was a distinct thing that has
    been destroyed, then when I use a demonstrative like `that' I
    cannot be referring to the non-existent statue.  My audience may
    interpret me as referring to the lump, and meaning that there used
    to be a statue co-located with the lump.  But if we suppose that
    the statue was not a distinct thing from the sum (and from the
    lump), then what I said is literally true.  For ``that''---that
    sum---was a statue, but is no longer.  It no longer satisfies the
    criteria for being a statue.  (You could dispute this; after I say
    ``That was my statue'', you could say ``It still is your statue;
    it's just a flatter statue than it was''.)  Likewise, suppose I
    have a lump of clay on Monday:

    \stage{Alex}{}{This will be a statue!}

    On Tuesday I make a statue out of the clay:

    \stage{Alex}{}{Yesterday this was nothing more than a lump of
      clay!  Now look at it!}

    On Wednesday you squish the statue:

    \stage{Alex}{}{Well, it's not a statue anymore.}
\end{enumerate}

With these examples, I am trying to motivate the idea that we refer to
the sum when we use words like `it' and `this' and `that'.  The sum
referred to on Monday is the same (or nearly the same) sum referred to
on Tuesday and on Wednesday.

This idea is a part of the thesis of essentialism---the thesis that
things do not change their parts.  This is a controversial thesis, and
I will address the objections to it in Section \ref{essentialism}.
The primary objection is that it is simply {\em wrong} to claim that
most of our cross-temporal talk---for example, ``That is the same
chair as yesterday''---is literally false.

\section{Problems with essentialism}
\label{essentialism}
Essentialism is the thesis that, strictly speaking, things don't
change their parts.  One can endorse or oppose essentialism in various
domains.  For example, almost everyone is a set essentialist; I can
think of nobody who claims that sets can change their parts.  But not
everyone is a {\em mereological} essentialist.

People who deny mereological essentialism are, I think, making one of
two claims:

\begin{enumerate}
  \item They may be claiming that ordinary things like chairs are not
    mereological sums; chairs can change their parts, so essentialism
    {\em with regard to chairs} is false.  A philosopher who makes
    this claim might allow that mereological sums, if there are such
    things, cannot change their parts.
  \item They may be claiming that mereological sums, whether or not
    they are identical with ordinary things like chairs, can change
    their parts.
\end{enumerate}

Fine appears at least sympathetic to the first claim (see Sections
\ref{fine-h}--\ref{fine-c}); van Inwagen and Hovda argue for the
second (see Sections \ref{change} and \ref{hovda}).  But a philosopher
who makes either claim will reject the theory I have been building.
They will say that my theory flies in the face of common sense (and I
make so much of common sense in earlier sections!).  They will say
things like this:

\begin{squote}
According to [the essentialist], it is never literally correct to say
that a thing survives a change in parts.  This is a point of massive
departure from ordinary belief \citep[184]{sider2001}.
\end{squote}

This is more or less the argument against essentialism.  You point at
a chair and say ``I'm supposed to believe that if that chair loses
{\em one atom}, it's literally a different chair?''

It is interesting to note that in the past, many philosophers were
more than willing to affirm this.  Roderick Chisholm points out that

\begin{squote}
Abelard held that ``no thing has more or less parts at one time than
at another''\,\ldots [and] Leibniz said ``we cannot say, speaking
according to the great truth of things, that the same whole is
preserved when a part is lost'' \citeyearpar[145]{chisholm1979}.
\end{squote}

Joseph Butler, writing in 1736, also held that ``when a man swears to
the same tree, as having stood fifty years in the same place, he means
only the same as to all the purposes of property and uses of common
life, and not that the tree has been all that time the same in the
strict philosophical sense of the word''
\citeyearpar[100]{butler1975a}.

One might object that these philosophers were simply failing to
distinguish {\em descriptive} and {\em numerical} sameness.  When I
say I have the same guitar as you, all I mean is that it is
descriptively the same, not that I have your guitar.  Likewise perhaps
Abelard, Leibniz, and Butler observed that a sapling is descriptively
different from the mature tree that it grows into, and then drew the
unwarranted conclusion that the sapling and mature tree are not
therefore the same.

This seems a bit uncharitable, but in any case there are arguments
supporting the same conclusion---that things cannot change their
parts.  The first comes from Chisholm:

\begin{squote}
Let us picture to ourselves a very simple table, improvised from a
stump and a board.  Now one might have constructed a very similar
table by using the same stump and a different board, or by using the
same board and a different stump.  But the only way of constructing
precisely {\em that} table is to use that particular stump and that
particular board.  It would seem, therefore, that that particular
table is {\em necessarily} made up of that particular stump and that
particular board \citeyearpar[146]{chisholm1979}.
\end{squote}

It may be objected that, {\em once the table is built}, it is possible
to change its parts without thereby destroying one table and
constructing another.  Once I have built a table, it seems true that I
could take it apart and reassemble the very same table.  It even seems
that I could take it apart and reassemble the very same table with a
slight modification; for example, I could put it back together with
one new leg.  It may be, then, that all Chisholm's argument shows is
that this particular table necessarily {\em began} its existence with
a particular stump and board.  But there is nothing in the argument
that shows that it necessarily cannot go on to change its parts while
remaining numerically identical.

But if a chair can remain numerically identical after changing a part,
it is difficult to say {\em how large} a part the chair can lose while
remaining the (numerically) same chair.  If most of the chair is
blasted away, then we may very well say that the chair is no more.
But {\em how much} must be blasted away?  Or suppose we have a portion
of gold.  How many atoms of gold can be stripped off before it is no
longer the same portion?  Thomson claims that ordinary uses of
`portion' are context-dependent:

\begin{squote}
The ordinary use of the term `portion' is heavily context-dependent.
If an atom drifts away from your portion of gold, do you still have
the same portion of gold?  You will say no if you are a scientist
engaged in an experiment for which every atom matters. You will say
yes if you are a jeweler about to make a ring.  Similarly, in fact,
for clay.  If you have just bought a load of clay, and a handful falls
off while you are on your way home, is the portion you have when you
get home the same as the portion you bought?  You will say no if you
had carefully measured and bought exactly as much as you need.  You
will say yes if loss of a handful makes no difference to you
\citeyearpar[163]{thomson1998a}.
\end{squote}

When we say that a use of a term is context-dependent, that can mean
one of two things.  First, it may mean that whether an utterance
involving a use of the term is {\em correct}, or {\em appropriate},
depends on the context.  It would not be appropriate for the scientist
to say that she has the same portion after the loss of several atoms,
because those atoms matter for the experiment.  Second, to say that
the use of a term is context-dependent may mean that whether an
utterance involving a use of the term is {\em true} depends on the
context.  In the quoted passage above, do the scientist and jeweler
both say true things?  If they do, then the truth-conditions of
`portion' are context-dependent.  This would mean that whether an
utterance involving `portion' is true depends on the context of the
utterance.  This would also suggest that the {\em meaning} of
`portion' depends on the context, for the truth-value of a sentence is
generally thought to be a function of the meaning of its constituent
elements, including words.

But just as I do not think there are different senses of `there is'
(see Section \ref{eng-quant}), so I do not think that there are
multiple senses of `portion'.  I find it far more plausible to think
that only the scientist says something that is, {\em strictly
  speaking}, true.  The jeweler, when she affirms that she has the
same portion of gold, may say something correct or appropriate, given
the context, but it is not {\em true}.  Strictly speaking, a portion
cannot change its parts; why should we assume that a chair can?

\section{Talk over time}
\label{time-talk}
One may object that, while this is all well and good, the essentialist
thesis---that groups are sets and ordinary things are sums---fails the
most important test.  The thesis seems to predict that propositions
about change over time, such as `The Supreme Court was formed in 1789'
or `The Brick House did not exist last Tuesday' are almost always
literally false.

In the case of sets, what is going on when I say something like ``The
Supreme Court was formed in 1789''?  If `the Supreme Court' designates
the set of current justices, then such a claim is false.  But there is
obviously {\em something} right about what I say.  The set identity
thesis must be supplemented with an explanation of what is right about
`The Supreme Court was formed in 1789' and what is wrong about `The
Supreme Court was formed in 1200'.  (Compare this to van Inwagen and
Merricks' attempts---discussed in Section \ref{stroud}---to explain
what is right about utterances like ``There is a chair in the
kitchen''.)

In the case of sums, how are we to understand the proposition `The
Brick House did not exist last Tuesday'?  This example is part of an
argument by Peter van Inwagen aiming to show that sums can change
their parts.  Suppose, first, that sums cannot change change their
parts:

\begin{squote}
Call the bricks that were piled in the yard last Tuesday the `Tuesday
bricks'.  Between last Tuesday and today, the Wise Pig has built a
house---the `Brick House'---out of the Tuesday bricks (using them all
and using no other materials).  The Brick House did not exist last
Tuesday (that is, it was not then a pile of bricks, a thing that was
not yet a house but would become a house).  The Brick House is not,
therefore, a mereological sum; for if it were, it would have been (it
would have ``existed as'') a pile of bricks last Tuesday
\citeyearpar[616]{inwagen2006}.
\end{squote}

But since the Brick House {\em is} a mereological sum, van Inwagen
concludes that our supposition that sums can't change their parts is
false; he claims that mereological sums {\em can} change their parts.
If we are to maintain both that the Brick House is a sum {\em and}
that sums can't change their parts, we must say that (strictly
speaking) the Brick House {\em did} exist last Tuesday, despite the
fact that it had not yet been built. \\

Below I will look at two different ways of making sense of
cross-temporal utterances about sets and sums.  The first is an
adaptation of Ted Sider's temporal counterpart theory; it maintains
that even though `the Supreme Court' designates the current set of
justices, ``The Supreme Court was formed in 1789'' is literally true.
The second is an adaptation of Roderick Chisholm's notion of an ``ens
successivum''; this theory claims that the Supreme Court and the Brick
House are ``fictions'' that are constituted by different things at
different times; cross-temporal talk is generally false, but can be
correct or accurate.  I will suggest that Chisholm's theory is
superior to Sider's.

\subsection{Ordinary speech and temporal counterpart theory}
\label{counterpart}
According to the essentialist thesis, it is literally false to say
``The Supreme Court was formed in 1789''.  Likewise it is false to say
``The Brick House did not exist last Tuesday''.  The thing designated
by `the Supreme Court' is a set; sets exist when their members exist,
and so the set in question did not even exist in 1789.  (Even if the
set {\em did} exist then, the only way the utterance would be true
would be if the last member of the set was born in 1789; then the set
would come into existence in 1789.  This {\em might} make it true that
the set was ``formed'' in 1789.)  The thing designated by `the Brick
House' is a sum; sums exist whenever their parts exist, and the parts
of the Brick House existed last Tuesday.

These are obviously unintuitive conclusions.  Ted Sider's temporal
counterpart theory offers a possible way to avoid them.

Sider's counterpart theory is part of the theory of
four-dimensionalism he once promoted \citeyearpar{sider2001}.  Unlike
most four-dimensionalists who claim that we use terms like `chair' to
refer to ``spacetime worms'' or ``aggregates of chair-stages'', Sider
argued that we use such terms to refer to instantaneous stages, not
``continuant'' worms or aggregates.  What this means is that in
ordinary talk we never refer to the same thing twice; the chair I
refer to at $t_1$ is one temporal part (chair-at-$t_1$) and the chair
I refer to at $t_2$ is another.  When I use `Ted' to refer to Ted
Sider, I am not referring to the temporally extended object that
includes a childhood; I am referring to something that lasts only for
an instant.

Nonetheless Sider claims that when I say ``Ted was once a boy'', I say
something literally true.  How can this be?  The object I am referring
to was never a boy.  It is here that Sider introduces temporal
counterparts:

\begin{squote}
According to my temporal counterpart theory, the truth condition of an
utterance like ``Ted was once a boy'' is this: there exists some
person stage $x$ prior to the time of the utterance, such that $x$ was
a boy, and $x$ bears the temporal counterpart relation to Ted.  Since
there is such a stage, the claim is
true. \citeyearpar[193]{sider2001}.
\end{squote}

This theory can be adapted to our purposes.  We may say that sets and
sums, like stages, have temporal counterparts.  The set that is
currently designated by `the Supreme Court' bears a temporal
counterpart relation to other sets at other times.  The truth
condition of `The Supreme Court was formed in 1789' is perhaps the
fact that there was a set $S$ such that $S$ in 1789 bears the temporal
counterpart relation to the (current) Supreme Court and it is not the
case that there was some set $S^{\prime}$ and time $t$ such that $t$
is earlier than 1789 and $S^{\prime}$ in $t$ bears the temporal
counterpart relation to the Supreme Court.  It would then be literally
true to say ``The Supreme Court was formed in 1789''.  Likewise the
sum that is currently designated by `the Brick House' bears temporal
counterpart relations to other sums.  The truth condition of `The
Brick House did not exist last Tuesday' is perhaps the fact that the
Brick House does not bear a temporal counterpart relation to anything
on last Tuesday (or prior).

These are only rough formulations; they must be adapted to account for
the possibility of co-location (in a loose sense).  Recall that the
set that is currently designated by `the Supreme Court' might also
currently be designated by `the Special Committee'.  It is neither
true nor correct in any sense to say that the Special Committee was
formed in 1789.  If we were to adopt the theory of temporal
counterparts, we would have to recognize different kinds of
counterpart relations.  The set that is currently designated by `the
Supreme Court' bears the ``Supreme-Court-counterpart'' relation to the
set that was designated by `the Supreme Court' in 1789, but not the
``Special-Committee-counterpart'' relation.  It bears {\em that}
relation to other sets at other times.

I will not elaborate on this, however, because Sider's temporal
counterpart theory makes false assumptions about meaning.

Sider explicitly states that stages have an instantaneous temporal
duration---any given stage exists only for an instant
\citeyearpar[xiv]{sider2001}.  If we suppose that Ted is a
person-stage that exists only at instant $t$, then it is obviously not
true that it was the case that Ted existed at any previous time.  That
is, at $t$, the following is true:

\begin{displaymath}
\neg \exists t^{\prime} (t^{\prime}\ \text{is earlier than}\ t \wedge
\exists x\ (Ext^{\prime} \wedge x = \text{Ted}))
\end{displaymath}
(`$Ext$' means `$x$ exists at $t$'.)

But Sider also claims that `Ted was once a boy' is true.  This seems
to be equivalent to `There was some thing such that it was a boy and
it was Ted'.  That is, it appears that Sider is also committed to this
being true at $t$:

\begin{displaymath}
\exists t^{\prime} (t^{\prime}\ \text{is earlier than}\ t \wedge
\exists x\ (Ext^{\prime} \wedge Bx \wedge x = \text{Ted}))
\end{displaymath}
(Here `$Bx$' means `$x$ is a boy'.)

These are contradictory claims.  Sider must therefore be supposing
either that `$\exists$' is semantically ambiguous or that `Ted was
once a boy' {\em does not mean} `There was some thing such that it was
a boy and it was Ted'.  Sider is vehemently opposed to the idea that
there are multiple, equally suitable, meanings for quantifiers
\citeyearpar{sider2001,sider2011b,sider2011d}.  Therefore, I think
Sider is assuming that what `Ted was once a boy' means is `There
exists some person stage $x$ prior to the time of the utterance, such
that $x$ was a boy, and $x$ bears the temporal counterpart relation to
Ted'.

If `Ted was once a boy' does not mean `There exists some person stage
$x$ prior to the time of the utterance, such that $x$ was a boy, and
$x$ bears the temporal counterpart relation to Ted', then there is no
reason to think that the truth-condition of the former are the latter.
For it seems initially obvious that the truth-condition of `Ted was
once a boy' is that Ted (the stage) was once a boy.  If this is not
the truth-condition, then it must be because `Ted was once a boy' does
not actually mean that Ted was once a boy, but instead means that
there exists some person stage $x$ prior to the time of the utterance,
such that $x$ was a boy, and $x$ bears the temporal counterpart
relation to Ted.

In order to maintain that `Ted was once a boy' is literally true,
Sider must claim that it means something other than that Ted was once
a boy.  This is a highly implausible and unmotivated claim; the only
reason I can think of as to why Sider might make such a claim would be
because he holds a truth-conditional theory of meaning (see Section
\ref{verbal}) and believes that `Ted was once a boy' is true if and
only if `There exists some person stage $x$\,\ldots ' is true.  But a
truth-conditional theory of meaning is controversial and susceptible
to numerous counter-examples.  It seems far more reasonable to admit
that `Ted was once a boy' means that Ted was once a boy, and is
literally false.

\subsection{Chisholm's entia successiva}
\label{chisholm}
Roderick Chisholm was a mereological essentialist, claiming that
ordinary things cannot change their parts:

\begin{squote}
Familiar physical things such as trees, ships, bodies and houses
persist ``only in a loose and popular sense''.  This thesis may be
construed as presupposing that these things are ``fictions'', logical
constructions or {\em entia per alio} \citeyearpar[97]{chisholm1979}.
\end{squote}

Chisholm paraphrases talk involving persistence by stipulating a
technical sense of `successor' and `successive'.  He gives the
following definitions:

\begin{enumerate}[ref=\arabic*]
  \item $x$ is at $t$ a direct chair successor of $y$ at $t^{\prime}
    =_{df}$ (i) $t$ does not begin before $t^{\prime}$; (ii) $x$ is a
    chair at $t$ and $y$ is a chair at $t^{\prime}$; and (iii) there
    is a $z$, such that $z$ is a part of $x$ at $t$ and a part of $y$
    at $t^{\prime}$, and at every moment between $t$ and $t^{\prime}$,
    inclusive, $z$ is itself a chair. \label{suc1}
  \item $x$ is at $t$ a chair successor of $y$ at $t^{\prime} =_{df}$
    (i) $t$ does not begin before $t^{\prime}$; (ii) $x$ is a chair at
    $t$ and $y$ is a chair at $t^{\prime}$; and (iii) $x$ has at $t$
    every property \textsc{p} such that (a) $y$ has \textsc{p} at
    $t^{\prime}$ and (b) all direct chair successors of anything
    having \textsc{p} have \textsc{p}. \label{suc2}
  \item $x$ constitutes at $t$ the same successive chair that $y$
    constitutes at $t^{\prime} =_{df}$ Either (a) $x$ and only $x$ is
    at $t$ a chair successor of $y$ at $t^{\prime}$ or (b) $y$ and
    only $y$ is at $t^{\prime}$ a chair successor of $x$ at $t$
    \citep[99--100]{chisholm1979}. \label{suc3}
\end{enumerate}

Before we see how these definitions are used, there are two
misinterpretations (in my opinion) of Chisholm's position.  First, one
might take Chisholm to be claiming that ``successive chairs'' are {\em
  things} that are composed of or constituted by different bits of
matter at different times.  I think this is not how Chisholm should be
understood, for it would undermine his claim that successive chairs
are ``fictions'' that persist only in a ``loose and popular'' sense.

Second, one might take Chisholm's four definitions above to be giving
the literal meaning of the definienda.  That is, one might take
Chisholm to be claiming that what `$x$ constitutes at $t$ a successive
chair' {\em means} is `There are a $y$ and a $t^{\prime}$ such that
$y$ is other than $x$ and $x$ constitutes at $t$ the same chair that
$y$ constitutes at $t^{\prime}$'.  Whether or not this is what
Chisholm intended, I think it is false for two reasons.  First, it is
a highly implausible thesis about sentence meaning; why should we
think the the former sentence is synonymous with the latter, except
that it makes Chisholm's theory more palatable?  Second, if Chisholm's
definitions gave the literal meaning of the definienda, then it would
be true in the ``strict and philosophical sense'', as well as in the
``loose and popular sense'', that a successive chair persists over
time.  But Chisholm explicitly denies this
\citeyearpar[96--97]{chisholm1979}.

The interpretation of Chisholm that I prefer is this: when we speak of
a successive chair persisting over time, what we say is, strictly
speaking false.  However, we should be understood to {\em mean}
something other than what we say; what we mean can be captured with
the definitions given by Chisholm.  For example, when someone says
``That chair was made in 1900'', what they say is literally false, but
can be {\em paraphrased} by applying Definitions
\ref{suc1}--\ref{suc3}.  First we understand `That chair was made in
1900' to be equivalent to `$x$ (the present chair) constitutes now the
same successive chair that some $y$ constituted in 1900 and there is
no $z$ such that $z$ constitutes before 1900 the same chair that $x$
constitutes now'.  This is false, but someone making either utterance
should be taken to mean something else.  We can determine exactly what
is (or should be) meant by applying Chisholm's definitions in reverse:

\begin{enumerate}[start=3]
  \item $x$ and only $x$ is now a chair successor of some $y$ in 1900
    and there is no $z$ such that $x$ is now a chair successor of $z$
    before 1900.
\end{enumerate}
This is turn can be understood as

\begin{enumerate}[start=2]
  \item First, $x$ is a chair now and $y$ is a chair in 1900, and $x$
    has now every property \textsc{p} such that (a) $y$ has \textsc{p}
    in 1900 and (b) all direct chair successors of anything having
    \textsc{p} have \textsc{p}.  Second, there is no $z$ and $t$ such
    that $t$ begins before 1900 and $x$ is now a chair successor of
    $z$ at $t$.
\end{enumerate}
The meaning of `direct chair successor' is given by Definition
\ref{suc1}.

We can say the same thing about groups and ordinary things.  When
someone says ``The Supreme Court was formed in 1789'', what they say
is false, but should be paraphrased as something like this:

\begin{itemize}
  \item First, $S$ is the Supreme Court now and $T$ is the Supreme
    Court in 1789, and $S$ has now every property \textsc{p} such that
    (a) $T$ has \textsc{p} in 1789 and (b) all direct Supreme-Court
    successors of anything having \textsc{p} have \textsc{p}.  Second,
    there is no $V$ and $t$ such that $t$ begins before 1789 and $x$
    is now a chair successor of $V$ at $t$.
\end{itemize}

Likewise when someone says ``The Brick House did not exist last
Tuesday'', what they say should be paraphrased as `There is no $x$ and
$t$ such that $t$ begins before last Monday and the Brick House is now
a house successor of $x$'.

This solution is superior to Sider's theory of temporal counterparts
because it does not make questionable assumptions about meaning.  It
is false that the Supreme Court was formed in 1789, but it is correct
(in a ``loose and popular sense'') because the Supreme Court is a
successor of the ``original'' Supreme Court.  It is also false to say
that the Brick House did not exist last Tuesday, but it is correct, in
a loose and popular sense.

As given, however, Chisholm's definitions assume eternalism.  If the
set of 1789 justices no longer exists---if some matter that was part
of a justice is destroyed---then, without assuming that what {\em did}
exist always {\em does} exist, it is not true that there is a set $S$
such that the Supreme Court is a Supreme-Court-successor of $S$.  This
can only be true if $S$ exists.  If $S$ does not exist, then the
Supreme Court cannot be a successor of it.

However, it seems possible to reformulate Chisholm's definitions (or
write entirely new ones) so as to avoid this assumption.  The
following revisions of Definitions \ref{suc1}--\ref{suc3} illustrate
how this might be done ($t$ is the present time and `\textsc{Always}'
means `it is always the case that'):

\begin{enumerate}[label=\arabic*a., ref=\arabic*a]
  \item $x$ is at $t$ a direct chair successor of $y$ at $t^{\prime}
    =_{df}$ \textsc{Always}(if it is $t^{\prime} \rightarrow \exists
    y$ such that $y$ is a chair and such that \textsc{Always}(if it is
    $t \rightarrow \exists x$ such that $x$ is a chair and such that
    \textsc{Always}(if it is between $t^{\prime}$ and $t$ inclusive
    $\rightarrow \exists z$ such that $z$ is a chair and such that
    \textsc{Always}(if it is $t^{\prime} \rightarrow z$ is part of
    $y$) and such that \textsc{Always}(if it is $t \rightarrow z$ is
    part of $x$)))). \label{pres1}
  \item $x$ is at $t$ a chair successor of $y$ at $t^{\prime} =_{df}$
    \textsc{Always}(if it is $t^{\prime} \rightarrow \exists y$ such
    that $y$ is a chair and such that \textsc{Always}(if it is $t
    \rightarrow \exists x$ such that $x$ is a chair and such that
    \textsc{Always}(if it is $t \rightarrow x$ has every property
    \textsc{p} such that \textsc{Always}(if it is $t^{\prime}
    \rightarrow y$ has \textsc{p}) and such that \textsc{Always}(all
    direct chair successors of anything having \textsc{p} have
    \textsc{p})))). \label{pres2}
  \item $x$ constitutes at $t$ the same successive chair that $y$
    constitutes at $t^{\prime} =_{df}$ Either (a) $x$ and only $x$ is
    at $t$ a chair successor of $y$ at $t^{\prime}$ or (b) $y$ and
    only $y$ is at $t^{\prime}$ a chair successor of $x$ at
    $t$. \label{pres3}
\end{enumerate}
(Definition \ref{pres3} is identical to \ref{suc3}.)

We can again ``paraphrase'' talk about chairs over time.  As before,
we understand `That chair was made in 1900' to be equivalent to `$x$
(the present chair) constitutes now the same successive chair that $y$
constitutes at 1900 and there is no $z$ such that $z$ constitutes
before 1900 the same chair that $x$ constitutes now'.  This is false,
but we can determine exactly what is (or should be) meant by applying
our new definitions in reverse:

\begin{enumerate}[label=3a.]
  \item $x$ and only $x$ is now (at $t$) a chair successor of $y$ in
    1900 and there is no $z$ such that $x$ is now a chair successor of
    $z$ before 1900.
\end{enumerate}
This is turn can be understood as

\begin{enumerate}[label=2a.]
  \item \textsc{Always}(if it is 1900 $\rightarrow \exists y$ such
    that $y$ is a chair and such that \textsc{Always}(if it is $t
    \rightarrow \exists x$ such that $x$ is a chair and such that
    \textsc{Always}(if it is $t \rightarrow x$ has every property
    \textsc{p} such that \textsc{Always}(if it is 1900 $\rightarrow y$
    has \textsc{p}) and such that \textsc{Always}(all direct chair
    successors of anything having \textsc{p} have \textsc{p}) and such
    that \textsc{Always}(it is not the case that (if it is before 1900
    $\rightarrow \exists z$ such that $z$ is a chair and $z$ has
    \textsc{p}))))).
\end{enumerate}
The meaning of `direct chair successor' is given by Definition
\ref{pres1}.

Similar paraphrases can now be performed on utterances about the
Supreme Court, the Brick House, and other things.

Unfortunately, adopting this solution requires that we reject {\em
  serious presentism}.  Presentism is the thesis that only presently
existing things exist.  Serious presentism is the conjunction of that
thesis with the further claim that relations and properties can hold
only of existing things.  Serious presentism has the consequence that
it is literally false that I am smaller than Socrates.  This is not
because I am very large, but because Socrates does not exist.
Likewise, Socrates is not identical with himself because he does not
exist.

The last clause of Definition \ref{pres2}---``\textsc{Always}(all
direct chair successors of anything having \textsc{p} have
\textsc{p})''---violates serious presentism by positing a relation
(what we might call the ``chair successor'' relation) between
cross-temporal entities.  It may be possible to rewrite that clause to
avoid this assumption and make Chisholm's solution compatible with
serious presentism.  Then again, it may not.  If we do not want to
reject serious presentism, we may be forced to look for a different
solution.

\section{The conventions of persistence}
\label{set-convention}
Even supposing that our talk over time can be sorted out, there is
still more to be said.  Although chairs are sums that cannot change
their parts, we talk as if they can.  Likewise, although the Supreme
Court is a set, we talk as if the Supreme Court can change its
members.  The most reasonable way to make sense of this is to suppose
that for a given ``successive chair'' we use `chair' to refer to
different sums at different times; likewise, we use `the Supreme
Court' to refer to different sets at different times.  Chisholm's
definitions specify that a ``chair successor'' must be a chair, and a
``Supreme Court successor'' must be the Supreme Court, but they do not
specify how to determine what counts as a chair or as the Supreme
Court at any given time.  What makes it true that some sum is a chair,
or that some set is the Supreme Court?

What makes it true that some sum is a chair is just the fact that it
meets our conventional criteria for being a chair.  These criteria
probably cannot be given in terms of necessary and sufficient
conditions; the concept {\em chair} is too broad:

\begin{squote}
When one says chair, one thinks vaguely of an average chair.  But
collect individual instances, think of arm-chairs and reading chairs,
and dining-room chairs and kitchen chairs, chairs that pass into
benches, chairs that cross the boundary and become settees, dentists'
chairs, thrones, opera stalls, seats of all sorts, those miraculous
fungoid growths that cumber the floor of the Arts and Crafts
Exhibition, and you will perceive what a lax bundle in fact is this
simple straightforward term.  In co-operation with an intelligent
joiner I would undertake to defeat any definition of chair or
chairishness that you gave me \citep[384--385]{wells1904}.
\end{squote}
This is not a problem, since we can agree on paradigm examples of
chairs.  The term `chair' is obviously meaningful; this suggests that
the criteria for what counts as a chair are relatively well-defined,
even if we cannot adequately formalize them.  Thus what makes it true
at a given time that some sum is the ``chair successor'' of another
sum is the fact that both sums satisfy the criteria for being chairs
(at their respective times), and are related in the ways specified by
Chisholm's definitions.

But what about the Supreme Court?  What makes it true at a given time
that some set is then the Supreme Court?  I suggest that, again, there
are conventional criteria governing `the Supreme Court', and that
which set is at a given time the Supreme Court is a matter of
convention.  In the case of the Supreme Court, the conventions are
{\em legal} conventions.  The Constitution authorizes the recognition
of a set of justices as the Supreme Court.  Which set is recognized as
the Supreme Court is decided by the legislative and executive
branches.  The president nominates a set (the sitting justices and the
nominated justice) and the legislative branch votes.  The outcome of
the vote makes it true or false that a given set is the Supreme Court.
Thus what makes it true at a given time that some set is the ``Supreme
Court successor'' of another set is that fact that both sets satisfy
the criteria for being the Supreme Court (at their respective times),
and are related in the ways specified by Chisholm's definitions.

The correctness ({\em not} the truth) of cross-temporal talk about
groups is governed by convention.  This is plausible; groups are
social entities, and it is plausible that their ``change'' over time
should be due to convention.  But if this is right, it suggests that
the same holds for ordinary things.

\section{Am I a mereological sum?}
\label{i-sum}
I have proposed that ordinary things like chairs and statues are
mereological sums.  Their apparent persistence through change is a
result of certain conventions---a chair $x$ at $t_1$ is the ``same
successive chair'' as a chair $y$ at $t_2$ if the two are related in
the ways specified by Chisholm's definitions.

If ordinary things like chairs are sums, then are other things sums as
well?  I will suppose that sums are ``material things'' as opposed to
``abstract things'' (whatever that distinction comes to), but are {\em
  all} material things sums?  If we are material things, are we
therefore sums?

\subsection{All material things are sums}
\label{material-sum}
If we think that ordinary things are sums, and that ordinary things
are material things, I think it is extremely plausible to conclude
that all material things are sums.  For what else would they be?

What is included under the concept {\em material thing}?  I would
include things like chairs, and desks, and desk lamps, and doors, and
doorways, and houses, and gardens, and plants.  I would also include
minuscule objects like molecules and massive objects like planets and
galaxies.  What would these things be, if not sums?

I proposed that ordinary things are sums so as to avoid the conclusion
that there is a plurality of different kinds of ordinary things
(statues and lumps only scratch the surface) all overlapping each
other.  This essentialist proposal was made so as to avoid positing
many different kinds of things.  So anyone who accepts the
essentialist theory should be sympathetic to the idea that all
material things are sums.

I don't have much of an argument for this conclusion, but I don't see
the {\em point} of supposing that all and only ordinary things are
sums, but other material things are some different kind of object.

\subsection{We are material things}
\label{material-beings}
Even if the idea that all material things are sums is relatively
uncontroversial, the idea that {\em we} are material beings will not
be unanimously accepted.  For it does have some unintuitive
consequences.

First, it rules out identifying us with our mental states.  Suppose
all my psychological characteristics---memory, personality---is
somehow transferred to another body.  The brain in that body is
``wiped'' before my psychology is transferred, and after the operation
my old brain is similarly wiped.  There is a temptation to say that I
exist in the new body.  But saying this commits us to the claim that I
am not a material thing, because I ``left'' my old material body and
came to ``inhabit'' a new one:

\begin{squote}
 If I am identical with the thinking substance in which I am thus
 placed, then I cannot be transferred {\em from} that substance to
 another substance \citep[107]{chisholm1979}.
\end{squote}

Claiming that we are material things entails that psychological
continuity is not a criterion of identity.  The body into which my
psychology is transferred is not me, according to the materialist
claim.  Psychological continuity is often taken to be {\em the}
criterion of identity, so one might take this consequence as a
refutation of the claim that we are material things.

But if we are not material things, what are we?  The only alternative
I see is to claim that we are immaterial minds or souls.  These
positions seem, to me, to be more implausible than the claim that we
are material things.  (Much, of course, can be said in defense of this
alternative.)

Claiming that we are material things, however, gives rise to another
question: what material things are we?  Are we identical with our
brains, or with our bodies?

I suggest, though somewhat tentatively, that we are identical with our
bodies.  I agree with Peter van Inwagen on this much:

\begin{squote}
I suppose that [the objects of mental predicates]---Descartes, you,
I---are material objects, in the sense that they are ultimately
composed entirely of quarks and electrons.  They are, moreover, a very
special sort of material object.  They are not brains or cerebral
hemispheres.  They are living animals; being {\em human} animals, they
are things shaped roughly like statues of human beings
\citeyearpar[6]{inwagen1995}.
\end{squote}

Eric Olson has a very plausible argument for the same conclusion:

\begin{enumerate}
  \item There is a human animal sitting in your chair.
  \item The human animal sitting in your chair is thinking. (If you
    like, every human animal sitting there is thinking.)
  \item You are the thinking being sitting in your chair. The one and
    only thinking being sitting in your chair is none other than
    you. Hence, you are that animal \citeyearpar[354]{olson2003a}.
\end{enumerate}

One apparent consequence of the claim that we are material human
animals is that if my brain is removed from my body and put into
another body, that new person is not me.  Claiming that we are
material things required denying that psychological continuity is a
criterion of identity; claiming that we are material human animals
requires denying that even brain continuity is a criterion of
identity.

This may seem to be a troubling consequence, but it is much less
troubling if we accept the essentialist theory.  If material objects
are sums, and if we are material objects, then we are sums.  And if
sums do not, strictly speaking, change their parts over time, then,
like the ``persistence'' conditions for ``successive chairs'' and
other ordinary things, the ``persistence'' conditions over time for
{\em us} is conventional.

Another difficulty with identifying us with human animals disappears
if we accept an essentialist theory.  Dean Zimmerman has objected to
Olson's argument by claiming that ``human animal'' can be replaced
with ``human body'' without making the argument invalid
\citeyearpar[24]{zimmerman2008a}.  The problem, however, is that it
seems true that we cease to exist when we die.  So Zimmerman concludes
that we are not bodies or animals.

If we accept an essentialist theory, however, the problem disappears.
If, strictly speaking, I can't change my parts over time, then I am
not (strictly speaking) the same person that will be designated by
`Alex' a month from now (or even a week).  I will certainly not be
identical with a dead body further down the road.

\subsection{How do I ``persist'' over time?}
\label{person-persist}
The idea that, strictly speaking, I don't change my parts over time
seems crazy.  And maybe it is.  But I don't think it is obviously
false.

Someone who thinks that I do, strictly speaking, persist over time
might say that it is obvious that I persist.  After all, I engage in
activities that take long periods of time, I remember things from long
ago, and I bear unique attitudes toward my past and future selves.  I
feel pride or regret at past actions, and anticipation or apprehension
at future ones.  How could these past and future selves not be me?

One reply begins by pointing out that, whether or not we persist in a
strict sense, the world will look the same.  I will still engage in
activities that take time; but it will not be I who completes them.  I
will still remember things from long ago; but it will not be I who
experienced them.  I will bear attitudes towards past and future
people, but those people will not, strictly speaking, be me.  But it
will {\em seem} as if they are me, and they may be ``Alex successors''
in the sense defined by Chisholm (Section \ref{chisholm}).  As in the
case of tables and chairs, there are conventional ``persistence''
conditions for people over time.  Like tables and chairs, these
criteria will involve causal and spatiotemporal continuity.  What
person is designated by `Alex' a week from now will depend on a causal
chain connected to me.

Psychological continuity may also play a role.  For example, if by
some miracle I am vaporized and---quite coincidentally---a
qualitatively identical person is summoned into existence nearby, that
person will not, strictly speaking be me.  But it may be that the
person meets the criteria for being designated by `Alex'.  Then again,
it may not.  It may ultimately indeterminate whether or not that
person is Alex.  (My friends and family might have to {\em decide}
whether it is or not.)

The criteria for the ``persistence'' of people over time is not fully
precise, as shown by our indecision over whether we would use `Alex'
to refer to a spontaneous duplicate of me.  Another, more realistic,
situation in which this indecision manifests itself is in death.
Suppose I die, and a wake is held for my body.  It is perfectly
correct for someone to point and say, ``That was Alex''.  But it is
equally correct to say ``That's Alex''.  (The latter may be more
appropriate if it is necessary to identify my body.)  Is the
mereological sum that is the (deceased) body really me, or not?  If we
accept the essentialist theory, it is (strictly speaking) not, but it
may be correct or appropriate to use `Alex' to refer to the body.  If
it is, this will be because the body satisfies (or nearly satisfies)
the conventional criteria for being me.

\section{Lessons}
\label{lessons-e}
In Section \ref{parts} I examined three different versions of the
``plurality thesis''; the view that there are pluralities of
co-located objects.  In this section I offered an alternative.  I am
not sure whether my theory or one of the plurality theories is
correct, but I suspect that it must be one or the other.  My
conclusion is largely the same as that of Karen Bennett:

\begin{squote}
The only live options, then, are to be either a one-thinger or a
bazillion-thinger.  We must either think that there is only one thing per
spatio-temporal location, or else that there are lots and \emph{lots} of
spatio-temporally coincident things \citeyearpar[358]{bennett2004}.
\end{squote}

I would prefer to be a ``one-thinger'' because it does not commit me
to a ``bazillion'' things all in the same place.  That is not a
decisive objection, of course.  It may well be that such an explosion
is more plausible than certain consequences of the ``one-thinger''
theory.  But I think one of the two theories must be right.

Just as we demanded that the plurality theories could be equipped with
an explanation as to why we don't believe there to be as many things
as there are, so this essentialist thesis should be supplemented with
an explanation as to why we {\em do} believe that things change their
parts, when they in fact don't.

\subsection{Can the essentialist theory explain what we believe?}
\label{explain-e}
In Section \ref{explain-p} I assessed whether any of the three
plurality theories could explain why we hold beliefs that conflicted
with certain consequences of the theories.  The same assessment may be
conducted with regard to the essentialist theory I have sketched here.
If the essentialist thesis is right, why do we believe that chairs can
change their parts?

One explanation is simply that we {\em don't} believe that things
literally persist over time.  When asked ``Is it {\em literally} the
same chair without its leg?'' some of us may waver, and perhaps
concede that we don't think it is really the same chair.  But I doubt
this reply will convince any philosopher who has already made up her
mind about essentialism.

Another reply is that we are fooled by the great similarity between
``successive chairs'', both with regard to appearance and with regard
to their spatiotemporal location.  If we see a certain chair in the
sitting room, and while we are away it is replaced by a different
chair (someone carries one out and places another in the room), then
we will likewise be fooled by the similarities between the two, and
mistake them for one and the same thing.  This idea is largely due to
Hume:

\begin{squote}
Nothing is more apt to make us mistake one idea for another, than any
relation betwixt them, which associates them together in the
imagination, and makes it pass with facility from one to the other.
Of all relations, that of resemblance is in this respect the most
efficacious; and that because it not only causes an association of
ideas, but also of dispositions, and makes us conceive the one idea by
an act or operation of the mind, similar to that by which we conceive
the other.  This circumstance I have observ'd to be of great moment;
and we may establish it for a general rule, that whatever ideas place
the mind in the same disposition or in similar ones, are very apt to
be confounded\,\ldots

Now what other objects, besides identical ones, are capable of placing
the mind in the same disposition, when it considers them, and of
causing the same uninterrupted passage of the imagination from one
idea to another?\,\ldots I immediately reply, that a succession of
related objects places the mind in this disposition, and is consider'd
with the same smooth and uninterrupted progress of the imagination, as
attends the view of the same invariable object.  The very nature and
essence of relation is to connect our ideas with each other, and upon
the appearance of the one, to facilitate the transition to its
correlative.  The passage betwixt related ideas is, therefore, so
smooth and easy, that it produces little alteration on the mind, and
seems like the continuation of the same action; and as the
continuation of the same action is an effect of the continu'd view of
the same object, 'tis for this reason we attribute sameness to every
succession of related objects.  The thought slides along the
succession with equal facility, as if it considered only one object;
and therefore confounds the succession with the identity
\citep[135]{hume2000}.
\end{squote}

Hume claims that from a succession of similar impressions, we come to
believe, through a ``fiction of the imagination'', that there is a
single enduring object causing the succession of impressions.
Likewise, I am suggesting that if this essentialist theory is true,
then we come to believe that chairs can change their parts through a
fiction of the imagination.  When looking at a ``successive chair'',
we see a series of sums that resemble each other in their appearance
and spatiotemporal location.  Due to such great similarities, we
mistakenly take them to be a single enduring thing.

\subsection{What can we learn from Fine's theory?}
\label{need-fine}
I have argued that we can identify ordinary things like chairs as
mereological sums, and we can identify things like groups as sets.  It
is therefore not necessary to use Fine's theory of operators (Section
\ref{fine-c}) to describe these things.  Is there anything we can take
away from Fine's theory?

At the very least, Fine's theory is valuable for its insight that
there are different ways of being a part.  It shows that sums and sets
both have parts, but in different ways.  It suggests that there are
also sequences, strings, words, poems, events, and quantities, each
perhaps having their parts in different ways.

Some philosophers who adhere to a more or less classical mereology
believe that physical or material things are the only things that
exist (van Inwagen is one).  For such philosophers, there is only one
way of being a part, and anything that has parts (which is everything)
is a mereological sum.  I do not share this view; I think there are
also sets, and probably other kinds of things.  I do not think,
therefore, that anything that has parts is a mereological sum.  Sets
have parts, and sets are not sums.  My theory of essentialism must
therefore operate with a definition of mereology that does not entail
that everything that has parts is a sum.  One way (though perhaps not
the best way) to ensure this is to say that mereological sums are all
and only physical things.  Appropriate qualifications may be added to
the definitions in Section \ref{tech}.

But if everything is not a sum, if there are sets and probably other
kinds of things as well, how {\em many} kinds of things are there?  If
the essentialist theory in this section was meant to avoid many
different kinds of overlapping things, how can I allow that, in
addition to sums and sets, there might also be strings, and sequences,
and words, and poems, and an unknown number of other things?

I am not sure.  But one, perhaps minor, advantage of my theory is that
it allows us to retain at least a semblance of our pre-reflective
categorization of ordinary things.  According to Fine, chairs,
statues, lumps, boats, and kittens are all different kinds of things,
occupying different ontological categories.  According to the
essentialist, they are all the same kind of thing---they are all
physical sums.  The essentialist theory may recognize different kinds
of things, but it does not multiply kinds beyond necessity.

\subsection{Deflationary metaphysics}
\label{deflate}
Kathrin Koslicki has an interesting objection to universalist theses
such as the one I appear committed to.  Her objection amounts to this:
if every set of objects (such as the London Bridge, a particle in the
moon, and Cal Ripkin, Jr.) is a thing in its own right (a sum), then
metaphysics becomes uninteresting.  There is no longer any debate
about whether chairs or dogbushes are more ``real'' or have a stronger
claim to existence.  They both exist, and the difference between
chairs and lumpkins is not ontological but conceptual: `chair' is more
embedded in our talk, and so chairs have greater importance to {\em
  us}.  But metaphysically, or ontologically, chairs and dogbushes are
on the same level.  There is no sense in which chairs exist and
lumpkins do not.

In the quote below, Koslicki is criticizing a version of
four-dimensionalism that Sider has previously defended.  Sider's
position was that any collection of objects-at-times composes a sum.
(Sider uses `fusions' to refer to sums.)  For example, a chair is a
fusion of a large number of {\em temporal part} of things (wood
molecules, or atoms, or simples).  Each thing (wood molecule, atom, or
simple) is a fusion of {\em its} temporal parts.  Each temporal part
of the chair is also a thing (a fusion).

Again, I take no stand on whether objects have temporal parts or
rather ``endure'' through time.  But Koslicki's comments are relevant
nonetheless:

\begin{squote}
There is room, in Sider's theory, for {\em some} genuine ontological
disagreements: for example, the universalist, the nihilist and the
holder of the intermediary position genuinely disagree over how many
and which fusions that exist.  But the only genuine ontological
disagreements for which there is room, in Sider's world, are ones that
concern disagreements over ``bare'' fusions, so to speak.  What has
happened to the houses, trees, people, and cars, the familiar concrete
objects of common-sense, whose persistence this account set out to
analyze?  There are no ``deep'' ontological facts as to whether a
given fusion should count as a house or not\,\ldots

[By claiming that there can be genuine ontological disputes while also
  promoting four-dimensionalism,] Sider is guilty of a bit of false
advertising: his account is really a way of saying that, at the end of
the day, there is no interesting {\em ontological} story to be told
about the persistence of our familiar concrete objects of
common-sense; whatever there is to say about the persistence of
houses, trees, people and cars concerns the organization of our
conceptual household \citeyearpar[124--125]{koslicki2003}.
\end{squote}

Koslicki seems to think that we ought to be able to find some
ontological difference between ``the familiar concrete objects of
common-sense'' and ``bare fusions'' like lumpkins or chairs-at-times.
But according to Sider's four-dimensional mereology, anything with
parts is, {\em by definition}, a fusion.  Fusions are just things with
parts.  Lumpkins have parts, and are therefore fusions.  Chairs and
houses have parts, and are therefore fusions.  To complain that
ordinary things should be something more than ``bare fusions'' appears
to exhibit a confusion about what fusions are.

Moreover, as I remarked above (Section \ref{universalism}), why should
what interests us (familiar objects like chairs) be a guide to what
exists?  The only difference between ordinary things like chairs and
unusual things like dogbushes seems to be the fact that we care about
the former and not about the latter.  There does not seem to be any
metaphysical or ontological difference between the two; both are sums
or fusions.  The conclusion that ``the persistence [and other
  properties] of houses, trees, people and cars concerns the
organization of our conceptual household'' therefore seems to be
correct.

However, there {\em is} an ontological difference between some things,
if not between chairs and dogbushes.  One lesson of Kit Fine's theory
of parts is that mereological sums may not be the only kind of
composite thing.  There are apparently sets as well, and strings, and
sequences, and perhaps many other types of thing.  The difference
between a set and a sum is probably an ontological difference, and
identifying what distinguishes sets from sums (and from other kinds of
things) is an interesting metaphysical question.  The field of
metaphysics is not then so barren, as Koslicki seems to have feared.
But it is true that many interesting questions---When are we willing
to call something a chair, and why?  What conditions must be
fulfilled?---are not ontological questions.  They are questions about
our ``conceptual household.''


\chapter*{Conclusion}
\label{concl}

\fancyhead[CE]{\textit{\thetitle}}%
\fancyhead[CO]{\textit{\thetitle}}
\renewcommand{\headrulewidth}{0.0pt}

\chapterpig{Conclusion}
\addcontentsline{toc}{chapter}{Conclusion}
\chaptermark{Conclusion}
\markboth{Conclusion}{Conclusion}
I have argued for a number of claims in the preceding sections.

First, debates in metaphysics such as the one I have been engaged in
are conducted in English (or French, or German) and not in
``Ontologese'' or some other pseudo-language.  If it is ``really'' or
``fundamentally'' the case that there are no chairs, then `there are
no chairs' is true in English.

Second, philosophers who deny that there are chairs have some
difficulty explaining why we nonetheless believe that there are
chairs.  Van Inwagen's explanation fails outright.  Trenton Merricks
claims that because things arranged chairwise matter to us, we have
introduced the word `chair' to refer to them; we are fooled by the
singular nature of the word `chair' and come to think that there is
some single {\em thing} that we are referring to, when in fact there
is not.  This explanation, however, is equally compatible with
universalism: the claim that for every set of things, there is some
other thing they compose.  And universalism is a much more plausible
thesis than the nihilism of Merricks.

Third, if we assume that universalism is true, we have a choice to
make.  We can either adopt a ``plurality theory'' that posits a huge
number of things (and possibly different {\em kinds} of things), or we
can adopt a version of essentialism, maintaining that, strictly
speaking, things do not change their parts over time.  I have
suggested that the essentialist theory avoids some of the excesses of
co-location that plague the plurality theories while offering some
neat solutions to problems of personal identity over time.  But
neither route is obviously superior, and both are defensible.\\

In the Introduction and in Section \ref{stroud}, I emphasized that my
opposition to metaphysical nihilism was based, largely, on the fact
that it is {\em obviously true} that there are chairs.  I claimed to
be arguing for what is clearly so, and rejecting what is clearly not.

But surely, mereological essentialism is not {\em obviously} true.
Some would say it is obviously false.  In either case, I can no longer
claim to be arguing for what is clearly so.

But not everything is clear in metaphysics.  ({\em This} is obviously
true.)  There are a few things that are obviously true; many other
things are not, but they are no less true.  It is obvious that there
are chairs; given that, what are they like?  What sort of thing are
they?  Can they change their parts?  If these questions have answers,
they are not obvious.

Moreover we seem to be forced to choose between two possibilities,
both of which might be decried as obviously false.  If there are
chairs, and all the other things that universalism entails, then
either things change their parts or they do not.  If things change
their parts, then (again ignoring four-dimensionalism) there must be
very many co-located things.  Someone who takes this to be false may
be forced to conclude that things do not change their parts.

This thesis therefore ends with no fully-formed theory.  I have
offered a disjunction: either a plurality theory or an essentialist
theory is correct. I have no decisive intuitions here.  I appreciate
the minimalism of the essentialist solution, which dissolves problems
of persistence and identity over time.  But I recognize its
strangeness, and see also the strengths of theories that posit
pluralities of things.

%	\setcounter{chapter}{4}
%	\setcounter{section}{0}
	
  \backmatter % backmatter makes the index and bibliography appear
              % properly in the t.o.c...
%% \pagestyle{fancy}
\fancyhead[CE]{\textit{References}}
\fancyhead[CO]{\textit{References}}
%% \renewcommand{\headrulewidth}{0.0pt}
\bibliographystyle{chicago}
\bibliography{everything}\newpage

\end{spacing}
\end{document}
