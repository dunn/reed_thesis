% This is the Reed College LaTeX thesis template. Most of the work 
% for the document class was done by Sam Noble (SN), as well as this
% template. Later comments etc. by Ben Salzberg (BTS). Additional
% restructuring and APA support by Jess Youngberg (JY).
% Your comments and suggestions are more than welcome; please email
% them to cus@reed.edu
%

\documentclass[12pt,twoside]{reedfancy}
\usepackage{graphicx,latexsym} 
\usepackage{amssymb,amsthm,amsmath}
\usepackage{longtable,booktabs,setspace} 
\usepackage{chemarr} %% Useful for one reaction arrow, useless if you're not a chem major
\usepackage{url}
\usepackage{natbib}
% \usepackage{times} % other fonts are available like times, bookman, charter, palatino

\usepackage{enumitem}
\setcitestyle{aysep={}}
\synctex=1

\DeclareSymbolFont{symbolsC}{U}{txsyc}{m}{n}
\DeclareMathSymbol{\strictif}{\mathrel}{symbolsC}{74}
\DeclareMathSymbol{\boxright}{\mathrel}{symbolsC}{128}

\newenvironment{squote}{\begin{quote}\begin{singlespace}}{\end{singlespace}\end{quote}}

\newcommand{\stager}[4]%
{%
	\begin{spacing}{1}%
	\vspace{0pt}
		\begin{description}[style=nextline, noitemsep, parsep=0pt, topsep=0pt, leftmargin=15mm, itemindent=-10mm, font=\mdseries]
			\item[\textsc{#1} \emph{#2}] #3
			\item[]%
			\begin{flushright}#4\end{flushright}
		\end{description}%
	\end{spacing}%
}

\newcommand{\stage}[3]%
{%
	\begin{spacing}{1}%
	\vspace{0pt}
		\begin{description}[style=nextline, parsep=0pt, leftmargin=15mm, itemindent=-10mm, font=\mdseries]
			\item[\textsc{#1} \emph{#2}] #3
		\end{description}%
	\end{spacing}%
}

% moved to reedfancy.cls:
%\def\thetitle{oink}
%\renewcommand{\firstmark}{\thetitle}
%\newcommand{\chapterpig}[1]{\def\thetitle{#1}}

\title{A World Without Us}
\author{Alexander A. Dunn}
% The month and year that you submit your FINAL draft TO THE LIBRARY (May or December)
\date{May 2012}
\division{Philosophy and Other Things}
\advisor{Paul Hovda}
\department{Philosophy}

\setlength{\parskip}{0pt}

%%%%%%%%%%%%%%%%%%%%%%%%%%%%%%%%%%%%%%%%%%%%

\begin{document}

  \maketitle
  \frontmatter % this stuff will be roman-numbered
  \pagestyle{empty} % this removes page numbers from the frontmatter

% is this a good idea:?
%\begin{spacing}{1.35}

% Acknowledgements (Acceptable American spelling) are optional
% So are Acknowledgments (proper English spelling)
    \chapter*{Acknowledgements}
	I want to thank Peter Unger for infuriating me. Everything he has written is stupendously false.

% The preface is optional
% To remove it, comment it out or delete it.
    \chapter*{Preface}
	This is an example of a thesis setup to use the reed thesis document class.

    \tableofcontents
% if you want a list of tables, optional
   % \listoftables
% if you want a list of figures, also optional
   % \listoffigures

% The abstract is not required if you're writing a creative thesis (but aren't they all?)
% If your abstract is longer than a page, there may be a formatting issue.
    \chapter*{Abstract}
	The preface pretty much says it all.

  \mainmatter % here the regular arabic numbering starts
  \pagestyle{fancyplain} % turns page numbering back on

%The \introduction command is provided as a convenience.
%if you want special chapter formatting, you'll probably want to avoid using it altogether

    \chapter*{Introduction}
         \addcontentsline{toc}{chapter}{Introduction}
	\chaptermark{Introduction}
	\markboth{Introduction}{Introduction}
	% The three lines above are to make sure that the headers are right, that the intro gets included in the table of contents, and that it doesn't get numbered 1 so that chapter one is 1.
	
	Welcome to the \LaTeX\ thesis template. If you've never used \TeX\ or \LaTeX\ before, you'll have an initial learning period to go through, but the results of a nicely formatted thesis are worth it for more than the aesthetic benefit: markup like \LaTeX\ is more consistent than the output of a word processor, much less prone to corruption or crashing and the resulting file is smaller than a Word file. While you may have never had problems using Word in the past, your thesis is going to be about twice as large and complex as anything you've written before, taxing Word's capabilities. If you're still on the fence about  using \LaTeX, read the Introduction to LaTeX on the CUS site as well as skim the following template and give it a few weeks. Pretty soon all the markup gibberish will become second nature.

\section{Why use it?}
	
\LaTeX\ does a great job of formatting tables and paragraphs. Its line-breaking algorithm was the subject of a PhD.\thinspace thesis. It does a fine job of automatically inserting ligatures, and to top it all off it is the only way to typeset good-looking mathematics.

\section{Who should use it?}

Anyone who needs to use math, tables, a lot of figures, complex cross-references, IPA or who just cares about the final appearance of their document should use \LaTeX. At Reed, math majors are required to use it, most physics majors will want to use it, and many other science majors may want it also.

\documentclass[11pt]{article}
\usepackage[margin=1.25in]{geometry}
\geometry{letterpaper}
\usepackage{graphicx}
%\usepackage{tipa}
%\usepackage{exaccent}
%\usepackage{txfonts}
%\usepackage{pxfonts}
\usepackage{enumitem}
%\usepackage{amssymb}
\usepackage{amsmath}
\usepackage{epstopdf}
\usepackage{setspace}
\usepackage{natbib}
\setcitestyle{aysep={}}
\synctex=1

\DeclareSymbolFont{symbolsC}{U}{txsyc}{m}{n}
\DeclareMathSymbol{\strictif}{\mathrel}{symbolsC}{74}
\DeclareMathSymbol{\boxright}{\mathrel}{symbolsC}{128}

\newenvironment{squote}{\begin{quote}\begin{singlespace}}{\end{singlespace}\end{quote}}

\newcommand{\stager}[4]%
{%
	\begin{spacing}{1}%
	\vspace{0pt}
		\begin{description}[style=nextline, noitemsep, parsep=0pt, topsep=0pt, leftmargin=15mm, itemindent=-10mm, font=\mdseries]
			\item[\textsc{#1} \emph{#2}] #3
			\item[]%
			\begin{flushright}#4\end{flushright}
		\end{description}%
	\end{spacing}%
}

\newcommand{\stage}[3]%
{%
	\begin{spacing}{1}%
	\vspace{0pt}
		\begin{description}[style=nextline, parsep=0pt, leftmargin=15mm, itemindent=-10mm, font=\mdseries]
			\item[\textsc{#1} \emph{#2}] #3
		\end{description}%
	\end{spacing}%
}

\title{Alien Explorers and Conceptual Schemes}
\author{Alexander A. Dunn}
\begin{document}
%\maketitle
%\begin{spacing}{1}

\section{Intuitions}
Gareth Evans once claimed that ``with pliant enough intuitions you can swallow anything in philosophy''~(\citeyear[192]{evans1973}). That's an hypothesis that demands testing. So I will claim that, if humans did not exist, neither would dogs, cats, mountains, forests, lakes, rivers, pigs, apples, shrubs, stars, and pretty much everything else. But before I try to get you to swallow this, I'm going to tell a story.

\subsection{Alien explorers meet a metaphysician}
{\em (Our protagonist sits in a chair in front of the fire.)} \\

\stager{Protagonist}{}{Suppose some years in the future we explore a distant planet very unlike our own. There seems to be organic life, but whether it is intelligent or not is hard to say. There seem to be artificial modifications to the environment, but again it's difficult to say what is natural and what isn't. But our intrepid scientists refuse to be baffled by this strange world. After some years of research they manage to pin down the physical processes and produce a model of the planet with really quite good predictive capabilities.}{{\em (A child's head droops.)}}

\stager{Protagonist}{(continuing)}{We've classified some phenomena as organisms with their set of biological underpinnings, and we've classified others as natural though non-living processes, such as weather patterns and geological change. There might be some things we've overlooked, but it doesn't look like there are going to be many more surprises. So I think it's safe to say that we've recognized most of what's on this planet, don't you?}{{\em (Vague nods. A sleeping \textsc{metaphysician} stirs briefly.)}}

\stager{Protagonist}{}{ Now I'm afraid I must admit that I haven't been entirely honest with you. The explorers in this story are not us, but an intelligent alien species. The planet is not distant at all; they are exploring our own Earth. And yet I think we must agree that we were correct---}{{\em (The metaphysician awakes with a start.)}}

\stage{Metaphysician}{(rising angrily)}{Now look here! These fool aliens don't recognize their own ignorance, let alone half of what exists on Earth! They forgot about adzes and Axminsters, boats and books, catalogs and cups, doors and dumbwaiters, earrings and elegance! Their ontology would fit in a knapsack!}

And so the question is asked to you, readers: are the things that went unseen by the aliens {\em really there?}

%\bibliography{everything}
%\bibliographystyle{ChicagoReedweb}
%
%\end{spacing}
\end{document}
	
\documentclass[11pt]{standalone} \newif\ifstandlone \standalonetrue
\usepackage{standalone}
\usepackage[left=1.75in, right=1.75in, top=1.25in, bottom=1.25in]{geometry}
\geometry{letterpaper}
\usepackage{graphicx}
%\usepackage{tipa}
%\usepackage{exaccent}
%\usepackage{txfonts}
%\usepackage{pxfonts}
\usepackage{enumitem}
%\usepackage{amssymb}
\usepackage{amsmath}
\usepackage{epstopdf}
\usepackage{setspace}
\usepackage{natbib}
\setcitestyle{aysep={}}
\usepackage{hyperref}
		
\synctex=1

\DeclareSymbolFont{symbolsC}{U}{txsyc}{m}{n}
\DeclareMathSymbol{\strictif}{\mathrel}{symbolsC}{74}
\DeclareMathSymbol{\boxright}{\mathrel}{symbolsC}{128}

\newenvironment{squote}{%
	\begin{quote}\begin{singlespace}%
	}{%
	\end{singlespace}\end{quote}}

\newcommand{\stager}[4]%
{%
	\begin{spacing}{1}%
	\vspace{0pt}
		\begin{description}[style=nextline, noitemsep,
                    parsep=0pt, topsep=0pt, leftmargin=15mm,
                    itemindent=-10mm, font=\mdseries]
			\item[\textsc{#1} \emph{#2}] #3
			\item[]%
			\begin{flushright}#4\end{flushright}
		\end{description}%
	\end{spacing}%
}

\newcommand{\stage}[3]%
{%
	\begin{spacing}{1}%
	\vspace{0pt}
		\begin{description}[style=nextline, parsep=0pt,
                    leftmargin=15mm, itemindent=-10mm, font=\mdseries]
			\item[\textsc{#1} \emph{#2}] #3
		\end{description}%
	\end{spacing}%
}

\newenvironment{inq}{\vspace{0pt}%
	\begin{list}{}%
	{\setlength\labelwidth{0pt}%
	\setlength\leftmargin{2.5\oddsidemargin}%
	\setlength\rightmargin{\leftmargin}}
	\begin{spacing}{1}
	\item[]%
	}{
	\end{spacing}
	\end{list}
	\vspace{10pt}
	%\noindent%
	}

\title{Brute facts and arbitrary objects}
\author{Alex Dunn}
\begin{document}
\ifstandalone
\maketitle
\begin{spacing}{1.5}
\fi

\section{Unger's nihilism}
\label{unger}
Just as resolutely as we would deny the existence of ghosts, so Peter
Unger has denied the existence of all `ordinary things'---such things
as ``tables and chairs and spears\,\ldots\,swizzle sticks and
sousaphones\,\ldots\,stones and rocks and twigs, and also tumbleweeds
and fingernails''~(\citeyear[117]{unger1979}).  He does not consider
them merely `subjective' as opposed to objectively so---like van
Inwagen, he claims that they simply do not exist.  He comes to this
conclusion from a different direction, however.  As we will see, van
Inwagen's denial of the existence of `ordinary things' is a
consequence of his theory of composition (under what conditions some
things compose another thing).  Unger, on the other hand, claims that
terms for ordinary things, like `chair', are {\em incoherent}.  Unger
claims that incoherent terms cannot apply to anything in the world;
therefore he concludes that there are no chairs (or any other ordinary
thing).

Unger has two largely independent motivations for his claim that terms
like `chair' are incoherent.  One is the sorites paradox, and the
other is the `problem of the many'.

\subsection{Sorites paradoxes}
A typical instance of the sorites paradox begins by having us imagine
some ordinary object; let us use a heap of sand.  Now suppose we
remove a single grain of sand.  If we were inclined to believe that
the initial quantity of sand did in fact constitute a heap, then after
the removal of a single grain, we should presumably still have a heap
(albeit a slightly smaller one).  It seems very implausible to think
that one grain of sand more or less could {\em ever} make a difference
as to whether something is or is not a heap.

But having conceded (a) that there is a heap and (b) that the removal
of a single grain cannot make the difference as to whether a quantity
of sand is a heap, we have unwittingly put our foot in it.  For if the
removal of a single grain {\em never} transforms a heap into a
non-heap, then by repeatedly removing one grain after another, we will
eventually find ourselves with a heap that consists of no sand at
all.  But it seems absurd to suppose that there could be a heap of
sand that is composed of no sand---indeed, of nothing whatsover.

This is the sorites paradox.  While a heap is a useful example,
because it is so ill-defined, similar problems appear to afflict all
ordinary things.  Unger illustrates the difficulty for stones:

\begin{squote}
Consider a stone, consisting of a certain finite number of atoms.  If
we or some physical process should remove one atom, without
replacement, then there is left that number minus one, presumably
constituting a stone still\,\ldots\,after another atom is removed,
there is that original number minus two; so far, so good.  But after
that certain number has been removed, in similar stepwise fashion,
there are no atoms at all in the situation, while we must still be
supposing that there is a stone there.  But as we have already agreed,
if there is a stone present, then there must be some atoms\,\ldots\,I
suggest that any adequate response to this contradiction must
include\,\ldots\,the denial of the existence of even a single
stone.~\citep[121--122]{unger1979}
\end{squote}
Unger understands this dilemma to apply across the board, and
correspondingly argues that we should deny the existence of even a
single ordinary thing.

\subsection{The problem of the many}
The `problem of the many', as Unger terms this second difficulty for
ordinary things, follows a similar line of reasoning.  If we consider
an ordinary thing---take a cloud, for instance---it is presumably
composed of molecules.  There is probably then a set of molecules, the
members of which compose the cloud.  Call that set $S$.  Now consider
$S_1$.  This is a set of molecules that includes all of the members of
$S$ as well as one additional molecule.  Do the members of $S_1$
compose a cloud?  Surely they are just as well suited to do so.  Now
consider $S_2$\,\ldots

Because these numerous 'candidates' are equally (or nearly equally)
well suited to be clouds, we seem forced to conclude that there are
either many clouds where we supposed there to be one, or rather no
clouds at all:

\begin{squote}
No matter where we start, the complex first chosen has nothing
objectively in its favor to make it a better candidate for cloudhood
than so many of its overlappers are.  Putting the matter somewhat
personally, each one's claim to be a cloud is just as good, no better
and no worse, than each of the many others.  And, by all odds, each
complex has \emph{at least} as good a claim as any still further real
entity in the situation.  So, either \emph{all} of \emph{them} make it
or else \emph{nothing} does; in this real situation, either there are
many clouds or else there really are no clouds at all
\citep[415--??]{unger1980a}.
\end{squote}

The problem of the many can also arise by considering the {\em
  boundary} of a given cloud.  It is natural to suppose that a cloud
has a determinate boundary.  But if we look at the edge of the cloud,
where we suppose the boundary to be, ``we may find, side by side, or
themselves overlapping, a great many potential boundaries for
clouds\,\ldots\,if our alleged typical item {[}the cloud{]} is indeed
a typical cloud, then many of these candidates, millions at least, do
not fail to be clouds altogether but are clouds of some
sort''~\citep[420--421]{unger1980a}.

The pattern of argumentation is the same for both approaches.  For a
certain cloud, a given set of members or a given boundary is supposed,
and it is argued that a set or boundary that differs minimally from
the original must also compose our bound a cloud.  The new set or
boundary does not appear to differ from the original in any relevant
way; there seems no principled way to deny that if the first set's
members compose a cloud, the second set's members do too.  And since
there are a great deal of very similar sets and boundaries, we find
ourselves with a plurality of clouds.

And of course, Unger does not rest content with applying the problem
of the many to clouds.  All ordinary objects get the same treatment;
he concludes that either there are a great many of them, or there are
none at all.  He claims, predictably, that the latter disjunct is
preferable.

\section{So what's the problem?}
We find ourselves wanting to hold three theses, which appear mutually
inconsistent:

\begin{enumerate}
  \item There is at least one chair (stone, cloud).
  \item If a chair (stone, cloud) exists, it must be composed of
    matter.
  \item If a chair (stone, cloud), exists, the removal of a single
    molecule (or otherwise insignificant quantity of matter) from it
    cannot destroy it or cause it to cease to exist.

We seem to be clearly caught in a paradox; the only question is where
we have gone wrong.

But have we, in fact, gone wrong?  Peter Unger thinks that we are
right on target:

\begin{squote}
While Eubulides' contribution has often been labeled `the sorites
paradox', there is nothing here which is a paradox in any
philosophically important sense\,\ldots\,Accepting our negative
conclusions here does not mean important logical trouble for us; we
only think we have troubles while we refuse to admit their validity
(\citeyear[145]{unger1979}).
\end{squote}

Our situation is only paradoxical, says Unger, while we unreflectingly
cling to the first thesis.  If, however, we come to see that there are
no chairs (stones, clouds), then we happily escape paradox: if there
are no chairs (stones, clouds) to begin with, we do not have to worry
about what the addition or removal of small amounts of matter would do
to them; nor do we need concern ourselves with what they would be made
of.

But thing are not quite so simple.  First, adding to the
implausibility of Unger's view, he must deny that our use of ordinary
terms like `chair' (`stone', `cloud') follow any sort of pattern or
display any competence at all.  Second, even if we managed to swallow
that consequence, Unger has no explanation as to why we believe that
there are chairs (stones, clouds).

\subsection{Competence and correctness}
Setting aside whether or not expressions of propositions like ``that's
a chair'' are ever \emph{true}, it seems right to say that there are at
least correct and incorrect uses of the terms.  For a word like
`chair' (`stone', `cloud') we generally do not say that a child has
learned how to use it until she is capable of using it in a certain
way.  We admit that she understands what `chair' (`stone', `cloud')
means or what a chair (stone, cloud) is when she displays a certain
competence with the term.  If instead of using `chair' to refer to
chairs she used it to refer to dogs or people, we would say that she
is confused and attempt to correct her use.

But Unger maintains that this is all an illusion, and that there is no
such thing as the correct or incorrect use of a term like `chair'
(`stone', `cloud'):

\begin{squote}
Concerning words and kinds, now, we might say this.  First, we might
say that it is in connection with \emph{semantics} that our reasonings have
what are their most obvious implications and, second, that their most
obvious semantic implications concern certain \emph{sortal nouns}, namely,
those which purport to denote ordinary things.  Thus, it appears quite
obvious to us now that there will be no application to things for such
nouns as `stone' and `rock', `twig' and `log', `planet' and `sun',
`mountain' and `lake', `sweater' and `cardigan', `telescope' and
`microscope', and so on, and so forth.  Simple positive sentences
containing these terms will never, given their current meanings,
express anything true, correct, accurate, etc., or even anything which
is anywhere close to being any of those things
(\citeyear[148]{unger1979}).
\end{squote}

This seems simply bizarre.  On what grounds, then, do parents correct
their children with respect to their use of ordinary terms?  Are they
compelled by some irrational force to consider certain utterances
correct and others incorrect?  One may question whether or not we use
ordinary term entirely consistently, but it seems simply false that,
necessarily, we {\em never} use (or have used, or will use) ordinary
terms in correct, as opposed to incorrect, ways:

\begin{squote}
It is\,\ldots\,unclear how far our use of e.g. the vocabulary of
colours \emph{is} consistent.  The descriptions given of awkward cases
may vary from occasion to occasion.  Besides that, the notion of using
a predicate consistently would appear to require some objective
criteria for variation in relevant respects among items to be
described in terms of it; but what is distinctive about observational
predicates is exactly the lack of such criteria.  So it would be
unwise to lean too heavily, as though it were a matter of hard fact,
upon the consistency of our employment of colour predicates.  What,
however, may be depended upon is that our use of these predicates is
largely \emph{successful}; the expectations which we form on the basis
of others' ascriptions of colour are not usually disappointed.
Agreement is generally possible about how colours are to be described;
and this, of course, is equivalent to saying that others \emph{seem}
to use colour predicates in a largely consistent way
\citep[361]{wright1975}.
\end{squote}

None of this {\em proves} that Unger is wrong, of course.  But it is
worth remembering how really implausible his view is.  Moreover, in
the next section we will see that certain questions about our beliefs
arise as a result of his denial of ordinary things.  Unfortunately for
Unger, he has no means to answer these questions, which remain a
(probably unsuperable) barrier to the acceptability of his thesis.

\section{Beliefs in things}
Having denied that ordinary things exist, Unger must either explain
how our beliefs about stones should be properly understood (van
Inwagen, as we shall see has an interesting paraphrasing strategy) or
he must deny that we really {\em do} have any beliefs about stones.
Unger opts for the latter and claims that, like the person who thought
she had a belief about a ghost, we are wrong to think that we have
any coherent beliefs about stones or any other ordinary things.  Unger
says that, like `ghost', our ``terms for ordinary things are
incoherent [and] cannot apply to anything
real''~\citep[147]{unger1979}.  A consequence of this is that our
language and thought concerning all such things is directed toward
{\em nothing at all}: ``it may well be that I have never {\em thought
  of} any stones at all, or tables, or even human hands.  If that is
so, then it would seem that {\em a fortiori} I do not {\em know}
anything {\em about these entities}, however commonly I might
otherwise suppose''~(\citeyear[458]{unger1980a}).

This all seems very strange.  Concerning ghosts, ``it is difficult
even to find a fully coherent belief that might be exposed as false;
we discover, at best, obscurity or perhaps confusion\,\ldots\,do we
really understand what sort of thing a ghost is supposed to
be''~\citep[76]{stroud2000a}?  If someone tries to tell me about the
ghost that visited him the previous night, it does not seem unjust to
say that he doesn't really know what he is talking about.  But can
this line be extended to some of the most common objects of
experience?

When we denied the existence of ghosts, we denied also others' beliefs
in them.  We did not, however, deny that people have beliefs which
they take to be about ghosts.  But we were able to show that these
beliefs were not {\em about} ghosts; in most cases they were about
nothing at all.  Likewise, Unger cannot deny that we have beliefs that
we take to be about tables, chairs, and all the other things that he
denies exist.  If our beliefs about tables and chairs are really
beliefs about nothing at all, then what causes us to form these
beliefs?  Why do we believe in ordinary things to begin with?

\subsection{Causes of belief}
\label{unger-cause}
People who believe in ghosts probably do so because they have
unreflectively embraced the superstitions of their culture.  They may
initially come to believe that ghosts exist on the testimony of other
people---older siblings, perhaps---or by reading too many ghost
stories.  Much as Catherine in Jane Austen's {\em Northanger Abbey}
jumps to the most macabre conclusions as a result of having absorbed
too many gothic novels, so might a gullible reader of ghost stories go
on to interpret such innocent phenomena as reflections of the moon as
ghostly assailants.  Those of us who have not taken our cues from
fiction would be more likely to recognize such phenomena as tricks of
the light.  Even if we were to see something that was definitely {\em
  not} a trick of the light, we would sooner attribute it to an
hallucination than countenance the possibility of ghosts.  Suppose
{\em you} saw what you took to be a ghost in an empty, well-lit room.
Most of us would still, even if presented with such a vision, {\em
  refuse to believe in ghosts}.  This is because we know that the
probability of there being such spirits is far less than the
probability of us experiencing cracks in our sanity.  Undermining my
belief that ghosts don't exist would require a great deal---for
example, my friend and I both apparently seeing the {\em same} ghost
at the same time, and knowing that we were each experiencing the same
vision.  (Even then, we would want further confirmed sightings to
convince us that we weren't, in fact, crazy.)

If this is an accurate characterization of our beliefs concerning
ghosts, it is a very different characterization than one we might give
of how we learn about and come to believe in chairs.  Chairs are not
something that children learn about from stories.  A child probably
learns what a chair is as an answer to the question, ``What is {\em
  that?}\,''  

Let us suppose that the child is pointing at a chair in the center of
a well-lit room containing no other furniture.  The chair is clearly
visible.  If someone were to believe they were pointing at a ghost in
a similarly well-lit situation, we could safely assume that they would
be experiencing a hallucination.  But clearly the child is not
hallucinating.  There is {\em something} (or some things) in the
center of the room; what Unger wants to deny is that there is a {\em
  chair} there.

Unger would admit, I think, that there is a quantity of matter,
arranged in a certain way, in the center of the room.  (Unger
presumably also denies the existence of rooms, so this would have to
be expressed differently, but never mind.)  He sees it, just as well
as we do; it's not as though Unger can't see straight.  All he's
saying is that what he is looking at is not a chair.

But if all that is in the center of the room is a mass of matter, {\em
  why do we believe that there is a chair there?}  To say that there
is a chair in the center of the room would, according to Unger, be
neither true, nor accurate, nor correct, nor ``anything which is
anywhere close to being any of those things'' \citep[148]{unger1979}.
So where on earth do we get the idea that there is a chair there?

\hline

\subsection{Loose truth}
\label{loose-u}
One sympathetic to Unger's thesis might admit that they are
communicating about something, but deny that the subject of their
communication is a chair.  This philosopher would take refuge in the
notion of `loose truth'.  She would maintain that it is strictly false
that there is a chair in the room, but that it is loosely true; it is
close enough to the truth for practical purposes.  These practical
purposes include the communication we have observed above.  She will
appeal to such examples as this:

\stage{Countess}{}{Where on earth am I going to find someone to invest
  in my eel farm?}

\stage{Count}{(pointing)}{There's a millionaire for you.}

\stage{Countess}{(incredulous)}{Henry? A millionaire? He hasn't got
  above nine hundred ninety-five thousand pounds.}

\stage{Count}{}{Oh? Well, it's close enough.}

We are supposing that there is no millionaire in the room; strictly
speaking, the count said something false with ``There's a
millionaire''.  Nonetheless, communication occurred because the term
`millionaire' made the count's referential intention clear: he
intended to refer to a person who was {\em almost} a millionaire.
(The term is regularly used to refer to non-millionaires who have
relatively great wealth.)  The Ungerian is claiming that this is
analogous to the case of the parent and child.  Strictly speaking,
what the parent said (``That's a chair'') was false, but it allowed
for communication by making the parent's referential intention clear.

Is this a coherent objection?  Without concerning ourselves too much
with the nature of loose truth, I think it is fair to claim that, just
as a (strict) truth has a `truthmaker', so a loose truth must have a
`loose-truthmaker'.  In the example above, the truthmaker for
``There's a millionaire'' would have been the fact that the count was
referring to a millionaire.  This fact did not obtain, so the
statement is, strictly speaking, false.  The loose-truthmaker is
evidently the fact that the count is referring to someone who is {\em
  almost} a millionaire.  (What counts as `almost' will no doubt vary
between contexts, but in this context I am supposing it is true that
Henry is almost a millionaire.)

The truthmaker for ``That's a chair'' must obviously be the fact that
the parent is referring to a chair.  According to Unger, there are no
chairs, so nobody can refer to them.  The parent's statement would
therefore be, strictly speaking, false.  Now what is the
loose-truthmaker for the parent's use of ``That's a chair''?  It cannot
be the fact that there is {\em almost} a chair (a partially built
chair?), at least not if that entails that there could ever be a
chair.  Unger maintains that the kind of object picked out by `chair'
is ``never instanced''~(\citeyear[147]{unger1979}).  Is there,
perhaps, something closely resembling a chair in the room, and the
parent is referring to {\em that} thing instead?  This raises two
objections of its own.  First, what is there in the room that
``closely resembles'' a chair, other than the chair itself?  Second, if
we cannot ever have coherent thoughts about chairs (and therefore
cannot know anything about chairs), how are we supposed to know what
resembles a chair?

I do not think there are satisfactory answers to these questions.
Moreover, I do not think Unger ever espoused a `loose-truth' nihilism,
so we are not slighting him by moving on.

\subsection{The moral}
\label{moral}
There are limits to what one can resolutely deny the existence of.  We
can deny that certain things, like ghosts, exist {\em and} deny that
people have beliefs in them.  We can do this because in each situation
where a person has a belief about what they take to be a ghost, we can
show that their belief is really about nothing at all.  If someone
sees a reflection of the moon or experiences a hallucination, and so
thinks she is seeing a ghost, we can say that she is afraid of nothing
at all.  Under no circumstances must we say that her belief is really
about a ghost.  Moreover, we can explain how people come to believe in
ghosts---they read too many ghost stories, or believe the lies of
others.

This is not something we can do with tables, chairs, and other
``ordinary things'', let alone people.  For one, Unger has no
explanation of how we come to form our beliefs in these things, if not
by {\em seeing them}.  Secondly, to deny that our thought and talk
about such things are really about nothing at all is to deny that
communication regularly occurs.  Bizarrely, this is a consequence
Unger appears willing to accept:
\begin{squote}
Now, it must of course be admitted that these arguments [for his
  strain of nihilism] undermine the possibility of any endeavor I
should try to propose, or even the putative thought that I should
propose anything, just as all of my putative essay is undermined.  But
even so, I shall (incoherently) propose that what we have now to do is
invent new expressions which are not inconsistent ones, and by means
of which we may, to some significant extent, think coherently about
concrete reality~(\citeyear[544]{unger1980b}).
\end{squote}
If Unger seriously believes this, then he could not expect us even to
understand his essay ({\em why would he write it?}).  But I think it
is safe to say that Unger does {\em not} actually believe that there
are no people or ordinary things.  In a book on ethics, Unger has
unambiguously expressed his belief in people:

\begin{squote}
Each year millions of children die from easy to beat disease, from
malnutrition, and from bad drinking water\,\ldots\,As UNICEF has made
clear to millions of us well-off American adults at one time or
another, with a packet of oral rehydration salts that costs 15 cents,
a child can be saved from dying soon~(\citeyear[3]{unger1996}).
\end{squote}
There are only two possibilities: either Unger does believe that
people (at least children and Americans) do exist, or he takes himself
to be flat-out lying in the quoted passage.

As far as other ordinary things go, Unger claims to ``often now
believe that there really are no tables or rocks, and never so firmly
believe that there are such things as I once
did''~(\citeyear[543]{unger1980b}).  All I can say is that I don't
believe him.  (To show that he does believe in these things, we would
need to spend some time with him, observing his behavior.  We could
invite him for a walk along a trail with lots of low-hanging branches,
then warn him about them.)

\section{The problem of the many}
\label{many}
In section~\ref{unger} we looked at a version of metaphysical
nihilism.  Peter Unger attempted to deny that any of the `ordinary
things' in the world (tables, chairs, apples, people, \&c.) actually
exist.  His motivation for this claim was drawn from an apparent
paradox involving the terms for ordinary things.  If we have a stone,
then removing one atom of matter will not destroy the stone.  Nor will
removing another atom.  But if we remove enough atoms, there will not
be a stone.  One solution to this puzzle is to deny that there ever is
a stone.  But this, we have seen, is not workable.

Another solution is to claim that there are {\em many} stones where we
once thought there was only one.  The motivation for this claim
sometimes comes from what Unger calls ``the problem of the many''.
There are a number of different formulations of this problem.  Van
Inwagen nicely summarizes one:
\begin{squote}
Assume I exist.  Then certain simples compose me.  Call them `M'.
Now, a single simple is a negligible item indeed.  Let $x$ be one of
these negligible parts of me---one that is somewhere in my right arm,
say.  Now consider the simples that compose me {\em other than} $x$
(`M -- $x$').  Since $x$ is so very negligible, M -- $x$ {\em could}
[my emphasis] compose a human being just as well as M could.  We may
say that M and M -- $x$ are ``equally well suited'' to compose human
beings.  And, of course, for {\em any} simple $y$, ``M -- $y$ will be
as well suited to compose a human being as M are.  Moreover, it would
be surprising indeed if there were not a simple $z$ such that ``M +
$z$'' were as well suited to compose a human being as M are.  It
would, in fact (if I may once more use this phrase), be intolerably
arbitrary to say that M composed a human being although M -- $x$ {\em
  didn't} [my emphasis] and M -- $y$ {\em didn't} [my emphasis] and M
+ $z$ {\em didn't} [my emphasis].  Suppose, therefore, that M -- $x$
et al.\ {\em do} [my emphasis] compose human
beings~(\citeyear[215]{inwagen1995}).
\end{squote}

I think this formulation is problematic.  We are supposing that M does
compose a human being.  But it does not immediately follow from this
that M -- $x$ also composes a human being.  As I have pointed out with
italics, there is a slide from the claim that M -- $x$ {\em could}
compose a human being to the claim that M -- $x$ {\em does} compose a
human being.  As an analogy, take a house of blocks.  Suppose that the
blocks do compose the house.  Is there also something composed by the
blocks minus one? There {\em could} be; but intuitively, there would
be only if that one block were removed.  Then we would have a house
with a missing roof.  But we do not obviously have that second thing
already, without having removed the block.

Van Inwagen understands there to be a number of additional premises
required for this argument.  He formulates them thus:
\begin{enumerate}
	\item In every situation of which we should ordinarily say
          that it contained just one man, there are many sets of
          simples whose members are as suitably arranged to compose
          men as any simples could be.  \label{many1}
	\item The members of each of these sets compose
          something.  \label{many2}
	\item Each of these ``somethings'' is a man, provided there
          are any men at all.  \label{many3}
	\item If I exist, there is a
          man~(\citeyear[216]{inwagen1995}).  \label{many4}
\end{enumerate}
I think only~\ref{many4} is uncontroversially true.  There are many
difficult questions raised by~\ref{many2}, and I'm not sure how to
answer them.  I'm quite sure, however, that the first and third
premises are false.  These two are closely related, so we'll examine
them together.

\subsection{Problems with the first and third premises}
\label{many13p}
Unger begins his presentation of the problem of the many by directing
our attention to a cloud; looking at the edge of the cloud, ``all that
is there to be seen is a {\em gradual transition} from the more dense
[concentration of water molecules] to the less so\,\ldots\,there is no
natural break, or boundary, or stopping place, for any would-be cloud
to have''~(\citeyear[415]{unger1980a}).  Therefore any boundary we
choose for the cloud will be somewhat arbitrary.  A minutely larger or
smaller boundary would be just as `suitable' for bounding a cloud;
``if our alleged typical item [the cloud] is indeed a typical cloud,
then many of these candidates, millions at least, do not fail to be
clouds altogether but are clouds of some
sort''~(\citeyear[421]{unger1980a}).  Unger draws the conclusion that
either there are millions of such things there, or none.  This
argument can be generalized to chairs, stones, and people.  Regarding
chairs, at least, we have already shown that the conclusion that there
are none is false; perhaps then, there are millions of chairs in our
well-lit room.  But there are two problems with this argument.

\paragraph{An objection regarding communication.}
First, the idea of there being millions of objects where we supposed
there was only one poses a problem for referential communication.  In
fact, it is quite the same problem that Unger's nihilism faced above:
how is referential communication possible under this hypothesis? When
we supposed that there was nothing being referred to by `chair', we
were at a loss to explain how people nonetheless managed to
communicate using that term.  Now we are supposing there to be
millions of chairs, each eligible to be referred to by ``that chair''.
What needs explaining now is how, if there are millions of chairs that
might be referred to, how two speakers can be sure that they are
talking about the {\em same} chair.  When the child points and says
``What's that?'', the parent knows that the child is pointing at {\em
  a} chair, but how is she to determine {\em which} chair the child is
referring to? The child did {\em not} say ``What are those?'' because
she took herself to be referring to {\em one} chair, and expressed her
referential intention accordingly.

Even if the parent and child got lucky and happened to think of the
same chair, how would they know it? If there really were millions of
chairs in the center of the room, it seems implausible to suggest that
anyone could distinguish between each of them.  There would be nothing
the parent or child could do to make it clear to the other which of
the millions of chairs they meant to refer to.  Indeed, they could not
make it clear to themselves.  If I am holding a rock in my hand,
``there are millions of ``overlapping stones'' before me\,\ldots\,how
am I to think of a single one of them, while not then equally thinking
of so many others''~\citep[456]{unger1980a}?  Unger's brand of
universalism, like his nihilism, precludes successful communication,
and so must be rejected.

\paragraph{An objection regarding boundaries.}
Second, it seems to be simply false that all boundaries drawn about a
cloud are equally arbitrary.  If a cloud is a concentration of water
molecules in the air, then the boundary of the cloud is the edge of
the concentration, beyond which the level of water in the air is
normal.  There are not millions of clouds with millions of different
boundaries; any stipulated boundary that falls inside or outside the
actual boundary is arbitrary because it does not track the
concentration of water molecules.  For simples to be `suitably
arranged' so as to compose a cloud, their boundary must be the edge of
the concentration of water molecules.  I'm not sure, but it may be
possible to extend this argument to cover most `ordinary
things'---tables and chairs (and even people) have molecular
boundaries.

Premise~\ref{many1} is therefore false.  For clouds, at least, it is
not true that there are ``many sets of simples whose members are as
suitably arranged'' to compose them.  These closely related sets do
not have boundaries that track the concentration of water molecules in
the air, so they are not suitably arranged to compose clouds.  If the
boundary of the cloud shrunk, then a smaller set of molecules {\em
  would} be suitably arranged to compose the cloud, and so it would.
But just as the blocks minus one {\em would} compose a house (but
don't), so this smaller set {\em would} compose a cloud (but doesn't).
So premise~\ref{many3} seems false too; at least, if these sets
compose anything, they compose something other than a cloud.

\ifstandalone
\bibliography{everything}
\bibliographystyle{ChicagoReedweb}
\end{spacing}
\fi
\end{document}


\chapter*{Conclusion}
         \addcontentsline{toc}{chapter}{Conclusion}
	\chaptermark{Conclusion}
	\markboth{Conclusion}{Conclusion}
	\setcounter{chapter}{4}
	\setcounter{section}{0}
	
Here's a conclusion, demonstrating the use of all that manual incrementing and table of contents adding that has to happen if you use the starred form of the chapter command. The deal is, the chapter command in \LaTeX\ does a lot of things: it increments the chapter counter, it resets the section counter to zero, it puts the name of the chapter into the table of contents and the running headers, and probably some other stuff. 

So, if you remove all that stuff because you don't like it to say ``Chapter 4: Conclusion'', then you have to manually add all the things \LaTeX\ would normally do for you. Maybe someday we'll write a new chapter macro that doesn't add ``Chapter X'' to the beginning of every chapter title.

\section{More info}
And here's some other random info: the first paragraph after a chapter title or section head \emph{shouldn't be} indented, because indents are to tell the reader that you're starting a new paragraph. Since that's obvious after a chapter or section title, proper typesetting doesn't add an indent there. 


%If you feel it necessary to include an appendix, it goes here.
    \appendix
      \chapter{The First Appendix}
      \chapter{The Second Appendix, for Fun}


%This is where endnotes are supposed to go, if you have them.

  \backmatter % backmatter makes the index and bibliography appear properly in the t.o.c...

% Make my bibliography be called "Bibliography" and not "References" (or "Works Cited" or...):
% \renewcommand{\bibname}{Works Cited}
    \bibliographystyle{ChicagoReedweb} % there are a variety of styles available; 
% replace ``plainnat'' with the style of choice. You can refer to files in the bsts or APA 
% subfolder, e.g. 
% \bibliographystyle{APA/apa-good}  % or
% \bibliographystyle{bsts/mla-good} 

% if you're using bibtex, the next line forces every entry in the bibtex file to be included
% in your bibliography, regardless of whether or not you've cited it in the thesis.
    %\nocite{*}
    \bibliography{everything}

%\end{spacing}
% Finally, an index would go here... but it is also optional.
\end{document}
