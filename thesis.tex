% This is the Reed College LaTeX thesis template. Most of the work 
% for the document class was done by Sam Noble (SN), as well as this
% template. Later comments etc. by Ben Salzberg (BTS). Additional
% restructuring and APA support by Jess Youngberg (JY).
% Your comments and suggestions are more than welcome; please email
% them to cus@reed.edu
%

\documentclass[12pt,twoside]{reedfancy}
\usepackage{standalone}
\usepackage{anyfontsize}
\usepackage{graphicx,latexsym} 
\usepackage{amssymb,amsthm,amsmath}
\usepackage{longtable,booktabs,setspace} 
\usepackage{chemarr} %% Useful for one reaction arrow, useless if you're not a chem major
\usepackage{url}
\usepackage{natbib}
% \usepackage{times} % other fonts are available like times, bookman, charter, palatino

\usepackage{enumitem}
\setcitestyle{aysep={}}
\synctex=1

\DeclareSymbolFont{symbolsC}{U}{txsyc}{m}{n}
\DeclareMathSymbol{\strictif}{\mathrel}{symbolsC}{74}
\DeclareMathSymbol{\boxright}{\mathrel}{symbolsC}{128}

\newenvironment{squote}{\begin{quote}\begin{singlespace}}{\end{singlespace}\end{quote}}

\newcommand{\stager}[4]%
{%
	\begin{spacing}{1}%
	\vspace{0pt}
		\begin{description}[style=nextline, noitemsep, parsep=0pt, topsep=0pt, leftmargin=15mm, itemindent=-10mm, font=\mdseries]
			\item[\textsc{#1} \emph{#2}] #3
			\item[]%
			\begin{flushright}#4\end{flushright}
		\end{description}%
	\end{spacing}%
}

\newcommand{\stage}[3]%
{%
	\begin{spacing}{1}%
	\vspace{0pt}
		\begin{description}[style=nextline, parsep=0pt, leftmargin=15mm, itemindent=-10mm, font=\mdseries]
			\item[\textsc{#1} \emph{#2}] #3
		\end{description}%
	\end{spacing}%
}

%squeezing intro quotes
\newenvironment{inq}{%
%	\begin{quote}
	\begin{center}
	\begin{spacing}{1}
%	\linewidth3in
	\list{}{%
	%\textwidth2in
	\leftmargin1in   % this is the adjusting screw
    \rightmargin\leftmargin
	\item\relax}
	}{%
	\endlist
	\end{spacing}
	\end{center}
	%\end{quote}
	%\vspace{10pt}
	}

% moved to reedfancy.cls:
%\def\thetitle{oink}
%\renewcommand{\firstmark}{\thetitle}
%\newcommand{\chapterpig}[1]{\def\thetitle{#1}}

\title{A World Without Us}
\author{Alexander A. Dunn}
% The month and year that you submit your FINAL draft TO THE LIBRARY (May or December)
\date{May 2012}
\division{Philosophy and Other Things}
\advisor{Paul Hovda}
\department{Philosophy}

\setlength{\parskip}{0pt}

%%%%%%%%%%%%%%%%%%%%%%%%%%%%%%%%%%%%%%%%%%%%

\begin{document}

  \maketitle
  \frontmatter % this stuff will be roman-numbered
  \pagestyle{empty} % this removes page numbers from the frontmatter

\begin{spacing}{1.25}

% Acknowledgements (Acceptable American spelling) are optional
% So are Acknowledgments (proper English spelling)
    \chapter*{Acknowledgements}
	I want to thank Peter Unger for infuriating me. Everything he has written is stupendously false.

% The preface is optional
% To remove it, comment it out or delete it.
    \chapter*{Preface}
	This is an example of a thesis setup to use the reed thesis document class.

    \tableofcontents
% if you want a list of tables, optional
   % \listoftables
% if you want a list of figures, also optional
   % \listoffigures

% The abstract is not required if you're writing a creative thesis (but aren't they all?)
% If your abstract is longer than a page, there may be a formatting issue.
    \chapter*{Abstract}
	The preface pretty much says it all.

  \mainmatter % here the regular arabic numbering starts
  \pagestyle{fancyplain} % turns page numbering back on

%The \introduction command is provided as a convenience.
%if you want special chapter formatting, you'll probably want to avoid using it altogether

    \chapter{Introduction}
    \chapterpig{Introduction}
         %\addcontentsline{toc}{chapter}{Introduction}
	%\chaptermark{Introduction}
	%\markboth{Introduction}{Introduction}
	% The three lines above are to make sure that the headers are right, that the intro gets included in the table of contents, and that it doesn't get numbered 1 so that chapter one is 1.
	
	Welcome to the \LaTeX\ thesis template. If you've never used \TeX\ or \LaTeX\ before, you'll have an initial learning period to go through, but the results of a nicely formatted thesis are worth it for more than the aesthetic benefit: markup like \LaTeX\ is more consistent than the output of a word processor, much less prone to corruption or crashing and the resulting file is smaller than a Word file. While you may have never had problems using Word in the past, your thesis is going to be about twice as large and complex as anything you've written before, taxing Word's capabilities. If you're still on the fence about  using \LaTeX, read the Introduction to LaTeX on the CUS site as well as skim the following template and give it a few weeks. Pretty soon all the markup gibberish will become second nature.


\section{Why use it?}
	
\LaTeX\ does a great job of formatting tables and paragraphs. Its line-breaking algorithm was the subject of a PhD.\thinspace thesis. It does a fine job of automatically inserting ligatures, and to top it all off it is the only way to typeset good-looking mathematics.

\section{Who should use it?}

Anyone who needs to use math, tables, a lot of figures, complex cross-references, IPA or who just cares about the final appearance of their document should use \LaTeX. At Reed, math majors are required to use it, most physics majors will want to use it, and many other science majors may want it also.

%\chapter{A story about things}
%\chapterpig{A Story About Things}
%%\documentclass[11pt]{article}
%\usepackage[margin=1.25in]{geometry}
%\geometry{letterpaper}
%\usepackage{graphicx}
%%\usepackage{tipa}
%%\usepackage{exaccent}
%%\usepackage{txfonts}
%%\usepackage{pxfonts}
%\usepackage{enumitem}
%%\usepackage{amssymb}
%\usepackage{amsmath}
%\usepackage{epstopdf}
%\usepackage{setspace}
%\usepackage{natbib}
%\setcitestyle{aysep={}}
%\synctex=1
%
%\DeclareSymbolFont{symbolsC}{U}{txsyc}{m}{n}
%\DeclareMathSymbol{\strictif}{\mathrel}{symbolsC}{74}
%\DeclareMathSymbol{\boxright}{\mathrel}{symbolsC}{128}
%
%\newenvironment{squote}{\begin{quote}\begin{singlespace}}{\end{singlespace}\end{quote}}
%
%\newcommand{\stager}[4]%
%{%
%	\begin{spacing}{1}%
%	\vspace{0pt}
%		\begin{description}[style=nextline, noitemsep, parsep=0pt, topsep=0pt, leftmargin=15mm, itemindent=-10mm, font=\mdseries]
%			\item[\textsc{#1} \emph{#2}] #3
%			\item[]%
%			\begin{flushright}#4\end{flushright}
%		\end{description}%
%	\end{spacing}%
%}
%
%\newcommand{\stage}[3]%
%{%
%	\begin{spacing}{1}%
%	\vspace{0pt}
%		\begin{description}[style=nextline, parsep=0pt, leftmargin=15mm, itemindent=-10mm, font=\mdseries]
%			\item[\textsc{#1} \emph{#2}] #3
%		\end{description}%
%	\end{spacing}%
%}
%
%\title{Alien Explorers and Conceptual Schemes}
%\author{Alexander A. Dunn}
%\begin{document}
%\maketitle
%\begin{spacing}{1}

\chapter{A story about things}
\chapterpig{A Story About Things}

\section{Intuitions}
Gareth Evans once claimed that ``with pliant enough intuitions you can swallow anything in philosophy''~(\citeyear[192]{evans1973}). That's an hypothesis that demands testing. So I will claim that, if humans did not exist, neither would dogs, cats, mountains, forests, lakes, rivers, pigs, apples, shrubs, stars, and pretty much everything else. But before I try to get you to swallow this, I'm going to tell a story.

\subsection{Alien explorers meet a metaphysician}
{\em (Our protagonist sits in a chair in front of the fire.)} \\

\stager{Protagonist}{}{Suppose some years in the future we explore a distant planet very unlike our own. There seems to be organic life, but whether it is intelligent or not is hard to say. There seem to be artificial modifications to the environment, but again it's difficult to say what is natural and what isn't. But our intrepid scientists refuse to be baffled by this strange world. After some years of research they manage to pin down the physical processes and produce a model of the planet with really quite good predictive capabilities.}{{\em (A child's head droops.)}}

\stager{Protagonist}{(continuing)}{We've classified some phenomena as organisms with their set of biological underpinnings, and we've classified others as natural though non-living processes, such as weather patterns and geological change. There might be some things we've overlooked, but it doesn't look like there are going to be many more surprises. So I think it's safe to say that we've recognized most of what's on this planet, don't you?}{{\em (Vague nods. A sleeping \textsc{metaphysician} stirs briefly.)}}

\stager{Protagonist}{}{ Now I'm afraid I must admit that I haven't been entirely honest with you. The explorers in this story are not us, but an intelligent alien species. The planet is not distant at all; they are exploring our own Earth. And yet I think we must agree that we were correct---}{{\em (The metaphysician awakes with a start.)}}

\stage{Metaphysician}{(rising angrily)}{Now look here! These fool aliens don't recognize their own ignorance, let alone half of what exists on Earth! They forgot about adzes and Axminsters, boats and books, catalogs and cups, doors and dumbwaiters, earrings and elegance! Their ontology would fit in a knapsack!}

And so the question is asked to you, readers: are the things that went unseen by the aliens {\em really there?}

%\bibliography{everything}
%\bibliographystyle{ChicagoReedweb}
%
%\end{spacing}
%\end{document}


\chapter{Denying the ordinary}
\label{deny}
\chapterpig{Denying the Ordinary}
\documentclass[11pt]{article}
\usepackage{standalone} \newif\ifstandlone \standalonetrue
\usepackage[left=1.75in, right=1.75in, top=1.25in, bottom=1.25in]{geometry}
\geometry{letterpaper}
\usepackage{graphicx}
\usepackage{enumitem}
%\usepackage{amssymb}
\usepackage{amsmath}
\usepackage{verbatim}
\usepackage{epstopdf}
\usepackage{setspace}
\usepackage{natbib}
\setcitestyle{aysep={}}
\usepackage%[colorlinks=true, citecolor=blue, linkcolor=black]%
{hyperref}

\synctex=1

\DeclareSymbolFont{symbolsC}{U}{txsyc}{m}{n}
\DeclareMathSymbol{\strictif}{\mathrel}{symbolsC}{74}
\DeclareMathSymbol{\boxright}{\mathrel}{symbolsC}{128}

\newcommand{\stager}[4]%
{%
	\begin{spacing}{1}%
	\vspace{0pt}
		\begin{description}[style=nextline, noitemsep,
                    parsep=0pt, topsep=0pt, leftmargin=15mm,
                    itemindent=-10mm, font=\mdseries]
			\item[\textsc{#1} \emph{#2}] #3
			\item[]%
			\begin{flushright}#4\end{flushright}
		\end{description}%
	\end{spacing}%
}

\newcommand{\stage}[3]%
{%
	\begin{spacing}{1}%
	\vspace{0pt}
		\begin{description}[style=nextline, parsep=0pt,
                    leftmargin=15mm, itemindent=-10mm, font=\mdseries]
			\item[\textsc{#1} \emph{#2}] #3
		\end{description}%
	\end{spacing}%
}

\newenvironment{squote}{%
	\begin{spacing}{1}
	\begin{list}{}{%
	\setlength{\labelwidth}{0pt}%
	\rightmargin\leftmargin%
	}
	%\begin{singlespace}%
	\item\relax
	}{%
	%\end{singlespace}%
	\end{list}%
	\end{spacing}
	}

\newenvironment{inq}{\vspace{0pt}%
	\begin{list}{}%
	{\setlength\labelwidth{0pt}%
	\setlength\leftmargin{2.5\oddsidemargin}%
	\setlength\rightmargin{\leftmargin}}
	\begin{spacing}{1}
	\item[]%
	}{
	\end{spacing}
	\end{list}
	\vspace{10pt}
	%\noindent%
	}

\title{Why do you think that?}
\author{Alexander A. Dunn}
\begin{document}
\ifstandalone
\maketitle
\begin{spacing}{1.5}
\fi
\label{stroud}

% \begin{inq}
% The philosophical quest must start somewhere. It needs a set of
% beliefs about what the world is like. Without some attitudes,
% perceptions, beliefs, or theories to start with, it would have
% nothing to reflect on.~\citep[16]{stroud2000a}
%\end{inq}

\noindent Section \ref{intro-beliefs} will motivate my claim that a
nihilistic metaphysical thesis should be accompanied by an explanation
of why people nonetheless believe that there are chairs and other
ordinary things.  I will then look at the specific theses of Peter van
Inwagen and Trenton Merricks.  After assessing the ability of each to
explain our beliefs, I will myself try to explain why they think that
it is not obviously true that there are chairs.  Van Inwagen and
Merricks claim that it is not obviously true because they overestimate
what is required for ``there are chairs'' to be true.

\section{Explaining the beliefs of others}
\label{intro-beliefs}
\noindent Many people have false beliefs.  They believe things that
misrepresent (in some sense) how the world is.  For example, some
people believe that ghosts exist.  These people each hold a false
belief, for it is not true that ghosts exist.  There are no ghosts in
the world.  Despite this fact---that there are no ghosts---some people
believe that there are.  Why?  What explanation can we give as to why
someone believes a falsehood like this?

In explaining why someone holds a belief, we appeal to {\em reasons}.
Even people who hold beliefs that we may consider irrational (like the
belief that there are ghosts) have reasons for holding these beliefs.
They may not be good reasons; someone might believe that there are
ghosts because her older sister told her that there are ghosts, or
read ghost stories as a child and took them seriously.  Someone who
believes in ghosts might even think that she has {\em seen} a ghost.
This too would be a false belief; there are no ghosts, so nobody can
have seen one.  But here too there will be a reason why she holds this
false belief.  Perhaps she saw a strange play of light on a distant
wall, or the reflection of the moon filtered through an attic window.
What she actually saw was perhaps one of these things, but she somehow
took what she saw to be a ghost.  Probably she already believed that
there were ghosts, and so, when confronted with a deceptive or
confusing sight, was predisposed to form the mistaken belief that she
was seeing a ghost.

Here and in what follows, when I say that there is a reason why
someone believes something, I mean that there is some {\em cause} that
produced the belief.  Above, I told a causal story about why the
person who believes that she saw a ghost holds that belief.  She had
been told that there were ghosts by a person who she thought
trustworthy, so she came to believe that there are ghosts.  Holding
that believe caused her to be predisposed to interpret unusual
phenomena as ghosts.  This disposition caused her to believe that she
was seeing a ghost when she saw a reflection of the moon.

My use of `reason', therefore, should be taken in this causal sense.
There are other ways that people use the word `reason'.  If someone
asks ``What reason do you have to believe that $((P \rightarrow Q )
\wedge P) \rightarrow Q$?''  I might reply that it is a theorem of
first-order logic.  Here I am not telling a causal story.  I am rather
{\em justifying} my belief that $((P \rightarrow Q ) \wedge P)
\rightarrow Q$.  But in this case it is perfectly correct to say that
I am giving a reason as to why I hold a belief.  It is just not a {\em
  causal} reason.  A causal reason would be something like the
following: $((P \rightarrow Q ) \wedge P) \rightarrow Q$ is true, and
I have learned the rules of logic, and so I can prove that $((P
\rightarrow Q ) \wedge P) \rightarrow Q$.

(Another example: suppose someone falsely believes that $((P
\rightarrow Q ) \wedge Q) \rightarrow P$ is a theorem of first-order
logic.  There will be some (causal) reason why they hold this belief;
probably they attempted to deduce it from no premises and therefore
believe that they succeeded.  There will, in turn, be a reason why
they hold {\em this} false belief; maybe they were not concentrating
on the proof steps, or they forgot certain rules of deduction.)

An example involving an apparently obviously true belief might help
clarify the distinction between causal reason and justifying reasons.
If someone were to ask me why I believe that the sky is blue during
the day, my immediate answer would probably be ``well, because it
is!''  There's not much else I can say to {\em justify} my belief.
But this not a {\em causal} explanation.  The fact that something is
true (the sky {\em is} blue) does not cause me to believe it.
Otherwise I would believe every truth, and I do not.  There are
doubtless many truths that I do not believe.  There must therefore be
another (causal) reason why I believe that the sky is blue, other than
the fact that the sky is blue.

I believe that the sky is blue because, first, it is blue, and second,
I have {\em seen} that it is blue.  My vision is generally reliable
(or at least seems to be), so the fact that my eyes `tell' me
something is good reason to believe it.  The same is true of my other
senses: they are generally reliable, so the fact that they `tell' me
something is a good reason to believe it.  It does not follow that it
is {\em true}, however (though no doubt we believe that it is true);
our eyes can be deceived.

A skeptic might claim that we cannot rule out the possibility that we
are {\em constantly} deceived.  They attempt to undermine the
reliability of our senses.  I will not be addressing such arguments.
Rather, in what follows I will examine arguments that deny (or appear
to deny) that many of our beliefs about `ordinary things' are true.
The philosophers making these denials do not claim that our eyes are
unreliable sources of information.  Their arguments are metaphysical
rather than epistemic; they deny that certain objects are {\em
  possible}.  

For example, Peter Unger claims that chairs do not exist.  He relies
on a number of metaphysical arguments to motivate this claim.  If he
is right, however, then it seems to follow from this that beliefs like
the following are necessarily false:

\begin{itemize}
  \item Some chairs are made of wood.
  \item There have existed many chairs which no longer exist.
  \item There are chairs.
\end{itemize}

I, however, believe that all of these propositions are true.  Even if
Unger is right, and they are all false, it still seems to be the case
that there are reasons why I believe these propositions.

If someone were to ask {\em me} why I believe that there are chairs, I
would probably answer ``because there are, and I have seen them (and
sat upon them)!''  It seems obviously true, just like the fact that
the sky is blue.  I have seen lots of chairs, and I can't have been
confused or deceived {\em every} time.

Nonetheless, Peter Unger and other philosophers (who we will call
`nihilists' or `eliminativists') say that I am mistaken.  They claim
that I have not in fact seen lots of chairs, though I may believe that
I have.  There are several different arguments by which nihilists seek
to establish that chairs (and other `ordinary things') do not exist;
we will examine some of these arguments below.  Having made these
arguments, however, the nihilists must reject our causal explanation
of why we believe that there are chairs.  Our explanation was that
there are chairs and we can see them.  But the nihilist denies that
there are chairs, and so should admit that, if we believe that there
are chairs, there must be a different explanation as to why we hold
this belief.

\subsection{Why bother?}
A metaphysical thesis that involves denying the existence of ordinary
things like chairs entails that the simplest explanation of why we
believe that there are chairs is incorrect.  I believe that such a
thesis should therefore be supplemented with a new explanation.  This
new explanation would identify the reasons why we would believe that
there are chairs if there are in fact none.  But why should I demand
this of a metaphysical theory?  Is it a reasonable request?

As an analogy, consider color.  Most people believe that things are
colored.  A simple causal story about why people believe that things
are colored might go like this:  things are colored, and people see
that things are colored.  

But imagine a philosopher who holds some version of {\em physicalism}
and claims that the world as described by physics is all that there
is.  This view is often thought to have the consequence that things
aren't actually colored.  In the `vocabulary of physics', things might
be described in such a way that the things color gets somehow left
out.  We may be unable to determine from the `physical description'
what color the object is.  The colors of objects are not included in
this philosopher's description of the world.

If the philosopher admits that people believe that things are colored,
she cannot explain this using the same story that I used above.  I
said that people believe that things are colored and that they see
that things are colored.  But the physicalist maintains that things
are not colored.  {\em If} she admits that people believe that things
are colored, then she needs a different explanation as to why people
believe that things are colored.

She might, however, deny that people believe that things are colored.
(This would be a rather bold claim.)  She could say that the notion of
color is entirely illusory.  If we believe that we see colors, she may
tell us we are wrong.  When we think that something is colored, we are
mistaken.  If we think that an apple is red, we have a false belief.
She might claim that color does not pose a difficulty for her view,
because humans do not experience `color'.

This, as I said, is a rather bold claim.  It seems simply true that we
see colors and that the apple looks red.  If a philosopher were to
deny these things, I would have difficulty understanding what she
meant.  This is not to say she is {\em wrong}; I have no argument
proving that her thesis is false.  But the claim that humans do not
experience color seems bizarre and unmotivated.  Fortunately I do not
know of anyone who actually holds this view.

Our imagined philosopher might make a less bold claim.  She might
instead claim that color is one of those things that are `subjective'
rather than `objective' or `absolute' features of the world.  A
subjective feature of the world is a feature that is present only
because we (or some other being) exists to experience it:

\begin{squote}
Whatever is due only to us and to our own ways of responding to and
interacting with the world does not reflect or correspond to anything
present in the world as it is independently of us.  The aim of an
``absolute'' conception, then, is to form a description of the way the
world is, not just independently of its being believed to be that way,
but independently, too, of all the ways in which it happens to present
itself to us human beings from our particular standpoint within
it\,\ldots\,[So we] form some conception of that independent reality
and come to understand parts or aspects of our original conception of
the world as not representing it as it is.  If we see them as products
or reflections of something peculiar to human experience or to the
human perspective on the universe, we assign them a merely
``subjective'' or dependent status and eliminate them from our
conception of the world as it is independently of
us~\citep[30--31]{stroud2000a}.
\end{squote}

A philosopher who adheres to this distinction might claim that our
conception of the world as colored does not represent the world as it
is independently of us.  Colors, she would claim, are not objectively
real.  She allows, however, that they are subjectively real.  She
admits that people do see colors.  Because of our color vision, we
come to believe that the things we see are colored.  A philosopher who
denies the objective reality of color does not thereby ``deny that we
perceive many different colours or that we believe physical objects to
be coloured'' \citep[145]{stroud2000a}.  What this philosopher claims
is something to the effect that, while we see things {\em as} colored,
things are not {\em themselves} colored.  The red color of a tomato,
on this view, obtains only in our perception of the tomato; there is
nothing {\em in} the tomato that is the redness (other species may not
see the redness when they see the tomato).

The philosopher who is denying the objective reality of color does
``recognize the presence in the world of perceptions of and beliefs
about the colours of things'' \citep[199]{stroud2000a}.  The challenge
then is for her to explain why we do have these perceptions and
beliefs.  If she believes that only the world of physics is
objectively real, she must explain why we hold these beliefs, and she
must give this explanation in such a way that commits her only to the
existence of physical things.  If she claims that the world as
described by physics is the only world there is, then she must explain
why, in a world that contains only physical things, we come to believe
that there are colors and colored objects.

Again: if our physicalist philosopher admits that people believe that
they experience color, and admits that people believe that things are
colored, {\em then} she commits herself to explaining why we form
beliefs that are, according to her, false.  Here is the analogy with
metaphysicians like Peter Unger: {\em if} they admit that many of us
believe that there are chairs and other ordinary objects, then they
commit themselves to explaining why we form these false beliefs.  For
as we have seen, even false beliefs are generally held for a reason.

\subsection{Paraphrasing beliefs}
\label{paraphrase}
Peter Unger denies that chairs exist, and claims that, if we believe
that chairs exist, we are mistaken.  His task will be to explain why
we form these false beliefs.  But not all nihilistic philosophers deny
that we are, in fact, mistaken.  They deny that there are any chairs,
but maintain that beliefs like the following might still be true:

\begin{itemize}
  \item There are two chairs in the next room.
  \item I own some very nice 17th-century chairs.
  \item Some chairs are heavier than some tables.
\end{itemize}

Peter van Inwagen is one of these philosophers.  He denies the
existence of tables, chairs, apples, and all other inanimate composite
objects (van Inwagen's technical definition of `composite' will be
discussed below in section~\ref{scq}).  He takes pains to make clear
that his denial of these things is not a relegation of tables and
chairs to `subjective reality'.  He wants to claim that such things do
not exist in any way, subjective or objective:
\begin{squote}
I want to do what I can to disown a certain apparently almost
irresistible characterization of my view, or of that part of my view
that pertains to inanimate objects.  Many philosophers, in
conversation and correspondence, have insisted, despite repeated
protests on my part, on describing my position in words like these:
``Van Inwagen says that tables are not real''; ``\ldots\,not true
objects''; ``\ldots\,not actually {\em things}''; ``\ldots\,not
substances''; ``\ldots\,not unified wholes''; ``\ldots\,nothing more
than collections of particles.''  These are words that darken counsel.
They are, in fact, perfectly meaningless.  My position vis-\`{a}-vis
tables and other inanimate objects is simply that there {\em are}
none~(\citeyear[99]{inwagen1995}).
\end{squote}

Van Inwagen asserts, quite seriously, that ``there are no tables or
chairs or any other visible objects except living organisms''
(\citeyear[1]{inwagen1995}).  This is a somewhat more bold claim than
that of the physicalist's with regard to color.  She at least granted
that we do see colors, even if we don't actually see things that are
(objectively) colored.  If, as van Inwagen claims, the only {\em
  visible} objects are living organisms, then we certainly can't {\em
  see} chairs at all.  But just as our physicalist could not claim
that we don't believe that there are colors, van Inwagen cannot deny
that we at least {\em believe} that there are chairs.

Van Inwagen does not attempt to deny that we hold beliefs like those
listed above.  He admits that many of us hold beliefs that we would
express as ``there are two chairs in the next room'' or ``I bought a
new chair today''.  Indeed, he admits that such beliefs are often {\em
  true}: ``when people say things in the ordinary business of life by
uttering sentences that start `There are chairs\,\ldots ' or `There
are stars\,\ldots ', they very often say things that are literally
true''~(\citeyear[102]{inwagen1995}).  Van Inwagen, when denying that
we have beliefs about chairs, appears to maintain that the beliefs
that we (erroneously) take to be about chairs are not, in fact,
beliefs about chairs.  If a belief expressed as ``that is a fine
chair'' was actually about a chair, then it could only be true if
there was at least one chair (a fine one).  But van Inwagen denies
that there is at least one chair, but nonetheless says that such a
belief might be true.  He accordingly recognizes the need to explain
what our beliefs really are about.  If he explains what the {\em
  content} of our beliefs is, then he will also be able to explain
{\em why} we hold such beliefs.

\ifstandalone
\end{spacing}
\bibliography{everything}
\bibliographystyle{ChicagoReedweb}
\fi
\end{document}

	
\chapter{Brute facts}
\label{brute}
\chapterpig{Brute Facts}
\documentclass[11pt]{article}
\usepackage{standalone} \newif\ifstandlone \standalonetrue
\usepackage[left=1.75in, right=1.75in, top=1.25in, bottom=1.25in]{geometry}
\geometry{letterpaper}
\usepackage{graphicx}
\usepackage{enumitem}
%\usepackage{amssymb}
\usepackage{amsmath}
\usepackage{epstopdf}
\usepackage{setspace}
\usepackage{natbib}
\setcitestyle{aysep={}}
\usepackage{hyperref}
		
\synctex=1

\DeclareSymbolFont{symbolsC}{U}{txsyc}{m}{n}
\DeclareMathSymbol{\strictif}{\mathrel}{symbolsC}{74}
\DeclareMathSymbol{\boxright}{\mathrel}{symbolsC}{128}

\newenvironment{squote}{%
	\begin{quote}\begin{singlespace}%
	}{%
	\end{singlespace}\end{quote}}

\newcommand{\stager}[4]%
{%
	\begin{spacing}{1}%
	\vspace{0pt}
		\begin{description}[style=nextline, noitemsep,
                    parsep=0pt, topsep=0pt, leftmargin=15mm,
                    itemindent=-10mm, font=\mdseries]
			\item[\textsc{#1} \emph{#2}] #3
			\item[]%
			\begin{flushright}#4\end{flushright}
		\end{description}%
	\end{spacing}%
}

\newcommand{\stage}[3]%
{%
	\begin{spacing}{1}%
	\vspace{0pt}
		\begin{description}[style=nextline, parsep=0pt,
                    leftmargin=15mm, itemindent=-10mm, font=\mdseries]
			\item[\textsc{#1} \emph{#2}] #3
		\end{description}%
	\end{spacing}%
}

\newenvironment{inq}{\vspace{0pt}%
	\begin{list}{}%
	{\setlength\labelwidth{0pt}%
	\setlength\leftmargin{2.5\oddsidemargin}%
	\setlength\rightmargin{\leftmargin}}
	\begin{spacing}{1}
	\item[]%
	}{
	\end{spacing}
	\end{list}
	\vspace{10pt}
	%\noindent%
	}

\title{Unger's arguments}
\author{Alex Dunn}
\begin{document}
\ifstandalone
\maketitle
\begin{spacing}{1.5}
\fi

Peter Unger has presented several arguments that threaten the kind of
universalism I sketched in section \ref{universalism}.  The versions
of the sorites paradox that he presents make trouble for all vague
concepts, including those that Merricks relies on for his version of
nihilism.  Unger's `problem of the many', on the other hand, poses a
problem specifically for versions of universalism, including my own.

\section{Incoherence and pluralities}
\label{unger}
Like Peter van Inwagen and Trenton Merricks, Peter Unger has denied
the existence of all `ordinary things'---such things as ``tables and
chairs and spears\,\ldots swizzle sticks and
sousaphones\,\ldots\,stones and rocks and twigs, and also tumbleweeds
and fingernails'' (\citeyear[117]{unger1979}).  Merricks has a
different motivation for his nihilism than does van Inwagen, and Unger
has a different motivation again.  In fact, he has two different
motivations; one is the sorites paradox and the other is the problem
of the many.  

Unger claims that the sorites paradox shows that terms for ordinary
things, like `chair', are {\em incoherent}.  He claims that incoherent
terms cannot apply to anything in the world; therefore he concludes
that there are no chairs (or any other ordinary thing).

Unger's presentation of the problem of the many is aimed to trouble
the concept of `chair' in a different way.  The conclusion of that
argument is not that `chair' is necessarily incoherent.  Rather, the
conclusion is a disjunction: {\em either} there are no chairs, {\em
  or} there are a plurality (possibly an infinity) of chairs where we
would normally take there to be only one.

We will examine these two arguments in turn.

\section{Sorites paradoxes}
\label{sorites}
A typical instance of the sorites paradox begins by having us imagine
some ordinary object; let us use a heap of sand.  Now suppose we
remove a single grain of sand.  If we were inclined to believe that
the initial quantity of sand did in fact constitute a heap, then after
the removal of a single grain, we should presumably still have a heap
(albeit a slightly smaller one).  It seems very implausible to think
that one grain of sand more or less could {\em ever} make a difference
as to whether something is or is not a heap.

But having conceded (a) that there is a heap and (b) that the removal
of a single grain cannot make the difference as to whether a quantity
of sand is a heap, we have unwittingly put our foot in it.  For if the
removal of a single grain {\em never} transforms a heap into a
non-heap, then by repeatedly removing one grain after another, we will
eventually find ourselves with a heap that consists of no sand at
all.  But it seems absurd to suppose that there could be a heap of
sand that is composed of no sand---indeed, of nothing whatsoever.

This is the sorites paradox.  While a heap is a useful example,
because it is so ill-defined, similar problems appear to afflict all
ordinary things.  Unger illustrates the difficulty for stones:

\begin{squote}
Consider a stone, consisting of a certain finite number of atoms.  If
we or some physical process should remove one atom, without
replacement, then there is left that number minus one, presumably
constituting a stone still\,\ldots after another atom is removed,
there is that original number minus two; so far, so good.  But after
that certain number has been removed, in similar stepwise fashion,
there are no atoms at all in the situation, while we must still be
supposing that there is a stone there.  But as we have already agreed,
if there is a stone present, then there must be some atoms\,\ldots I
suggest that any adequate response to this contradiction must
include\,\ldots the denial of the existence of even a single
stone.~\citep[121--122]{unger1979}
\end{squote}
Unger understands this dilemma to apply across the board, and
correspondingly argues that we should deny the existence of even a
single ordinary thing.

\subsection{The sorites paradox in relation to Merricks' nihilism}
\label{sorites-m}
The purpose of Unger's argument is to show that terms that are
susceptible to the sorites paradox are incoherent and cannot apply to
anything in the world.  Unger claims that terms like `chair' therefore
cannot apply to anything in the world---from which it follows that
there are no chairs.

Trenton Merricks' version of nihilism (section \ref{merricks}) denies
that there are chairs.  In this, Merricks is in agreement with Unger.
But unlike Unger, Merricks maintains that beliefs like ``there are
chairs'' are {\em justified} and {\em nearly as good as true}.  He
claims that such beliefs are nearly as good as true if they are caused
by simples arranged chairwise.  Merricks does not believe that there
are chairs, so ``there are chairs'' is, strictly speaking, false.  But
Merricks believes that there are simples arranged chairwise, and the
presence of such arrangements cause and justify false beliefs such as
``there are chairs''.

We have seen, however, that the sorites paradox threatens the
coherency of concepts like `chair'.  If we suppose that a given
collection of atoms composes a chair, we can remove them one by one
and at no point feel justified in saying that the chair suddenly
ceases to exist.  Now suppose that a given arrangement of simples is
arranged chairwise.  Remove one.  Is the arrangement still arranged
chairwise?  Just as in the case of the chair, it seems bizarre to
think that a single simple (or atom) can make a difference as to
whether `chairwise' applies.  But now remove another atom\,\ldots

If Unger's argument shows that the concept of `chair' is incoherent,
then it seems that the same argument shows the concept of `chairwise'
to be incoherent.  If so, then `chairwise' can apply to nothing in the
world.  There can be no chairwise arrangements of simples.  If there
are no chairwise arrangements of simples, then our belief that there
are chairs cannot be nearly as good as true.  If someone believes that
there is a ghost, that belief, according to Merricks, is not nearly as
good as true because there are no ghostwise arrangements of simples to
cause or justify the belief that there is a ghost.  If `chairwise' is
incoherent, then there are no chairwise arrangements of simples to
cause or justify the belief that there are chairs.  The belief that
there are chairs is (if Unger is right) simply false, just as is the
belief that there are ghosts.

Merricks cannot allow this conclusion.  But the sorites paradox can be
used to show that {\em any} vague term is incoherent.  Much of our
language is vague, but we are not therefore tempted to conclude that
our speech is rarely (if ever) coherent.  The problem of the sorites
paradox is a very general problem that requires a general theory of
vagueness.  Most metaphysical theories are threatened by the sorites
paradox; Merricks is not a special case.

With that in mind, it should not be surprising that Merricks does not
have full answer.  A full answer to the sorites paradox would be a
theory of vagueness.  Merricks is not trying to establish a theory of
vagueness, but it attempting to motivate a metaphysical thesis about
ordinary things.  That said, he does have a {\em partial} answer.  He
points out that while the sorites paradox does threaten the concept of
`chairwise', it does so in a less troubling way than the way in which
it threatens `chair'.  

How is the sorites paradox ``less troubling'' for chairwise
arrangements than for chairs?  Very roughly, it is because accepting
the vagueness of `chair' can lead us to {\em metaphysical} vagueness,
while accepting the vagueness of `chairwise' can only lead us to {\em
  linguistic} vagueness.  And linguistic vagueness is generally
considered to be less troubling than metaphysical vagueness.

We can get a sense of how this is so by imagining, as Merricks does,
the sorites paradox being played out as a series of questions.  Let us
suppose that there is a chair.  We then ask ourselves (Merricks asks
God), ``is `there is a chair' true?''  Presumably we will answer
``yes''.  Then we remove one atom (or simple, or other thing) from the
chair.  ``Is `there is a chair' true?''

If we agree that there is no single atom whose removal would destroy
the chair, then we must accept that at some point it becomes
indeterminate whether ``there is a chair'' is true.  If it is
indeterminate whether ``there is a chair'' is true, then it is
indeterminate whether there is a chair.  The idea that it could be
indeterminate whether something exists is taken by many to be
problematic (\textbf{CITE}).

We can compare this case with that of the things arranged chairwise.
Let us suppose that there are things arranged chairwise.  We ask ``is
`there things arranged chairwise' true?'' and answer ``yes''.  Then we
remove a thing and ask ``is `there things arranged chairwise' true?''
As in the chair case, if we are unwilling to allow that the removal of
a single thing could take us from ``yes'' to ``no'', then we must
admit that at some point it is indeterminate whether or not ``there
are things arranged chairwise'' is true.  It would then be
indeterminate whether there are things arranged chairwise.  {\em But
  this does have the result that it is indeterminate whether something
  exists.}  It may be perfectly determinate that there are the things
there are; all that is indeterminate is whether the things (which
there determinately are) are arranged in a certain way.  This kind of
indeterminacy seems less troubling than indeterminacy as to whether
something (a chair or anything else) exists.

Merricks therefore sees the problem posed by the sorites paradox as
less threatening to his chairwise arrangements than to chairs.  He
does not attempt to provide a solution, for that would require solving
the problem of linguistic vagueness.  But he finds the problem of
linguistic vagueness less troubling than the problem of metaphysical
vagueness, and he shows that the latter does not threaten his version
of nihilism.

\subsection{So what's the problem?}
\label{sorites-3}
We find ourselves wanting to hold three theses, which appear mutually
inconsistent:

\begin{enumerate}
  \item There is at least one chair (stone, cloud).
  \item If a chair (stone, cloud) exists, it must be made up of
    matter.
  \item If a chair (stone, cloud), exists, the removal of a single
    molecule (or otherwise insignificant quantity of matter) from it
    cannot destroy it or cause it to cease to exist.
\end{enumerate}

We seem to be clearly caught in a paradox; the only question is where
we have gone wrong.

But have we, in fact, gone wrong?  Peter Unger thinks that we are
right on target:

\begin{squote}
While Eubulides' contribution has often been labeled `the sorites
paradox', there is nothing here which is a paradox in any
philosophically important sense\,\ldots Accepting our negative
conclusions here does not mean important logical trouble for us; we
only think we have troubles while we refuse to admit their validity
(\citeyear[145]{unger1979}).
\end{squote}

Our situation is only paradoxical, says Unger, while we unreflectingly
cling to the first thesis.  If, however, we come to see that there are
no chairs (stones, clouds), then we happily escape paradox: if there
are no chairs (stones, clouds) to begin with, we do not have to worry
about what the addition or removal of small amounts of matter would do
to them; nor do we need concern ourselves with what they would be made
of.

But things are not quite so simple.  First, adding to the
implausibility of Unger's view, he must deny that our use of ordinary
terms like `chair' (`stone', `cloud') follow any sort of pattern or
display any competence at all.  Second, even if we manage to swallow
that consequence, Unger has no explanation as to why we believe that
there are chairs (stones, clouds).

\subsection{Competence and correctness}
\label{correct}
Setting aside whether or not expressions of propositions like ``that's
a chair'' are ever \emph{true}, it seems right to say that there are
at least correct and incorrect uses of the terms.  For a word like
`chair' (`stone', `cloud') we generally do not say that a child has
learned how to use it until she is capable of deploying it in certain
ways.  We admit that she understands what `chair' (`stone', `cloud')
means or what a chair (stone, cloud) is when she displays a certain
competence with the term.  If instead of using `chair' to refer to
chairs she used it to refer to dogs or people, we would say that she
is confused and attempt to correct her use.

But Unger maintains that this is all an illusion, and that there is no
such thing as the correct or incorrect use of a term like `chair'
(`stone', `cloud'):

\begin{squote}
Concerning words and kinds, now, we might say this.  First, we might
say that it is in connection with \emph{semantics} that our reasonings have
what are their most obvious implications and, second, that their most
obvious semantic implications concern certain \emph{sortal nouns}, namely,
those which purport to denote ordinary things.  Thus, it appears quite
obvious to us now that there will be no application to things for such
nouns as `stone' and `rock', `twig' and `log', `planet' and `sun',
`mountain' and `lake', `sweater' and `cardigan', `telescope' and
`microscope', and so on, and so forth.  Simple positive sentences
containing these terms will never, given their current meanings,
express anything true, correct, accurate, etc., or even anything which
is anywhere close to being any of those things
(\citeyear[148]{unger1979}).
\end{squote}

This seems simply bizarre.  On what grounds, then, do parents correct
their children with respect to their use of ordinary terms?  Are they
compelled by some irrational force to consider certain utterances
correct and others incorrect?  One may question whether or not we use
ordinary term entirely consistently, but it seems simply false to say that,
necessarily, we {\em never} use (or have used, or will use) ordinary
terms in correct, as opposed to incorrect, ways.  

The fact that Unger's position means that terms like `chair' are never
used correctly gives us reason to think his argument goes awry
somewhere.  However, attempting to identify the issue with his
argument and proposing a solution would be tantamount to attempting to
solve the problem of vagueness.  That is not something I will attempt,
at least not until we have examined the problem of the many.

\section{The problem of the many}
\label{many}
The `problem of the many', as Unger terms this second difficulty for
ordinary things, follows a similar line of reasoning as that of the
sorites paradox.  If we consider an ordinary thing---a cloud, for
instance---it is natural to think that it is made up of molecules.
There is probably then a set of molecules, the members of which make
up the cloud.  Call that set $S$.  Now consider $S_1$.  This is a set
of molecules that includes all of the members of $S$ as well as one
additional molecule.  Do the members of $S_1$ make up a cloud?  Surely
they are just as well suited to do so.  Now consider $S_2$\,\ldots

Because these numerous 'candidates' are equally (or nearly equally)
well suited to be clouds, we seem forced to conclude that there are
either many clouds where we supposed there to be one, or rather no
clouds at all:

\begin{squote}
No matter where we start, the complex first chosen has nothing
objectively in its favor to make it a better candidate for cloudhood
than so many of its overlappers are.  Putting the matter somewhat
personally, each one's claim to be a cloud is just as good, no better
and no worse, than each of the many others.  And, by all odds, each
complex has \emph{at least} as good a claim as any still further real
entity in the situation.  So, either \emph{all} of \emph{them} make it
or else \emph{nothing} does; in this real situation, either there are
many clouds or else there really are no clouds at all
\citep[415]{unger1980a}.
\end{squote}

The problem of the many can also arise by considering the {\em
  boundary} of a given cloud.  It is natural to suppose that a cloud
has a determinate boundary.  But if we look at the edge of the cloud,
where we suppose the boundary to be, ``we may find, side by side, or
themselves overlapping, a great many potential boundaries for
clouds\,\ldots if our alleged typical item {[}the cloud{]} is indeed
a typical cloud, then many of these candidates, millions at least, do
not fail to be clouds altogether but are clouds of some
sort'' \citep[420--421]{unger1980a}.

The pattern of argumentation is the same for both approaches to the
problem of the many.  For a given cloud, a certain set of members or a
certain boundary is supposed, and it is argued that a set or boundary
that differs minimally from the original must also make up or bound a
cloud.  The new set or boundary does not appear to differ from the
original in any relevant way; there seems no principled reason to deny
that if the first set's members make up a cloud, the second set's
members do too.  And since there are a great deal of very similar sets
and boundaries, we find ourselves threatened with a plurality of
clouds.

And of course, Unger does not rest content with applying the problem
of the many to clouds.  All ordinary objects get the same treatment;
he concludes that either there are a great many of them, or there are
none at all.  He claims, predictably, that the latter disjunct is
preferable.

\subsection{Is the problem of the many a problem for Merricks?}
\label{many-merricks}
It might seem that the problem of the many, if it makes things
difficult for chairs and other ordinary things, also causes trouble
for the chairwise arrangements upon which Merricks relies so heavily.
But there is an important disanalogy.  The problem of the many is only
a problem as long as we are unwilling to accept one of Unger's
disjuncts: that there are no chairs or that there are a plurality
where we took there to be one.  Whether or not it is part of the
meaning of `chair' that there is not an overlapping plurality of
chairs, it is simply unacceptable that this be the case.  (It is
likewise unacceptable that there be no chairs.)  But it {\em is}
acceptable, at least initially, that there be a plurality of chairwise
arrangements.  The idea that there is a plurality of different sets,
the members of which overlap and of which all are arranged chairwise,
is not particularly bizarre.  A potential difficulty for Merricks
would be in regard to his criterion of `arranged chairwise' (or
statuewise, as the case may be):

\begin{squote}
Atoms are \emph{arranged statuewise} if and only if they both have the
properties and also stand in the relations to microscopica upon which,
if statues existed, those atoms' \emph{composing a statue} would
non-trivially supervene (\citeyear[4]{merricks2001a}).
\end{squote}

If Merricks allows that there may be a plurality of statuewise
arrangements, then he is committed to the proposition that, if statues
existed, there may be pluralities of statues.  But Merricks may simply
take this as more evidence that his counterpossible conditional (``if
there were statues\,\ldots '') really is impossible.

There may be, however, a problem for Merricks with regard to the
notion of `singular thought'.  If Merricks says to me, ``those things
arranged chairwise are arranged very comfortably'', how can I know
which things he is talking about?  If there are numerous different
sets of things arranged chairwise, the chance that I am thinking of
the same things as Merricks is very low.  Are we really communicating,
then?  (Moreover, is it true that I am thinking of {\em any}
determinate set of things?  I certainly couldn't specify which
particular things I am thinking of.)

If the problem of the many is a problem for Merricks, then so much the
worse for his nihilism.  However, the problem is a problem for us as
well.

\section{Beliefs in things}
\label{u-belief}
When we examined previous versions of nihilism, we asked that their
proponents explained why, if there are no chairs, we nonetheless
believe that there are chairs.  Van Inwagen and Merricks both made a
claim to the effect that our beliefs are caused and (in some sense)
justified by arrangements of simples.  Although Merricks denies that
beliefs like ``there is a fine chair'' are strictly true, he agrees
with van Inwagen that they `get something right' in a way that beliefs
like ``there is a dancing chair'' do not.  Their explanation for our
belief that there are chairs is that they are caused (and justified)
by a `nearby' or somehow related fact---that there are simples
arranged chairwise.

It should therefore seem reasonable to demand a similar explanation
from Unger.  This explanation would be expected to take a different
form, depending on whether it accompanies the sorites paradox or the
problem of the many.  However, Unger offers no explanations.  

\subsection{Explaining our beliefs given the sorites paradox}
\label{expl-sorites}
Unger claims that our belief that there are chairs, like our beliefs
that there are other ordinary things, are not justified, `nearly as
good as true', or even coherent.  Unger says that ``terms for ordinary
things are incoherent [and] cannot apply to anything real''
\citep[147]{unger1979}.

Unger should not deny that believe that there are ordinary things.  If
our beliefs about tables and chairs are invariably false (even
incoherent), then what causes us to form these beliefs?  Why do we
believe in ordinary things to begin with?

Unlike van Inwagen and Merricks, Unger does not offer an explanation.
Having denied the existence of all ordinary things, he makes no
attempt to explain why we have so many false beliefs or what gives the
impression of coherency to our use of them in communication.  He seems
almost to revel in the strangeness of his position:

\begin{squote}
Now, it must of course be admitted that these arguments undermine the
possibility of any endeavor I should try to propose, or even the
putative thought that I should propose anything, just as all of my
putative essay is undermined.  But even so, I shall (incoherently)
propose that what we have now to do is invent new expressions which
are not inconsistent ones, and by means of which we may, to some
significant extent, think coherently about concrete reality
(\citeyear[544]{unger1980b}).
\end{squote}

I do not have an argument against the proposition that nearly all of
our language is hopelessly incoherent.  I do have a very hard time
believing that this is true, however; I'm not sure Unger believes it
himself.

\subsection{Explaining beliefs given the problem of the many}
\label{expl-many}
When presenting the problem of the many, Unger declares that he
prefers to maintain that there are no chairs, rather than that there
are pluralities of chairs.  Having made this nihilistic claim,
however, he does not offer an explanation of why we do in fact believe
that there are chairs.  However, he does not deny (at least not
explicitly) that there are things arranged chairwise, so Merricks'
explanation (see section \ref{connection}) might serve for Unger too.
We might propose, on Unger's behalf, that we believe that there are
chairs because there are things arranged chairwise.

In this case, however, I am tempted to repeat my arguments against
Merricks (section \ref{dogbush}).  I claimed that if there are things
arranged chairwise, {\em then there are chairs}.  If our belief that
there are chairs is caused by things arranged chairwise, then it is a
true, not a false, belief.

If we reject Unger's conclusion that there are no chairs, we are still
faced with a problem.  For we seem to run ourselves into the other
disjunct of the problem of the many.  If there are chairs, then there
are pluralities of chairs where we expect there to be only one.

This is an unacceptable conclusion, but not due to any explanatory
deficiency.  If we supposed that there was a plurality of chairs, then
{\em that} would explain why we believe that there are chairs.

\section{Referring to the many}
\label{refer}
Above (section \ref{many-merricks}) I mentioned that notions of
singular thought are threatened by the problem of the many.  More
should be said on this.

\ifstandalone
\bibliography{everything}
\bibliographystyle{ChicagoReedweb}
\end{spacing}
\fi
\end{document}


\chapter*{Conclusion}
         \addcontentsline{toc}{chapter}{Conclusion}
	\chaptermark{Conclusion}
	\markboth{Conclusion}{Conclusion}
	\setcounter{chapter}{4}
	\setcounter{section}{0}
	
Here's a conclusion, demonstrating the use of all that manual incrementing and table of contents adding that has to happen if you use the starred form of the chapter command. The deal is, the chapter command in \LaTeX\ does a lot of things: it increments the chapter counter, it resets the section counter to zero, it puts the name of the chapter into the table of contents and the running headers, and probably some other stuff. 

So, if you remove all that stuff because you don't like it to say ``Chapter 4: Conclusion'', then you have to manually add all the things \LaTeX\ would normally do for you. Maybe someday we'll write a new chapter macro that doesn't add ``Chapter X'' to the beginning of every chapter title.

\section{More info}
And here's some other random info: the first paragraph after a chapter title or section head \emph{shouldn't be} indented, because indents are to tell the reader that you're starting a new paragraph. Since that's obvious after a chapter or section title, proper typesetting doesn't add an indent there. 


%If you feel it necessary to include an appendix, it goes here.
    \appendix
      \chapter{The First Appendix}
      \chapter{The Second Appendix, for Fun}


%This is where endnotes are supposed to go, if you have them.

  \backmatter % backmatter makes the index and bibliography appear properly in the t.o.c...

% Make my bibliography be called "Bibliography" and not "References" (or "Works Cited" or...):
% \renewcommand{\bibname}{Works Cited}
    \bibliographystyle{ChicagoReedweb} % there are a variety of styles available; 
% replace ``plainnat'' with the style of choice. You can refer to files in the bsts or APA 
% subfolder, e.g. 
% \bibliographystyle{APA/apa-good}  % or
% \bibliographystyle{bsts/mla-good} 

% if you're using bibtex, the next line forces every entry in the bibtex file to be included
% in your bibliography, regardless of whether or not you've cited it in the thesis.
    %\nocite{*}
    \bibliography{everything}

\end{spacing}
% Finally, an index would go here... but it is also optional.
\end{document}
