% This is the Reed College LaTeX thesis template. Most of the work 
% for the document class was done by Sam Noble (SN), as well as this
% template. Later comments etc. by Ben Salzberg (BTS). Additional
% restructuring and APA support by Jess Youngberg (JY).
% Your comments and suggestions are more than welcome; please email
% them to cus@reed.edu

\documentclass[12pt,twoside]{reedfancy}
\usepackage{standalone}
\usepackage{anyfontsize}
\usepackage{graphicx,latexsym} 
\usepackage{amssymb,amsthm,amsmath}
\usepackage{tipa}
\usepackage{longtable,booktabs,setspace} 
\usepackage{verbatim}
\usepackage{url}
\usepackage{natbib}
\usepackage{enumitem}
\usepackage{url}
\usepackage{hyperref}
\setcitestyle{aysep={}}
\synctex=1

\DeclareSymbolFont{symbolsC}{U}{txsyc}{m}{n}
\DeclareMathSymbol{\strictif}{\mathrel}{symbolsC}{74}
\DeclareMathSymbol{\boxright}{\mathrel}{symbolsC}{128}

\newenvironment{squote}{%
	\begin{spacing}{1}
	\begin{list}{}{%
	\setlength{\labelwidth}{0pt}%
	\rightmargin\leftmargin%
	}
	\item\relax
	}{%
	\end{list}%
	\end{spacing}
	}

\newcommand{\stager}[4]%
{%
	\begin{spacing}{1}%
	\vspace{0pt}
		\begin{description}[style=nextline, noitemsep,
                    parsep=0pt, topsep=0pt, leftmargin=15mm,
                    itemindent=-10mm, font=\mdseries]
			\item[\textsc{#1} \emph{#2}] #3
			\item[]%
			\begin{flushright}#4\end{flushright}
		\end{description}%
	\end{spacing}%
}

\newcommand{\stage}[3]%
{%
	\begin{spacing}{1}%
	\vspace{0pt}
		\begin{description}[style=nextline, parsep=0pt,
                    leftmargin=15mm, itemindent=-10mm, font=\mdseries]
			\item[\textsc{#1} \emph{#2}] #3
		\end{description}%
	\end{spacing}%
}

\newenvironment{inq}{\vspace{0pt}%
	\begin{list}{}{%
	\setlength{\labelwidth}{0pt}%
	\setlength{\leftmargin}{2.5\oddsidemargin}%
	\setlength{\rightmargin}{\leftmargin}}
	\begin{spacing}{1}
	\item[]%
	}{
	\end{spacing}
	\end{list}
	\vspace{10pt}
	%\noindent%
	}
	
\newenvironment{epigram}{%
	\begin{minipage}[c]{0.75\textwidth}
	\vspace{2.5in}
	\begin{spacing}{1}
	\begin{list}{}{%
	\setlength{\labelwidth}{0pt}
	\setlength{\leftmargin}{1.4in}
	\setlength{\rightmargin}{.25in}}
	\item[]
	}{%
	\end{list}
	\end{spacing}
	\end{minipage}
	\newline
	}

% moved to reedfancy.cls:
%\def\thetitle{oink}
%\renewcommand{\firstmark}{\thetitle}
%\newcommand{\chapterpig}[1]{\def\thetitle{#1}}

\title{Why are there chairs?}
\author{Alexander A. Dunn}
% The month and year that you submit your FINAL draft TO THE LIBRARY
% (May or December)
\date{May 2012}
\division{Philosophy, Religion, Psychology, and Linguistics}
\advisor{Paul Hovda}
\department{Philosophy}

\setlength{\parskip}{0pt}

%%%%%%%%%%%%%%%%%%%%%%%%%%%%%%%%%%%%%%%%%%%%

\begin{document}

  \maketitle
  \frontmatter % this stuff will be roman-numbered
  \pagestyle{empty} % this removes page numbers from the frontmatter

%% \begin{epigram}
%% A curious thing about the ontological problem is its simplicity.  It
%% can be put in three Anglo-Saxon monosyllables: ``What is there?''  It
%% can be answered, moreover, in a word---``Everything''---and everyone
%% will accept this answer as true. \\ \\ \textsc{W.\,V.\,O.\ Quine
%%   \citeyearpar{quine1948}}
%% \end{epigram}

\begin{spacing}{1.25}

% Acknowledgements (Acceptable American spelling) are optional
% So are Acknowledgments (proper English spelling)
%% \chapter*{Acknowledgements}
%% \textsc{Thank you people.}

% The preface is optional
% To remove it, comment it out or delete it.
%    \chapter*{Preface}
%	This is an example of a thesis setup to use the reed thesis document class.

    \tableofcontents
% if you want a list of tables, optional
   % \listoftables
% if you want a list of figures, also optional
   % \listoffigures

% If your abstract is longer than a page, there may be a formatting issue.
\chapter*{Abstract}
Theories of metaphysical nihilism claim that there are no (or nearly
no) objects with parts.  Such theories claim that there are no chairs,
houses, mountains; some versions claim that there are no people.  The
philosophers who make these claims have trouble explaining why we
nonetheless believe that there are chairs.  The only successful
explanation is compatible both with there not being chairs and with
there being many things, including chairs.  Metaphysical universalism
is that claim that for every set of things, there is something else
made up of those things; this thesis is intuitively more plausible
than nihilism.  But if we assume that universalism is true, we have to
choose between two different versions of universalism: one that posits
a plurality of co-located (entirely overlapping) objects, or one that
denies that things (including chairs) can change their parts over
time.

\mainmatter % here the regular arabic numbering starts
\pagestyle{fancyplain} % turns page numbering back on
  
\chapter*{Introduction}
\chapterpig{Introduction}
\addcontentsline{toc}{chapter}{Introduction}
\chaptermark{Introduction}
\markboth{Introduction}{Introduction}
\documentclass[11pt]{article}
\usepackage{standalone} \newif\ifstandlone \standalonetrue
\usepackage[left=1.75in, right=1.75in, top=1.25in, bottom=1.25in]{geometry}
\geometry{letterpaper}
\usepackage{graphicx}
\usepackage{enumitem}
%\usepackage{amssymb}
\usepackage{amsmath}
\usepackage{epstopdf}
\usepackage{verbatim}
\usepackage{setspace}
\usepackage{natbib}
\setcitestyle{aysep={}}
\usepackage{url}
\usepackage{hyperref}
\synctex=1

\DeclareSymbolFont{symbolsC}{U}{txsyc}{m}{n}
\DeclareMathSymbol{\strictif}{\mathrel}{symbolsC}{74}
\DeclareMathSymbol{\boxright}{\mathrel}{symbolsC}{128}

\newenvironment{squote}{%
\begin{spacing}{1}
\begin{list}{}{%
\setlength{\labelwidth}{0pt}%
\rightmargin\leftmargin%
}
\item\relax
}{%
\end{list}%
\end{spacing}
}

\title{Introduction}
\author{Alexander A. Dunn}
\begin{document}
\ifstandalone
\maketitle
\begin{spacing}{1.5}
\fi

%\section{How to deny what is true}
It is true that there are chairs.  This, as far as I'm concerned, is
obvious.  If someone says that it is not true that there are
chairs---that there are not chairs---then it is clear to me that
somehow they have gone astray.  If they have an argument for this
conclusion, there must be something wrong with the argument.  There
must be something wrong because it is true---obviously true---that
there are chairs.

In many parts of what follows, I will often say things such as ``I
believe that there are chairs''.  This should not be taken as an
unwillingness to assert a stronger claim---that there are chairs.  The
stronger claim is, I have said, obviously true.  But I will use the
weaker claim---```I believe that there are chairs''---because while
the philosophers I am criticizing deny that there are chairs, they do
not deny that I believe that there are chairs.  And I am going to
argue that this fact alone---that I believe that there are
chairs---causes some trouble for their views, and gives us reason to
doubt their extraordinary conclusions.

So I believe that there are chairs.  I also believe that there are
desks, and desk lamps, and doors, and doorways, and houses, and
gardens, and plants.  Such things, and many others, are commonly
termed `ordinary things'.  This term is extremely vague in its
application, but is taken to refer to macroscopic objects, such as
those listed above, that are parts of our everyday lives.

Many philosophers have denied that ordinary things exist.  Until
recently, such a denial was generally a consequence of the
philosopher's views on other matters.  If a philosopher claimed that
there was no external world, or that the world was not at all like it
appears, then they might deny that there were any physical things, or
any things that exist outside the mind, or anything at all.  It would
follow from such a claim that there would be no ordinary things, like
chairs.  But the philosopher would not be specifically interesting in
denying that chairs exist.  They were interested in denying that {\em
  anything} exists; the denial of chairs was a minor consequence.

In the past 30 years, however, philosophers have begun to construct
arguments specifically aiming to show that there are no ordinary
things.  (Peter Unger was one of the first, with the aptly titled
paper, ``There are no ordinary things''.)  These philosophers do not
deny that there is an external world, or that it contains many
physical things; these propositions are readily granted to be true.
But the philosophers are unwilling to admit that such a world
does---or even possibly could---contain chairs.

Most philosophers making this sort of claim admit that it is strange
and unintuitive.  But they believe that the benefits of denying the
existence of ordinary things outweighs the costs.  Different
philosophers cite different benefits: consistency with regard to our
notion of composition, theoretical simplicity, or greater coherency
among our beliefs.  

The benefits do not out-weight the cost.  Moreover, I am unable to
imagine that any argument could convince me that there are no ordinary
things.  I believe that any argument that has the nonexistence of
chairs as a consequence is flawed.  Whether or not we can immediately
identify the flaw in the argument, the fact that it entails a
falsehood shows that something has gone amiss.

It will be objected that this is merely a fact about myself; other
philosophers are perfectly willing to deny that there are chairs.  (It
is another fact about me that I doubt that they really believe that
there are no chairs.)  It may be argued that the fact that I consider
``there are chairs'' to be true regardless of arguments against its
truth shows that I consider it to be, in some sense, a conceptual
truth.  It may be further argued that, since there are philosophers
willing to deny that ``there are chairs'' is true, what I mean by
``there are chairs'' is something different than what these
philosophers mean by ``there are chairs''.  We may be thought to be
using our words in different ways.

In section \ref{verbal} I will argue that we are {\em not} using our
words in different ways.  When I say ``there are chairs'' and someone
else says ``there are not chairs'', we are having a real disagreement.
Moreover, we are disagreeing in English; there is no special
`ontological language' in which we do metaphysical philosophy.

In section \ref{stroud} I will argue that any philosopher who attempts
to deny that there are chairs should be able to explain why we
nonetheless believe that there are chairs.  This is, I think, a
reasonable request, but it is surprisingly hard to satisfy.  The
difficulties that `nihilistic' philosophers have in explaining why we
believe that there are chairs should give us reason to suspect their
conclusions.

But even if we show that there are problems with the arguments of
philosophers who deny that ordinary things exist, we have not thereby
proved that they {\em do} exist.  The philosophers who say that there
are no chairs are motivated by a number of questions about the nature
of ordinary things.  For example, why are there chairs and tables, but
not chair-tables (single objects composed of an adjacent table and
chair)?

In section \ref{universe}, however, I will argue that some of the
considerations that philosophers take to be good reasons to deny that
chairs exist are not good reasons at all.  In effect, these
philosopher take the apparent non-existence of chair-tables to tell
against the existence of chairs and tables.  On the contrary, as we
will see, the {\em obvious} existence of chairs and tables tells for
the existence of chair-tables.  Additionally, I will argue that teams,
families, crews and other `groups' exist and have parts, just like
`material objects' like chairs.  

In section \ref{parts} I will argue (following Kit Fine) that things
can be `parts' of other things in many different ways.  The way that
something is part of a chair differs from the way a thing is part of a
group.  I will examine three different theries that seek to make sense
of the different ways that things are parts.  I will argue that all
three have the consequence that there are a {\em plurality of
  co-located objects}---that where we might think there is just one
think (a lump of clay), there are actually millions or more.  I will
suggest that this consequence is a reason to be suspicious of all
three theories, and should encourage us to look for a theory without
this consequence.

In section \ref{essential} I will attempt to develop such a theory.  I
will claim, for example, that the way in which a thing is part of a
group is the way that things are parts of sets---I will argue that
groups can be identified with sets.  Since sets cannot change their
membership over time, the set referred to by a term for a group (`the
Supreme Court') will change over time.  The most interesting
consequence of this is that the {\em identity conditions} for a group
over time---the conditions in which a certain set is identical with
the group---are wholly conventional.  Even more interesting is that
this seems to be the case with ordinary things like chairs as well;
whether and how a chair persists over time is largely up to us.

Throughout this thesis, there are certain things I will {\em not}
presuppose.  First, I will take no stand on whether or not things have
`temporal parts'.  I am not sure I fully understand the doctrine of
temporal parts, but it is often summarized thus: if a thing has
temporal parts, then for each time at which it exist, there exists at
that time (and only at that time) another thing---a temporal part or
`slice' of the larger object.  The (temporally) larger object is
somehow `built up' from these temporally smaller part.  If a thing
does not have temporal parts, then it is not divided into temporal
`slices'---it is ``wholly present'' at every moment of its existence.
Whatever this debate comes to, I will try to avoid relying on the
truth or falsity of the doctrine of temporal parts.

Second, I will not presuppose {\em eternalism}.  Eternalism is,
roughly, the view that past and future time are just as `real' as the
present.  An analogy is often drawn with space; what's behind me and
in front of me is just as real as what is under me.  There is nothing
special about `here' rather than `there'.  Likewise the eternalist
claims that `now' is no more special that `then'.  Eternalism is
generally opposed to {\em presentism}, which is the view that only the
present is real.  The presentist and the eternalist both agree that
there {\em were} dinosaurs, but for the eternalist there is a sense in
which there {\em are} dinosaurs (they are just not `now').  Again, I
will attempt to avoid committing myself to either of these positions.

\ifstandalone
\end{spacing}
\bibliography{everything}
\bibliographystyle{ChicagoReedweb}
\fi
\end{document}


% The three lines above are to make sure that the headers are right,
% that the intro gets included in the table of contents, and that it
% doesn't get numbered 1 so that chapter one is 1.

\chapter{What do I mean when I say there are chairs?}
\chapterpig{What do I mean when I say there are chairs?}
\documentclass[11pt]{article}
\usepackage{standalone} \newif\ifstandlone \standalonetrue
\usepackage[left=1.75in, right=1.75in, top=1.25in, bottom=1.25in]{geometry}
\geometry{letterpaper}
\usepackage{graphicx}
\usepackage{enumitem}
%\usepackage{amssymb}
\usepackage{amsmath}
\usepackage{epstopdf}
\usepackage{verbatim}
\usepackage{setspace}
\usepackage{natbib}
\setcitestyle{aysep={}}
\usepackage{hyperref}
\usepackage{url}
\synctex=1

\DeclareSymbolFont{symbolsC}{U}{txsyc}{m}{n}
\DeclareMathSymbol{\strictif}{\mathrel}{symbolsC}{74}
\DeclareMathSymbol{\boxright}{\mathrel}{symbolsC}{128}

\newenvironment{squote}{%
\begin{spacing}{1}
       	\begin{list}{}{%
\setlength{\labelwidth}{0pt}%
\rightmargin\leftmargin%
}
\item\relax
}{%
\end{list}%
\end{spacing}
}

\title{English ontology}
\author{Alexander A. Dunn}
\begin{document}
\ifstandalone
\maketitle
\begin{spacing}{1.25}
\fi

% \section{What are we doing?}
I believe that there are chairs.  I believe that there are desks, and
desk lamps, and doors, and doorways, and houses, and gardens, and
plants, and people.  Such things, and many others, are commonly termed
`ordinary things'.  This term is extremely vague in its application,
but is taken to refer to macroscopic objects, such as those listed
above, that are parts of our everyday lives.

Many philosophers have denied that ordinary things exist.  Until
recently, such a denial was generally a consequence of the
philosopher's views on other matters.  If a philosopher claimed that
there was no external world, or that the world was not at all like it
appears, then they might deny that there were any physical things, or
any things that exist outside the mind, or anything at all.  It would
follow from such a claim that there would be no ordinary things, like
chairs.  But the philosopher would not be specifically interesting in
denying that chairs exist.  They were interested in denying that {\em
  anything} exists; the denial of chairs was a minor consequence.

In the past 30 years, however, philosophers have begun to construct
arguments specifically aiming to show that there are no ordinary
things.  (Peter Unger was one of the first, with the aply titled
paper, ``There are no ordinary things''.)  These philosophers do not
deny that there is an external world, or that it contains many
physical things; these propositions are readily granted to be true.
But the philosophers are unwilling to admit that such a world
does---or even possibly could---contain chairs.

Most philosophers making this sort of claim admit that it is strange
and unintuitive.  But they believe that the benefits of denying the
existence of ordinary things outweighs the costs.  Different
philosophers cite different benefits: consistency with regard to our
notion of composition, theoretical simplicity, or greater coherency
among our beliefs.  

The benefits do not outweight the cost.  Moreover, I am unable to
imagine that any argument could convince me that there are no ordinary
things.  I believe that any argument that has the nonexistence of
chairs as a consequence is flawed.  Whether or not we can immediately
identify the flaw in the argument, the fact that it entails a
falsehood shows that something has gone amiss.

It will be objected that this is merely a fact about myself; other
philosophers are perfectly willing to deny that there are chairs.  (It
is another fact about me that I doubt that they really believe that
there are no chairs.)  However, there is another reason to resist
their conclusions, one that is independent of my inability to believe
that there are no chairs.  As we will see in Section \ref{stroud},
philosophers who deny that there are chairs have a difficult time
explaining why we believe that there are chairs.  To the extent that
they cannot explain why we hold this belief (and others concerning
ordinary things), we have reason to suspect that their denials might
be unfounded.

Even if we show that there are problems with the arguments of
philosophers who deny that ordinary things exist, we have not thereby
proved that they {\em do} exist.  The philosophers who say that there
are no chairs are motivated by a number of questions about the nature
of ordinary things.  For example, why are there chairs and tables, but
not chair-tables (single objects composed of an adjecent table and
chair)?  If chairs are physical things, then they are made up of atoms
(or even smaller things); is there a determinate number of this
microscopic objects in a given chair, and can we know what the number
is?

These and other questions have troubled some philosophers to the point
that they choose to deny that there are chairs at all.  If there are
no chairs, then there is no question of why there are chairs but not
chair-tables; if there are no chairs then there is no question of how
many atoms compose them.  But such a position is incorrect, because
there {\em are} chairs.  This thesis, therefore, is an attempt to
begin to answer some of the difficult questions about chairs and other
ordinary things.  I may be compelled to give some strange and
implausible answers of my own.  But no matter how strange, if they do
not have the consequence that there are no chairs, then I will
consider myself to have succeeded.

\ifstandalone
\end{spacing}
\bibliography{everything}
\bibliographystyle{ChicagoReedweb}
\fi
\end{document}


\chapter{Why do I believe that there are chairs?}
\chapterpig{Why do I believe that there are chairs?}
\documentclass[11pt]{article}
\usepackage{standalone} \newif\ifstandlone \standalonetrue
\usepackage[left=1.75in, right=1.75in, top=1.25in, bottom=1.25in]{geometry}
\geometry{letterpaper}
\usepackage{graphicx}
\usepackage{enumitem}
%\usepackage{amssymb}
\usepackage{amsmath}
\usepackage{verbatim}
\usepackage{epstopdf}
\usepackage{setspace}
\usepackage{natbib}
\setcitestyle{aysep={}}
\usepackage%[colorlinks=true, citecolor=blue, linkcolor=black]%
{hyperref}

\synctex=1

\DeclareSymbolFont{symbolsC}{U}{txsyc}{m}{n}
\DeclareMathSymbol{\strictif}{\mathrel}{symbolsC}{74}
\DeclareMathSymbol{\boxright}{\mathrel}{symbolsC}{128}

\newcommand{\stager}[4]%
{%
	\begin{spacing}{1}%
	\vspace{0pt}
		\begin{description}[style=nextline, noitemsep,
                    parsep=0pt, topsep=0pt, leftmargin=15mm,
                    itemindent=-10mm, font=\mdseries]
			\item[\textsc{#1} \emph{#2}] #3
			\item[]%
			\begin{flushright}#4\end{flushright}
		\end{description}%
	\end{spacing}%
}

\newcommand{\stage}[3]%
{%
	\begin{spacing}{1}%
	\vspace{0pt}
		\begin{description}[style=nextline, parsep=0pt,
                    leftmargin=15mm, itemindent=-10mm, font=\mdseries]
			\item[\textsc{#1} \emph{#2}] #3
		\end{description}%
	\end{spacing}%
}

\newenvironment{squote}{%
	\begin{spacing}{1}
	\begin{list}{}{%
	\setlength{\labelwidth}{0pt}%
	\rightmargin\leftmargin%
	}
	%\begin{singlespace}%
	\item\relax
	}{%
	%\end{singlespace}%
	\end{list}%
	\end{spacing}
	}

\newenvironment{inq}{\vspace{0pt}%
	\begin{list}{}%
	{\setlength\labelwidth{0pt}%
	\setlength\leftmargin{2.5\oddsidemargin}%
	\setlength\rightmargin{\leftmargin}}
	\begin{spacing}{1}
	\item[]%
	}{
	\end{spacing}
	\end{list}
	\vspace{10pt}
	%\noindent%
	}

\title{Why do you think that?}
\author{Alexander A. Dunn}
\begin{document}
\ifstandalone
\maketitle
\begin{spacing}{1.5}
\fi
\label{stroud}

% \begin{inq}
% The philosophical quest must start somewhere. It needs a set of
% beliefs about what the world is like. Without some attitudes,
% perceptions, beliefs, or theories to start with, it would have
% nothing to reflect on.~\citep[16]{stroud2000a}
%\end{inq}

\noindent Section \ref{intro-beliefs} will motivate my claim that a
nihilistic metaphysical thesis should be accompanied by an explanation
of why people nonetheless believe that there are chairs and other
ordinary things.  I will then look at the specific theses of Peter van
Inwagen and Trenton Merricks.  After assessing the ability of each to
explain our beliefs, I will myself try to explain why they think that
it is not obviously true that there are chairs.  Van Inwagen and
Merricks claim that it is not obviously true because they overestimate
what is required for ``there are chairs'' to be true.

\section{Explaining the beliefs of others}
\label{intro-beliefs}
\noindent Many people have false beliefs.  They believe things that
misrepresent (in some sense) how the world is.  For example, some
people believe that ghosts exist.  These people each hold a false
belief, for it is not true that ghosts exist.  There are no ghosts in
the world.  Despite this fact---that there are no ghosts---some people
believe that there are.  Why?  What explanation can we give as to why
someone believes a falsehood like this?

In explaining why someone holds a belief, we appeal to {\em reasons}.
Even people who hold beliefs that we may consider irrational (like the
belief that there are ghosts) have reasons for holding these beliefs.
They may not be good reasons; someone might believe that there are
ghosts because her older sister told her that there are ghosts, or
read ghost stories as a child and took them seriously.  Someone who
believes in ghosts might even think that she has {\em seen} a ghost.
This too would be a false belief; there are no ghosts, so nobody can
have seen one.  But here too there will be a reason why she holds this
false belief.  Perhaps she saw a strange play of light on a distant
wall, or the reflection of the moon filtered through an attic window.
What she actually saw was perhaps one of these things, but she somehow
took what she saw to be a ghost.  Probably she already believed that
there were ghosts, and so, when confronted with a deceptive or
confusing sight, was predisposed to form the mistaken belief that she
was seeing a ghost.

Here and in what follows, when I say that there is a reason why
someone believes something, I mean that there is some {\em cause} that
produced the belief.  Above, I told a causal story about why the
person who believes that she saw a ghost holds that belief.  She had
been told that there were ghosts by a person who she thought
trustworthy, so she came to believe that there are ghosts.  Holding
that believe caused her to be predisposed to interpret unusual
phenomena as ghosts.  This disposition caused her to believe that she
was seeing a ghost when she saw a reflection of the moon.

My use of `reason', therefore, should be taken in this causal sense.
There are other ways that people use the word `reason'.  If someone
asks ``What reason do you have to believe that $((P \rightarrow Q )
\wedge P) \rightarrow Q$?''  I might reply that it is a theorem of
first-order logic.  Here I am not telling a causal story.  I am rather
{\em justifying} my belief that $((P \rightarrow Q ) \wedge P)
\rightarrow Q$.  But in this case it is perfectly correct to say that
I am giving a reason as to why I hold a belief.  It is just not a {\em
  causal} reason.  A causal reason would be something like the
following: $((P \rightarrow Q ) \wedge P) \rightarrow Q$ is true, and
I have learned the rules of logic, and so I can prove that $((P
\rightarrow Q ) \wedge P) \rightarrow Q$.

(Another example: suppose someone falsely believes that $((P
\rightarrow Q ) \wedge Q) \rightarrow P$ is a theorem of first-order
logic.  There will be some (causal) reason why they hold this belief;
probably they attempted to deduce it from no premises and therefore
believe that they succeeded.  There will, in turn, be a reason why
they hold {\em this} false belief; maybe they were not concentrating
on the proof steps, or they forgot certain rules of deduction.)

An example involving an apparently obviously true belief might help
clarify the distinction between causal reason and justifying reasons.
If someone were to ask me why I believe that the sky is blue during
the day, my immediate answer would probably be ``well, because it
is!''  There's not much else I can say to {\em justify} my belief.
But this not a {\em causal} explanation.  The fact that something is
true (the sky {\em is} blue) does not cause me to believe it.
Otherwise I would believe every truth, and I do not.  There are
doubtless many truths that I do not believe.  There must therefore be
another (causal) reason why I believe that the sky is blue, other than
the fact that the sky is blue.

I believe that the sky is blue because, first, it is blue, and second,
I have {\em seen} that it is blue.  My vision is generally reliable
(or at least seems to be), so the fact that my eyes `tell' me
something is good reason to believe it.  The same is true of my other
senses: they are generally reliable, so the fact that they `tell' me
something is a good reason to believe it.  It does not follow that it
is {\em true}, however (though no doubt we believe that it is true);
our eyes can be deceived.

A skeptic might claim that we cannot rule out the possibility that we
are {\em constantly} deceived.  They attempt to undermine the
reliability of our senses.  I will not be addressing such arguments.
Rather, in what follows I will examine arguments that deny (or appear
to deny) that many of our beliefs about `ordinary things' are true.
The philosophers making these denials do not claim that our eyes are
unreliable sources of information.  Their arguments are metaphysical
rather than epistemic; they deny that certain objects are {\em
  possible}.  

For example, Peter Unger claims that chairs do not exist.  He relies
on a number of metaphysical arguments to motivate this claim.  If he
is right, however, then it seems to follow from this that beliefs like
the following are necessarily false:

\begin{itemize}
  \item Some chairs are made of wood.
  \item There have existed many chairs which no longer exist.
  \item There are chairs.
\end{itemize}

I, however, believe that all of these propositions are true.  Even if
Unger is right, and they are all false, it still seems to be the case
that there are reasons why I believe these propositions.

If someone were to ask {\em me} why I believe that there are chairs, I
would probably answer ``because there are, and I have seen them (and
sat upon them)!''  It seems obviously true, just like the fact that
the sky is blue.  I have seen lots of chairs, and I can't have been
confused or deceived {\em every} time.

Nonetheless, Peter Unger and other philosophers (who we will call
`nihilists' or `eliminativists') say that I am mistaken.  They claim
that I have not in fact seen lots of chairs, though I may believe that
I have.  There are several different arguments by which nihilists seek
to establish that chairs (and other `ordinary things') do not exist;
we will examine some of these arguments below.  Having made these
arguments, however, the nihilists must reject our causal explanation
of why we believe that there are chairs.  Our explanation was that
there are chairs and we can see them.  But the nihilist denies that
there are chairs, and so should admit that, if we believe that there
are chairs, there must be a different explanation as to why we hold
this belief.

\subsection{Why bother?}
A metaphysical thesis that involves denying the existence of ordinary
things like chairs entails that the simplest explanation of why we
believe that there are chairs is incorrect.  I believe that such a
thesis should therefore be supplemented with a new explanation.  This
new explanation would identify the reasons why we would believe that
there are chairs if there are in fact none.  But why should I demand
this of a metaphysical theory?  Is it a reasonable request?

As an analogy, consider color.  Most people believe that things are
colored.  A simple causal story about why people believe that things
are colored might go like this:  things are colored, and people see
that things are colored.  

But imagine a philosopher who holds some version of {\em physicalism}
and claims that the world as described by physics is all that there
is.  This view is often thought to have the consequence that things
aren't actually colored.  In the `vocabulary of physics', things might
be described in such a way that the things color gets somehow left
out.  We may be unable to determine from the `physical description'
what color the object is.  The colors of objects are not included in
this philosopher's description of the world.

If the philosopher admits that people believe that things are colored,
she cannot explain this using the same story that I used above.  I
said that people believe that things are colored and that they see
that things are colored.  But the physicalist maintains that things
are not colored.  {\em If} she admits that people believe that things
are colored, then she needs a different explanation as to why people
believe that things are colored.

She might, however, deny that people believe that things are colored.
(This would be a rather bold claim.)  She could say that the notion of
color is entirely illusory.  If we believe that we see colors, she may
tell us we are wrong.  When we think that something is colored, we are
mistaken.  If we think that an apple is red, we have a false belief.
She might claim that color does not pose a difficulty for her view,
because humans do not experience `color'.

This, as I said, is a rather bold claim.  It seems simply true that we
see colors and that the apple looks red.  If a philosopher were to
deny these things, I would have difficulty understanding what she
meant.  This is not to say she is {\em wrong}; I have no argument
proving that her thesis is false.  But the claim that humans do not
experience color seems bizarre and unmotivated.  Fortunately I do not
know of anyone who actually holds this view.

Our imagined philosopher might make a less bold claim.  She might
instead claim that color is one of those things that are `subjective'
rather than `objective' or `absolute' features of the world.  A
subjective feature of the world is a feature that is present only
because we (or some other being) exists to experience it:

\begin{squote}
Whatever is due only to us and to our own ways of responding to and
interacting with the world does not reflect or correspond to anything
present in the world as it is independently of us.  The aim of an
``absolute'' conception, then, is to form a description of the way the
world is, not just independently of its being believed to be that way,
but independently, too, of all the ways in which it happens to present
itself to us human beings from our particular standpoint within
it\,\ldots\,[So we] form some conception of that independent reality
and come to understand parts or aspects of our original conception of
the world as not representing it as it is.  If we see them as products
or reflections of something peculiar to human experience or to the
human perspective on the universe, we assign them a merely
``subjective'' or dependent status and eliminate them from our
conception of the world as it is independently of
us~\citep[30--31]{stroud2000a}.
\end{squote}

A philosopher who adheres to this distinction might claim that our
conception of the world as colored does not represent the world as it
is independently of us.  Colors, she would claim, are not objectively
real.  She allows, however, that they are subjectively real.  She
admits that people do see colors.  Because of our color vision, we
come to believe that the things we see are colored.  A philosopher who
denies the objective reality of color does not thereby ``deny that we
perceive many different colours or that we believe physical objects to
be coloured'' \citep[145]{stroud2000a}.  What this philosopher claims
is something to the effect that, while we see things {\em as} colored,
things are not {\em themselves} colored.  The red color of a tomato,
on this view, obtains only in our perception of the tomato; there is
nothing {\em in} the tomato that is the redness (other species may not
see the redness when they see the tomato).

The philosopher who is denying the objective reality of color does
``recognize the presence in the world of perceptions of and beliefs
about the colours of things'' \citep[199]{stroud2000a}.  The challenge
then is for her to explain why we do have these perceptions and
beliefs.  If she believes that only the world of physics is
objectively real, she must explain why we hold these beliefs, and she
must give this explanation in such a way that commits her only to the
existence of physical things.  If she claims that the world as
described by physics is the only world there is, then she must explain
why, in a world that contains only physical things, we come to believe
that there are colors and colored objects.

Again: if our physicalist philosopher admits that people believe that
they experience color, and admits that people believe that things are
colored, {\em then} she commits herself to explaining why we form
beliefs that are, according to her, false.  Here is the analogy with
metaphysicians like Peter Unger: {\em if} they admit that many of us
believe that there are chairs and other ordinary objects, then they
commit themselves to explaining why we form these false beliefs.  For
as we have seen, even false beliefs are generally held for a reason.

\subsection{Paraphrasing beliefs}
\label{paraphrase}
Peter Unger denies that chairs exist, and claims that, if we believe
that chairs exist, we are mistaken.  His task will be to explain why
we form these false beliefs.  But not all nihilistic philosophers deny
that we are, in fact, mistaken.  They deny that there are any chairs,
but maintain that beliefs like the following might still be true:

\begin{itemize}
  \item There are two chairs in the next room.
  \item I own some very nice 17th-century chairs.
  \item Some chairs are heavier than some tables.
\end{itemize}

Peter van Inwagen is one of these philosophers.  He denies the
existence of tables, chairs, apples, and all other inanimate composite
objects (van Inwagen's technical definition of `composite' will be
discussed below in section~\ref{scq}).  He takes pains to make clear
that his denial of these things is not a relegation of tables and
chairs to `subjective reality'.  He wants to claim that such things do
not exist in any way, subjective or objective:
\begin{squote}
I want to do what I can to disown a certain apparently almost
irresistible characterization of my view, or of that part of my view
that pertains to inanimate objects.  Many philosophers, in
conversation and correspondence, have insisted, despite repeated
protests on my part, on describing my position in words like these:
``Van Inwagen says that tables are not real''; ``\ldots\,not true
objects''; ``\ldots\,not actually {\em things}''; ``\ldots\,not
substances''; ``\ldots\,not unified wholes''; ``\ldots\,nothing more
than collections of particles.''  These are words that darken counsel.
They are, in fact, perfectly meaningless.  My position vis-\`{a}-vis
tables and other inanimate objects is simply that there {\em are}
none~(\citeyear[99]{inwagen1995}).
\end{squote}

Van Inwagen asserts, quite seriously, that ``there are no tables or
chairs or any other visible objects except living organisms''
(\citeyear[1]{inwagen1995}).  This is a somewhat more bold claim than
that of the physicalist's with regard to color.  She at least granted
that we do see colors, even if we don't actually see things that are
(objectively) colored.  If, as van Inwagen claims, the only {\em
  visible} objects are living organisms, then we certainly can't {\em
  see} chairs at all.  But just as our physicalist could not claim
that we don't believe that there are colors, van Inwagen cannot deny
that we at least {\em believe} that there are chairs.

Van Inwagen does not attempt to deny that we hold beliefs like those
listed above.  He admits that many of us hold beliefs that we would
express as ``there are two chairs in the next room'' or ``I bought a
new chair today''.  Indeed, he admits that such beliefs are often {\em
  true}: ``when people say things in the ordinary business of life by
uttering sentences that start `There are chairs\,\ldots ' or `There
are stars\,\ldots ', they very often say things that are literally
true''~(\citeyear[102]{inwagen1995}).  Van Inwagen, when denying that
we have beliefs about chairs, appears to maintain that the beliefs
that we (erroneously) take to be about chairs are not, in fact,
beliefs about chairs.  If a belief expressed as ``that is a fine
chair'' was actually about a chair, then it could only be true if
there was at least one chair (a fine one).  But van Inwagen denies
that there is at least one chair, but nonetheless says that such a
belief might be true.  He accordingly recognizes the need to explain
what our beliefs really are about.  If he explains what the {\em
  content} of our beliefs is, then he will also be able to explain
{\em why} we hold such beliefs.

\ifstandalone
\end{spacing}
\bibliography{everything}
\bibliographystyle{ChicagoReedweb}
\fi
\end{document}

\documentclass[11pt]{standalone}
\usepackage{standalone} \newif\ifstandlone \standalonetrue
\usepackage[left=1.75in, right=1.75in, top=1.25in, bottom=1.25in]{geometry}
\geometry{letterpaper}
\usepackage{verbatim}
\usepackage{graphicx}
\usepackage{enumitem}
%\usepackage{amssymb}
\usepackage{amsmath}
\usepackage{epstopdf}
\usepackage{setspace}
\usepackage{natbib}
\setcitestyle{aysep={}}
\usepackage{hyperref}
\usepackage{url}
\synctex=1

\DeclareSymbolFont{symbolsC}{U}{txsyc}{m}{n}
\DeclareMathSymbol{\strictif}{\mathrel}{symbolsC}{74}
\DeclareMathSymbol{\boxright}{\mathrel}{symbolsC}{128}

\newenvironment{squote}{%
\begin{spacing}{1}
       	\begin{list}{}{%
\setlength{\labelwidth}{0pt}%
\rightmargin\leftmargin%
}
\item\relax
}{%
\end{list}%
\end{spacing}
}

\title{1.25}
\author{Alexander A. Dunn}
\begin{document}
\ifstandalone
\maketitle
\begin{spacing}{1.5}
\fi

\section{Paraphrases}
I have proposed that any attempt to deny the existence of ordinary
things such as tables and chairs must be supplemented by an
explanation as to why we believe in the existence of ordinary things.
As we will see, Peter Unger's claim that there are no ordinary things
and his claim that propositions like ``that is a chair'' are uniformly
false leaves it quite mysterious why we take there to be chairs in the
first place.  Some nihilistic philosophers, therefore, have attempted
to maintain Unger's first thesis---that there are no ordinary
things---while rejecting the second---that ordinary thing discourse is
invariably false.  Such a philosopher will claim that such discourse
is {\em compatible} with the nonexistence of chairs.  This may involve
the claim that while we take ourselves to have beliefs about chairs
and other ordinary things, our beliefs do not actually concern such
(non-existent) entities.  Rather, our thought and talk is (or should
be seen as) relating to such things as do exist.  Strategies that
follow this pattern can be called paraphrasing strategies, and Peter
van Inwagen has presented a well-known version.

Like Unger, van Inwagen claims that (necessarily) there are simply no
such things as tables and chairs in the world.  But unlike Unger, he
does not claim that when we take ourselves to be thinking and talking
about such things, we are thinking and talking about nothing at all.
At least, ``when people say things in the ordinary business of life by
uttering sentences that start `There are chairs\,\ldots ' or `There
are stars\,\ldots ', they very often say things that are literally
true'' \citep[102]{inwagen1995}.  

One might assume that if such statements are true, then it follows
that there are chairs and stars.  But van Inwagen denies that chairs
and stars exist.  How can he claim, then, that what was said was true?
What van Inwagen does is attempt to show that the statements in
question can be {\em paraphrased}---they can be reformulated to show
that they have no `ontological commitments'.  According to van
Inwagen, one can assert that there is a chair without being committed
to the existence of chairs.

Section~\ref{comp} will summarize the motivation for van Inwagen's
denial.  Section~\ref{inwagen} will introduce and criticize van
Inwagen's paraphrasing strategy.

\subsection{Composition}
\label{comp}
Van Inwagen's conclusion that there are no chairs is a consequence of
his views on {\em composition} (or `constitution').  Some things are
said to compose another thing if the former are {\em parts} of the
latter; the latter is `made up of' the former.  Van Inwagen believes
that ``the metaphysically puzzling features of material objects are
connected in deep and essential ways with metaphysically puzzling
features of the constitution of material objects by their
parts''~\citep[18]{inwagen1995}.  An example of such a puzzle is the
Ship of Theseus.  The Ship of Theseus is (presumably) an object
composed of many parts, including planks of wood.  As the planks (and
other parts of the ship) wear out, they are replaced.  These
replacements happen each by themselves; the entire ship (or even a
large section) is not replaced all at once.  But eventually no part of
the original ship remains; it is build of entirely different planks,
nails, rigging, etc.  And yet we would commonly say that it is still
the same ship.  But why should we think that the present ship is
identical with a past ship with which it shares no parts?

\subsection{The Special Composition Question}
\label{scq}
Answering the question `why is this ship identical with that past
ship?' requires first figuring out why (and how) these planks and
rigging and sails (et.\ al.) compose a ship in the first place.  Van
Inwagen asks ``in what circumstances do planks\footnote{For
  simplicity's sake, van Inwagen ignores the rigging and sails.}
compose (add up to, form) something?'' (\citeyear[21]{inwagen1995}) 
For some $x$s, then, van Inwagen asks us to consider when
\begin{equation}
\exists y\ \text{the}\ x\text{s compose}\ y
\end{equation}
is true.%
\footnote{Van Inwagen explains in some detail how plural referring
  expressions (like ``the planks'') can be given a logical
  formalization (\citeyear[23--28]{inwagen1995}), but suffice to say
  they work just as one would expect.}
%
\ Less formally, van Inwagen asks: ``suppose one had certain
(nonoverlapping) objects, the $x$s, at one's disposal; what would one
have to do---what {\em could} one do---to get the $x$s to compose
something?'' (\citeyear[31]{inwagen1995})  This is the Special
Composition Question.

(`Composition' is used in a technical sense with regard to the Special
Composition Question.  Van Inwagen defines it thus: ``the $x$s compose
$y$'' means that ``the $x$s are all parts of $y$ and no two of the
$x$s overlap and every part of $y$ overlaps at least one of the
$x$s\,\ldots\,a thing {\em overlaps} a thing---or: they overlap---if
they have a common part'' (\citeyear[29]{inwagen1995}).  For van
Inwagen, everything is a part of itself; some $x$ is a {\em proper}
part of some $y$ only if $x \neq y$.)

\subsection{The usual answers}
\label{scq-ans}
There are several prominent answers to the Special Composition
Question, including the following (These formulations are from
\citet{markosian1998a}):
\begin{description}
	\item[Nihilism] Necessarily, for any $x$s, there is an object
          composed of the $x$s iff there is only one of the $x$s,
          i.e., the only objects that exist are
          simples (\citeyear[219]{markosian1998a}).
	%
	\footnote{\label{flip} Note that this may not be Unger's view.
          He denies that people, apples, cheese, tables, chairs, and
          other ``ordinary things'' are nonexistence but he does not,
          as far as I know, take a stand on whether anything at all
          exists.  His view can be (flippantly) summarized thus: ``if
          we have a word for it, it doesn't exist.''}
	%\footnote{\label{gunk} Of course, it may be that the world is
	%not fundamentally particulate, and is filled not with simples
	%but with `gunk'; see \citet{schaffer2003}.  Nihilism (and van
	%Inwagen's second condition below) can be formulated to take
	%this possibility into account: ``for any quantity of gunk,
	%there is nothing composed of it.''}%
	%
	\item[Universalism] Necessarily, for any $x$s, there is an
          object composed of the $x$s iff no two of the $x$s
          overlap (\citeyear[227]{markosian1998a}).
	\item[Van Inwagenism] Necessarily, for any $x$s, there is an
          object composed of the $x$s iff either (i) the activity of
          the $x$s constitutes a life or (ii) there is only one of the
          $x$s (\citeyear[221]{markosian1998a}).
\end{description}

We will discuss Unger's version of nihilism in section \ref{unger}.
As I will argue, any version of nihilism that does not explain our
beliefs in the existence of ordinary objects is problematic.

Universalism raises a number of problems in relation to Peter Unger's
`problem of the many' (see section \ref{many}).  I will therefore
postpone discussion of this view until later.

Van Inwagen examines and rejects universalism and the version of
nihilism given above.  He also rejects a number of other answers to
the Special Composition Question.  Some are too strong: `some $x$s
compose a $y$ iff the $x$s are in contact' would entail that two
people shaking hands will result in a new object coming into being.
Others are too strong in some ways and too weak in others: `some $x$s
compose a $y$ iff the $x$s are fastened together' would entail that
two people being glued together would result in a new object; and it
would deny that an object can be composed without fastening its parts
together (such as when building a house of cards).  The only answer
van Inwagen finds consistent is what we have dubbed {\em van
  Inwagenism}, which entails that tables and chairs do not exist.

Because of this consequence, van Inwagenism should include an
explanation why we nonetheless believe that there are tables and
chairs.  Happily, van Inwagen recognizes this and is prepared with a
{\em paraphrasing strategy}.  This strategy aims to show that the
beliefs that we take to be about tables and chairs are really about
something else, and are not true beliefs.  If such beliefs are true,
then it should be relatively easy to explain why hold them: they are
true, and we learn of them through some reliable means (like our
eyes).

Unfortunately, van Inwagen's paraphrasing strategy does not work.

\section{Van Inwagen's paraphrasing strategy}
\label{inwagen}
Peter Unger maintains that terms like `chair' are incoherent; if this
is so, a statement involving the phrase `There is a chair\,\ldots '
could surely not be true.  Van Inwagen, on the other hand, admits that
``when people say things in the ordinary business of life by uttering
sentences that start `There are chairs\,\ldots ' or `There are
stars\,\ldots ', they very often say things that are literally true''
\cite[102]{inwagen1995}.  It does not seem unreasonable to assume that
if what people say with ``There are chairs\,\ldots '' and the like are
true, then chairs exist.  But van Inwagen denies this entailment.

How can van Inwagen maintain this?  Someone can say, truly, ``There
are simples arranged chairwise\,\ldots '' without committing oneself
to the existence of chairs.  Van Inwagen might then claim that when
someone says ``There is a chair\,\ldots '' she {\em means} `There are
simples arranged chairwise.'  This is, of course, a bold hypothesis
about the speech practices of ordinary speakers.  Certainly very few
speakers would, if asked, affirm that what they meant to say had
anything to do with simples; they would say that when they said that
there was a chair, they meant just that.  Van Inwagen recognizes that
this is not a viable position: ``The only thing I have to say about
what the ordinary man really means by `There are two valuable chairs
in the next room' is that he really means that there are two valuable
chairs in the next room'' (\citeyear[106]{inwagen1995}).

One might then assume that van Inwagen is thinking in analogy with
Russell.  He could attempt to claim that, despite the surface
appearance of language (`There is a chair\,\ldots '), the underlying
logical form does not make any mention of chairs (or tables); the
offending concept is analyzed away, leaving `There are simples
arranged chairwise\,\ldots '.  Van Inwagen notes that his ``suggested
technique of paraphrasing enables us to escape some of the more
embarrassing consequences of this position.  When someone says `Some
tables are heavier than some chairs,' there is obviously something
right about what he says.  Our technique of paraphrasis enables us to
capture what it is that is right about what he says''
(\citeyear[111]{inwagen1995}).  However, on the very next page he
admits that the ordinary language proposition and his paraphrased
version are different propositions: ``When the ordinary man utters the
sentence `Some chairs are heavier than some tables' (in an appropriate
context, and so on and so on), he expresses a certain proposition, and
one that is almost certainly true.  But I do not claim that this
proposition {\em is} the proposition that, for some $x$s, those $x$s
are arranged chairwise and for some $y$s, those $y$s are arranged
tablewise, and the $x$s are heavier than the $y$s''
(\citeyear[112]{inwagen1995}).  So van Inwagen is not making an appeal
to some notion of `logical form'.  But then what is the purpose of the
paraphrasing project?

Van Inwagen attempts to justify his method of paraphrasis by asserting
the following parallels between the original and paraphrased
propositions:
\begin{enumerate}[label=(\Alph*)]
	\item The paraphrase describes the same fact as the
          original.  \label{para-a}
	\item The paraphrase, unlike the original, does not even
          appear to imply that there are any objects that occupy
          chair-receptacles.  \label{para-b}
	\item The paraphrase is neutral with respect to competing
          metaphysical theories, {\em viz}.  the ``received'' theory,
          that there are objects that occupy chair-receptacles, and
          the theory I have proposed, according to which there are no
          such objects.  \label{para-c}
	\item The original, though it doubtless does not express the
          same proposition as the paraphrase, has the feature ascribed
          to the paraphrase in \ref{para-c}: It is neutral with
          respect to the question whether there are objects that fit
          exactly into
          chair-receptacles~(\citeyear[113]{inwagen1995}).  \label{para-d}
\end{enumerate}

I am rather dubious as to the truth of \ref{para-a}, but I am quite
sure that \ref{para-d} is false, and van Inwagen's thesis appears to
depend on it.  He admits in \ref{para-b} that the original sentence
(`There are chairs\,\ldots ') {\em implies} that there are chairs, but
claims in \ref{para-d} that it does not {\em entail} this.  But why
wouldn't it?

\subsection{Propositions and ontological commitment}
Let us review the situation.  First, van Inwagen agrees that when
someone says things like ``There is a chair\,\ldots '' they mean just
that.  Second, he admits that his `paraphrases' of such propositions
are not so faithful to the original that they can be called the same
proposition; the original and the paraphrase are two different
propositions.  Third, he claims nonetheless that {\em neither} the
original nor the paraphrase entail the existence of chairs.

This seems obviously untrue.  How can he claim that when someone says
``There is a chair\,\ldots '' and means just that, that the
proposition they express does not entail the existence of chairs?  To
defend his claim, van Inwagen appeals to his `Copernican analogy':

\begin{squote}
I accept the Copernican Hypothesis.  One day you hear me say, ``It was
cooler in the garden after the sun had moved behind the elms.''  You
say, ``You see, you can't consistently maintain your Copernicanism
outside the astronomer's study.  You say that the sun moved behind the
elms; yet, according to your official theory, the sun does not move.''
I reply that the proposition I expressed by saying ``It was cooler in
the garden after the sun had moved behind the elms'' is consistent
with the Copernican Hypothesis (\citeyear[101]{inwagen1995}).
\end{squote}
That is, van Inwagen claims that the proposition he expressed with
``It was cooler in the garden after the sun had moved behind the
elms'' does not entail that the sun actually moved.  And he argues
that this is analogous to our talk of chairs: most propositions
expressed with ``There is a chair\,\ldots '' do not entail that chairs
actually exist.

First, does the proposition van Inwagen expresses with ``The sun moved
behind the elms'' entail that the sun moved? I am inclined to say that
it does.  If I were to say simply ``The sun moved'' (meaning just
that), I think I would have committed myself to the movement of the
sun.  Why should we think that the addition of `behind the elms'
defeats this entailment?  Without some explanation of what the
difference is, I see no reason to think that saying ``The sun moved
behind the elms'' (and meaning it) does not entail the movement of the
sun.  But van Inwagen may be forced to say here that neither
proposition entails that the sun moved.  For he certainly won't allow
that either entails that the sun {\em exists}.

There is an analogy here, though perhaps not the one van Inwagen had
in mind.  He claims that a proposition expressed by ``There are two
very valuable chairs in the next room'' does not necessarily entail
the existence of chairs.  If this proposition does not entail that
chairs exist, then what about `There are two valuable chairs left in
the world' or `There are at least two chairs in the world' or `There
are at least two chairs' or simply `There are chairs'?  Van Inwagen
appears committed to the claim that the proposition I would express
with ``There are chairs'' does not entail that there are chairs.

Why on earth should this be?  Does not the proposition expressed by my
saying ``There are simples arranged chairwise\,\ldots '' entail the
existence of simples?  If van Inwagen says that there are simples
arranged chairwise, and means just that, then it would appear to
follow that there are simples.  Indeed, van Inwagen's argument relies
rather heavily on the assumption that simples exist.
%
%% \footnote{Ted Sider takes him to task for this
%%   assumption~(\citeyear{sider1993}), claiming that the possibility of
%%   `gunk'---the possibility that the matter of the world is not
%%   fundamentally particulate but infinitely divisible---falsifies van
%%   Inwagen's thesis.  I think it may be possible for van Inwagen to
%%   adapt to a gunky world (he might be able to claim that nothing
%%   exists but organisms, who are composed of other organisms and/or
%%   gunk), but I think van Inwagen's thesis is false either way.}
%
\ But if `There are chairs' does not entail that there are chairs and
if `The sun moved behind the trees' entails neither that the sun moved
nor that the sun exists, then how can van Inwagen maintain that `There
are simples arranged chairwise' entails that there are simples, or
that they are arranged chairwise?  He has given us no reason to
believe one and not the other.

\subsection{Loose truth}
\label{loose-v}
Van Inwagen's attempt to maintain that there are no chairs and that
``there are chairs'' is true does not appear promising.  But he may
instead claim that ``there are chairs'' is not true, but {\em loosely
  true}.  He admits this position as a possibility:

\begin{squote}
I can say this [that `There are chairs\,\ldots ' can be true yet not
  entail that there are chairs] because I accept certain theses in the
philosophy of language.  I can say this because I accept certain
theses in the philosophy of language.  Some people, I suppose, would
reject these theses.  These people would say that when I
said\,\ldots\ The sun moved behind the elms,' I said something
false\,\ldots If someone maintains that `The sun moved behind the
elms' expresses a falsehood, he must still have some way to
distinguish between this sentence and those sentences (like `The sun
exploded' and `The sun turned green') that the vulgar would regard as
the sentences that expressed falsehoods about the sun\,\ldots [if I
  took this position,] I should not be willing to say that people who
uttered things like `There are two valuable chairs in the next room'
very often said what was true.  I should be willing to say only that
they very often say what might be treated as a truth for all practical
purposes (\citeyear[102--103]{inwagen1995}).
\end{squote}

Van Inwagen admits that `There are two very valuable chairs in the
next room', ``when it is successfully used to report a fact, does
report a fact about the existence of {\em something}''
(\citeyear[102]{inwagen1995}).  Presumably van Inwagen thinks that
`something' is the chairwise arrangements of simples.  Van Inwagen
must therefore explain how a chairwise arrangement of simples can
cause people to believe that there are chairs.  Van Inwagen attempts
to bolster his case by drawing an analogy with the imaginary bliger.
The bliger, according to van Inwagen, is what happens when a monkey,
four owls, and a tiger attach themselves together temporarily.  The
conglomeration appears to the untrained observer to be a single
animal, and gullible farmers dubbed it a `bliger'.  Van Inwagen's
point is that there are no bligers, but that a farmer saying ``there's
a bliger'' when pointing at such a conglomeration would be saying
something loosely true.  The fact being reported by ``there's a
bliger'' is the fact that a monkey, four owls, and a tiger are there.
People believe that there are bligers because they mistake the group
of animals for a single thing, which has been dubbed `bliger'.

Likewise, van Inwagen maintains that people mistake chairwise
arrangements of simples for chairs.  When someone says ``there's a
chair'' what she says is loosely true because there is a chairwise
arrangement of simples there.  People believe that there are chairs
because they mistake the group of animals for a single thing, which
has been dubbed `chair'.

I agree with van Inwagen that these cases are analogous.  However,
where van Inwagen takes this analogy to show that there are no chairs,
I take it to show that there {\em are} bligers in van Inwagen's
imaginary scenario.  When it is discovered that bligers are built up
from six other creatures, we are learning something about bligers:

\begin{squote}
\ldots {\em of course} there are bligers in [van Inwagen's] story.
Bligers are what the story is about.  The zoologists do not report
that there are no bligers.  Rather they tell us what a bliger is.
They explain that a bliger is not a single large carnivorous animal
but a transient symbiotic union of six animals
\citep[704]{rosenberg1993}.
\end{squote}

In short, van Inwagen's analogy does not provide us with an
explanation of why we would believe in chairs even if there were none.
Just as the Ungerian treatment of loose truth failed to explain our
beliefs in chairs, van Inwagen's appeal to loose truth does not
explain why we believe that there are chairs, rather than chairwise
arrangements of simples.

\ifstandalone
\end{spacing}
\fi
\end{document}

\documentclass[11pt]{article}
\usepackage{standalone} \newif\ifstandlone \standalonetrue
\usepackage[left=1.75in, right=1.75in, top=1.25in, bottom=1.25in]{geometry}
\geometry{letterpaper}
\usepackage{graphicx}
\usepackage{enumitem}
\usepackage{amssymb}
\usepackage{amsmath}
\usepackage{tipa}
\usepackage{epstopdf}
\usepackage{verbatim}
\usepackage{setspace}
\usepackage{natbib}
\setcitestyle{aysep={}}
\usepackage{url}
\synctex=1
\usepackage{hyperref}

\newenvironment{squote}{%
\begin{spacing}{1}
       	\begin{list}{}{%
\setlength{\labelwidth}{0pt}%
\rightmargin\leftmargin%
}
\item\relax
}{%
\end{list}%
\end{spacing}
}

\title{``Nearly as good as true''}
\author{Alexander A. Dunn}
\begin{document}
\ifstandalone
\maketitle
\begin{spacing}{1.25}
\fi

\section{How does Merricks explain what we believe?}
\label{universe}
\label{merricks}
Trenton Merricks comes to the same metaphysical conclusions as does
van Inwagen.  That is, he claims that there are no physical objects
other than human beings.  However, he comes to this conclusion through
a different path of reasoning.  He claims, roughly, that positing
ordinary things (excluding people) is causally redundant; everything
that ordinary things are said to do can be described in terms of their
parts. (The details are unimportant; what matters is how Merricks
explains why we nonetheless believe that there are ordinary things.)

Despite the fact that Merricks has a different motivation for his
nihilism, we can pose the same question to him as we posed to van
Inwagen.  Why, if there are no chairs, do we believe that there are
chairs?  Happily, Merricks addresses our concern.  Even more happily,
he has a better explanation than van Inwagen.  He explains why, if
nihilism is true, we might nonetheless believe that there are chairs.
But when giving his explanation, he presupposes that {\em
  universalism} is false (see sections \ref{scq-ans} and
\ref{universalism}).  Universalism, like nihilism, seems to contradict
certain of our beliefs, but Merricks' explanation can also explain
why, if universalism is true, we nonetheless hold these certain
beliefs.  Merricks' explanation does not therefore provide nihilism
any advantage over universalism, and the latter is far more plausible.

\subsection{Nearly as good as true}
\label{near}
Merricks claims that `folk' beliefs, such as the belief that there are
chairs, are false, but nonetheless are {\em nearly as good as true}.
What does this mean?

\begin{squote}
People who believe in unicorns [or ghosts] are few and far between.
And those few are generally unjustified.  On the other hand, people
who believe in statues are legion.  And they are generally justified
in so believing.  Given the truth of eliminativism [what I have been
  calling nihilism], we might ask {\em why} the belief in statues is
more common, and more commonly justified, than the belief in unicorns.

The answer is that statue beliefs are nearly as good as true.  For, so
I claim here, {\em atoms arranged statuewise} often play a key role in
producing, and grounding the justification of, the belief that statues
exist.  In general, a false belief's being nearly as good as true
explains how {\em reasonable} people come to hold it.  And, relatedly,
its being nearly as good as true can ground its justification.
Because the belief that unicorns exist is not nearly as good as true
(i.e.\ because there are no things arranged unicornwise), there is no
similar explanation of its production or similar reason to think it is
justified (\citeyear[171--172]{merricks2001a}).
\end{squote}

To say that something is ``nearly as good as true'' seems to be
equivalent to saying that it is `loosely true', or `true for practical
purposes'.  In each case, the proposition in question is false, but it
is somehow close enough to the truth for a given purpose or situation.
For example, suppose we have decided to buy a fake holiday tree for
the holidays this year.  We are looking at a number of different fake
trees.  I point to one and say ``that is a nice tree''.  What I have
said is false; that is not a tree.  It is a fake tree.  But what I
mean---and what my audience recognizes me to mean---is that it is a
nice {\em fake} tree.  We both know that we are looking at fake trees;
there is no point qualifying every use of `tree' with `fake'.  When I
say ``that is a nice tree'', therefore, what I say is quite sufficient
to allow for successful communication, despite being false.  Merricks
claims that propositions expressed by things like ``there are chairs''
are also loosely true.  They are false, but are nonetheless good
enough for certain purposes.

Initially, this seems like a bizarre claim.  After all, Merricks is
claiming that chairs {\em necessarily} do not exist.  According to
Merricks, ``chairs exist'', given its current meaning, could {\em
  never} be true.  If the proposition expressed by ``chairs exist'' is
necessarily false, how could it nonetheless be ``nearly as good as
true''?

\subsection{The conceptual connection}
\label{connection}
Merricks' argument relies on a very close conceptual connection
between ``chair'' and ``chairwise'' (and likewise for all ordinary
terms).  Despite claiming that chairs are impossible, Merricks admits
that we understand perfectly what chairs {\em would} be, if they
existed.  Because we understand the concept of `chair', we can
recognize {\em actually existing} things that are arranged
`chairwise':

\begin{squote}
The folk concept of \emph{statue} plays a role in determining which
atomic arrangements are statuewise. I would even go so far as to say
that if \emph{being arranged statuewise} were not derivative upon
folk-ontological concepts\,\ldots something would be amiss
(\citeyear[8]{merricks2001a}).
\end{squote}

For Merricks, to know what things are actually arranged statue- or
chairwise requires knowing what things would compose a statue or a
chair, if such things were possible:

\begin{squote}
Atoms are \emph{arranged statuewise} if and only if they both have the
properties and also stand in the relations to microscopica upon which,
if statues existed, those atoms' \emph{composing a statue} would
non-trivially supervene (\citeyear[4]{merricks2001a}).
\end{squote}

Merricks' explanation of why we believe that there are chairs relies
on this conceptual connection.  It also is structurally similar to the
explanation we gave in section \ref{intro-beliefs}.  Recall that our
explanation of why we believe that there are chairs (or statues) is
that, first, there are chairs, and, second, we see that there are
chairs (or learn that there are chairs through a similarly reliable
mechanism).

Merricks' definition of `nearly as good as true' allows us to produce
a parallel explanation.  His definition is this:

\begin{squote}
Any folk-ontological claim of the form `F exists' is \emph{nearly as
  good as true} if and only if (i) `F exists' is false and (ii) there
are things arranged F-wise. So, for example, `the statue \emph{David}
exists' is nearly as good as true because (it is false and) there are
some things arranged Davidwise (\citeyear[171]{merricks2001a}).
\end{squote}

We may now say on behalf of Merricks that we believe that there are
chairs (and statues) because, first, there are things arranged
chairwise and, second, we see that there are things arranged
chairwise.

The structure of the two explanations is analogous, but there is an
apparent disanalogy in the content of the two.  The disanalogy does
not favor Merricks.  For it is easy enough to understand why there
being chairs, and us seeing that there are chairs, would cause us to
believe that there are chairs.  But it is less obvious why there being
things arranged chairwise, and us seeing that there are things
arranged chairwise, would cause us to believe {\em not} that there are
things arranged chairwise, but that there are {\em chairs}.

(While it is certainly true that we believe that there are chairs, I
am not sure if all or even most of us {\em also} believe that there
are things arranged chairwise.  Let us suppose for now that we do.)

The close conceptual connection between `chair' and `chairwise' is
very important for Merricks.  It is this {\em connection} that is
doing the explanatory work.  The only thing that can explain why there
being things arranged chairwise would cause us to believe that there
are chairs is this connection between the concepts.  The existence of
things arranged chairwise, and the belief that there are things
arranged chairwise, is supposed to cause the {\em additional} belief
that there are chairs.  How does this happen?

Merricks' answer appears to go something like this: certain
arrangements of things---chairwise arrangements, statuewise
arrangements, and all ordinary arrangements---play important roles in
our lives.  These arrangements of things are of interest to us, so we
have developed words that allow us to refer to them.  For whatever
reason---historical, psychological, or otherwise---we think of each
arrangement as a single thing, rather than as things.  Words like
`chair' and `statue', being singular, reflect this (incorrect) view of
the world.  We are, in a sense, fooled by grammar.

This is more than Merricks says himself.  I have not found a passage
in which he explicitly describes the nature of the conceptual
connection between concepts like `chair' and `chairwise', and explains
why, from our belief that there are things arranged chairwise, we
invariably infer that there are chairs.  But I think he would endorse
something like this.  In the first chapter of his book, he claims that
whether there is a statue or merely things arranged statuewise is not
an empirical question.  He claims that were there not a statue and
merely things arranged chairwise, our ``visual evidence'' would be the
same.  He supports this claim with an analogy:

\begin{squote}
{[}Consider{]} the claim that the atoms arranged
my-neighbour's-dogwise and the-top-half-of-the-tree-in-my-backyardwise
compose an object\ldots it won't do to defend this claim with nothing
more than `I can \emph{just see} the object composed of the atoms
arranged dog-and-treetopwise'. Part of why this won't do, presumably,
is that one's visual evidence would be the same \emph{whether or not}
those atoms composed something (\citeyear[8--9]{merricks2001a}).
\end{squote}

He assumes, of course, that we do not believe that there is a thing
composed of a dog and some of a tree.  Later he suggests that it is
arbitrary to claim that there are statues but not dog-tree things:
``we ought to see that the only difference between arbitrary sums and
statues is a matter of conventional wisdom and local custom''
\citeyearpar[75]{merricks2001a}.  He seems sympathetic to the idea
that the reason we believe that there are statues, and not dog-tree
composites, is due to our conventional speech practices: ``it is at
least somewhat plausible that atoms arranged statuewise are united not
by composing something but, instead and in part, by how we speak and
think'' \citeyearpar[121]{merricks2001a}.

On this picture, whether we see an arrangement of things as composing
an object or not depends more on our own interests than features of
the things themselves.  We have words for chairs and statues because
things arranged chairwise and statuewise interest us.  We don't have a
word for things arranged ``my-neighbor's-dogwise and
the-top-half-of-the-tree-in-my-backyardwise'' because such an
arrangement does not hold much interest for us.  But each of these
arrangements exist, and it seems arbitrary to say that the chairwise
and statuewise arrangements compose chairs and statues while the other
arrangement composes nothing.

Merricks might explain why we believe there are things arranged
my-neighbor's-dogwise and the-top-half-of-the-tree-in-my-backyardwise
thus: there are things arranged my-neighbor's-dogwise and
the-top-half-of-the-tree-in-my-backyardwise, and we see that there are
things so arranged.  This is exactly the same explanation that I would
give.

Now Merricks explains why we believe that there are chairs thus: there
are things arranged chairwise, and we see that there are things
arranged chairwise.  {\em And incidentally, due to our own human
  peculiarities, we have found it convenient to refer to and think
  about things arranged chairwise as if they were `chairs'---single
  unified objects}.

\section{Strange objects}
\label{dogbush}
This is a somewhat plausible explanation of why we would believe that
there are chairs if there were not.  It is certainly much better than
van Inwagen's.  But I think that it fails.  I think that when we look
closer at Merricks' attempts to motivate nihilism, we will see that
they do not support nihilism at all.  If anything they support a
version of {\em universalism}.

Merricks observes that one might object to nihilism simply by saying,
``I just {\em see} the chair!''  He claims that if this objection
moves us, we should think about an analogous objection, which he finds
much less moving:

\begin{squote}
Whether atoms arranged statuewise compose a statue is analogous to
whether atoms arranged my-neighbour's-dogwise and
the-top-half-of-the-tree-in-my-backyardwise compose an object\,\ldots
it would not do to support an affirmative answer to the latter
question simply by saying `I can just see that object'
\citeyearpar[73]{merricks2001a}.
\end{squote}

It does indeed seem initially plausible to say that the top half of a
tree and my neighbor's dog do not compose anything.  But I think this
is ultimately incorrect.

Recall the bliger story that van Inwagen used to motivate his version
of nihilism (section \ref{prop-ont}).  A bliger was supposed to be
four monkeys, an owl, and a sloth, who arrange themselves into a
temporary symbiotic configuration.  Van Inwagen thought we would agree
that bligers did not exist.  He claimed that it is not true that ``six
animals arranged in bliger fashion compose anything, and that is what
I mean to deny when I say that there are no bligers''
\citeyearpar[104]{inwagen1995}.

But as we saw, it is simply false that there are no bligers:

\begin{squote}
\ldots {\em of course} there are bligers in [van Inwagen's] story.
Bligers are what the story is about.  The zoologists do not report
that there are no bligers.  Rather they tell us what a bliger is.
They explain that a bliger is not a single large carnivorous animal
but a transient symbiotic union of six animals
\citep[704]{rosenberg1993}.
\end{squote}

One only reason we might be tempted to say that there are no bligers
is that van Inwagen presents the question in an unintuitive way.  He
asks us if there is some thing, some object, that is composed of the
other six animals.  This gives one the impression that, were there to
be such a thing, it would perhaps be another animal (a seventh); were
there such a thing, it should somehow pop out at us.  But all we see
when we picture the scene are the six animals together, so we feel
that van Inwagen might be right.  There is no {\em other} thing.  But
if we phrase the question differently, things become clearer.  Rather
than ask if there is some thing composed of such and such other
things, we simply ask, ``are there bligers?''  And of course there
are.  Van Inwagen's use of the word `composition' led our intuitions
astray.

Merricks makes the same mistake in his passage above.  Imagine if he
had said, ``Consider five discontinuous islands.  One cannot argue
that they compose some further thing by simply saying `I just see
it!'\,'' If these five islands are an archipelago, then one {\em can}
say ``I just see the archipelago!''  {\em Of course} there are
archipelagos.  They are, as one might put it, `scattered objects'.
The archipelago is made up of a number of separate islands, but it is
nonetheless a thing.  It is an archipelago.  Now let us suppose there
is an archipelago in the Mediterranean Sea (this example is adapted
from \citet{hawthorne2008}).  This archipelago is called the Roman
Archipelago, due to the fact that there are a number of Roman ruins on
one of its islands.  There are several research camps on the islands,
where archaeologists dig for artifacts.  Their researches result in a
surprising discovery: one of the islands {\em is} a Roman ruin.  What
was thought to a rocky and curiously shaped island is in fact a
massive collapsed temple.  Further investigation reveals that another
island is made up of the bones of an extinct sea monster, and another
island is a crashed UFO.

Despite these extraordinary circumstances, it is nonetheless true that
the Roman Archipelago exists.  It just happens to be composed of
several islands, a Roman ruin, a pile of old bones, and an alien
spacecraft.  To say the Roman Archipelago does not exist would entail
that these things are {\em not} sitting in the Mediterranean Sea.  (Of
course I made this story up, so the Roman Archipelago in fact doesn't
exist; but it does in the story.)

If Merricks or someone else asks us ``could scattered islands, Roman
ruins, old bones and alien spacecraft ever compose anything?'' we
should reply ``{\em of course}''.  Now take this example:

\begin{quote}
Pranksters break into a museum to install joke pieces of art.  One one
wall they put up a bathroom mirror and towel ring (complete with
towel).  Under the mirror they put a little sign reading ``Wash your
hands''.  The installation is accepted as art by the gullible curator,
who gets an equally gullible journalist to write about it.  {\em Wash
  Your Hands} quickly becomes a valuable piece of art---valuable
enough that art thieves target it.  They break into the museum in
order to steal {\em Wash Your Hands}, but trip an alarm and are forced
to flee.  All they get away with is the towel.  In the morning the
guards tell the curator that part of {\em Wash Your Hands} is missing.
The curator orders them to remove the rest of the piece and informs
crestfallen visitors that {\em Wash Your Hands} is no longer in the
museum's collection.
\end{quote}

Here, the only point at which is it true to say that {\em Wash Your
  Hands} is not in the museum is when it is finally removed.  Someone
who claimed that it was {\em never} in the museum because it doesn't
exist would be saying something quite clearly false.  Thus if Merricks
asks us ``do mirrors and towels ever compose anything?'' we should say
``{\em of course!}\,''

In these two examples, it is clear that the things in question really
do exist.  Nobody will deny that there are archipelagos and works of
art without having first been moved by a philosophical argument.  But
it may be that people {\em will} deny that there are things composed
of the tops of trees and dogs, even before hearing an argument.

Call the things composed of dogs and treetops `dogbushes'.  For
example, in a park that contains one tree and one dog, there is also
one dogbush.  Is it {\em obvious} that there are dogbushes?  Is it
just as obvious as that there are archipelagos and chairs and pieces
of crappy modern art like the {\em Wash Your Hands}?  If not, why?
What is the difference between things like archipelagos and things
like dogbushes?

One obvious difference is that things like archipelagos interest us.
I argued above that Merricks motivates his nihilism by drawing our
attention to the role of tradition and convention in our talk.  We
have a word for archipelagos because they {\em matter} to us.  We
don't have a word for dogbushes because they {\em don't} matter.
Merricks argued, in effect, that since we are not inclined to say that
there are dogbushes, and since there is no metaphysical difference
between dogbushes and archipelagos, we should not be inclined to say
that there are archipelagos.

But we can reverse Merricks' argument.  Since we {\em are} inclined to
say that there are archipelagos, and since there is no metaphysical
difference between archipelagos and dogbushes, we should not be
inclined to deny that there are dogbushes.

\section{Universalism}
\label{universalism}
I claimed in section \ref{connection} that Merricks' explanation of
why we believe that there are chairs is something like this: there are
things arranged chairwise, and we see that there are things arranged
chairwise.  {\em And incidentally, due to our own human peculiarities,
  we have found it convenient to refer to and think about things
  arranged chairwise as if they were `chairs'---single unified
  objects}.  I attributed to Merricks the idea that just because
things arranged chairwise interest us, we should not therefore suppose
that there are chairs.  What interests us should not be a guide to
what exists.  But now there is an obvious counter against this move.
Just because dogbushes do {\em not} interest us, we should not
therefore suppose that there are not dogbushes.  In this spirit,
Judith Thomson suggests that we ``think of Reality as like an
over-crowded attic, some of its contents interesting, and most merely
junk.  There is no need to deny the junk; we can simply leave it to
gather dust'' \citep[167]{thomson1998a}.  This is the intuition behind
{\em universalism}, one of the answers to the Special Composition
Question (section \ref{scq}):

\begin{description}
\item[Universalism] Necessarily, for any $x$s, there is an object
  composed of the $x$s if and only if no two of the $x$s overlap
  \citep[227]{markosian1998a}.
\end{description}

The following argument suggests itself:

\begin{enumerate}[ref=(\arabic*)]
  \item Chairs exist. \label{u-1}
  \item Things that do not differ from chairs (or archipelagos, or
    works of art) in metaphysically significant ways also exist.
  \item Dogbushes do not differ from chairs in metaphysically
    significant ways.
  \item {\em Therefore}, dogbushes exist. \label{u-4}
\end{enumerate}

I imagine that Merricks would deny the conclusion \ref{u-4} and so, by
{\em modus tollens}, deny one or more premises (and we have seen that
he denies \ref{u-1}).  But I affirm the premises and so, by {\em modus
  ponens}, affirm the conclusion.

This argument helps us see what is wrong with Ned Markosian's response
to the Special Composition Question.  Markosian defends what he calls
`brutal composition'.  The thesis of brutal composition is that, while
there is indeed no ``no true, non-trivial, and finitely long answer to
[the Special Composition Question]''
\citeyearpar[214]{markosian1998a}, this is not because we should refer
questions of composition to the empirical sciences.  Rather, whether
or not some things compose another is simply a {\em brute fact}.

This is a clever reply, but whether true or not I do not think it does
the work that Markosian expects it to.  He presents his theory as
``consistent with standard, pre-philosophical intuitions about the
universe's composite objects'' \citeyearpar[211]{markosian1998a}.  But
his theory will only be consistent with such intuitions if, first, it
is a brute fact that all of the things we ordinarily take to exist
(tables, chairs, etc.) do in fact exist, and, second, that it is a
brute fact that the things that we don't take to exist don't in fact
exist.  But why should we expect there to be a {\em metaphysical}
difference between things that interest us and things that don't?  The
chance that the brute facts of composition happen to line up with our
(or Markosian's) intuitions seems to be incredibly low.

But accepting the above argument for universalism has some strange
consequences that are not immediately apparent.  Ned Markosian brings
out such a consequence in this passage:

\begin{squote}
There is what seems to me a fatal objection to Universalism:
Universalism entails that there are far more composite objects than
common sense intuitions can allow.  To give just one example,
Universalism entails that the following sentence is true:\,\ldots
There is an object composed of (i) London Bridge, (ii) a certain
sub-atomic particle located far beneath the surface of the moon, and
(iii) Cal Ripken, Jr.  My intuitions tell me that there is no such
object, and I suspect that the intuitions of the man on the street
would agree with mine on this point \citeyearpar[228]{markosian1998a}.
\end{squote}

If this is a compelling objection, it is because such an object (call
it `Lumpkin') does not interest us in the least.  As Merricks observed
(see section \ref{connection}), the things that we believe to exist
are largely the things that interest us.  We believe that there are
archipelagos; van Inwagen's imaginary farmers believe that there are
bligers.  If we do not believe that there are dogbushes or Lumpkins,
this may be because they do not interest us.

Suppose Markosian wrote this instead:

\begin{squote}
Universalism entails that the following sentence is true:\,\ldots
There is an object composed of (i) an island, (ii) a Roman ruin, and
(iii) the bones of a sea monster.
\end{squote}

But this is just the Roman Archipelago I mentioned in section
\ref{dogbush}.  We are (or should be) happy to admit that it exists.
If the Roman Archipelago exists, and if it does not differ from the
Lumpkin in any metaphysically significant ways, why shouldn't we admit
that the Lumpkin exists?  Of course we don't {\em care} about the
Lumpkin.  We have no need to refer to it; it doesn't matter to our
lives.  But why should we expect---as Markosian seems to---that our
intuitions should perfectly track what exists?

\section{Lessons, part 2}
\label{lessons-m}
What we have learned from examining Merricks' arguments is not that
there are no chairs.  What we have learned is that since there {\em
  are} chairs, and since dogbushes do not differ from chairs in
metaphysically significant ways, there are therefore also dogbushes.

If we agree that there are chairs and archipelagos and dogbushes and
the Lumpkin, however, new questions arise. For example: What are the
parts of a chair?  How do they compose the chair?  Do the parts of the
chair change over time?  How?  We will address these questions in the
next section.  

\ifstandalone
\end{spacing}
\bibliography{everything}
\bibliographystyle{ChicagoReedweb}
\fi
\end{document}


\chapter{What are the parts of a chair?}
\chapterpig{What are the parts of a chair?}
\documentclass[11pt]{article}
\usepackage{standalone} \newif\ifstandlone \standalonetrue
\usepackage[left=1.75in, right=1.75in, top=1.25in, bottom=1.25in]{geometry}
\geometry{letterpaper}
\usepackage{graphicx}
\usepackage{enumitem}
\usepackage{amssymb}
\usepackage{amsmath}
\usepackage{epstopdf}
\usepackage{verbatim}
\usepackage{setspace}
\usepackage{natbib}
\setcitestyle{aysep={}}
\usepackage{url}
\usepackage{hyperref}
\synctex=1

\DeclareSymbolFont{symbolsC}{U}{txsyc}{m}{n}
\DeclareMathSymbol{\strictif}{\mathrel}{symbolsC}{74}
\DeclareMathSymbol{\boxright}{\mathrel}{symbolsC}{128}

\newenvironment{squote}{%
\begin{spacing}{1}
\begin{list}{}{%
\setlength{\labelwidth}{0pt}%
\rightmargin\leftmargin%
}
\item\relax
}{%
\end{list}%
\end{spacing}
}

\title{How do chairs persist over time?}
\author{Alexander A. Dunn}
\begin{document}
\ifstandalone
\maketitle
\begin{spacing}{1.5}
\fi

\label{parts}

In the previous section I argued that not only do `material things'
like chairs exist, but other kinds of things---like teams and
families---exist as well.  In this section I will adopt Kit Fine's
theory of parthood, which claims that there are many different ways of
being a part of something.  However, I will argue that the way in
which people are members of groups is the way that that things are
parts of sets, and that teams and other groups are actually identical
with sets.  Since sets do not change their parts, which set is
identical with a group at a time is variable and dictated by {\em
  convention}.  I will argue that the situation is analogous with
material things: which `mereological sum' is identical with a chair at
a time is a conventional matter.

\section{Parthood and the language of composition}
\label{parthood}
Things like teams, crews, and families are indeed {\em things}.  They
are not disguised references to plurals, nor are they ``mere
collections'' \citep[29]{inwagen2009}.  Moreover, things like teams
are things with {\em parts}.  The rugby players are each {\em part} of
the Reed College women's rugby team.  The team is made up of---it is
composed of---the players.

When I say that the players are part of the team, or that the
crewmembers are part of the crew, or that I am part of my family, is
that use of `part' the same as when I say that the tree is part of the
dogbush, or that the seat is part of the chair?  Are {\em any} of
these uses of `part' the same?

Technical notions of composition are often defined in terms of
parthood.  How I am using the term `part' will therefore influence how
I construct formal equivalences for propositions like ``there are
chairs'', ``there are dogbushes'', and ``there are teams''.  Is the
appropriate formalization for each of these the same?  Can chairs,
dogbushes, and teams each `fit in' to the schema ``there is an $x$
such that it is composed of the $y$s''?  Or is `composition' one of
many {\em operations} that `produce' things?

\subsection{Van Inwagen's notion of parthood}
\label{van-part}
Van Inwagen defines his technical notion of composition (see section
\ref{scq}) in terms of a largely intuitive notion of parthood.  Van
Inwagen's interest, however, is restricted to `material' objects
(objects made exclusively of quarks and protons, or whatever the basic
atoms of the physical world turn out to be).  While he goes on to use
`part' only in reference to material objects, he recognizes that the
term has much wider application:

\begin{squote}
Parthood will occupy a central place in the present study of material
objects.  It is therefore worth noting that the word `part' is applied
to many things besides material objects.  We have already noted that
submicroscopic objects like quarks and protons are at least not clear
cases of material objects; nevertheless, every material object would
seem pretty clearly to have quarks and protons as \emph{parts}, and,
it would seem, in exactly the same sense of \emph{part} as that in
which a paradigmatic material object might have another paradigmatic
material object as a part.  A ``part,'' therefore, need not be a thing
that is clearly a material object.  Moreover, the word `part' is
applied to things that are clearly \emph{not} material objects---or at
least it is on the assumption that these things really exist and that
apparent reference to them is not a mere manner of speaking.  A stanza
is a part of a poem; Botvinnik was in trouble for part of the game;
the part of the curve that lies below the x-axis contains two minima;
parts of his story are hard to believe\,\ldots\,such examples can be
multiplied indefinitely.  Does this word `part' mean the same thing
when we speak of parts of cats, parts of poems, parts of games, parts
of curves, and parts of stories \citeyearpar[18--19]{inwagen1995}?
\end{squote} 

Van Inwagen suggests that `part' does have a number of different
meanings.  Later he says that ``there is one relation called
`parthood' whose field comprises material objects\,\ldots\,another
relation called `parthood' defined on events, another still defined on
stories, yet another defined on curves, and so on''
\citeyearpar[19]{inwagen1995}.

One reason why we might resist this conclusion is that it appears to
rule out the silly example of the wish sandwich.  A wish sandwich,
recall, is the kind of sandwich where you have two slices of bread and
wish you had some meat.  The slices of bread and the wish for meat are
all parts of the wish sandwich.  But if they are parts of the sandwich
in different ways, then in what sense did I use `part' in the
preceding sentence?  If it was part$_b$---the parthood relation for
foodstuffs---then the sentence was false, because a wish does not
partake in that sort of relation.  If the relation was that of
parthood$_w$---parthood for wishes---then the sentence would again be
false, because {\em that} relation does not govern foodstuffs.  I
could instead say ``the slices of bread are part$_b$ of the sandwich
and the wish is part$_w$ of the sandwich'', but it still seems to me
that the original sentence is {\em true}.  Might this suggest that
there is really just one parthood relation that both foodstuffs and
wishes partake in?

Moreover, our modified sentence still faces a difficulty.  For the
notion of composition is generally defined in terms of parthood.
Since van Inwagen's technical definition of composition is given in
terms of his notion of parthood, `composition' for van Inwagen can
only apply to things whose parts are all material things.  Van
Inwagen's notion of composition cannot make sense of the wish
sandwich, or any thing with both material and non-material parts.
(Van Inwagen does not see this as a disadvantage; he finds mysterious
the idea that there could be something composed of ``you and I and the
number two'' \citeyearpar[20]{inwagen1995}.)

We could, perhaps, define `composition' in terms of not just one
parthood relation but all of them.  Composition would take into
account all possible ways there are of being a part.  Kit Fine has
proposed a theory of parthood that takes seriously the possibility
that there are a plurality of different parthood relations.

\section{Fine's theory of part}
\label{fine}
Fine agrees with van Inwagen that the notion of parthood should not be
reserved only for material things:

\begin{squote}
Philosophers have often supposed the notion of part only has proper
application to material things or the like and that its application
to abstract objects such as sets or properties is somehow improper
and not sanctioned by ordinary use.  But I suspect that this is
something of a philosopher’s myth.  We happily talk of a sentence
being composed of words and of the words being composed of
letters---and not just the sentence and work tokens, mind, but also
the types.  And similarly, a symphony (and not just its performance)
will be composed of movements, a play of acts, a proof of steps.  I
wonder how many of these philosophers have said such things as ``this
paper is in three parts.''  When they have, then I very much doubt
that they would have any inclination, as ordinary speakers of the
language, to add ``but not, of course, in a strict or literal sense of
the term''; and the intended reference here is not primarily---or
perhaps not at all---to the tokens of the paper but to the type of
which they are the tokens.  The evidence concerning our ordinary talk
of part is mixed and complicated, but it does not seem especially to
favor taking material things to be the only true relata of the
relation \citeyearpar[561]{fine2010}.
\end{squote}

But even if one accepts the idea that there are things other than
`material things' that have parts, one might object that they are
still all parts in the same sense.  This might be the parthood
relation of classical mereology, or it might be some other, general
relation.  Moreover, one who maintained the univocality of parthood
can still concede that there are different ways of being a part.  But
for the believer in the univocality of parthood---the `monist'---these
are only {\em derivative} kinds of parthood.  For example, for any
given mereological sum, there are bigger and smaller parts of it. But
these are bigger and smaller parts of the same {\em kind}.  The
pluralist goes further and claims that there are parts of different
kinds. These different kinds are not derivative but \emph{basic}; they
are ``not definable in terms of other ways of being a part''
\citep[561]{fine2010}.  Fine gives a number of reasons to think that
there might be different basic parthood relations:

\begin{squote}
Now, on the face of it, there would appear to be a wide variety of
basic ways in which one object can be a part of another.  The letter
`n' would appear to be a part of the expression `no', for example, and
a particular pint of milk part of a particular quart; and if these two
relations of part are not themselves basic (perhaps through being
restricted to expressions or quantities), there would appear to be
basic relations of part that hold between `n' and `no' or the pint and
the quart.  It is also plausible that the way in which `n' is a part
of `no' is different from the way in which the pint is a part of the
quart.  For if the two ways were the same, then how could it be that
two pints were only capable of composing a single quart, while the two
letters `n' and `o' were capable of composing two expressions, `no'
and `on' \citeyearpar[562]{fine2010}?
\end{squote}

The parthood relation for sets is again different.  The set containing
the only the letters `n' and `o' has the letters as parts.  When the
letters are parts of a set, their order is irrelevant, but when the
letters are parts of a word, order matters; hence `no' and `on'.  The
parthood relation for sets is also different from the parthood
relation for quantities (of milk):

\begin{squote}
If four quarts compose a gallon the pints which compose the quarts
will compose the gallon in the same way in which they compose the
quarts, whereas, if four sets compose a further set the members of the
sets will not compose the further set in the same way in which they
compose the component sets.  Thus we would now appear to have three
different basic ways in which one object can be a part of another
(pint/gallon, letter/word, and member/set); and once these cases have
been granted, it is plausible that there will be many more
\citeyearpar[562]{fine2010}.
\end{squote}

One might, of course, refuse to grant these cases.  But one would have
to refuse them {\em all}; for if it can be established that there are
even two different (basic) ways of being a part, then the pluralist
position is established.  Once it is established that there are at
least two ways of being a part, it becomes much more plausible that
there might be three ways, or more.  Fine therefore attempts to
motivate the idea that the members of a set are, quite literally,
parts of the set.

\subsection{Parts of sets}
\label{sets}
The first objection is that while parthood is supposed to be
transitive, the membership relation of sets is not.  The letter `n' is
a member of the set \{`n',\{`n',`o'\}\}, but `o' is not.  The
objection claims that sets have {\em members}, not parts, and that
Fine has confused the two.

But while it is true that the membership relation is not the parthood
relation, this is no reason to think that sets do not have parts.  A
given set will have certain members---the $x$s---and certain
parts---the $y$s---and only sometimes will the $x$s and the $y$s be
the very same things.  The set \{`n',\{`n',`o'\}\} has two members
but three parts.  The parthood relation for sets can even be defined
in set-theoretic terms:

\begin{squote}
It may well be thought that the way in which a member is a part of a
set is given, not by the membership relation itself, but by the
ancestral of the membership relation, where this is the relation that
holds between $x$ and $y$ when $x$ is a member of $y$ or a member of a
member of $y$ or a member of a member of a member of $y$, and so on
\citep[563]{fine2010}.
\end{squote}

A second objection is that talk of parthood in connection with things
like sets is somehow metaphorical or non-literal.  We saw above that
van Inwagen admits that many different things are said to have parts.
However, he qualifies this in two ways.  First, he seems to have
doubts (or at least is sympathetic with those who have doubts) as to
whether the non-material things that are said to have parts really
exist:

\begin{squote}
The word `part' is applied to things that are clearly \emph{not}
material objects---or at least it is on the assumption that these
things really exist and that apparent reference to them is not a mere
manner of speaking \citep[19]{inwagen1995}.
\end{squote}

If there are no such things as tennis matches or poems or papers, then
of course they do not have parts.  But I think it is obviously true
that there are such things.  This being so, what does it mean to say
that they have parts?  This is where van Inwagen's second
qualification comes in.  For he suggests not only that the `parts' of
tennis matches and poems are parts in a different way than are the
parts of a table, but that these different relations of parthood are
only tenuously connected.  Van Inwagen says that the various relations
of parthood (if such there be) are connected only by the ``unity of
analogy'' \citeyearpar[19]{inwagen1995}.  If the only similarity
between the parthood relation for poems and the parthood relation for
chairs is that they share the `analogy' of parthood, then is there
anything important or interesting about `parts' of poems?  Is the
parthood relation for sets likewise only interesting because of the
analogy with the parthood relation for chairs?

At least in the case of parthood for sets, the notion does not appear
to be wholly metaphorical:

\begin{squote}
In the case of set-membership, there would appear to be nothing that
might plausibly be taken to indicate that the talk of part-whole is
not to be taken literally. A set is indeed composed of or built up
from its members, and we should add that we may meaningfully
talk---and in the intended way---of \emph{replacing} one member of a
set with another.  Thus Aristotle in the set \{Plato, Aristotle\} may
be replaced with Socrates to obtain the set \{Plato, Socrates\}, with
the given set becoming a different set from what it was. In the case
of sets, our conception of members as parts seems to extend all the
way \citep[564]{fine2010}.
\end{squote}

But the second worry raised by van Inwagen remains.  Why should we
think that there is any {\em real} similarity between these different
parthood relations, other than the fact that we call them all
`parthood'?

\subsection{Operationalism}
\label{operation}
Fine's theory of {\em operationalism} helps answer this worry.
Various {\em operations} produce different things---mereological
summation produces mereological sums or fusions, the set-builder
produces sets, and so forth.  Parts are therefore {\em things} that
have been `combined', through one or more such operations, into a
single {\em thing}.  What is common to all parthood relations is that
from each set of parts is produced a {\em whole}.  This may simply be
metaphorical, but it is nonetheless an accurate description of the
result of the composition operator.  From parts (letters, atoms) we
can `make' something new (a word, a set, a chair).  What ties together
all the ways of being a part is that they are involved in an operation
that produces a single thing from a number of things:

\begin{squote}
In formulating the principles of mereology, it has been usual to take
the relation of part-whole or some associated relation (such as
overlap) as primitive.  But I believe that, in formulating a more
general theory, it is important to take the operation of composition
as primitive rather than the more familiar relation of part-whole.  In
the case of classical mereology, the operation of composition will
take some objects into the sum or fusion of those objects, while, in
the set-theoretic case, it will take some objects into the set of
those objects; and, in general, the operation of composition will be
the characteristic means (summation, set-builder, and so on) by which
a given kind of whole is formed from its parts \citep[565]{fine2010}.
\end{squote}

Each way of being a part can then be defined in terms of the related
composition operation:

\begin{squote}
Once given a compositional operation, a corresponding relation of part
may be defined in two steps.  We say first that $x$ is a component of
$y$ if $y$ is the result of applying $\sum$ to $x$ or to $x$ and some
other objects.  In other words, $y$ should be of the form $\sum
(x_{1}, x_{2}, \mathellipsis )$, where at least one of $x_1$, $x_2,
\mathellipsis$ is $x$.  Thus when $\sum$ is mereological summation the
components of an object will be mere parts, and where $\sum$ is the
set-builder the components of an object will be its members.  We may
then define $x$ to be a part of $y$ if there is a sequence of objects
$x_1$, $x_2, \mathellipsis x_n$, $n$ \textgreater{} $0$, for which $x
= x_1$, $y = x_n$, and $x_i$ is a component of $x_{i+1}$ for $i = 1$,
$2, \mathellipsis, n-1$. The parts of an object are the object itself,
or its components, or the components of the components, and so on
\citep[567--568]{fine2010}.
\end{squote}

The parthood relation for summation can therefore be seen to exhibit
reflexivity, transitivity and anti-symmetry:

\begin{description}
\item[Reflexivity] Each object is a part of itself.
\item[Transitivity] If $x$ is a part of $y$ and $y$ of $z$, then $x$
  is a part of $z$.
\item[Anti-symmetry] $x$ is a part of $y$ and $y$ of $x$ only when $x
  = y$ \citep[568]{fine2010}.
\end{description}

But not all definitions of parthood that issue from a composition
operator will exhibit these features:

\begin{squote}
When the underlying operation is summation, each object will be a part
of itself, since the unit sum of any object is the object itself, but
when the underlying operation is the set-builder, no object will be a
part of itself, since no object is ever an ancestral member of itself
\citep[569]{fine2010}.
\end{squote}

\subsection{Principles}
\label{principle}
Each composition operation will, according to Fine, be governed by
various principles:

\begin{squote}
I believe that the principles governing the basic forms of composition
will conform to a general template.  Variations in the principles for
the different forms of composition will then arise from variations in
how the template is to be filled in.  The template will comprise two
broad categories of principle---the {\em formal} and the {\em
  material} (though not quite in the sense of Husserl).  Among the
formal principles, we may distinguish between those that provide
conditions of application for the operation and those that provide
identity conditions; among the material principles, we may distinguish
between those that provide conditions for the presence of a whole (in
space and time or at a world) and those that specify the descriptive
character of the whole.  The presence conditions, in their turn, may
concern either the existence of the whole or its extension
\citeyearpar[569--570]{fine2010}.
\end{squote}

\paragraph{Formal principles}
The formal principles govern when composition occurs and when two
products of a composition operation are identical:

\begin{description}
  \item[Application] The application conditions are ``the conditions
    under which there are wholes of a given sort---which, on the
    operational approach, is a matter of stating when the result of
    applying the compositional operation to various objects will be
    defined'' \citep[570]{fine2010}.  For the summation operation, the
    application conditions are very lax: for any $n$ physical objects
    (where $n$ \textgreater{} 1), there is a thing composed of those
    objects.  The application conditions for the set-builder will be
    more limited; and other operations will be more restricted
    still. \label{fine-app}
  \item[Identity] Setting out identity conditions on Fine's
    operational approach ``is a matter of stating when a whole formed
    in one way by means of the compositional operation is the same as
    a given object or a whole that has been formed in some other way''
    \citeyearpar[570]{fine2010}.  For example, suppose $y$ is the
    result of applying the summation operator to $x$, and suppose
    $y^\prime$ is the result of applying the set-builder to $x$.  If
    $y \neq y^\prime$, it will be because of some difference in the
    respective operations that produced these composites.  As we will
    see below, the summation operator is defined such that its
    application to a single thing ($x$) produces that very thing.  The
    set-builder, however, produces the singleton of $x$, which is not
    $x$.  And whereas applying the set-builder to nothing produces the
    null set, the result of applying the summation operator to nothing
    might be undefined---nothing would result.
\end{description}

\paragraph{Material principles}
As above, there are two subcategories:

\begin{description}
  \item[Presence] Fine claims that ``there are two fundamentally
    different ways in which an object might be present in space or
    time; it may \emph{exist} in space or time, or it may be
    \emph{extended} (or \emph{located}) in space or time. Thus a
    material thing will exist in time but be extended in space while
    an event will be extended in both space and time''
    \citeyearpar[570]{fine2010}.  Whether or not this is actually true
    seems to depend on whether a `three-dimensional' theory of
    persistence is correct.  However, that question will remain unasked.
  \item[Character] ``The character conditions will tend to have a much
    more ad hoc character than the other conditions that we have
    considered. The color of a house, for example, is the color of its
    siding; the color of an egg, the color of its shell; the color of
    a pencil, the color of its lead. In the case of the `intrinsic'
    character of a thing---such as its mass or color---the character
    of the whole will be some sort of function of the character of the
    parts. But the function in question will vary from case to case''
    \citep[571]{fine2010}.  What we decide is the color of a house
    will turn on our concept `house', rather than anything about the
    house itself (anything beyond its having that color to {\em some}
    extent).  When the composition operation is summation, interesting
    questions also arise about weight; is the weight of a sum equal to
    the combined weights of the parts?  This seems intuitively so, but
    if someone's parts weight 150 pounds, and therefore they weight
    150, it does not follow that when they (and their parts) stand on
    a scale, the scale reads 300 pounds.  Figuring out what to say in
    these cases may ultimately feel ad hoc.
\end{description}

\subsection{Fine's pluralist account of classical mereology}
\label{classical}
Of the principles sketched above, Fine gives most attention to the
identity conditions for composition operations.  The composition
operation used as a paradigm is the summation operation of classical
mereology.  Fine's exposition of identity conditions for sums relies
on the notion of `regularity':

\begin{squote}
Call an identity condition $s = t$ {\em regular} if the variables
appearing in $s$ and in $t$ are the same.  Thus $\sum (x, y) = \sum
(y, x)$ is regular while $\sum (x, y) = x$ is not
\citeyearpar[572]{fine2010}.
\end{squote}

With this notion in hand, Fine proposes this condition for identity of
sums:

\begin{description}
  \item[Summative Identity] $s = t$ whenever `$s = t$' is a regular
    identity \citeyearpar[572]{fine2010}.
\end{description}

One particularly interesting aspect of this condition is that it
entails four more principles of the summation operation:

\begin{description}
  \item[Absorption] $\sum (\mathellipsis, x, x, \mathellipsis,
    \mathellipsis, y, y, \mathellipsis, \mathellipsis = \sum (
    \mathellipsis, x, \mathellipsis, y, \mathellipsis )$;
\item[Collapse] $\sum (x) = x$;
\item[Leveling] $\sum (\mathellipsis, \sum (x, y, z, \mathellipsis ),
  \mathellipsis, \sum (u, v, w, \mathellipsis ), \mathellipsis ) \\ =
  \sum (\mathellipsis, x, y, z, \mathellipsis, \mathellipsis, u, v, w,
  \mathellipsis, \mathellipsis )$;
\item[Permutation] $\sum (x, y, z, \mathellipsis ) = \sum (y, z, x,
  \mathellipsis )$ (and similarly for all other permutations)
  \citep[573]{fine2010}.
\end{description}

We can define other compositional identity criteria (e.g., sequences)
in terms of which of these principles apply to their compositional
operation.  But we may also devise new principles by which we may then
define new types of composition:

\begin{squote}
We should note that there would appear to be no good reason to require
that the defining principles for the various operations should be
limited to the particular principles (C [collapse], L [leveling], A
[absorption], and P [permutation]) that we used in characterizing
sums; for any set of regular identities would appear to be equally
well suited to defining a basic form of composition, so long as they
conform to Anti-cyclicity.  Indeed, I would conjecture that any such
set of principles in fact will correspond to a form of composition and
a corresponding form of whole.  How the resulting forms of composition
and whole might be organized is an interesting question, but it should
be apparent that the approach will lead to an infinitude of forms of
composition, each differing from one another in how exactly the
identity of the resulting wholes is to be
determined. \citep[575--576]{fine2010}.
\end{squote}

It is at this point that the importance of Fine's theory becomes
obvious.  Above I stressed that things like teams and families are
really {\em things}; moreover I made this claim as part of an attempt
to motivate a sort of universalistic outlook on metaphysics.  I argued
that the term `composition' was potentially misleading, but that it
was nevertheless correct to say that things like dogbushes, wish
sandwiches, and teams are composed of their parts.  But now it is
apparent that `composition' will mean something different when applied
to each of these things.  Each thing will be the product of a
different composition operation.

Fine's theory reveals new {\em kinds} of universalism.  One might be
committed to the existence of dogbushes---and so to unrestricted
mereological composition---but deny the existence of teams, groups,
crews, and families.  Or one might defend unrestricted composition of
groups while claiming a restriction on mereological composition.

Below I will look at how a definition of the composition operator for
groups might be formulated.  But I will first return to Fine's theory.

\subsection{Hybrid parts}
\label{hybrid}
Above, we saw that one objection to the idea of sets having parts was
that parthood is transitive and set-membership is not.  Moreover it
was supposed (rather plausibly) that the only reason we think that
sets have parts is {\em because} they have members; it is the members
of a set that we are tempted to call parts.  But, the objection goes,
it is a mistake to think of members as part.  I am the only member of
my singleton (the singleton of $x$ is the set resulting from applying
the set-builder to $x$ alone).  My hand, for instance, is not a member
of my singleton.  But my hand is a part of me.  If I was a part of my
singleton, then---because parthood is transitive---my hand would be a
part of my singleton.  And if that means that my hand is a {\em
  member} of my singleton, that is clearly wrong.

Fine points out, of course, that the objection makes the mistake of
supposing that something (me, my hand) can be a part in only one way
(in this case, through set-membership).  Once we recognize that there
are a plurality of ways of being a part, it becomes clear that my hand
is part of the set in one way, but not in another:

\begin{squote}
Given the specific relations of part, we may derive various {\em
  hybrid} relations of part.  Suppose, for example, that we are given
the relations of set-theoretic and mereological part---which we may
designate as \textepsilon -part and $m$-part. We may then take one
object to be an \textepsilon ,$m$-part of another if it is an
\textepsilon -part or an $m$-part or an $m$-part of an \textepsilon
-part or an \textepsilon -part of an $m$-part, or an $m$-part of an
\textepsilon -part of an $m$-part, and so on. More generally, if $K$
is a family of specific ways of being a part, we may take an object to
be a {\em K-part} of another if $x$ and $y$ can be linked by
relationships of $k$-part for $k$ in $K$ \citep[579]{fine2010}.
\end{squote}

My hand is a \textepsilon ,$m$-part of my singleton, but not a
\textepsilon -part.

By conjoining every way of being a part, we arrive at the most general
notion of part:

\begin{squote}
Among the hybrid relations of part, of special interest is the
relation of $K$-part where $K$ is the family of {\em all} the specific
ways of being a part.  This is the relation of $K$-part that holds
between two objects when they may be linked by relationships of
$k$-part without restriction on $k$.  We might call it the {\em
  general} relation of part, and it is a relation that holds between
$x$ and $y$ whenever $x$ is in any way whatever a part of $y$
\citep[580]{fine2010}.
\end{squote}

\subsection{Generating kinds}
\label{generate}
On this theory, what kind a thing is depends on what operation
produced it.  If a chair or a dogbush is a mereological sum, then this
is because they are produced by the summation operation.  The Dunn
family is `produced' by the family operation.  Groups are produced by
the group operation (see section \ref{group}).

But there is a difficulty to be avoided here.  As we saw in section
\ref{classical}, the mereological sum of a single thing $x$ is just
$x$.  Therefore there is a sense in which every physical thing,
including every simple, is a mereological sum, for the application of
the summation operation would just produce that thing.  To avoid this
consequence Fine introduces the notion of a {\em generative}
application of an operation:

\begin{squote}
We might say that the application $y = \Gamma (x_1, x_2, x_3,
\mathellipsis )$ of an operation $\Gamma$ is {\em generative} if there
is an explanation of the identity of $y$ as $\Gamma (x_1, x_2, x_3,
\mathellipsis )$; and we might say that the operation $\Gamma$ is
itself {\em generative} if it permits a generative application. Thus
both the set-builder and the operation of predication will be
generative in this sense \citeyearpar[582]{fine2010}.
\end{squote}

Whether or not the summation operation is generative depends on the
things it is being applied to.  When summing a dog and a tree, it is
generative; when summing a dog by itself, it is not.

For any operation, there will be things it applies to that it cannot
produce.  The summation operator fuses simples, but cannot produce
them; the set-builder combines many things that it cannot produce
(like letters).  For any given operation, there is a `level 0'
consisting of the things that the operator itself cannot produce:

\begin{squote}
We suppose that certain objects are simply given.  These are the
objects whose identity does not require an explanation in terms of
$\Gamma$.  Thus, when $\Gamma$ is the set-builder, they are the
objects that are not sets and, when $\Gamma$ is summation, they are
the objects that are not sums or, rather, the objects that do not need
to be seen as sums.

We now `generate' objects in stages.  At stage 0 are the givens; at
stage 1, we add the objects that result from a single application of
the generative operation $\Gamma$ to the givens \citep[583]{fine2010}.
\end{squote}

An application can now be identified as generative in a strong or a
weak sense:

\begin{description}
  \item[Strong generative application] Also called `strict' by Fine, a
    ``[strong] generative application of $\Gamma$ to the objects $x_1,
    x_2, \mathellipsis$ can now be defined as one in which $y = \Gamma
    (x_1, x_2, \mathellipsis )$ is of a higher level than each of
    $x_1, x_2, \mathellipsis$'' \citeyearpar[584]{fine2010}.  For
    example, summing the simples $x$ and $y$ to produce the fusion $z$
    would be a strong generative application of the summation
    operator; the simples are level 0 and $z$ is level 1.  Summing two
    composites, or a composite and a simple, would not be strongly
    generative; one or both of the parts would be the same level (1)
    as the product.
  \item[Weak generative application] To illuminate this notion Fine
    introduces another, that of a {\em putative generative
      application}: ``Let us say, in the first place, that $y = \Gamma
    (x_1, x_2, \mathellipsis )$ is a putative generative application
    of $\Gamma$ if $y$ is of a higher or of the same level as each of
    $x_1, x_2, \mathellipsis$.  This gives us the notions of a
    putative prior component and of a putative prior in the usual way.
    We now say that the application $y = \Gamma (x_1, x_2,
    \mathellipsis )$ of $\Gamma$ is a {\em weak} generative
    application if it is the putative generative application and if
    $y$ is not putatively prior to any of $x1, x2, \mathellipsis$.  We
    can get from $x_1, x_2, \mathellipsis$ to $y$ without an ascent in
    level but not from $y$ to any of $x_1, x_2, \mathellipsis$''
    \citeyearpar[584]{fine2010}.
\end{description}

Applying the summation operator to a simple is neither strongly nor
weakly generative.  It is not strongly generative because the result
is a simple, which is at level 0---the same level as its part
(itself).  It is not weakly generative because the result of the
operation is putatively prior to its parts.

\section{Groups and sets}
\label{group}
There are a number of reasons to think that, along with sets, there
are also groups.  The most salient reason is the apparent fact that
groups can change their parts, while sets cannot.  If groups are
distinct from sets, we should be able to define a group-builder that
is distinct from a set-builder.  However, defining a group-builder
that allows a group to change its parts requires that we assume {\em
  eternalism}.  If we wish to remain neutral on the
presentism/eternalism debate, we have no means of distinguishing
groups from sets.

\subsection{Motivating groups}
The set-builder, on Fine's theory, takes things (such as jurists) and
produces a set composed of them.  There is a set $S$ composed of the
2004 Supreme Court justices:

\begin{squote}
$S = \sum _{\in}$ (Rehnquist, Stevens, O'Connor, Scalia, Kennedy,
  Souter, Thomas, \\ Ginsburg, Breyer) $ = $ \{Rehnquist, Stevens,
  O'Connor, Scalia, Kennedy, \\ Souter, Thomas, Ginsburg, Breyer\}
\end{squote}

But some claim that, in addition to sets, there are also {\em groups}.
There may, in addition to the set \{Rehnquist, Stevens, O'Connor,
Scalia, Kennedy, Souter, Thomas, Ginsburg, Breyer\}, a group
containing the very same people.

One might ask why this is necessary.  Groups, it might be objected,
are really no different than sets.  When we speak of a group of
people, we are actually referring to the set of which they are
members.

But there are some reasons why it seems incorrect to identify groups
with sets.  Take the Supreme Court.  It seems that any attempt to
identify the Supreme Court with the set of the Supreme Court justices
will not succeed.  This is because the membership of the Supreme Court
changes over time, while the members of a set do not.  The set
containing the 1990 justices is a {\em different} set from the set
containing the 2012 justices, but the 2012 Supreme Court is not a
different entity than the 1990 Court.  (We may of course say things
like ``it's a different court now'', but by that we mean only that it
is composed of different people, and so may rule differently---note
that we do {\em not} say ``it's a different Court now''.)

If one grants that groups such as the Supreme Court are not sets, it
may still be objected that they are therefore simply mereological
sums.  But 

\begin{squote}
membership in the Supreme Court is very different from
the part-whole relation on material objects.  The part-whole relation
on material objects is a transitive relation.  Thus if one identified
the Supreme Court with a material object and Justice Breyer with a
part of it, then one would be forced to conclude that Justice Breyer's
arm must be a part of the Supreme Court as well.  Yet, it is plain
that Justice Breyer's arm is neither a part nor a member of the
Supreme Court \citep[136--137]{uzquiano2004a}.
\end{squote}

If the Supreme Court were a mereological sum, it would behave very
strangely.  What its parts would be on a given occasion would depend
on the appointment decisions of the President.  (If we accept a
`four-dimensional' version of universalism, then objects have {\em
  temporal} as well as spatial parts.  There would then be a
mereological sum of the parts of the justices that existed during
their appointments.  This would be an object whose existence would not
depend on the President.)

There is at least some motivation to posit a new {\em kind} of thing
that is not a set or a sum.  This new kind is the group.
Unfortunately, once we try to spell out the identity conditions of
groups, we will see that, without assuming eternalism, we cannot
distinguish groups from sets.

\subsection{The place of the group in Fine's template}
\label{group-temp}
In section \ref{classical} above, Kit Fine showed how the summation
operator relates to four different properties: Collapse, Leveling,
Absorption, and Permutation.

\begin{description}
  \item[Absorption] $\sum (\mathellipsis, x, x, \mathellipsis,
    \mathellipsis, y, y, \mathellipsis, \mathellipsis = \sum (
    \mathellipsis, x, \mathellipsis, y, \mathellipsis )$;
\item[Collapse] $\sum (x) = x$;
\item[Leveling] $\sum (\mathellipsis, \sum (x, y, z, \mathellipsis ),
  \mathellipsis, \sum (u, v, w, \mathellipsis ), \mathellipsis ) \\ =
  \sum (\mathellipsis, x, y, z, \mathellipsis, \mathellipsis, u, v, w,
  \mathellipsis, \mathellipsis )$;
\item[Permutation] $\sum (x, y, z, \mathellipsis ) = \sum (y, z, x,
  \mathellipsis )$ (and similarly for all other permutations)
  \citep[573]{fine2010}.
\end{description}

Sums have all four properties, while sets have only Permutation and
Absorption.  We can begin to define our group operator by thinking
about which of these properties it has.

I tentatively suggest that groups mimic sets with regard to these
four properties.  Groups possess Absorption due to the fact that one
cannot be twice a member of the same group.  Groups possess
Permutation, since there is no `order' with regard to membership of a
group (there may be {\em temporal} order---I joined the group
first!---but that is not the same thing).  Groups do not possess
Collapse, since (I am inclined to think) a group can have a single
member without thereby {\em being} that member.  If, for example, a
task force is created and only one individual assigned to it, the
findings of the task force will be of the {\em task force} and not of
the individual.  Groups do not possess Leveling either, since there
can be groups made up of groups.

So far we have seen that groups are quite similar to sets.  But there
are some differences.  As we have seen, groups can change their
membership, while sets cannot.  Additionally, I suggest that the
application conditions (see section \ref{fine-app}) for groups is
different from that of sets.  While sets can have more or less
anything as members (letters, people, other sets), I propose that {\em
  groups may be composed only of living things and other groups}.

This claim is made on intuitive grounds.  It simply seems odd to talk
about a group of rocks, or a group of sets.  Talk of groups implies
some sort of activity, and so it is more natural to use `group' to
refer to people and other animals.  Even a `group' of trees is not
wholly bizarre.  And a group---for example, the Special Committee on
Judicial Ethics---might be part of another group---the Committee of
Ethics Committees \citep[145]{uzquiano2004a}.

%% Another point on application conditions: I think that group
%% composition is {\em unrestricted} among living things and groups.
%% That is, for any set of living things and/or groups, there is a
%% group of them.  Any restriction seems arbitrary.

\subsection{How do groups change their members?}
The most important apparent difference between sets and groups is that
while the members of a set are necessarily so, the members of a group
may change over time.  How does this work?

We could think of the group composition operator (the group-builder)
as operating {\em not} on things (living things and groups) but on
things-at-times.  The group-builder for the Supreme Court takes the
various justices during the times of their service and produces the
group---the Supreme Court---from those people-at-times.

One problem with this proposal is that it appears to presuppose {\em
  temporal parts}.  For if the group-builder works in similar fashion
to the set-builder and summation operator, then it operates on {\em
  things}.  If our group-builder is going to operate on
people-at-times, then we seem to commit ourselves to the claim that
people-at-times are {\em things}.  And what things could they be but
temporal parts of other things?

Thinking of groups as being composed of things-at-times rather than
things is unintuitive in any case.  Sandra Day O'Connor \emph{was} a
member of the Supreme Court.  Taking temporal parts seriously would
require us saying instead that her 1981---2006 part \emph{is} a member
of the Supreme Court.  But Sandra Day O'Connor is no longer a member
{\em at all}.

If we don't want to presuppose temporal parts, the group operator has
to be somehow \emph{dynamic}. It can't just take things, compose them
and be done---it has to \emph{add and remove things over time}.

Making sense of a dynamic group operator might allow us to avoid
presupposing {\em eternalism} as well (ultimately it will not).  If
the group-builder made the Supreme Court `in one go', then future
justices would have to already exist in some sense.  How else could
the group-builder operate on them?

The most readily apparent way of making sense of a dynamic operator is
by relativizing the group-builder to times.  We can think of the
operator as taking a set at a time and producing a group: $G = \sum
_{t} (S)$.  Following \citet{uzquiano2004a}, we can say that set $S$
composes group $G$ at time $t$ if and only if:

\begin{enumerate}[label=(\arabic*)]
  \item $\forall x\ (x \in S \leftrightarrow x\ \text{is a member of}\ G
  \  @\ t)$
  \item $\exists x {[} x\ \text{is a member
      of}\ G\ @\ t\ \wedge\ \square ( x \in S )\ \wedge
    \\ \diamond\ \exists t^{\prime} ( G\ \text{exists}\ @
    \ t^{\prime}\ \wedge\ \neg ( x\ \text{is a member
      of}\ G\ @\ t^{\prime} )) {]}\ \vee \\ \exists x^{\prime} {[}
    \neg ( x^{\prime}\ \text{is a member of}\ G\ @\ t)\ \wedge \\ \neg
    \diamond (x^{\prime} \in S ) \wedge \ \diamond \exists t^{\prime
      \prime} (x^{\prime} \text{is a member of}\ G\ @\ t^{\prime
      \prime}) {]}$\ \citeyearpar[150]{uzquiano2004a}
\end{enumerate}

I will assume that group composition is {\em unrestricted}.  That is,
for any people and/or groups at any time $t$, there is a group
composed of them at that time.

\subsection{Identity conditions}
Given that groups can change their parts over time, there will be
cases in which $G = \sum _{t_1} ( S )$ and $G^{\prime} = \sum _{t_2}
( S^{\prime} )$ and $G = G^{\prime}$ and $t_1 \neq t_2$ and $S \neq
S^{\prime}$.  When will this occur?  Under what conditions does $G =
G^{\prime}$?

For example, $t_1$ might be 2004, $t_2$ might be 2012, $S$ might be
\{Rehnquist, Stevens, O'Connor, Scalia, Kennedy, Souter, Thomas,
Ginsburg, Breyer\} and $S^{\prime}$ might be \{Roberts, Stevens,
O'Connor, Scalia, Kennedy, Souter, Thomas, Ginsburg, Breyer\}.  If we
suppose that $G$ is the Supreme Court in 2004 and that $G^{\prime}$
is the Supreme Court, then we want to be able to say that $G =
G^{\prime}$.  

But now we have hit a snag.  For as Uzquiano points out, a set of
people can compose more than one group at a time.  Suppose that all
and only the members of the Supreme Court in 2004 are part of the
Special Committee on Judicial Ethics.  In this case ``the Supreme
Court share[s] all of its members with the Special Committee on
Judicial Ethics as of a certain time [2004]''
\citep[151]{uzquiano2004a}.  It seems false to say that, in 2004, the
Supreme Court was identical with the Special Committee.  But if the
Supreme Court, $G$, is $\sum _{t} ( S )$ and the Special Committee,
$C$, is also $\sum _{t} ( S )$, then how can we deny that $G = C$?

%% One way is to allow that the group-builder can produce numerically
%% distinct groups from repeated applications of the same operation.

The obvious solution is to look at the past and future histories of
$G$ and $C$.  The Supreme Court is composed of $S^{\prime}$ in 2012,
while the Special Committee has been dissolved.  But I do not see how
we can appeal to identity across time without assuming eternalism.  If
we assume eternalism, we can say that $G = \sum ( \mathellipsis , \sum
_{2004}(S), \sum _{2012}(S^{\prime}), \mathellipsis )$.  But if we do
not assume eternalism, we will have to use temporal operators like
\textsc{was} and \textsc{will}.  We can therefore say only that
\textsc{was}($G = \sum (S)$), $G = \sum (S^{\prime})$ and
\textsc{will}($G = \sum (S^{\prime \prime})$).  We are still left with
its being presently the case that $G = C$.  Without assuming
eternalism, we seem forced to admit that, in 2004, the Supreme Court
{\em is} the Special Committee on Judicial Ethics.

The point of positing the existence of groups in addition to sets was
to avoid the identity of the Supreme Court with the Special
Committee.  But positing groups is not sufficient; we also need to
assume eternalism.  If we do not assume eternalism, then groups do not
help us.  Therefore, we should reconsider whether the positing of
groups is necessary.

\subsection{Re-examining the set identity thesis}
\label{set-id}
Given that we cannot seem to distinguish the Supreme Court and the
Special Committee, we might abandon groups altogether and claim that
the Supreme Court is not a group, but a set.  This view, of course,
has difficulties of its own.

The primary motivation cited above for positing groups was the fact
that the Supreme Court changes its members over time.  For example,
both of the following sentences are true:

\begin{enumerate}[label=(\arabic*)]
  \item The Supreme Court ruled on Roe vs.\ Wade in 1973. \label{roe1}

  \item The set of justices now serving as Supreme Court Justices did
    not rule on Roe vs.\ Wade in 1973
    \citep[135]{uzquiano2004a}. \label{roe2}
\end{enumerate}

One way to accommodate these facts is to ``insist that the Supreme
Court is a set, but to abandon the assumption that there is a single
set to which the phrase `the Supreme Court' refers in sentences
\ref{roe1} and \ref{roe2}'' \citep[138]{uzquiano2004a}.  To
successfully use the term `the Supreme Court' to refer to a set of
justices, there must be an implicit or explicit temporal reference.
If an utterance of \ref{roe1} is true it will be true because it the
speaker intends her audience to recognize her intention to refer to
the set of justices that was the Supreme Court in 1973.  If her
audience, for whatever reason, takes her to be referring to the
current Court, then they will evaluate \ref{roe1} as false.

Considered in this light, `the Supreme Court' is used to express a
relation between sets and times; ``$x$ is the Supreme Court at $t$''
\citep[140]{uzquiano2004a}.  There is some precedent for this sort of
interpretation:

\begin{squote}
Our use of the phrase `the Supreme Court' to express a relation a set
of justices bears to a time is much like our use of the phrase `the
president of the United States' to express a relation an individual
bears to a time.  Different persons may be the president of the United
States at different times, but there is at most one person that bears
that relation to each time \citep[138]{uzquiano2004a}.
\end{squote}

``But,'' it will be objected, ``there is an important difference here.
We use both terms---`the Supreme Court' and `the president'---to refer
to a past, present or future set that `is' the thing, but we also use
`the Supreme Court' to refer to {\em the Supreme Court}, which has
changed its membership over time.  If I say, `the Supreme Court has
become more conservative over the past century', there is no one set I
am referring to.  I must be referring to something else; the obvious
candidate is the {\em group} that is the Court.''

One reply here is to claim that all that what ``the Supreme Court has
become more conservative over the past century'' actually means is
that the members of the sets that have been the Supreme Court have
become more conservative.  Another, similar reply is that someone who
utters ``the Supreme Court has become more conservative over the past
century'' is saying something literally false (either because there is
no unique set that is being referred to, or because there is a unique
set referred to, but one that does not make the proposition true), but
can generally be understood to mean something else; namely, that the
members of the sets that have been the Supreme Court have become more
conservative.

Neither reply is {\em very} unintuitive; indeed, there is something
attractive about a thesis that reserves application of adjectives like
`conservative' for people, rather than other things like groups.

But there is a more pressing worry for the set identity thesis.
Recall that the set that is the Supreme Court at a given time might
also be the Special Committee on Judicial Ethics.  We must admit that
the Supreme Court in 2004 is the set \{Rehnquist, Stevens, O'Connor,
Scalia, Kennedy, Souter, Thomas, Ginsburg, Breyer\}, and the Special
Committee in 2004 is that very same set.  But now we are committed to
this argument:

\begin{enumerate}[ref=(\arabic*)]
  \item The Special Committee on Judicial Ethics is one of the
    committees assembled by the Senate.

  \item The Special Committee on Judicial Ethics is identical with the
    Supreme Court.

  \item {\em Therefore} the Supreme Court is one of the committees
    assembled by the
    Senate. \citep[144]{uzquiano2004a} \label{sup-com}
\end{enumerate}

And \ref{sup-com} seems false.

But it may be possible to argue that \ref{sup-com} is not false but
only {\em misleading} (indeed, very misleading).  For it
(conversationally) implies that future sets referred to by `the
Supreme Court' will be identical to future sets referred to by `the
Special Committee'.  And it is {\em this} that is certainly false.

\subsection{Set membership and implicature}
\label{implicate}
Suppose we arrive at a meeting of the Special Committee on Judicial
Ethics.  Rehnquist, Stevens, O'Connor, Scalia, Kennedy, Souter,
Thomas, Ginsburg, and Breyer are sitting around a center table.  As we
take our seats you turn to me and say, ``they look rather familiar,
don't they?''  I say ``that's also the Supreme Court.''

What am I referring to with the demonstrative expression ``that''?  If
one thinks that I am referring to a {\em group}---the Special
Committee---that is distinct from the Supreme Court, my utterance will
have to be interpreted as non-literal.  I will have to be understood
to mean that the {\em members} of the Special Committee are also the
members of the Supreme Court.

Suppose instead that you ask me who the members of the Special
Committee are.  I say ``Rehnquist, Stevens, O'Connor, Scalia, Kennedy,
Souter, Thomas, Ginsburg, and Breyer.  The Special Committee is just
the Supreme Court.''  Here again one could argue that I am speaking
non-literally; what I mean is that the members of the Special
Committee are just the members of the Supreme Court.

Now suppose that the Special Committee is dissolved in 2004.  In 2005,
we see the members of the Supreme Court (still Rehnquist, Stevens,
O'Connor, Scalia, Kennedy, Souter, Thomas, Ginsburg, and Breyer) out
to lunch together.  I point and say ``that was the Special Committee
on Judicial Ethics.''  Now what is the referent of ``that''?  It
cannot be the Special Committee, for that has ceased to be.  It must
either be the Supreme Court or the set \{Rehnquist, Stevens, O'Connor,
Scalia, Kennedy, Souter, Thomas, Ginsburg, and Breyer\}.  Either way,
the proponent of groups will have to interpret this utterance as
non-literal.

Now suppose that the Special Committee is dissolved in 2004 and
Rehnquist retired before dying in 2005 (let's pretend he retired in
May).  Now in August we see Rehnquist, Stevens, O'Connor, Scalia,
Kennedy, Souter, Thomas, Ginsburg, and Breyer out to lunch together.
I point and say ``that was the Supreme Court {\em and} the Special
Committee on Judicial Ethics.''  I can only be referring to the set of
justices.  Why not suppose that I have only {\em ever} been referring
to the set of justices?  If I am in fact referring to the set
\{Rehnquist, Stevens, O'Connor, Scalia, Kennedy, Souter, Thomas,
Ginsburg, Breyer\}, then when I say ``that was the Supreme Court {\em
  and} the Special Committee'', I say something literally true.

I also say something literally true when I say ``the Supreme Court is
one of the committees assembled by the Senate'' or ``the Supreme Court
is the Special Committee on Judicial Ethics''.  But it is very
misleading to say either.  By saying ``the Supreme Court is the
Special Committee'' I imply that future referents of `the Supreme
Court' will be identical to future referents of `the Special
Committee'.  It is less misleading to say ``the current Supreme Court
is the Special Committee on Judicial Ethics''.  (It is even less
misleading to say ``the current Supreme Court is also the Special
Committee''.)

We say above that `the Supreme Court' is sometimes used to refer
(whether literally or non-literally) to more than one set.  When I say
``the Supreme Court has become more diverse'' I mean that the members
of the sets that have been the Supreme Court have become more diverse.
It may be due to this fact that we so easily misinterpret uttered
propositions like ``the Supreme Court is the Special Committee''.  A
listener might take this to mean that the members of the sets that
have been the Supreme Court are identical with the members of the sets
that have been the Special Committee.  They would therefore evaluate
the utterance as false.

%% This is because it is mutually assumed that the set referred to by
%% `the Supreme Court' will change; the current set of justices will
%% not always be the Supreme Court.  It is a contingent fact that one
%% set---\{Rehnquist, Stevens, O'Connor, Scalia, Kennedy, Souter,
%% Thomas, Ginsburg, and Breyer\}---is both the Supreme Court and the
%% Special Committee in 2004, But since the form of the sentence ``the
%% Supreme Court is the Special Committee'' is that of an identity
%% statement, and sincethe audience will understandably

\subsection{Conventional identity conditions}
\label{set-convention}
Even supposing everything above is right, there is still more to be
said.  What are the identity conditions for the Supreme Court?  What
makes it true that one set is the Supreme Court in 2004 and a
different set is in 2012?

What makes it true that a given set is the Supreme Court at a given
time is simply our legal conventions.  The Constitution authorizes the
recognition of a set of justices as `the Supreme Court'.  Which set is
recognized as the Supreme Court is decided by the legislative and
executive branches.  The president nominates a set (the sitting
justices and the nominated justice) and the legislative branch votes.
The outcome of the vote make it true or false that a given set is the
Supreme Court.

This is analogous for all `groups'.  Teams, bands, militias---what
makes it true that a certain set is a team, band or militia is just
the conventions governing the group.  If I desert my militia and the
other members of the militia recognize my absence as a desertion, then
it is understood that I am no longer part of the militia; for that
reason it is then true that the set containing me is no longer the
militia.  A smaller set, not containing me, is now the militia.

It seems, then, that {\em identity conditions over time for groups are
  wholly conventional}.  This is plausible; groups are social
entities, and it makes sense that their composition should be a matter
of social convention.  But if this is true, it suggests something more
radical: that identity conditions over time for physical objects like
chairs are conventional as well.

\section{The conventions of ordinary things}
[Fine's definition of classical mereology; essentialism and its
  parallel with sets; objects are identical with sums; when we use
  `chair' we express a relation between a sum and a time; which sum is
  the referent of chair at any given time is conventional; the
  persistence of any given `chair' over time is conventional.]

%%%%%%
\begin{comment}
\section{Deflationary metaphysics}
Kathrin Koslicki has an interesting objection to universalist theses
such as the one I appear committed to.  Her objection amounts to this:
if every `collection' of objects (such as the London Bridge, a
particle in the moon, and Cal Ripkin, Jr.) is a thing in its own
right, then metaphysics becomes uninteresting.  There is no longer any
debate about whether chairs or dogbushes are more `real' or have a
stronger claim to existence.  They both (obviously) exist, and the
difference between chairs and dogbushes is not ontological but
conceptual: `chair' is more embedded in our talk, and so chairs have
greater importance to {\em us}.  But metaphysically, or ontologically,
chairs and dogbushes are on the same level.  There is no sense in
which chairs exist and dogbushes do not.

In the quoted material below, Koslicki is criticizing a version of
four-dimensionalism that Sider has previously defended.  Sider's
position was that any collection of objects-at-times is a thing in its
own right.  Sider calls these things `fusions'.  For example, a chair
is a fusion of a large number of {\em temporal part} of things (wood
molecules, or atoms, or simples).  Each thing (wood molecule, atom, or
simple) is a fusion of {\em its} temporal parts.  Each temporal part
of the chair is also a thing (a fusion).

I take no stance on whether objects have temporal parts or rather
`endure' through time.  Moreover, if we accept Fine's theory of parts
then we reject the idea that there is just one composition operation;
the operation that produces fusions is one among many.  But Koslicki's
comments are relevant nonetheless:

\begin{squote}
There is room, in Sider's theory, for {\em some} genuine ontological
disagreements: for example, the universalist, the nihilist and the
holder of the intermediary position genuinely disagree over how many
and which fusions that exist.  But the only genuine ontological
disagreements for which there is room, in Sider's world, are ones that
concern disagreements over `bare' fusions, so to speak.  What has
happened to the houses, trees, people, and cars, the familiar concrete
objects of common-sense, whose persistence this account set out to
analyze?  There are no `deep' ontological facts as to whether a given
fusion should count as a house or not\,\ldots

[By claiming that there can be genuine ontological disputes,] Sider is
guilty of a bit of false advertising: his account is really a way of
saying that, at the end of the day, there is no interesting {\em
  ontological} story to be told about the persistence of our familiar
concrete objects of common-sense; whatever there is to say about the
persistence of houses, trees, people and cars concerns the
organization of our conceptual household
\citeyearpar[124--125]{koslicki2003}.
\end{squote}

Koslicki seems to think that we ought to be able to find some
ontological difference between ``the familiar concrete objects of
common-sense'' and things like dogbushes or chairs-at-times.  But as I
remarked above, why should what interests us (familiar objects like
chairs) be a guide to what exists?  The conclusion that ``the
persistence [and other properties] of houses, trees, people and cars
concerns the organization of our conceptual household'' seems to be a
most welcome one.

However, there {\em is} an ontological difference between some things,
if no between chairs and dogbushes.  One lesson of Kit Fine's theory
of parts is that mereological sums are not the only kind of composite
thing.  There are sets as well, and groups, and sequences, and perhaps
infinitely many other types of thing.  The difference between a set
and a sum is an ontological difference.  Within each type, however,
we must rely on our own conceptual `scheme' to organize things.

In section \ref{lessons-v} I considered Jay Rosenberg's claim that
the Special Composition Question is the wrong question to be asking.
Rosenberg's position seems to be that there is {\em no} answer to the
Special Composition Question.  Rather, he thinks what it takes to
`compose' something depends on what that something is---making a chair
is not like making a pie.

This insight of Rosenberg's can be connected with the insights of
Fine's theory.  If we understand `composition' in the Special
Composition Question to mean {\em mereological composition}, then
Rosenberg was wrong if he held that there is no correct answer to the
Special Composition Question.  It seems intuitively true that
mereological composition is unrestricted.  But if take `composition'
in the Special Composition Question to be $K$-composition---any
composition operator at all---then Rosenberg was {\em right} that
there is no answer.  What the application conditions are for a
composition operator depends on {\em which} composition operation is
being applied.  

Moreover, determining these application conditions, and determining
identity conditions, and determining the other properties of these
various composition operations is a task for metaphysics.  The field
is not then so barren as Koslicki seems to have feared.  But it is
true that perhaps the most interesting questions---When are we willing
to call something a chair, and why?  What conditions must be
fulfilled?---are not ontological questions anymore.  They are
questions about our ``conceptual household.''

\end{comment}

\ifstandalone
\end{spacing}
\bibliography{everything}
\bibliographystyle{ChicagoReedweb}
\fi
\end{document}


\chapter{Can a chair change its parts?}
\chapterpig{Can a chair change its parts?}
\documentclass[11pt]{article}
\usepackage{standalone} \newif\ifstandlone \standalonetrue
\usepackage[left=1.75in, right=1.75in, top=1.25in, bottom=1.25in]{geometry}
\geometry{letterpaper}
\usepackage{graphicx}
\usepackage{enumitem}
\usepackage{amssymb}
\usepackage{amsmath}
\usepackage{epstopdf}
\usepackage{verbatim}
\usepackage{setspace}
\usepackage{natbib}
\setcitestyle{aysep={}}
\usepackage{hyperref}
\usepackage{url}
\synctex=1

\DeclareSymbolFont{symbolsC}{U}{txsyc}{m}{n}
\DeclareMathSymbol{\strictif}{\mathrel}{symbolsC}{74}
\DeclareMathSymbol{\boxright}{\mathrel}{symbolsC}{128}

\newenvironment{squote}{%
\begin{spacing}{1}
\begin{list}{}{%
\setlength{\labelwidth}{0pt}%
\rightmargin\leftmargin%
}
\item\relax
}{%
\end{list}%
\end{spacing}
}

\title{Essentialism}
\author{Alexander A. Dunn}
\begin{document}
\ifstandalone
\maketitle
\begin{spacing}{1.5}
\fi

\label{essential}

In section \ref{parts} I presented three different theories that
modified classical mereology.  These modification were made to
explain, among other things, how objects change their parts over time
and how co-located objects (like the statue and the lump) have
different properties.  But each of these theories require us to posit
an extraordinary plurality of co-located objects.  A theory that posits
co-located people is, I think, clearly false, but any theory that
entails the existence of a plurality of overlapping objects is at
least intuitively objectionable.

In this section, therefore, I will attempt to sketch an {\em
  essentialist} theory of things.  This theory will allow us to reject
the `plurality thesis'---that there are pluralities of co-located
objects---but it will have problems of its own.  The most glaring is
the consequence that, strictly speaking, things don't change their
parts.

In section \ref{problem3} I argued that using Fine's theory of
composition operators to account for groups like the Supreme Court led
to a plurality of different {\em kinds} of things rather than a
plurality of a single kind of group.  I suggested that this was a
drawback of the theory, and that we should re-examine the idea that
groups like the Supreme Court are actually identical with sets.

Before we assess the merits of that controversial thesis, there is
another possibility that should be addressed.  If we assume the theory
of {\em four-dimensionalism}, many of our problems appear to go away.
Unfortunately, new ones arise.

\section{What if we assume four-dimensionalism?}
\label{4d}
So far I have been supposing that four-dimensionalism is false.  That
is, I have assumed neither that things are composed of temporal as
well as spatial parts, nor that the past and future exist.  But if we
{\em do} assume four-dimensionalism, we have access to new solutions
to the problems relating to ordinary things.

What is four-dimensionalism?  I am supposing that
`four-dimensionalism' refers to the conjunction of two theories.  The
first is that things have {\em temporal parts}.  The second is {\em
  eternalism}.

Ted Sider presents a relatively clear picture of the first part of the
theory of four-dimensionalism, that of temporal parts:

\begin{squote}
Think of your life as a long story.  Let the story be a rather
narcissistic story: cut out all details about everything else except
you.  So the story begins with an infant (or perhaps a fetus).  It
describes the infant developing into a child and then an adolescent.
The adolescent passes into young adulthood, then adulthood, middle
age, and finally old age and death.  Like all stories, this story has
parts.  We can distinguish the part of the story concerning childhood
from the part concerning adulthood.  Given enough details, there will
be parts concerning individual days, minutes, or even instants.

According to the `four-dimensionalist' conception of persons (all all
other objects that persist over time), persons are a lot like their
stories.  Just as my story has a part for my childhood, so {\em I}
have a part consisting just of my childhood.  Just as my story has a
part describing just this instant, so I have a part that is
me-at-this-very-instant \citeyearpar[1]{sider2001}.
\end{squote}

The claim that we have these {\em temporal
  parts}---me-at-this-instant, or me-as-a-child---relies on a close
analogy between space and time.  It is relatively uncontroversial to
claim that we have {\em spatial} parts.  My foot is a part of me, for
instance, but it is not {\em all} of me (it is a proper part, to use
mereological terms).  The philosopher who claims that we have temporal
parts is saying that, likewise, my adulthood is a part of me, but it
is not all of me.  My childhood is---or was, if we do not assume
eternalism---another part of me.  My infancy, childhood, adulthood,
etc. together {\em compose} me.

This theory of temporal parts is often conjoined with a theory about
time.  This theory is commonly referred to as {\em eternalism}.
According to eternalism, ``time is like space.  There is nothing
special about the things here; things at other places are just as
real; no place is metaphysically distinguished.  Similarly, for the
eternalist, there is nothing special about the present; things at
other times are just as real; no time is metaphysically
distinguished'' \citep[122]{hinchliff1996}.  For the eternalist, there
is a sense in which ``there are dinosaurs'' is true.  Everyone agrees
that there are no dinosaurs {\em now}; the question is whether the
dinosaurs of the past still exist {\em in the past}.  

I have no firm intuition on whether either conjunct of the
four-dimensionalist theory is correct.  I do not know whether things
have temporal parts, and I do not know if the past (and future) exist.
Nonetheless I will assume in this section that four-dimensionalism is
true; {\em if} this assumption is correct, we can explain the
existence of ordinary things in new and interesting ways.

\subsection{Four-dimensional essentialism}
\label{4de}
According to four-dimensionalism, ordinary things like chairs and
statues are {\em four-dimensional spacetime worms}.  They are composed
of temporal parts or {\em slices}; a chair might be made up of
`chair-slices' at $t_{1}$, $t_{2}$, $t_{3}$\,\ldots\,, etc.  Depending
on who you ask, these `slices' have a very small temporal duration or
none at all.  If the latter, they are {\em extended} in only three
dimensions; their temporal extension is point-sized (this is what I
will assume).

Four-dimensionalism is very commonly conjoined with universalism---the
theory, defended in section \ref{universe}, that for any things, there
is something composed of them.  If we assume universalism, then
four-dimensionalism entails that for every set of temporal slices,
there is something composed of them.  There is an object composed of
the first ten years of my life, the Kremlin from 1970--1990, and one
second of a puppy's existence in 2020.  This thing is not, of course,
a person; nor is something composed of the first 10 years of my life
and the last ten years of someone else's.  Certain causal or
psychological connections must hold between the temporal parts of a
thing in order for it to be a person.

The objects composed of these temporal slices are mereological sums in
the classical sense.  Let us use `Krupkin' to designate the object
made of the first ten years of my life, the Kremlin from 1970--1990,
and one second of a puppy's existence in 2020.  Because the past and
future exist (we're assuming eternalism), Krupkin always has the same
parts.  Strictly speaking, it doesn't ever change its parts.  In 1991
it is true to say ``the Kremlin is not {\em now} part of Krupkin'',
but it is not true to say ``the Kremlin is not part of Krupkin.

If we assume universalism in addition to four-dimensionalism, then not
only does Krupkin not change its parts, it {\em cannot} change its
parts.  It cannot change its parts for the reason given in section
\ref{change}.  Let us use `Alkin' to designate the object composed of
the first 10 years of my life and the Kremlin from 1970--1990.  Now if
Krupkin could change its parts, it could lose a part.  Suppose it lost
its puppy part.  Then, if it still exists, it would be the object
composed of the first 10 years of my life and the Kremlin from
1970--1990.  But {\em that} object is Alkin; Krupkin would therefore
become identical with Alkin.  Alkin and Krupkin are not identical,
however, because Krupkin has a property that Alkin does not: the
property of having had a puppy as a part.  So Krupkin cannot, in fact,
lose a part; otherwise we would have a contradiction.

Technically, therefore, four-dimensional universalism is a version of
{\em essentialism}---the thesis that things cannot change their parts.
Four-dimensionalists explain change by relativizing things to times:
for a thing to change its color is just for it to have the relation of
being green at one time and red at another, or having a part at one
time and not at another.

I am somewhat sympathetic to this view.  In section \ref{essential} I
will sketch an essentialist theory of things, but one that presupposes
neither temporal parts nor eternalism.  But here I will briefly
examine how a four-dimensionalist essentialism addresses the issues
related to ordinary things that we have been concerned with.

Four-dimensionalism has two advantages and two disadvantages, when
compared with the three theories above.  The first advantage is that
four-dimensionalism does not posits a plurality of {\em kinds} of
things.  The second is that it does not posit co-located objects.  The
material objects that a four-dimensionalist recognizes are all
mereological sums in the classical sense.  The first disadvantage is
that four-dimensionalism, when conjoined with universalism, produces a
plurality of objects, just as the three theories above do.  The second
disadvantage is that four-dimensionalism has difficulty distinguishing
objects that are co-located for the entirety of their existence.

\subsection{Four-dimensional solutions}
\label{4ds}
The first advantage of four-dimensionalism---that it does not have to posit
a plurality of kinds of things---is primarily an advantage relative to
Fine's theory of composition operators (section \ref{fine-c}).  That
theory produced an incredible plurality, not only of things in
general, but of different kinds of things.  Four-dimensional things
are simply mereological sums, in the classical sense.

The second advantage of four-dimensionalism is that, unlike all three
theories presented above, it does not posit co-located objects.  The
theories in section \ref{parts}, in order to distinguish objects like
the statue and the lump---objects that (currently) share all their
parts---had do posit co-located objects.  But on the four-dimensional
picture, this is unnecessary.  Suppose that the lump is formed on
Monday, and the statue on Tuesday.  The lump therefore has temporal
parts that are `earlier' than any of the statue's parts.  They do not
share all their parts, and so are not co-located.  It is true that
they share all their Tuesday parts; the temporal slices that compose
the lump on Tuesday are the same that compose the statue on Tuesday.
But they share parts only at certain times.  They do not share all
their parts at all times.  (This leads into a problem for
four-dimensionalism, however; it does not appear to let us
differentiate a statue and a lump that {\em always} share their parts.
See section \ref{4dp}).

Four-dimensionalism can also readily account for the apparent change
in membership of the Supreme Court.  Since people (including the
justices) are composed of temporal slices, the Supreme Court can be
identified with the set of person-slices that correspond to the
various justices' terms.  For example, the 25 September 1981--31
January 2006 temporal parts of Sandra Day O'Connor are part of the
set.  The phenomenon of the Supreme Court `changing' its members
(e.g., Day O'Connor retiring) is simply it being 1 February 2006 and
that temporal part of Day O'Connor not being part of the set.

\subsection{Problems for four-dimensionalism}
\label{4dp}
But there are two disadvantages to four-dimensional universalism.  The
first is that while four-dimensionalism does not posit a plurality of
kinds of things or a plurality of co-located objects, there is still a
sense in which it is a `plurality thesis'.  Any given temporal slice
is part of a plurality of things.  When I point at my chair, I am also
pointing at a thing composed of my chair and a black bear from the
1800s, as well as a thing composed of my chair and the head of Thomas
Aquinas.  All those things (and {\em many} more) are currently located
in the very same place.

This is certainly bizarre, but it no more {\em disproves}
four-dimensionalism than does it disprove the three theories
previously examined.  Unfortunately there is another disadvantage to
four-dimensionalism, one that does threaten it as a theory.

The second disadvantage of four-dimensionalism is that it has trouble
distinguishing between objects that are co-located for the entirety of
their existence.  Suppose that I have two lumps of clay; I form one
into the top half of a figure and I shape the other into the bottom
half.  Having done this, I stick the two pieces of clay together,
forming a statue.  When I do this I also form a new, larger lump of
clay.  I admire the statue and the lump for a little while, then smash
them with a hammer.

Let $S$ be the thing composed of all the statue-slices.  Let $L$ be
the thing composed of all the larger-lump-slices.  $S$ and $L$ are
mereological sums composed of the very same parts; $S = L$.  But if
the statue had been squashed instead of smashed, $L$ would have
survived; but $S$ would not have survived being squashed.  $L$ has a
property that $S$ does not---the property `could survive being
squashed'---and therefore $S \neq L$.  This is a problem.

The four-dimensionalist could say that the lump would not have
survived being squashed, or that the statue would have survived.
Since there is only one thing under investigation (since $S = L$),
that thing must have a consistent set of properties.  It can't be such
that it would both survive and not survive a squashing.  So the
four-dimensionalist will have to say that one of our two intuitions is
wrong.

But there is another, related, difficulty.  In the case just
presented, the statue is the lump ($S = L$).  But suppose there is a
situation exactly like the one presented, but in which the statue and
lump are first squashed, then smashed.  In this case, we are inclined
to say that the lump $L^{\prime}$ continues to exist after the
squashing.  Its parts include temporal slices of the clay after it has
been squashed.  In the case of the statue $S^{\prime}$, however, we
are inclined to say that the statue does not have any temporal parts
after the squashing.  The statue is destroyed when it is squashed.
Since $L^{\prime}$ and $S^{\prime}$ have different parts, they are not
the same thing; $L^{\prime} \neq S^{\prime}$.  The four-dimensionalist
is committed to the claim that whether there is one thing (a statue
that is also a lump) or two things (a statue and a lump) on the table,
and what that thing's (or those things') modal properties are depends
upon whether I squash or smash it.  This seems highly implausible.  By
choosing to squash the statue rather than smash it, do I thereby {\em
  make it the case} that there were two things, rather than one?

The four-dimensionalist will object that, since the future already
exists, it was {\em already} true that there were two things (it has
always been true).  But claiming that it is already the case that I
will squash the statue seems to commit the four-dimensionalist to some
version of {\em determinism}---the thesis that, roughly, the events of
the future are determined, or fixed, to occur.  This may well be true,
but it is largely an empirical hypothesis; to rely on it here would be
unwise.  (If the four-dimensionalist does not assume determinism, and
instead assumes indeterminism, they will presumably have to say that,
since it is indeterminate whether or not I will squash the statue, the
number of things on the table is therefore also indeterminate.  This
seems even worse.)

\subsection{Moving along}
\label{4dc}
Four-dimensionalism allows for the resolution of a number of puzzles
related to ordinary things.  It does not resolve everything, however,
and it introduces a few problems of its own.  Moreover, it requires a
number of controversial assumptions: the theory of temporal parts,
eternalism, and possibly determinism.  I will therefore set aside
four-dimensionalism, and suppose henceforth that the past and future
do not exist, and that things do not have temporal parts.

\section{Re-examining the set identity thesis}
\label{set-id}
The primary motivation cited in section \ref{why-group} for positing
groups was the fact that the Supreme Court changes its members over
time.  For example, both of the following sentences are true:

\begin{enumerate}[label=(\arabic*)]
  \item The Supreme Court ruled on Roe vs.\ Wade in 1973. \label{roe1}

  \item The set of justices now serving as Supreme Court Justices did
    not rule on Roe vs.\ Wade in 1973
    \citep[135]{uzquiano2004a}. \label{roe2}
\end{enumerate}

One way to accommodate these facts is to ``insist that the Supreme
Court is a set, but to abandon the assumption that there is a single
set to which the phrase `the Supreme Court' refers in sentences
\ref{roe1} and \ref{roe2}'' \citep[138]{uzquiano2004a}.  To
successfully use the term `the Supreme Court' to refer to a set of
justices, there must be an implicit or explicit temporal reference.
If an utterance of \ref{roe1} is true it will be true because it the
speaker intends her audience to recognize her intention to refer to
the set of justices that was the Supreme Court in 1973.  If her
audience, for whatever reason, takes her to be referring to the
current Court, then they will evaluate \ref{roe1} as false.

Considered in this light, `the Supreme Court' is used to express a
relation between sets and times; ``$x$ is the Supreme Court at $t$''
\citep[140]{uzquiano2004a}.  There is some precedent for this sort of
interpretation:

\begin{squote}
Our use of the phrase `the Supreme Court' to express a relation a set
of justices bears to a time is much like our use of the phrase `the
president of the United States' to express a relation an individual
bears to a time.  Different persons may be the president of the United
States at different times, but there is at most one person that bears
that relation to each time \citep[138]{uzquiano2004a}.
\end{squote}

``But,'' it will be objected, ``there is an important difference here.
We use both terms---`the Supreme Court' and `the president'---to refer
to a past, present or future set that `is' the thing, but we also use
`the Supreme Court' to refer to {\em the Supreme Court}, which has
changed its membership over time.  If I say, `the Supreme Court has
become more conservative over the past century', there is no one set I
am referring to.  I must be referring to something else; the obvious
candidate is the {\em group} that is the Court.''

One reply here is to claim that all that what ``the Supreme Court has
become more conservative over the past century'' actually means is
that the members of the sets that have been the Supreme Court have
become more conservative.  Another, similar reply is that someone who
utters ``the Supreme Court has become more conservative over the past
century'' is saying something literally false (either because there is
no unique set that is being referred to, or because there is a unique
set referred to, but one that does not make the proposition true), but
can generally be understood to mean something else; namely, that the
members of the sets that have been the Supreme Court have become more
conservative.

Neither reply is {\em very} unintuitive; indeed, there is something
attractive about a thesis that reserves application of adjectives like
`conservative' for people, rather than other things like groups.

But there is a more pressing worry for the set identity thesis.
Recall that the set that is the Supreme Court at a given time might
also be the Special Committee on Judicial Ethics.  We must admit that
the Supreme Court in 2004 is the set \{Rehnquist, Stevens, O'Connor,
Scalia, Kennedy, Souter, Thomas, Ginsburg, Breyer\}, and the Special
Committee in 2004 is that very same set.  But now we are committed to
this argument:

\begin{enumerate}[ref=(\arabic*)]
  \item The Special Committee on Judicial Ethics is one of the
    committees assembled by the Senate.

  \item The Special Committee on Judicial Ethics is identical with the
    Supreme Court.

  \item {\em Therefore} the Supreme Court is one of the committees
    assembled by the
    Senate. \citep[144]{uzquiano2004a} \label{sup-com}
\end{enumerate}

And \ref{sup-com} seems false.

But it may be possible to argue that \ref{sup-com} is not false but
only {\em misleading} (indeed, very misleading).  For it
(conversationally) implies that future sets referred to by `the
Supreme Court' will be identical to future sets referred to by `the
Special Committee'.  And it is {\em this} that is certainly false.

\subsection{Set membership and literal speech}
\label{implicate}
I argued in section \ref{eng-quant} that ordinary uses of `there is'
are often false.  For example, if I say ``there is no beer'', what I
say is almost certainly false---there is beer {\em somewhere}---but
what I mean is that there is no beer in the house.

It is very likely that much of our ordinary talk is similarly
non-literal (see \citet{bach1987}).  For example, we should interpret
``the chair is mine'' as non-literal, because ``the chair is mine''
entails that there is only one chair in the world.  Even propositions
involving proper names might be non-literal.  If `Alex' designates
every person named `Alex', then ``Alex is lying down'' is literally
false, since it entails either that there is only one `Alex' or that
every `Alex' is lying down.

Therefore, if a theory predicts that some of our talk about groups is
non-literal, we should not necessarily be worried.  But not {\em all}
of our talk about groups is non-literal, and when a theory can
preserve the intuition that certain thinks are literally true, that
should be taken as an advantage.  At least for a certain class of
examples, the set-identity thesis preserves more of our intuitive
judgements about literal speech than does the theory that posits
groups as distinct from sets.

\begin{enumerate}
  \item Suppose we arrive at a meeting of the Special Committee on
    Judicial Ethics.  Rehnquist, Stevens, O'Connor, Scalia, Kennedy,
    Souter, Thomas, Ginsburg, and Breyer are sitting around a center
    table.  As we take our seats you turn to me and say, ``they look
    rather familiar, don't they?''  I say ``that's also the Supreme
    Court.''

    What am I referring to with the demonstrative expression ``that''?
    \begin{itemize}
      \item If one thinks that I am referring to a {\em group}---the
        Special Committee---that is distinct from the Supreme Court,
        my utterance will have to be interpreted as non-literal.  I
        will have to be understood to mean that the {\em members} of
        the Special Committee are also the members of the Supreme
        Court.
      \item On the other hand, if I am referring to the {\em set}
        of justices, what I said is literally true.
     \end{itemize}

  \item Suppose instead that you ask me who the members of the Special
    Committee are.  I say ``Rehnquist, Stevens, O'Connor, Scalia,
    Kennedy, Souter, Thomas, Ginsburg, and Breyer.  The Special
    Committee is just the Supreme Court.''  
    \begin{itemize}
      \item Here again one could argue that I am speaking
        non-literally; what I mean is that the members of the Special
        Committee are just the members of the Supreme Court.  
      \item But if the Supreme Court and the Special Committee are
        just sets---the same set---I have again said something
        literally true.
    \end{itemize}

  \item Now suppose that the Special Committee is dissolved in 2004.
    In 2005, we see the members of the Supreme Court (still Rehnquist,
    Stevens, O'Connor, Scalia, Kennedy, Souter, Thomas, Ginsburg, and
    Breyer) out to lunch together.  I point and say ``that was the
    Special Committee on Judicial Ethics.''  Now what is the referent
    of ``that''?  It cannot be the Special Committee, for that has
    ceased to be.  It must either be the Supreme Court or the set
    \{Rehnquist, Stevens, O'Connor, Scalia, Kennedy, Souter, Thomas,
    Ginsburg, and Breyer\}.  
    \begin{itemize}
      \item Either way, the proponent of groups will
    have to interpret this utterance as non-literal.  
      \item The set-identity theorist can interpret this utterance as
        literally true, however; that set was the Special Committee
        before the dissolution.
    \end{itemize}

  \item Now suppose that the Special Committee is dissolved in 2004
    and Rehnquist retired before dying in 2005 (let's pretend he
    retired in May).  Now in August we see Rehnquist, Stevens,
    O'Connor, Scalia, Kennedy, Souter, Thomas, Ginsburg, and Breyer
    out to lunch together.  I point and say ``that was the Supreme
    Court {\em and} the Special Committee on Judicial Ethics.''  I can
    only be referring to the set of justices.  Why not suppose that I
    have only {\em ever} been referring to the set of justices?  If I
    am in fact referring to the set \{Rehnquist, Stevens, O'Connor,
    Scalia, Kennedy, Souter, Thomas, Ginsburg, Breyer\}, then when I
    say ``that was the Supreme Court {\em and} the Special
    Committee'', I say something literally true.
\end{enumerate}

These examples provide some support for the set identity thesis.  At
the very least they show that identifying groups with sets does not
mean that all our talk about groups must be interpreted as
non-literal.  However, the set identity thesis also predicts that some
propositions will be literally true, when intuitively we may believe
that they are not.  For example, according to the set identity thesis,
I say something literally true when I say ``the Supreme Court is one
of the committees assembled by the Senate'' or ``the Supreme Court is
the Special Committee on Judicial Ethics''.  But it is very misleading
to say either.  By saying ``the Supreme Court is the Special
Committee'' I imply that future referents of `the Supreme Court' will
be identical to future referents of `the Special Committee'.  It is
less misleading to say ``the current Supreme Court is the Special
Committee on Judicial Ethics''.  (It is even less misleading to say
``the current Supreme Court is also the Special Committee''.)

One may object that, while this is all well and good, the set identity
thesis fails the most important test.  The set identity thesis
predicts that propositions about the Supreme Court evolution over
time, such as ``the Supreme Court has become more diverse'' are
literally false.

This is a drawback for the set identity theorist, but I do not think
it is a great one.  As a parallel case, take the proposition ``the
temperature is dropping''.  For this to be literally true, there would
have to be some thing---the temperature---that is, in some sense,
dropping.  But it is plausible to interpret a speaker who says ``the
temperature is dropping'' as meaning that soon, the number that is the
referent of `the temperature' will be lower than the number currently
referred to by `the temperature'.  Likewise, when I say ``the Supreme
Court has become more diverse'' I mean that the members of the sets
that have been the Supreme Court have become more diverse.

It may be due to this fact that we so easily misinterpret uttered
propositions like ``the Supreme Court is the Special Committee''.  A
listener might take this to mean that the members of the sets that
have been (and will be) the Supreme Court are identical with the
members of the sets that have been (and will be) the Special
Committee.  They would therefore evaluate the utterance as false.

Another example that the set identity thesis predicts as non-literal
is ``the Supreme Court has become more conservative''.  If we claim
that the Supreme Court is a set, we cannot interpret this utterance as
literally true; {\em sets} do not have political leanings.  Someone
who utters this, according to the set identity thesis, must be taken
to mean that the members of the sets that have been the referents of
``the Supreme Court'' have become more conservative.

But can a philosopher who distinguishes groups from sets interpret
this literally?  Can groups {\em literally} have political leanings?
Or must the speaker be interpreted as meaning that the members of the
group have become more conservative?  I think it is plausible that
``the Supreme Court has become more conservative'' must be interpreted
as non-literal, whether the Supreme Court is a group or a set.

What about an utterance such as ``the Supreme Court ruled against the
defendant''?  If the Supreme Court is a set, this utterance will have
to be interpreted as non-literal.  Sets don't {\em do} things; we will
have to interpret the speaker as meaning that the Supreme Court
justices ruled against the defendant.  But what if the Court was
divided over the ruling?  If several justices wrote dissenting
opinions, it seems that {\em they} didn't rule against the defendant.
Rather, we want to say that the {\em group} ruled against the
defendant.  The proponent of groups may be in a slightly stronger
position here.  But if we identify the Supreme Court with a set, we
can still say that the {\em majority} of the Supreme Court justices
ruled against the defendant.  And that is more or less what we mean
when we say ``the Supreme Court ruled against the defendant''.

Whether or not we identify the Supreme Court and other groups with
sets, we will have to interpret some apparently literal speech as
non-literal.  But the set identity thesis, at least with regard to
this slate of examples, treats talk about groups more consistently
than does the theory that groups are distinct from sets.

\subsection{Conventional identity conditions}
\label{set-convention}
Even supposing everything above is right, there is still more to be
said.  What are the `identity conditions' for the Supreme Court over
time?  Although the Supreme Court is a set, we use `the Supreme Court'
to refer to different sets at different times.  What governs this
shifting reference?  What makes it true that one set is the Supreme
Court in 2004 and a different set is in 2012?

What makes it true that a given set is the Supreme Court at a given
time is simply our legal conventions.  The Constitution authorizes the
recognition of a set of justices as `the Supreme Court'.  Which set is
recognized as the Supreme Court is decided by the legislative and
executive branches.  The president nominates a set (the sitting
justices and the nominated justice) and the legislative branch votes.
The outcome of the vote make it true or false that a given set is the
Supreme Court.

(Now what do we mean when we say ``the Supreme Court was established
in 1789''?  Perhaps that the convention of referring to a set of
justices as `the Supreme Court'---and the granting of legal powers to
them---began in 1789.)

This is analogous for all `groups'.  Teams, bands, militias---what
makes it true that a certain set is a team, band or militia is just
the conventions governing the group.  If I desert my militia and the
other members of the militia recognize my absence as a desertion, then
it is understood that I am no longer part of the militia; for that
reason it is then true that the set containing me is no longer the
militia.  A smaller set, not containing me, is now the militia.

It seems, then, that {\em `identity' conditions over time for groups
  are wholly conventional}.  This is plausible; groups are social
entities, and it makes sense that their composition should be a matter
of social convention.  But if this is true, it suggests something more
radical: that identity conditions over time for physical objects like
chairs are conventional as well.

\section{The conventions of ordinary things}
\label{chair-ref}
Just as we identified groups like the Supreme Court with different
sets at different times, so we can identify things like chairs with
different sums at different times.

This will have the consequence that, at $t_1$, the statue is identical
with the lump of clay.  This will be true, since they are both the set
$S$.  The statue {\em is} the lump of clay.  This seems correct.  If
someone were to ask ``I see the statue, but where did the lump of clay
go?'' we would reply, ``the statue {\em is} the lump''.

If we feel tempted to say that the statue is not identical with the
lump, this is because future (and {\em possible}) utterances of
`statue' may not be used to refer to the same sum as future utterances
of `lump'.  (Thus we can account for their apparent modal differences
as well.)

The identification of statues (and lumps) with sums allows us to
explain some sorts of talk that would be otherwise problematic.  For
example, suppose I make a statue out of a lump of clay.  You come
along and squish the statue (thereby destroying it):

\stage{Alex}{}{That was my statue!}

If we thought that the statue was a distinct thing from the sum (and
from the lump), we would have to interpret what I say non-literally.
For if the statue was a distinct thing that has been destroyed, then
when I use a demonstrative like ``that'' I cannot be referring to the
non-existence statue.  My audience may interpret me as referring to
the lump, and meaning that there used to be a statue co-located with
the lump.  But if we suppose that the statue was not a distinct thing
from the sum (and from the lump), then what I said is literally true.
For ``that''---that sum---was a statue, but is no longer.  It is no
longer a statue because it no longer satisfies our conventional
criteria for what counts as a statue.  (You could dispute these
criteria; after I say ``that was my statue'', you could say ``it still
is your statue; it's just a flatter statue than it was''.)

One objection that may arise here is that, {\em strictly speaking},
the statue still exists.  Worse, if the statue is just the sum, then
the statue existed even before it was sculpted!  How can that be?

Suppose I have a lump of clay on Monday:

\stage{Alex}{}{This will be a statue!}

On Tuesday I make a statue out of the clay:

\stage{Alex}{}{Yesterday this was nothing more than a lump of clay!
  Now look at it!}

On Wednesday you squish the statue:

\stage{Alex}{}{Well, it's not a statue anymore.}

With these examples, I am trying to motivate the idea that we refer to
the sum when we use terms like `it' and `this'.  The sum referred to
on Monday is the same (or nearly the same) sum referred to on Tuesday
and on Wednesday.  When I say ``this will be a statue'', therefore, I
mean that this sum will meet the criteria for being referred to as a
statue.  When I say ``this was nothing more than a lump of clay'', I
mean that this sum previously met only the criteria for being referred
to as a lump of clay.  When I say ``it's not a statue anymore'', I
mean that it no longer meets the criteria for being referred to as a
statue.

But what if I say ``that statue doesn't exist anymore''?  I suggest
that this should be interpreted in almost all cases as meaning that
the sum that is understood to have been the referent of `that statue'
no longer meets the criteria for being the referent of `statue'.  If
``that statue doesn't exist'' is taken in a more literal sense, then
it is false.

Peter van Inwagen has a similar objection to the idea that sums cannot
change their parts (he assumes that ordinary things are sums).
Suppose that sums cannot change change their parts:

\begin{squote}
Call the bricks that were piled in the yard last Tuesday the ``Tuesday
bricks.''  Between last Tuesday and today, the Wise Pig has built a
house---the ``Brick House''---out of the Tuesday bricks (using them
all and using no other materials).  The Brick House did not exist last
Tuesday (that is, it was not then a pile of bricks, a thing that was
not yet a house but would become a house).  The Brick House is not,
therefore, a mereological sum; for if it were, it would have been (it
would have ``existed as'') a pile of bricks last Tuesday
\citeyearpar[616]{inwagen2006}.
\end{squote}

But since the Brick House {\em is} a mereological sum, van Inwagen
concludes that our supposition that sums can't change their parts is
false; he claims that mereological sums {\em can} change their parts.

However, I suggest that, strictly speaking, the Brick House {\em did}
exist last Tuesday.  But last Tuesday it (the sum) did not meet the
criteria for being referred to as a house.  Today I point to the Brick
House and say ``last Tuesday that was just a pile of bricks.  Now it's
a house!''  By `that' I mean the Brick House---the sum---which was a
pile of bricks on Tuesday.  If I say ``the Brick House did not exist
last Tuesday'' I should be taken to mean just that the Brick House did
not meet the criteria for being referred to as `the Brick House' last
Tuesday.

\subsection{Criteria and convention}
\label{criteria}
Like the `identity' of groups over time, the `identity' over time of a
given thing---like my chair---will be conventional.

But this does not mean that {\em I} have the final say over how my
chair persists through time.  The identity over time of the Supreme
Court is conventional, but it is not up to me.  There are entrenched
legal conventions governing the persistence of the Supreme Court.
Likewise, for different things, different conventions will govern
their persistence.  Recall Rosenberg's discussion of the problem with
van Inwagen's Special Composition Question (section \ref{lessons-v}).
He rejected the idea that there is just one way in which `material
objects' are composed:

\begin{squote}
Microphysics explains how protons, neutrons, and electrons compose
different species of atoms, and physical chemistry, how atoms of
various species compose different sorts of molecules
\citep[706]{rosenberg1993}.
\end{squote}

I think it is plausible to claim that atoms, molecules, animals,
chairs, statues and lumps are, in fact, identical with sums.  If this
is correct, then there is really just {\em one} way in which these
things are composed.  But they {\em persist through time} in very
different ways; which sum is identical with a given animal, for
example, depends on the `conventions' of the biological sciences.

\subsection{Essentialism}
\label{essentialism}
This theory is a version of {\em essentialism}.  Essentialism is the
thesis that, strictly speaking, things don't change their parts.  One
can endorse or oppose essentialism in various domains.  For example,
everyone is a set essentialist; nobody (as far as I know) claims that
sets can change their parts.  But not everyone is a {\em mereological}
essentialist.

People who deny mereological essentialism are, I think, making one of
two claims:

\begin{enumerate}
  \item They may be claiming that ordinary things like chairs are not
    mereological sums; chairs can change their parts, so essentialism
    {\em with regard to chairs} is false.  A philosopher who makes
    this claim might allow that mereological sums, if there are such
    things, cannot change their parts.
  \item They may be claiming that mereological sums, whether or not
    they are identical with ordinary things like chairs, can change
    their parts.
\end{enumerate}

I argued against the second claim in section \ref{change}.  But a
philosopher who makes either claim will reject the theory I have been
building.  They will say that my theory flies in the face of common
sense (and I made so much of common sense in earlier sections!).  They
will say things like this:

\begin{squote}
According to [the essentialist], it is never literally correct to say
that a thing survives a change in parts.  This is a point of massive
departure from ordinary belief \citep[184]{sider2001}.
\end{squote}

This is more or less the argument against essentialism.  You point at
a chair and say ``I'm supposed to believe that if that chair loses
{\em one atom}, it's literally a different chair?''

One argument for the claim that the change of a single part results in
a numerically distinct chair comes from Chisholm:

\begin{squote}
Let us picture to ourselves a very simple table, improvised from a
stump and a board.  Now one might have constructed a very similar
table by using the same stump and a different board, or by using the
same board and a different stump.  But the only way of constructing
precisely {\em that} table is to use that particular stump and that
particular board.  It would seem, therefore, that that particular
table is {\em necessarily} made up of that particular stump and that
particular board \citeyearpar[146]{chisholm1976}.
\end{squote}

It may be objected that, {\em once the table is built}, it is possible
to change its parts without thereby destroying one table and
constructing another.  But what is the relevant difference between
building a table for the first time---when the addition of a different
part results in a different table---and adding a new part to an
already existing table?  There does not seem to be one; for all we
know, before Chisholm build his table out of a stump and a board,
there was a table constructed out of the same stump and a different
board.  But since the table that Chisholm build is necessary made up
of its parts, it cannot be identical to a previous table that is made
of different parts.

There is another argument for the same conclusion.  If a chair can
remain numerically distinct after losing a part, it is difficult to
say {\em how large} a part the chair can lose while remaining the
(numerically) same chair.  If most of the chair is blasted away, then
we may very well say that the chair is no more.  But {\em how much}
must be blasted away?  Or suppose we have a portion of gold.  How many
atoms of gold can be stripped off before it is no longer the same
portion?  Thomson claims that ordinary uses of `portion' are
context-dependent:

\begin{squote}
The ordinary use of the term ``portion'' is heavily context-dependent.
If an atom drifts away from your portion of gold, do you still have
the same portion of gold?  You will say no if you are a scientist
engaged in an experiment for which every atom matters. You will say
yes if you are a jeweler about to make a ring.  Similarly, in fact,
for clay.  If you have just bought a load of clay, and a handful falls
off while you are on your way home, is the portion you have when you
get home the same as the portion you bought?  You will say no if you
had carefully measured and bought exactly as much as you need.  You
will say yes if loss of a handful makes no difference to you
\citeyearpar[163]{thomson1998a}.
\end{squote}

When we say that a use of a term is `context-dependent', that can mean
one of two things.  First, it may mean that whether an utterance
involving a use of the term is {\em correct}, or {\em appropriate},
depends on the context.  It would not be appropriate for the scientist
to say that she has the same portion after the loss of several atoms,
because those atoms matter for the experiment.  Second, to say that
the use of a term is `context-dependent' may mean that whether an
utterance involving a use of the term is {\em true} depends on the
context.  In the quoted passage above, do the scientist and jeweler
both say true things?  If they do, then the truth-conditions of
`portion' are context-dependent.  This would mean that whether an
utterance involving `portion' is true depends on the context of the
utterance.  This would also suggest that the {\em meaning} of
`portion' depends on the context, for the truth-value of a sentence is
generally thought to be a function of the meaning of its constituent
elements, including words.

But just as I do not think there are different senses of `there is'
(see section \ref{eng-quant}), so I do not think that there are
multiple senses of `portion'.  I find it far more plausible to think
that only the scientist says something that is, {\em strictly
  speaking}, true.  The jeweler, when she affirms that she has the
same portion of gold, may say something correct or appropriate, given
the context, but it is not {\em true}.  Strictly speaking, a portion
cannot change its parts; why should we assume that a chair can?

But even if ``that's the same chair as yesterday'' is literally false,
the sum that yesterday satisfies the criteria for being the referent
of ``chair'' no longer satisfies those criteria; a different sum that
has many of the same parts now satisfies those criteria, and so
qualifies as `the same chair'.  (How similar the two sums has to be
is, again, a conventional matter.)

Maybe I am wrong and it is true that (strictly speaking) the chair can
lose its parts and yet remain the (numerically) same chair.  If so,
then we must accept the mereological firmament that comes with Fine's
theory.

\section{Am I a mereological sum?}
\label{i-sum}
I have proposed that ordinary things like chairs and statues are
mereological sums.  Their apparent persistence through change is a
result of certain conventions---a chair at $t_1$ is the `same' chair
at $t_2$ if, first, there is a sum at each time that meets our
criteria for being a referent of `chair' and, second, either the sum
that is the referent of `chair' at $t_1$ is the same sum that is the
referent of `chair' at $t_2$, or the sum that was the referent of
`chair' at $t_1$ does not meet the criteria for being the referent
of `chair' at $t_2$ and a different sum does meet these criteria at
$t_2$ and is sufficiently related to the first sum.  What is
`sufficiently related' is again a conventional matter, but will
presumably involve causal and spatiotemporal continuity.

If ordinary things are sums, then are other things sums as well?  I
will suppose that sums are `material things' as opposed to `abstract
things' (whatever that distinction comes to), but are {\em all}
material things sums?  If we are material things, are we therefore
sums?

% all material things are sums

\subsection{All material things are sums}
\label{material-sum}
If we think that ordinary things are sums, and that ordinary things
are material things, I think it is extremely plausible to conclude
that all material things are sums.  For what else would they be?

What is included under the label `material thing'?  I would include
things like chairs, and desks, and desk lamps, and doors, and
doorways, and houses, and gardens, and plants.  I would also include
minuscule objects like molecules and massive objects like planets and
galaxies.  What would these things be, if not sums?

I proposed that ordinary things are sums so as to avoid the conclusion
that there is a plurality of different kinds of ordinary things
(statues and lumps only scratch the surface) all overlapping each
other.  This essentialist proposal was made so as to avoid positing
many different kinds of things.  So anyone who accepts the
essentialist theory should be sympathetic to the idea that all
material things are sums.

I don't have a powerful argument for this conclusion, but I don't see
the {\em point} of supposing that all and only ordinary things are
sums, but other material things are some different kind of object.

\subsection{We are material things}
\label{material-beings}
Even if the idea that all material things are sums is (or should be)
relatively uncontroversial, the idea that {\em we} are material beings
will not be unanimously accepted.  For it does have some unintuitive
consequences.  

First, it rules out identifying us with our mental states.  Suppose
all my psychological characteristics---memory, personality---is
somehow transferred to another body.  The brain in that body is
`wiped' before my psychology is transferred, and after the operation
my old brain is similarly `wiped'.  There is a temptation to say that
I exist in the new body.  But saying this commits us to the claim that
I am not a material thing, because I `left' my old material body and
came to `inhabit' a new one:

\begin{squote}
 If I am identical with the thinking substance in which I am thus
 placed, then I cannot be transferred {\em from} that substance to
 another substance \citep[107]{chisholm1979}.
\end{squote}

Claiming that we are material things entails that psychological
continuity is not a criterion of identity.  The body into which my
psychology is transferred is not me, according to the materialist
claim.  Psychological continuity is often taken to be {\em the}
criterion of identity, so one might take this consequence as a
refutation of the claim that we are material things.

But if we are not material things, what are we?  The only alternative
I see is to claim that we are immaterial minds or souls.  These
positions seem, to me, to be more implausible than the claim that we
are material things.  (Much, of course, can be said in defense of such
a position.)

Claiming that we are material things, however, gives rise to another
question: what material things are we?  Are we identical with our
brains, or with our bodies?

I suggest, though somewhat tentatively, that we are identical with our
bodies.  I agree with Peter van Inwagen on this much:

\begin{squote}
I suppose that [the objects of mental predicates]---Descartes, you,
I---are material objects, in the sense that they are ultimately
composed entirely of quarks and electrons.  They are, moreover, a very
special sort of material object.  They are not brains or cerebral
hemispheres.  They are living animals; being {\em human} animals, they
are things shaped roughly like statues of human beings
\citeyearpar[6]{inwagen1995}.
\end{squote}

Eric Olson has a very plausible argument for the same conclusion:

\begin{enumerate}
  \item There is a human animal sitting in your chair.
  \item The human animal sitting in your chair is thinking. (If you
    like, every human animal sitting there is thinking.)
  \item You are the thinking being sitting in your chair. The one and
    only thinking being sitting in your chair is none other than
    you. Hence, you are that animal \citeyearpar[354]{olson2003a}.
\end{enumerate}

One apparent consequence of the claim that we are material human
animals is that if my brain is removed from my body and put into
another body, that new person is not me.  Claiming that we are
material things required denying that psychological continuity is a
criterion of identity; claiming that we are material human animals
requires denying that even brain continuity is a criterion of
identity.

This may seem to be a troubling consequence, but it is much less
troubling if we accept the essentialist theory.  If material objects
are sums, and if we are material objects, then we are sums.  And if
sums do not, strictly speaking, change their parts over time, then,
like the `identity' conditions for chairs and other ordinary things,
the `identity' conditions over time for {\em us} is conventional.

Another difficulty with identifying us with human animals disappears
if we accept an essentialist theory.  Dean Zimmerman has objected to
Olson's argument by claiming that `human animal' can be replaced with
`human body' without making the argument invalid
\citeyearpar[24]{zimmerman2008a}.  The problem, however, is that it
seems true that we cease to exist when we die.  So Zimmerman concludes
that we are not bodies or animals.

If we accept an essentialist theory, however, the problem disappears.
If, strictly speaking, I can't change my parts over time, then I am
not (strictly speaking) the same person that will be the referent of
`Alex' a month from now (or even a week).  I will certainly not be
identical with a dead body further down the road.

\subsection{How do I `persist' over time?}
\label{person-persist}
The idea that, strictly speaking, I don't change my parts over time
seems crazy.  And maybe it is.  But I don't think it is obviously
false.

Someone who thinks that I do, strictly speaking, persist over time
might say that it is obvious that I persist.  After all, I engage in
activities that take long periods of time, I remember things from long
ago, and I bear unique attitudes toward my past and future selves.  I
feel pride or regret at past actions, and anticipation or apprehension
at future ones.  How could these past and future selves not be me?

One reply begins by pointing out that, whether or not we persist in a
strict sense, the world will look the same.  I will still engage in
activities that take time; but it will not be I who completes them.  I
will still remember things from long ago; but it will not be I who
experienced them.  I will bear attitudes towards past and future
people, but those people will not, strictly speaking, be me.  But it
will {\em seem} as if they are me, and they will meet the conventional
criteria for being the referent of `me'.  As in the case of tables and
chairs, there are conventional `identity' conditions for people over
time.  Like tables and chairs, these criteria will involve causal and
spatiotemporal continuity.  What person is the referent of `Alex' a
week from now will depend on a causal chain connected to me.

Psychological continuity may also play a role.  For example, if by
some miracle I am vaporized and (coincidentally) an qualitatively
identical person is summoned into existence nearby, that person will
not, strictly speaking be me.  But it may be agreed that the person
meets the criteria for being the referent of `Alex'.  Then again, it
may not.  If this person meets these criteria, however, it will be on
account of the apparent psychological continuity between us.

The criteria for the `identity' over time of people is not fully
precise, as shown by our indecision over whether a spontaneous
duplicate of me ought to be referred to as `Alex'.  Another, more
realistic, situation in which this indecision manifests itself is in
death.  Suppose I die, and a wake is held for my body.  It is
perfectly correct for someone to point and say, ``that was Alex''.
But it is equally correct to say ``that's Alex''.  (The latter may be
more appropriate if it is necessary to identify my body.)  Is the
mereological sum that is the (deceased) body really me, or not?  If we
accept the essentialist theory, it is (strictly speaking) not, but it
may be correct or appropriate to refer to the body as `Alex'.

\section{Can the essentialist theory explain what we believe?}
\label{explain-e}
In section \ref{explain-p} I assessed whether any of the three
`plurality' theories could explain why we held beliefs that conflicted
with certain consequences of the theories.  The same assessment may
be conducted with regard to the essentialist theory I have sketched in
this section.  If the essentialist thesis is right, why do we believe
that chairs are literally identical over time?  If we want to defend
the essentialist theory, we should try to explain why we generally
seem to think that things like chairs literally persist over time, and
can change their parts over time as well.

One reply is simply to claim that we {\em don't} believe that things
literally persist over time.  When asked ``is it {\em literally} the
same chair without its leg?'' some of us may waver, and perhaps
concede that we don't think it is really the same chair.  But I doubt
this reply will convince any philosopher who has already made up her
mind about essentialism.

Another reply\,\ldots

\section{Lessons}
\label{lessons-e}
In section \ref{parts} I examined three different versions of the
`plurality thesis'; the view that there are pluralities of co-located
objects.  In this section I offered an alternative.  I am not sure
whether my theory or one of the plurality theses is correct, but I
suspect that it is one or the other.  My conclusion is largely the
same as that of Karen Bennett:

\begin{squote}
The only live options, then, are to be either a one-thinger or a
bazillion-thinger.  We must either think that there is only one thing per
spatio-temporal location, or else that there are lots and \emph{lots} of
spatio-temporally coincident things \citeyearpar[358]{bennett2004}.
\end{squote}

I would prefer to be a `one-thinger' because it does not commit me to
a `bazillion' things all in the same place.  That is not a decisive
objection, of course.  It may well be that such an explosion is more
plausible than certain consequences of the `one-thinger' theory.  But
I think one of the two theories must be right.

\subsection{Do we need Fine's theory at all?}
\label{need-fine}
I have argued that we can identify ordinary things like chairs as
mereological sums, and we can identify things like groups as sets.  It
is therefore not necessary to use Fine's theory of parthood to
describe these things.  Do we need Fine's theory at all?

I think the theory is still useful.  For it shows that sums and sets
both have parts, but in different ways.  There are also sequences,
strings, and other things produced by various composition operators.
And there are other things that exist, like words, poems, events, and
quantities, that have parts in other ways.  Fine's theory may help us
define composition operators for these things, if they cannot be
described in terms of sets and sums (which I suspect they cannot be).

Fine's theory also suggests a novel way of describing temporal parts.
He claims that not all operators are compositional; some are {\em
  decompositional}.  He introduces the segmentation operation that
takes partless simples and produces their `parts'---the top and
bottom, left and right half, etc.  These `parts' are derived from the
wholes, rather than the wholes being built up from the parts:

\begin{squote}
Let us suppose that the universe consists of physical atoms which are
physically indivisible but of finite volume.  We might then
distinguish between the upper and lower parts of the atom (relative to
its orientation at a given time); and it is plausible that the atoms
are to be taken as givens, there being no explanation of their
identity in more basic terms, while the identity of the upper and
lower parts of an atom is to be explained in terms of their {\em
  being} the upper and lower parts of the atom.  Thus the account of
the part is in terms of the whole rather than the other way around
\citep[585]{fine2010}.
\end{squote}

Likewise, if we maintain that objects are not {\em composed} of
temporal parts, but are `temporally indivisible', we can nonetheless
have a temporal-segmentation operator that generates temporal `parts'
from temporally `simple' wholes.  These parts would have theoretical
utility; we could quantify over them without committing ourselves to
their existence (in a `basic' sense).  This sort of derived temporal
part might allow for a form of four-dimensionalism (the conjunction of
universalism and the doctrine of temporal parts) that avoids the
`spinning disc' problem, among others.

So whether or not we accept the ``vast mereological firmament'' that
Fine envisions, his theory of part is useful.

\subsection{Deflationary metaphysics}
\label{deflate}
Kathrin Koslicki has an interesting objection to universalist theses
such as the one I appear committed to.  Her objection amounts to this:
if every `collection' of objects (such as the London Bridge, a
particle in the moon, and Cal Ripkin, Jr.) is a thing in its own right
(a sum), then metaphysics becomes uninteresting.  There is no longer
any debate about whether chairs or dogbushes are more `real' or have a
stronger claim to existence.  They both (obviously) exist, and the
difference between chairs and lumpkins is not ontological but
conceptual: `chair' is more embedded in our talk, and so chairs have
greater importance to {\em us}.  But metaphysically, or ontologically,
chairs and dogbushes are on the same level.  There is no sense in
which chairs exist and lumpkins do not.

In the quoted material below, Koslicki is criticizing a version of
four-dimensionalism that Sider has previously defended.  Sider's
position was that any collection of objects-at-times is a thing in its
own right.  Sider calls these things `fusions'.  For example, a chair
is a fusion of a large number of {\em temporal part} of things (wood
molecules, or atoms, or simples).  Each thing (wood molecule, atom, or
simple) is a fusion of {\em its} temporal parts.  Each temporal part
of the chair is also a thing (a fusion).

I take no stance on whether objects have temporal parts or rather
`endure' through time.  Moreover, if we accept Fine's theory of parts
then we reject the idea that there is just one composition operation;
the operation that produces fusions is one among many.  But Koslicki's
comments are relevant nonetheless:

\begin{squote}
There is room, in Sider's theory, for {\em some} genuine ontological
disagreements: for example, the universalist, the nihilist and the
holder of the intermediary position genuinely disagree over how many
and which fusions that exist.  But the only genuine ontological
disagreements for which there is room, in Sider's world, are ones that
concern disagreements over `bare' fusions, so to speak.  What has
happened to the houses, trees, people, and cars, the familiar concrete
objects of common-sense, whose persistence this account set out to
analyze?  There are no `deep' ontological facts as to whether a given
fusion should count as a house or not\,\ldots

[By claiming that there can be genuine ontological disputes,] Sider is
guilty of a bit of false advertising: his account is really a way of
saying that, at the end of the day, there is no interesting {\em
  ontological} story to be told about the persistence of our familiar
concrete objects of common-sense; whatever there is to say about the
persistence of houses, trees, people and cars concerns the
organization of our conceptual household
\citeyearpar[124--125]{koslicki2003}.
\end{squote}

Koslicki seems to think that we ought to be able to find some
ontological difference between ``the familiar concrete objects of
common-sense'' and things like lumpkins or chairs-at-times.  But as I
remarked above (section \ref{universalism}), why should what interests
us (familiar objects like chairs) be a guide to what exists?  The
conclusion that ``the persistence [and other properties] of houses,
trees, people and cars concerns the organization of our conceptual
household'' seems to be correct.

However, there {\em is} an ontological difference between some things,
if not between chairs and dogbushes.  One lesson of Kit Fine's theory
of parts is that mereological sums are not the only kind of composite
thing.  There are sets as well, and strings, and sequences, and
perhaps infinitely many other types of thing.  The difference between
a set and a sum is an ontological difference.  Within each type,
however, we must rely on our own conceptual `scheme' to organize
things.

In section \ref{lessons-v} I considered Jay Rosenberg's claim that
the Special Composition Question is the wrong question to be asking.
Rosenberg's position seems to be that there is {\em no} answer to the
Special Composition Question.  Rather, he thinks what it takes to
`compose' something depends on what that something is---making a chair
is not like making a pie.

This insight of Rosenberg's can be connected with the insights of
Fine's theory.  If we understand `composition' in the Special
Composition Question to mean {\em mereological composition}, then
Rosenberg was wrong if he held that there is no correct answer to the
Special Composition Question.  It seems intuitively true that
mereological composition is unrestricted.  But if take `composition'
in the Special Composition Question to be $K$-composition---any
composition operator at all---then Rosenberg was {\em right} that
there is no answer.  What the application conditions are for a
composition operator depends on {\em which} composition operation is
being applied.  

Moreover, determining the application conditions for these composition
operators, and determining their identity conditions, and determining
the other properties of these various operators is a task for
metaphysics.  The field is not then so barren as Koslicki seems to
have feared.  But it is true that many interesting questions---When
are we willing to call something a chair, and why?  What conditions
must be fulfilled?---are not ontological questions anymore.  They are
questions about our ``conceptual household.''

\ifstandalone
\end{spacing}
\bibliography{everything}
\bibliographystyle{ChicagoReedweb}
\fi
\end{document}


\chapter*{Conclusion}
\label{concl}
\chapterpig{Conclusion}
\addcontentsline{toc}{chapter}{Conclusion}
\chaptermark{Conclusion}
\markboth{Conclusion}{Conclusion}
\documentclass[11pt]{article}
\usepackage{standalone} \newif\ifstandlone \standalonetrue
\usepackage[left=1.75in, right=1.75in, top=1.25in, bottom=1.25in]{geometry}
\geometry{letterpaper}
\usepackage{graphicx}
\usepackage{enumitem}
%\usepackage{amssymb}
\usepackage{amsmath}
\usepackage{epstopdf}
\usepackage{verbatim}
\usepackage{setspace}
\usepackage{natbib}
\setcitestyle{aysep={}}
\usepackage{hyperref}
\usepackage{url}
\synctex=1

\DeclareSymbolFont{symbolsC}{U}{txsyc}{m}{n}
\DeclareMathSymbol{\strictif}{\mathrel}{symbolsC}{74}
\DeclareMathSymbol{\boxright}{\mathrel}{symbolsC}{128}

\newenvironment{squote}{%
\begin{spacing}{1}
\begin{list}{}{%
\setlength{\labelwidth}{0pt}%
\rightmargin\leftmargin%
}
\item\relax
}{%
\end{list}%
\end{spacing}
}

\title{The End: Oh Make it Stop}
\author{Alexander A. Dunn}
\begin{document}
\ifstandalone
\maketitle
\begin{spacing}{1.5}
\fi

I have argued for a number of claims in the previous sections.

First, debates in metaphysics such as the one I have been engaged in
are conducted in English (or French, or German) and not in
`Ontologese' or some other pseudo-language.  If it is `really' or
`fundamentally' the case that there are no chairs, then it is true in
English that there are no chairs.

Second, philosophers who deny that there are chairs appear unable to
explain why we nonetheless believe that there are chairs.  Trenton
Merricks, who denies that there are chairs, has an explanation of why
we believe nonetheless that there are chairs.  He claims that because
``things arranged chairwise'' matter to us, we have introduced the
word `chair' to refer to them; we are fooled by the singular nature of
the word `chair' and come to think that there is some single {\em
  thing} that we are referring to, when in fact there is not.  His
explanation, however, is equally compatible with universalism: the
claim that for every set of things, there is some other thing they
compose.  And universalism is a much more plausible thesis than the
nihilism of Merricks.

Third, if we assume that universalism is true then we have a choice to
make.  We can either adopt a `plurality' theory that posits a huge
plurality of things (and different {\em kinds} of things), or we can
adopt a version of {\em essentialism}, maintaining that, strictly
speaking, things do not change their parts over time.  I have
suggested that while both routes are defensible, the essentialist
theory avoids some of the excesses of co-location that plague the
plurality theories while offering some neat solutions to problems of
personal identity over time.  However, neither solution appears to
offer a ready answer to the mental problem of the many: how it is that
with many mereological sums that are all equally `suitable' to be
identified with the one and only thinking thing in a particular
location, one and only one of those sums is a thinking thing.

This thesis therefore ends with no fully-formed theory.  I have, in
effect, offered a disjunction: either a plurality theory or an
essentialist theory is correct.  I have my own sympathies, but I
recognize that there is plenty of room for disagreement.

\ifstandalone
\end{spacing}
\bibliography{everything}
\bibliographystyle{ChicagoReedweb}
\fi
\end{document}

%	\setcounter{chapter}{4}
%	\setcounter{section}{0}
	
%If you feel it necessary to include an appendix, it goes here.
%    \appendix
%      \chapter{The First Appendix}
%      \chapter{The Second Appendix, for Fun}


%This is where endnotes are supposed to go, if you have them.

  \backmatter % backmatter makes the index and bibliography appear
              % properly in the t.o.c...

\bibliographystyle{chicago}
\bibliography{everything}

\end{spacing}
% Finally, an index would go here\,\ldots\,but it is also optional.
\end{document}
