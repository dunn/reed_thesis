% This is the Reed College LaTeX thesis template. Most of the work 
% for the document class was done by Sam Noble (SN), as well as this
% template. Later comments etc. by Ben Salzberg (BTS). Additional
% restructuring and APA support by Jess Youngberg (JY).
% Your comments and suggestions are more than welcome; please email
% them to cus@reed.edu
%

\documentclass[12pt,twoside]{reedthesis}
\usepackage{graphicx,latexsym} 
\usepackage{amssymb,amsthm,amsmath}
\usepackage{longtable,booktabs,setspace} 
\usepackage{chemarr} %% Useful for one reaction arrow, useless if you're not a chem major
\usepackage{url}
\usepackage{natbib}
% \usepackage{times} % other fonts are available like times, bookman, charter, palatino

\usepackage{enumitem}
\setcitestyle{aysep={}}
\synctex=1

\DeclareSymbolFont{symbolsC}{U}{txsyc}{m}{n}
\DeclareMathSymbol{\strictif}{\mathrel}{symbolsC}{74}
\DeclareMathSymbol{\boxright}{\mathrel}{symbolsC}{128}

\newenvironment{squote}{\begin{quote}\begin{singlespace}}{\end{singlespace}\end{quote}}

\newcommand{\stager}[4]%
{%
	\begin{spacing}{1}%
	\vspace{0pt}
		\begin{description}[style=nextline, noitemsep, parsep=0pt, topsep=0pt, leftmargin=15mm, itemindent=-10mm, font=\mdseries]
			\item[\textsc{#1} \emph{#2}] #3
			\item[]%
			\begin{flushright}#4\end{flushright}
		\end{description}%
	\end{spacing}%
}

\newcommand{\stage}[3]%
{%
	\begin{spacing}{1}%
	\vspace{0pt}
		\begin{description}[style=nextline, parsep=0pt, leftmargin=15mm, itemindent=-10mm, font=\mdseries]
			\item[\textsc{#1} \emph{#2}] #3
		\end{description}%
	\end{spacing}%
}

\title{A World Without Us}
\author{Alexander A. Dunn}
% The month and year that you submit your FINAL draft TO THE LIBRARY (May or December)
\date{May 2012}
\division{Philosophy and Other Things}
\advisor{Paul Hovda}
\department{Philosophy}

\setlength{\parskip}{0pt}

%%%%%%%%%%%%%%%%%%%%%%%%%%%%%%%%%%%%%%%%%%%%

\begin{document}

  \maketitle
  \frontmatter % this stuff will be roman-numbered
  \pagestyle{empty} % this removes page numbers from the frontmatter

% Acknowledgements (Acceptable American spelling) are optional
% So are Acknowledgments (proper English spelling)
    \chapter*{Acknowledgements}
	I want to thank Peter Unger for infuriating me. Everything he has written is stupendously false.

% The preface is optional
% To remove it, comment it out or delete it.
    \chapter*{Preface}
	This is an example of a thesis setup to use the reed thesis document class.

    \tableofcontents
% if you want a list of tables, optional
    \listoftables
% if you want a list of figures, also optional
    \listoffigures

% The abstract is not required if you're writing a creative thesis (but aren't they all?)
% If your abstract is longer than a page, there may be a formatting issue.
    \chapter*{Abstract}
	The preface pretty much says it all.

  \mainmatter % here the regular arabic numbering starts
  \pagestyle{fancyplain} % turns page numbering back on

%The \introduction command is provided as a convenience.
%if you want special chapter formatting, you'll probably want to avoid using it altogether

    \chapter*{Introduction}
         \addcontentsline{toc}{chapter}{Introduction}
	\chaptermark{Introduction}
	\markboth{Introduction}{Introduction}
	% The three lines above are to make sure that the headers are right, that the intro gets included in the table of contents, and that it doesn't get numbered 1 so that chapter one is 1.

% Double spacing: if you want to double space, or one and a half 
% space, uncomment one of the following lines. You can go back to 
% single spacing with the \singlespacing command.
% \onehalfspacing
% \doublespacing
	
	Welcome to the \LaTeX\ thesis template. If you've never used \TeX\ or \LaTeX\ before, you'll have an initial learning period to go through, but the results of a nicely formatted thesis are worth it for more than the aesthetic benefit: markup like \LaTeX\ is more consistent than the output of a word processor, much less prone to corruption or crashing and the resulting file is smaller than a Word file. While you may have never had problems using Word in the past, your thesis is going to be about twice as large and complex as anything you've written before, taxing Word's capabilities. If you're still on the fence about  using \LaTeX, read the Introduction to LaTeX on the CUS site as well as skim the following template and give it a few weeks. Pretty soon all the markup gibberish will become second nature.

\section{Why use it?}
	
\LaTeX\ does a great job of formatting tables and paragraphs. Its line-breaking algorithm was the subject of a PhD.\thinspace thesis. It does a fine job of automatically inserting ligatures, and to top it all off it is the only way to typeset good-looking mathematics.

\section{Who should use it?}

Anyone who needs to use math, tables, a lot of figures, complex cross-references, IPA or who just cares about the final appearance of their document should use \LaTeX. At Reed, math majors are required to use it, most physics majors will want to use it, and many other science majors may want it also.
	
\documentclass[11pt]{article}
\usepackage{standalone} \newif\ifstandlone \standalonetrue
\usepackage[left=1.75in, right=1.75in, top=1.25in, bottom=1.25in]{geometry}
\geometry{letterpaper}
\usepackage{graphicx}
\usepackage{enumitem}
%\usepackage{amssymb}
\usepackage{amsmath}
\usepackage{epstopdf}
\usepackage{setspace}
\usepackage{natbib}
\setcitestyle{aysep={}}
\usepackage{hyperref}
		
\synctex=1

\DeclareSymbolFont{symbolsC}{U}{txsyc}{m}{n}
\DeclareMathSymbol{\strictif}{\mathrel}{symbolsC}{74}
\DeclareMathSymbol{\boxright}{\mathrel}{symbolsC}{128}

\newenvironment{squote}{%
	\begin{quote}\begin{singlespace}%
	}{%
	\end{singlespace}\end{quote}}

\newcommand{\stager}[4]%
{%
	\begin{spacing}{1}%
	\vspace{0pt}
		\begin{description}[style=nextline, noitemsep,
                    parsep=0pt, topsep=0pt, leftmargin=15mm,
                    itemindent=-10mm, font=\mdseries]
			\item[\textsc{#1} \emph{#2}] #3
			\item[]%
			\begin{flushright}#4\end{flushright}
		\end{description}%
	\end{spacing}%
}

\newcommand{\stage}[3]%
{%
	\begin{spacing}{1}%
	\vspace{0pt}
		\begin{description}[style=nextline, parsep=0pt,
                    leftmargin=15mm, itemindent=-10mm, font=\mdseries]
			\item[\textsc{#1} \emph{#2}] #3
		\end{description}%
	\end{spacing}%
}

\newenvironment{inq}{\vspace{0pt}%
	\begin{list}{}%
	{\setlength\labelwidth{0pt}%
	\setlength\leftmargin{2.5\oddsidemargin}%
	\setlength\rightmargin{\leftmargin}}
	\begin{spacing}{1}
	\item[]%
	}{
	\end{spacing}
	\end{list}
	\vspace{10pt}
	%\noindent%
	}

\title{Unger's arguments}
\author{Alex Dunn}
\begin{document}
\ifstandalone
\maketitle
\begin{spacing}{1.5}
\fi

Peter Unger has presented several arguments that threaten the kind of
universalism I sketched in section \ref{universalism}.  The versions
of the sorites paradox that he presents make trouble for all vague
concepts, including those that Merricks relies on for his version of
nihilism.  Unger's `problem of the many', on the other hand, poses a
problem specifically for versions of universalism, including my own.

\section{Incoherence and pluralities}
\label{unger}
Like Peter van Inwagen and Trenton Merricks, Peter Unger has denied
the existence of all `ordinary things'---such things as ``tables and
chairs and spears\,\ldots swizzle sticks and
sousaphones\,\ldots\,stones and rocks and twigs, and also tumbleweeds
and fingernails'' (\citeyear[117]{unger1979}).  Merricks has a
different motivation for his nihilism than does van Inwagen, and Unger
has a different motivation again.  In fact, he has two different
motivations; one is the sorites paradox and the other is the problem
of the many.  

Unger claims that the sorites paradox shows that terms for ordinary
things, like `chair', are {\em incoherent}.  He claims that incoherent
terms cannot apply to anything in the world; therefore he concludes
that there are no chairs (or any other ordinary thing).

Unger's presentation of the problem of the many is aimed to trouble
the concept of `chair' in a different way.  The conclusion of that
argument is not that `chair' is necessarily incoherent.  Rather, the
conclusion is a disjunction: {\em either} there are no chairs, {\em
  or} there are a plurality (possibly an infinity) of chairs where we
would normally take there to be only one.

We will examine these two arguments in turn.

\section{Sorites paradoxes}
\label{sorites}
A typical instance of the sorites paradox begins by having us imagine
some ordinary object; let us use a heap of sand.  Now suppose we
remove a single grain of sand.  If we were inclined to believe that
the initial quantity of sand did in fact constitute a heap, then after
the removal of a single grain, we should presumably still have a heap
(albeit a slightly smaller one).  It seems very implausible to think
that one grain of sand more or less could {\em ever} make a difference
as to whether something is or is not a heap.

But having conceded (a) that there is a heap and (b) that the removal
of a single grain cannot make the difference as to whether a quantity
of sand is a heap, we have unwittingly put our foot in it.  For if the
removal of a single grain {\em never} transforms a heap into a
non-heap, then by repeatedly removing one grain after another, we will
eventually find ourselves with a heap that consists of no sand at
all.  But it seems absurd to suppose that there could be a heap of
sand that is composed of no sand---indeed, of nothing whatsoever.

This is the sorites paradox.  While a heap is a useful example,
because it is so ill-defined, similar problems appear to afflict all
ordinary things.  Unger illustrates the difficulty for stones:

\begin{squote}
Consider a stone, consisting of a certain finite number of atoms.  If
we or some physical process should remove one atom, without
replacement, then there is left that number minus one, presumably
constituting a stone still\,\ldots after another atom is removed,
there is that original number minus two; so far, so good.  But after
that certain number has been removed, in similar stepwise fashion,
there are no atoms at all in the situation, while we must still be
supposing that there is a stone there.  But as we have already agreed,
if there is a stone present, then there must be some atoms\,\ldots I
suggest that any adequate response to this contradiction must
include\,\ldots the denial of the existence of even a single
stone.~\citep[121--122]{unger1979}
\end{squote}
Unger understands this dilemma to apply across the board, and
correspondingly argues that we should deny the existence of even a
single ordinary thing.

\subsection{The sorites paradox in relation to Merricks' nihilism}
\label{sorites-m}
The purpose of Unger's argument is to show that terms that are
susceptible to the sorites paradox are incoherent and cannot apply to
anything in the world.  Unger claims that terms like `chair' therefore
cannot apply to anything in the world---from which it follows that
there are no chairs.

Trenton Merricks' version of nihilism (section \ref{merricks}) denies
that there are chairs.  In this, Merricks is in agreement with Unger.
But unlike Unger, Merricks maintains that beliefs like ``there are
chairs'' are {\em justified} and {\em nearly as good as true}.  He
claims that such beliefs are nearly as good as true if they are caused
by simples arranged chairwise.  Merricks does not believe that there
are chairs, so ``there are chairs'' is, strictly speaking, false.  But
Merricks believes that there are simples arranged chairwise, and the
presence of such arrangements cause and justify false beliefs such as
``there are chairs''.

We have seen, however, that the sorites paradox threatens the
coherency of concepts like `chair'.  If we suppose that a given
collection of atoms composes a chair, we can remove them one by one
and at no point feel justified in saying that the chair suddenly
ceases to exist.  Now suppose that a given arrangement of simples is
arranged chairwise.  Remove one.  Is the arrangement still arranged
chairwise?  Just as in the case of the chair, it seems bizarre to
think that a single simple (or atom) can make a difference as to
whether `chairwise' applies.  But now remove another atom\,\ldots

If Unger's argument shows that the concept of `chair' is incoherent,
then it seems that the same argument shows the concept of `chairwise'
to be incoherent.  If so, then `chairwise' can apply to nothing in the
world.  There can be no chairwise arrangements of simples.  If there
are no chairwise arrangements of simples, then our belief that there
are chairs cannot be nearly as good as true.  If someone believes that
there is a ghost, that belief, according to Merricks, is not nearly as
good as true because there are no ghostwise arrangements of simples to
cause or justify the belief that there is a ghost.  If `chairwise' is
incoherent, then there are no chairwise arrangements of simples to
cause or justify the belief that there are chairs.  The belief that
there are chairs is (if Unger is right) simply false, just as is the
belief that there are ghosts.

Merricks cannot allow this conclusion.  But the sorites paradox can be
used to show that {\em any} vague term is incoherent.  Much of our
language is vague, but we are not therefore tempted to conclude that
our speech is rarely (if ever) coherent.  The problem of the sorites
paradox is a very general problem that requires a general theory of
vagueness.  Most metaphysical theories are threatened by the sorites
paradox; Merricks is not a special case.

With that in mind, it should not be surprising that Merricks does not
have full answer.  A full answer to the sorites paradox would be a
theory of vagueness.  Merricks is not trying to establish a theory of
vagueness, but it attempting to motivate a metaphysical thesis about
ordinary things.  That said, he does have a {\em partial} answer.  He
points out that while the sorites paradox does threaten the concept of
`chairwise', it does so in a less troubling way than the way in which
it threatens `chair'.  

How is the sorites paradox ``less troubling'' for chairwise
arrangements than for chairs?  Very roughly, it is because accepting
the vagueness of `chair' can lead us to {\em metaphysical} vagueness,
while accepting the vagueness of `chairwise' can only lead us to {\em
  linguistic} vagueness.  And linguistic vagueness is generally
considered to be less troubling than metaphysical vagueness.

We can get a sense of how this is so by imagining, as Merricks does,
the sorites paradox being played out as a series of questions.  Let us
suppose that there is a chair.  We then ask ourselves (Merricks asks
God), ``is `there is a chair' true?''  Presumably we will answer
``yes''.  Then we remove one atom (or simple, or other thing) from the
chair.  ``Is `there is a chair' true?''

If we agree that there is no single atom whose removal would destroy
the chair, then we must accept that at some point it becomes
indeterminate whether ``there is a chair'' is true.  If it is
indeterminate whether ``there is a chair'' is true, then it is
indeterminate whether there is a chair.  The idea that it could be
indeterminate whether something exists is taken by many to be
problematic (\textbf{CITE}).

We can compare this case with that of the things arranged chairwise.
Let us suppose that there are things arranged chairwise.  We ask ``is
`there things arranged chairwise' true?'' and answer ``yes''.  Then we
remove a thing and ask ``is `there things arranged chairwise' true?''
As in the chair case, if we are unwilling to allow that the removal of
a single thing could take us from ``yes'' to ``no'', then we must
admit that at some point it is indeterminate whether or not ``there
are things arranged chairwise'' is true.  It would then be
indeterminate whether there are things arranged chairwise.  {\em But
  this does have the result that it is indeterminate whether something
  exists.}  It may be perfectly determinate that there are the things
there are; all that is indeterminate is whether the things (which
there determinately are) are arranged in a certain way.  This kind of
indeterminacy seems less troubling than indeterminacy as to whether
something (a chair or anything else) exists.

Merricks therefore sees the problem posed by the sorites paradox as
less threatening to his chairwise arrangements than to chairs.  He
does not attempt to provide a solution, for that would require solving
the problem of linguistic vagueness.  But he finds the problem of
linguistic vagueness less troubling than the problem of metaphysical
vagueness, and he shows that the latter does not threaten his version
of nihilism.

\subsection{So what's the problem?}
\label{sorites-3}
We find ourselves wanting to hold three theses, which appear mutually
inconsistent:

\begin{enumerate}
  \item There is at least one chair (stone, cloud).
  \item If a chair (stone, cloud) exists, it must be made up of
    matter.
  \item If a chair (stone, cloud), exists, the removal of a single
    molecule (or otherwise insignificant quantity of matter) from it
    cannot destroy it or cause it to cease to exist.
\end{enumerate}

We seem to be clearly caught in a paradox; the only question is where
we have gone wrong.

But have we, in fact, gone wrong?  Peter Unger thinks that we are
right on target:

\begin{squote}
While Eubulides' contribution has often been labeled `the sorites
paradox', there is nothing here which is a paradox in any
philosophically important sense\,\ldots Accepting our negative
conclusions here does not mean important logical trouble for us; we
only think we have troubles while we refuse to admit their validity
(\citeyear[145]{unger1979}).
\end{squote}

Our situation is only paradoxical, says Unger, while we unreflectingly
cling to the first thesis.  If, however, we come to see that there are
no chairs (stones, clouds), then we happily escape paradox: if there
are no chairs (stones, clouds) to begin with, we do not have to worry
about what the addition or removal of small amounts of matter would do
to them; nor do we need concern ourselves with what they would be made
of.

But things are not quite so simple.  First, adding to the
implausibility of Unger's view, he must deny that our use of ordinary
terms like `chair' (`stone', `cloud') follow any sort of pattern or
display any competence at all.  Second, even if we manage to swallow
that consequence, Unger has no explanation as to why we believe that
there are chairs (stones, clouds).

\subsection{Competence and correctness}
\label{correct}
Setting aside whether or not expressions of propositions like ``that's
a chair'' are ever \emph{true}, it seems right to say that there are
at least correct and incorrect uses of the terms.  For a word like
`chair' (`stone', `cloud') we generally do not say that a child has
learned how to use it until she is capable of deploying it in certain
ways.  We admit that she understands what `chair' (`stone', `cloud')
means or what a chair (stone, cloud) is when she displays a certain
competence with the term.  If instead of using `chair' to refer to
chairs she used it to refer to dogs or people, we would say that she
is confused and attempt to correct her use.

But Unger maintains that this is all an illusion, and that there is no
such thing as the correct or incorrect use of a term like `chair'
(`stone', `cloud'):

\begin{squote}
Concerning words and kinds, now, we might say this.  First, we might
say that it is in connection with \emph{semantics} that our reasonings have
what are their most obvious implications and, second, that their most
obvious semantic implications concern certain \emph{sortal nouns}, namely,
those which purport to denote ordinary things.  Thus, it appears quite
obvious to us now that there will be no application to things for such
nouns as `stone' and `rock', `twig' and `log', `planet' and `sun',
`mountain' and `lake', `sweater' and `cardigan', `telescope' and
`microscope', and so on, and so forth.  Simple positive sentences
containing these terms will never, given their current meanings,
express anything true, correct, accurate, etc., or even anything which
is anywhere close to being any of those things
(\citeyear[148]{unger1979}).
\end{squote}

This seems simply bizarre.  On what grounds, then, do parents correct
their children with respect to their use of ordinary terms?  Are they
compelled by some irrational force to consider certain utterances
correct and others incorrect?  One may question whether or not we use
ordinary term entirely consistently, but it seems simply false to say that,
necessarily, we {\em never} use (or have used, or will use) ordinary
terms in correct, as opposed to incorrect, ways.  

The fact that Unger's position means that terms like `chair' are never
used correctly gives us reason to think his argument goes awry
somewhere.  However, attempting to identify the issue with his
argument and proposing a solution would be tantamount to attempting to
solve the problem of vagueness.  That is not something I will attempt,
at least not until we have examined the problem of the many.

\section{The problem of the many}
\label{many}
The `problem of the many', as Unger terms this second difficulty for
ordinary things, follows a similar line of reasoning as that of the
sorites paradox.  If we consider an ordinary thing---a cloud, for
instance---it is natural to think that it is made up of molecules.
There is probably then a set of molecules, the members of which make
up the cloud.  Call that set $S$.  Now consider $S_1$.  This is a set
of molecules that includes all of the members of $S$ as well as one
additional molecule.  Do the members of $S_1$ make up a cloud?  Surely
they are just as well suited to do so.  Now consider $S_2$\,\ldots

Because these numerous 'candidates' are equally (or nearly equally)
well suited to be clouds, we seem forced to conclude that there are
either many clouds where we supposed there to be one, or rather no
clouds at all:

\begin{squote}
No matter where we start, the complex first chosen has nothing
objectively in its favor to make it a better candidate for cloudhood
than so many of its overlappers are.  Putting the matter somewhat
personally, each one's claim to be a cloud is just as good, no better
and no worse, than each of the many others.  And, by all odds, each
complex has \emph{at least} as good a claim as any still further real
entity in the situation.  So, either \emph{all} of \emph{them} make it
or else \emph{nothing} does; in this real situation, either there are
many clouds or else there really are no clouds at all
\citep[415]{unger1980a}.
\end{squote}

The problem of the many can also arise by considering the {\em
  boundary} of a given cloud.  It is natural to suppose that a cloud
has a determinate boundary.  But if we look at the edge of the cloud,
where we suppose the boundary to be, ``we may find, side by side, or
themselves overlapping, a great many potential boundaries for
clouds\,\ldots if our alleged typical item {[}the cloud{]} is indeed
a typical cloud, then many of these candidates, millions at least, do
not fail to be clouds altogether but are clouds of some
sort'' \citep[420--421]{unger1980a}.

The pattern of argumentation is the same for both approaches to the
problem of the many.  For a given cloud, a certain set of members or a
certain boundary is supposed, and it is argued that a set or boundary
that differs minimally from the original must also make up or bound a
cloud.  The new set or boundary does not appear to differ from the
original in any relevant way; there seems no principled reason to deny
that if the first set's members make up a cloud, the second set's
members do too.  And since there are a great deal of very similar sets
and boundaries, we find ourselves threatened with a plurality of
clouds.

And of course, Unger does not rest content with applying the problem
of the many to clouds.  All ordinary objects get the same treatment;
he concludes that either there are a great many of them, or there are
none at all.  He claims, predictably, that the latter disjunct is
preferable.

\subsection{Is the problem of the many a problem for Merricks?}
\label{many-merricks}
It might seem that the problem of the many, if it makes things
difficult for chairs and other ordinary things, also causes trouble
for the chairwise arrangements upon which Merricks relies so heavily.
But there is an important disanalogy.  The problem of the many is only
a problem as long as we are unwilling to accept one of Unger's
disjuncts: that there are no chairs or that there are a plurality
where we took there to be one.  Whether or not it is part of the
meaning of `chair' that there is not an overlapping plurality of
chairs, it is simply unacceptable that this be the case.  (It is
likewise unacceptable that there be no chairs.)  But it {\em is}
acceptable, at least initially, that there be a plurality of chairwise
arrangements.  The idea that there is a plurality of different sets,
the members of which overlap and of which all are arranged chairwise,
is not particularly bizarre.  A potential difficulty for Merricks
would be in regard to his criterion of `arranged chairwise' (or
statuewise, as the case may be):

\begin{squote}
Atoms are \emph{arranged statuewise} if and only if they both have the
properties and also stand in the relations to microscopica upon which,
if statues existed, those atoms' \emph{composing a statue} would
non-trivially supervene (\citeyear[4]{merricks2001a}).
\end{squote}

If Merricks allows that there may be a plurality of statuewise
arrangements, then he is committed to the proposition that, if statues
existed, there may be pluralities of statues.  But Merricks may simply
take this as more evidence that his counterpossible conditional (``if
there were statues\,\ldots '') really is impossible.

There may be, however, a problem for Merricks with regard to the
notion of `singular thought'.  If Merricks says to me, ``those things
arranged chairwise are arranged very comfortably'', how can I know
which things he is talking about?  If there are numerous different
sets of things arranged chairwise, the chance that I am thinking of
the same things as Merricks is very low.  Are we really communicating,
then?  (Moreover, is it true that I am thinking of {\em any}
determinate set of things?  I certainly couldn't specify which
particular things I am thinking of.)

If the problem of the many is a problem for Merricks, then so much the
worse for his nihilism.  However, the problem is a problem for us as
well.

\section{Beliefs in things}
\label{u-belief}
When we examined previous versions of nihilism, we asked that their
proponents explained why, if there are no chairs, we nonetheless
believe that there are chairs.  Van Inwagen and Merricks both made a
claim to the effect that our beliefs are caused and (in some sense)
justified by arrangements of simples.  Although Merricks denies that
beliefs like ``there is a fine chair'' are strictly true, he agrees
with van Inwagen that they `get something right' in a way that beliefs
like ``there is a dancing chair'' do not.  Their explanation for our
belief that there are chairs is that they are caused (and justified)
by a `nearby' or somehow related fact---that there are simples
arranged chairwise.

It should therefore seem reasonable to demand a similar explanation
from Unger.  This explanation would be expected to take a different
form, depending on whether it accompanies the sorites paradox or the
problem of the many.  However, Unger offers no explanations.  

\subsection{Explaining our beliefs given the sorites paradox}
\label{expl-sorites}
Unger claims that our belief that there are chairs, like our beliefs
that there are other ordinary things, are not justified, `nearly as
good as true', or even coherent.  Unger says that ``terms for ordinary
things are incoherent [and] cannot apply to anything real''
\citep[147]{unger1979}.

Unger should not deny that believe that there are ordinary things.  If
our beliefs about tables and chairs are invariably false (even
incoherent), then what causes us to form these beliefs?  Why do we
believe in ordinary things to begin with?

Unlike van Inwagen and Merricks, Unger does not offer an explanation.
Having denied the existence of all ordinary things, he makes no
attempt to explain why we have so many false beliefs or what gives the
impression of coherency to our use of them in communication.  He seems
almost to revel in the strangeness of his position:

\begin{squote}
Now, it must of course be admitted that these arguments undermine the
possibility of any endeavor I should try to propose, or even the
putative thought that I should propose anything, just as all of my
putative essay is undermined.  But even so, I shall (incoherently)
propose that what we have now to do is invent new expressions which
are not inconsistent ones, and by means of which we may, to some
significant extent, think coherently about concrete reality
(\citeyear[544]{unger1980b}).
\end{squote}

I do not have an argument against the proposition that nearly all of
our language is hopelessly incoherent.  I do have a very hard time
believing that this is true, however; I'm not sure Unger believes it
himself.

\subsection{Explaining beliefs given the problem of the many}
\label{expl-many}
When presenting the problem of the many, Unger declares that he
prefers to maintain that there are no chairs, rather than that there
are pluralities of chairs.  Having made this nihilistic claim,
however, he does not offer an explanation of why we do in fact believe
that there are chairs.  However, he does not deny (at least not
explicitly) that there are things arranged chairwise, so Merricks'
explanation (see section \ref{connection}) might serve for Unger too.
We might propose, on Unger's behalf, that we believe that there are
chairs because there are things arranged chairwise.

In this case, however, I am tempted to repeat my arguments against
Merricks (section \ref{dogbush}).  I claimed that if there are things
arranged chairwise, {\em then there are chairs}.  If our belief that
there are chairs is caused by things arranged chairwise, then it is a
true, not a false, belief.

If we reject Unger's conclusion that there are no chairs, we are still
faced with a problem.  For we seem to run ourselves into the other
disjunct of the problem of the many.  If there are chairs, then there
are pluralities of chairs where we expect there to be only one.

This is an unacceptable conclusion, but not due to any explanatory
deficiency.  If we supposed that there was a plurality of chairs, then
{\em that} would explain why we believe that there are chairs.

\section{Referring to the many}
\label{refer}
Above (section \ref{many-merricks}) I mentioned that notions of
singular thought are threatened by the problem of the many.  More
should be said on this.

\ifstandalone
\bibliography{everything}
\bibliographystyle{ChicagoReedweb}
\end{spacing}
\fi
\end{document}


\chapter{Mathematics and Science}	
\section{Math}
	\TeX\ is the best way to typeset mathematics. Donald Knuth designed \TeX\ when he got frustrated at how long it was taking the typesetters to finish his book, which contained a lot of mathematics. 
	
	If you are doing a thesis that will involve lots of math, you will want to read the following section which has been commented out. If you're not going to use math, skip over this next big red section. (It's red in the .tex file but does not show up in the .pdf.)

\section{Chemistry 101: Symbols}
Chemical formulas will look best if they are not italicized. Get around math mode's automatic italicizing by using the argument \verb=$\mathrm{formula here}$=, with your formula inside the curly brackets.

So, $\mathrm{Fe_2^{2+}Cr_2O_4}$ is written \verb=$\mathrm{Fe_2^{2+}Cr_2O_4}$=\\
Exponent or Superscript: O$^{-}$\\
Subscript: CH$_{4}$\\

To stack numbers or letters as in $\mathrm{Fe_2^{2+}}$, the subscript is defined first, and then the superscript is defined.\\
Angstrom: {\AA}\\
Bullet: CuCl $\bullet$ 7H${_2}$O\\
Double Dagger: \ddag \/\\
Delta: $\Delta$\\
Reaction Arrows: $\longrightarrow$ or  $\xrightarrow{solution}$\\
Resonance Arrows: $\leftrightarrow$\\
Reversible Reaction Arrows: $\rightleftharpoons$ or $\xrightleftharpoons[ ]{solution}$ (the latter requires the chemarr package)\\


\subsection{Typesetting reactions}
You may wish to put your reaction in a figure environment, which means that LaTeX will place the reaction where it fits and you can have a figure legend if desired:
\begin{figure}[htbp]
\begin{center}
$\mathrm{C_6H_{12}O_6  + 6O_2} \longrightarrow \mathrm{6CO_2 + 6H_2O}$
\caption{Combustion of glucose}
\label{combustion of glucose}
\end{center}
\end{figure}

\subsection{Other examples of reactions}
$\mathrm{NH_4Cl_{(s)}} \rightleftharpoons \mathrm{NH_{3(g)}+HCl_{(g)}}$\\
$\mathrm{MeCH_2Br + Mg} \xrightarrow[below]{above} \mathrm{MeCH_2\bullet Mg \bullet Br}$

\section{Physics}

Many of the symbols you will need can be found on the math page (\url{http://web.reed.edu/cis/help/latex/math.html}) and the Comprehensive \LaTeX\ Symbol Guide (enclosed in this template download).  You may wish to create custom commands for commonly used symbols, phrases or equations, as described in Chapter \ref{commands}.

\section{Biology}
You will probably find the resources at \url{http://www.lecb.ncifcrf.gov/~toms/latex.html} helpful, particularly the links to bsts for various journals. You may also be interested in TeXShade for nucleotide typesetting (\url{http://homepages.uni-tuebingen.de/beitz/txe.html}).  Be sure to read the proceeding chapter on graphics and tables, and remember that the thesis template has versions of Ecology and Science bsts which support webpage citation formats. 

\chapter{Tables and Graphics}

\section{Tables}
	The following section contains examples of tables, most of which have been commented out for brevity. (They will show up in the .tex document in red, but not at all in the .pdf). For more help in constructing a table (or anything else in this document), please see the LaTeX pages on the CUS site. 

\begin{table}[htdp] % begins the table floating environment. This enables LaTeX to fit the table where it works best and lets you add a caption.
\caption[Basic Table 1]{A Basic Table: Correlation of Factors between Parents and Child, Showing Inheritance} 
% The words in square brackets of the caption command end up in the Table of Tables. The words in curly braces are the caption directly over the table.
\begin{center} 
% makes the table centered
\begin{tabular}{c c c c} 
% the tabular environment is used to make the table itself. The {c c c c} specify that the table will have four columns and they will all be center-aligned. You can make the cell contents left aligned by replacing the Cs with Ls or right aligned by using Rs instead. Add more letters for more columns, and pipes (the vertical line above the backslash) for vertical lines. Another useful type of column is the p{width} column, which forces text to wrap within whatever width you specify e.g. p{1in}. Text will wrap badly in narrow columns though, so beware.
\toprule % a horizontal line, slightly thicker than \hline, depends on the booktabs package
  Factors &  Correlation between Parents \& Child & Inherited \\ % the first row of the table. Separate columns with ampersands and end the line with two backslashes. An environment begun in one cell will not carry over to adjacent rows.
  \midrule % another horizontal line
Education & -0.49 & Yes \\ % another row
Socio-Economic Status & 0.28 & Slight \\
Income & 0.08 & No\\
Family Size & 0.19 & Slight \\
Occupational Prestige &0.21 & Slight \\
\bottomrule % yet another horizontal line
\end{tabular}
\end{center}
\label{inheritance} % labels are useful when you have more than one table or figure in your document. See our online documentation for more on this.
\end{table}

	\clearpage 
%% \clearpage ends the page, and also dumps out all floats. 
%% Floats are things like tables and figures.
   

\chapter*{Conclusion}
         \addcontentsline{toc}{chapter}{Conclusion}
	\chaptermark{Conclusion}
	\markboth{Conclusion}{Conclusion}
	\setcounter{chapter}{4}
	\setcounter{section}{0}
	
Here's a conclusion, demonstrating the use of all that manual incrementing and table of contents adding that has to happen if you use the starred form of the chapter command. The deal is, the chapter command in \LaTeX\ does a lot of things: it increments the chapter counter, it resets the section counter to zero, it puts the name of the chapter into the table of contents and the running headers, and probably some other stuff. 

So, if you remove all that stuff because you don't like it to say ``Chapter 4: Conclusion'', then you have to manually add all the things \LaTeX\ would normall do for you. Maybe someday we'll write a new chapter macro that doesn't add ``Chapter X'' to the beginning of every chapter title.

\section{More info}
And here's some other random info: the first paragraph after a chapter title or section head \emph{shouldn't be} indented, because indents are to tell the reader that you're starting a new paragraph. Since that's obvious after a chapter or section title, proper typesetting doesn't add an indent there. 


%If you feel it necessary to include an appendix, it goes here.
    \appendix
      \chapter{The First Appendix}
      \chapter{The Second Appendix, for Fun}


%This is where endnotes are supposed to go, if you have them.

  \backmatter % backmatter makes the index and bibliography appear properly in the t.o.c...

% Make my bibliography be called "Bibliography" and not "References" (or "Works Cited" or...):
% \renewcommand{\bibname}{Works Cited}
    \bibliographystyle{ChicagoReedweb} % there are a variety of styles available; 
% replace ``plainnat'' with the style of choice. You can refer to files in the bsts or APA 
% subfolder, e.g. 
% \bibliographystyle{APA/apa-good}  % or
% \bibliographystyle{bsts/mla-good} 

% if you're using bibtex, the next line forces every entry in the bibtex file to be included
% in your bibliography, regardless of whether or not you've cited it in the thesis.
    %\nocite{*}
    \bibliography{everything}

% Finally, an index would go here... but it is also optional.
\end{document}
