% This is the Reed College LaTeX thesis template. Most of the work 
% for the document class was done by Sam Noble (SN), as well as this
% template. Later comments etc. by Ben Salzberg (BTS). Additional
% restructuring and APA support by Jess Youngberg (JY).
% Your comments and suggestions are more than welcome; please email
% them to cus@reed.edu
%

\documentclass[12pt,twoside]{reedfancy}
\usepackage{standalone}
\usepackage{anyfontsize}
\usepackage{graphicx,latexsym} 
\usepackage{amssymb,amsthm,amsmath}
\usepackage{longtable,booktabs,setspace} 
\usepackage{chemarr} %% Useful for one reaction arrow, useless if you're not a chem major
\usepackage{url}
\usepackage{natbib}
% \usepackage{times} % other fonts are available like times, bookman, charter, palatino

\usepackage{enumitem}
\setcitestyle{aysep={}}
\synctex=1

\DeclareSymbolFont{symbolsC}{U}{txsyc}{m}{n}
\DeclareMathSymbol{\strictif}{\mathrel}{symbolsC}{74}
\DeclareMathSymbol{\boxright}{\mathrel}{symbolsC}{128}

\newenvironment{squote}{\begin{quote}\begin{singlespace}}{\end{singlespace}\end{quote}}

\newcommand{\stager}[4]%
{%
	\begin{spacing}{1}%
	\vspace{0pt}
		\begin{description}[style=nextline, noitemsep, parsep=0pt, topsep=0pt, leftmargin=15mm, itemindent=-10mm, font=\mdseries]
			\item[\textsc{#1} \emph{#2}] #3
			\item[]%
			\begin{flushright}#4\end{flushright}
		\end{description}%
	\end{spacing}%
}

\newcommand{\stage}[3]%
{%
	\begin{spacing}{1}%
	\vspace{0pt}
		\begin{description}[style=nextline, parsep=0pt, leftmargin=15mm, itemindent=-10mm, font=\mdseries]
			\item[\textsc{#1} \emph{#2}] #3
		\end{description}%
	\end{spacing}%
}

% moved to reedfancy.cls:
%\def\thetitle{oink}
%\renewcommand{\firstmark}{\thetitle}
%\newcommand{\chapterpig}[1]{\def\thetitle{#1}}

\title{A World Without Us}
\author{Alexander A. Dunn}
% The month and year that you submit your FINAL draft TO THE LIBRARY (May or December)
\date{May 2012}
\division{Philosophy and Other Things}
\advisor{Paul Hovda}
\department{Philosophy}

\setlength{\parskip}{0pt}

%%%%%%%%%%%%%%%%%%%%%%%%%%%%%%%%%%%%%%%%%%%%

\begin{document}

  \maketitle
  \frontmatter % this stuff will be roman-numbered
  \pagestyle{empty} % this removes page numbers from the frontmatter

% is this a good idea:?
%\begin{spacing}{1.35}

% Acknowledgements (Acceptable American spelling) are optional
% So are Acknowledgments (proper English spelling)
    \chapter*{Acknowledgements}
	I want to thank Peter Unger for infuriating me. Everything he has written is stupendously false.

% The preface is optional
% To remove it, comment it out or delete it.
    \chapter*{Preface}
	This is an example of a thesis setup to use the reed thesis document class.

    \tableofcontents
% if you want a list of tables, optional
   % \listoftables
% if you want a list of figures, also optional
   % \listoffigures

% The abstract is not required if you're writing a creative thesis (but aren't they all?)
% If your abstract is longer than a page, there may be a formatting issue.
    \chapter*{Abstract}
	The preface pretty much says it all.

  \mainmatter % here the regular arabic numbering starts
  \pagestyle{fancyplain} % turns page numbering back on

%The \introduction command is provided as a convenience.
%if you want special chapter formatting, you'll probably want to avoid using it altogether

    \chapter*{Introduction}
         \addcontentsline{toc}{chapter}{Introduction}
	\chaptermark{Introduction}
	\markboth{Introduction}{Introduction}
	% The three lines above are to make sure that the headers are right, that the intro gets included in the table of contents, and that it doesn't get numbered 1 so that chapter one is 1.
	
	Welcome to the \LaTeX\ thesis template. If you've never used \TeX\ or \LaTeX\ before, you'll have an initial learning period to go through, but the results of a nicely formatted thesis are worth it for more than the aesthetic benefit: markup like \LaTeX\ is more consistent than the output of a word processor, much less prone to corruption or crashing and the resulting file is smaller than a Word file. While you may have never had problems using Word in the past, your thesis is going to be about twice as large and complex as anything you've written before, taxing Word's capabilities. If you're still on the fence about  using \LaTeX, read the Introduction to LaTeX on the CUS site as well as skim the following template and give it a few weeks. Pretty soon all the markup gibberish will become second nature.

\section{Why use it?}
	
\LaTeX\ does a great job of formatting tables and paragraphs. Its line-breaking algorithm was the subject of a PhD.\thinspace thesis. It does a fine job of automatically inserting ligatures, and to top it all off it is the only way to typeset good-looking mathematics.

\section{Who should use it?}

Anyone who needs to use math, tables, a lot of figures, complex cross-references, IPA or who just cares about the final appearance of their document should use \LaTeX. At Reed, math majors are required to use it, most physics majors will want to use it, and many other science majors may want it also.

%\chapter{A story about things}
%\chapterpig{A Story About Things}
%\documentclass[11pt]{article}
\usepackage[margin=1.25in]{geometry}
\geometry{letterpaper}
\usepackage{graphicx}
%\usepackage{tipa}
%\usepackage{exaccent}
%\usepackage{txfonts}
%\usepackage{pxfonts}
\usepackage{enumitem}
%\usepackage{amssymb}
\usepackage{amsmath}
\usepackage{epstopdf}
\usepackage{setspace}
\usepackage{natbib}
\setcitestyle{aysep={}}
\synctex=1

\DeclareSymbolFont{symbolsC}{U}{txsyc}{m}{n}
\DeclareMathSymbol{\strictif}{\mathrel}{symbolsC}{74}
\DeclareMathSymbol{\boxright}{\mathrel}{symbolsC}{128}

\newenvironment{squote}{\begin{quote}\begin{singlespace}}{\end{singlespace}\end{quote}}

\newcommand{\stager}[4]%
{%
	\begin{spacing}{1}%
	\vspace{0pt}
		\begin{description}[style=nextline, noitemsep, parsep=0pt, topsep=0pt, leftmargin=15mm, itemindent=-10mm, font=\mdseries]
			\item[\textsc{#1} \emph{#2}] #3
			\item[]%
			\begin{flushright}#4\end{flushright}
		\end{description}%
	\end{spacing}%
}

\newcommand{\stage}[3]%
{%
	\begin{spacing}{1}%
	\vspace{0pt}
		\begin{description}[style=nextline, parsep=0pt, leftmargin=15mm, itemindent=-10mm, font=\mdseries]
			\item[\textsc{#1} \emph{#2}] #3
		\end{description}%
	\end{spacing}%
}

\title{Alien Explorers and Conceptual Schemes}
\author{Alexander A. Dunn}
\begin{document}
%\maketitle
%\begin{spacing}{1}

\section{Intuitions}
Gareth Evans once claimed that ``with pliant enough intuitions you can swallow anything in philosophy''~(\citeyear[192]{evans1973}). That's an hypothesis that demands testing. So I will claim that, if humans did not exist, neither would dogs, cats, mountains, forests, lakes, rivers, pigs, apples, shrubs, stars, and pretty much everything else. But before I try to get you to swallow this, I'm going to tell a story.

\subsection{Alien explorers meet a metaphysician}
{\em (Our protagonist sits in a chair in front of the fire.)} \\

\stager{Protagonist}{}{Suppose some years in the future we explore a distant planet very unlike our own. There seems to be organic life, but whether it is intelligent or not is hard to say. There seem to be artificial modifications to the environment, but again it's difficult to say what is natural and what isn't. But our intrepid scientists refuse to be baffled by this strange world. After some years of research they manage to pin down the physical processes and produce a model of the planet with really quite good predictive capabilities.}{{\em (A child's head droops.)}}

\stager{Protagonist}{(continuing)}{We've classified some phenomena as organisms with their set of biological underpinnings, and we've classified others as natural though non-living processes, such as weather patterns and geological change. There might be some things we've overlooked, but it doesn't look like there are going to be many more surprises. So I think it's safe to say that we've recognized most of what's on this planet, don't you?}{{\em (Vague nods. A sleeping \textsc{metaphysician} stirs briefly.)}}

\stager{Protagonist}{}{ Now I'm afraid I must admit that I haven't been entirely honest with you. The explorers in this story are not us, but an intelligent alien species. The planet is not distant at all; they are exploring our own Earth. And yet I think we must agree that we were correct---}{{\em (The metaphysician awakes with a start.)}}

\stage{Metaphysician}{(rising angrily)}{Now look here! These fool aliens don't recognize their own ignorance, let alone half of what exists on Earth! They forgot about adzes and Axminsters, boats and books, catalogs and cups, doors and dumbwaiters, earrings and elegance! Their ontology would fit in a knapsack!}

And so the question is asked to you, readers: are the things that went unseen by the aliens {\em really there?}

%\bibliography{everything}
%\bibliographystyle{ChicagoReedweb}
%
%\end{spacing}
\end{document}

\chapter{Denials}
\label{deny}
\chapterpig{Denials}
\documentclass[11pt]{article}
\usepackage{standalone} \newif\ifstandlone \standalonetrue
\usepackage[left=1.75in, right=1.75in, top=1.25in, bottom=1.25in]{geometry}
\geometry{letterpaper}
\usepackage{graphicx}
%\usepackage{tipa}
%\usepackage{exaccent}
%\usepackage{txfonts}
%\usepackage{pxfonts}
\usepackage{enumitem}
%\usepackage{amssymb}
\usepackage{amsmath}
\usepackage{epstopdf}
\usepackage{setspace}
\usepackage{natbib}
\setcitestyle{aysep={}}
\synctex=1

\DeclareSymbolFont{symbolsC}{U}{txsyc}{m}{n}
\DeclareMathSymbol{\strictif}{\mathrel}{symbolsC}{74}
\DeclareMathSymbol{\boxright}{\mathrel}{symbolsC}{128}

\newcommand{\stager}[4]%
{%
	\begin{spacing}{1}%
	\vspace{0pt}
		\begin{description}[style=nextline, noitemsep, parsep=0pt, topsep=0pt, leftmargin=15mm, itemindent=-10mm, font=\mdseries]
			\item[\textsc{#1} \emph{#2}] #3
			\item[]%
			\begin{flushright}#4\end{flushright}
		\end{description}%
	\end{spacing}%
}

\newcommand{\stage}[3]%
{%
	\begin{spacing}{1}%
	\vspace{0pt}
		\begin{description}[style=nextline, parsep=0pt, leftmargin=15mm, itemindent=-10mm, font=\mdseries]
			\item[\textsc{#1} \emph{#2}] #3
		\end{description}%
	\end{spacing}%
}

\newenvironment{squote}{%
	\begin{spacing}{1}
	\begin{list}{}{%
	\setlength{\labelwidth}{0pt}%
	\rightmargin\leftmargin%
	}
	%\begin{singlespace}%
	\item\relax
	}{%
	%\end{singlespace}%
	\end{list}%
	\end{spacing}
	}

\newenvironment{inq}{\vspace{0pt}%
	\begin{list}{}%
	{\setlength\labelwidth{0pt}%
	\setlength\leftmargin{2.5\oddsidemargin}%
	\setlength\rightmargin{\leftmargin}}
	\begin{spacing}{1}
	\item[]%
	}{
	\end{spacing}
	\end{list}
	\vspace{10pt}
	%\noindent%
	}

\title{Denying the Ordinary}
\author{Alexander A. Dunn}
\begin{document}
\ifstandalone
\maketitle
\begin{spacing}{1.5}
\fi

%\begin{inq}
%The philosophical quest must start somewhere. It needs a set of beliefs about what the world is like. Without some attitudes, perceptions, beliefs, or theories to start with, it would have nothing to reflect on.~\citep[16]{stroud2000a}
%\end{inq}

%	\begin{inq}\textbf{quine}, v. To deny resolutely the existence or importance of something real or significant.
%	\footnote{This is of course from the {\em Philosophical Lexicon}~\citep{dennett2008}.}%}
%	\end{inq}%

%\section{How to defy common sense}
%\label{denials}
\noindent Every so often a philosopher will claim that some aspect of what we take to be our world is somehow illusory, bogus, or simply nonexistent. This sort of denial will take various forms. One might claim that a certain phenomenon does indeed find expression in the world, but that it is somehow subjective; without humans to experience it, there would be no such phenomena. Or one might deny that some ordinary object of experience is actually non-existent. Peter van Inwagen claims that tables do not exist. He recognizes, of course, that people talk and have beliefs about (what they take to be) tables, so he must find a way to explain our beliefs in tables in terms of those things that he does believe to exist. Finally, one might claim that some things are non-existent, and that we in fact {\em don't} really talk and have beliefs about them. This is the sort of denial we make of the existence of ghosts. Unlike the philosopher who denies the existence of tables, we who deny the existence of ghosts don't have to explain people's beliefs in ghosts---people simply don't {\em have} coherent beliefs about them, because they are entirely unreal. Few philosophers make denials of this sort about ``real or significant'' things, because the beliefs we have about such things (tables, chairs, people, custard) are deeply integrated into our daily lives, and to say that these beliefs are utterly incoherent is to risk crossing the line into nonsense.

\section{Kinds of denials}
In this section I will briefly look at each of these types of denial and how the philosophers making them manage to explain the beliefs that people have about the objects of the denials.

\subsection{The relegation}
\label{relegate}
Among the various phenomena we observe in the world, it can be tempting to draw a distinction between those that we somehow imprint upon the world and those that are independent of any human experience. The former are `subjective' while the latter are `objective' or absolute:
\begin{squote}
Whatever is due only to us and to our own ways of responding to and interacting with the world does not reflect or correspond to anything present in the world as it is independently of us. The aim of an ``absolute'' conception, then, is to form a description of the way the world is, not just independently of its being believed to be that way, but independently, too, of all the ways in which it happens to present itself to us human beings from our particular standpoint within it\,\ldots\,[So we] form some conception of that independent reality and come to understand parts or aspects of our original conception of the world as not representing it as it is. If we see them as products or reflections of something peculiar to human experience or to the human perspective on the universe, we assign them a merely ``subjective'' or dependent status and eliminate them from our conception of the world as it is independently of us~\citep[30--31]{stroud2000a}.
\end{squote}

A philosopher who adheres to this distinction might claim that our conception of the world as colored does not represent the world as it is independently of us. Colors, she would claim, are not objectively real. She allows that they are subjectively real, of course. People {\em do} see colors; we have color vision while some species do not. Because of our color vision, we come to believe that the things we see are colored. A philosopher who is skeptical of the objective reality of color ``cannot deny that we perceive many different colours or that we believe physical objects to be coloured''~\citep[145]{stroud2000a}. What the skeptic has to claim is something to the effect that, while we see things {\em as} colored, things are not {\em themselves} colored. The red color of a tomato, on this view, obtains only in our perception of the tomato; there is nothing {\em in} the tomato that is the redness (other species may not see the redness when they see the tomato).% The belief most commonly motivating this type of view, according to Stroud, is a belief that ``the world as it is independently of us'' is simply the world described by an ideal physics: ``physical science can describe every aspect of the figure or shape and the number and motions of the bodies that make up the world. We have words for what we think of as the colours, odours, and tastes of those objects as well, but those words stand for nothing that exists in reality''~(\citeyear[8]{stroud2000a}).

Someone who makes this argument does not, therefore, deny that we perceive colors, or that we believe that things are colored. To claim that we {\em actually don't} think things are colored---that we don't actually believe that tomatoes are red---would obviously be false. We certainly do believe that tomatoes (at least most of them) are red; this is what makes the denial of color interesting. If the philosopher claimed that colors aren't objectively real {\em and that we don't believe them to be}, we ought to wonder why the philosopher is even bothering to make the argument.%
%
%\footnote{}
%
\ It would be like the claim that ghosts don't exist; this is not controversial or interesting, because we don't believe that ghosts exist.

So the philosopher who is denying the objective reality of color must ``recognize the presence in the world of perceptions of and beliefs about the colours of things''~\citep[199]{stroud2000a}. The challenge then is to explain why we do have these perceptions and beliefs. For example, a philosopher who believes that only the world of physics is objectively real must explain the color phenomena in the vocabulary of the physical sciences. (And before this can be attempted, the question arises as to what this vocabulary is: ``Physical science changes. Physicists do not just change their minds as they learn more and more about the world; the very conception of what is to be included in physics changes''~\citep[53]{stroud2000a}. So the philosopher relegating colors---or anything else---to subjective reality must have a clear idea of what is left in objective reality.)%

\subsection{The paraphrase}
\label{paraphrase}
The second sort of denial goes further in denying any kind of reality at all to the subject of inquiry. The philosopher above denied that colors were `objectively real', but not that they were `subjectively real'; she did not deny that we do at least perceive colors. But there are some things that are taken by some philosophers to be neither objectively or subjectively real; they simply do not exist.% A philosopher in a cynical mood might deny that love exists. Depending on how recently she was jilted, however, she might not deny that an utterance of ``there is love in the world'' is true. What she would do is paraphrase it as ``there are people in the world who are in symmetrical or asymmetrical relations with other people that can be described as `loving relationships'\,'' (assuming that the cynical philosopher doesn't deny the existence of people as well). If it can be determined that the original speaker meant something along these lines, then our philosopher has performed `the paraphrase' on love. (The philosopher may or may not go on to claim that the loving relationship is `subjective'; that it would not exist without humans to instantiate it.)
%\ (The original speaker might reject this paraphrase, of course. She might insist that $\exists x(x=Love)$. In this situation we might perform a resolute denial of the sort described in section~\ref{resolute} below and say that there is no existing object that is love.)

\ Peter van Inwagen denies the existence of tables, chairs, apples, and every other inanimate composite object.%
%
\footnote{The notion of `composite' will be discussed below in section~\ref{scq}.}
%
\ He takes pains to make clear that his denial of these things is not a relegation of tables and chairs to `subjective reality'. He wants to claim that such things do not exist in any way, subjective or objective:
\begin{squote}
I want to do what I can to disown a certain apparently almost irresistible characterization of my view, or of that part of my view that pertains to inanimate objects. Many philosophers, in conversation and correspondence, have insisted, despite repeated protests on my part, on describing my position in words like these: ``Van Inwagen says that tables are not real''; ``\ldots\,not true objects''; ``\ldots\,not actually {\em things}''; ``\ldots\,not substances''; ``\ldots\,not unified wholes''; ``\ldots\,nothing more than collections of particles.'' These are words that darken counsel. They are, in fact, perfectly meaningless. My position vis-\`{a}-vis tables and other inanimate objects is simply that there {\em are} none~(\citeyear[99]{inwagen1995}).
\end{squote}
Van Inwagen asserts, quite seriously, that ``there are no tables or chairs or any other visible objects except living organisms''~(\citeyear[1]{inwagen1995}). This is a somewhat more bold claim than that of the philosopher skeptical of color. She at least granted that we do see colors, even if we don't actually see things that are (objectively) colored. If, as van Inwagen claims, the only {\em visible} objects are living organisms, then we certainly can't {\em see} tables at all. But just as our color skeptic could not claim that we don't believe in colors, van Inwagen cannot deny that we at least {\em believe} there to be apples. We have many beliefs about what we take to be apples. We believe that they grow on trees, and go well with many types of cheese (which we also believe to exist). Van Inwagen has, therefore, a rather daunting task: he must explain our beliefs about apples (and about cheese, and tables, and chairs, and about everything else he denies) in terms of whatever it is that he does claim to exist (living organisms and the basic particles that make up the physical universe).
%
%\footnote{Van Inwagen assumes, without defense, ``that matter is ultimately particulate\,\ldots\,every material thing is composed of things that have no proper parts: `elementary particles' or `mereological atoms' or `metaphysical simples'\,''~(\citeyear[5]{inwagen1995}). Ted Sider takes him to task for this assumption~(\citeyear{sider1993}), claiming that the possibility of `gunk'---the possibility that the matter of the world is not fundamentally particulate but infinitely divisible---falsifies van Inwagen's thesis. I think it may be possible for van Inwagen to adapt to a gunky world (see Section~\ref{brute}, note~\ref{gunk}), but I think van Inwagen's thesis is false either way.}
%

Van Inwagen does not attempt to deny that we have beliefs about what we take to be apples, cheeses, \&c. Indeed, he admits that ``when people say things in the ordinary business of life by uttering sentences that start `There are chairs\,\ldots ' or `There are stars\,\ldots ', they very often say things that are literally true''~(\citeyear[102]{inwagen1995}). By conceding this, he distances himself from a skeptical philosopher (see section~\ref{unger} below) whose denial of apples et.\ al.\ is on par with a denial of ghosts. When someone has a belief about what they took to be a ghost, we their belief is not about whatever actually caused their fright; it is most charitable to say that their belief is really about nothing at all (I explain why in section~\ref{resolute}). Van Inwagen, when denying that we have beliefs about apples, appears to maintain that the beliefs that we erroneously take to be about apples are not beliefs about {\em nothing}. They are not empty; they are rather beliefs about something {\em else}, something other than what we took them to be about (apples). Van Inwagen accordingly recognizes the need to explain what our beliefs really are about.

What van Inwagen says is that, when saying things like ``Some chairs are heavier than some tables'', if we are talking about anything at all, we are talking about simple particles ``that are arranged chairwise and\,\ldots\,that are arranged tablewise''~(\citeyear[109]{inwagen1995}). He develops an elaborate `paraphrasing strategy' that is an attempt to show that many (if not all) propositions expressed by ``There are chairs\,\ldots '' and related sentences do not actually entail the existence of chairs.

I do not think van Inwagen's defense is ultimately successful (see section~\ref{pigletwise}), but it is precisely the sort of defense required when denying the existence of ordinary things like tables and chairs. He cannot claim, without courting absurdity, that we {\em don't} believe there to be tables and chairs in the world. We do believe so. We might, of course, be {\em wrong} about the existence of tables and chairs; ``from the fact that we believe a certain thing it does not follow that it is true''~\citep[21]{stroud2000a}. Nevertheless it is true that we believe there to be such things, and van Inwagen needs to explain the source of this belief. If there are no tables and chairs, then we must be able to understand the beliefs that we thought to be about the furniture to be about something else instead, and we need a story about what that something else is.
% ``you cannot hope to explain something unless you grant that there is such a thing and you have at least some idea of what it is''~\citep[97]{stroud2000a}.
\ As I said, I don't think van Inwagen's explanation is a good one. But before discussing why, we should consider the third and most radical sort of denial.

\subsection{The resolute denial}
\label{resolute}
The third kind of denial is that which I likened to a denial of ghosts. When I deny the existence of ghosts, I also deny that people talk about and have beliefs about ghosts. Even if someone says ``let me tell you about the ghost I saw last night!'', I can maintain that there is {\em nothing} in particular that they are talking about. They are certainly not talking about a certain ghost; there are no ghosts, and never have been.

Of course I cannot deny that some people believe that ghosts exist. Some people do believe this. But my resolute denial of the existence of ghosts does not prevent me from explaining someone's belief that there are ghosts. This someone, for example, might tell me that she saw a ghost on the landing. I walk out and see a light from a high window flickering strangely on the wall. (If I squint, the pattern of the light looks almost humanoid.) So I tell her that what she thought was a ghost was really just a curious play of the light. I am not here conceding that she had a belief {\em about} a ghost. On the contrary, I have tried to show her that her belief was about anything but a ghost. There was in fact no ghost that she could have formed a belief about; ``if we show that what a frightened person saw in the attic on a particular occasion was a rippling reflection of the moon through the window, we implicitly deny the presence of a ghost in giving the explanation of the person's belief and fear''~\citep[76]{stroud2000a}. We do not deny that people have beliefs about what they take to be ghosts; what we deny is that they are correct in taking their beliefs to be about ghosts.

Likewise, suppose she and I are looking in on an empty, well-lit room. Suddenly she points and cries, ``Look, a ghost!'' In this case there is nothing in the room that I can assume to have caused this belief. There are no reflections of the moon or curious plays of the light; as far as I can tell, she is pointing at nothing.

\stage{Me}{}{What on earth do you mean?}

\stage{Her}{(pointing)}{That ghost, there! See?}

If this continues, the most probable explanation is that she is having an hallucination. She has what she takes to be a belief about a ghost in the room. There is no ghost in the room, so what is her belief about? Likewise, if someone says she saw a ghost in the attic, what should we say her belief is really about?

In Stroud's example above, we might say that the person is frightened of the reflection of the moon. This is similar to how we might talk about a child's night terrors: ``she was afraid of the chair in her room (she thought it was a monster).'' Looking at things this way, the person's beliefs is {\em about} something, but it is something very different from what they took it to be about. I think, however, that this is not a fully accurate characterization of the object of the person's belief. It may be true that the reflection of the moon {\em caused} her belief (which caused her fright), but it would be at least a little misleading to say that Mrs --------- is frightened of the reflection of the moon. This is misleading because, if we succeed in showing her that there {\em is} no ghost in the attic (only a reflection of the moon), she will not still be afraid. When she understands that her belief in the ghost was mistaken, she will see that there is {\em nothing} to be afraid of. The belief that caused her fright, that there was a ghost in the attic, was in fact a belief about nothing at all. (The same goes for the person who sees the play of light on the wall and believes that there is a ghost.)

If someone claims to see a ghost in an empty, well-lit room, is her belief actually about the {\em hallucination}\,? Again, charity demands that we not say this. If she is convinced that nobody else sees a ghost, she may recognize that she was hallucinating. Were she to come to this conclusion, she would no longer be afraid of what she saw (though she will no doubt be afraid that she is going mad). The belief that she took to be about a ghost was in fact about nothing at all.

\section{Unger's nihilism}
\label{unger}
Just as resolutely as we denied the existence of ghosts, so Peter Unger has denied the existence of such things as ``tables and chairs and spears\,\ldots\,swizzle sticks and sousaphones\,\ldots\,stones and rocks and twigs, and also tumbleweeds and fingernails''~(\citeyear[117]{unger1979}). He does not consider them merely `subjectively real' as opposed to objectively so---like van Inwagen, he claims that they simply do not exist. He comes to this conclusion from a different direction, however. As we will see, van Inwagen's denial of the existence of `ordinary things' is a consequence of his theory of composition (under what conditions some things compose another thing). Unger, on the other hand, draws his conclusion from an application of the sorites paradox:%
%
\footnote{Unger also motivates his nihilism by way of `the problem of the many'. We will examine this problem in section~\ref{many}.}
%
\begin{squote}
Consider a stone, consisting of a certain finite number of atoms. If we or some physical process should remove one atom, without replacement, then there is left that number minus one, presumably constituting a stone still\,\ldots\,after another atom is removed, there is that original number minus two; so far, so good. But after that certain number has been removed, in similar stepwise fashion, there are no atoms at all in the situation, while we must still be supposing that there is a stone there. But as we have already agreed, if there is a stone present, then there must
be some atoms\,\ldots\,I suggest that any adequate response to this contradiction must include\,\ldots\,the denial of the existence of even a single stone.~\citep[121--122]{unger1979}
\end{squote}

Having made this denial, Unger must either explain how our beliefs about stones should be understood (van Inwagen has his paraphrasing strategy) or he must deny that we really {\em do} have any beliefs about stones. It appears that he selects the latter option: Unger seems to claim that, like the person who thought they had a belief about a ghost, we are wrong to think that we have any coherent beliefs about stones or any other ordinary things. Unger says that, like `ghost', our ``terms for ordinary things are incoherent [and] cannot apply to anything real''~\citep[147]{unger1979}. A consequence of this is that our language and thought concerning all such things is directed toward {\em nothing at all}: ``it may well be that I have never {\em thought of} any stones at all, or tables, or even human hands. If that is so, then it would seem that {\em a fortiori} I do not {\em know} anything {\em about these entities}, however commonly I might otherwise suppose''~(\citeyear[458]{unger1980a}).

This all seems very strange. Concerning ghosts, ``it is difficult even to find a fully coherent belief that might be exposed as false; we discover, at best, obscurity or perhaps confusion\,\ldots\,do we really understand what sort of thing a ghost is supposed to be''~\citep[76]{stroud2000a}? If someone tries to tell me about the ghost that visited him the previous night, it does not seem unjust to say that he doesn't really know what he is talking about. But can this be extended to some of the most common objects of experience?

When we denied the existence of ghosts, we denied also others' beliefs in them. We did not, however, deny that people have beliefs which they take to be about ghosts. But we were able to show that these beliefs were not {\em about} ghosts; in most cases they were about nothing at all. Likewise, Unger cannot deny that we have beliefs that we take to be about tables, chairs, and all the other things that he denies exist. If our beliefs about tables and chairs are really beliefs about nothing at all, there are two questions that must be answered: first, what causes us to form these beliefs?\ and second, if the utterances containing these empty terms are about nothing at all, how do we manage to communicate so effectively using them?

\subsection{Causes of belief}
\label{unger-cause}
People who believe in ghosts probably do so because they have unreflectively embraced the superstitions of their culture. They may initially come to believe that ghosts exist on the testimony of other people---older siblings, perhaps---or by reading too many ghost stories. Much as Catherine in Jane Austen's {\em Northanger Abbey} jumps to the most macabre conclusions as a result of having absorbed too many gothic novels, so might our gullible reader of ghost stories interpret such innocent phenomena as reflections of the moon as ghostly assailants. Those of us who have not taken our cues from fiction would be more likely to recognize such phenomena as tricks of the light. Even if we were to see something that was definitely {\em not} a trick of the light, we would sooner attribute it to an hallucination than countenance the possibility of ghosts. Suppose {\em you} saw what you took to be a ghost in an empty, well-lit room.%
%
%\begin{squote}
%She rose, not as if she had heard me, but with an indescribable grand melancholy of indifference and detachment, and, within a dozen feet of me, stood there as my vile predecessor. Dishonored and tragic, she was all before me; but even as I fixed, and, for memory, secured it, the awful image passed away~\citep[58]{james1991}.
%\end{squote}
%
\ Most of us would still, even if presented with such a vision, {\em refuse to believe in ghosts}. This is because we know that the probability of there being such spirits is far less than the probability of us experiencing cracks in our sanity. Undermining my belief that ghosts don't exist would require a great deal---for example, my friend and I both seeing the {\em same} apparent ghost at the same time, and knowing that we were each experiencing the same vision. (Even then, we would want further confirmed sightings to convince us that we weren't, in fact, crazy.)

If this is an accurate characterization of our beliefs concerning ghosts, it is a very different characterization than one we might give of how we learn about and come to believe in chairs. Chairs are not something that children learn about from stories. A child learns what a chair is as an answer to the question, ``What is {\em that?}\,'' Let us suppose that the child is pointing at a chair in the center of a well-lit room containing no other furniture. The chair is clearly visible. If someone were to believe they were pointing at a ghost in a similarly well-lit situation, we could safely assume that they would be experiencing a hallucination. Hallucinations are not shared experiences; if one person is hallucinating a ghost, nobody else can see that ghost, not even if they were {\em also} hallucinating a ghost (they can't both hallucinate the {\em same} ghost). Such a ghost sighting, therefore, would necessarily be experienced by a single person. A chair, on the other hand, can be sighted by multiple people simultaneously. The parent sees the same chair that the child sees. This is what allows the parent to answer the child's question (``That's a chair'') and this is how the child learns about chairs. We learn about chairs by coming upon them in the world, and being {\em told} what it is we have come upon. This brings us to communication.

\subsection{Communication and incoherence}
\label{unger-comm}
A consequence of Peter Unger's thesis is that most of our communication is empty of content, if not entirely incoherent. For me to warn my friend of a low-hanging tree branch, I must refer her to it. For her to heed my warning, she must recognize my intention to refer to the branch in question, and, as a result of that intention, herself come to think of the branch (and then duck). But according to Unger, ``when we are under the impression that we are thinking about an object in the world\,\ldots\,our impression is mistaken''~(\citeyear[149]{unger1979}). Unger is denying that I can ever possibly warn my friend about a low-hanging branch.

This seems plainly false. For, after all, my friend ducked. She heard my warning and avoided the low-hanging branch. Unless we are to suppose that it was sheer luck that she moved her head in time to dodge the branch, we must conclude that I did, in fact, warn my friend. If I did, in fact, warn my friend, then we both thought of the branch. %(It doesn't really matter how Unger argues for the impossibility of such communication; it {\em does} succeed.)% (I want to set it down as an axiom that \textsc{communication occurs}.)

That ought to be enough to prove Unger wrong. Just to be thorough, however, let's return to our well-lit room. There is a chair in the center of it. A child and her parent enter.

\stage{Child}{(pointing at the chair)}{What's that?}

Now, if the parent is going to be in a position to understand the question, she must recognize the intention of the child to refer to the chair; recognizing this intention, she comes herself to think of the chair. Only if this process occurs can the parent know what the child means by ``that''. Having performed this feat, the parent can then tell the child ``That's a chair''. If the child was hallucinating the chair (if the room was really empty), then the parent could not recognize the referential intention of the child. She could not recognize a referential intention, because that would require her to think of the object that the child is thinking of (and intending to refer to). But as I have argued, the child, if hallucinating a chair, has a belief about {\em nothing} in particular. The parent cannot come to think {\em of} the child's hallucination, at least not as a result of recognizing the child's referential intention (she may come to think {\em that} her child is hallucinating, but that's a different matter). Rather, she will look at an empty room and say

\stage{Parent}{}{What's what?}

But if the room is not empty, and the child is not hallucinating, then the parent {\em will} recognize the child's referential intention. If the child is referring to a chair, then the parent will say ``That's a chair''.

%And yet somehow Unger maintains that the kind of object picked out by `chair' is ``never instanced''~(\citeyear[147]{unger1979}). That is, the word ``chair'' necessarily has no application in the world. But suppose that this happens instead:
%
%\stage{Child}{}{I love the color of this chair!}
%
%\stage{Parent}{}{I painted it just for you.}

%\noindent %
How is Unger supposed to explain their communication? The child, according to Unger, has no determinate thought; she is certainly not thinking of a {\em chair}. Nor is the parent thinking of a chair. Unger might say that, just as people see things (reflections of the moon) that they mistakenly take to be of a ghost, so the child and parent are seeing something that each mistakenly takes to be a chair.

But now what could they be seeing? This is a well-lit, sparsely furnished room. There is no chance of curious plays of the light or visual tricks that might deceive the child and her parent. If the child claimed to be seeing a ghost in such a room, then (as above) we should have to say that she is hallucinating. Is the child then hallucinating a chair? If this were the case, then (as above) we should not expect the parent to understand the referential intention. Yet she apparently does. Are {\em both} of them hallucinating? That would allow us to say that communication appears to succeed when in fact it does not. But for the parent and child to have such similar hallucinations is incredibly unlikely. It may happen on rare occasions, but to say that such a coincidence is actually a daily occurrence is absurd. There is clearly {\em something} that the parent and child are communicating about. The default assumption is that it is a chair.

One sympathetic to Unger's thesis might admit that they are communicating about something, but deny that the subject of their communication is a chair. This philosopher would take refuge in the notion of `loose truth'. She would maintain that it is strictly false that there is a chair in the room, but that it is loosely true; it is close enough to the truth for practical purposes. These practical purposes include the communication we have observed above. She will appeal to such examples as this:

\stage{Countess}{}{Where on earth am I going to find someone to invest in my eel farm?}

\stage{Count}{(pointing)}{There's a millionaire for you.}

\stage{Countess}{(incredulous)}{Henry? A millionaire? He hasn't got above nine hundred ninety-five thousand pounds.}

\stage{Count}{}{Oh, it's close enough.}

We are supposing that there is no millionaire in the room; strictly speaking, the count said something false with ``There's a millionaire''. Nonetheless, communication occurred because the term `millionaire' made the count's referential intention clear: he intended to refer to the person who was {\em almost} a millionaire. (The term is regularly used to refer to non-millionaires who have relatively great wealth.) The Ungerian is claiming that this is analogous to the case of the parent and child. Strictly speaking, what the parent said (``That's a chair'') was false, but it allowed for communication by making the parent's referential intention clear.

Is this a coherent objection? Without concerning ourselves too much with the nature of loose truth, I think it is fair to claim that, just as a (strict) truth has a `truthmaker', so a loose truth must have a `loose-truthmaker'. In the example above, the truthmaker for ``There's a millionaire'' would have been the fact that the count was referring to a millionaire. This fact did not obtain, so the statement is, strictly speaking, false. The loose-truthmaker is evidently the fact that the count is referring to someone who is {\em almost} a millionaire. (What counts as `almost' will no doubt vary between contexts, but in this context I am supposing it is true that Henry is almost a millionaire.)

The truthmaker for ``That's a chair'' must obviously be the fact that the parent is referring to a chair. According to Unger, there are no chairs, so nobody can refer to them. The parent's statement would therefore be, strictly speaking, false. Now what is the loose-truthmaker for the parent's use of ``That's a chair''? It cannot be the fact that there is {\em almost} a chair (a partially built chair?), at least not if that entails that there could ever be a chair. Unger maintains that the kind of object picked out by `chair' is ``never instanced''~(\citeyear[147]{unger1979}). Is there, perhaps, something closely resembling a chair in the room, and the parent is referring to {\em that} thing instead? This raises two objections of its own. First, what is there in the room that ``closely resembles'' a chair, other than the chair itself? Second, if we cannot ever have coherent thoughts about chairs (and therefore cannot know anything about chairs), how are we supposed to know what resembles a chair?

I do not think there are satisfactory answers to these questions. Moreover, I do not think Unger ever espoused a `loose-truth' nihilism, so we are not slighting him by moving on.

%\stage{Peter Unger}{}{I never said the room was {\em empty}. It isn't empty. But that doesn't mean that there is a {\em chair} in the room. There's something else entirely.}
%
%\noindent Now we may ask, ``What {\em is} in the room, if not a chair?'' Unger might walk over to the chair, saying ``There seems to be a concentration of solid matter in this vicinity which, when placed appropriately, may be sat on.'' And now we say, ``Oh, you mean a {\em chair!} That's what it's called, you see.''

\subsection{The moral}
There are limits to what one can resolutely deny the existence of. We can deny that certain things, like ghosts, exist {\em and} deny that people have beliefs in them. We can do this because in each situation where a person has a belief about what they take to be a ghost, we can show that their belief is really about nothing at all. If someone sees a reflection of the moon or experiences a hallucination, and so thinks she is seeing a ghost, we can say that she is afraid of nothing at all. Under no circumstances must we say that her belief is really about a ghost. Moreover, we can explain how people come to believe in ghosts---they read too many ghost stories, or believe the lies of others.

This is not something we can do with tables, chairs, and other ``ordinary things'', let alone people. For one, Unger has no explanation of how we come to form our beliefs in these things, if not by {\em seeing them}. Secondly, to deny that our thought and talk about such things are really about nothing at all is to deny that communication regularly occurs. Bizarrely, this is a consequence Unger appears willing to accept:
\begin{squote}
Now, it must of course be admitted that these arguments [for his strain of nihilism] undermine the possibility of any endeavor I should try to propose, or even the putative thought that I should propose anything, just as all of my putative essay is undermined. But even so, I shall (incoherently) propose that what we have now to do is invent new expressions which are not inconsistent ones, and by means of which we may, to some significant extent, think coherently about concrete reality~(\citeyear[544]{unger1980b}).
\end{squote}
If Unger seriously believes this, then he could not expect us even to understand his essay ({\em why would he write it?}). But I think it is safe to say that Unger does {\em not} actually believe that there are no people or ordinary things. In a book on ethics, Unger has unambiguously expressed his belief in people:

\begin{squote}
Each year millions of children die from easy to beat disease, from malnutrition, and from bad drinking water\,\ldots\,As UNICEF has made clear to millions of us well-off American adults at one time or another, with a packet of oral rehydration salts that costs 15 cents, a child can be saved from dying soon~(\citeyear[3]{unger1996}).
\end{squote}
There are only two possibilities: either Unger does believe that people (at least children and Americans) do exist, or he takes himself to be flat-out lying in the quoted passage.

As far as other ordinary things go, Unger claims to ``often now believe that there really are no tables or rocks, and never so firmly believe that there are such things as I once did''~(\citeyear[543]{unger1980b}). All I can say is that I don't believe him. (To show that he does believe in these things, we would need to spend some time with him, observing his behavior. We could invite him for a walk along a trail with lots of low-hanging branches, then warn him about them.)

\section{The Problem of the Many}
\label{many}
In section~\ref{unger} we looked at a version of metaphysical nihilism. Peter Unger attempted to deny that any of the `ordinary things' in the world (tables, chairs, apples, people, \&c.) actually exist. His motivation for this claim was drawn from an apparent paradox involving the terms for ordinary things. If we have a stone, then removing one atom of matter will not destroy the stone. Nor will removing another atom. But if we remove enough atoms, there will not be a stone. One solution to this puzzle is to deny that there ever is a stone. But this, we have seen, is not workable.

Another solution is to claim that there are {\em many} stones where we once thought there was only one. The motivation for this claim sometimes comes from what Unger calls ``the problem of the many''. There are a number of different formulations of this problem. Van Inwagen nicely summarizes one:
\begin{squote}
Assume I exist. Then certain simples compose me. Call them `M'. Now, a single simple is a negligible item indeed. Let $x$ be one of these negligible parts of me---one that is somewhere in my right arm, say. Now consider the simples that compose me {\em other than} $x$ (`M -- $x$'). Since $x$ is so very negligible, M -- $x$ {\em could} [my emphasis] compose a human being just as well as M could. We may say that M and M -- $x$ are ``equally well suited'' to compose human beings. And, of course, for {\em any} simple $y$, ``M -- $y$ will be as well suited to compose a human being as M are. Moreover, it would be surprising indeed if there were not a simple $z$ such that ``M + $z$'' were as well suited to compose a human being as M are. It would, in fact (if I may once more use this phrase), be intolerably arbitrary to say that M composed a human being although M -- $x$ {\em didn't} [my emphasis] and M -- $y$ {\em didn't} [my emphasis] and M + $z$ {\em didn't} [my emphasis]. Suppose, therefore, that M -- $x$ et al.\ {\em do} [my emphasis] compose human beings~(\citeyear[215]{inwagen1995}).
\end{squote}

I think this formulation is problematic. We are supposing that M does compose a human being. But it does not immediately follow from this that M -- $x$ also composes a human being. As I have pointed out with italics, there is a slide from the claim that M -- $x$ {\em could} compose a human being to the claim that M -- $x$ {\em does} compose a human being. As an analogy, take a house of blocks. Suppose that the blocks do compose the house. Is there also something composed by the blocks minus one? There {\em could} be; but intuitively, there would be only if that one block were removed. Then we would have a house with a missing roof. But we do not obviously have that second thing already, without having removed the block.

Van Inwagen understands there to be a number of additional premises required for this argument. He formulates them thus:
\begin{enumerate}
	\item In every situation of which we should ordinarily say that it contained just one man, there are many sets of simples whose members are as suitably arranged to compose men as any simples could be. \label{many1}
	\item The members of each of these sets compose something. \label{many2}
	\item Each of these ``somethings'' is a man, provided there are any men at all. \label{many3}
	\item If I exist, there is a man~(\citeyear[216]{inwagen1995}). \label{many4}
\end{enumerate}
I think only~\ref{many4} is uncontroversially true. There are many difficult questions raised by~\ref{many2}, and I'm not sure how to answer them. I'm quite sure, however, that the first and third premises are false. These two are closely related, so we'll examine them together.

\subsection{Problems with the first and third premises}
\label{many13p}
Unger begins his presentation of the problem of the many by directing our attention to a cloud; looking at the edge of the cloud, ``all that is there to be seen is a {\em gradual transition} from the more dense [concentration of water molecules] to the less so\,\ldots\,there is no natural break, or boundary, or stopping place, for any would-be cloud to have''~(\citeyear[415]{unger1980a}). Therefore any boundary we choose for the cloud will be somewhat arbitrary. A minutely larger or smaller boundary would be just as `suitable' for bounding a cloud; ``if our alleged typical item [the cloud] is indeed a typical cloud, then many of these candidates, millions at least, do not fail to be clouds altogether but are clouds of some sort''~(\citeyear[421]{unger1980a}). Unger draws the conclusion that either there are millions of such things there, or none. This argument can be generalized to chairs, stones, and people. Regarding chairs, at least, we have already shown that the conclusion that there are none is false; perhaps then, there are millions of chairs in our well-lit room. But there are two problems with this argument.

\paragraph{An objection regarding communication.}
First, the idea of there being millions of objects where we supposed there was only one poses a problem for referential communication. In fact, it is quite the same problem that Unger's nihilism faced above: how is referential communication possible under this hypothesis? When we supposed that there was nothing being referred to by `chair', we were at a loss to explain how people nonetheless managed to communicate using that term. Now we are supposing there to be millions of chairs, each eligible to be referred to by ``that chair''. What needs explaining now is how, if there are millions of chairs that might be referred to, how two speakers can be sure that they are talking about the {\em same} chair. When the child points and says ``What's that?'', the parent knows that the child is pointing at {\em a} chair, but how is she to determine {\em which} chair the child is referring to? The child did {\em not} say ``What are those?'' because she took herself to be referring to {\em one} chair, and expressed her referential intention accordingly.

Even if the parent and child got lucky and happened to think of the same chair, how would they know it? If there really were millions of chairs in the center of the room, it seems implausible to suggest that anyone could distinguish between each of them. There would be nothing the parent or child could do to make it clear to the other which of the millions of chairs they meant to refer to. Indeed, they could not make it clear to themselves. If I am holding a rock in my hand, ``there are millions of ``overlapping stones'' before me\,\ldots\,how am I to think of a single one of them, while not then equally thinking of so many others''~\citep[456]{unger1980a}? Unger's brand of universalism, like his nihilism, precludes successful communication, and so must be rejected.

\paragraph{An objection regarding boundaries.}
Second, it seems to be simply false that all boundaries drawn about a cloud are equally arbitrary. If a cloud is a concentration of water molecules in the air, then the boundary of the cloud is the edge of the concentration, beyond which the level of water in the air is normal. There are not millions of clouds with millions of different boundaries; any stipulated boundary that falls inside or outside the actual boundary is arbitrary because it does not track the concentration of water molecules. For simples to be `suitably arranged' so as to compose a cloud, their boundary must be the edge of the concentration of water molecules. I'm not sure, but it may be possible to extend this argument to cover most `ordinary things'---tables and chairs (and even people) have molecular boundaries.

Premise~\ref{many1} is therefore false. For clouds, at least, it is not true that there are ``many sets of simples whose members are as suitably arranged'' to compose them. These closely related sets do not have boundaries that track the concentration of water molecules in the air, so they are not suitably arranged to compose clouds. If the boundary of the cloud shrunk, then a smaller set of molecules {\em would} be suitably arranged to compose the cloud, and so it would. But just as the blocks minus one {\em would} compose a house (but don't), so this smaller set {\em would} compose a cloud (but doesn't). So premise~\ref{many3} seems false too; at least, if these sets compose anything, they compose something other than a cloud.

\section{Paraphrasing and composition}
In section~\ref{unger} we found that an attempt to deny the existence of ordinary things such as tables and chairs cannot succeed unless an explanation is given for our successful communication using the terms `table' and `chair'. Unger's nihilistic thesis failed to explain what we are really thinking and talking about when we take ourselves to be thinking and talking about chairs. We found no reason to suppose that we are not, in fact, thinking and talking about tables and chairs. If we are able to think and talk about tables and chairs, then it seems to follow that tables and chairs exist. (Remember that the fact that ghosts do not exist gives us reason to believe that we do not think and talk about ghosts.)

Peter van Inwagen has developed a more sophisticated thesis denying the existence of tables and chairs. Like Unger, he claims that (necessarily) there are simply no such things as tables and chairs in the world. But unlike Unger, he does not claim that when we take ourselves to be thinking and talking about such things, we are thinking and talking about nothing at all. Or at least ``when people say things in the ordinary business of life by uttering sentences that start `There are chairs\,\ldots\,' or `There are stars\,\ldots\,', they very often say things that are literally true''~\citep[102]{inwagen1995}. One would generally assume that if such statements are true, then it follows that chairs and stars exist. But van Inwagen denies that chairs and stars exist. How can he claim, then, that what was said was true? As we mentioned in section~\ref{paraphrase}, van Inwagen attempts to show that the statements in question can be {\em paraphrased}---they can be reformulated to show that they have no ``ontological commitments''. According to van Inwagen, one can assert that there is a chair without being committed to the existence of chairs.

Section~\ref{comp} will summarize the motivation for van Inwagen's denial. Section~\ref{pigletwise} will introduce and criticize van Inwagen's paraphrasing strategy.

\subsection{Composition}
\label{comp}
The Special Composition Question was given a precise formulation by van Inwagen, who finds that ``the metaphysically puzzling features of material objects are connected in deep and essential ways with metaphysically puzzling features of the constitution of material objects by their parts''~\citep[18]{inwagen1995}. A ready example is the Ship of Theseus: the planks and rigging and sails (and every part of the ship) are replaced as they individually wear out. These replacements happen each by themselves; it's not the case that the entire ship (or even a large section) is swapped out all at once. But eventually no part of the original ship remains. And yet we would commonly say that it is still the same ship. But why should we think that the present ship is identical with a past ship with which it shares no parts?

\subsection{The question}
\label{scq}
Answering the question ``why is this ship identical with that past ship?'' requires first figuring out why (and how) these planks and rigging and sails (et.\ al.) compose a ship in the first place. Van Inwagen asks ``in what circumstances do planks\footnote{For simplicity's sake, van Inwagen ignores the rigging and sails.} compose (add up to, form) something?''~(\citeyear[21]{inwagen1995}) For some $x$s, then, van Inwagen asks us to consider when
\begin{equation}
\exists y\ \text{the}\ x\text{s compose}\ y
\end{equation}
is true.%
\footnote{Van Inwagen explains in some detail how plural referring expressions (like ``the planks'') can be given a logical formalization (\citeyear[23--28]{inwagen1995}), but suffice to say they work just as one would expect.}%
%
\ Less formally, van Inwagen asks: ``suppose one had certain (nonoverlapping) objects, the $x$s, at one's disposal; what would one have to do---what {\em could} one do---to get the $x$s to compose something?''~(\citeyear[31]{inwagen1995})

({\em Composition} is a technical term for van Inwagen. He understands it thus: ``the $x$s compose $y$'' means that ``the $x$s are all parts of $y$ and no two of the $x$s overlap and every part of $y$ overlaps at least one of the $x$s\,\ldots\,a thing {\em overlaps} a thing---or: they overlap---if they have a common part''~(\citeyear[29]{inwagen1995}). For van Inwagen, everything is a part of itself; some $x$ is a {\em proper} part of some $y$ only if $x \neq y$.)

\subsection{The usual answers}
There are several prominent answers to the Special Composition Question, including the following:\footnote{These formulations are from~\citet{markosian1998a}.}
\begin{description}
	\item[Nihilism] Necessarily, for any $x$s, there is an object composed of the $x$s iff there is only one of the $x$s, i.e., the only objects that exist are simples~(\citeyear[219]{markosian1998a}).%
	%
	\footnote{\label{flip} Note that this may not be Unger's view. He denies that people, apples, cheese, tables, chairs, and other ``ordinary things'' are nonexistence but he does not, as far as I know, take a stand on whether anything at all exists. His view can be (flippantly) summarized thus: ``if we have a word for it, it doesn't exist.''}
	%\footnote{\label{gunk} Of course, it may be that the world is not fundamentally particulate, and is filled not with simples but with `gunk'; see \citet{schaffer2003}. Nihilism (and van Inwagen's second condition below) can be formulated to take this possibility into account: ``for any quantity of gunk, there is nothing composed of it.''}%
	%
	\item[Universalism] Necessarily, for any $x$s, there is an object composed of the $x$s iff no two of the $x$s overlap~(\citeyear[227]{markosian1998a}).
	\item[Van Inwagenism] Necessarily, for any $x$s, there is an object composed of the $x$s iff either (i) the activity of the $x$s constitutes a life or (ii) there is only one of the $x$s~(\citeyear[221]{markosian1998a}).
\end{description}

As we have seen, any version of nihilism that does explain our putative communication about tables and chairs is false.

Whether or not universalism is false is a more difficult question. As we saw in section~\ref{many13p}, any version of universalism that entails the existence of millions of chairs where we assumed there to be only one is false. This is because such a thesis would entail that we are unable to successfully communicate about ordinary things; we would not be able to know that were were talking about the {\em same thing} (not things) as our audience. It may be that a version of universalism without this consequence is true; but we will leave that possibility aside for now.

Van Inwagen examines and rejects a number of answers to the Special Composition Question that would entail the existence of tables and chairs. Some are too strong: `some $x$s compose a $y$ iff the $x$s are in contact' would entail that two people shaking hands will result in a new object coming into being. Others are too strong in some ways and too weak in others: `some $x$s compose a $y$ iff the $x$s are fastened together' would entail that two people being glued together would result in a new object; and it would deny that an object can be composed without fastening its parts together (such as when building a house of cards). The only answer van Inwagen finds consistent is what we have dubbed {\em van Inwagenism}, which entails that tables and chairs do not exist.

However, without a good explanation of what our beliefs about tables and chairs are really about, the fact that van Inwagenism entails the nonexistence of tables and chairs only shows that van Inwagenism is false. Happily, though, van Inwagen recognizes this and is prepared with a paraphrasing strategy aimed to show that the beliefs that we take to be about tables and chairs are really about something else.

\section{Van Inwagen's paraphrasing strategy}
\label{pigletwise}
Van Inwagen distances himself from the kind of resolute denial we saw Unger attempting in section~\ref{unger}. Unger maintained that terms like `chair' are incoherent; were this so, a statement involving the phrase ``There is a chair\,\ldots '' could surely not be true. Van Inwagen, on the other hand, admits that ``when people say things in the ordinary business of life by uttering sentences that start `There are chairs\,\ldots\,' or `There are stars\,\ldots\,', they very often say things that are literally true''~\cite[102]{inwagen1995}. One would generally assume that if what people say with ``There are chairs\,\ldots '' and the like are true, then chairs exist. But van Inwagen denies this entailment.

How can van Inwagen maintain this? He claims that one can also say, truly, ``There are simples arranged chairwise\,\ldots '' without committing oneself to the existence of chairs. He could, therefore, claim that when someone says ``There is a chair\,\ldots '' she {\em means} ``There are simples arranged chairwise.'' This is, of course, a bold hypothesis about the speech practices of ordinary speakers. Certainly very few speakers would, if asked, affirm that what they meant to say had anything to do with simples; they would say that when they said that there was a chair, they meant just that. Van Inwagen recognizes that this is not a viable position: ``The only thing I have to say about what the ordinary man really means by `There are two valuable chairs in the next room' is that he really means that there are two valuable chairs in the next room''~(\citeyear[106]{inwagen1995}).

One might then assume that van Inwagen is thinking in analogy with Russell. He could attempt to claim that, despite the surface appearance of language (``There is a chair\,\ldots ''), the underlying logical form does not make any mention of chairs (or tables); the offending concept is analyzed away, leaving ``There are simples arranged chairwise\,\ldots ''. Van Inwagen notes that his ``suggested technique of paraphrasing enables us to escape some of the more embarrassing consequences of this position. When someone says ``Some tables are heavier than some chairs,'' there is obviously something right about what he says. Our technique of paraphrasis enables us to capture what it is that is right about what he says''~(\citeyear[111]{inwagen1995}). However, on the very next page he admits that the ordinary language proposition and his paraphrased version are different propositions: ``When the ordinary man utters the sentence `Some chairs are heavier than some tables' (in an appropriate context, and so on and so on), he expresses a certain proposition, and one that is almost certainly true. But I do not claim that this proposition {\em is} the proposition that, for some $x$s, those $x$s are arranged chairwise and for some $y$s, those $y$s are arranged tablewise, and the $x$s are heavier than the $y$s''~(\citeyear[112]{inwagen1995}). So van Inwagen is not making an appeal to some notion of `logical form'. But then what is the purpose of the paraphrasing project?

Van Inwagen attempts to justify his method of paraphrasis by asserting the following parallels between the original and paraphrased propositions:
\begin{enumerate}[label=(\Alph*)]
	\item The paraphrase describes the same fact as the original. \label{para-a}
	\item The paraphrase, unlike the original, does not even appear to imply that there are any objects that occupy chair-receptacles. \label{para-b}
	\item The paraphrase is neutral with respect to competing metaphysical theories, {\em viz}. the ``received'' theory, that there are objects that occupy chair-receptacles, and the theory I have proposed, according to which there are no such objects. \label{para-c}
	\item The original, though it doubtless does not express the same proposition as the paraphrase, has the feature ascribed to the paraphrase in \ref{para-c}: It is neutral with respect to the question whether there are objects that fit exactly into chair-receptacles~(\citeyear[113]{inwagen1995}). \label{para-d}
\end{enumerate}
I am rather dubious as to the truth of \ref{para-a}, but I am quite sure that \ref{para-d} is false, and van Inwagen's thesis appears to depend on it. He admits in~\ref{para-b} that the original sentence (e.g., ``There are chairs\,\ldots '') {\em implies} that there are chairs, but claims in~\ref{para-d} that it does not {\em entail} this. But why wouldn't it?

\subsection{Propositions and ontological commitment}
Let us review the situation so as to appreciate the mess van Inwagen has gotten himself into. First, he agrees that when someone says thinks like ``There is a chair\,\ldots '' they mean just that. Second, he admits that his `paraphrases' of such propositions are not so faithful to the original that they can be called the same proposition; the original and the paraphrase are two different propositions. Third, he claims nonetheless that {\em neither} the original nor the paraphrase entail the existence of chairs.

This may strike one as obviously untrue. How can he claim that when someone says ``There is a chair\,\ldots '' and means just that, that the proposition they express does not entail the existence of chairs? To defend his claim, van Inwagen appeals to his `Copernican analogy':
\begin{squote}
I accept the Copernican Hypothesis. One day you hear me say, ``It was cooler in the garden after the sun had moved behind the elms.'' You say, ``You see, you can't consistently maintain your Copernicanism outside the astronomer's study. You say that the sun moved behind the elms; yet, according to your official theory, the sun does not move.'' I reply that the proposition I expressed by saying ``It was cooler in the garden after the sun had moved behind the elms'' is consistent with the Copernican Hypothesis~(\citeyear[101]{inwagen1995}).
\end{squote}
That is, van Inwagen claims that the proposition he expressed with ``It was cooler in the garden after the sun had moved behind the elms'' does not entail that the sun actually moved. And he argues that this is analogous to our talk of chairs: most propositions expressed with ``There is a chair\,\ldots '' do not entail that chairs actually exist.

First, does the proposition van Inwagen expresses with ``The sun moved behind the elms'' entail that the sun moved? I am inclined to say that it does. If I were to say simply ``The sun moved'' (meaning just that), I think I would have committed myself to the movement of the sun. Why should we think that the addition of ``behind the elms'' defeats this entailment? Without some explanation of what the difference is, I see no reason to think that saying ``The sun moved behind the elms'' (and meaning it) does not entail the movement of the sun. But van Inwagen may be forced to say here that neither proposition entails that the sun moved. For he certainly won't allow that either entails that the sun {\em exists.}

There is an analogy here, though perhaps not the one van Inwagen had in mind. He claims that a proposition expressed by ``There are two very valuable chairs in the next room'' does not necessarily entail the existence of chairs. If this proposition does not entail that chairs exist, then what about ``There are two valuable chairs left in the world'' or ``There are at least two chairs in the world'' or ``There are at least two chairs'' or simply ``There are chairs''? Van Inwagen appears committed to the claim that the proposition I would express with ``There are chairs'' does not entail that there are chairs.

Why on earth should this be? Does not the proposition expressed by my saying ``There are simples arranged chairwise\,\ldots '' entail the existence of simples? If van Inwagen says that there are simples arranged chairwise, and means just that, then it would appear to follow that there are simples. Van Inwagen's argument relies rather heavily on the assumption that simples exist.%
%
\footnote{Ted Sider takes him to task for this assumption~(\citeyear{sider1993}), claiming that the possibility of `gunk'---the possibility that the matter of the world is not fundamentally particulate but infinitely divisible---falsifies van Inwagen's thesis. I think it may be possible for van Inwagen to adapt to a gunky world (he might be able to claim that nothing exists but organisms, who are composed of other organisms and/or gunk), but I think van Inwagen's thesis is false either way.}
%
\ But if ``There are chairs'' does not entail that there are chairs and if ``The sun moved behind the trees'' entails neither that the sun moved nor that the sun exists, then how can van Inwagen maintain that ``There are simples arranged chairwise'' entails that there are simples, or that they are arranged chairwise? He has given us no reason to believe one and not the other.

\subsection{Loose truth, again}
When criticizing Peter Unger's nihilism, we imagined a defense of his thesis based on the notion of `loose truth'. Our Unger partisan claimed that while such claims as ``There is a chair\,\ldots '' are invariably false (because incoherent), they may be loosely true. Unfortunately for Unger, his defender was unable to give a loose-truthmaker for these supposed loose truths.

Might van Inwagen also appeal to loose truths? He admits it as a last-ditch possibility:
\begin{squote}
I can say this [that ``There are chairs\,\ldots '' can be true yet not entail that there are chairs] because I accept certain theses in the philosophy of language. I can say this because I accept certain theses in the philosophy of language. Some people, I suppose, would reject these theses. These people would say that when I said\,\ldots\,`The sun moved behind the elms,' I said something false\,\ldots\,If someone maintains that `The sun moved behind the elms' expresses a falsehood, he must still have some way to distinguish between this sentence and those sentences (like `The sun exploded' and `The sun turned green') that the vulgar would regard as the sentences that expressed falsehoods about the sun\,\ldots\,[if I took this line,] I should not be willing to say that people who uttered things like `There are two valuable chairs in the next room' very often said what was true. I should be willing to say only that they very often say what might be treated as a truth for all practical purposes~(\citeyear[102--103]{inwagen1995}).
\end{squote}
We can distinguish the given propositions based on the fact that only ``The sun moved behind the elms'' has a loose-truthmaker:
\begin{squote}
Owing to a change in the relative positions and orientations of the earth and the sun, it came to pass that a straight line drawn between the sun and this point (which is on the surface of the earth) would have passed through the elms~\citep[112--113]{inwagen1995}.
\end{squote}

But now what is the loose-truthmaker for ``There are chairs''? Presumably van Inwagen will say that it is the fact that there are simples arranged chairwise. Therefore the beliefs that we take to be about tables and chairs are, if they are about anything at all, about the arrangements of simples. For this to be a legitimate move, however, van Inwagen needs to give a non-circular definition of `chairwise'. The loose-truthmaker for the sun's movement contained no appeal to movement; the movement in question was defined in other terms. Likewise, the loose-truthmaker for the chair's existence must not make a covert appeal to chairs.

\subsection{Chair(wise)}
We are demanding that van Inwagen give us definitions of `chairwise' and `chair' that are not definitions in terms of each other. Van Inwagen claims to be able to do this; responding to criticism by Jay Rosenberg, he says that ``it is easy to see how to define `chairwise' in terms of `chair' without supposing that there are any chairs. Let a ``chair'' be defined as an object that has the properties $C_{1}, C_{2},\,\dots\,C_{n}$''~(\citeyear[719]{inwagen1993b}). In order for the definitions to be non-circular, these properties must not include things like ``is a chair'', ``is shaped chairwise'', etc.

I am dubious that van Inwagen can provide us with necessary and sufficient property-list definitions of chair and chairwise that are not circular. And I am not alone in my skepticism:
\begin{squote}
When one says chair, one thinks vaguely of an average chair. But collect individual instances, think of arm-chairs and reading chairs, and dining-room chairs and kitchen chairs, chairs that pass into benches, chairs that cross the boundary and become settees, dentists' chairs, thrones, opera stalls, seats of all sorts, those miraculous fungoid growths that cumber the floor of the Arts and Crafts Exhibition, and you will perceive what a lax bundle in fact is this simple straightforward term. In co-operation with an intelligent joiner I would undertake to defeat any definition of chair or chairishness that you gave me.% Chairs just as much as individual organisms, just as much as mineral and rock specimens, are unique things---if you know them well enough you will find an individual difference even in a set of machine-made chairs---and it is only because we do not possess minds of unlimited capacity, because our brain has only a limited number of pigeon-holes for our correspondences with an unlimited universe of objective uniques, that we have to delude ourselves into the belief that there is a chairishness in this species common to and distinctive of all chairs
~\citep[384--385]{wells1904}.
\end{squote}

Without a proposal as to how we can capture all these types of chairs under an exclusive definition (leaving none out and bringing nothing else in), I suggest that `chair' has no necessary and sufficient property-list definition; it is a `family resemblance'-type concept. We can say well enough whether a given object is a chair or not, but this is not because it has all and only those properties that belong to chairs. [It is because\,\ldots ?]

\section{Now what?}
We have come to the following conclusions:
\begin{enumerate}
	\item Ordinary things (tables, chairs, people, \&c.) exist, and we communicate intelligibly about them. Therefore, most versions of nihilism are false.
	\item Because we communicate intelligibly about ordinary things, we reject versions of universalism that entail a profusion of tables where we assume a single table to be.
	\item Van Inwagen's paraphrasing strategy could not coherently show that a literally true proposition expressed by ``There are chairs\,\ldots '' does not entail that there are chairs. He cannot appeal to loose truth until he has given non-circular definitions of `chair' and `chairwise'.
	\item No such definition seems to be forthcoming.
\end{enumerate}

Although the theories presented by Unger and van Inwagen are unsatisfactory, the problems they are meant to solve do require some answer. Despite the fact that we communicate perfectly well with terms like `chair' and `person', it is difficult to spell out exactly what it is that we use these terms to refer to. A chair may lose some of its matter (we might sand it down) or gain more matter (we might paint it), yet we generally assume that the same chair endures the change. We shed cells constantly, but it would be overhasty to conclude that we do not persist through time, because we are not continuously constituted out the very same matter. When we refer to these chairs and people, then, it seems that the things we are referring to are not simply the particles that make them up. But then what are they?

\ifstandalone
\end{spacing}
\bibliography{everything}
\bibliographystyle{ChicagoReedweb}
\fi
\end{document}
	
\chapter{Brute Facts}
\label{brute}
\chapterpig{Brute Facts}
\documentclass[11pt]{standalone} \newif\ifstandlone \standalonetrue
\usepackage{standalone}
\usepackage[left=1.75in, right=1.75in, top=1.25in, bottom=1.25in]{geometry}
\geometry{letterpaper}
\usepackage{graphicx}
%\usepackage{tipa}
%\usepackage{exaccent}
%\usepackage{txfonts}
%\usepackage{pxfonts}
\usepackage{enumitem}
%\usepackage{amssymb}
\usepackage{amsmath}
\usepackage{epstopdf}
\usepackage{setspace}
\usepackage{natbib}
\setcitestyle{aysep={}}
\usepackage{hyperref}
		
\synctex=1

\DeclareSymbolFont{symbolsC}{U}{txsyc}{m}{n}
\DeclareMathSymbol{\strictif}{\mathrel}{symbolsC}{74}
\DeclareMathSymbol{\boxright}{\mathrel}{symbolsC}{128}

\newenvironment{squote}{%
	\begin{quote}\begin{singlespace}%
	}{%
	\end{singlespace}\end{quote}}

\newcommand{\stager}[4]%
{%
	\begin{spacing}{1}%
	\vspace{0pt}
		\begin{description}[style=nextline, noitemsep,
                    parsep=0pt, topsep=0pt, leftmargin=15mm,
                    itemindent=-10mm, font=\mdseries]
			\item[\textsc{#1} \emph{#2}] #3
			\item[]%
			\begin{flushright}#4\end{flushright}
		\end{description}%
	\end{spacing}%
}

\newcommand{\stage}[3]%
{%
	\begin{spacing}{1}%
	\vspace{0pt}
		\begin{description}[style=nextline, parsep=0pt,
                    leftmargin=15mm, itemindent=-10mm, font=\mdseries]
			\item[\textsc{#1} \emph{#2}] #3
		\end{description}%
	\end{spacing}%
}

\newenvironment{inq}{\vspace{0pt}%
	\begin{list}{}%
	{\setlength\labelwidth{0pt}%
	\setlength\leftmargin{2.5\oddsidemargin}%
	\setlength\rightmargin{\leftmargin}}
	\begin{spacing}{1}
	\item[]%
	}{
	\end{spacing}
	\end{list}
	\vspace{10pt}
	%\noindent%
	}

\title{Brute facts and arbitrary objects}
\author{Alex Dunn}
\begin{document}
\ifstandalone
\maketitle
\begin{spacing}{1.5}
\fi

\section{Unger's nihilism}
\label{unger}
Just as resolutely as we would deny the existence of ghosts, so Peter
Unger has denied the existence of all `ordinary things'---such things
as ``tables and chairs and spears\,\ldots\,swizzle sticks and
sousaphones\,\ldots\,stones and rocks and twigs, and also tumbleweeds
and fingernails''~(\citeyear[117]{unger1979}).  He does not consider
them merely `subjective' as opposed to objectively so---like van
Inwagen, he claims that they simply do not exist.  He comes to this
conclusion from a different direction, however.  As we will see, van
Inwagen's denial of the existence of `ordinary things' is a
consequence of his theory of composition (under what conditions some
things compose another thing).  Unger, on the other hand, claims that
terms for ordinary things, like `chair', are {\em incoherent}.  Unger
claims that incoherent terms cannot apply to anything in the world;
therefore he concludes that there are no chairs (or any other ordinary
thing).

Unger has two largely independent motivations for his claim that terms
like `chair' are incoherent.  One is the sorites paradox, and the
other is the `problem of the many'.

\subsection{Sorites paradoxes}
A typical instance of the sorites paradox begins by having us imagine
some ordinary object; let us use a heap of sand.  Now suppose we
remove a single grain of sand.  If we were inclined to believe that
the initial quantity of sand did in fact constitute a heap, then after
the removal of a single grain, we should presumably still have a heap
(albeit a slightly smaller one).  It seems very implausible to think
that one grain of sand more or less could {\em ever} make a difference
as to whether something is or is not a heap.

But having conceded (a) that there is a heap and (b) that the removal
of a single grain cannot make the difference as to whether a quantity
of sand is a heap, we have unwittingly put our foot in it.  For if the
removal of a single grain {\em never} transforms a heap into a
non-heap, then by repeatedly removing one grain after another, we will
eventually find ourselves with a heap that consists of no sand at
all.  But it seems absurd to suppose that there could be a heap of
sand that is composed of no sand---indeed, of nothing whatsover.

This is the sorites paradox.  While a heap is a useful example,
because it is so ill-defined, similar problems appear to afflict all
ordinary things.  Unger illustrates the difficulty for stones:

\begin{squote}
Consider a stone, consisting of a certain finite number of atoms.  If
we or some physical process should remove one atom, without
replacement, then there is left that number minus one, presumably
constituting a stone still\,\ldots\,after another atom is removed,
there is that original number minus two; so far, so good.  But after
that certain number has been removed, in similar stepwise fashion,
there are no atoms at all in the situation, while we must still be
supposing that there is a stone there.  But as we have already agreed,
if there is a stone present, then there must be some atoms\,\ldots\,I
suggest that any adequate response to this contradiction must
include\,\ldots\,the denial of the existence of even a single
stone.~\citep[121--122]{unger1979}
\end{squote}
Unger understands this dilemma to apply across the board, and
correspondingly argues that we should deny the existence of even a
single ordinary thing.

\subsection{The problem of the many}
The `problem of the many', as Unger terms this second difficulty for
ordinary things, follows a similar line of reasoning.  If we consider
an ordinary thing---take a cloud, for instance---it is presumably
composed of molecules.  There is probably then a set of molecules, the
members of which compose the cloud.  Call that set $S$.  Now consider
$S_1$.  This is a set of molecules that includes all of the members of
$S$ as well as one additional molecule.  Do the members of $S_1$
compose a cloud?  Surely they are just as well suited to do so.  Now
consider $S_2$\,\ldots

Because these numerous 'candidates' are equally (or nearly equally)
well suited to be clouds, we seem forced to conclude that there are
either many clouds where we supposed there to be one, or rather no
clouds at all:

\begin{squote}
No matter where we start, the complex first chosen has nothing
objectively in its favor to make it a better candidate for cloudhood
than so many of its overlappers are.  Putting the matter somewhat
personally, each one's claim to be a cloud is just as good, no better
and no worse, than each of the many others.  And, by all odds, each
complex has \emph{at least} as good a claim as any still further real
entity in the situation.  So, either \emph{all} of \emph{them} make it
or else \emph{nothing} does; in this real situation, either there are
many clouds or else there really are no clouds at all
\citep[415--??]{unger1980a}.
\end{squote}

The problem of the many can also arise by considering the {\em
  boundary} of a given cloud.  It is natural to suppose that a cloud
has a determinate boundary.  But if we look at the edge of the cloud,
where we suppose the boundary to be, ``we may find, side by side, or
themselves overlapping, a great many potential boundaries for
clouds\,\ldots\,if our alleged typical item {[}the cloud{]} is indeed
a typical cloud, then many of these candidates, millions at least, do
not fail to be clouds altogether but are clouds of some
sort''~\citep[420--421]{unger1980a}.

The pattern of argumentation is the same for both approaches.  For a
certain cloud, a given set of members or a given boundary is supposed,
and it is argued that a set or boundary that differs minimally from
the original must also compose our bound a cloud.  The new set or
boundary does not appear to differ from the original in any relevant
way; there seems no principled way to deny that if the first set's
members compose a cloud, the second set's members do too.  And since
there are a great deal of very similar sets and boundaries, we find
ourselves with a plurality of clouds.

And of course, Unger does not rest content with applying the problem
of the many to clouds.  All ordinary objects get the same treatment;
he concludes that either there are a great many of them, or there are
none at all.  He claims, predictably, that the latter disjunct is
preferable.

\section{So what's the problem?}
We find ourselves wanting to hold three theses, which appear mutually
inconsistent:

\begin{enumerate}
  \item There is at least one chair (stone, cloud).
  \item If a chair (stone, cloud) exists, it must be composed of
    matter.
  \item If a chair (stone, cloud), exists, the removal of a single
    molecule (or otherwise insignificant quantity of matter) from it
    cannot destroy it or cause it to cease to exist.

We seem to be clearly caught in a paradox; the only question is where
we have gone wrong.

But have we, in fact, gone wrong?  Peter Unger thinks that we are
right on target:

\begin{squote}
While Eubulides' contribution has often been labeled `the sorites
paradox', there is nothing here which is a paradox in any
philosophically important sense\,\ldots\,Accepting our negative
conclusions here does not mean important logical trouble for us; we
only think we have troubles while we refuse to admit their validity
(\citeyear[145]{unger1979}).
\end{squote}

Our situation is only paradoxical, says Unger, while we unreflectingly
cling to the first thesis.  If, however, we come to see that there are
no chairs (stones, clouds), then we happily escape paradox: if there
are no chairs (stones, clouds) to begin with, we do not have to worry
about what the addition or removal of small amounts of matter would do
to them; nor do we need concern ourselves with what they would be made
of.

But thing are not quite so simple.  First, adding to the
implausibility of Unger's view, he must deny that our use of ordinary
terms like `chair' (`stone', `cloud') follow any sort of pattern or
display any competence at all.  Second, even if we managed to swallow
that consequence, Unger has no explanation as to why we believe that
there are chairs (stones, clouds).

\subsection{Competence and correctness}
Setting aside whether or not expressions of propositions like ``that's
a chair'' are ever \emph{true}, it seems right to say that there are at
least correct and incorrect uses of the terms.  For a word like
`chair' (`stone', `cloud') we generally do not say that a child has
learned how to use it until she is capable of using it in a certain
way.  We admit that she understands what `chair' (`stone', `cloud')
means or what a chair (stone, cloud) is when she displays a certain
competence with the term.  If instead of using `chair' to refer to
chairs she used it to refer to dogs or people, we would say that she
is confused and attempt to correct her use.

But Unger maintains that this is all an illusion, and that there is no
such thing as the correct or incorrect use of a term like `chair'
(`stone', `cloud'):

\begin{squote}
Concerning words and kinds, now, we might say this.  First, we might
say that it is in connection with \emph{semantics} that our reasonings have
what are their most obvious implications and, second, that their most
obvious semantic implications concern certain \emph{sortal nouns}, namely,
those which purport to denote ordinary things.  Thus, it appears quite
obvious to us now that there will be no application to things for such
nouns as `stone' and `rock', `twig' and `log', `planet' and `sun',
`mountain' and `lake', `sweater' and `cardigan', `telescope' and
`microscope', and so on, and so forth.  Simple positive sentences
containing these terms will never, given their current meanings,
express anything true, correct, accurate, etc., or even anything which
is anywhere close to being any of those things
(\citeyear[148]{unger1979}).
\end{squote}

This seems simply bizarre.  On what grounds, then, do parents correct
their children with respect to their use of ordinary terms?  Are they
compelled by some irrational force to consider certain utterances
correct and others incorrect?  One may question whether or not we use
ordinary term entirely consistently, but it seems simply false that,
necessarily, we {\em never} use (or have used, or will use) ordinary
terms in correct, as opposed to incorrect, ways:

\begin{squote}
It is\,\ldots\,unclear how far our use of e.g. the vocabulary of
colours \emph{is} consistent.  The descriptions given of awkward cases
may vary from occasion to occasion.  Besides that, the notion of using
a predicate consistently would appear to require some objective
criteria for variation in relevant respects among items to be
described in terms of it; but what is distinctive about observational
predicates is exactly the lack of such criteria.  So it would be
unwise to lean too heavily, as though it were a matter of hard fact,
upon the consistency of our employment of colour predicates.  What,
however, may be depended upon is that our use of these predicates is
largely \emph{successful}; the expectations which we form on the basis
of others' ascriptions of colour are not usually disappointed.
Agreement is generally possible about how colours are to be described;
and this, of course, is equivalent to saying that others \emph{seem}
to use colour predicates in a largely consistent way
\citep[361]{wright1975}.
\end{squote}

None of this {\em proves} that Unger is wrong, of course.  But it is
worth remembering how really implausible his view is.  Moreover, in
the next section we will see that certain questions about our beliefs
arise as a result of his denial of ordinary things.  Unfortunately for
Unger, he has no means to answer these questions, which remain a
(probably unsuperable) barrier to the acceptability of his thesis.

\section{Beliefs in things}
Having denied that ordinary things exist, Unger must either explain
how our beliefs about stones should be properly understood (van
Inwagen, as we shall see has an interesting paraphrasing strategy) or
he must deny that we really {\em do} have any beliefs about stones.
Unger opts for the latter and claims that, like the person who thought
she had a belief about a ghost, we are wrong to think that we have
any coherent beliefs about stones or any other ordinary things.  Unger
says that, like `ghost', our ``terms for ordinary things are
incoherent [and] cannot apply to anything
real''~\citep[147]{unger1979}.  A consequence of this is that our
language and thought concerning all such things is directed toward
{\em nothing at all}: ``it may well be that I have never {\em thought
  of} any stones at all, or tables, or even human hands.  If that is
so, then it would seem that {\em a fortiori} I do not {\em know}
anything {\em about these entities}, however commonly I might
otherwise suppose''~(\citeyear[458]{unger1980a}).

This all seems very strange.  Concerning ghosts, ``it is difficult
even to find a fully coherent belief that might be exposed as false;
we discover, at best, obscurity or perhaps confusion\,\ldots\,do we
really understand what sort of thing a ghost is supposed to
be''~\citep[76]{stroud2000a}?  If someone tries to tell me about the
ghost that visited him the previous night, it does not seem unjust to
say that he doesn't really know what he is talking about.  But can
this line be extended to some of the most common objects of
experience?

When we denied the existence of ghosts, we denied also others' beliefs
in them.  We did not, however, deny that people have beliefs which
they take to be about ghosts.  But we were able to show that these
beliefs were not {\em about} ghosts; in most cases they were about
nothing at all.  Likewise, Unger cannot deny that we have beliefs that
we take to be about tables, chairs, and all the other things that he
denies exist.  If our beliefs about tables and chairs are really
beliefs about nothing at all, then what causes us to form these
beliefs?  Why do we believe in ordinary things to begin with?

\subsection{Causes of belief}
\label{unger-cause}
People who believe in ghosts probably do so because they have
unreflectively embraced the superstitions of their culture.  They may
initially come to believe that ghosts exist on the testimony of other
people---older siblings, perhaps---or by reading too many ghost
stories.  Much as Catherine in Jane Austen's {\em Northanger Abbey}
jumps to the most macabre conclusions as a result of having absorbed
too many gothic novels, so might a gullible reader of ghost stories go
on to interpret such innocent phenomena as reflections of the moon as
ghostly assailants.  Those of us who have not taken our cues from
fiction would be more likely to recognize such phenomena as tricks of
the light.  Even if we were to see something that was definitely {\em
  not} a trick of the light, we would sooner attribute it to an
hallucination than countenance the possibility of ghosts.  Suppose
{\em you} saw what you took to be a ghost in an empty, well-lit room.
Most of us would still, even if presented with such a vision, {\em
  refuse to believe in ghosts}.  This is because we know that the
probability of there being such spirits is far less than the
probability of us experiencing cracks in our sanity.  Undermining my
belief that ghosts don't exist would require a great deal---for
example, my friend and I both apparently seeing the {\em same} ghost
at the same time, and knowing that we were each experiencing the same
vision.  (Even then, we would want further confirmed sightings to
convince us that we weren't, in fact, crazy.)

If this is an accurate characterization of our beliefs concerning
ghosts, it is a very different characterization than one we might give
of how we learn about and come to believe in chairs.  Chairs are not
something that children learn about from stories.  A child probably
learns what a chair is as an answer to the question, ``What is {\em
  that?}\,''  

Let us suppose that the child is pointing at a chair in the center of
a well-lit room containing no other furniture.  The chair is clearly
visible.  If someone were to believe they were pointing at a ghost in
a similarly well-lit situation, we could safely assume that they would
be experiencing a hallucination.  But clearly the child is not
hallucinating.  There is {\em something} (or some things) in the
center of the room; what Unger wants to deny is that there is a {\em
  chair} there.

Unger would admit, I think, that there is a quantity of matter,
arranged in a certain way, in the center of the room.  (Unger
presumably also denies the existence of rooms, so this would have to
be expressed differently, but never mind.)  He sees it, just as well
as we do; it's not as though Unger can't see straight.  All he's
saying is that what he is looking at is not a chair.

But if all that is in the center of the room is a mass of matter, {\em
  why do we believe that there is a chair there?}  To say that there
is a chair in the center of the room would, according to Unger, be
neither true, nor accurate, nor correct, nor ``anything which is
anywhere close to being any of those things'' \citep[148]{unger1979}.
So where on earth do we get the idea that there is a chair there?

\hline

\subsection{Loose truth}
\label{loose-u}
One sympathetic to Unger's thesis might admit that they are
communicating about something, but deny that the subject of their
communication is a chair.  This philosopher would take refuge in the
notion of `loose truth'.  She would maintain that it is strictly false
that there is a chair in the room, but that it is loosely true; it is
close enough to the truth for practical purposes.  These practical
purposes include the communication we have observed above.  She will
appeal to such examples as this:

\stage{Countess}{}{Where on earth am I going to find someone to invest
  in my eel farm?}

\stage{Count}{(pointing)}{There's a millionaire for you.}

\stage{Countess}{(incredulous)}{Henry? A millionaire? He hasn't got
  above nine hundred ninety-five thousand pounds.}

\stage{Count}{}{Oh? Well, it's close enough.}

We are supposing that there is no millionaire in the room; strictly
speaking, the count said something false with ``There's a
millionaire''.  Nonetheless, communication occurred because the term
`millionaire' made the count's referential intention clear: he
intended to refer to a person who was {\em almost} a millionaire.
(The term is regularly used to refer to non-millionaires who have
relatively great wealth.)  The Ungerian is claiming that this is
analogous to the case of the parent and child.  Strictly speaking,
what the parent said (``That's a chair'') was false, but it allowed
for communication by making the parent's referential intention clear.

Is this a coherent objection?  Without concerning ourselves too much
with the nature of loose truth, I think it is fair to claim that, just
as a (strict) truth has a `truthmaker', so a loose truth must have a
`loose-truthmaker'.  In the example above, the truthmaker for
``There's a millionaire'' would have been the fact that the count was
referring to a millionaire.  This fact did not obtain, so the
statement is, strictly speaking, false.  The loose-truthmaker is
evidently the fact that the count is referring to someone who is {\em
  almost} a millionaire.  (What counts as `almost' will no doubt vary
between contexts, but in this context I am supposing it is true that
Henry is almost a millionaire.)

The truthmaker for ``That's a chair'' must obviously be the fact that
the parent is referring to a chair.  According to Unger, there are no
chairs, so nobody can refer to them.  The parent's statement would
therefore be, strictly speaking, false.  Now what is the
loose-truthmaker for the parent's use of ``That's a chair''?  It cannot
be the fact that there is {\em almost} a chair (a partially built
chair?), at least not if that entails that there could ever be a
chair.  Unger maintains that the kind of object picked out by `chair'
is ``never instanced''~(\citeyear[147]{unger1979}).  Is there,
perhaps, something closely resembling a chair in the room, and the
parent is referring to {\em that} thing instead?  This raises two
objections of its own.  First, what is there in the room that
``closely resembles'' a chair, other than the chair itself?  Second, if
we cannot ever have coherent thoughts about chairs (and therefore
cannot know anything about chairs), how are we supposed to know what
resembles a chair?

I do not think there are satisfactory answers to these questions.
Moreover, I do not think Unger ever espoused a `loose-truth' nihilism,
so we are not slighting him by moving on.

\subsection{The moral}
\label{moral}
There are limits to what one can resolutely deny the existence of.  We
can deny that certain things, like ghosts, exist {\em and} deny that
people have beliefs in them.  We can do this because in each situation
where a person has a belief about what they take to be a ghost, we can
show that their belief is really about nothing at all.  If someone
sees a reflection of the moon or experiences a hallucination, and so
thinks she is seeing a ghost, we can say that she is afraid of nothing
at all.  Under no circumstances must we say that her belief is really
about a ghost.  Moreover, we can explain how people come to believe in
ghosts---they read too many ghost stories, or believe the lies of
others.

This is not something we can do with tables, chairs, and other
``ordinary things'', let alone people.  For one, Unger has no
explanation of how we come to form our beliefs in these things, if not
by {\em seeing them}.  Secondly, to deny that our thought and talk
about such things are really about nothing at all is to deny that
communication regularly occurs.  Bizarrely, this is a consequence
Unger appears willing to accept:
\begin{squote}
Now, it must of course be admitted that these arguments [for his
  strain of nihilism] undermine the possibility of any endeavor I
should try to propose, or even the putative thought that I should
propose anything, just as all of my putative essay is undermined.  But
even so, I shall (incoherently) propose that what we have now to do is
invent new expressions which are not inconsistent ones, and by means
of which we may, to some significant extent, think coherently about
concrete reality~(\citeyear[544]{unger1980b}).
\end{squote}
If Unger seriously believes this, then he could not expect us even to
understand his essay ({\em why would he write it?}).  But I think it
is safe to say that Unger does {\em not} actually believe that there
are no people or ordinary things.  In a book on ethics, Unger has
unambiguously expressed his belief in people:

\begin{squote}
Each year millions of children die from easy to beat disease, from
malnutrition, and from bad drinking water\,\ldots\,As UNICEF has made
clear to millions of us well-off American adults at one time or
another, with a packet of oral rehydration salts that costs 15 cents,
a child can be saved from dying soon~(\citeyear[3]{unger1996}).
\end{squote}
There are only two possibilities: either Unger does believe that
people (at least children and Americans) do exist, or he takes himself
to be flat-out lying in the quoted passage.

As far as other ordinary things go, Unger claims to ``often now
believe that there really are no tables or rocks, and never so firmly
believe that there are such things as I once
did''~(\citeyear[543]{unger1980b}).  All I can say is that I don't
believe him.  (To show that he does believe in these things, we would
need to spend some time with him, observing his behavior.  We could
invite him for a walk along a trail with lots of low-hanging branches,
then warn him about them.)

\section{The problem of the many}
\label{many}
In section~\ref{unger} we looked at a version of metaphysical
nihilism.  Peter Unger attempted to deny that any of the `ordinary
things' in the world (tables, chairs, apples, people, \&c.) actually
exist.  His motivation for this claim was drawn from an apparent
paradox involving the terms for ordinary things.  If we have a stone,
then removing one atom of matter will not destroy the stone.  Nor will
removing another atom.  But if we remove enough atoms, there will not
be a stone.  One solution to this puzzle is to deny that there ever is
a stone.  But this, we have seen, is not workable.

Another solution is to claim that there are {\em many} stones where we
once thought there was only one.  The motivation for this claim
sometimes comes from what Unger calls ``the problem of the many''.
There are a number of different formulations of this problem.  Van
Inwagen nicely summarizes one:
\begin{squote}
Assume I exist.  Then certain simples compose me.  Call them `M'.
Now, a single simple is a negligible item indeed.  Let $x$ be one of
these negligible parts of me---one that is somewhere in my right arm,
say.  Now consider the simples that compose me {\em other than} $x$
(`M -- $x$').  Since $x$ is so very negligible, M -- $x$ {\em could}
[my emphasis] compose a human being just as well as M could.  We may
say that M and M -- $x$ are ``equally well suited'' to compose human
beings.  And, of course, for {\em any} simple $y$, ``M -- $y$ will be
as well suited to compose a human being as M are.  Moreover, it would
be surprising indeed if there were not a simple $z$ such that ``M +
$z$'' were as well suited to compose a human being as M are.  It
would, in fact (if I may once more use this phrase), be intolerably
arbitrary to say that M composed a human being although M -- $x$ {\em
  didn't} [my emphasis] and M -- $y$ {\em didn't} [my emphasis] and M
+ $z$ {\em didn't} [my emphasis].  Suppose, therefore, that M -- $x$
et al.\ {\em do} [my emphasis] compose human
beings~(\citeyear[215]{inwagen1995}).
\end{squote}

I think this formulation is problematic.  We are supposing that M does
compose a human being.  But it does not immediately follow from this
that M -- $x$ also composes a human being.  As I have pointed out with
italics, there is a slide from the claim that M -- $x$ {\em could}
compose a human being to the claim that M -- $x$ {\em does} compose a
human being.  As an analogy, take a house of blocks.  Suppose that the
blocks do compose the house.  Is there also something composed by the
blocks minus one? There {\em could} be; but intuitively, there would
be only if that one block were removed.  Then we would have a house
with a missing roof.  But we do not obviously have that second thing
already, without having removed the block.

Van Inwagen understands there to be a number of additional premises
required for this argument.  He formulates them thus:
\begin{enumerate}
	\item In every situation of which we should ordinarily say
          that it contained just one man, there are many sets of
          simples whose members are as suitably arranged to compose
          men as any simples could be.  \label{many1}
	\item The members of each of these sets compose
          something.  \label{many2}
	\item Each of these ``somethings'' is a man, provided there
          are any men at all.  \label{many3}
	\item If I exist, there is a
          man~(\citeyear[216]{inwagen1995}).  \label{many4}
\end{enumerate}
I think only~\ref{many4} is uncontroversially true.  There are many
difficult questions raised by~\ref{many2}, and I'm not sure how to
answer them.  I'm quite sure, however, that the first and third
premises are false.  These two are closely related, so we'll examine
them together.

\subsection{Problems with the first and third premises}
\label{many13p}
Unger begins his presentation of the problem of the many by directing
our attention to a cloud; looking at the edge of the cloud, ``all that
is there to be seen is a {\em gradual transition} from the more dense
[concentration of water molecules] to the less so\,\ldots\,there is no
natural break, or boundary, or stopping place, for any would-be cloud
to have''~(\citeyear[415]{unger1980a}).  Therefore any boundary we
choose for the cloud will be somewhat arbitrary.  A minutely larger or
smaller boundary would be just as `suitable' for bounding a cloud;
``if our alleged typical item [the cloud] is indeed a typical cloud,
then many of these candidates, millions at least, do not fail to be
clouds altogether but are clouds of some
sort''~(\citeyear[421]{unger1980a}).  Unger draws the conclusion that
either there are millions of such things there, or none.  This
argument can be generalized to chairs, stones, and people.  Regarding
chairs, at least, we have already shown that the conclusion that there
are none is false; perhaps then, there are millions of chairs in our
well-lit room.  But there are two problems with this argument.

\paragraph{An objection regarding communication.}
First, the idea of there being millions of objects where we supposed
there was only one poses a problem for referential communication.  In
fact, it is quite the same problem that Unger's nihilism faced above:
how is referential communication possible under this hypothesis? When
we supposed that there was nothing being referred to by `chair', we
were at a loss to explain how people nonetheless managed to
communicate using that term.  Now we are supposing there to be
millions of chairs, each eligible to be referred to by ``that chair''.
What needs explaining now is how, if there are millions of chairs that
might be referred to, how two speakers can be sure that they are
talking about the {\em same} chair.  When the child points and says
``What's that?'', the parent knows that the child is pointing at {\em
  a} chair, but how is she to determine {\em which} chair the child is
referring to? The child did {\em not} say ``What are those?'' because
she took herself to be referring to {\em one} chair, and expressed her
referential intention accordingly.

Even if the parent and child got lucky and happened to think of the
same chair, how would they know it? If there really were millions of
chairs in the center of the room, it seems implausible to suggest that
anyone could distinguish between each of them.  There would be nothing
the parent or child could do to make it clear to the other which of
the millions of chairs they meant to refer to.  Indeed, they could not
make it clear to themselves.  If I am holding a rock in my hand,
``there are millions of ``overlapping stones'' before me\,\ldots\,how
am I to think of a single one of them, while not then equally thinking
of so many others''~\citep[456]{unger1980a}?  Unger's brand of
universalism, like his nihilism, precludes successful communication,
and so must be rejected.

\paragraph{An objection regarding boundaries.}
Second, it seems to be simply false that all boundaries drawn about a
cloud are equally arbitrary.  If a cloud is a concentration of water
molecules in the air, then the boundary of the cloud is the edge of
the concentration, beyond which the level of water in the air is
normal.  There are not millions of clouds with millions of different
boundaries; any stipulated boundary that falls inside or outside the
actual boundary is arbitrary because it does not track the
concentration of water molecules.  For simples to be `suitably
arranged' so as to compose a cloud, their boundary must be the edge of
the concentration of water molecules.  I'm not sure, but it may be
possible to extend this argument to cover most `ordinary
things'---tables and chairs (and even people) have molecular
boundaries.

Premise~\ref{many1} is therefore false.  For clouds, at least, it is
not true that there are ``many sets of simples whose members are as
suitably arranged'' to compose them.  These closely related sets do
not have boundaries that track the concentration of water molecules in
the air, so they are not suitably arranged to compose clouds.  If the
boundary of the cloud shrunk, then a smaller set of molecules {\em
  would} be suitably arranged to compose the cloud, and so it would.
But just as the blocks minus one {\em would} compose a house (but
don't), so this smaller set {\em would} compose a cloud (but doesn't).
So premise~\ref{many3} seems false too; at least, if these sets
compose anything, they compose something other than a cloud.

\ifstandalone
\bibliography{everything}
\bibliographystyle{ChicagoReedweb}
\end{spacing}
\fi
\end{document}


\chapter*{Conclusion}
         \addcontentsline{toc}{chapter}{Conclusion}
	\chaptermark{Conclusion}
	\markboth{Conclusion}{Conclusion}
	\setcounter{chapter}{4}
	\setcounter{section}{0}
	
Here's a conclusion, demonstrating the use of all that manual incrementing and table of contents adding that has to happen if you use the starred form of the chapter command. The deal is, the chapter command in \LaTeX\ does a lot of things: it increments the chapter counter, it resets the section counter to zero, it puts the name of the chapter into the table of contents and the running headers, and probably some other stuff. 

So, if you remove all that stuff because you don't like it to say ``Chapter 4: Conclusion'', then you have to manually add all the things \LaTeX\ would normally do for you. Maybe someday we'll write a new chapter macro that doesn't add ``Chapter X'' to the beginning of every chapter title.

\section{More info}
And here's some other random info: the first paragraph after a chapter title or section head \emph{shouldn't be} indented, because indents are to tell the reader that you're starting a new paragraph. Since that's obvious after a chapter or section title, proper typesetting doesn't add an indent there. 


%If you feel it necessary to include an appendix, it goes here.
    \appendix
      \chapter{The First Appendix}
      \chapter{The Second Appendix, for Fun}


%This is where endnotes are supposed to go, if you have them.

  \backmatter % backmatter makes the index and bibliography appear properly in the t.o.c...

% Make my bibliography be called "Bibliography" and not "References" (or "Works Cited" or...):
% \renewcommand{\bibname}{Works Cited}
    \bibliographystyle{ChicagoReedweb} % there are a variety of styles available; 
% replace ``plainnat'' with the style of choice. You can refer to files in the bsts or APA 
% subfolder, e.g. 
% \bibliographystyle{APA/apa-good}  % or
% \bibliographystyle{bsts/mla-good} 

% if you're using bibtex, the next line forces every entry in the bibtex file to be included
% in your bibliography, regardless of whether or not you've cited it in the thesis.
    %\nocite{*}
    \bibliography{everything}

%\end{spacing}
% Finally, an index would go here... but it is also optional.
\end{document}
