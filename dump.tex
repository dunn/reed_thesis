\documentclass[11pt]{article}
\usepackage[left=1.75in, right=1.75in, top=1.25in, bottom=1.25in]{geometry}
\geometry{letterpaper}
\usepackage{graphicx}
\usepackage{enumitem}
\usepackage{amssymb}
\usepackage{amsmath}
\usepackage{epstopdf}
\usepackage{setspace}
\usepackage{natbib}
\setcitestyle{aysep={}}
\usepackage{hyperref}
\usepackage{url}
\synctex=1

\DeclareSymbolFont{symbolsC}{U}{txsyc}{m}{n}
\DeclareMathSymbol{\strictif}{\mathrel}{symbolsC}{74}
\DeclareMathSymbol{\boxright}{\mathrel}{symbolsC}{128}		

\newenvironment{squote}{%
       \begin{spacing}{1}
       \begin{list}{}{%
       \setlength{\labelwidth}{0pt}%
       \rightmargin\leftmargin%
       }
       \item\relax
       }{%
       \end{list}%
       \end{spacing}
}

\title{The dump}
\author{Alexander A. Dunn}
\begin{document}
\maketitle
\begin{spacing}{1}


\section{Other answers that are false}
This section will mostly involve me registering agreement with
Markosian on the insufficiency of other proposed answers to the
SCQ. For example:
\begin{description}
	\item[Fastenation] Necessarily, for any $x$s, there is an
          object composed of the $x$s iff the $x$s are fastened
          together.~\citep[223]{markosian1998a}
\end{description}
No general notion of `fasten' will make this plausible. ``Suppose that
van Inwagen and I shake hands, and suppose that just as we do so, our
hands becomes paralyzed [or super-glued together], so that we cannot
pull them apart. Then, according to Fastenation, there is a new
composite object in the
world''~\citep[224]{markosian1998a}.\footnote{This is based on an
  example of van Inwagen's~(\citeyear[57--58]{inwagen1995}.} But we do
not recognize van Inwagen and Markosian as composing an object; `van
Mark', thankfully, does not emerge. Fastenation is not then
sufficient. Nor is it necessary; I can create a house of cards without
fastening any of them together.

Another approach is this:
\begin{description}
	\item[The Serial Response] The correct answer to the SCQ is an
          instance of this schema: There is an object composed of the
          $x$s iff {\em either} the $x$s are F1s and related by R1,
          {\em or} the $x$s are F2s and are related by R2, {\em
            or}\,\ldots\,the $x$s are Fns and are related by
          Rn.~\citep[230]{markosian1998a}
\end{description}
But seeing as we have not {\em one} plausible relation between objects
that is necessary and sufficient for composition, it seems unlikely
that we can find a series of them, limited in scope though they may
be. (For what are the necessary and sufficient conditions for a deck
of cards composing a house of cards?)

\section{Arbitrariness}
Where does this leave us? [stuff about realism]

\hline

\section{Now what?}
We have come to the following conclusions:
\begin{enumerate}
	\item Ordinary things (tables, chairs, people, \&c.) exist,
          and we communicate intelligibly about them.  Therefore, most
          versions of nihilism are false.
	\item Because we communicate intelligibly about ordinary
          things, we reject versions of universalism that entail a
          profusion of tables where we assume a single table to be.
	\item Van Inwagen's paraphrasing strategy could not coherently
          show that a literally true proposition expressed by `There
          are chairs\,\ldots ' does not entail that there are chairs.
          He cannot appeal to loose truth until he has given
          non-circular definitions of `chair' and `chairwise'.
	\item No such definition seems to be forthcoming.
\end{enumerate}

Although the theories presented by Unger and van Inwagen are
unsatisfactory, the problems they are meant to solve do require some
answer.  Despite the fact that we communicate perfectly well with
terms like `chair' and `person', it is difficult to spell out exactly
what it is that we use these terms to refer to.  A chair may lose some
of its matter (we might sand it down) or gain more matter (we might
paint it), yet we generally assume that the same chair endures the
change.  We shed cells constantly, but it would be overhasty to
conclude that we do not persist through time, because we are not
continuously constituted out the very same matter.  When we refer to
these chairs and people, then, it seems that the things we are
referring to are not simply the particles that make them up.  But then
what are they?

\hline

\section{Unger's intolerance}
\noindent Peter Unger denies that chairs exist.  (It is possible that
he has since recanted.)  Unger claims that this is a consequence of
the incoherency of the term `chair'.  As he argues, `chair' is an
incoherent term; being incoherent, he says, it cannot have any
application to things in the world; therefore he concludes that there
are no chairs.

The strength of Unger's argument for the incoherency of `chair' rests
on two things: the sorites paradox and the `problem of the many'.

\subsection{Sorites paradoxes}
A typical case of a sorites paradox involves some object
(paradigmatically, a heap of sand) from which a minute quantity of
matter is removed.  If we are inclined to suppose that the initial
quantity of matter (in this case, sand) was really a heap, then the
removal of a single grain of sand should leave the heap intact.  At
least on an intuitive level, it seems false that the removal of a
single grain of sand could {\em ever} transform a heap to something
less than a heap.  The heap will of course be a slightly smaller heap,
but it seems that it must be a heap nonetheless.

Once we have conceded these two points, however, we have unwittingly
put our foot in it.  For if the removal of a grain of sand {\em never}
transforms a heap into a non-heap, then by repeatedly removing single
grains of sand, we will eventually find ourselves with a heap of sand
that consists of absolutely no sand at all!

Unger follows this line of reasoning with respect to not only chairs,
but stones as well:

\begin{squote}
Consider a stone, consisting of a certain finite number of atoms.  If
we or some physical process should remove one atom, without
replacement, then there is left that number minus one, presumably
constituting a stone still\,\ldots\,after another atom is removed,
there is that original number minus two; so far, so good.  But after
that certain number has been removed, in similar stepwise fashion,
there are no atoms at all in the situation, while we must still be
supposing that there is a stone there.  But as we have already agreed,
if there is a stone present, then there must be some atoms\,\ldots\,I
suggest that any adequate response to this contradiction must
include\,\ldots\,the denial of the existence of even a single
stone.~\citep[121--122]{unger1979}
\end{squote}

Unger generalizes this argument and denies the existence of all
``ordinary things''.  An object that is generally thought to endure
minuscule losses of matter is therefore banished.

\subsection{The problem of the many}
Here we find a rather similar method.  A cloud is, presumably,
composed of molecules.

\hline

First, however, something must be said as to why we need to `get out'
in the first place.  Why not conclude with Unger that there just
aren't any chairs?

\subsection{Unger's skeptical solution}
Unger's objective, in his series of papers on ordinary things, is to
show that there are no chairs, tables, etc.  He therefore claims that
the sorites paradox is not a paradox at all; it simply shows that
there are no chairs et.\ al.:



\subsection{Sources of belief}
(Integrate old material)

\begin{squote}
Concerning words and kinds, now, we might say this.  First, we might
say that it is in connection with \emph{semantics} that our reasonings have
what are their most obvious implications and, second, that their most
obvious semantic implications concern certain \emph{sortal nouns}, namely,
those which purport to denote ordinary things.  Thus, it appears quite
obvious to us now that there will be no application to things for such
nouns as `stone' and `rock', `twig' and `log', `planet' and `sun',
`mountain' and `lake', `sweater' and `cardigan', `telescope' and
`microscope', and so on, and so forth.  Simple positive sentences
containing these terms will never, given their current meanings,
express anything true, correct, accurate, etc., or even anything which
is anywhere close to being any of those things
(\citeyear[148]{unger1979}).
\end{squote}

This seems simply bizarre.  On what grounds, then, do parents correct
their children with respect to their use of ordinary terms?  Are they
compelled by some irrational force to consider certain utterances
correct and others incorrect?  Unger's view is extremely implausible,
and it becomes more so when we consider color words.  As mentioned
above, Crispin Wright uses `red' as a example of a vague term.  It is,
according to the governing view, semantically incoherent in just the
same way that ordinary terms like `chair' and `table' are.  Unger does
not address color words, but it seems that he would be compelled to
treat them as incoherent.  He would have to conclude that, as `red' is
incoherent, it has no application; he would have to say that there are
no red things, and that no expressions of propositions like ``that is
red'' could ever express anything correct or accurate, let alone true.
But what brings us to consider the problem of vagueness with regard to
terms like `red' is that they \emph{do} seem to be applied in correct
and incorrect ways.  It seemed correct to apply `red' to the leftmost
color patch and incorrect to apply it to the rightmost one.  The
difficulties that arise in the middle of the spectrum are only
troubling \emph{because} they conflict with the obvious correctness of
the application (or withholding) of `red' at the edge of the spectrum.
The sorites paradox might make us doubt whether our application of
certain terms is consistent, but it should not convince us that we do
not ever apply the terms \emph{correctly}:

\begin{squote}
It is, to begin with, unclear how far our use of e.g. the vocabulary
  of colours \emph{is} consistent.  The descriptions given of awkward cases
  may vary from occasion to occasion.  Besides that, the notion of
  using a predicate consistently would appear to require some
  objective criteria for variation in relevant respects among items to
  be described in terms of it; but what is distinctive about
  observational predicates is exactly the lack of such criteria.  So
  it would be unwise to lean too heavily, as though it were a matter
  of hard fact, upon the consistency of our employment of colour
  predicates.  What, however, may be depended upon is that our use of
  these predicates is largely \emph{successful}; the expectations which we
  form on the basis of others' ascriptions of colour are not usually
  disappointed.  Agreement is generally possible about how colours are
  to be described; and this, of course, is equivalent to saying that
  others \emph{seem} to use colour predicates in a largely consistent way
  \citep[361]{wright1975}.
\end{squote}

Just as Unger cannot deny that we believe there to be chairs, so he
cannot deny that there is a pattern of use for ordinary terms such
that some uses are correct and others incorrect.

\section{van Inwagen's presentation of the problem of the many}
\begin{squote}
Assume I exist.  Then certain simples compose me.  Call them `M'.
Now, a single simple is a negligible item indeed.  Let $x$ be one of
these negligible parts of me---one that is somewhere in my right arm,
say.  Now consider the simples that compose me {\em other than} $x$
(`M -- $x$').  Since $x$ is so very negligible, M -- $x$ {\em could}
[my emphasis] compose a human being just as well as M could.  We may
say that M and M -- $x$ are ``equally well suited'' to compose human
beings.  And, of course, for {\em any} simple $y$, ``M -- $y$ will be
as well suited to compose a human being as M are.  Moreover, it would
be surprising indeed if there were not a simple $z$ such that ``M +
$z$'' were as well suited to compose a human being as M are.  It
would, in fact (if I may once more use this phrase), be intolerably
arbitrary to say that M composed a human being although M -- $x$ {\em
  didn't} [my emphasis] and M -- $y$ {\em didn't} [my emphasis] and M
+ $z$ {\em didn't} [my emphasis].  Suppose, therefore, that M -- $x$
et al.\ {\em do} [my emphasis] compose human
beings~(\citeyear[215]{inwagen1995}).
\end{squote}

I think this formulation is problematic.  We are supposing that M does
compose a human being.  But it does not immediately follow from this
that M -- $x$ also composes a human being.  As I have pointed out with
italics, there is a slide from the claim that M -- $x$ {\em could}
compose a human being to the claim that M -- $x$ {\em does} compose a
human being.  As an analogy, take a house of blocks.  Suppose that the
blocks do compose the house.  Is there also something composed by the
blocks minus one? There {\em could} be; but intuitively, there would
be only if that one block were removed.  Then we would have a house
with a missing roof.  But we do not obviously have that second thing
already, without having removed the block.

Van Inwagen understands there to be a number of additional premises
required for this argument.  He formulates them thus:
\begin{enumerate}
	\item In every situation of which we should ordinarily say
          that it contained just one man, there are many sets of
          simples whose members are as suitably arranged to compose
          men as any simples could be.  \label{many1}
	\item The members of each of these sets compose
          something.  \label{many2}
	\item Each of these ``somethings'' is a man, provided there
          are any men at all.  \label{many3}
	\item If I exist, there is a
          man~(\citeyear[216]{inwagen1995}).  \label{many4}
\end{enumerate}
I think only~\ref{many4} is uncontroversially true.  There are many
difficult questions raised by~\ref{many2}, and I'm not sure how to
answer them.  I'm quite sure, however, that the first and third
premises are false.  These two are closely related, so we'll examine
them together.



\subsection{Chair(wise)}
We are demanding that van Inwagen give us definitions of `chairwise'
and `chair' that are not definitions in terms of each other.  Van
Inwagen claims to be able to do this; responding to criticism by Jay
Rosenberg, he says that ``it is easy to see how to define `chairwise'
in terms of `chair' without supposing that there are any chairs.  Let
a ``chair'' be defined as an object that has the properties $C_{1},
C_{2},\,\dots\,C_{n}$''~(\citeyear[719]{inwagen1993b}).  In order for
the definitions to be non-circular, these properties must not include
things like `is a chair', `is shaped chairwise', etc.

I am dubious that van Inwagen can provide us with necessary and
sufficient property-list definitions of chair and chairwise that are
not circular.  And I am not alone in my skepticism:


Without a proposal as to how we can capture all these types of chairs
under an exclusive definition (leaving none out and bringing nothing
else in), I suggest that `chair' has no necessary and sufficient
property-list definition; it is a `family resemblance'-type concept.
We can say well enough whether a given object is a chair or not, but
this is not because it has all and only those properties that belong
to chairs.  [It is because\,\ldots ?]

\subsection{Van Inwagen's notion of parthood}
\label{van-part}
Van Inwagen defines his technical notion of composition (see section
\ref{scq}) in terms of a largely intuitive notion of parthood.  Van
Inwagen's interest, however, is restricted to `material' objects
(objects made exclusively of quarks and protons, or whatever the basic
atoms of the physical world turn out to be).  While he goes on to use
`part' only in reference to material objects, he recognizes that the
term has much wider application:

\begin{squote}
Parthood will occupy a central place in the present study of material
objects.  It is therefore worth noting that the word `part' is applied
to many things besides material objects.  We have already noted that
submicroscopic objects like quarks and protons are at least not clear
cases of material objects; nevertheless, every material object would
seem pretty clearly to have quarks and protons as \emph{parts}, and,
it would seem, in exactly the same sense of \emph{part} as that in
which a paradigmatic material object might have another paradigmatic
material object as a part.  A ``part,'' therefore, need not be a thing
that is clearly a material object.  Moreover, the word `part' is
applied to things that are clearly \emph{not} material objects---or at
least it is on the assumption that these things really exist and that
apparent reference to them is not a mere manner of speaking.  A stanza
is a part of a poem; Botvinnik was in trouble for part of the game;
the part of the curve that lies below the x-axis contains two minima;
parts of his story are hard to believe\,\ldots\,such examples can be
multiplied indefinitely.  Does this word `part' mean the same thing
when we speak of parts of cats, parts of poems, parts of games, parts
of curves, and parts of stories \citeyearpar[18--19]{inwagen1995}?
\end{squote} 

Van Inwagen suggests that `part' does have a number of different
meanings.  Later he says that ``there is one relation called
`parthood' whose field comprises material objects\,\ldots\,another
relation called `parthood' defined on events, another still defined on
stories, yet another defined on curves, and so on''
\citeyearpar[19]{inwagen1995}.

This may very well be, but what is the similarity between these
relations?  The parthood relation in classical mereology is
well-defined, as is the membership relation in set theory.  But there
is no equally well-defined relation for event parthood, or poem parthood.

However, Kit Fine has proposed a theory of parthood that takes
seriously the possibility that there are a plurality of different
parthood relations, and has the resources to define numerous parthood
relations.  The parthood relation for poems may be given as rigorous a
treatment as the parthood relation for sets.
\end{spacing}
\end{document}

%%%%%



%%%%%%%%%


%%%%
Suppose now that $b$ is the diachronic fusion of the condition of
being Tibbles.  In other words, it is always the case that $b$
synchronically fuses the condition of being Tibbles.  At all times
during which something has the condition of being Tibbles, $b$ is that
thing.  Likewise, suppose that $c$ is the diachronic fusion of the
condition of being all of Tibbles except for his whiskers.  It is
always the case that $c$ synchronically fuses the condition of being
all of Tibbles except for his whiskers.  At all times during which
something has the condition of being all of Tibbles except for his
whiskers, $c$ is that thing.  When Tibbles loses his whiskers at
$t_2$, $b$ and $c$ will share all their parts, and so by strong
supplementation they will be parts of each other, but it is not {\em
  always} the case that they are parts of each other.  By restricting
anti-symmetry to things that are {\em always} parts of each other, we
avoid the conclusion that $b = c$.

%% But there is one further wrinkle.  Above I claimed that the synchronic
%% fusion of the condition of being Tibbles at $t_1$ (when he has his
%% whiskers) was not identical with the synchronic fusion of the
%% condition of being Tibbles at $t_2$.  That claim followed from our
%% assumption that if there exists a fusion of some things at one time
%% and if there exists a fusion of the same things at another time, then
%% the two fusions are really the same fusion.

%% But here we are claiming the opposite.  If $b$ is the diachronic
%% fusion of the condition of being Tibbles, then at $t_1$, $b$ is
%% composed of Tibbles (including his whiskers) and at $t_2$, $b$ is
%% composed of Tibbles (excluding his whiskers).  But if $c$ is the
%% diachronic fusion of the condition of being all of Tibbles except for
%% his whiskers, then at $t_1$, $c$ is composed of Tibbles (excluding his
%% whiskers) and at $t_2$, $c$ is still composed of Tibbles (excluding
%% his whiskers).

%% \subsection{Do synchronic fusions persist over time?  How?}
%% \label{synch-p}

By rejecting strict anti-symmetry in favor of the assumption that
things that {\em always} have the same parts are identical, it is
possible for diachronic fusions to change parts.  Any two diachronic
fusions that always have the same parts (are always both composed of
the same synchronic fusions) are really the same, but two diachronic
fusions that are only sometimes composed of the same synchronic
fusions are not the same.

Since a diachronic fusion is defined as something that {\em always}
fuses some particular condition, it is a straightforward question as
to whether two diachronic fusions are really the same.  But the
definition of a {\em synchronic fusion} does not contain any temporal
term like `always'.  Recall that a thing is a synchronic fusion of
some condition just in case ``both (1) every [$u$] that meets the
condition is part of [$v$]; and (2) every part of [$v$] overlaps
something that meets the condition'' \citeyearpar[??]{hovda2011}.
Above we supposed that $z$ and $w$ were synchronic fusions at $t_2$
that shared all the same parts (Tibbles without whiskers).  The
objects $z$ and $w$ currently share the same parts, and so are parts
of each other, but do $z$ and $w$ {\em always} share the same parts?

I think they must.  At $t_2$, they are identical.  This is because at
$t_2$ objects $b$ and $c$ are composed of the same parts; if they were
not, we would not have had to reject anti-symmetry.  Now if two things
are {\em ever} identical, they are {\em always} identical; there is no
such thing as temporary identity.  Therefore $z$ and $w$ are always
identical, and so therefore they always share the same parts.

I believe it follows that synchronic fusions like $z$ exist only
momentarily.  

%% A discouraging analogy might be drawn with classical mereology.  The
%% definitions of mereology that I quoted in section \ref{tech} used
%% plural quantifiers like ``the $x$s''.  But they are often formulated
%% using sets instead.  For example, instead of saying ``$y$ is composed
%% of the $x$s'', we can say ``$y$ fuses $S$'', where $S$ is a set whose
%% members are all and only the $x$s.  When classical mereology is given
%% this set-theoretic formulation, it is assumed that for every set,
%% there is one and only one thing that fuses that set.

\subsection{Why bother?}
A metaphysical thesis that involves denying the existence of ordinary
things like chairs entails that the simplest explanation of why we
believe that there are chairs is incorrect.  I believe that such a
thesis should therefore be supplemented with a new explanation.  This
new explanation would identify the reasons why we would believe that
there are chairs if there are in fact none.  But why should I demand
this of a metaphysical theory?  Is it a reasonable request?

As an analogy, consider color.  Most people believe that things are
colored.  A simple causal story about why people believe that things
are colored might go like this:  things are colored, and people see
that things are colored.  

But imagine a philosopher who holds some version of {\em physicalism}
and claims that the world as described by physics is all that there
is.  This view is often thought to have the consequence that things
aren't actually colored.  In the `vocabulary of physics', things might
be described in such a way that the things color gets somehow left
out.  We may be unable to determine from the `physical description'
what color the object is.  The colors of objects are not included in
this philosopher's description of the world.

If the philosopher admits that people believe that things are colored,
she cannot explain this using the same story that I used above.  I
said that people believe that things are colored and that they see
that things are colored.  But the physicalist maintains that things
are not colored.  {\em If} she admits that people believe that things
are colored, then she needs a different explanation as to why people
believe that things are colored.

She might, however, deny that people believe that things are colored.
(This would be a rather bold claim.)  She could say that the notion of
color is entirely illusory.  If we believe that we see colors, she may
tell us we are wrong.  When we think that something is colored, we are
mistaken.  If we think that an apple is red, we have a false belief.
She might claim that color does not pose a difficulty for her view,
because humans do not experience `color'.

This, as I said, is a rather bold claim.  It seems simply true that we
see colors and that the apple looks red.  If a philosopher were to
deny these things, I would have difficulty understanding what she
meant.  This is not to say she is {\em wrong}; I have no argument
proving that her thesis is false.  But the claim that humans do not
experience color seems bizarre and unmotivated.  Fortunately I do not
know of anyone who actually holds this view.

Our imagined philosopher might make a less bold claim.  She might
instead claim that color is one of those things that are `subjective'
rather than `objective' or `absolute' features of the world.  A
subjective feature of the world is a feature that is present only
because we (or some other being) exists to experience it:

\begin{squote}
Whatever is due only to us and to our own ways of responding to and
interacting with the world does not reflect or correspond to anything
present in the world as it is independently of us.  The aim of an
``absolute'' conception, then, is to form a description of the way the
world is, not just independently of its being believed to be that way,
but independently, too, of all the ways in which it happens to present
itself to us human beings from our particular standpoint within
it\,\ldots\,[So we] form some conception of that independent reality
and come to understand parts or aspects of our original conception of
the world as not representing it as it is.  If we see them as products
or reflections of something peculiar to human experience or to the
human perspective on the universe, we assign them a merely
``subjective'' or dependent status and eliminate them from our
conception of the world as it is independently of
us~\citep[30--31]{stroud2000a}.
\end{squote}

A philosopher who adheres to this distinction might claim that our
conception of the world as colored does not represent the world as it
is independently of us.  Colors, she would claim, are not objectively
real.  She allows, however, that they are subjectively real.  She
admits that people do see colors.  Because of our color vision, we
come to believe that the things we see are colored.  A philosopher who
denies the objective reality of color does not thereby ``deny that we
perceive many different colours or that we believe physical objects to
be coloured'' \citep[145]{stroud2000a}.  What this philosopher claims
is something to the effect that, while we see things {\em as} colored,
things are not {\em themselves} colored.  The red color of a tomato,
on this view, obtains only in our perception of the tomato; there is
nothing {\em in} the tomato that is the redness (other species may not
see the redness when they see the tomato).

The philosopher who is denying the objective reality of color does
``recognize the presence in the world of perceptions of and beliefs
about the colours of things'' \citep[199]{stroud2000a}.  The challenge
then is for her to explain why we do have these perceptions and
beliefs.  If she believes that only the world of physics is
objectively real, she must explain why we hold these beliefs, and she
must give this explanation in such a way that commits her only to the
existence of physical things.  If she claims that the world as
described by physics is the only world there is, then she must explain
why, in a world that contains only physical things, we come to believe
that there are colors and colored objects.

Again: if our physicalist philosopher admits that people believe that
they experience color, and admits that people believe that things are
colored, {\em then} she commits herself to explaining why we form
beliefs that are, according to her, false.  Here is the analogy with
metaphysicians like Trenton Merricks: {\em if} they admit that many of
us believe that there are chairs and other ordinary objects, then they
commit themselves to explaining why we form these false beliefs.  For
as we have seen, even false beliefs are generally held for a reason.
