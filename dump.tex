\documentclass[11pt]{article}
\usepackage[left=1.75in, right=1.75in, top=1.25in, bottom=1.25in]{geometry}
\geometry{letterpaper}
\usepackage{graphicx}
%\usepackage{xyling}
%\usepackage{tipa}
%\usepackage{exaccent}
%\usepackage{txfonts}
%\usepackage{pxfonts}
\usepackage{enumitem}
%\usepackage{amssymb}
\usepackage{amsmath}
\usepackage{epstopdf}
\usepackage{setspace}
\usepackage{natbib}
\setcitestyle{aysep={}}
\usepackage{hyperref}
\usepackage{url}
\synctex=1

\DeclareSymbolFont{symbolsC}{U}{txsyc}{m}{n}
\DeclareMathSymbol{\strictif}{\mathrel}{symbolsC}{74}
\DeclareMathSymbol{\boxright}{\mathrel}{symbolsC}{128}		

\newenvironment{squote}{%
       \begin{spacing}{1}
       \begin{list}{}{%
       \setlength{\labelwidth}{0pt}%
       \rightmargin\leftmargin%
       }
       \item\relax
       }{%
       \end{list}%
       \end{spacing}
}

\title{The dump}
\author{Alexander A. Dunn}
\begin{document}
\maketitle
\begin{spacing}{1}


\section{Other answers that are false}
This section will mostly involve me registering agreement with
Markosian on the insufficiency of other proposed answers to the
SCQ. For example:
\begin{description}
	\item[Fastenation] Necessarily, for any $x$s, there is an
          object composed of the $x$s iff the $x$s are fastened
          together.~\citep[223]{markosian1998a}
\end{description}
No general notion of `fasten' will make this plausible. ``Suppose that
van Inwagen and I shake hands, and suppose that just as we do so, our
hands becomes paralyzed [or super-glued together], so that we cannot
pull them apart. Then, according to Fastenation, there is a new
composite object in the
world''~\citep[224]{markosian1998a}.\footnote{This is based on an
  example of van Inwagen's~(\citeyear[57--58]{inwagen1995}.} But we do
not recognize van Inwagen and Markosian as composing an object; `van
Mark', thankfully, does not emerge. Fastenation is not then
sufficient. Nor is it necessary; I can create a house of cards without
fastening any of them together.

Another approach is this:
\begin{description}
	\item[The Serial Response] The correct answer to the SCQ is an
          instance of this schema: There is an object composed of the
          $x$s iff {\em either} the $x$s are F1s and related by R1,
          {\em or} the $x$s are F2s and are related by R2, {\em
            or}\,\ldots\,the $x$s are Fns and are related by
          Rn.~\citep[230]{markosian1998a}
\end{description}
But seeing as we have not {\em one} plausible relation between objects
that is necessary and sufficient for composition, it seems unlikely
that we can find a series of them, limited in scope though they may
be. (For what are the necessary and sufficient conditions for a deck
of cards composing a house of cards?)

\section{Arbitrariness}
Where does this leave us? [stuff about realism]

\hline

\section{Now what?}
We have come to the following conclusions:
\begin{enumerate}
	\item Ordinary things (tables, chairs, people, \&c.) exist,
          and we communicate intelligibly about them.  Therefore, most
          versions of nihilism are false.
	\item Because we communicate intelligibly about ordinary
          things, we reject versions of universalism that entail a
          profusion of tables where we assume a single table to be.
	\item Van Inwagen's paraphrasing strategy could not coherently
          show that a literally true proposition expressed by `There
          are chairs\,\ldots ' does not entail that there are chairs.
          He cannot appeal to loose truth until he has given
          non-circular definitions of `chair' and `chairwise'.
	\item No such definition seems to be forthcoming.
\end{enumerate}

Although the theories presented by Unger and van Inwagen are
unsatisfactory, the problems they are meant to solve do require some
answer.  Despite the fact that we communicate perfectly well with
terms like `chair' and `person', it is difficult to spell out exactly
what it is that we use these terms to refer to.  A chair may lose some
of its matter (we might sand it down) or gain more matter (we might
paint it), yet we generally assume that the same chair endures the
change.  We shed cells constantly, but it would be overhasty to
conclude that we do not persist through time, because we are not
continuously constituted out the very same matter.  When we refer to
these chairs and people, then, it seems that the things we are
referring to are not simply the particles that make them up.  But then
what are they?

\hline

\section{Unger's intolerance}
\noindent Peter Unger denies that chairs exist.  (It is possible that
he has since recanted.)  Unger claims that this is a consequence of
the incoherency of the term `chair'.  As he argues, `chair' is an
incoherent term; being incoherent, he says, it cannot have any
application to things in the world; therefore he concludes that there
are no chairs.

The strength of Unger's argument for the incoherency of `chair' rests
on two things: the sorites paradox and the `problem of the many'.

\subsection{Sorites paradoxes}
A typical case of a sorites paradox involves some object
(paradigmatically, a heap of sand) from which a minute quantity of
matter is removed.  If we are inclined to suppose that the initial
quantity of matter (in this case, sand) was really a heap, then the
removal of a single grain of sand should leave the heap intact.  At
least on an intuitive level, it seems false that the removal of a
single grain of sand could {\em ever} transform a heap to something
less than a heap.  The heap will of course be a slightly smaller heap,
but it seems that it must be a heap nonetheless.

Once we have conceded these two points, however, we have unwittingly
put our foot in it.  For if the removal of a grain of sand {\em never}
transforms a heap into a non-heap, then by repeatedly removing single
grains of sand, we will eventually find ourselves with a heap of sand
that consists of absolutely no sand at all!

Unger follows this line of reasoning with respect to not only chairs,
but stones as well:

\begin{squote}
Consider a stone, consisting of a certain finite number of atoms.  If
we or some physical process should remove one atom, without
replacement, then there is left that number minus one, presumably
constituting a stone still\,\ldots\,after another atom is removed,
there is that original number minus two; so far, so good.  But after
that certain number has been removed, in similar stepwise fashion,
there are no atoms at all in the situation, while we must still be
supposing that there is a stone there.  But as we have already agreed,
if there is a stone present, then there must be some atoms\,\ldots\,I
suggest that any adequate response to this contradiction must
include\,\ldots\,the denial of the existence of even a single
stone.~\citep[121--122]{unger1979}
\end{squote}

Unger generalizes this argument and denies the existence of all
``ordinary things''.  An object that is generally thought to endure
minuscule losses of matter is therefore banished.

\subsection{The problem of the many}
Here we find a rather similar method.  A cloud is, presumably,
composed of molecules.

\hline

First, however, something must be said as to why we need to `get out'
in the first place.  Why not conclude with Unger that there just
aren't any chairs?

\subsection{Unger's skeptical solution}
Unger's objective, in his series of papers on ordinary things, is to
show that there are no chairs, tables, etc.  He therefore claims that
the sorites paradox is not a paradox at all; it simply shows that
there are no chairs et.\ al.:



\subsection{Sources of belief}
(Integrate old material)

\begin{squote}
Concerning words and kinds, now, we might say this.  First, we might
say that it is in connection with \emph{semantics} that our reasonings have
what are their most obvious implications and, second, that their most
obvious semantic implications concern certain \emph{sortal nouns}, namely,
those which purport to denote ordinary things.  Thus, it appears quite
obvious to us now that there will be no application to things for such
nouns as `stone' and `rock', `twig' and `log', `planet' and `sun',
`mountain' and `lake', `sweater' and `cardigan', `telescope' and
`microscope', and so on, and so forth.  Simple positive sentences
containing these terms will never, given their current meanings,
express anything true, correct, accurate, etc., or even anything which
is anywhere close to being any of those things
(\citeyear[148]{unger1979}).
\end{squote}

This seems simply bizarre.  On what grounds, then, do parents correct
their children with respect to their use of ordinary terms?  Are they
compelled by some irrational force to consider certain utterances
correct and others incorrect?  Unger's view is extremely implausible,
and it becomes more so when we consider color words.  As mentioned
above, Crispin Wright uses `red' as a example of a vague term.  It is,
according to the governing view, semantically incoherent in just the
same way that ordinary terms like `chair' and `table' are.  Unger does
not address color words, but it seems that he would be compelled to
treat them as incoherent.  He would have to conclude that, as `red' is
incoherent, it has no application; he would have to say that there are
no red things, and that no expressions of propositions like ``that is
red'' could ever express anything correct or accurate, let alone true.
But what brings us to consider the problem of vagueness with regard to
terms like `red' is that they \emph{do} seem to be applied in correct
and incorrect ways.  It seemed correct to apply `red' to the leftmost
color patch and incorrect to apply it to the rightmost one.  The
difficulties that arise in the middle of the spectrum are only
troubling \emph{because} they conflict with the obvious correctness of
the application (or withholding) of `red' at the edge of the spectrum.
The sorites paradox might make us doubt whether our application of
certain terms is consistent, but it should not convince us that we do
not ever apply the terms \emph{correctly}:

\begin{squote}
It is, to begin with, unclear how far our use of e.g. the vocabulary
  of colours \emph{is} consistent.  The descriptions given of awkward cases
  may vary from occasion to occasion.  Besides that, the notion of
  using a predicate consistently would appear to require some
  objective criteria for variation in relevant respects among items to
  be described in terms of it; but what is distinctive about
  observational predicates is exactly the lack of such criteria.  So
  it would be unwise to lean too heavily, as though it were a matter
  of hard fact, upon the consistency of our employment of colour
  predicates.  What, however, may be depended upon is that our use of
  these predicates is largely \emph{successful}; the expectations which we
  form on the basis of others' ascriptions of colour are not usually
  disappointed.  Agreement is generally possible about how colours are
  to be described; and this, of course, is equivalent to saying that
  others \emph{seem} to use colour predicates in a largely consistent way
  \citep[361]{wright1975}.
\end{squote}

Just as Unger cannot deny that we believe there to be chairs, so he
cannot deny that there is a pattern of use for ordinary terms such
that some uses are correct and others incorrect.
\end{spacing}
\end{document}
