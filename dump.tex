\documentclass[11pt]{article}
\usepackage[left=1.75in, right=1.75in, top=1.25in, bottom=1.25in]{geometry}
\geometry{letterpaper}
\usepackage{graphicx}
\usepackage{enumitem}
\usepackage{amssymb}
\usepackage{amsmath}
\usepackage{epstopdf}
\usepackage{setspace}
\usepackage{natbib}
\setcitestyle{aysep={}}
\usepackage{hyperref}
\usepackage{url}
\synctex=1

\DeclareSymbolFont{symbolsC}{U}{txsyc}{m}{n}
\DeclareMathSymbol{\strictif}{\mathrel}{symbolsC}{74}
\DeclareMathSymbol{\boxright}{\mathrel}{symbolsC}{128}		

\newenvironment{squote}{%
       \begin{spacing}{1}
       \begin{list}{}{%
       \setlength{\labelwidth}{0pt}%
       \rightmargin\leftmargin%
       }
       \item\relax
       }{%
       \end{list}%
       \end{spacing}
}

\title{The dump}
\author{Alexander A. Dunn}
\begin{document}
\maketitle
\begin{spacing}{1}


\section{Other answers that are false}
This section will mostly involve me registering agreement with
Markosian on the insufficiency of other proposed answers to the
SCQ. For example:
\begin{description}
	\item[Fastenation] Necessarily, for any $x$s, there is an
          object composed of the $x$s iff the $x$s are fastened
          together.~\citep[223]{markosian1998a}
\end{description}
No general notion of `fasten' will make this plausible. ``Suppose that
van Inwagen and I shake hands, and suppose that just as we do so, our
hands becomes paralyzed [or super-glued together], so that we cannot
pull them apart. Then, according to Fastenation, there is a new
composite object in the
world''~\citep[224]{markosian1998a}.\footnote{This is based on an
  example of van Inwagen's~(\citeyear[57--58]{inwagen1995}.} But we do
not recognize van Inwagen and Markosian as composing an object; `van
Mark', thankfully, does not emerge. Fastenation is not then
sufficient. Nor is it necessary; I can create a house of cards without
fastening any of them together.

Another approach is this:
\begin{description}
	\item[The Serial Response] The correct answer to the SCQ is an
          instance of this schema: There is an object composed of the
          $x$s iff {\em either} the $x$s are F1s and related by R1,
          {\em or} the $x$s are F2s and are related by R2, {\em
            or}\,\ldots\,the $x$s are Fns and are related by
          Rn.~\citep[230]{markosian1998a}
\end{description}
But seeing as we have not {\em one} plausible relation between objects
that is necessary and sufficient for composition, it seems unlikely
that we can find a series of them, limited in scope though they may
be. (For what are the necessary and sufficient conditions for a deck
of cards composing a house of cards?)

%%%%%%

\subsection{Van Inwagen's notion of parthood}
\label{van-part}
Van Inwagen defines his technical notion of composition (see section
\ref{scq}) in terms of a largely intuitive notion of parthood.  Van
Inwagen's interest, however, is restricted to `material' objects
(objects made exclusively of quarks and protons, or whatever the basic
atoms of the physical world turn out to be).  While he goes on to use
`part' only in reference to material objects, he recognizes that the
term has much wider application:

\begin{squote}
Parthood will occupy a central place in the present study of material
objects.  It is therefore worth noting that the word `part' is applied
to many things besides material objects.  We have already noted that
submicroscopic objects like quarks and protons are at least not clear
cases of material objects; nevertheless, every material object would
seem pretty clearly to have quarks and protons as \emph{parts}, and,
it would seem, in exactly the same sense of \emph{part} as that in
which a paradigmatic material object might have another paradigmatic
material object as a part.  A ``part,'' therefore, need not be a thing
that is clearly a material object.  Moreover, the word `part' is
applied to things that are clearly \emph{not} material objects---or at
least it is on the assumption that these things really exist and that
apparent reference to them is not a mere manner of speaking.  A stanza
is a part of a poem; Botvinnik was in trouble for part of the game;
the part of the curve that lies below the x-axis contains two minima;
parts of his story are hard to believe\,\ldots\,such examples can be
multiplied indefinitely.  Does this word `part' mean the same thing
when we speak of parts of cats, parts of poems, parts of games, parts
of curves, and parts of stories \citeyearpar[18--19]{inwagen1995}?
\end{squote} 

Van Inwagen suggests that `part' does have a number of different
meanings.  Later he says that ``there is one relation called
`parthood' whose field comprises material objects\,\ldots\,another
relation called `parthood' defined on events, another still defined on
stories, yet another defined on curves, and so on''
\citeyearpar[19]{inwagen1995}.

This may very well be, but what is the similarity between these
relations?  The parthood relation in classical mereology is
well-defined, as is the membership relation in set theory.  But there
is no equally well-defined relation for event parthood, or poem parthood.

However, Kit Fine has proposed a theory of parthood that takes
seriously the possibility that there are a plurality of different
parthood relations, and has the resources to define numerous parthood
relations.  The parthood relation for poems may be given as rigorous a
treatment as the parthood relation for sets.
\end{spacing}
\end{document}

%%%%%%%%%

\subsection{Why bother?}
A metaphysical thesis that involves denying the existence of ordinary
things like chairs entails that the simplest explanation of why we
believe that there are chairs is incorrect.  I believe that such a
thesis should therefore be supplemented with a new explanation.  This
new explanation would identify the reasons why we would believe that
there are chairs if there are in fact none.  But why should I demand
this of a metaphysical theory?  Is it a reasonable request?

As an analogy, consider color.  Most people believe that things are
colored.  A simple causal story about why people believe that things
are colored might go like this:  things are colored, and people see
that things are colored.  

But imagine a philosopher who holds some version of {\em physicalism}
and claims that the world as described by physics is all that there
is.  This view is often thought to have the consequence that things
aren't actually colored.  In the `vocabulary of physics', things might
be described in such a way that the things color gets somehow left
out.  We may be unable to determine from the `physical description'
what color the object is.  The colors of objects are not included in
this philosopher's description of the world.

If the philosopher admits that people believe that things are colored,
she cannot explain this using the same story that I used above.  I
said that people believe that things are colored and that they see
that things are colored.  But the physicalist maintains that things
are not colored.  {\em If} she admits that people believe that things
are colored, then she needs a different explanation as to why people
believe that things are colored.

She might, however, deny that people believe that things are colored.
(This would be a rather bold claim.)  She could say that the notion of
color is entirely illusory.  If we believe that we see colors, she may
tell us we are wrong.  When we think that something is colored, we are
mistaken.  If we think that an apple is red, we have a false belief.
She might claim that color does not pose a difficulty for her view,
because humans do not experience `color'.

This, as I said, is a rather bold claim.  It seems simply true that we
see colors and that the apple looks red.  If a philosopher were to
deny these things, I would have difficulty understanding what she
meant.  This is not to say she is {\em wrong}; I have no argument
proving that her thesis is false.  But the claim that humans do not
experience color seems bizarre and unmotivated.  Fortunately I do not
know of anyone who actually holds this view.

Our imagined philosopher might make a less bold claim.  She might
instead claim that color is one of those things that are `subjective'
rather than `objective' or `absolute' features of the world.  A
subjective feature of the world is a feature that is present only
because we (or some other being) exists to experience it:

\begin{squote}
Whatever is due only to us and to our own ways of responding to and
interacting with the world does not reflect or correspond to anything
present in the world as it is independently of us.  The aim of an
``absolute'' conception, then, is to form a description of the way the
world is, not just independently of its being believed to be that way,
but independently, too, of all the ways in which it happens to present
itself to us human beings from our particular standpoint within
it\,\ldots\,[So we] form some conception of that independent reality
and come to understand parts or aspects of our original conception of
the world as not representing it as it is.  If we see them as products
or reflections of something peculiar to human experience or to the
human perspective on the universe, we assign them a merely
``subjective'' or dependent status and eliminate them from our
conception of the world as it is independently of
us~\citep[30--31]{stroud2000a}.
\end{squote}

A philosopher who adheres to this distinction might claim that our
conception of the world as colored does not represent the world as it
is independently of us.  Colors, she would claim, are not objectively
real.  She allows, however, that they are subjectively real.  She
admits that people do see colors.  Because of our color vision, we
come to believe that the things we see are colored.  A philosopher who
denies the objective reality of color does not thereby ``deny that we
perceive many different colours or that we believe physical objects to
be coloured'' \citep[145]{stroud2000a}.  What this philosopher claims
is something to the effect that, while we see things {\em as} colored,
things are not {\em themselves} colored.  The red color of a tomato,
on this view, obtains only in our perception of the tomato; there is
nothing {\em in} the tomato that is the redness (other species may not
see the redness when they see the tomato).

The philosopher who is denying the objective reality of color does
``recognize the presence in the world of perceptions of and beliefs
about the colours of things'' \citep[199]{stroud2000a}.  The challenge
then is for her to explain why we do have these perceptions and
beliefs.  If she believes that only the world of physics is
objectively real, she must explain why we hold these beliefs, and she
must give this explanation in such a way that commits her only to the
existence of physical things.  If she claims that the world as
described by physics is the only world there is, then she must explain
why, in a world that contains only physical things, we come to believe
that there are colors and colored objects.

Again: if our physicalist philosopher admits that people believe that
they experience color, and admits that people believe that things are
colored, {\em then} she commits herself to explaining why we form
beliefs that are, according to her, false.  Here is the analogy with
metaphysicians like Trenton Merricks: {\em if} they admit that many of
us believe that there are chairs and other ordinary objects, then they
commit themselves to explaining why we form these false beliefs.  For
as we have seen, even false beliefs are generally held for a reason.

%%%%

If we defend this `essentialist' answer, we will have to interpret
talk about `the same chair' over time as being talk about {\em
  different} sums over time.  The appearance of persistence over time
(and through change) is a product of {\em conventions} that govern
terms like `chair'.  What mereological sum is the referent of `the
chair' from one day to the next depends not upon the actual identity
of the two referents, but upon convention.

%%%%

I argued in section \ref{plural-ref} that things like teams, crews,
and families are indeed {\em things}.  Terms like `team', `crew', and
`family' are not disguised references to plurals.  Moreover, things
like teams are things with {\em parts}.  The rugby players are each
{\em part} of the Reed College women's rugby team.  The team is made
up of---it is composed of---the players.

When I say that the players are part of the team, or that the
crewmembers are part of the crew, or that I am part of my family, is
that use of `part' the same as when I say that the tree is part of the
dogbush, or that the seat is part of the chair?  Are {\em any} of
these uses of `part' the same?

%%%%%% GROUPS AND THING
Take, for example, the Reed College women's rugby team.  There is,
obviously, such a team (whether or not it has College funding).  The
rugby team exists.  Having established this, there are various
consequences.  For instance, each player is part of the team.  The
team is made up of the players and the coach.  Expressing this
formally, we might say that there is some thing (the team) composed of
the $x$s (the players and coach).  If this formal treatment is
equivalent to the informal ``there is a team'', and if the informal
phrasing is uncontroversial, the formal phrasing should not be
controversial either.

Merricks has a related example:

\begin{squote}
Consider whether `the Crew of the USS {\em Enterprise}' is a plural
referring expression---akin to `Locke, Berkeley, and Hume'---or,
instead, the name of a single large object with each crew member as a
proper part.  Note, in fact, that there are two questions here.
First, there is the semantic question of what `the Crew of the USS
{\em Enterprise}' is supposed to mean.  Second, there is the
metaphysical question of whether there really is a big physical object
that has all and only the crew members as its parts (at one level of
decomposition), a scattered object that weighs as much as the sum of
the weights of those people taken individually.

I am not sure how to answer the first question.  But, I say, the
answer to the second question is `no'.  Some philosophers would
disagree.  No matter.  The point here---in this section of this
chapter---is not to settle either the metaphysical or the semantic
dispute surrounding `the Crew of the USS {\em Enterprise}'.  It is,
rather, that such disputes are neither here nor there with respect to
everyday uses of `the Crew of the USS {\em Enterprise}'.  `The Crew of
the USS {\em Enterprise}' will continue to perform its ordinary duties
regardless of how or whether the semantic and metaphysical disputes
get settled \citeyearpar[10]{merricks2001a}.
\end{squote}

I will discuss plural referring expressions in a moment.  But first I
want to point out that the last sentence of this quote by Merricks
does not seem to be true.  Suppose I believed that the crew of the
{\em Enterprise}, were it to exist, would be a thing composed of the
crewmembers, {\em and on those grounds} I denied that the crew exists.
If the crew does not exist, if there is no crew, then that could only
mean that the ship was unmanned.  If there are crewmembers, there is a
crew.  If there is a crew, then there are crewmembers.  If Merricks or
anyone denies that there is a crew, what would it mean for them to say
that `the crew' ``will continue to perform its ordinary duties''?

\subsection{What are groups?}
\label{group}
  This new kind is the group.  In section
\ref{parts} will will look at three theories that attempt to make
room for groups.  Unfortunately, they make room for an incredible
amount of other things as well.

%%%%% GROUPS AND THINGS
But I also argued that in addition to this plurality of `material
objects', there are other {\em kinds} of things, such as teams,
families, and judicial bodies.  For example, in addition to the
material object made up of the justices on the Supreme Court, there is
also the Supreme Court itself.  The former thing is located at each
point at which the justices are located, but it seems odd to say that
the Supreme Court is a large, scattered object.  It is more natural to
say that the Court is a {\em group} of physical objects (the
justices), not a physical object itself.  The same goes for the Reed
College women's rugby team, and my family.  There is at least some
reason to recognize kinds of things in addition to material objects.

How, though, do these new kinds of things differ from material
objects?  The material object made up of the Supreme Court justices
and the Supreme Court itself appear to have the same parts: the
justices.  But we are supposing that they are not the same thing.  If
they do not differ in their parts, they must differ in some other way.

In this section we will assess several different theories.  These
theories have to do with how different kinds of things exist and how
they relate to their parts.  We will see that each theory, in order to
differentiate different kinds of things (like the material object made
up of the justices and the Supreme Court), has the consequence that
there are potentially {\em infinite} different kinds of things, and,
worse, potentially infinite kinds of things in any given location.

A theory that posits such {\em co-located objects} has some merits.
For example, some philosophers claim that a statue is an object
distinct from the clay of which it is made.  A problem for this view
is how to distinguish the statue from the clay, when both have the
same parts (the parallel with the Supreme Court and the justices is
obvious).  The theories that we will examine each solve this problem,
though in different ways.

%%%%%% K PARTS AND LUMPS
We can perhaps make this point more clearly by recalling Fine's notion
of $K$-part from section \ref{hybrid}.  $K$-parthood is ``the {\em
  general} relation of part\,\ldots a relation that holds between $x$
and $y$ whenever $x$ is in any way whatever a part of $y$''
\citep[580]{fine2010}.  The very same things (atoms, bits of clay)
that are $K$-parts of the statue are $K$-parts of the lump.  These
things are $K$-parts of at least two different objects.  The following
argument suggests itself:

\begin{enumerate}
  \item Suppose $x \neq y$ but $\forall z$\,($z$ is a proper $K$-part
    of $x$ if and only if $z$ is a proper $K$-part of $y$).
  \item Then for some $\sum_{i}$ and some $\sum_{j}$, $\sum_{i} \neq
    \sum_{j}$ and $x = \sum_{i}(a, b, c, \mathellipsis ,)$ and $y =
    \sum_{j}(a, b, c, \mathellipsis , )$
\end{enumerate}

%% I say ``perhaps'' because several qualifications are required.  First,
%% the applications of $\sum_{i}$ and $\sum_{j}$ that produce $x$ and
%% $y$, respectively, must be {\em generative} applications (see section
%% \ref{generate}).  

One might claim that repeated applications of a particular operator on
the same things may produce multiple distinct objects.  For example,
one might claim that applying the sum operator to some things $a$,
$b$, $c$\,\ldots will result in a lump on one application and a statue
on another.  But this is inconsistent with the rest of Fine's theory.
Call the lump $L$ and the statue $S$.  Fine represents the application
of a composition operator as $L = \sum (a, b, c, \mathellipsis , )$.
The product $L$ {\em just is} the application of the operator; they
are {\em identical}.  If the application of the sum operator produces
both the statue and the lump, they are therefore identical:

\begin{enumerate}
  \item $L = \sum_{s} (a, b, c, \mathellipsis , )$
  \item $S = \sum_{s} (a, b, c, \mathellipsis , )$
  \item[$\therefore$] $L = S$
\end{enumerate}

%% \subsection{Hybrid parts}
%% \label{hybrid}
%% The idea that there are different ways of being a part, corresponding
%% to the different kinds of things produced by different composition
%% operators, allows us to solve certain puzzles about parthood.

%% For example, I am the only member of my singleton (the singleton of
%% $x$ is the set resulting from applying the set-builder to $x$ alone).
%% My hand, for instance, is not a member of my singleton.  But my hand
%% is a part of me.  If I was a part of my singleton, then---because
%% parthood is transitive---my hand would be a part of my singleton.  And
%% if that means that my hand is a {\em member} of my singleton, that is
%% clearly wrong.

%% Fine points out, of course, that the objection makes the mistake of
%% supposing that something (me, my hand) can be a part in only one way
%% (in this case, through set-membership).  Once we recognize that there
%% are a plurality of ways of being a part, it becomes clear that my hand
%% is part of the set in one way, but not in another:

%% \begin{squote}
%% Given the specific relations of part, we may derive various {\em
%%   hybrid} relations of part.  Suppose, for example, that we are given
%% the relations of set-theoretic and mereological part---which we may
%% designate as \textepsilon -part and $m$-part. We may then take one
%% object to be an \textepsilon ,$m$-part of another if it is an
%% \textepsilon -part or an $m$-part or an $m$-part of an \textepsilon
%% -part or an \textepsilon -part of an $m$-part, or an $m$-part of an
%% \textepsilon -part of an $m$-part, and so on. More generally, if $K$
%% is a family of specific ways of being a part, we may take an object to
%% be a {\em K-part} of another if $x$ and $y$ can be linked by
%% relationships of $k$-part for $k$ in $K$ \citep[579]{fine2010}.
%% \end{squote}

%% My hand is a \textepsilon ,$m$-part of my singleton, but not a
%% \textepsilon -part.

%% By conjoining every way of being a part, we arrive at the most general
%% notion of part:

%% \begin{squote}
%% Among the hybrid relations of part, of special interest is the
%% relation of $K$-part where $K$ is the family of {\em all} the specific
%% ways of being a part.  This is the relation of $K$-part that holds
%% between two objects when they may be linked by relationships of
%% $k$-part without restriction on $k$.  We might call it the {\em
%%   general} relation of part, and it is a relation that holds between
%% $x$ and $y$ whenever $x$ is in any way whatever a part of $y$
%% \citep[580]{fine2010}.
%% \end{squote}

%% So in addition to saying that my hand is a \textepsilon ,$m$-part of
%% my singleton, it is also a $K$-part of my singleton.

%% %% Related to this notion of hybrid parts is the distinction Fine draws
%% %% between {\em parts} and {\em components}.  To illustrate this, take
%% %% the set \{Socrates, \{Socrates, Plato\} \}.  This set was produced by
%% %% applying the set-builder to Socrates and \{Socrates, Plato\}.  These
%% %% two objects are the {\em components} of the set: ``$x$ is a {\em
%% %%   component} of $y$ if $y$ is the result of applying $\sum$ to $x$ or
%% %% to $x$ and some other objects'' \citep[567]{fine2010}.  A {\em part},
%% %% as 

%% \subsection{Generating kinds}
%% \label{generate}
%% On this theory, what kind a thing is depends on what operation
%% produced it.  If a chair or a dogbush is a mereological sum, then this
%% is because they are produced by the summation operation.  The Dunn
%% family is `produced' by the family operation.  Groups are produced by
%% the group operation (see section \ref{group}).

%% But there is a difficulty to be avoided here.  As we saw in section
%% \ref{classical}, the mereological sum of a single thing $x$ is just
%% $x$.  Therefore there is a sense in which every physical thing,
%% including every simple, is a mereological sum, for the application of
%% the summation operation would just produce that thing.  To avoid this
%% consequence Fine introduces the notion of a {\em generative}
%% application of an operation:

%% \begin{squote}
%% We might say that the application $y = \Gamma (x_1, x_2, x_3,
%% \mathellipsis )$ of an operation $\Gamma$ is {\em generative} if there
%% is an explanation of the identity of $y$ as $\Gamma (x_1, x_2, x_3,
%% \mathellipsis )$; and we might say that the operation $\Gamma$ is
%% itself {\em generative} if it permits a generative application. Thus
%% both the set-builder and the operation of predication will be
%% generative in this sense \citeyearpar[582]{fine2010}.
%% \end{squote}

%% Whether or not the summation operation is generative depends on the
%% things it is being applied to.  When summing a dog and a tree, it is
%% generative; when summing a dog by itself, it is not.

%% For any operation, there will be things it applies to that it cannot
%% produce.  The summation operator fuses simples, but cannot produce
%% them; the set-builder combines many things that it cannot produce
%% (like letters).  For any given operation, there is a `level 0'
%% consisting of the things that the operator itself cannot produce:

%% \begin{squote}
%% We suppose that certain objects are simply given.  These are the
%% objects whose identity does not require an explanation in terms of
%% $\Gamma$.  Thus, when $\Gamma$ is the set-builder, they are the
%% objects that are not sets and, when $\Gamma$ is summation, they are
%% the objects that are not sums or, rather, the objects that do not need
%% to be seen as sums.

%% We now `generate' objects in stages.  At stage 0 are the givens; at
%% stage 1, we add the objects that result from a single application of
%% the generative operation $\Gamma$ to the givens \citep[583]{fine2010}.
%% \end{squote}

%% An application can now be identified as generative in a strong or a
%% weak sense:

%% \begin{description}
%%   \item[Strong generative application] Also called `strict' by Fine, a
%%     ``[strong] generative application of $\Gamma$ to the objects $x_1,
%%     x_2, \mathellipsis$ can now be defined as one in which $y = \Gamma
%%     (x_1, x_2, \mathellipsis )$ is of a higher level than each of
%%     $x_1, x_2, \mathellipsis$'' \citeyearpar[584]{fine2010}.  For
%%     example, summing the simples $x$ and $y$ to produce the fusion $z$
%%     would be a strong generative application of the summation
%%     operator; the simples are level 0 and $z$ is level 1.  Summing two
%%     composites, or a composite and a simple, would not be strongly
%%     generative; one or both of the parts would be the same level (1)
%%     as the product.
%%   \item[Weak generative application] To illuminate this notion Fine
%%     introduces another, that of a {\em putative generative
%%       application}: ``Let us say, in the first place, that $y = \Gamma
%%     (x_1, x_2, \mathellipsis )$ is a putative generative application
%%     of $\Gamma$ if $y$ is of a higher or of the same level as each of
%%     $x_1, x_2, \mathellipsis$.  This gives us the notions of a
%%     putative prior component and of a putative prior in the usual way.
%%     We now say that the application $y = \Gamma (x_1, x_2,
%%     \mathellipsis )$ of $\Gamma$ is a {\em weak} generative
%%     application if it is the putative generative application and if
%%     $y$ is not putatively prior to any of $x1, x2, \mathellipsis$.  We
%%     can get from $x_1, x_2, \mathellipsis$ to $y$ without an ascent in
%%     level but not from $y$ to any of $x_1, x_2, \mathellipsis$''
%%     \citeyearpar[584]{fine2010}.
%% \end{description}

%% Applying the summation operator to a simple is neither strongly nor
%% weakly generative.  It is not strongly generative because the result
%% is a simple, which is at level 0---the same level as its part
%% (itself).  It is not weakly generative because the result of the
%% operation is putatively prior to its parts.

%%%%

However, I suggest that, strictly speaking, the Brick House {\em did}
exist last Tuesday.  But last Tuesday it (the sum) did not meet the
criteria for being referred to as a house.  Today I point to the Brick
House and say ``last Tuesday that was just a pile of bricks.  Now it's
a house!''  By `that' I mean the Brick House---the sum---which was a
pile of bricks on Tuesday.  If I say ``the Brick House did not exist
last Tuesday'' I should be taken to mean just that the Brick House did
not meet the criteria for being referred to as `the Brick House' last
Tuesday.

%%%%%

This is a drawback for the set identity theorist, but I do not think
it is a great one.  As a parallel case, take the proposition ``the
temperature is rising''.  For this to be literally true, there would
have to be some thing---the temperature---that is, in some sense,
rising.  But it is plausible to interpret a speaker who says ``the
temperature is rising'' as meaning that the number $x$ that is the
referent of `the temperature' is lower than the number $y$ that will
be the referent of `the temperature'.  Likewise, when I say ``the
Supreme Court is now more diverse'' I mean that the members of the set
that is the referent of `the Supreme Court' are more diverse than the
members of the set that was (or of the sets that were) the referent of
`the Supreme Court'.

It may be due to this fact that we so easily misinterpret uttered
propositions like ``the Supreme Court is the Special Committee''.  A
listener might take this to mean that the members of the sets that
have been (and will be) the Supreme Court are identical with the
members of the sets that have been (and will be) the Special
Committee.  They would therefore evaluate the utterance as false.

Another example that the set identity thesis predicts as non-literal
is ``the Supreme Court has become more conservative''.  If we claim
that the Supreme Court is a set, we cannot interpret this utterance as
literally true; {\em sets} do not have political leanings.  Someone
who utters this, according to the set identity thesis, must be taken
to mean that the members of the sets that have been the referents of
``the Supreme Court'' have become more conservative.

But can a philosopher who distinguishes groups from sets interpret
this literally?  Can groups {\em literally} have political leanings?
Or must the speaker be interpreted as meaning that the members of the
group have become more conservative?  I think it is plausible that
``the Supreme Court has become more conservative'' must be interpreted
as non-literal, whether the Supreme Court is a group or a set.

What about an utterance such as ``the Supreme Court ruled against the
defendant''?  If the Supreme Court is a set, this utterance will have
to be interpreted as non-literal.  Sets don't {\em do} things; we will
have to interpret the speaker as meaning that the Supreme Court
justices ruled against the defendant.  But what if the Court was
divided over the ruling?  If several justices wrote dissenting
opinions, it seems that {\em they} didn't rule against the defendant.
Rather, we want to say that the {\em group} ruled against the
defendant.  The proponent of groups may be in a slightly stronger
position here.  But if we identify the Supreme Court with a set, we
can still say that the {\em majority} of the Supreme Court justices
ruled against the defendant.  And that is more or less what we mean
when we say ``the Supreme Court ruled against the defendant''.

Whether or not we identify the Supreme Court and other groups with
sets, we will have to interpret some apparently literal speech as
non-literal.  But the set identity thesis, at least with regard to
this slate of examples, treats talk about groups more consistently
than does the theory that groups are distinct from sets.

%%%%

Markosian himself suggests a fourth response to the Special
Composition Question.  He claims that, while there is indeed no ``no
true, non-trivial, and finitely long answer to [the Special
  Composition Question]'' \citeyearpar[214]{markosian1998a}, this is
not because we should refer questions of composition to the empirical
sciences.  Rather, he claims that whether or not some things compose
another is simply a {\em brute fact}.

This is a clever reply, but whether true or not I do not think it
serves the purpose that Markosian expects it to.  He proposes his
theory of `brutal composition' to be ``consistent with standard,
pre-philosophical intuitions about the universe's composite objects''
\citeyearpar[211]{markosian1998a}.  But his theory will only be
consistent with such intuitions if, first, it is a brute fact that all
(or nearly all) of the things we ordinarily take to exist (chairs
etc.) do in fact exist, and, second, that it is a brute fact that the
things that we don't take to exist (or that Markosian takes not to
exist) don't in fact exist.  The chance that the brute facts of
composition happen to line up with our (or Markosian's) intuitions
seems to be incredibly low.

I think that if Markosian is right that there is no interesting answer
to the Special Composition Question, it is not because whether things
compose other things is a brute fact.\\
