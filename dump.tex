\documentclass[11pt]{article}
\usepackage[left=1.75in, right=1.75in, top=1.25in, bottom=1.25in]{geometry}
\geometry{letterpaper}
\usepackage{graphicx}
%\usepackage{xyling}
%\usepackage{tipa}
%\usepackage{exaccent}
%\usepackage{txfonts}
%\usepackage{pxfonts}
\usepackage{enumitem}
%\usepackage{amssymb}
\usepackage{amsmath}
\usepackage{epstopdf}
\usepackage{setspace}
\usepackage{natbib}
\setcitestyle{aysep={}}
\usepackage{hyperref}
\usepackage{url}
\synctex=1

\DeclareSymbolFont{symbolsC}{U}{txsyc}{m}{n}
\DeclareMathSymbol{\strictif}{\mathrel}{symbolsC}{74}
\DeclareMathSymbol{\boxright}{\mathrel}{symbolsC}{128}		

\newenvironment{squote}{%
       \begin{spacing}{1}
       \begin{list}{}{%
       \setlength{\labelwidth}{0pt}%
       \rightmargin\leftmargin%
       }
       \item\relax
       }{%
       \end{list}%
       \end{spacing}
}

\title{The dump}
\author{Alexander A. Dunn}
\begin{document}
\maketitle
\begin{spacing}{1}


\section{Other answers that are false}
This section will mostly involve me registering agreement with
Markosian on the insufficiency of other proposed answers to the
SCQ. For example:
\begin{description}
	\item[Fastenation] Necessarily, for any $x$s, there is an
          object composed of the $x$s iff the $x$s are fastened
          together.~\citep[223]{markosian1998a}
\end{description}
No general notion of `fasten' will make this plausible. ``Suppose that
van Inwagen and I shake hands, and suppose that just as we do so, our
hands becomes paralyzed [or super-glued together], so that we cannot
pull them apart. Then, according to Fastenation, there is a new
composite object in the
world''~\citep[224]{markosian1998a}.\footnote{This is based on an
  example of van Inwagen's~(\citeyear[57--58]{inwagen1995}.} But we do
not recognize van Inwagen and Markosian as composing an object; `van
Mark', thankfully, does not emerge. Fastenation is not then
sufficient. Nor is it necessary; I can create a house of cards without
fastening any of them together.

Another approach is this:
\begin{description}
	\item[The Serial Response] The correct answer to the SCQ is an
          instance of this schema: There is an object composed of the
          $x$s iff {\em either} the $x$s are F1s and related by R1,
          {\em or} the $x$s are F2s and are related by R2, {\em
            or}\,\ldots\,the $x$s are Fns and are related by
          Rn.~\citep[230]{markosian1998a}
\end{description}
But seeing as we have not {\em one} plausible relation between objects
that is necessary and sufficient for composition, it seems unlikely
that we can find a series of them, limited in scope though they may
be. (For what are the necessary and sufficient conditions for a deck
of cards composing a house of cards?)

\section{Arbitrariness}
Where does this leave us? [stuff about realism]

\section{Now what?}
We have come to the following conclusions:
\begin{enumerate}
	\item Ordinary things (tables, chairs, people, \&c.) exist,
          and we communicate intelligibly about them.  Therefore, most
          versions of nihilism are false.
	\item Because we communicate intelligibly about ordinary
          things, we reject versions of universalism that entail a
          profusion of tables where we assume a single table to be.
	\item Van Inwagen's paraphrasing strategy could not coherently
          show that a literally true proposition expressed by `There
          are chairs\,\ldots ' does not entail that there are chairs.
          He cannot appeal to loose truth until he has given
          non-circular definitions of `chair' and `chairwise'.
	\item No such definition seems to be forthcoming.
\end{enumerate}

Although the theories presented by Unger and van Inwagen are
unsatisfactory, the problems they are meant to solve do require some
answer.  Despite the fact that we communicate perfectly well with
terms like `chair' and `person', it is difficult to spell out exactly
what it is that we use these terms to refer to.  A chair may lose some
of its matter (we might sand it down) or gain more matter (we might
paint it), yet we generally assume that the same chair endures the
change.  We shed cells constantly, but it would be overhasty to
conclude that we do not persist through time, because we are not
continuously constituted out the very same matter.  When we refer to
these chairs and people, then, it seems that the things we are
referring to are not simply the particles that make them up.  But then
what are they?

\end{spacing}
\end{document}
