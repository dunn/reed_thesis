\documentclass[11pt]{article}
\usepackage{standalone} \newif\ifstandlone \standalonetrue
\usepackage[left=1.75in, right=1.75in, top=1.25in, bottom=1.25in]{geometry}
\geometry{letterpaper}
\usepackage{graphicx}
\usepackage{enumitem}
\usepackage{amssymb}
\usepackage{amsmath}
\usepackage{epstopdf}
\usepackage{verbatim}
\usepackage{setspace}
\usepackage{natbib}
\setcitestyle{aysep={}}
\usepackage{hyperref}
\usepackage{url}
\synctex=1

\DeclareSymbolFont{symbolsC}{U}{txsyc}{m}{n}
\DeclareMathSymbol{\strictif}{\mathrel}{symbolsC}{74}
\DeclareMathSymbol{\boxright}{\mathrel}{symbolsC}{128}

\newenvironment{squote}{%
\begin{spacing}{1}
\begin{list}{}{%
\setlength{\labelwidth}{0pt}%
\rightmargin\leftmargin%
}
\item\relax
}{%
\end{list}%
\end{spacing}
}

\title{Essentialism}
\author{Alexander A. Dunn}
\begin{document}
\ifstandalone
\maketitle
\begin{spacing}{1.5}
\fi

\label{essential}

In section \ref{parts} I presented three different theories that
modified classical mereology.  These modification were made to
explain, among other things, how objects change their parts over time
and how co-located objects (like the statue and the lump) have
different properties.  But each of these theories require us to posit
an extraordinary plurality of co-located objects.  A theory that posits
co-located people is, I think, clearly false, but any theory that
entails the existence of a plurality of overlapping objects is at
least intuitively objectionable.

In this section, therefore, I will attempt to sketch an {\em
  essentialist} theory of things.  This theory will allow us to reject
the `plurality thesis'---that there are pluralities of co-located
objects---but it will have problems of its own.  The most glaring is
the consequence that, strictly speaking, things don't change their
parts.

In section \ref{problem3} I argued that using Fine's theory of
composition operators to account for groups like the Supreme Court led
to a plurality of different {\em kinds} of things rather than a
plurality of a single kind of group.  I suggested that this was a
drawback of the theory, and that we should re-examine the idea that
groups like the Supreme Court are actually identical with sets.

Before we assess the merits of that controversial thesis, there is
another possibility that should be addressed.  If we assume the theory
of {\em four-dimensionalism}, many of our problems appear to go away.
Unfortunately, new ones arise.

\section{What if we assume four-dimensionalism?}
\label{4d}
So far I have been supposing that four-dimensionalism is false.  That
is, I have assumed neither that things are composed of temporal as
well as spatial parts, nor that the past and future exist.  But if we
{\em do} assume four-dimensionalism, we have access to new solutions
to the problems relating to ordinary things.

What is four-dimensionalism?  I am supposing that
`four-dimensionalism' refers to the conjunction of two theories.  The
first is that things have {\em temporal parts}.  The second is {\em
  eternalism}.

Ted Sider presents a relatively clear picture of the first part of the
theory of four-dimensionalism, that of temporal parts:

\begin{squote}
Think of your life as a long story.  Let the story be a rather
narcissistic story: cut out all details about everything else except
you.  So the story begins with an infant (or perhaps a fetus).  It
describes the infant developing into a child and then an adolescent.
The adolescent passes into young adulthood, then adulthood, middle
age, and finally old age and death.  Like all stories, this story has
parts.  We can distinguish the part of the story concerning childhood
from the part concerning adulthood.  Given enough details, there will
be parts concerning individual days, minutes, or even instants.

According to the `four-dimensionalist' conception of persons (all all
other objects that persist over time), persons are a lot like their
stories.  Just as my story has a part for my childhood, so {\em I}
have a part consisting just of my childhood.  Just as my story has a
part describing just this instant, so I have a part that is
me-at-this-very-instant \citeyearpar[1]{sider2001}.
\end{squote}

The claim that we have these {\em temporal
  parts}---me-at-this-instant, or me-as-a-child---relies on a close
analogy between space and time.  It is relatively uncontroversial to
claim that we have {\em spatial} parts.  My foot is a part of me, for
instance, but it is not {\em all} of me (it is a proper part, to use
mereological terms).  The philosopher who claims that we have temporal
parts is saying that, likewise, my adulthood is a part of me, but it
is not all of me.  My childhood is---or was, if we do not assume
eternalism---another part of me.  My infancy, childhood, adulthood,
etc. together {\em compose} me.

This theory of temporal parts is often conjoined with a theory about
time.  This theory is commonly referred to as {\em eternalism}.
According to eternalism, ``time is like space.  There is nothing
special about the things here; things at other places are just as
real; no place is metaphysically distinguished.  Similarly, for the
eternalist, there is nothing special about the present; things at
other times are just as real; no time is metaphysically
distinguished'' \citep[122]{hinchliff1996}.  For the eternalist, there
is a sense in which ``there are dinosaurs'' is true.  Everyone agrees
that there are no dinosaurs {\em now}; the question is whether the
dinosaurs of the past still exist {\em in the past}.  

I have no firm intuition on whether either conjunct of the
four-dimensionalist theory is correct.  I do not know whether things
have temporal parts, and I do not know if the past (and future) exist.
Nonetheless I will assume in this section that four-dimensionalism is
true; {\em if} this assumption is correct, we can explain the
existence of ordinary things in new and interesting ways.

\subsection{Four-dimensional essentialism}
\label{4de}
According to four-dimensionalism, ordinary things like chairs and
statues are {\em four-dimensional spacetime worms}.  They are composed
of temporal parts or {\em slices}; a chair might be made up of
`chair-slices' at $t_{1}$, $t_{2}$, $t_{3}$\,\ldots\,, etc.  Depending
on who you ask, these `slices' have a very small temporal duration or
none at all.  If the latter, they are {\em extended} in only three
dimensions; their temporal extension is point-sized (this is what I
will assume).

Four-dimensionalism is very commonly conjoined with universalism---the
theory, defended in section \ref{universe}, that for any things, there
is something composed of them.  If we assume universalism, then
four-dimensionalism entails that for every set of temporal slices,
there is something composed of them.  There is an object composed of
the first ten years of my life, the Kremlin from 1970--1990, and one
second of a puppy's existence in 2020.  This thing is not, of course,
a person; nor is something composed of the first 10 years of my life
and the last ten years of someone else's.  Certain causal or
psychological connections must hold between the temporal parts of a
thing in order for it to be a person.

The objects composed of these temporal slices are mereological sums in
the classical sense.  Let us use `Krupkin' to designate the object
made of the first ten years of my life, the Kremlin from 1970--1990,
and one second of a puppy's existence in 2020.  Because the past and
future exist (we're assuming eternalism), Krupkin always has the same
parts.  Strictly speaking, it doesn't ever change its parts.  In 1991
it is true to say ``the Kremlin is not {\em now} part of Krupkin'',
but it is not true to say ``the Kremlin is not part of Krupkin.

If we assume universalism in addition to four-dimensionalism, then not
only does Krupkin not change its parts, it {\em cannot} change its
parts.  It cannot change its parts for the reason given in section
\ref{change}.  Let us use `Alkin' to designate the object composed of
the first 10 years of my life and the Kremlin from 1970--1990.  Now if
Krupkin could change its parts, it could lose a part.  Suppose it lost
its puppy part.  Then, if it still exists, it would be the object
composed of the first 10 years of my life and the Kremlin from
1970--1990.  But {\em that} object is Alkin; Krupkin would therefore
become identical with Alkin.  Alkin and Krupkin are not identical,
however, because Krupkin has a property that Alkin does not: the
property of having had a puppy as a part.  So Krupkin cannot, in fact,
lose a part; otherwise we would have a contradiction.

Technically, therefore, four-dimensional universalism is a version of
{\em essentialism}---the thesis that things cannot change their parts.
Four-dimensionalists explain change by relativizing things to times:
for a thing to change its color is just for it to have the relation of
being green at one time and red at another, or having a part at one
time and not at another.

I am somewhat sympathetic to this view.  In section \ref{essential} I
will sketch an essentialist theory of things, but one that presupposes
neither temporal parts nor eternalism.  But here I will briefly
examine how a four-dimensionalist essentialism addresses the issues
related to ordinary things that we have been concerned with.

Four-dimensionalism has two advantages and two disadvantages, when
compared with the three theories above.  The first advantage is that
four-dimensionalism does not posits a plurality of {\em kinds} of
things.  The second is that it does not posit co-located objects.  The
material objects that a four-dimensionalist recognizes are all
mereological sums in the classical sense.  The first disadvantage is
that four-dimensionalism, when conjoined with universalism, produces a
plurality of objects, just as the three theories above do.  The second
disadvantage is that four-dimensionalism has difficulty distinguishing
objects that are co-located for the entirety of their existence.

\subsection{Four-dimensional solutions}
\label{4ds}
The first advantage of four-dimensionalism---that it does not have to posit
a plurality of kinds of things---is primarily an advantage relative to
Fine's theory of composition operators (section \ref{fine-c}).  That
theory produced an incredible plurality, not only of things in
general, but of different kinds of things.  Four-dimensional things
are simply mereological sums, in the classical sense.

The second advantage of four-dimensionalism is that, unlike all three
theories presented above, it does not posit co-located objects.  The
theories in section \ref{parts}, in order to distinguish objects like
the statue and the lump---objects that (currently) share all their
parts---had do posit co-located objects.  But on the four-dimensional
picture, this is unnecessary.  Suppose that the lump is formed on
Monday, and the statue on Tuesday.  The lump therefore has temporal
parts that are `earlier' than any of the statue's parts.  They do not
share all their parts, and so are not co-located.  It is true that
they share all their Tuesday parts; the temporal slices that compose
the lump on Tuesday are the same that compose the statue on Tuesday.
But they share parts only at certain times.  They do not share all
their parts at all times.  (This leads into a problem for
four-dimensionalism, however; it does not appear to let us
differentiate a statue and a lump that {\em always} share their parts.
See section \ref{4dp}).

Four-dimensionalism can also readily account for the apparent change
in membership of the Supreme Court.  Since people (including the
justices) are composed of temporal slices, the Supreme Court can be
identified with the set of person-slices that correspond to the
various justices' terms.  For example, the 25 September 1981--31
January 2006 temporal parts of Sandra Day O'Connor are part of the
set.  The phenomenon of the Supreme Court `changing' its members
(e.g., Day O'Connor retiring) is simply it being 1 February 2006 and
that temporal part of Day O'Connor not being part of the set.

\subsection{Problems for four-dimensionalism}
\label{4dp}
But there are two disadvantages to four-dimensional universalism.  The
first is that while four-dimensionalism does not posit a plurality of
kinds of things or a plurality of co-located objects, there is still a
sense in which it is a `plurality thesis'.  Any given temporal slice
is part of a plurality of things.  When I point at my chair, I am also
pointing at a thing composed of my chair and a black bear from the
1800s, as well as a thing composed of my chair and the head of Thomas
Aquinas.  All those things (and {\em many} more) are currently located
in the very same place.

This is certainly bizarre, but it no more {\em disproves}
four-dimensionalism than does it disprove the three theories
previously examined.  Unfortunately there is another disadvantage to
four-dimensionalism, one that does threaten it as a theory.

The second disadvantage of four-dimensionalism is that it has trouble
distinguishing between objects that are co-located for the entirety of
their existence.  Suppose that I have two lumps of clay; I form one
into the top half of a figure and I shape the other into the bottom
half.  Having done this, I stick the two pieces of clay together,
forming a statue.  When I do this I also form a new, larger lump of
clay.  I admire the statue and the lump for a little while, then smash
them with a hammer.

Let $S$ be the thing composed of all the statue-slices.  Let $L$ be
the thing composed of all the larger-lump-slices.  $S$ and $L$ are
mereological sums composed of the very same parts; $S = L$.  But if
the statue had been squashed instead of smashed, $L$ would have
survived; but $S$ would not have survived being squashed.  $L$ has a
property that $S$ does not---the property `could survive being
squashed'---and therefore $S \neq L$.  This is a problem.

The four-dimensionalist could say that the lump would not have
survived being squashed, or that the statue would have survived.
Since there is only one thing under investigation (since $S = L$),
that thing must have a consistent set of properties.  It can't be such
that it would both survive and not survive a squashing.  So the
four-dimensionalist will have to say that one of our two intuitions is
wrong.

But there is another, related, difficulty.  In the case just
presented, the statue is the lump ($S = L$).  But suppose there is a
situation exactly like the one presented, but in which the statue and
lump are first squashed, then smashed.  In this case, we are inclined
to say that the lump $L^{\prime}$ continues to exist after the
squashing.  Its parts include temporal slices of the clay after it has
been squashed.  In the case of the statue $S^{\prime}$, however, we
are inclined to say that the statue does not have any temporal parts
after the squashing.  The statue is destroyed when it is squashed.
Since $L^{\prime}$ and $S^{\prime}$ have different parts, they are not
the same thing; $L^{\prime} \neq S^{\prime}$.  The four-dimensionalist
is committed to the claim that whether there is one thing (a statue
that is also a lump) or two things (a statue and a lump) on the table,
and what that thing's (or those things') modal properties are depends
upon whether I squash or smash it.  This seems highly implausible.  By
choosing to squash the statue rather than smash it, do I thereby {\em
  make it the case} that there were two things, rather than one?

The four-dimensionalist will object that, since the future already
exists, it was {\em already} true that there were two things (it has
always been true).  But claiming that it is already the case that I
will squash the statue seems to commit the four-dimensionalist to some
version of {\em determinism}---the thesis that, roughly, the events of
the future are determined, or fixed, to occur.  This may well be true,
but it is largely an empirical hypothesis; to rely on it here would be
unwise.  (If the four-dimensionalist does not assume determinism, and
instead assumes indeterminism, they will presumably have to say that,
since it is indeterminate whether or not I will squash the statue, the
number of things on the table is therefore also indeterminate.  This
seems even worse.)

\subsection{Moving along}
\label{4dc}
Four-dimensionalism allows for the resolution of a number of puzzles
related to ordinary things.  It does not resolve everything, however,
and it introduces a few problems of its own.  Moreover, it requires a
number of controversial assumptions: the theory of temporal parts,
eternalism, and possibly determinism.  I will therefore set aside
four-dimensionalism, and suppose henceforth that the past and future
do not exist, and that things do not have temporal parts.

\section{Re-examining the set identity thesis}
\label{set-id}
The primary motivation cited in section \ref{why-group} for positing
groups was the fact that the Supreme Court changes its members over
time.  For example, both of the following sentences are true:

\begin{enumerate}[label=(\arabic*)]
  \item The Supreme Court ruled on Roe vs.\ Wade in 1973. \label{roe1}

  \item The set of justices now serving as Supreme Court Justices did
    not rule on Roe vs.\ Wade in 1973
    \citep[135]{uzquiano2004a}. \label{roe2}
\end{enumerate}

One way to accommodate these facts is to ``insist that the Supreme
Court is a set, but to abandon the assumption that there is a single
set to which the phrase `the Supreme Court' refers in sentences
\ref{roe1} and \ref{roe2}'' \citep[138]{uzquiano2004a}.  To
successfully use the term `the Supreme Court' to refer to a set of
justices, there must be an implicit or explicit temporal reference.
If an utterance of \ref{roe1} is true it will be true because it the
speaker intends her audience to recognize her intention to refer to
the set of justices that was the Supreme Court in 1973.  If her
audience, for whatever reason, takes her to be referring to the
current Court, then they will evaluate \ref{roe1} as false.

Considered in this light, `the Supreme Court' is used to express a
relation between sets and times; ``$x$ is the Supreme Court at $t$''
\citep[140]{uzquiano2004a}.  There is some precedent for this sort of
interpretation:

\begin{squote}
Our use of the phrase `the Supreme Court' to express a relation a set
of justices bears to a time is much like our use of the phrase `the
president of the United States' to express a relation an individual
bears to a time.  Different persons may be the president of the United
States at different times, but there is at most one person that bears
that relation to each time \citep[138]{uzquiano2004a}.
\end{squote}

``But,'' it will be objected, ``there is an important difference here.
We use both terms---`the Supreme Court' and `the president'---to refer
to a past, present or future set that `is' the thing, but we also use
`the Supreme Court' to refer to {\em the Supreme Court}, which has
changed its membership over time.  If I say, `the Supreme Court has
become more conservative over the past century', there is no one set I
am referring to.  I must be referring to something else; the obvious
candidate is the {\em group} that is the Court.''

One reply here is to claim that all that what ``the Supreme Court has
become more conservative over the past century'' actually means is
that the members of the sets that have been the Supreme Court have
become more conservative.  Another, similar reply is that someone who
utters ``the Supreme Court has become more conservative over the past
century'' is saying something literally false (either because there is
no unique set that is being referred to, or because there is a unique
set referred to, but one that does not make the proposition true), but
can generally be understood to mean something else; namely, that the
members of the sets that have been the Supreme Court have become more
conservative.

Neither reply is {\em very} unintuitive; indeed, there is something
attractive about a thesis that reserves application of adjectives like
`conservative' for people, rather than other things like groups.

But there is a more pressing worry for the set identity thesis.
Recall that the set that is the Supreme Court at a given time might
also be the Special Committee on Judicial Ethics.  We must admit that
the Supreme Court in 2004 is the set \{Rehnquist, Stevens, O'Connor,
Scalia, Kennedy, Souter, Thomas, Ginsburg, Breyer\}, and the Special
Committee in 2004 is that very same set.  But now we are committed to
this argument:

\begin{enumerate}[ref=(\arabic*)]
  \item The Special Committee on Judicial Ethics is one of the
    committees assembled by the Senate.

  \item The Special Committee on Judicial Ethics is identical with the
    Supreme Court.

  \item {\em Therefore} the Supreme Court is one of the committees
    assembled by the
    Senate. \citep[144]{uzquiano2004a} \label{sup-com}
\end{enumerate}

And \ref{sup-com} seems false.

But it may be possible to argue that \ref{sup-com} is not false but
only {\em misleading} (indeed, very misleading).  For it
(conversationally) implies that future sets referred to by `the
Supreme Court' will be identical to future sets referred to by `the
Special Committee'.  And it is {\em this} that is certainly false.

\subsection{Set membership and literal speech}
\label{implicate}
I argued in section \ref{eng-quant} that ordinary uses of `there is'
are often false.  For example, if I say ``there is no beer'', what I
say is almost certainly false---there is beer {\em somewhere}---but
what I mean is that there is no beer in the house.

It is very likely that much of our ordinary talk is similarly
non-literal (see \citet{bach1987}).  For example, we should interpret
``the chair is mine'' as non-literal, because ``the chair is mine''
entails that there is only one chair in the world.  Even propositions
involving proper names might be non-literal.  If `Alex' designates
every person named `Alex', then ``Alex is lying down'' is literally
false, since it entails either that there is only one `Alex' or that
every `Alex' is lying down.

Therefore, if a theory predicts that some of our talk about groups is
non-literal, we should not necessarily be worried.  But not {\em all}
of our talk about groups is non-literal, and when a theory can
preserve the intuition that certain thinks are literally true, that
should be taken as an advantage.  At least for a certain class of
examples, the set-identity thesis preserves more of our intuitive
judgements about literal speech than does the theory that posits
groups as distinct from sets.

\begin{enumerate}
  \item Suppose we arrive at a meeting of the Special Committee on
    Judicial Ethics.  Rehnquist, Stevens, O'Connor, Scalia, Kennedy,
    Souter, Thomas, Ginsburg, and Breyer are sitting around a center
    table.  As we take our seats you turn to me and say, ``they look
    rather familiar, don't they?''  I say ``that's also the Supreme
    Court.''

    What am I referring to with the demonstrative expression ``that''?
    \begin{itemize}
      \item If one thinks that I am referring to a {\em group}---the
        Special Committee---that is distinct from the Supreme Court,
        my utterance will have to be interpreted as non-literal.  I
        will have to be understood to mean that the {\em members} of
        the Special Committee are also the members of the Supreme
        Court.
      \item On the other hand, if I am referring to the {\em set}
        of justices, what I said is literally true.
     \end{itemize}

  \item Suppose instead that you ask me who the members of the Special
    Committee are.  I say ``Rehnquist, Stevens, O'Connor, Scalia,
    Kennedy, Souter, Thomas, Ginsburg, and Breyer.  The Special
    Committee is just the Supreme Court.''  
    \begin{itemize}
      \item Here again one could argue that I am speaking
        non-literally; what I mean is that the members of the Special
        Committee are just the members of the Supreme Court.  
      \item But if the Supreme Court and the Special Committee are
        just sets---the same set---I have again said something
        literally true.
    \end{itemize}

  \item Now suppose that the Special Committee is dissolved in 2004.
    In 2005, we see the members of the Supreme Court (still Rehnquist,
    Stevens, O'Connor, Scalia, Kennedy, Souter, Thomas, Ginsburg, and
    Breyer) out to lunch together.  I point and say ``that was the
    Special Committee on Judicial Ethics.''  Now what is the referent
    of ``that''?  It cannot be the Special Committee, for that has
    ceased to be.  It must either be the Supreme Court or the set
    \{Rehnquist, Stevens, O'Connor, Scalia, Kennedy, Souter, Thomas,
    Ginsburg, and Breyer\}.  
    \begin{itemize}
      \item Either way, the proponent of groups will
    have to interpret this utterance as non-literal.  
      \item The set-identity theorist can interpret this utterance as
        literally true, however; that set was the Special Committee
        before the dissolution.
    \end{itemize}

  \item Now suppose that the Special Committee is dissolved in 2004
    and Rehnquist retired before dying in 2005 (let's pretend he
    retired in May).  Now in August we see Rehnquist, Stevens,
    O'Connor, Scalia, Kennedy, Souter, Thomas, Ginsburg, and Breyer
    out to lunch together.  I point and say ``that was the Supreme
    Court {\em and} the Special Committee on Judicial Ethics.''  I can
    only be referring to the set of justices.  Why not suppose that I
    have only {\em ever} been referring to the set of justices?  If I
    am in fact referring to the set \{Rehnquist, Stevens, O'Connor,
    Scalia, Kennedy, Souter, Thomas, Ginsburg, Breyer\}, then when I
    say ``that was the Supreme Court {\em and} the Special
    Committee'', I say something literally true.
\end{enumerate}

These examples provide some support for the set identity thesis.  At
the very least they show that identifying groups with sets does not
mean that all our talk about groups must be interpreted as
non-literal.  However, the set identity thesis also predicts that some
propositions will be literally true, when intuitively we may believe
that they are not.  For example, according to the set identity thesis,
I say something literally true when I say ``the Supreme Court is one
of the committees assembled by the Senate'' or ``the Supreme Court is
the Special Committee on Judicial Ethics''.  But it is very misleading
to say either.  By saying ``the Supreme Court is the Special
Committee'' I imply that future referents of `the Supreme Court' will
be identical to future referents of `the Special Committee'.  It is
less misleading to say ``the current Supreme Court is the Special
Committee on Judicial Ethics''.  (It is even less misleading to say
``the current Supreme Court is also the Special Committee''.)

One may object that, while this is all well and good, the set identity
thesis fails the most important test.  The set identity thesis
predicts that propositions about the Supreme Court evolution over
time, such as ``the Supreme Court has become more diverse'' are
literally false.

This is a drawback for the set identity theorist, but I do not think
it is a great one.  As a parallel case, take the proposition ``the
temperature is dropping''.  For this to be literally true, there would
have to be some thing---the temperature---that is, in some sense,
dropping.  But it is plausible to interpret a speaker who says ``the
temperature is dropping'' as meaning that soon, the number that is the
referent of `the temperature' will be lower than the number currently
referred to by `the temperature'.  Likewise, when I say ``the Supreme
Court has become more diverse'' I mean that the members of the sets
that have been the Supreme Court have become more diverse.

It may be due to this fact that we so easily misinterpret uttered
propositions like ``the Supreme Court is the Special Committee''.  A
listener might take this to mean that the members of the sets that
have been (and will be) the Supreme Court are identical with the
members of the sets that have been (and will be) the Special
Committee.  They would therefore evaluate the utterance as false.

Another example that the set identity thesis predicts as non-literal
is ``the Supreme Court has become more conservative''.  If we claim
that the Supreme Court is a set, we cannot interpret this utterance as
literally true; {\em sets} do not have political leanings.  Someone
who utters this, according to the set identity thesis, must be taken
to mean that the members of the sets that have been the referents of
``the Supreme Court'' have become more conservative.

But can a philosopher who distinguishes groups from sets interpret
this literally?  Can groups {\em literally} have political leanings?
Or must the speaker be interpreted as meaning that the members of the
group have become more conservative?  I think it is plausible that
``the Supreme Court has become more conservative'' must be interpreted
as non-literal, whether the Supreme Court is a group or a set.

What about an utterance such as ``the Supreme Court ruled against the
defendant''?  If the Supreme Court is a set, this utterance will have
to be interpreted as non-literal.  Sets don't {\em do} things; we will
have to interpret the speaker as meaning that the Supreme Court
justices ruled against the defendant.  But what if the Court was
divided over the ruling?  If several justices wrote dissenting
opinions, it seems that {\em they} didn't rule against the defendant.
Rather, we want to say that the {\em group} ruled against the
defendant.  The proponent of groups may be in a slightly stronger
position here.  But if we identify the Supreme Court with a set, we
can still say that the {\em majority} of the Supreme Court justices
ruled against the defendant.  And that is more or less what we mean
when we say ``the Supreme Court ruled against the defendant''.

Whether or not we identify the Supreme Court and other groups with
sets, we will have to interpret some apparently literal speech as
non-literal.  But the set identity thesis, at least with regard to
this slate of examples, treats talk about groups more consistently
than does the theory that groups are distinct from sets.

\subsection{Conventional identity conditions}
\label{set-convention}
Even supposing everything above is right, there is still more to be
said.  What are the `identity conditions' for the Supreme Court over
time?  Although the Supreme Court is a set, we use `the Supreme Court'
to refer to different sets at different times.  What governs this
shifting reference?  What makes it true that one set is the Supreme
Court in 2004 and a different set is in 2012?

What makes it true that a given set is the Supreme Court at a given
time is simply our legal conventions.  The Constitution authorizes the
recognition of a set of justices as `the Supreme Court'.  Which set is
recognized as the Supreme Court is decided by the legislative and
executive branches.  The president nominates a set (the sitting
justices and the nominated justice) and the legislative branch votes.
The outcome of the vote make it true or false that a given set is the
Supreme Court.

(Now what do we mean when we say ``the Supreme Court was established
in 1789''?  Perhaps that the convention of referring to a set of
justices as `the Supreme Court'---and the granting of legal powers to
them---began in 1789.)

This is analogous for all `groups'.  Teams, bands, militias---what
makes it true that a certain set is a team, band or militia is just
the conventions governing the group.  If I desert my militia and the
other members of the militia recognize my absence as a desertion, then
it is understood that I am no longer part of the militia; for that
reason it is then true that the set containing me is no longer the
militia.  A smaller set, not containing me, is now the militia.

It seems, then, that {\em `identity' conditions over time for groups
  are wholly conventional}.  This is plausible; groups are social
entities, and it makes sense that their composition should be a matter
of social convention.  But if this is true, it suggests something more
radical: that identity conditions over time for physical objects like
chairs are conventional as well.

\section{The conventions of ordinary things}
\label{chair-ref}
Just as we identified groups like the Supreme Court with different
sets at different times, so we can identify things like chairs with
different sums at different times.

This will have the consequence that, at $t_1$, the statue is identical
with the lump of clay.  This will be true, since they are both the set
$S$.  The statue {\em is} the lump of clay.  This seems correct.  If
someone were to ask ``I see the statue, but where did the lump of clay
go?'' we would reply, ``the statue {\em is} the lump''.

If we feel tempted to say that the statue is not identical with the
lump, this is because future (and {\em possible}) utterances of
`statue' may not be used to refer to the same sum as future utterances
of `lump'.  (Thus we can account for their apparent modal differences
as well.)

The identification of statues (and lumps) with sums allows us to
explain some sorts of talk that would be otherwise problematic.  For
example, suppose I make a statue out of a lump of clay.  You come
along and squish the statue (thereby destroying it):

\stage{Alex}{}{That was my statue!}

If we thought that the statue was a distinct thing from the sum (and
from the lump), we would have to interpret what I say non-literally.
For if the statue was a distinct thing that has been destroyed, then
when I use a demonstrative like ``that'' I cannot be referring to the
non-existence statue.  My audience may interpret me as referring to
the lump, and meaning that there used to be a statue co-located with
the lump.  But if we suppose that the statue was not a distinct thing
from the sum (and from the lump), then what I said is literally true.
For ``that''---that sum---was a statue, but is no longer.  It is no
longer a statue because it no longer satisfies our conventional
criteria for what counts as a statue.  (You could dispute these
criteria; after I say ``that was my statue'', you could say ``it still
is your statue; it's just a flatter statue than it was''.)

One objection that may arise here is that, {\em strictly speaking},
the statue still exists.  Worse, if the statue is just the sum, then
the statue existed even before it was sculpted!  How can that be?

Suppose I have a lump of clay on Monday:

\stage{Alex}{}{This will be a statue!}

On Tuesday I make a statue out of the clay:

\stage{Alex}{}{Yesterday this was nothing more than a lump of clay!
  Now look at it!}

On Wednesday you squish the statue:

\stage{Alex}{}{Well, it's not a statue anymore.}

With these examples, I am trying to motivate the idea that we refer to
the sum when we use terms like `it' and `this'.  The sum referred to
on Monday is the same (or nearly the same) sum referred to on Tuesday
and on Wednesday.  When I say ``this will be a statue'', therefore, I
mean that this sum will meet the criteria for being referred to as a
statue.  When I say ``this was nothing more than a lump of clay'', I
mean that this sum previously met only the criteria for being referred
to as a lump of clay.  When I say ``it's not a statue anymore'', I
mean that it no longer meets the criteria for being referred to as a
statue.

But what if I say ``that statue doesn't exist anymore''?  I suggest
that this should be interpreted in almost all cases as meaning that
the sum that is understood to have been the referent of `that statue'
no longer meets the criteria for being the referent of `statue'.  If
``that statue doesn't exist'' is taken in a more literal sense, then
it is false.

Peter van Inwagen has a similar objection to the idea that sums cannot
change their parts (he assumes that ordinary things are sums).
Suppose that sums cannot change change their parts:

\begin{squote}
Call the bricks that were piled in the yard last Tuesday the ``Tuesday
bricks.''  Between last Tuesday and today, the Wise Pig has built a
house---the ``Brick House''---out of the Tuesday bricks (using them
all and using no other materials).  The Brick House did not exist last
Tuesday (that is, it was not then a pile of bricks, a thing that was
not yet a house but would become a house).  The Brick House is not,
therefore, a mereological sum; for if it were, it would have been (it
would have ``existed as'') a pile of bricks last Tuesday
\citeyearpar[616]{inwagen2006}.
\end{squote}

But since the Brick House {\em is} a mereological sum, van Inwagen
concludes that our supposition that sums can't change their parts is
false; he claims that mereological sums {\em can} change their parts.

However, I suggest that, strictly speaking, the Brick House {\em did}
exist last Tuesday.  But last Tuesday it (the sum) did not meet the
criteria for being referred to as a house.  Today I point to the Brick
House and say ``last Tuesday that was just a pile of bricks.  Now it's
a house!''  By `that' I mean the Brick House---the sum---which was a
pile of bricks on Tuesday.  If I say ``the Brick House did not exist
last Tuesday'' I should be taken to mean just that the Brick House did
not meet the criteria for being referred to as `the Brick House' last
Tuesday.

\subsection{Criteria and convention}
\label{criteria}
Like the `identity' of groups over time, the `identity' over time of a
given thing---like my chair---will be conventional.

But this does not mean that {\em I} have the final say over how my
chair persists through time.  The identity over time of the Supreme
Court is conventional, but it is not up to me.  There are entrenched
legal conventions governing the persistence of the Supreme Court.
Likewise, for different things, different conventions will govern
their persistence.  Recall Rosenberg's discussion of the problem with
van Inwagen's Special Composition Question (section \ref{lessons-v}).
He rejected the idea that there is just one way in which `material
objects' are composed:

\begin{squote}
Microphysics explains how protons, neutrons, and electrons compose
different species of atoms, and physical chemistry, how atoms of
various species compose different sorts of molecules
\citep[706]{rosenberg1993}.
\end{squote}

I think it is plausible to claim that atoms, molecules, animals,
chairs, statues and lumps are, in fact, identical with sums.  If this
is correct, then there is really just {\em one} way in which these
things are composed.  But they {\em persist through time} in very
different ways; which sum is identical with a given animal, for
example, depends on the `conventions' of the biological sciences.

\subsection{Essentialism}
\label{essentialism}
This theory is a version of {\em essentialism}.  Essentialism is the
thesis that, strictly speaking, things don't change their parts.  One
can endorse or oppose essentialism in various domains.  For example,
everyone is a set essentialist; nobody (as far as I know) claims that
sets can change their parts.  But not everyone is a {\em mereological}
essentialist.

People who deny mereological essentialism are, I think, making one of
two claims:

\begin{enumerate}
  \item They may be claiming that ordinary things like chairs are not
    mereological sums; chairs can change their parts, so essentialism
    {\em with regard to chairs} is false.  A philosopher who makes
    this claim might allow that mereological sums, if there are such
    things, cannot change their parts.
  \item They may be claiming that mereological sums, whether or not
    they are identical with ordinary things like chairs, can change
    their parts.
\end{enumerate}

I argued against the second claim in section \ref{change}.  But a
philosopher who makes either claim will reject the theory I have been
building.  They will say that my theory flies in the face of common
sense (and I made so much of common sense in earlier sections!).  They
will say things like this:

\begin{squote}
According to [the essentialist], it is never literally correct to say
that a thing survives a change in parts.  This is a point of massive
departure from ordinary belief \citep[184]{sider2001}.
\end{squote}

This is more or less the argument against essentialism.  You point at
a chair and say ``I'm supposed to believe that if that chair loses
{\em one atom}, it's literally a different chair?''

One argument for the claim that the change of a single part results in
a numerically distinct chair comes from Chisholm:

\begin{squote}
Let us picture to ourselves a very simple table, improvised from a
stump and a board.  Now one might have constructed a very similar
table by using the same stump and a different board, or by using the
same board and a different stump.  But the only way of constructing
precisely {\em that} table is to use that particular stump and that
particular board.  It would seem, therefore, that that particular
table is {\em necessarily} made up of that particular stump and that
particular board \citeyearpar[146]{chisholm1976}.
\end{squote}

It may be objected that, {\em once the table is built}, it is possible
to change its parts without thereby destroying one table and
constructing another.  But what is the relevant difference between
building a table for the first time---when the addition of a different
part results in a different table---and adding a new part to an
already existing table?  There does not seem to be one; for all we
know, before Chisholm build his table out of a stump and a board,
there was a table constructed out of the same stump and a different
board.  But since the table that Chisholm build is necessary made up
of its parts, it cannot be identical to a previous table that is made
of different parts.

There is another argument for the same conclusion.  If a chair can
remain numerically distinct after losing a part, it is difficult to
say {\em how large} a part the chair can lose while remaining the
(numerically) same chair.  If most of the chair is blasted away, then
we may very well say that the chair is no more.  But {\em how much}
must be blasted away?  Or suppose we have a portion of gold.  How many
atoms of gold can be stripped off before it is no longer the same
portion?  Thomson claims that ordinary uses of `portion' are
context-dependent:

\begin{squote}
The ordinary use of the term ``portion'' is heavily context-dependent.
If an atom drifts away from your portion of gold, do you still have
the same portion of gold?  You will say no if you are a scientist
engaged in an experiment for which every atom matters. You will say
yes if you are a jeweler about to make a ring.  Similarly, in fact,
for clay.  If you have just bought a load of clay, and a handful falls
off while you are on your way home, is the portion you have when you
get home the same as the portion you bought?  You will say no if you
had carefully measured and bought exactly as much as you need.  You
will say yes if loss of a handful makes no difference to you
\citeyearpar[163]{thomson1998a}.
\end{squote}

When we say that a use of a term is `context-dependent', that can mean
one of two things.  First, it may mean that whether an utterance
involving a use of the term is {\em correct}, or {\em appropriate},
depends on the context.  It would not be appropriate for the scientist
to say that she has the same portion after the loss of several atoms,
because those atoms matter for the experiment.  Second, to say that
the use of a term is `context-dependent' may mean that whether an
utterance involving a use of the term is {\em true} depends on the
context.  In the quoted passage above, do the scientist and jeweler
both say true things?  If they do, then the truth-conditions of
`portion' are context-dependent.  This would mean that whether an
utterance involving `portion' is true depends on the context of the
utterance.  This would also suggest that the {\em meaning} of
`portion' depends on the context, for the truth-value of a sentence is
generally thought to be a function of the meaning of its constituent
elements, including words.

But just as I do not think there are different senses of `there is'
(see section \ref{eng-quant}), so I do not think that there are
multiple senses of `portion'.  I find it far more plausible to think
that only the scientist says something that is, {\em strictly
  speaking}, true.  The jeweler, when she affirms that she has the
same portion of gold, may say something correct or appropriate, given
the context, but it is not {\em true}.  Strictly speaking, a portion
cannot change its parts; why should we assume that a chair can?

But even if ``that's the same chair as yesterday'' is literally false,
the sum that yesterday satisfies the criteria for being the referent
of ``chair'' no longer satisfies those criteria; a different sum that
has many of the same parts now satisfies those criteria, and so
qualifies as `the same chair'.  (How similar the two sums has to be
is, again, a conventional matter.)

Maybe I am wrong and it is true that (strictly speaking) the chair can
lose its parts and yet remain the (numerically) same chair.  If so,
then we must accept the mereological firmament that comes with Fine's
theory.

\section{Am I a mereological sum?}
\label{i-sum}
I have proposed that ordinary things like chairs and statues are
mereological sums.  Their apparent persistence through change is a
result of certain conventions---a chair at $t_1$ is the `same' chair
at $t_2$ if, first, there is a sum at each time that meets our
criteria for being a referent of `chair' and, second, either the sum
that is the referent of `chair' at $t_1$ is the same sum that is the
referent of `chair' at $t_2$, or the sum that was the referent of
`chair' at $t_1$ does not meet the criteria for being the referent
of `chair' at $t_2$ and a different sum does meet these criteria at
$t_2$ and is sufficiently related to the first sum.  What is
`sufficiently related' is again a conventional matter, but will
presumably involve causal and spatiotemporal continuity.

If ordinary things are sums, then are other things sums as well?  I
will suppose that sums are `material things' as opposed to `abstract
things' (whatever that distinction comes to), but are {\em all}
material things sums?  If we are material things, are we therefore
sums?

% all material things are sums

\subsection{All material things are sums}
\label{material-sum}
If we think that ordinary things are sums, and that ordinary things
are material things, I think it is extremely plausible to conclude
that all material things are sums.  For what else would they be?

What is included under the label `material thing'?  I would include
things like chairs, and desks, and desk lamps, and doors, and
doorways, and houses, and gardens, and plants.  I would also include
minuscule objects like molecules and massive objects like planets and
galaxies.  What would these things be, if not sums?

I proposed that ordinary things are sums so as to avoid the conclusion
that there is a plurality of different kinds of ordinary things
(statues and lumps only scratch the surface) all overlapping each
other.  This essentialist proposal was made so as to avoid positing
many different kinds of things.  So anyone who accepts the
essentialist theory should be sympathetic to the idea that all
material things are sums.

I don't have a powerful argument for this conclusion, but I don't see
the {\em point} of supposing that all and only ordinary things are
sums, but other material things are some different kind of object.

\subsection{We are material things}
\label{material-beings}
Even if the idea that all material things are sums is (or should be)
relatively uncontroversial, the idea that {\em we} are material beings
will not be unanimously accepted.  For it does have some unintuitive
consequences.  

First, it rules out identifying us with our mental states.  Suppose
all my psychological characteristics---memory, personality---is
somehow transferred to another body.  The brain in that body is
`wiped' before my psychology is transferred, and after the operation
my old brain is similarly `wiped'.  There is a temptation to say that
I exist in the new body.  But saying this commits us to the claim that
I am not a material thing, because I `left' my old material body and
came to `inhabit' a new one:

\begin{squote}
 If I am identical with the thinking substance in which I am thus
 placed, then I cannot be transferred {\em from} that substance to
 another substance \citep[107]{chisholm1979}.
\end{squote}

Claiming that we are material things entails that psychological
continuity is not a criterion of identity.  The body into which my
psychology is transferred is not me, according to the materialist
claim.  Psychological continuity is often taken to be {\em the}
criterion of identity, so one might take this consequence as a
refutation of the claim that we are material things.

But if we are not material things, what are we?  The only alternative
I see is to claim that we are immaterial minds or souls.  These
positions seem, to me, to be more implausible than the claim that we
are material things.  (Much, of course, can be said in defense of such
a position.)

Claiming that we are material things, however, gives rise to another
question: what material things are we?  Are we identical with our
brains, or with our bodies?

I suggest, though somewhat tentatively, that we are identical with our
bodies.  I agree with Peter van Inwagen on this much:

\begin{squote}
I suppose that [the objects of mental predicates]---Descartes, you,
I---are material objects, in the sense that they are ultimately
composed entirely of quarks and electrons.  They are, moreover, a very
special sort of material object.  They are not brains or cerebral
hemispheres.  They are living animals; being {\em human} animals, they
are things shaped roughly like statues of human beings
\citeyearpar[6]{inwagen1995}.
\end{squote}

Eric Olson has a very plausible argument for the same conclusion:

\begin{enumerate}
  \item There is a human animal sitting in your chair.
  \item The human animal sitting in your chair is thinking. (If you
    like, every human animal sitting there is thinking.)
  \item You are the thinking being sitting in your chair. The one and
    only thinking being sitting in your chair is none other than
    you. Hence, you are that animal \citeyearpar[354]{olson2003a}.
\end{enumerate}

One apparent consequence of the claim that we are material human
animals is that if my brain is removed from my body and put into
another body, that new person is not me.  Claiming that we are
material things required denying that psychological continuity is a
criterion of identity; claiming that we are material human animals
requires denying that even brain continuity is a criterion of
identity.

This may seem to be a troubling consequence, but it is much less
troubling if we accept the essentialist theory.  If material objects
are sums, and if we are material objects, then we are sums.  And if
sums do not, strictly speaking, change their parts over time, then,
like the `identity' conditions for chairs and other ordinary things,
the `identity' conditions over time for {\em us} is conventional.

Another difficulty with identifying us with human animals disappears
if we accept an essentialist theory.  Dean Zimmerman has objected to
Olson's argument by claiming that `human animal' can be replaced with
`human body' without making the argument invalid
\citeyearpar[24]{zimmerman2008a}.  The problem, however, is that it
seems true that we cease to exist when we die.  So Zimmerman concludes
that we are not bodies or animals.

If we accept an essentialist theory, however, the problem disappears.
If, strictly speaking, I can't change my parts over time, then I am
not (strictly speaking) the same person that will be the referent of
`Alex' a month from now (or even a week).  I will certainly not be
identical with a dead body further down the road.

\subsection{How do I `persist' over time?}
\label{person-persist}
The idea that, strictly speaking, I don't change my parts over time
seems crazy.  And maybe it is.  But I don't think it is obviously
false.

Someone who thinks that I do, strictly speaking, persist over time
might say that it is obvious that I persist.  After all, I engage in
activities that take long periods of time, I remember things from long
ago, and I bear unique attitudes toward my past and future selves.  I
feel pride or regret at past actions, and anticipation or apprehension
at future ones.  How could these past and future selves not be me?

One reply begins by pointing out that, whether or not we persist in a
strict sense, the world will look the same.  I will still engage in
activities that take time; but it will not be I who completes them.  I
will still remember things from long ago; but it will not be I who
experienced them.  I will bear attitudes towards past and future
people, but those people will not, strictly speaking, be me.  But it
will {\em seem} as if they are me, and they will meet the conventional
criteria for being the referent of `me'.  As in the case of tables and
chairs, there are conventional `identity' conditions for people over
time.  Like tables and chairs, these criteria will involve causal and
spatiotemporal continuity.  What person is the referent of `Alex' a
week from now will depend on a causal chain connected to me.

Psychological continuity may also play a role.  For example, if by
some miracle I am vaporized and (coincidentally) an qualitatively
identical person is summoned into existence nearby, that person will
not, strictly speaking be me.  But it may be agreed that the person
meets the criteria for being the referent of `Alex'.  Then again, it
may not.  If this person meets these criteria, however, it will be on
account of the apparent psychological continuity between us.

The criteria for the `identity' over time of people is not fully
precise, as shown by our indecision over whether a spontaneous
duplicate of me ought to be referred to as `Alex'.  Another, more
realistic, situation in which this indecision manifests itself is in
death.  Suppose I die, and a wake is held for my body.  It is
perfectly correct for someone to point and say, ``that was Alex''.
But it is equally correct to say ``that's Alex''.  (The latter may be
more appropriate if it is necessary to identify my body.)  Is the
mereological sum that is the (deceased) body really me, or not?  If we
accept the essentialist theory, it is (strictly speaking) not, but it
may be correct or appropriate to refer to the body as `Alex'.

\section{Can the essentialist theory explain what we believe?}
\label{explain-e}
In section \ref{explain-p} I assessed whether any of the three
`plurality' theories could explain why we held beliefs that conflicted
with certain consequences of the theories.  The same assessment may
be conducted with regard to the essentialist theory I have sketched in
this section.  If the essentialist thesis is right, why do we believe
that chairs are literally identical over time?  If we want to defend
the essentialist theory, we should try to explain why we generally
seem to think that things like chairs literally persist over time, and
can change their parts over time as well.

One reply is simply to claim that we {\em don't} believe that things
literally persist over time.  When asked ``is it {\em literally} the
same chair without its leg?'' some of us may waver, and perhaps
concede that we don't think it is really the same chair.  But I doubt
this reply will convince any philosopher who has already made up her
mind about essentialism.

Another reply\,\ldots

\section{Lessons}
\label{lessons-e}
In section \ref{parts} I examined three different versions of the
`plurality thesis'; the view that there are pluralities of co-located
objects.  In this section I offered an alternative.  I am not sure
whether my theory or one of the plurality theses is correct, but I
suspect that it is one or the other.  My conclusion is largely the
same as that of Karen Bennett:

\begin{squote}
The only live options, then, are to be either a one-thinger or a
bazillion-thinger.  We must either think that there is only one thing per
spatio-temporal location, or else that there are lots and \emph{lots} of
spatio-temporally coincident things \citeyearpar[358]{bennett2004}.
\end{squote}

I would prefer to be a `one-thinger' because it does not commit me to
a `bazillion' things all in the same place.  That is not a decisive
objection, of course.  It may well be that such an explosion is more
plausible than certain consequences of the `one-thinger' theory.  But
I think one of the two theories must be right.

\subsection{Do we need Fine's theory at all?}
\label{need-fine}
I have argued that we can identify ordinary things like chairs as
mereological sums, and we can identify things like groups as sets.  It
is therefore not necessary to use Fine's theory of parthood to
describe these things.  Do we need Fine's theory at all?

I think the theory is still useful.  For it shows that sums and sets
both have parts, but in different ways.  There are also sequences,
strings, and other things produced by various composition operators.
And there are other things that exist, like words, poems, events, and
quantities, that have parts in other ways.  Fine's theory may help us
define composition operators for these things, if they cannot be
described in terms of sets and sums (which I suspect they cannot be).

Fine's theory also suggests a novel way of describing temporal parts.
He claims that not all operators are compositional; some are {\em
  decompositional}.  He introduces the segmentation operation that
takes partless simples and produces their `parts'---the top and
bottom, left and right half, etc.  These `parts' are derived from the
wholes, rather than the wholes being built up from the parts:

\begin{squote}
Let us suppose that the universe consists of physical atoms which are
physically indivisible but of finite volume.  We might then
distinguish between the upper and lower parts of the atom (relative to
its orientation at a given time); and it is plausible that the atoms
are to be taken as givens, there being no explanation of their
identity in more basic terms, while the identity of the upper and
lower parts of an atom is to be explained in terms of their {\em
  being} the upper and lower parts of the atom.  Thus the account of
the part is in terms of the whole rather than the other way around
\citep[585]{fine2010}.
\end{squote}

Likewise, if we maintain that objects are not {\em composed} of
temporal parts, but are `temporally indivisible', we can nonetheless
have a temporal-segmentation operator that generates temporal `parts'
from temporally `simple' wholes.  These parts would have theoretical
utility; we could quantify over them without committing ourselves to
their existence (in a `basic' sense).  This sort of derived temporal
part might allow for a form of four-dimensionalism (the conjunction of
universalism and the doctrine of temporal parts) that avoids the
`spinning disc' problem, among others.

So whether or not we accept the ``vast mereological firmament'' that
Fine envisions, his theory of part is useful.

\subsection{Deflationary metaphysics}
\label{deflate}
Kathrin Koslicki has an interesting objection to universalist theses
such as the one I appear committed to.  Her objection amounts to this:
if every `collection' of objects (such as the London Bridge, a
particle in the moon, and Cal Ripkin, Jr.) is a thing in its own right
(a sum), then metaphysics becomes uninteresting.  There is no longer
any debate about whether chairs or dogbushes are more `real' or have a
stronger claim to existence.  They both (obviously) exist, and the
difference between chairs and lumpkins is not ontological but
conceptual: `chair' is more embedded in our talk, and so chairs have
greater importance to {\em us}.  But metaphysically, or ontologically,
chairs and dogbushes are on the same level.  There is no sense in
which chairs exist and lumpkins do not.

In the quoted material below, Koslicki is criticizing a version of
four-dimensionalism that Sider has previously defended.  Sider's
position was that any collection of objects-at-times is a thing in its
own right.  Sider calls these things `fusions'.  For example, a chair
is a fusion of a large number of {\em temporal part} of things (wood
molecules, or atoms, or simples).  Each thing (wood molecule, atom, or
simple) is a fusion of {\em its} temporal parts.  Each temporal part
of the chair is also a thing (a fusion).

I take no stance on whether objects have temporal parts or rather
`endure' through time.  Moreover, if we accept Fine's theory of parts
then we reject the idea that there is just one composition operation;
the operation that produces fusions is one among many.  But Koslicki's
comments are relevant nonetheless:

\begin{squote}
There is room, in Sider's theory, for {\em some} genuine ontological
disagreements: for example, the universalist, the nihilist and the
holder of the intermediary position genuinely disagree over how many
and which fusions that exist.  But the only genuine ontological
disagreements for which there is room, in Sider's world, are ones that
concern disagreements over `bare' fusions, so to speak.  What has
happened to the houses, trees, people, and cars, the familiar concrete
objects of common-sense, whose persistence this account set out to
analyze?  There are no `deep' ontological facts as to whether a given
fusion should count as a house or not\,\ldots

[By claiming that there can be genuine ontological disputes,] Sider is
guilty of a bit of false advertising: his account is really a way of
saying that, at the end of the day, there is no interesting {\em
  ontological} story to be told about the persistence of our familiar
concrete objects of common-sense; whatever there is to say about the
persistence of houses, trees, people and cars concerns the
organization of our conceptual household
\citeyearpar[124--125]{koslicki2003}.
\end{squote}

Koslicki seems to think that we ought to be able to find some
ontological difference between ``the familiar concrete objects of
common-sense'' and things like lumpkins or chairs-at-times.  But as I
remarked above (section \ref{universalism}), why should what interests
us (familiar objects like chairs) be a guide to what exists?  The
conclusion that ``the persistence [and other properties] of houses,
trees, people and cars concerns the organization of our conceptual
household'' seems to be correct.

However, there {\em is} an ontological difference between some things,
if not between chairs and dogbushes.  One lesson of Kit Fine's theory
of parts is that mereological sums are not the only kind of composite
thing.  There are sets as well, and strings, and sequences, and
perhaps infinitely many other types of thing.  The difference between
a set and a sum is an ontological difference.  Within each type,
however, we must rely on our own conceptual `scheme' to organize
things.

In section \ref{lessons-v} I considered Jay Rosenberg's claim that
the Special Composition Question is the wrong question to be asking.
Rosenberg's position seems to be that there is {\em no} answer to the
Special Composition Question.  Rather, he thinks what it takes to
`compose' something depends on what that something is---making a chair
is not like making a pie.

This insight of Rosenberg's can be connected with the insights of
Fine's theory.  If we understand `composition' in the Special
Composition Question to mean {\em mereological composition}, then
Rosenberg was wrong if he held that there is no correct answer to the
Special Composition Question.  It seems intuitively true that
mereological composition is unrestricted.  But if take `composition'
in the Special Composition Question to be $K$-composition---any
composition operator at all---then Rosenberg was {\em right} that
there is no answer.  What the application conditions are for a
composition operator depends on {\em which} composition operation is
being applied.  

Moreover, determining the application conditions for these composition
operators, and determining their identity conditions, and determining
the other properties of these various operators is a task for
metaphysics.  The field is not then so barren as Koslicki seems to
have feared.  But it is true that many interesting questions---When
are we willing to call something a chair, and why?  What conditions
must be fulfilled?---are not ontological questions anymore.  They are
questions about our ``conceptual household.''

\ifstandalone
\end{spacing}
\bibliography{everything}
\bibliographystyle{ChicagoReedweb}
\fi
\end{document}
