\documentclass[11pt]{article}
\usepackage{standalone} \newif\ifstandlone \standalonetrue
\usepackage[left=1.75in, right=1.75in, top=1.25in, bottom=1.25in]{geometry}
\geometry{letterpaper}
\usepackage{graphicx}
\usepackage{enumitem}
\usepackage{amssymb}
\usepackage{amsmath}
\usepackage{epstopdf}
\usepackage{verbatim}
\usepackage{setspace}
\usepackage{natbib}
\setcitestyle{aysep={}}
\usepackage{hyperref}
\usepackage{url}
\synctex=1

\DeclareSymbolFont{symbolsC}{U}{txsyc}{m}{n}
\DeclareMathSymbol{\strictif}{\mathrel}{symbolsC}{74}
\DeclareMathSymbol{\boxright}{\mathrel}{symbolsC}{128}

\newenvironment{squote}{%
\begin{spacing}{1}
\begin{list}{}{%
\setlength{\labelwidth}{0pt}%
\rightmargin\leftmargin%
}
\item\relax
}{%
\end{list}%
\end{spacing}
}

\title{Essentialism}
\author{Alexander A. Dunn}
\begin{document}
\ifstandalone
\maketitle
\begin{spacing}{1.5}
\fi

\label{essential}

In Section \ref{parts} I presented three different theories that
modified classical mereology.  These modification were made to explain
how objects change their parts over time.  But each of these theories
require us to posit an extraordinary plurality of (if only
temporarily) co-located objects.  Such theories are, if not false, at
least very strange.

In this section, therefore, I will attempt to sketch an {\em
  essentialist} theory of things.  This theory will allow us to reject
the ``plurality thesis''---that there are pluralities of co-located
objects---but it will have problems of its own.  The most glaring is
the consequence that, strictly speaking, things don't change their
parts.

In Section \ref{groups} I claimed that three theories---Fine's theory
of embodiments, his theory of composition operators, and Hovda's
theory of tensed mereology---could each explain the existence of {\em
  groups} like the Supreme Court.  These theories are able to do so
because they each have a way of accounting for how things, including
groups, change their parts over time.  In this section, however, I am
suggesting that perhaps things cannot change their parts.  Must I
therefore deny the existence of groups?

Instead of denying that there are groups, instead I will argue that it
is possible to identify groups with {\em sets}; strictly speaking,
therefore, groups cannot change their parts.

\section{Re-examining the set identity thesis}
\label{set-id}
The primary motivation cited in Section \ref{group-set} for positing
groups was the fact that the Supreme Court appears to change its
members over time.  For example, both of the following sentences seem
to be true:

\begin{enumerate}[ref=(\arabic*)]
  \item The Supreme Court ruled on Roe vs.\ Wade in 1973. \label{roe1}

  \item The set of justices now serving as Supreme Court Justices did
    not rule on Roe vs.\ Wade in 1973
    \citep[135]{uzquiano2004a}. \label{roe2}
\end{enumerate}

One way to accommodate these facts is to ``insist that the Supreme
Court is a set, but to abandon the assumption that there is a single
set to which the phrase `the Supreme Court' refers in sentences
\ref{roe1} and \ref{roe2}'' \citep[138]{uzquiano2004a}.  To
successfully use `the Supreme Court' to refer to a set of justices,
there must be an implicit or explicit temporal reference.  If an
utterance of \ref{roe1} is true it will be true because it the speaker
intends her audience to recognize her intention to refer to the set of
justices that was the Supreme Court in 1973.  If her audience, for
whatever reason, takes her to be referring to the current Court, then
they will evaluate \ref{roe1} as false.

Considered in this light, `the Supreme Court' is used to express a
relation between sets and times; `$x$ is the Supreme Court at $t$'
\citep[140]{uzquiano2004a}.  There is some precedent for this sort of
interpretation:

\begin{squote}
Our use of the phrase `the Supreme Court' to express a relation a set
of justices bears to a time is much like our use of the phrase `the
president of the United States' to express a relation an individual
bears to a time.  Different persons may be the president of the United
States at different times, but there is at most one person that bears
that relation to each time \citep[138]{uzquiano2004a}.
\end{squote}

``But,'' it will be objected, ``there is an important difference here.
We use both phrases---`the Supreme Court' and `the president'---to
refer to a past, present or future set that `is' the thing, but we
also use `the Supreme Court' to refer to {\em the Supreme Court},
which has changed its membership over time.  If I say, `The Supreme
Court has become more conservative over the past century', there is no
one set I am referring to.  I must be referring to something else; the
obvious candidate is the {\em group} that is the Court.''

One reply here is to claim that all that what ``The Supreme Court has
become more conservative over the past century'' actually means is
that the members of the sets that have been the Supreme Court have
become more conservative.  Another, similar reply is that someone who
utters ``The Supreme Court has become more conservative over the past
century'' is saying something literally false (either because there is
no unique set that is being referred to, or because there is a unique
set referred to, but one that does not make the proposition true), but
can generally be understood to mean something else; namely, that the
members of the sets that have been the Supreme Court have become more
conservative.

Neither reply is {\em very} unintuitive; indeed, there is something
attractive about a thesis that reserves application of adjectives like
`conservative' for people, rather than other things like groups.

But there is a more pressing worry for the set identity thesis.
Recall that the set that is the Supreme Court at a given time might
also be the Special Committee on Judicial Ethics.  We must admit that
the Supreme Court in 2004 is the set \{Rehnquist, Stevens, O'Connor,
Scalia, Kennedy, Souter, Thomas, Ginsburg, Breyer\}, and the Special
Committee in 2004 is that very same set.  But now we are committed to
this argument:

\begin{enumerate}[ref=(\arabic*)]
  \item The Special Committee on Judicial Ethics is one of the
    committees assembled by the Senate.

  \item The Special Committee on Judicial Ethics is identical with the
    Supreme Court.

  \item {\em Therefore} the Supreme Court is one of the committees
    assembled by the
    Senate. \citep[144]{uzquiano2004a} \label{sup-com}
\end{enumerate}

And \ref{sup-com} seems false.

But it may be possible to argue that \ref{sup-com} is not false but
only {\em misleading} (indeed, very misleading).  For it
(conversationally) implies that future sets referred to by `the
Supreme Court' will be identical to future sets referred to by `the
Special Committee'.  And it is {\em this} that is certainly false.

This possibility raises another: that ordinary things like chairs are
identical with {\em sums} in the classical sense.  That is, just as it
may be that, strictly speaking, the Supreme Court cannot change its
parts (its members), so a chair cannot, strictly speaking, change its
parts.  Just as we identified groups like the Supreme Court with
different sets at different times, so we can identify things like
chairs with different sums at different times.

This is a bizarre possibility, and one apparently at odds with common
sense.  Isn't it {\em obvious} that things change their parts?  They
certainly seem to, and it may be argued that much of our talk
presupposes this.  But not all of our talk does, and some stretches of
discourse can actually be {\em better} interpreted on the assumption
that groups are sets, or that ordinary things are sums.

\section{Sets, sums, and literal speech}
\label{talk}
I argued in Section \ref{eng-quant} that ordinary uses of `there is'
are often false.  For example, if I say ``There is no beer'', what I
say is almost certainly false---there is beer {\em somewhere}---but
what I mean is that there is no beer in the house.

It is very likely that much of our ordinary talk is similarly
non-literal (see \citet{bach1987}).  For example, we should interpret
all uses of `The chair is mine' as non-literal, because saying ``The
chair is mine'' entails that there is only one chair in the world.
Even propositions involving proper names might be non-literal.  If
`Alex' designates every person named `Alex', then ``Alex is lying
down'' is literally false, since it entails either that there is only
one `Alex' or that every `Alex' is lying down.

Therefore, if a theory predicts that some of our talk is non-literal,
we should not necessarily be worried.  But not {\em all} of our talk
is non-literal, and when a theory can preserve the intuition that
certain things are literally true, that should be taken as an
advantage.

\subsection{Talking about sets}
\label{sets-talk}
At least for a certain class of examples, the set-identity thesis
preserves more of our intuitive judgments about literal speech than
does the theory that posits groups as distinct from sets.

\begin{enumerate}
  \item Suppose we arrive at a meeting of the Special Committee on
    Judicial Ethics.  Rehnquist, Stevens, O'Connor, Scalia, Kennedy,
    Souter, Thomas, Ginsburg, and Breyer are sitting around a center
    table.  As we take our seats you turn to me and say, ``They look
    rather familiar, don't they?''  I say ``That's also the Supreme
    Court.''

    What am I referring to with the demonstrative expression `that'?
    If one thinks that I am referring to a {\em group}---the Special
    Committee---that is distinct from the Supreme Court, my utterance
    will have to be interpreted as non-literal.  I will have to be
    understood to mean that the {\em members} of the Special Committee
    are also the members of the Supreme Court.  On the other hand, if
    I am referring to the {\em set} of justices, what I said is
    literally true.

  \item Suppose instead that you ask me who the members of the Special
    Committee are.  I say ``Rehnquist, Stevens, O'Connor, Scalia,
    Kennedy, Souter, Thomas, Ginsburg, and Breyer.  The Special
    Committee is just the Supreme Court.''  

    Here again one could argue that I am speaking non-literally; what
    I mean is that the members of the Special Committee are just the
    members of the Supreme Court.  But if the Supreme Court and the
    Special Committee are just sets---the same set---I have again said
    something literally true.

  \item Now suppose that the Special Committee is dissolved in 2004.
    In 2005, we see the members of the Supreme Court (still Rehnquist,
    Stevens, O'Connor, Scalia, Kennedy, Souter, Thomas, Ginsburg, and
    Breyer) out to lunch together.  I point and say ``That was the
    Special Committee on Judicial Ethics.''  Now what is `that' used
    to refer to?  It cannot be the Special Committee, for that has
    ceased to be.  It must either be the Supreme Court or the set
    \{Rehnquist, Stevens, O'Connor, Scalia, Kennedy, Souter, Thomas,
    Ginsburg, and Breyer\}.

    Either way, the proponent of groups will have to interpret this
    utterance as non-literal.  The set-identity theorist can interpret
    this utterance as literally true, however; the set in question was
    the Special Committee before the dissolution.

  \item Now suppose that the Special Committee is dissolved in 2004
    and Rehnquist retired before dying in 2005 (let's pretend he
    retired in May).  Now in August we see Rehnquist, Stevens,
    O'Connor, Scalia, Kennedy, Souter, Thomas, Ginsburg, and Breyer
    out to lunch together.  I point and say ``That was the Supreme
    Court {\em and} the Special Committee on Judicial Ethics.''  I can
    only be referring to the set of justices.  Why not suppose that I
    have only {\em ever} been referring to the set of justices?  If I
    am in fact referring to the set \{Rehnquist, Stevens, O'Connor,
    Scalia, Kennedy, Souter, Thomas, Ginsburg, Breyer\}, then when I
    say ``That was the Supreme Court {\em and} the Special
    Committee'', I say something literally true.
\end{enumerate}

These examples provide some support for the set identity thesis.  At
the very least they show that identifying groups with sets does not
mean that all our talk about groups must be interpreted as
non-literal.  However, the set identity thesis also predicts that some
propositions will be literally true, when intuitively we may believe
that they are not.  For example, according to the set identity thesis,
I say something literally true when I say ``The Supreme Court is one
of the committees assembled by the Senate'' or ``The Supreme Court is
the Special Committee on Judicial Ethics''.  But it is very misleading
to say either.  By saying ``The Supreme Court is the Special
Committee'' I imply that future things designated by `the Supreme
Court' will be identical to future things designated by `the Special
Committee'.  It is less misleading to say ``The current Supreme Court
is the Special Committee on Judicial Ethics''.  (It is even less
misleading to say ``The current Supreme Court is also the Special
Committee''.)

\subsection{Talking about sums}
\label{sums-talk}
The identification of statues (and lumps) with sums allows us to again
explain some sorts of talk that would be otherwise problematic:

\begin{enumerate}
  \item Suppose you help me carry a lump of clay into my workshop.  In
    the afternoon you drop by and see that I am sculpting a statue.
    You say, ``That looks familiar''.  I reply, ``It's the lump of
    clay from this morning''.

    What are you referring to with `that', and what am I referring to
    with `it'?  If we thought that the statue and the lump were two
    distinct things, we might assume that we are both referring to the
    statue.  We would therefore have to interpret what I say as
    somehow non-literal.  But if we are both referring to a single
    sum, then what I say is literally true.

  \item Suppose you're a little dense.  You see the statue and ask,
    ``What a great big statue!  Where did it come from, and where did
    the lump of clay disappear to?''  I reply, ``The statue {\em is}
    the lump''.

    If the statue and the lump are distinct things and my use of `is'
    is that of identity, then what I say is false.  It must be
    interpreted non-literally, as meaning that the statue is composed
    of the same matter as the lump.  But if the statue and the lump
    are the very same sum, then what I say is literally true.
  \item Suppose you come along and squish the statue, thereby
    destroying it.  I cry, ``That was my statue!''

    If we are imagining that the statue was a distinct thing from the
    sum (and from the lump), we would have to interpret what I say
    non-literally.  For if the statue was a distinct thing that has
    been destroyed, then when I use a demonstrative like `that' I
    cannot be referring to the non-existent statue.  My audience may
    interpret me as referring to the lump, and meaning that there used
    to be a statue co-located with the lump.  But if we suppose that
    the statue was not a distinct thing from the sum (and from the
    lump), then what I said is literally true.  For ``that''---that
    sum---was a statue, but is no longer.  It no longer satisfies the
    criteria for being a statue.  (You could dispute this; after I say
    ``That was my statue'', you could say ``It still is your statue;
    it's just a flatter statue than it was''.)  Likewise, suppose I
    have a lump of clay on Monday:

    \stage{Alex}{}{This will be a statue!}

    On Tuesday I make a statue out of the clay:

    \stage{Alex}{}{Yesterday this was nothing more than a lump of
      clay!  Now look at it!}

    On Wednesday you squish the statue:

    \stage{Alex}{}{Well, it's not a statue anymore.}
\end{enumerate}

With these examples, I am trying to motivate the idea that we refer to
the sum when we use words like `it' and `this' and `that'.  The sum
referred to on Monday is the same (or nearly the same) sum referred to
on Tuesday and on Wednesday.

This idea is a part of the thesis of essentialism---the thesis that
things do not change their parts.  This is a controversial thesis, and
I will address the objections to it in Section \ref{essentialism}.
The primary objection is that it is simply {\em wrong} to claim that
most of our cross-temporal talk---for example, ``That is the same
chair as yesterday''---is literally false.

\section{Problems with essentialism}
\label{essentialism}
Essentialism is the thesis that, strictly speaking, things don't
change their parts.  One can endorse or oppose essentialism in various
domains.  For example, almost everyone is a set essentialist; I can
think of nobody who claims that sets can change their parts.  But not
everyone is a {\em mereological} essentialist.

People who deny mereological essentialism are, I think, making one of
two claims:

\begin{enumerate}
  \item They may be claiming that ordinary things like chairs are not
    mereological sums; chairs can change their parts, so essentialism
    {\em with regard to chairs} is false.  A philosopher who makes
    this claim might allow that mereological sums, if there are such
    things, cannot change their parts.
  \item They may be claiming that mereological sums, whether or not
    they are identical with ordinary things like chairs, can change
    their parts.
\end{enumerate}

Fine appears at least sympathetic to the first claim (see Sections
\ref{fine-h}--\ref{fine-c}); van Inwagen and Hovda argue for the
second (see Sections \ref{change} and \ref{hovda}).  But a philosopher
who makes either claim will reject the theory I have been building.
They will say that my theory flies in the face of common sense (and I
make so much of common sense in earlier sections!).  They will say
things like this:

\begin{squote}
According to [the essentialist], it is never literally correct to say
that a thing survives a change in parts.  This is a point of massive
departure from ordinary belief \citep[184]{sider2001}.
\end{squote}

This is more or less the argument against essentialism.  You point at
a chair and say ``I'm supposed to believe that if that chair loses
{\em one atom}, it's literally a different chair?''

It is interesting to note that in the past, many philosophers were
more than willing to affirm this.  Roderick Chisholm points out that

\begin{squote}
Abelard held that ``no thing has more or less parts at one time than
at another''\,\ldots [and] Leibniz said ``we cannot say, speaking
according to the great truth of things, that the same whole is
preserved when a part is lost'' \citeyearpar[145]{chisholm1979}.
\end{squote}

Joseph Butler, writing in 1736, also held that ``when a man swears to
the same tree, as having stood fifty years in the same place, he means
only the same as to all the purposes of property and uses of common
life, and not that the tree has been all that time the same in the
strict philosophical sense of the word''
\citeyearpar[100]{butler1975a}.

One might object that these philosophers were simply failing to
distinguish {\em descriptive} and {\em numerical} sameness.  When I
say I have the same guitar as you, all I mean is that it is
descriptively the same, not that I have your guitar.  Likewise perhaps
Abelard, Leibniz, and Butler observed that a sapling is descriptively
different from the mature tree that it grows into, and then drew the
unwarranted conclusion that the sapling and mature tree are not
therefore the same.

This seems a bit uncharitable, but in any case there are arguments
supporting the same conclusion---that things cannot change their
parts.  The first comes from Chisholm:

\begin{squote}
Let us picture to ourselves a very simple table, improvised from a
stump and a board.  Now one might have constructed a very similar
table by using the same stump and a different board, or by using the
same board and a different stump.  But the only way of constructing
precisely {\em that} table is to use that particular stump and that
particular board.  It would seem, therefore, that that particular
table is {\em necessarily} made up of that particular stump and that
particular board \citeyearpar[146]{chisholm1979}.
\end{squote}

It may be objected that, {\em once the table is built}, it is possible
to change its parts without thereby destroying one table and
constructing another.  Once I have built a table, it seems true that I
could take it apart and reassemble the very same table.  It even seems
that I could take it apart and reassemble the very same table with a
slight modification; for example, I could put it back together with
one new leg.  It may be, then, that all Chisholm's argument shows is
that this particular table necessarily {\em began} its existence with
a particular stump and board.  But there is nothing in the argument
that shows that it necessarily cannot go on to change its parts while
remaining numerically identical.

But if a chair can remain numerically identical after changing a part,
it is difficult to say {\em how large} a part the chair can lose while
remaining the (numerically) same chair.  If most of the chair is
blasted away, then we may very well say that the chair is no more.
But {\em how much} must be blasted away?  Or suppose we have a portion
of gold.  How many atoms of gold can be stripped off before it is no
longer the same portion?  Thomson claims that ordinary uses of
`portion' are context-dependent:

\begin{squote}
The ordinary use of the term `portion' is heavily context-dependent.
If an atom drifts away from your portion of gold, do you still have
the same portion of gold?  You will say no if you are a scientist
engaged in an experiment for which every atom matters. You will say
yes if you are a jeweler about to make a ring.  Similarly, in fact,
for clay.  If you have just bought a load of clay, and a handful falls
off while you are on your way home, is the portion you have when you
get home the same as the portion you bought?  You will say no if you
had carefully measured and bought exactly as much as you need.  You
will say yes if loss of a handful makes no difference to you
\citeyearpar[163]{thomson1998a}.
\end{squote}

When we say that a use of a term is context-dependent, that can mean
one of two things.  First, it may mean that whether an utterance
involving a use of the term is {\em correct}, or {\em appropriate},
depends on the context.  It would not be appropriate for the scientist
to say that she has the same portion after the loss of several atoms,
because those atoms matter for the experiment.  Second, to say that
the use of a term is context-dependent may mean that whether an
utterance involving a use of the term is {\em true} depends on the
context.  In the quoted passage above, do the scientist and jeweler
both say true things?  If they do, then the truth-conditions of
`portion' are context-dependent.  This would mean that whether an
utterance involving `portion' is true depends on the context of the
utterance.  This would also suggest that the {\em meaning} of
`portion' depends on the context, for the truth-value of a sentence is
generally thought to be a function of the meaning of its constituent
elements, including words.

But just as I do not think there are different senses of `there is'
(see Section \ref{eng-quant}), so I do not think that there are
multiple senses of `portion'.  I find it far more plausible to think
that only the scientist says something that is, {\em strictly
  speaking}, true.  The jeweler, when she affirms that she has the
same portion of gold, may say something correct or appropriate, given
the context, but it is not {\em true}.  Strictly speaking, a portion
cannot change its parts; why should we assume that a chair can?

\section{Talk over time}
\label{time-talk}
One may object that, while this is all well and good, the essentialist
thesis---that groups are sets and ordinary things are sums---fails the
most important test.  The thesis seems to predict that propositions
about change over time, such as `The Supreme Court was formed in 1789'
or `The Brick House did not exist last Tuesday' are almost always
literally false.

In the case of sets, what is going on when I say something like ``The
Supreme Court was formed in 1789''?  If `the Supreme Court' designates
the set of current justices, then such a claim is false.  But there is
obviously {\em something} right about what I say.  The set identity
thesis must be supplemented with an explanation of what is right about
`The Supreme Court was formed in 1789' and what is wrong about `The
Supreme Court was formed in 1200'.  (Compare this to van Inwagen and
Merricks' attempts---discussed in Section \ref{stroud}---to explain
what is right about utterances like ``There is a chair in the
kitchen''.)

In the case of sums, how are we to understand the proposition `The
Brick House did not exist last Tuesday'?  This example is part of an
argument by Peter van Inwagen aiming to show that sums can change
their parts.  Suppose, first, that sums cannot change change their
parts:

\begin{squote}
Call the bricks that were piled in the yard last Tuesday the `Tuesday
bricks'.  Between last Tuesday and today, the Wise Pig has built a
house---the `Brick House'---out of the Tuesday bricks (using them all
and using no other materials).  The Brick House did not exist last
Tuesday (that is, it was not then a pile of bricks, a thing that was
not yet a house but would become a house).  The Brick House is not,
therefore, a mereological sum; for if it were, it would have been (it
would have ``existed as'') a pile of bricks last Tuesday
\citeyearpar[616]{inwagen2006}.
\end{squote}

But since the Brick House {\em is} a mereological sum, van Inwagen
concludes that our supposition that sums can't change their parts is
false; he claims that mereological sums {\em can} change their parts.
If we are to maintain both that the Brick House is a sum {\em and}
that sums can't change their parts, we must say that (strictly
speaking) the Brick House {\em did} exist last Tuesday, despite the
fact that it had not yet been built. \\

Below I will look at two different ways of making sense of
cross-temporal utterances about sets and sums.  The first is an
adaptation of Ted Sider's temporal counterpart theory; it maintains
that even though `the Supreme Court' designates the current set of
justices, ``The Supreme Court was formed in 1789'' is literally true.
The second is an adaptation of Roderick Chisholm's notion of an ``ens
successivum''; this theory claims that the Supreme Court and the Brick
House are ``fictions'' that are constituted by different things at
different times; cross-temporal talk is generally false, but can be
correct or accurate.  I will suggest that Chisholm's theory is
superior to Sider's.

\subsection{Ordinary speech and temporal counterpart theory}
\label{counterpart}
According to the essentialist thesis, it is literally false to say
``The Supreme Court was formed in 1789''.  Likewise it is false to say
``The Brick House did not exist last Tuesday''.  The thing designated
by `the Supreme Court' is a set; sets exist when their members exist,
and so the set in question did not even exist in 1789.  (Even if the
set {\em did} exist then, the only way the utterance would be true
would be if the last member of the set was born in 1789; then the set
would come into existence in 1789.  This {\em might} make it true that
the set was ``formed'' in 1789.)  The thing designated by `the Brick
House' is a sum; sums exist whenever their parts exist, and the parts
of the Brick House existed last Tuesday.

These are obviously unintuitive conclusions.  Ted Sider's temporal
counterpart theory offers a possible way to avoid them.

Sider's counterpart theory is part of the theory of
four-dimensionalism he once promoted \citeyearpar{sider2001}.  Unlike
most four-dimensionalists who claim that we use terms like `chair' to
refer to ``spacetime worms'' or ``aggregates of chair-stages'', Sider
argued that we use such terms to refer to instantaneous stages, not
``continuant'' worms or aggregates.  What this means is that in
ordinary talk we never refer to the same thing twice; the chair I
refer to at $t_1$ is one temporal part (chair-at-$t_1$) and the chair
I refer to at $t_2$ is another.  When I use `Ted' to refer to Ted
Sider, I am not referring to the temporally extended object that
includes a childhood; I am referring to something that lasts only for
an instant.

Nonetheless Sider claims that when I say ``Ted was once a boy'', I say
something literally true.  How can this be?  The object I am referring
to was never a boy.  It is here that Sider introduces temporal
counterparts:

\begin{squote}
According to my temporal counterpart theory, the truth condition of an
utterance like ``Ted was once a boy'' is this: there exists some
person stage $x$ prior to the time of the utterance, such that $x$ was
a boy, and $x$ bears the temporal counterpart relation to Ted.  Since
there is such a stage, the claim is
true. \citeyearpar[193]{sider2001}.
\end{squote}

This theory can be adapted to our purposes.  We may say that sets and
sums, like stages, have temporal counterparts.  The set that is
currently designated by `the Supreme Court' bears a temporal
counterpart relation to other sets at other times.  The truth
condition of `The Supreme Court was formed in 1789' is perhaps the
fact that there was a set $S$ such that $S$ in 1789 bears the temporal
counterpart relation to the (current) Supreme Court and it is not the
case that there was some set $S^{\prime}$ and time $t$ such that $t$
is earlier than 1789 and $S^{\prime}$ in $t$ bears the temporal
counterpart relation to the Supreme Court.  It would then be literally
true to say ``The Supreme Court was formed in 1789''.  Likewise the
sum that is currently designated by `the Brick House' bears temporal
counterpart relations to other sums.  The truth condition of `The
Brick House did not exist last Tuesday' is perhaps the fact that the
Brick House does not bear a temporal counterpart relation to anything
on last Tuesday (or prior).

These are only rough formulations; they must be adapted to account for
the possibility of co-location (in a loose sense).  Recall that the
set that is currently designated by `the Supreme Court' might also
currently be designated by `the Special Committee'.  It is neither
true nor correct in any sense to say that the Special Committee was
formed in 1789.  If we were to adopt the theory of temporal
counterparts, we would have to recognize different kinds of
counterpart relations.  The set that is currently designated by `the
Supreme Court' bears the ``Supreme-Court-counterpart'' relation to the
set that was designated by `the Supreme Court' in 1789, but not the
``Special-Committee-counterpart'' relation.  It bears {\em that}
relation to other sets at other times.

I will not elaborate on this, however, because Sider's temporal
counterpart theory makes false assumptions about meaning.

Sider explicitly states that stages have an instantaneous temporal
duration---any given stage exists only for an instant
\citeyearpar[xiv]{sider2001}.  If we suppose that Ted is a
person-stage that exists only at instant $t$, then it is obviously not
true that it was the case that Ted existed at any previous time.  That
is, at $t$, the following is true:

\begin{displaymath}
\neg \exists t^{\prime} (t^{\prime}\ \text{is earlier than}\ t \wedge
\exists x\ (Ext^{\prime} \wedge x = \text{Ted}))
\end{displaymath}
(`$Ext$' means `$x$ exists at $t$'.)

But Sider also claims that `Ted was once a boy' is true.  This seems
to be equivalent to `There was some thing such that it was a boy and
it was Ted'.  That is, it appears that Sider is also committed to this
being true at $t$:

\begin{displaymath}
\exists t^{\prime} (t^{\prime}\ \text{is earlier than}\ t \wedge
\exists x\ (Ext^{\prime} \wedge Bx \wedge x = \text{Ted}))
\end{displaymath}
(Here `$Bx$' means `$x$ is a boy'.)

These are contradictory claims.  Sider must therefore be supposing
either that `$\exists$' is semantically ambiguous or that `Ted was
once a boy' {\em does not mean} `There was some thing such that it was
a boy and it was Ted'.  Sider is vehemently opposed to the idea that
there are multiple, equally suitable, meanings for quantifiers
\citeyearpar{sider2001,sider2011b,sider2011d}.  Therefore, I think
Sider is assuming that what `Ted was once a boy' means is `There
exists some person stage $x$ prior to the time of the utterance, such
that $x$ was a boy, and $x$ bears the temporal counterpart relation to
Ted'.

If `Ted was once a boy' does not mean `There exists some person stage
$x$ prior to the time of the utterance, such that $x$ was a boy, and
$x$ bears the temporal counterpart relation to Ted', then there is no
reason to think that the truth-condition of the former are the latter.
For it seems initially obvious that the truth-condition of `Ted was
once a boy' is that Ted (the stage) was once a boy.  If this is not
the truth-condition, then it must be because `Ted was once a boy' does
not actually mean that Ted was once a boy, but instead means that
there exists some person stage $x$ prior to the time of the utterance,
such that $x$ was a boy, and $x$ bears the temporal counterpart
relation to Ted.

In order to maintain that `Ted was once a boy' is literally true,
Sider must claim that it means something other than that Ted was once
a boy.  This is a highly implausible and unmotivated claim; the only
reason I can think of as to why Sider might make such a claim would be
because he holds a truth-conditional theory of meaning (see Section
\ref{verbal}) and believes that `Ted was once a boy' is true if and
only if `There exists some person stage $x$\,\ldots ' is true.  But a
truth-conditional theory of meaning is controversial and susceptible
to numerous counter-examples.  It seems far more reasonable to admit
that `Ted was once a boy' means that Ted was once a boy, and is
literally false.

\subsection{Chisholm's entia successiva}
\label{chisholm}
Roderick Chisholm was a mereological essentialist, claiming that
ordinary things cannot change their parts:

\begin{squote}
Familiar physical things such as trees, ships, bodies and houses
persist ``only in a loose and popular sense''.  This thesis may be
construed as presupposing that these things are ``fictions'', logical
constructions or {\em entia per alio} \citeyearpar[97]{chisholm1979}.
\end{squote}

Chisholm paraphrases talk involving persistence by stipulating a
technical sense of `successor' and `successive'.  He gives the
following definitions:

\begin{enumerate}[ref=\arabic*]
  \item $x$ is at $t$ a direct chair successor of $y$ at $t^{\prime}
    =_{df}$ (i) $t$ does not begin before $t^{\prime}$; (ii) $x$ is a
    chair at $t$ and $y$ is a chair at $t^{\prime}$; and (iii) there
    is a $z$, such that $z$ is a part of $x$ at $t$ and a part of $y$
    at $t^{\prime}$, and at every moment between $t$ and $t^{\prime}$,
    inclusive, $z$ is itself a chair. \label{suc1}
  \item $x$ is at $t$ a chair successor of $y$ at $t^{\prime} =_{df}$
    (i) $t$ does not begin before $t^{\prime}$; (ii) $x$ is a chair at
    $t$ and $y$ is a chair at $t^{\prime}$; and (iii) $x$ has at $t$
    every property \textsc{p} such that (a) $y$ has \textsc{p} at
    $t^{\prime}$ and (b) all direct chair successors of anything
    having \textsc{p} have \textsc{p}. \label{suc2}
  \item $x$ constitutes at $t$ the same successive chair that $y$
    constitutes at $t^{\prime} =_{df}$ Either (a) $x$ and only $x$ is
    at $t$ a chair successor of $y$ at $t^{\prime}$ or (b) $y$ and
    only $y$ is at $t^{\prime}$ a chair successor of $x$ at $t$
    \citep[99--100]{chisholm1979}. \label{suc3}
\end{enumerate}

Before we see how these definitions are used, there are two
misinterpretations (in my opinion) of Chisholm's position.  First, one
might take Chisholm to be claiming that ``successive chairs'' are {\em
  things} that are composed of or constituted by different bits of
matter at different times.  I think this is not how Chisholm should be
understood, for it would undermine his claim that successive chairs
are ``fictions'' that persist only in a ``loose and popular'' sense.

Second, one might take Chisholm's four definitions above to be giving
the literal meaning of the definienda.  That is, one might take
Chisholm to be claiming that what `$x$ constitutes at $t$ a successive
chair' {\em means} is `There are a $y$ and a $t^{\prime}$ such that
$y$ is other than $x$ and $x$ constitutes at $t$ the same chair that
$y$ constitutes at $t^{\prime}$'.  Whether or not this is what
Chisholm intended, I think it is false for two reasons.  First, it is
a highly implausible thesis about sentence meaning; why should we
think the the former sentence is synonymous with the latter, except
that it makes Chisholm's theory more palatable?  Second, if Chisholm's
definitions gave the literal meaning of the definienda, then it would
be true in the ``strict and philosophical sense'', as well as in the
``loose and popular sense'', that a successive chair persists over
time.  But Chisholm explicitly denies this
\citeyearpar[96--97]{chisholm1979}.

The interpretation of Chisholm that I prefer is this: when we speak of
a successive chair persisting over time, what we say is, strictly
speaking false.  However, we should be understood to {\em mean}
something other than what we say; what we mean can be captured with
the definitions given by Chisholm.  For example, when someone says
``That chair was made in 1900'', what they say is literally false, but
can be {\em paraphrased} by applying Definitions
\ref{suc1}--\ref{suc3}.  First we understand `That chair was made in
1900' to be equivalent to `$x$ (the present chair) constitutes now the
same successive chair that some $y$ constituted in 1900 and there is
no $z$ such that $z$ constitutes before 1900 the same chair that $x$
constitutes now'.  This is false, but someone making either utterance
should be taken to mean something else.  We can determine exactly what
is (or should be) meant by applying Chisholm's definitions in reverse:

\begin{enumerate}[start=3]
  \item $x$ and only $x$ is now a chair successor of some $y$ in 1900
    and there is no $z$ such that $x$ is now a chair successor of $z$
    before 1900.
\end{enumerate}
This is turn can be understood as

\begin{enumerate}[start=2]
  \item First, $x$ is a chair now and $y$ is a chair in 1900, and $x$
    has now every property \textsc{p} such that (a) $y$ has \textsc{p}
    in 1900 and (b) all direct chair successors of anything having
    \textsc{p} have \textsc{p}.  Second, there is no $z$ and $t$ such
    that $t$ begins before 1900 and $x$ is now a chair successor of
    $z$ at $t$.
\end{enumerate}
The meaning of `direct chair successor' is given by Definition
\ref{suc1}.

We can say the same thing about groups and ordinary things.  When
someone says ``The Supreme Court was formed in 1789'', what they say
is false, but should be paraphrased as something like this:

\begin{itemize}
  \item First, $S$ is the Supreme Court now and $T$ is the Supreme
    Court in 1789, and $S$ has now every property \textsc{p} such that
    (a) $T$ has \textsc{p} in 1789 and (b) all direct Supreme-Court
    successors of anything having \textsc{p} have \textsc{p}.  Second,
    there is no $V$ and $t$ such that $t$ begins before 1789 and $x$
    is now a chair successor of $V$ at $t$.
\end{itemize}

Likewise when someone says ``The Brick House did not exist last
Tuesday'', what they say should be paraphrased as `There is no $x$ and
$t$ such that $t$ begins before last Monday and the Brick House is now
a house successor of $x$'.

This solution is superior to Sider's theory of temporal counterparts
because it does not make questionable assumptions about meaning.  It
is false that the Supreme Court was formed in 1789, but it is correct
(in a ``loose and popular sense'') because the Supreme Court is a
successor of the ``original'' Supreme Court.  It is also false to say
that the Brick House did not exist last Tuesday, but it is correct, in
a loose and popular sense.

As given, however, Chisholm's definitions assume eternalism.  If the
set of 1789 justices no longer exists---if some matter that was part
of a justice is destroyed---then, without assuming that what {\em did}
exist always {\em does} exist, it is not true that there is a set $S$
such that the Supreme Court is a Supreme-Court-successor of $S$.  This
can only be true if $S$ exists.  If $S$ does not exist, then the
Supreme Court cannot be a successor of it.

However, it seems possible to reformulate Chisholm's definitions (or
write entirely new ones) so as to avoid this assumption.  The
following revisions of Definitions \ref{suc1}--\ref{suc3} illustrate
how this might be done ($t$ is the present time and `\textsc{Always}'
means `it is always the case that'):

\begin{enumerate}[label=\arabic*a., ref=\arabic*a]
  \item $x$ is at $t$ a direct chair successor of $y$ at $t^{\prime}
    =_{df}$ \textsc{Always}(if it is $t^{\prime} \rightarrow \exists
    y$ such that $y$ is a chair and such that \textsc{Always}(if it is
    $t \rightarrow \exists x$ such that $x$ is a chair and such that
    \textsc{Always}(if it is between $t^{\prime}$ and $t$ inclusive
    $\rightarrow \exists z$ such that $z$ is a chair and such that
    \textsc{Always}(if it is $t^{\prime} \rightarrow z$ is part of
    $y$) and such that \textsc{Always}(if it is $t \rightarrow z$ is
    part of $x$)))). \label{pres1}
  \item $x$ is at $t$ a chair successor of $y$ at $t^{\prime} =_{df}$
    \textsc{Always}(if it is $t^{\prime} \rightarrow \exists y$ such
    that $y$ is a chair and such that \textsc{Always}(if it is $t
    \rightarrow \exists x$ such that $x$ is a chair and such that
    \textsc{Always}(if it is $t \rightarrow x$ has every property
    \textsc{p} such that \textsc{Always}(if it is $t^{\prime}
    \rightarrow y$ has \textsc{p}) and such that \textsc{Always}(all
    direct chair successors of anything having \textsc{p} have
    \textsc{p})))). \label{pres2}
  \item $x$ constitutes at $t$ the same successive chair that $y$
    constitutes at $t^{\prime} =_{df}$ Either (a) $x$ and only $x$ is
    at $t$ a chair successor of $y$ at $t^{\prime}$ or (b) $y$ and
    only $y$ is at $t^{\prime}$ a chair successor of $x$ at
    $t$. \label{pres3}
\end{enumerate}
(Definition \ref{pres3} is identical to \ref{suc3}.)

We can again ``paraphrase'' talk about chairs over time.  As before,
we understand `That chair was made in 1900' to be equivalent to `$x$
(the present chair) constitutes now the same successive chair that $y$
constitutes at 1900 and there is no $z$ such that $z$ constitutes
before 1900 the same chair that $x$ constitutes now'.  This is false,
but we can determine exactly what is (or should be) meant by applying
our new definitions in reverse:

\begin{enumerate}[label=3a.]
  \item $x$ and only $x$ is now (at $t$) a chair successor of $y$ in
    1900 and there is no $z$ such that $x$ is now a chair successor of
    $z$ before 1900.
\end{enumerate}
This is turn can be understood as

\begin{enumerate}[label=2a.]
  \item \textsc{Always}(if it is 1900 $\rightarrow \exists y$ such
    that $y$ is a chair and such that \textsc{Always}(if it is $t
    \rightarrow \exists x$ such that $x$ is a chair and such that
    \textsc{Always}(if it is $t \rightarrow x$ has every property
    \textsc{p} such that \textsc{Always}(if it is 1900 $\rightarrow y$
    has \textsc{p}) and such that \textsc{Always}(all direct chair
    successors of anything having \textsc{p} have \textsc{p}) and such
    that \textsc{Always}(it is not the case that (if it is before 1900
    $\rightarrow \exists z$ such that $z$ is a chair and $z$ has
    \textsc{p}))))).
\end{enumerate}
The meaning of `direct chair successor' is given by Definition
\ref{pres1}.

Similar paraphrases can now be performed on utterances about the
Supreme Court, the Brick House, and other things.

Unfortunately, adopting this solution requires that we reject {\em
  serious presentism}.  Presentism is the thesis that only presently
existing things exist.  Serious presentism is the conjunction of that
thesis with the further claim that relations and properties can hold
only of existing things.  Serious presentism has the consequence that
it is literally false that I am smaller than Socrates.  This is not
because I am very large, but because Socrates does not exist.
Likewise, Socrates is not identical with himself because he does not
exist.

The last clause of Definition \ref{pres2}---``\textsc{Always}(all
direct chair successors of anything having \textsc{p} have
\textsc{p})''---violates serious presentism by positing a relation
(what we might call the ``chair successor'' relation) between
cross-temporal entities.  It may be possible to rewrite that clause to
avoid this assumption and make Chisholm's solution compatible with
serious presentism.  Then again, it may not.  If we do not want to
reject serious presentism, we may be forced to look for a different
solution.

\section{The conventions of persistence}
\label{set-convention}
Even supposing that our talk over time can be sorted out, there is
still more to be said.  Although chairs are sums that cannot change
their parts, we talk as if they can.  Likewise, although the Supreme
Court is a set, we talk as if the Supreme Court can change its
members.  The most reasonable way to make sense of this is to suppose
that for a given ``successive chair'' we use `chair' to refer to
different sums at different times; likewise, we use `the Supreme
Court' to refer to different sets at different times.  Chisholm's
definitions specify that a ``chair successor'' must be a chair, and a
``Supreme Court successor'' must be the Supreme Court, but they do not
specify how to determine what counts as a chair or as the Supreme
Court at any given time.  What makes it true that some sum is a chair,
or that some set is the Supreme Court?

What makes it true that some sum is a chair is just the fact that it
meets our conventional criteria for being a chair.  These criteria
probably cannot be given in terms of necessary and sufficient
conditions; the concept {\em chair} is too broad:

\begin{squote}
When one says chair, one thinks vaguely of an average chair.  But
collect individual instances, think of arm-chairs and reading chairs,
and dining-room chairs and kitchen chairs, chairs that pass into
benches, chairs that cross the boundary and become settees, dentists'
chairs, thrones, opera stalls, seats of all sorts, those miraculous
fungoid growths that cumber the floor of the Arts and Crafts
Exhibition, and you will perceive what a lax bundle in fact is this
simple straightforward term.  In co-operation with an intelligent
joiner I would undertake to defeat any definition of chair or
chairishness that you gave me \citep[384--385]{wells1904}.
\end{squote}
This is not a problem, since we can agree on paradigm examples of
chairs.  The term `chair' is obviously meaningful; this suggests that
the criteria for what counts as a chair are relatively well-defined,
even if we cannot adequately formalize them.  Thus what makes it true
at a given time that some sum is the ``chair successor'' of another
sum is the fact that both sums satisfy the criteria for being chairs
(at their respective times), and are related in the ways specified by
Chisholm's definitions.

But what about the Supreme Court?  What makes it true at a given time
that some set is then the Supreme Court?  I suggest that, again, there
are conventional criteria governing `the Supreme Court', and that
which set is at a given time the Supreme Court is a matter of
convention.  In the case of the Supreme Court, the conventions are
{\em legal} conventions.  The Constitution authorizes the recognition
of a set of justices as the Supreme Court.  Which set is recognized as
the Supreme Court is decided by the legislative and executive
branches.  The president nominates a set (the sitting justices and the
nominated justice) and the legislative branch votes.  The outcome of
the vote makes it true or false that a given set is the Supreme Court.
Thus what makes it true at a given time that some set is the ``Supreme
Court successor'' of another set is that fact that both sets satisfy
the criteria for being the Supreme Court (at their respective times),
and are related in the ways specified by Chisholm's definitions.

The correctness ({\em not} the truth) of cross-temporal talk about
groups is governed by convention.  This is plausible; groups are
social entities, and it is plausible that their ``change'' over time
should be due to convention.  But if this is right, it suggests that
the same holds for ordinary things.

\section{Am I a mereological sum?}
\label{i-sum}
I have proposed that ordinary things like chairs and statues are
mereological sums.  Their apparent persistence through change is a
result of certain conventions---a chair $x$ at $t_1$ is the ``same
successive chair'' as a chair $y$ at $t_2$ if the two are related in
the ways specified by Chisholm's definitions.

If ordinary things like chairs are sums, then are other things sums as
well?  I will suppose that sums are ``material things'' as opposed to
``abstract things'' (whatever that distinction comes to), but are {\em
  all} material things sums?  If we are material things, are we
therefore sums?

\subsection{All material things are sums}
\label{material-sum}
If we think that ordinary things are sums, and that ordinary things
are material things, I think it is extremely plausible to conclude
that all material things are sums.  For what else would they be?

What is included under the concept {\em material thing}?  I would
include things like chairs, and desks, and desk lamps, and doors, and
doorways, and houses, and gardens, and plants.  I would also include
minuscule objects like molecules and massive objects like planets and
galaxies.  What would these things be, if not sums?

I proposed that ordinary things are sums so as to avoid the conclusion
that there is a plurality of different kinds of ordinary things
(statues and lumps only scratch the surface) all overlapping each
other.  This essentialist proposal was made so as to avoid positing
many different kinds of things.  So anyone who accepts the
essentialist theory should be sympathetic to the idea that all
material things are sums.

I don't have much of an argument for this conclusion, but I don't see
the {\em point} of supposing that all and only ordinary things are
sums, but other material things are some different kind of object.

\subsection{We are material things}
\label{material-beings}
Even if the idea that all material things are sums is relatively
uncontroversial, the idea that {\em we} are material beings will not
be unanimously accepted.  For it does have some unintuitive
consequences.

First, it rules out identifying us with our mental states.  Suppose
all my psychological characteristics---memory, personality---is
somehow transferred to another body.  The brain in that body is
``wiped'' before my psychology is transferred, and after the operation
my old brain is similarly wiped.  There is a temptation to say that I
exist in the new body.  But saying this commits us to the claim that I
am not a material thing, because I ``left'' my old material body and
came to ``inhabit'' a new one:

\begin{squote}
 If I am identical with the thinking substance in which I am thus
 placed, then I cannot be transferred {\em from} that substance to
 another substance \citep[107]{chisholm1979}.
\end{squote}

Claiming that we are material things entails that psychological
continuity is not a criterion of identity.  The body into which my
psychology is transferred is not me, according to the materialist
claim.  Psychological continuity is often taken to be {\em the}
criterion of identity, so one might take this consequence as a
refutation of the claim that we are material things.

But if we are not material things, what are we?  The only alternative
I see is to claim that we are immaterial minds or souls.  These
positions seem, to me, to be more implausible than the claim that we
are material things.  (Much, of course, can be said in defense of this
alternative.)

Claiming that we are material things, however, gives rise to another
question: what material things are we?  Are we identical with our
brains, or with our bodies?

I suggest, though somewhat tentatively, that we are identical with our
bodies.  I agree with Peter van Inwagen on this much:

\begin{squote}
I suppose that [the objects of mental predicates]---Descartes, you,
I---are material objects, in the sense that they are ultimately
composed entirely of quarks and electrons.  They are, moreover, a very
special sort of material object.  They are not brains or cerebral
hemispheres.  They are living animals; being {\em human} animals, they
are things shaped roughly like statues of human beings
\citeyearpar[6]{inwagen1995}.
\end{squote}

Eric Olson has a very plausible argument for the same conclusion:

\begin{enumerate}
  \item There is a human animal sitting in your chair.
  \item The human animal sitting in your chair is thinking. (If you
    like, every human animal sitting there is thinking.)
  \item You are the thinking being sitting in your chair. The one and
    only thinking being sitting in your chair is none other than
    you. Hence, you are that animal \citeyearpar[354]{olson2003a}.
\end{enumerate}

One apparent consequence of the claim that we are material human
animals is that if my brain is removed from my body and put into
another body, that new person is not me.  Claiming that we are
material things required denying that psychological continuity is a
criterion of identity; claiming that we are material human animals
requires denying that even brain continuity is a criterion of
identity.

This may seem to be a troubling consequence, but it is much less
troubling if we accept the essentialist theory.  If material objects
are sums, and if we are material objects, then we are sums.  And if
sums do not, strictly speaking, change their parts over time, then,
like the ``persistence'' conditions for ``successive chairs'' and
other ordinary things, the ``persistence'' conditions over time for
{\em us} is conventional.

Another difficulty with identifying us with human animals disappears
if we accept an essentialist theory.  Dean Zimmerman has objected to
Olson's argument by claiming that ``human animal'' can be replaced
with ``human body'' without making the argument invalid
\citeyearpar[24]{zimmerman2008a}.  The problem, however, is that it
seems true that we cease to exist when we die.  So Zimmerman concludes
that we are not bodies or animals.

If we accept an essentialist theory, however, the problem disappears.
If, strictly speaking, I can't change my parts over time, then I am
not (strictly speaking) the same person that will be designated by
`Alex' a month from now (or even a week).  I will certainly not be
identical with a dead body further down the road.

\subsection{How do I ``persist'' over time?}
\label{person-persist}
The idea that, strictly speaking, I don't change my parts over time
seems crazy.  And maybe it is.  But I don't think it is obviously
false.

Someone who thinks that I do, strictly speaking, persist over time
might say that it is obvious that I persist.  After all, I engage in
activities that take long periods of time, I remember things from long
ago, and I bear unique attitudes toward my past and future selves.  I
feel pride or regret at past actions, and anticipation or apprehension
at future ones.  How could these past and future selves not be me?

One reply begins by pointing out that, whether or not we persist in a
strict sense, the world will look the same.  I will still engage in
activities that take time; but it will not be I who completes them.  I
will still remember things from long ago; but it will not be I who
experienced them.  I will bear attitudes towards past and future
people, but those people will not, strictly speaking, be me.  But it
will {\em seem} as if they are me, and they may be ``Alex successors''
in the sense defined by Chisholm (Section \ref{chisholm}).  As in the
case of tables and chairs, there are conventional ``persistence''
conditions for people over time.  Like tables and chairs, these
criteria will involve causal and spatiotemporal continuity.  What
person is designated by `Alex' a week from now will depend on a causal
chain connected to me.

Psychological continuity may also play a role.  For example, if by
some miracle I am vaporized and---quite coincidentally---a
qualitatively identical person is summoned into existence nearby, that
person will not, strictly speaking be me.  But it may be that the
person meets the criteria for being designated by `Alex'.  Then again,
it may not.  It may ultimately indeterminate whether or not that
person is Alex.  (My friends and family might have to {\em decide}
whether it is or not.)

The criteria for the ``persistence'' of people over time is not fully
precise, as shown by our indecision over whether we would use `Alex'
to refer to a spontaneous duplicate of me.  Another, more realistic,
situation in which this indecision manifests itself is in death.
Suppose I die, and a wake is held for my body.  It is perfectly
correct for someone to point and say, ``That was Alex''.  But it is
equally correct to say ``That's Alex''.  (The latter may be more
appropriate if it is necessary to identify my body.)  Is the
mereological sum that is the (deceased) body really me, or not?  If we
accept the essentialist theory, it is (strictly speaking) not, but it
may be correct or appropriate to use `Alex' to refer to the body.  If
it is, this will be because the body satisfies (or nearly satisfies)
the conventional criteria for being me.

\section{Lessons}
\label{lessons-e}
In Section \ref{parts} I examined three different versions of the
``plurality thesis''; the view that there are pluralities of
co-located objects.  In this section I offered an alternative.  I am
not sure whether my theory or one of the plurality theories is
correct, but I suspect that it must be one or the other.  My
conclusion is largely the same as that of Karen Bennett:

\begin{squote}
The only live options, then, are to be either a one-thinger or a
bazillion-thinger.  We must either think that there is only one thing per
spatio-temporal location, or else that there are lots and \emph{lots} of
spatio-temporally coincident things \citeyearpar[358]{bennett2004}.
\end{squote}

I would prefer to be a ``one-thinger'' because it does not commit me
to a ``bazillion'' things all in the same place.  That is not a
decisive objection, of course.  It may well be that such an explosion
is more plausible than certain consequences of the ``one-thinger''
theory.  But I think one of the two theories must be right.

Just as we demanded that the plurality theories could be equipped with
an explanation as to why we don't believe there to be as many things
as there are, so this essentialist thesis should be supplemented with
an explanation as to why we {\em do} believe that things change their
parts, when they in fact don't.

\subsection{Can the essentialist theory explain what we believe?}
\label{explain-e}
In Section \ref{explain-p} I assessed whether any of the three
plurality theories could explain why we hold beliefs that conflicted
with certain consequences of the theories.  The same assessment may be
conducted with regard to the essentialist theory I have sketched here.
If the essentialist thesis is right, why do we believe that chairs can
change their parts?

One explanation is simply that we {\em don't} believe that things
literally persist over time.  When asked ``Is it {\em literally} the
same chair without its leg?'' some of us may waver, and perhaps
concede that we don't think it is really the same chair.  But I doubt
this reply will convince any philosopher who has already made up her
mind about essentialism.

Another reply is that we are fooled by the great similarity between
``successive chairs'', both with regard to appearance and with regard
to their spatiotemporal location.  If we see a certain chair in the
sitting room, and while we are away it is replaced by a different
chair (someone carries one out and places another in the room), then
we will likewise be fooled by the similarities between the two, and
mistake them for one and the same thing.  This idea is largely due to
Hume:

\begin{squote}
Nothing is more apt to make us mistake one idea for another, than any
relation betwixt them, which associates them together in the
imagination, and makes it pass with facility from one to the other.
Of all relations, that of resemblance is in this respect the most
efficacious; and that because it not only causes an association of
ideas, but also of dispositions, and makes us conceive the one idea by
an act or operation of the mind, similar to that by which we conceive
the other.  This circumstance I have observ'd to be of great moment;
and we may establish it for a general rule, that whatever ideas place
the mind in the same disposition or in similar ones, are very apt to
be confounded\,\ldots

Now what other objects, besides identical ones, are capable of placing
the mind in the same disposition, when it considers them, and of
causing the same uninterrupted passage of the imagination from one
idea to another?\,\ldots I immediately reply, that a succession of
related objects places the mind in this disposition, and is consider'd
with the same smooth and uninterrupted progress of the imagination, as
attends the view of the same invariable object.  The very nature and
essence of relation is to connect our ideas with each other, and upon
the appearance of the one, to facilitate the transition to its
correlative.  The passage betwixt related ideas is, therefore, so
smooth and easy, that it produces little alteration on the mind, and
seems like the continuation of the same action; and as the
continuation of the same action is an effect of the continu'd view of
the same object, 'tis for this reason we attribute sameness to every
succession of related objects.  The thought slides along the
succession with equal facility, as if it considered only one object;
and therefore confounds the succession with the identity
\citep[135]{hume2000}.
\end{squote}

Hume claims that from a succession of similar impressions, we come to
believe, through a ``fiction of the imagination'', that there is a
single enduring object causing the succession of impressions.
Likewise, I am suggesting that if this essentialist theory is true,
then we come to believe that chairs can change their parts through a
fiction of the imagination.  When looking at a ``successive chair'',
we see a series of sums that resemble each other in their appearance
and spatiotemporal location.  Due to such great similarities, we
mistakenly take them to be a single enduring thing.

\subsection{What can we learn from Fine's theory?}
\label{need-fine}
I have argued that we can identify ordinary things like chairs as
mereological sums, and we can identify things like groups as sets.  It
is therefore not necessary to use Fine's theory of operators (Section
\ref{fine-c}) to describe these things.  Is there anything we can take
away from Fine's theory?

At the very least, Fine's theory is valuable for its insight that
there are different ways of being a part.  It shows that sums and sets
both have parts, but in different ways.  It suggests that there are
also sequences, strings, words, poems, events, and quantities, each
perhaps having their parts in different ways.

Some philosophers who adhere to a more or less classical mereology
believe that physical or material things are the only things that
exist (van Inwagen is one).  For such philosophers, there is only one
way of being a part, and anything that has parts (which is everything)
is a mereological sum.  I do not share this view; I think there are
also sets, and probably other kinds of things.  I do not think,
therefore, that anything that has parts is a mereological sum.  Sets
have parts, and sets are not sums.  My theory of essentialism must
therefore operate with a definition of mereology that does not entail
that everything that has parts is a sum.  One way (though perhaps not
the best way) to ensure this is to say that mereological sums are all
and only physical things.  Appropriate qualifications may be added to
the definitions in Section \ref{tech}.

But if everything is not a sum, if there are sets and probably other
kinds of things as well, how {\em many} kinds of things are there?  If
the essentialist theory in this section was meant to avoid many
different kinds of overlapping things, how can I allow that, in
addition to sums and sets, there might also be strings, and sequences,
and words, and poems, and an unknown number of other things?

I am not sure.  But one, perhaps minor, advantage of my theory is that
it allows us to retain at least a semblance of our pre-reflective
categorization of ordinary things.  According to Fine, chairs,
statues, lumps, boats, and kittens are all different kinds of things,
occupying different ontological categories.  According to the
essentialist, they are all the same kind of thing---they are all
physical sums.  The essentialist theory may recognize different kinds
of things, but it does not multiply kinds beyond necessity.

\subsection{Deflationary metaphysics}
\label{deflate}
Kathrin Koslicki has an interesting objection to universalist theses
such as the one I appear committed to.  Her objection amounts to this:
if every set of objects (such as the London Bridge, a particle in the
moon, and Cal Ripkin, Jr.) is a thing in its own right (a sum), then
metaphysics becomes uninteresting.  There is no longer any debate
about whether chairs or dogbushes are more ``real'' or have a stronger
claim to existence.  They both exist, and the difference between
chairs and lumpkins is not ontological but conceptual: `chair' is more
embedded in our talk, and so chairs have greater importance to {\em
  us}.  But metaphysically, or ontologically, chairs and dogbushes are
on the same level.  There is no sense in which chairs exist and
lumpkins do not.

In the quote below, Koslicki is criticizing a version of
four-dimensionalism that Sider has previously defended.  Sider's
position was that any collection of objects-at-times composes a sum.
(Sider uses `fusions' to refer to sums.)  For example, a chair is a
fusion of a large number of {\em temporal part} of things (wood
molecules, or atoms, or simples).  Each thing (wood molecule, atom, or
simple) is a fusion of {\em its} temporal parts.  Each temporal part
of the chair is also a thing (a fusion).

Again, I take no stand on whether objects have temporal parts or
rather ``endure'' through time.  But Koslicki's comments are relevant
nonetheless:

\begin{squote}
There is room, in Sider's theory, for {\em some} genuine ontological
disagreements: for example, the universalist, the nihilist and the
holder of the intermediary position genuinely disagree over how many
and which fusions that exist.  But the only genuine ontological
disagreements for which there is room, in Sider's world, are ones that
concern disagreements over ``bare'' fusions, so to speak.  What has
happened to the houses, trees, people, and cars, the familiar concrete
objects of common-sense, whose persistence this account set out to
analyze?  There are no ``deep'' ontological facts as to whether a
given fusion should count as a house or not\,\ldots

[By claiming that there can be genuine ontological disputes while also
  promoting four-dimensionalism,] Sider is guilty of a bit of false
advertising: his account is really a way of saying that, at the end of
the day, there is no interesting {\em ontological} story to be told
about the persistence of our familiar concrete objects of
common-sense; whatever there is to say about the persistence of
houses, trees, people and cars concerns the organization of our
conceptual household \citeyearpar[124--125]{koslicki2003}.
\end{squote}

Koslicki seems to think that we ought to be able to find some
ontological difference between ``the familiar concrete objects of
common-sense'' and ``bare fusions'' like lumpkins or chairs-at-times.
But according to Sider's four-dimensional mereology, anything with
parts is, {\em by definition}, a fusion.  Fusions are just things with
parts.  Lumpkins have parts, and are therefore fusions.  Chairs and
houses have parts, and are therefore fusions.  To complain that
ordinary things should be something more than ``bare fusions'' appears
to exhibit a confusion about what fusions are.

Moreover, as I remarked above (Section \ref{universalism}), why should
what interests us (familiar objects like chairs) be a guide to what
exists?  The only difference between ordinary things like chairs and
unusual things like dogbushes seems to be the fact that we care about
the former and not about the latter.  There does not seem to be any
metaphysical or ontological difference between the two; both are sums
or fusions.  The conclusion that ``the persistence [and other
  properties] of houses, trees, people and cars concerns the
organization of our conceptual household'' therefore seems to be
correct.

However, there {\em is} an ontological difference between some things,
if not between chairs and dogbushes.  One lesson of Kit Fine's theory
of parts is that mereological sums may not be the only kind of
composite thing.  There are apparently sets as well, and strings, and
sequences, and perhaps many other types of thing.  The difference
between a set and a sum is probably an ontological difference, and
identifying what distinguishes sets from sums (and from other kinds of
things) is an interesting metaphysical question.  The field of
metaphysics is not then so barren, as Koslicki seems to have feared.
But it is true that many interesting questions---When are we willing
to call something a chair, and why?  What conditions must be
fulfilled?---are not ontological questions.  They are questions about
our ``conceptual household.''

\ifstandalone
\end{spacing}
\bibliography{everything}
\bibliographystyle{ChicagoReedweb}
\fi
\end{document}
