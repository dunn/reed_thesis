\documentclass[11pt]{article}
\usepackage{standalone} \newif\ifstandlone \standalonetrue
\usepackage[left=1.75in, right=1.75in, top=1.25in, bottom=1.25in]{geometry}
\geometry{letterpaper}
\usepackage{graphicx}
\usepackage{enumitem}
%\usepackage{amssymb}
\usepackage{amsmath}
\usepackage{epstopdf}
\usepackage{verbatim}
\usepackage{setspace}
\usepackage{natbib}
\setcitestyle{aysep={}}
\usepackage{url}
\usepackage{hyperref}
\synctex=1

\DeclareSymbolFont{symbolsC}{U}{txsyc}{m}{n}
\DeclareMathSymbol{\strictif}{\mathrel}{symbolsC}{74}
\DeclareMathSymbol{\boxright}{\mathrel}{symbolsC}{128}

\newenvironment{squote}{%
\begin{spacing}{1}
\begin{list}{}{%
\setlength{\labelwidth}{0pt}%
\rightmargin\leftmargin%
}
\item\relax
}{%
\end{list}%
\end{spacing}
}

\title{Introduction}
\author{Alexander A. Dunn}
\begin{document}
\ifstandalone
\maketitle
\begin{spacing}{1.5}
\fi

It is true that there are chairs.  This, as far as I'm concerned, is
obvious.  If someone says denies that there are chairs, then it seem
to me that somehow they have gone astray.  If they have an argument
for this conclusion, there must be something wrong with the argument.
There must be something wrong because it is true---obviously
true---that there are chairs.

In many parts of what follows, I will often say things such as ``I
believe that there are chairs''.  This is not due to an unwillingness
to assert the stronger claim---that there are chairs.  The stronger
claim is, I have said, obviously true.  But I will use the weaker
claim---``I believe that there are chairs''---because while the
philosophers I am criticizing deny that there are chairs, they do not
deny that I believe that there are chairs.  And this fact alone---that
I believe that there are chairs---causes some trouble for their views,
and gives us reason to doubt their extraordinary conclusions.

So I believe that there are chairs.  I also believe that there are
desks, and desk lamps, and doors, and doorways, and houses, and
gardens, and plants.  Such things, and many others, are commonly
referred to as `ordinary things'.  This phrase is extremely vague in
its application, but may be taken to designate macroscopic objects,
such as those listed above, that are parts of our everyday lives.

Many philosophers have denied that ordinary things exist.  Until
recently, such a denial was generally a consequence of the
philosopher's views on other matters.  If a philosopher claims that
there is no external world, or that the world is not at all like it
appears, then she might deny that there are any physical things, or
any things that exist outside the mind, or anything at all.  It
follows from such a claim that there are no ordinary things like
chairs.  But such a philosopher is not specifically interested in
denying that chairs exist.  She is interested in denying that {\em
  anything} exists; the denial of chairs is a minor consequence.

In the past 30 years, however, certain philosophers who we will refer
to collectively as {\em nihilists} have constructed arguments
specifically designed to show that there are no ordinary things.
(Peter Unger was one of the first, with the aptly titled paper,
``There are no ordinary things''.)  These philosophers do not deny
that there is an external world, or that it contains many physical
things; these propositions are readily granted to be true.  But they
are unwilling to admit that such a world does---or even possibly
could---contain chairs.

Most philosophers making this sort of claim admit that it is strange
and unintuitive.  But they believe that the benefits of denying the
existence of ordinary things outweighs the costs.  Different
philosophers cite different benefits: consistency with regard to our
notion of composition, theoretical simplicity, or greater coherence
with our other beliefs.

The benefits do not outweigh the costs.  Moreover, I am unable to
imagine that any argument could convince me that there are no ordinary
things.  I believe that any argument that has the nonexistence of
chairs as a consequence is flawed.  Whether or not we can immediately
identify the flaw in the argument, the fact that it entails a
falsehood shows that something has gone amiss.

It will be objected that this is merely a fact about myself; other
philosophers are perfectly willing to deny that there are chairs.  It
may be argued that since I consider ``there are chairs'' to be true no
matter what, I must consider it to be some sort of conceptual truth.
It may be further argued that, since there are philosophers willing to
deny that ``there are chairs'' is true, what I mean by ``there are
chairs'' is something different than what these philosophers mean by
``there are chairs''.  We may be thought to be using our words in
different ways.

In Section \ref{verbal} I will argue that we are {\em not} using our
words in different ways.  When I say ``there are chairs'' and someone
else says ``there are not chairs'', we are having a real disagreement.
Moreover, we are disagreeing in English; there is no special
``ontological language'' in which we do metaphysical philosophy.

In Section \ref{stroud} I will argue that any philosopher who attempts
to deny that there are chairs should be able to explain why we
nonetheless believe that there are chairs.  This seems to be a
reasonable request, but it is surprisingly hard to satisfy.  The
difficulties that nihilistic philosophers have in explaining why we
believe that there are chairs should make us suspicious of their
conclusions.

But even if we show that there are problems with the arguments of
philosophers who deny that ordinary things exist, we have not thereby
proved that they {\em do} exist.  The nihilistic philosophers who deny
that there are chairs are motivated to do so by a number of puzzles
about the nature of ordinary things.  For example, why are there
chairs and tables, but not chair-tables (single objects composed of an
adjacent table and chair)?

In Section \ref{universe}, however, I will argue that some of the
considerations that philosophers take to be good reasons to deny that
chairs exist are not good reasons at all.  In effect, these
philosophers take the apparent non-existence of chair-tables to tell
against the existence of chairs and tables.  On the contrary, as we
will see, the obvious existence of chairs and tables tells {\em for}
the existence of chair-tables, dogbushes, and other strange things.

In Section \ref{parts} I will examine three theories that seek to make
sense of all these different objects.  I will argue that all have the
consequence that there are a {\em plurality of overlapping
  objects}---that where we might think there is just one thing (a lump
of clay), there are actually millions or more.  I will suggest that
this unwelcome consequence should encourage us to look for a different
sort of theory.

In Section \ref{essential} I will attempt to defend the claim that
ordinary things are {\em mereological sums} which cannot change their
parts.  What we take to be a chair with a new leg, for instance, is
really a new chair.  The most interesting consequence of this is that
the ``persistence conditions'' for things over time---the conditions
in which a certain thing is the referent of a term like `the
Washington Monument'---are wholly conventional.

Throughout this thesis, there are certain things I will {\em not}
presuppose.  First, I will take no stand on whether or not things have
{\em temporal parts}.  I am not sure I fully understand the doctrine
of temporal parts, but it is often summarized thus: if a thing has
temporal parts, then for each time at which it exists, there exists at
that time (and only at that time) another thing---a temporal part or
``slice'' of the larger object.  The (temporally) larger object is
somehow `built up' from these temporally smaller parts.  If a thing
does not have temporal parts, then it is not divided into temporal
slices---it is ``wholly present'' at every moment of its existence.
Whatever this debate comes to, I will try to avoid relying on the
truth or falsity of the doctrine of temporal parts.

Second, I will not presuppose {\em eternalism}.  Eternalism is,
roughly, the view that past and future time are just as ``real'' as
the present.  An analogy is often drawn with space; what's behind me
and in front of me is just as real as what is under me.  There is
nothing special about {\em here} rather than {\em there}.  Likewise
the eternalist claims that {\em now} is no more special that {\em
  then}.  Eternalism is generally opposed to {\em presentism}, which
is the view that only the present is real.  The presentist and the
eternalist both agree that there {\em were} dinosaurs, but for the
eternalist there is a sense in which there {\em are} dinosaurs (they
just don't exist now).  I will attempt to avoid committing myself to
the reality of anything but the present.

\ifstandalone
\end{spacing}
\bibliography{everything}
\bibliographystyle{ChicagoReedweb}
\fi
\end{document}
