\documentclass[11pt]{article}
\usepackage{standalone} \newif\ifstandlone \standalonetrue
\usepackage[left=1.75in, right=1.75in, top=1.25in, bottom=1.25in]{geometry}
\geometry{letterpaper}
\usepackage{graphicx}
\usepackage{enumitem}
%\usepackage{amssymb}
\usepackage{amsmath}
\usepackage{epstopdf}
\usepackage{verbatim}
\usepackage{setspace}
\usepackage{natbib}
\setcitestyle{aysep={}}
\usepackage{url}
\usepackage{hyperref}
\synctex=1

\DeclareSymbolFont{symbolsC}{U}{txsyc}{m}{n}
\DeclareMathSymbol{\strictif}{\mathrel}{symbolsC}{74}
\DeclareMathSymbol{\boxright}{\mathrel}{symbolsC}{128}

\newenvironment{squote}{%
\begin{spacing}{1}
\begin{list}{}{%
\setlength{\labelwidth}{0pt}%
\rightmargin\leftmargin%
}
\item\relax
}{%
\end{list}%
\end{spacing}
}

\title{Introduction}
\author{Alexander A. Dunn}
\begin{document}
\ifstandalone
\maketitle
\begin{spacing}{1.5}
\fi

%\section{How to deny what is true}
It is true that there are chairs.  This, as far as I'm concerned, is
obvious.  If someone says that it is not true that there are
chairs---that there are not chairs---then it is clear to me that
somehow they have gone astray.  If they have an argument for this
conclusion, there must be something wrong with the argument.  There
must be something wrong because it is true---obviously true---that
there are chairs.

In many parts of what follows, I will often say things such as ``I
believe that there are chairs''.  This should not be taken as an
unwillingness to assert a stronger claim---that there are chairs.  The
stronger claim is, I have said, obviously true.  But I will use the
weaker claim---```I believe that there are chairs''---because while
the philosophers I am criticizing deny that there are chairs, they do
not deny that I believe that there are chairs.  And I am going to
argue that this fact alone---that I believe that there are
chairs---causes some trouble for their views, and gives us reason to
doubt their extraordinary conclusions.

So I believe that there are chairs.  I also believe that there are
desks, and desk lamps, and doors, and doorways, and houses, and
gardens, and plants.  Such things, and many others, are commonly
termed `ordinary things'.  This term is extremely vague in its
application, but is taken to refer to macroscopic objects, such as
those listed above, that are parts of our everyday lives.

Many philosophers have denied that ordinary things exist.  Until
recently, such a denial was generally a consequence of the
philosopher's views on other matters.  If a philosopher claimed that
there was no external world, or that the world was not at all like it
appears, then they might deny that there were any physical things, or
any things that exist outside the mind, or anything at all.  It would
follow from such a claim that there would be no ordinary things, like
chairs.  But the philosopher would not be specifically interesting in
denying that chairs exist.  They were interested in denying that {\em
  anything} exists; the denial of chairs was a minor consequence.

In the past 30 years, however, philosophers have begun to construct
arguments specifically aiming to show that there are no ordinary
things.  (Peter Unger was one of the first, with the aptly titled
paper, ``There are no ordinary things''.)  These philosophers do not
deny that there is an external world, or that it contains many
physical things; these propositions are readily granted to be true.
But the philosophers are unwilling to admit that such a world
does---or even possibly could---contain chairs.

Most philosophers making this sort of claim admit that it is strange
and unintuitive.  But they believe that the benefits of denying the
existence of ordinary things outweighs the costs.  Different
philosophers cite different benefits: consistency with regard to our
notion of composition, theoretical simplicity, or greater coherency
among our beliefs.  

The benefits do not out-weight the cost.  Moreover, I am unable to
imagine that any argument could convince me that there are no ordinary
things.  I believe that any argument that has the nonexistence of
chairs as a consequence is flawed.  Whether or not we can immediately
identify the flaw in the argument, the fact that it entails a
falsehood shows that something has gone amiss.

It will be objected that this is merely a fact about myself; other
philosophers are perfectly willing to deny that there are chairs.  (It
is another fact about me that I doubt that they really believe that
there are no chairs.)  It may be argued that the fact that I consider
``there are chairs'' to be true regardless of arguments against its
truth shows that I consider it to be, in some sense, a conceptual
truth.  It may be further argued that, since there are philosophers
willing to deny that ``there are chairs'' is true, what I mean by
``there are chairs'' is something different than what these
philosophers mean by ``there are chairs''.  We may be thought to be
using our words in different ways.

In section \ref{verbal} I will argue that we are {\em not} using our
words in different ways.  When I say ``there are chairs'' and someone
else says ``there are not chairs'', we are having a real disagreement.
Moreover, we are disagreeing in English; there is no special
`ontological language' in which we do metaphysical philosophy.

In section \ref{stroud} I will argue that any philosopher who attempts
to deny that there are chairs should be able to explain why we
nonetheless believe that there are chairs.  This is, I think, a
reasonable request, but it is surprisingly hard to satisfy.  The
difficulties that `nihilistic' philosophers have in explaining why we
believe that there are chairs should give us reason to suspect their
conclusions.

But even if we show that there are problems with the arguments of
philosophers who deny that ordinary things exist, we have not thereby
proved that they {\em do} exist.  The philosophers who say that there
are no chairs are motivated by a number of questions about the nature
of ordinary things.  For example, why are there chairs and tables, but
not chair-tables (single objects composed of an adjacent table and
chair)?

In section \ref{universe}, however, I will argue that some of the
considerations that philosophers take to be good reasons to deny that
chairs exist are not good reasons at all.  In effect, these
philosopher take the apparent non-existence of chair-tables to tell
against the existence of chairs and tables.  On the contrary, as we
will see, the {\em obvious} existence of chairs and tables tells for
the existence of chair-tables.  Additionally, I will argue that teams,
families, crews and other `groups' exist and have parts, just like
`material objects' like chairs.  

In section \ref{parts} I will examine two theories that seek to make
sense of all these different objects.  I will argue that both have the
consequence that there are a {\em plurality of co-located
  objects}---that where we might think there is just one think (a lump
of clay), there are actually millions or more.  I will suggest that
this consequence is a reason to be suspicious of both theories, and
should encourage us to look for a theory without this consequence.

In section \ref{essential} I will attempt to develop such a theory.  I
will claim, for example, that the way in which a thing is part of a
group is the way that things are parts of sets---I will argue that
groups can be identified with sets.  Since sets cannot change their
membership over time, the set referred to by a term for a group (`the
Supreme Court') will change over time.  The most interesting
consequence of this is that the {\em identity conditions} for a group
over time---the conditions in which a certain set is identical with
the group---are wholly conventional.  Even more interesting is that
this seems to be the case with ordinary things like chairs as well;
whether and how a chair persists over time is largely up to us.

Throughout this thesis, there are certain things I will {\em not}
presuppose.  First, I will take no stand on whether or not things have
`temporal parts'.  I am not sure I fully understand the doctrine of
temporal parts, but it is often summarized thus: if a thing has
temporal parts, then for each time at which it exist, there exists at
that time (and only at that time) another thing---a temporal part or
`slice' of the larger object.  The (temporally) larger object is
somehow `built up' from these temporally smaller part.  If a thing
does not have temporal parts, then it is not divided into temporal
`slices'---it is ``wholly present'' at every moment of its existence.
Whatever this debate comes to, I will try to avoid relying on the
truth or falsity of the doctrine of temporal parts.

Second, I will not presuppose {\em eternalism}.  Eternalism is,
roughly, the view that past and future time are just as `real' as the
present.  An analogy is often drawn with space; what's behind me and
in front of me is just as real as what is under me.  There is nothing
special about `here' rather than `there'.  Likewise the eternalist
claims that `now' is no more special that `then'.  Eternalism is
generally opposed to {\em presentism}, which is the view that only the
present is real.  The presentist and the eternalist both agree that
there {\em were} dinosaurs, but for the eternalist there is a sense in
which there {\em are} dinosaurs (they are just not `now').  Again, I
will attempt to avoid committing myself to either of these positions.

\ifstandalone
\end{spacing}
\bibliography{everything}
\bibliographystyle{ChicagoReedweb}
\fi
\end{document}
