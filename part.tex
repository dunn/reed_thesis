\documentclass[11pt]{article}
\usepackage{standalone} \newif\ifstandlone \standalonetrue
\usepackage[left=1.75in, right=1.75in, top=1.25in, bottom=1.25in]{geometry}
\geometry{letterpaper}
\usepackage{graphicx}
\usepackage{enumitem}
\usepackage{amssymb}
\usepackage{amsmath}
\usepackage{epstopdf}
\usepackage{verbatim}
\usepackage{setspace}
\usepackage{natbib}
\setcitestyle{aysep={}}
\usepackage{url}
\usepackage{hyperref}
\synctex=1

\DeclareSymbolFont{symbolsC}{U}{txsyc}{m}{n}
\DeclareMathSymbol{\strictif}{\mathrel}{symbolsC}{74}
\DeclareMathSymbol{\boxright}{\mathrel}{symbolsC}{128}

\newenvironment{squote}{%
\begin{spacing}{1}
\begin{list}{}{%
\setlength{\labelwidth}{0pt}%
\rightmargin\leftmargin%
}
\item\relax
}{%
\end{list}%
\end{spacing}
}

\title{Does a chair change its parts?}
\author{Alexander A. Dunn}
\begin{document}
\ifstandalone
\maketitle
\begin{spacing}{1.5}
\fi

\label{parts}

In the previous section I argued that not only are there ordinary
things like chairs, but that there are more unusual things like
archipelagos, scattered works of art, and perhaps even dogbushes.  I
tried to make it at least plausible to assume that some version of
{\em universalism} is true---that for any material things, there is
some material object made up of them.

If this is the case, a new problem arises.  The problem is that
universalism, along with a few plausible assumptions, rules out the
possibility that things change their parts.  But this seems clearly
false; things appear to change their parts all the time.  The
philosopher who accepts universalism and claims that things change
their parts must deny one or more of the plausible assumptions that,
with universalism, rule out the possibility of things' changing their
parts.  One of these plausible assumptions is the assumption that no
two things completely overlap one another; that is, are {\em
  co-located}.  Denying this assumption leads to the possibility that
there are very many things in any given location; for example, there
might be a plurality of objects co-located with my chair at this very
instant.

There are reasons to be suspicious of a theory that entails such a
plurality of things.  But the alternative, which I will present in
section \ref{essential}, is to deny that things really can change
their parts.

\section{Parthood and composition}
\label{part-comp}
This subsection will describe the `classical' notions of mereology.
The technical formulae of classical mereology are often defined in
terms of parthood, which is itself left undefined (except for the
stipulation that everything is a part of itself).  `Part', moreover,
is understood to have a single, univocal, meaning; the consequence is
that everything that has a part (or parts) is a mereological sum.
Since everything is by definition a part of itself, classical
mereology entails that everything is a mereological sum.  We will see
in section \ref{all-sum} that this consequence is problematic.

If we make several plausible assumptions, classical mereology also
entails that things cannot change their parts.  This consequence also
seems problematic, and the theories that I will present below are
motivated largely to avoid this consequence.

\subsection{Classical mereology}
\label{tech}
Peter van Inwagen provides the following definitions for classical
mereology:

\begin{enumerate}
  \item $x\ \text{is a part of}\ y =_{df} x\ \text{is a proper part
    of}\ y\ \text{or}\ x = y$
\end{enumerate}

It is assumed that everything is a part of itself.  A `proper part' of
some $x$ is a part that is not $x$ itself.

\begin{enumerate}[start=2]
  \item $x\ \text{overlaps}\ y =_{df} \text{For some}\ z, z\ \text{is
    a part of}\ x\ \text{and}\ z\ \text{is a part of}\ y$
\end{enumerate}

Just as everything is a part of itself, everything overlaps itself
(when $x = y = z$).

\begin{enumerate}[start=3]
  \item $x$ is a mereological sum of the $y$s $=_{df}$ For all $z$ (if
    $z$ is one of the $y$s, $z$ is a part of $x$) and for all $z$ (if
    $z$ is a part of $x$, then for some $w$, ($w$ is one of the $y$s
    and $z$ overlaps $w$)) \citeyearpar[618--619]{inwagen2006}.
\end{enumerate}

The first part of this last definition---``if $z$ is one of the $y$s,
$z$ is a part of $x$''---specifies that all of the $y$s are part of
$x$.  We cannot say that {\em only} the $y$s are part of $x$, because
the sum of half of the $y$s is also part of $x$.  But we can say that
every part of $x$ overlaps one of the $y$s.  The second part of the
definition---``if $z$ is a part of $x$, then for some $w$, $w$ is one
of the $y$s and $z$ overlaps $w$''---secures this.

There are at least two limitations to this classical formulation of
mereology.  First, it entails that everything that has parts is a
mereological sum.  Second, it does not explain how, if at all,
mereological sums can change their parts.

\subsection{Is everything a mereological sum?}
\label{all-sum}
One problem with the classical formulations of mereology is
that---like van Inwagen's definitions above---they entail that
everything that has parts is a mereological sum.  And though it may be
an unreflective prejudice, I am inclined to believe that mereological
sums are {\em physical}, or material, things.  Material things
certainly have parts, but they are not the only things:

\begin{squote}
The word `part' is applied to many things besides material objects.
We have already noted that submicroscopic objects like quarks and
protons are at least not clear cases of material objects;
nevertheless, every material object would seem pretty clearly to have
quarks and protons as \emph{parts}, and, it would seem, in exactly the
same sense of \emph{part} as that in which a paradigmatic material
object might have another paradigmatic material object as a part.  A
``part,'' therefore, need not be a thing that is clearly a material
object.  Moreover, the word `part' is applied to things that are
clearly \emph{not} material objects---or at least it is on the
assumption that these things really exist and that apparent reference
to them is not a mere manner of speaking.  A stanza is a part of a
poem; Botvinnik was in trouble for part of the game; the part of the
curve that lies below the x-axis contains two minima; parts of his
story are hard to believe\,\ldots\,such examples can be multiplied
indefinitely \citeyearpar[18--19]{inwagen1995}.
\end{squote}

Under our current conception of a mereological sum, things like poems
seem to be included.  For recall van Inwagen's definition:

\begin{squote}
$x$ is a mereological sum of the $y$s $=_{df}$ For all $z$ (if $z$ is
  one of the $y$s, $z$ is a part of $x$) and for all $z$ (if $z$ is a
  part of $x$, then for some $w$, ($w$ is one of the $y$s and $z$
  overlaps $w$)) \citeyearpar[618--619]{inwagen2006}.
\end{squote}

Stanzas are parts of the poem, as are lines, words, and letters.  For
simplicity's sake, though, let us pretend that only words are parts of
poems.  Let the $y$s therefore be all the words in a poem $x$.

Suppose $z$ is the word `bear'.  The word `bear' is a word in the
poem, so it is one of the $y$s.  The first part of van Inwagen's
definition tells us that `bear' is therefore part of the poem.  Now
take the second part of the definition.  We have established that $z$
(`bear') is part of $x$ (the poem), so the antecedent of the
conditional (``if $z$ is a part of $x$'') is true.  If the poem is a
mereological sum, then the consequent must also be true.  There must
be some one of the $y$s that overlaps $z$.  Since $z$ is one of the
$y$s, and everything overlaps itself, the consequent is true.

We can follow the same steps for every word in the poem.  It seems,
therefore, that poems are mereological sums.  Is this acceptable?

I can think of no principled reason to deny that poems in particular
are mereological sums, but there are good reasons to think that not
{\em everything} is a mereological sum.  This is because there are
some things that are correctly said to be parts of another thing, but
are apparently parts in a {\em different way} than are the parts of
mereological sums:

\begin{squote}
Now, on the face of it, there would appear to be a wide variety of
basic ways in which one object can be a part of another.  The letter
`n' would appear to be a part of the expression `no', for example, and
a particular pint of milk part of a particular quart; and if these two
relations of part are not themselves basic (perhaps through being
restricted to expressions or quantities), there would appear to be
basic relations of part that hold between `n' and `no' or the pint and
the quart.  It is also plausible that the way in which `n' is a part
of `no' is different from the way in which the pint is a part of the
quart.  For if the two ways were the same, then how could it be that
two pints were only capable of composing a single quart, while the two
letters `n' and `o' were capable of composing two expressions, `no'
and `on' \citep[562]{fine2010}?
\end{squote}

The parthood relation for sets is again different.  The set containing
the only the letters `n' and `o' has the letters as parts.  When the
letters are parts of a set, their order is irrelevant, but when the
letters are parts of a word, order matters; hence `no' and `on'.  The
parthood relation for sets is also different from the parthood
relation for quantities (of milk):

\begin{squote}
If four quarts compose a gallon the pints which compose the quarts
will compose the gallon in the same way in which they compose the
quarts, whereas, if four sets compose a further set the members of the
sets will not compose the further set in the same way in which they
compose the component sets.  Thus we would now appear to have three
different basic ways in which one object can be a part of another
(pint/gallon, letter/word, and member/set); and once these cases have
been granted, it is plausible that there will be many more
\citep[562]{fine2010}.
\end{squote}

Classical mereology does not appear to be capable of handling the
parthood relation that applies to sets, or that which applies to
words.  But if these things (sets, words) have parts, and if they are
not sums, then classical mereology is flawed, for it entails that
everything that has parts is a sum.

\subsection{Can mereological sums change their parts?}
\label{change}
Many philosophers believe that mereological sums cannot change their
parts.  Since many also believe that chairs and other ordinary things
{\em can} change their parts, concerns arise about the utility of
mereological sums.  If chairs and other ordinary things change their
parts, then they are not sums; what then {\em are} sums?

But not all philosophers do believe that sums cannot change their
parts.  Peter van Inwagen is one.  His argument is very
straightforward:  just as it follows from the definition of
`mereological sum' that things like poems are sums, so it follows that
things like chairs are sums.  Things like chairs can change their
parts.  Therefore, sums can change their parts.

This simple argument requires some supplementation, for there is an
(almost) equally simple argument that purports to show that sums {\em
  cannot} change their parts:

\begin{squote}
Consider an object $\alpha$ that is the mereological sum of $A$, $B$,
and $C$ (that is $\alpha = A + B + C$).  We suppose that $A$, $B$, and
$C$ are simples (that they have no proper parts), and that none of
them overlaps either of the others.  And let us suppose that nothing
{\em else} exists---that nothing exists besides $A$, $B$, $C$, $A +
B$, $B + C$, $A + C$, and $A + B + C$.  Now suppose that a little time
has passed since we supposed this, and that, during this brief
interval, $C$ has been annihilated (and that nothing has been created
{\em ex nihilo}).  Can it be that $\alpha$ still exists?  Well, here
is a complete inventory of the things that now exist: $A$, $B$, and $A
+ B$.  And $\alpha$ is none of these things, for, before the
annihilation of $C$, they existed and $\alpha$ existed and $\alpha$
was was not identical with any of them (all three of them were then
proper parts of $\alpha$).  And nothing can become identical with
something else: $x \neq y \rightarrow \square\ x \neq y$; a thing and
another thing cannot become a thing and itself.  We do not, in fact,
have to appeal to any modal principle to establish this conclusion,
for if $\alpha$ were (now) identical with, say, $A + B$, that identity
would constitute a violation of Leibniz's Law, since the object that
is both $\alpha$ and $A + B$ would both have and lack the property
``once having had $C$ as a part'' \citep[628]{inwagen2006}.
\end{squote}

Three assumptions are required for this argument to be valid.  First,
the thing that is the sum of $A$ and $B$ before $C$ is destroyed must
be the same thing that is the sum of $A$ and $B$ after $C$ is
destroyed.  That is,

\begin{squote}
If $A$ and $B$ had a unique mereological sum before the annihilation
of $C$, and if $A$ and $B$ had a unique mereological sum after the
annihilation of $C$, the object that was their sum before the
annihilation of $C$ and the object that was their sum after the
annihilation of $C$ are identical \citep[629]{inwagen2006}.
\end{squote}

This assumption seems very plausible.  If $C$ had not been destroyed,
we would have had little or no inclination to say that the sum of $A$
and $B$ at the earlier time is not identical with the sum of $A$ and
$B$ at the later time.  So I do not see why we should think that {\em
  if} $C$ is destroyed, then the sum of $A$ and $B$ at the earlier
time is not identical with the sum of $A$ and $B$ at the later time.

We can generalize this as the {\em existence assumption}:

\begin{enumerate}[ref=(\arabic*)]
  \item If there exists a sum $S$ of some things $A$ and $B$, then $S$
    exists when and only when $A$ and $B$ exist. \label{ass-ex}
\end{enumerate}

The second assumption is that composition is unrestricted (that is,
that universalism is true).  Van Inwagen escapes the conclusion that
mereological sums cannot change their parts by denying that
mereological composition is unrestricted.  He denies that for any
things, there is an object composed of them.  In the example above,
therefore, van Inwagen might deny that, before the annihilation of
$C$, there was a sum of $A + B$.  He would therefore be able to
maintain that $\alpha$ loses a part, going from $A + B + C$ to $A +
B$.  There would be no preexisting $A + B$ to compete with.  As I
argued in sections \ref{universe}--\ref{universalism}, I think
unrestricted composition---universalism---is true:

\begin{enumerate}[start=2, ref=(\arabic*)]
  \item Universalism is true. \label{ass-uni}
\end{enumerate}

The third assumption is that that there are no {\em co-located}
(completely overlapping) objects.  This assumption, along with the
principle that a thing is located where its parts are located, entails
that for any things, there is at most {\em one} thing composed of
them.  This consequence is generally referred to as {\em uniqueness}.

\begin{enumerate}[start=3, ref=(\arabic*)]
  \item There are no co-located things, and wholes are located where
    their parts are located. \label{ass-co}
\end{enumerate}

Anyone wishing to maintain that things {\em do} change their parts
must reject one of these three assumptions.  Philosophers like van
Inwagen and Merricks reject universalism (and, for good measure, the
existence assumption).  Anyone who accepts universalism will, I think,
also accept that a whole exists whenever its parts do; I can see no
reason why a universalist would deny the existence assumption.  But if
a universalist also claims that things can change their parts, she
must therefore reject the third assumption, that there are no
co-located objects.  Below I will examine three different theories
that allow for co-located objects.  If such co-location is
unacceptable, we will have to reject all three theories.

Section \ref{fine-h} will discuss Kit Fine's theory of `embodiments'.
Section \ref{fine-c} will discuss Fine's more recent theory of
composition operators.  Section \ref{hovda} will discuss Paul Hovda's
theory of temporal mereology.

Before we assess the merits of these theses, however, there is another
possibility that should be addressed.  If we assume the theory of {\em
  four-dimensionalism}, many of our problems appear to go away.
Unfortunately, new ones arise.

\section{What if we assume four-dimensionalism?}
\label{4d}
So far I have been supposing that four-dimensionalism is false.  That
is, I have assumed neither that things are composed of temporal as
well as spatial parts, nor that the past and future exist.  But if we
{\em do} assume four-dimensionalism, we have access to new solutions
to the problems relating to ordinary things.

I am using `four-dimensionalism' to refer to the conjunction of two
theories.  The first is that things have {\em temporal parts}.  The
second is {\em eternalism}.

Ted Sider presents a relatively clear picture of the doctrine of
temporal parts:

\begin{squote}
Think of your life as a long story.  Let the story be a rather
narcissistic story: cut out all details about everything else except
you.  So the story begins with an infant (or perhaps a fetus).  It
describes the infant developing into a child and then an adolescent.
The adolescent passes into young adulthood, then adulthood, middle
age, and finally old age and death.  Like all stories, this story has
parts.  We can distinguish the part of the story concerning childhood
from the part concerning adulthood.  Given enough details, there will
be parts concerning individual days, minutes, or even instants.

According to the `four-dimensionalist' conception of persons (and all
other objects that persist over time), persons are a lot like their
stories.  Just as my story has a part for my childhood, so {\em I}
have a part consisting just of my childhood.  Just as my story has a
part describing just this instant, so I have a part that is
me-at-this-very-instant \citeyearpar[1]{sider2001}.
\end{squote}

The claim that we have these {\em temporal
  parts}---me-at-this-instant, or me-as-a-child---relies on a close
analogy between space and time.  It is relatively uncontroversial to
claim that we have {\em spatial} parts.  My foot is a part of me, for
instance, but it is not {\em all} of me (it is a proper part, in
mereological terms).  A philosopher who holds this theory claims that,
likewise, my adulthood is a part of me, but it is not all of me.  My
childhood is---or was, if we do not assume eternalism---another part
of me.  My infancy, childhood, adulthood, etc. together {\em compose}
me.

This theory of temporal parts is often conjoined with a theory about
time.  This theory is commonly referred to as {\em eternalism}.
According to eternalism, ``time is like space.  There is nothing
special about the things here; things at other places are just as
real; no place is metaphysically distinguished.  Similarly, for the
eternalist, there is nothing special about the present; things at
other times are just as real; no time is metaphysically
distinguished'' \citep[122]{hinchliff1996}.  For the eternalist, there
is a sense in which ``there are dinosaurs'' is true.  Everyone agrees
that there are no dinosaurs {\em now}; the question is whether the
dinosaurs of the past still exist {\em in the past}.  

I have no firm intuition as to whether either conjunct of
four-dimensionalism is true.  I do not know whether things have
temporal parts, and I do not know if the past and future exist.  But
let us suppose for now that four-dimensionalism is true; {\em if} this
assumption is correct, we can explain the existence of ordinary things
in new and interesting ways.

\subsection{Four-dimensional essentialism}
\label{4de}
According to the standard versions of four-dimensionalism, ordinary
things like chairs and statues are {\em four-dimensional spacetime
  worms}.  They are composed of temporal parts or {\em slices}; a
chair might be made up of `chair-slices' at $t_{1}$, $t_{2}$,
$t_{3}$\,\ldots\,, etc.  These `slices' are generally supposed to have
no temporal duration.  They are {\em extended} in only three
dimensions; their temporal extension is point-sized.

Four-dimensionalism is very commonly conjoined with universalism---the
theory, defended in sections \ref{universe}--\ref{universalism}, that
for any things, there is something composed of them.  If we assume
universalism, then four-dimensionalism entails that for every set of
temporal slices, there is something composed of them.  There is an
object composed of the first ten years of my life, the Kremlin from
1970--1990, and one second of a puppy's existence in 2020.  This thing
is not, of course, a person; nor is something composed of the first 10
years of my life and the last ten years of someone else's.  Certain
causal or psychological connections must hold between the temporal
parts of a thing in order for it to be a person.

The objects composed of these temporal slices are mereological sums in
the classical sense.  Let us use `Krupkin' to designate the object
made of the first ten years of my life, the Kremlin from 1970--1990,
and one second of a puppy's existence in 2020.  Because the past and
future exist (we're assuming eternalism), Krupkin always has the same
parts.  Strictly speaking, it doesn't ever change its parts.  In 1991
it is true to say ``the Kremlin is not {\em now} part of Krupkin'',
but it is not true to say ``the Kremlin is not part of Krupkin.

If we assume universalism in addition to four-dimensionalism, then not
only does Krupkin not change its parts, it {\em cannot} change its
parts.  It cannot change its parts for the reason given in section
\ref{change}.  Let us use `Alkin' to designate the object composed of
the first 10 years of my life and the Kremlin from 1970--1990.  Now if
Krupkin could change its parts, it could lose a part.  Suppose it lost
its puppy part.  Then, if it still exists, it would be the object
composed of the first 10 years of my life and the Kremlin from
1970--1990.  But {\em that} object is Alkin; Krupkin would therefore
become identical with Alkin.  Alkin and Krupkin are not identical,
however, because Krupkin has a property that Alkin does not: the
property of having had a puppy as a part.  So Krupkin cannot, in fact,
lose a part; otherwise we would have a contradiction.

Technically, therefore, four-dimensional universalism is a version of
{\em essentialism}---the thesis that things cannot change their parts.
Saying that a thing `changes' a part just means that it has some but
not all of that part's temporal parts as parts.  For a chair to lose a
leg is for the chair to have the leg-at-$t_1$ as a part and not have
the leg-at-$t_2$ as a part.  The chair has one of the leg's temporal
parts (the leg-at-$t_1$) as a part, but not both (it does not have the
leg-at-$t_2$).

I am somewhat sympathetic to this view.  In section \ref{essential} I
will sketch an essentialist theory of things, but one that presupposes
neither temporal parts nor eternalism.  But here I will briefly
examine how a four-dimensionalist essentialism addresses the issues
related to ordinary things that we have been concerned with.

Four-dimensionalism has two advantages and two disadvantages, when
compared with the three theories above.  The first advantage is that
four-dimensionalism does not posits a plurality of {\em kinds} of
things.  The second is that it does not posit co-located objects.  The
material objects that a four-dimensionalist recognizes are all
mereological sums in the classical sense.  The first disadvantage is
that four-dimensionalism, when conjoined with universalism, produces a
plurality of objects, just as the three other theories (sections
\ref{fine-h}--\ref{hovda}) do.  The second disadvantage is that
four-dimensionalism has difficulty distinguishing objects that are
co-located for the entirety of their existence.

\subsection{Four-dimensional solutions}
\label{4ds}
The first advantage of four-dimensionalism---that it does not have to
posit a plurality of kinds of things---is primarily an advantage
relative to Fine's theory of composition operators (section
\ref{fine-c}).  That theory, as we will see, produces an incredible
plurality, not only of things in general, but of different kinds of
things.  Four-dimensional things are simply mereological sums, in the
classical sense.

The second advantage of four-dimensionalism is that, unlike the three
theories presented below, it does not posit co-located objects.  The
theories of Fine and Hovda, in order to distinguish objects like the
statue and the lump---objects that (currently) share all their
parts---had do posit co-located objects.  But on the four-dimensional
picture, this is unnecessary.  Suppose that the lump is formed on
Monday, and the statue on Tuesday.  The lump therefore has temporal
parts that are `earlier' than any of the statue's parts.  They do not
share all their parts, and so are not co-located.  It is true that
they share all their Tuesday parts; the temporal slices that compose
the lump on Tuesday are the same that compose the statue on Tuesday.
But they share parts only at certain times.  They do not share all
their parts at all times.  

This leads into a problem for four-dimensionalism, however; it does
not appear to let us differentiate a statue and a lump that {\em
  always} share their parts.

\subsection{Problems for four-dimensionalism}
\label{4dp}
But there are two disadvantages to four-dimensional universalism.  The
first is that while four-dimensionalism does not posit a plurality of
kinds of things or a plurality of co-located objects, there is still a
sense in which it is a `plurality thesis'.  Any given temporal slice
is part of a plurality of things.  When I point at my chair, I am also
pointing at a thing composed of my chair and a black bear from the
1800s, as well as a thing composed of my chair and the head of Thomas
Aquinas.  All those things (and {\em many} more) are currently located
in the very same place.

This is certainly bizarre, and it makes four-dimensionalism somewhat
unpalatable, but it does not {\em disproves} four-dimensionalism.
Unfortunately there is another disadvantage to four-dimensionalism,
one that does threaten it as a theory.

The second disadvantage of four-dimensionalism is that it has trouble
distinguishing between objects that are co-located for the entirety of
their existence.  Suppose that I have two lumps of clay; I form one
into the top half of a figure and I shape the other into the bottom
half.  Having done this, I stick the two pieces of clay together,
forming a statue.  When I do this I also form a new, larger lump of
clay.  I admire the statue and the lump for a little while, then smash
them with a hammer.

Let $S$ be the thing composed of all the statue-slices.  Let $L$ be
the thing composed of all the larger-lump-slices.  $S$ and $L$ are
mereological sums composed of the very same parts; $S = L$.  But if
the statue had been squashed instead of smashed, $L$ would have
survived; but $S$ would not have survived being squashed.  $L$ has a
property that $S$ does not---the property `could survive being
squashed'---and therefore $S \neq L$.  This is a problem.

The four-dimensionalist could say that the lump would not have
survived being squashed, or that the statue would have survived.
Since there is only one thing under investigation (since $S = L$),
that thing must have a consistent set of properties.  It can't be such
that it would both survive and not survive a squashing.  So the
four-dimensionalist will have to say that one of our two intuitions is
wrong.

But there is another, related, difficulty.  In the case just
presented, the statue is the lump ($S = L$).  But suppose there is a
situation exactly like the one presented, but in which the statue and
lump are first squashed, then smashed.  In this case, we are inclined
to say that the lump $L^{\prime}$ continues to exist after the
squashing.  Its parts include temporal slices of the clay after it has
been squashed.  In the case of the statue $S^{\prime}$, however, we
are inclined to say that the statue does not have any temporal parts
after the squashing.  The statue is destroyed when it is squashed.
Since $L^{\prime}$ and $S^{\prime}$ have different parts, they are not
the same thing; $L^{\prime} \neq S^{\prime}$.  The four-dimensionalist
is committed to the claim that whether there is one thing (a statue
that is also a lump) or two things (a statue and a lump) on the table,
and what that thing's (or those things') modal properties are depends
upon whether I squash or smash it.  This seems highly implausible.  By
choosing to squash the statue rather than smash it, do I thereby {\em
  make it the case} that there were two things, rather than one?

The four-dimensionalist will object that, since the future already
exists, it was {\em already} true that there were two things (it has
always been true).  But claiming that it is already the case that I
will squash the statue seems to commit the four-dimensionalist to some
version of {\em determinism}---the thesis that, roughly, the events of
the future are determined, or fixed, to occur.  This may well be true,
but it is largely an empirical hypothesis; to rely on it here would be
unwise.  (If the four-dimensionalist does not assume determinism, and
instead assumes indeterminism, they will presumably have to say that,
since it is indeterminate whether or not I will squash the statue, the
number of things on the table is therefore also indeterminate.  This
seems even worse.)\\

Four-dimensionalism allows for the resolution of a number of puzzles
related to ordinary things.  It does not resolve everything, however,
and it introduces a few problems of its own.  Moreover, it requires a
number of controversial assumptions: the theory of temporal parts,
eternalism, and possibly determinism.  I will therefore set aside
four-dimensionalism, and suppose henceforth that the past and future
do not exist, and that things do not have temporal parts.

In the section that follows, I will examine the first of two theories
by Kit Fine that attempt to explain how things can change their parts.

\section{First theory: rigid and variable embodiments}
\label{fine-h}
An outline of Kit Fine's `hylomorphic' theory is presented in his
paper ``Things and their parts'' \citeyearpar{fine1999}.  His
objective in this paper is to present a satisfactory account of how
things can change their parts over time.

In section \ref{change} I explained why, given three assumptions,
mereological sums do not change their parts.  One of those assumptions
is that a mereological sum exists whenever the things that compose it
exist.  The mereological sum of $a, b, c$ exists whenever (and
wherever) $a, b, c$ exist.

Now, however, there are some things that do not seem to obey this
assumption.  Take a ham sandwich, for example.  It has two slices of
bread and a piece of ham as parts.  It seems to fit the definition of
a mereological sum.  But

\begin{squote}
the sum $a + b + c + \mathellipsis $ will exist {\em whenever} any of
its components $a, b, c, \mathellipsis $ exists (just as it is
located, at any time, {\em wherever} any of its components are
located).  It follows that, under the proposed analysis of the ham
sandwich, it will exist as soon as the piece of ham or either slice of
bread exists.  Yet surely this is not so.  Surely the ham sandwich
will not exist until the ham is actually placed between the two slices
of bread.  After all, one {\em makes} a ham sandwich; and to make
something is to bring into existence something that formerly did not
exist \citep[62]{fine1999}.
\end{squote}

If it is true that the sandwich comes into existence only when the
bread and meat are put together, then the sandwich cannot be a
mereological sum in the classical sense.  How, then, is it composed?

\subsection{Composition relations}
\label{rigid}
Fine's suggestion is that things like the sandwich be seen not merely
as the sum of the bread and meat, but as an object composed of the
bread and the meat {\em standing in the relation of `betweenness'}.
Likewise, a bunch of flowers is not merely the sum of the individual
flowers, but as an object composed of the flowers {\em in the relation
  of being bunched}:

\begin{squote}
Given objects $a, b, c, \mathellipsis $ and given a relation $R$ that
may hold or fail to hold of those objects at any given time, we
suppose that there is a new object---what one may call ``the objects
$a, b, c, \mathellipsis $ in the relation $R$.''  So, for example,
given some flowers and given the relation of being bunched, there will
be a new object---the flowers in the relation of being bunched (what
might ordinarily be called a ``bunch of flowers'')
\citeyearpar[65]{fine1999}.
\end{squote}

Fine can be understood here to be modifying our existence
assumption---the assumption \ref{ass-ex} that a sum exists whenever
its parts do.  Instead, something composed of certain objects and a
relation---a composite object that Fine calls a {\em rigid
  embodiment}---exists whenever its parts stand in the given relation.

But rigid embodiments cannot change their parts.  The sandwich is {\em
  destroyed} when its parts fail to stand in the correct relation.
Fine must introduce another kind of thing---a {\em variable
  embodiment}---in order to make it possible for things to change
their parts.  In doing so, he allows for (very many) co-located
objects.

\subsection{How things change their parts}
\label{h-part}
Given certain assumptions, classical mereological sums cannot change
their parts.  Given the same assumptions, rigid embodiments cannot
change their parts either.

Fine stipulates that a thing $x$ composed of $a, b, c$ in relation $R$
exists at a time $t$ if and only if $R$ holds of $a, b, c$ at $t$.  If
$x$ exists at $t_1$, it is because $a, b, c$ are in $R$ at that time.
If at $t_2$, $a, b, c$ are not in $R$---say that only $b, c$ are in
that relation---then $x$ does not exist.  This is the analogue of our
assumption \ref{ass-ex} that a sum exists whenever its parts do.

If our assumption \ref{ass-uni} of universalism holds here, then for
any things ($z$s) in a relation $R$, there is an object composed of
the $z$s in that relation.  Suppose, as above, that there is an object
$x$ composed of $a, b, c$ in relation $R$.  If $a, b$ alone also stand
in $R$, then, {\em if} composition is unrestricted, there is also an
object $y$ composed of $a, b$ in relation $R$.  Objects $x$ and $y$
have different parts and are therefore different things.

If our assumption \ref{ass-co} holds, then there is at most one thing
composed of $a, b$ in relation $R$.

Now suppose $c$ is destroyed or somehow no longer stands in $R$ with
$a, b$.  If we assume that composition is unrestricted and that the
object composed of $a, b$ in $R$ before $c$ is destroyed is identical
with the object composed of $a, b$ in $R$ after $c$ is destroyed, then
we cannot say that $x$ has lost a part and is now composed of $a, b$
in $R$.  There is already an object composed of $a, b$ in $R$---the
object $y$.  If we said that $x$ has lost a part, we would be
committed to the claim that $x = y$, even though previously $x \neq
y$.  If $y$ exists, then we must say that $x$ ceases to exist when it
loses a part ($c$).

(Here I am assuming that relations like $R$ are {\em not} fixed
polyadic relations.  That is, there is not one relation $R$ that can
apply to three things---the schema being $Rxyz$---and a different
relation $R^{\prime}$ that can apply to two things---$Rxy$.  Rather, I
am assuming that relations like $R$ have a single variable `slot' that
can accommodate {\em plural variables}.  The schema is something like
$Rx$s, where $x$s is a plural variable that can designate any number
of things.  Therefore it is the {\em same} relation $R$ that applies
to $a$, $b$ and to $a$, $b$, $c$.)

Therefore Fine has a separate proposal for objects that can change
their parts.  These things Fine calls {\em variable embodiments}.
Variable embodiments have, at different times, different {\em rigid}
embodiments as parts.  What part a variable embodiment has at a given
time is determined by a function that assigns rigid embodiments to
times.  Fine illustrates this with the water of a river.  There is the
quantity of water that currently composes the river, but there is also
the `variable' water, that consists of different quantities of water
at different times:

\begin{squote}
I take it that the water in the river in the second sense---what we may
call the variable water---is now constituted by one quantity of water
and now by another. But what is the variable water?\,\ldots

I would like to take the bold step of supposing that there is here a
hitherto unrecognized method by which wholes may be formed from parts.
In the case of the variable water, there is a function, or
``principle,'' that determines which quantity of water constitutes the
variable water at any given time \citeyearpar[68]{fine1999}.
\end{squote}

In effect, {\em the water} of the river---the thing that is the
variable embodiment---is composed of other things---rigid embodiments
that are in turn composed of water molecules.  The water molecules are
not `directly' part of {\em the water}, but they are parts of its
parts.

\subsection{Problems with the first theory}
\label{problems1}
There are two problems with this theory.  First, it has the
consequence that relations (like `being bunched') are actually {\em
  parts} of things (the relation of being bunched is part of the bunch
of flowers).  Second, it produces a plurality of co-located objects.

It is certainly not true that a relation is part of a bunch of flowers
in the same way that the flowers are part of the bunch.  Fine
recognizes this; it constitutes one of his objections to a possible
extension of classical mereology.  He observes that one could claim
that mereological sums are made up of things like bread and meat as
well as {\em tropes}, or relations.  But

\begin{squote}
even if we grant that the trope is a part of the sandwich, it is hard
to believe that it is a part in the same way as the standard
ingredients.  Thus we should not regard the sandwich as a
straightforward mereological sum of $s_1$, $s_2$, $h$, and $r$, but in
some other way that has yet to be made clear \citep[64]{fine1999}.
\end{squote}

Fine's theory of embodiments recognizes relations as parts of things,
but in a different way than things like slices of bread are parts of
things.  This is suggested by his notation for a rigid embodiment of
$a$, $b$, and $c$ in relation $R$: $a, b, c / R$.  But this does not
explain in {\em what} way relations are parts of things.  Moreover, it
just seems false that the relation of being bunched {\em is} a part of
the bunch of flowers in any way.  The relation {\em holds} of the
flowers, and it explains why the flowers are a bunch, but that does
not convince me that the relation is in fact part of the bunch.  In
section \ref{all-sum} we saw examples of many different kinds of
things that have many different kinds of parts.  Tennis matches have
sets, sets have members, poems have stanzas, stanzas have lines, lines
have words.  But {\em relations} were not included in this catalog of
parts.  A theory that has the consequence that relations are parts is,
at least, unintuitive.  (Fine's theory can be modified to avoid this
consequence, as we will see in section \ref{fine-c}.)

The second problem with Fine's theory is one that will plague all
three theories: it posits a plurality of co-located objects, violating
our assumption \ref{ass-co} of non-co-location.  Every variable
embodiment is composed at different times of different rigid
embodiments.  At any given time, therefore, a variable embodiment and
the rigid embodiment that composes it at that time occupy the very
same location.

(Note that because Fine claims that rigid and variable embodiments are
different {\em kinds} of things, he is not required to deny {\em
  uniqueness}.  The parts of the rigid embodiments (including itself)
are parts of it in a different {\em way} than the parts of the
variable embodiment are part of it.  It remains true that for any
parts, there is one whole composed of them, but Fine considers this
statement ambiguous: there are at least two different {\em senses} of
`part' and `compose' that might be meant here, corresponding to the
rigid and variable notions.)

Fine illustrates how his theory entails that even people are
co-located with many other things:

\begin{squote}
An especially important class of cases are those in which the
principle of embodiment is a property $P$ rather than a polyadic
relation $R$.  The rigid embodiment is then of the form ``$a/P$'' and
may be read as ``$a$ qua $P$'' or as ``$a$ under the description
$P$.''  An airline passenger, for example, is not the same as the
person who is the passenger since, in counting the passengers who pass
through an airport on a given weekend, we may legitimately count the
same person several times.  This therefore suggests that we should
take an airline passenger to be someone under the description of being
flown on such and such a flight.  And similarly for mayors and judges
and other ``personages'' of this sort \citeyearpar[67--68]{fine1999}.
\end{squote}

One might take this to be an unacceptable consequence of Fine's
theory.  For persons can think, and airline passengers can think as
well.  Are we therefore being asked to accept that there are at least
{\em two} thinking things in every seat on the airplane?

This objection, however, comes from confusing rigid and variable
embodiments.  Rigid embodiments, like the person-as-passenger, cannot
change their parts.  As soon as the person-as-passenger loses {\em
  any} of its parts, it ceases to exist.  I think Fine would say that
rigid embodiments, because they cannot undergo change, cannot properly
be said to think.  Things that {\em do} think are variable
embodiments; for example, the human person that at one time is
composed of some of the same parts as the person-as-passenger (but not
all of the same parts, for the person-as-passenger has a relation as a
part).  If it is only variable embodiments that can think, then a
variable embodiment overlapped by one or more rigid embodiments cannot
result in co-located thinkers.

Moreover, if functions are identified by their assignments of things
to times---that is, extensionally---then there may not be {\em always}
co-located variable embodiments.  If there are no two functions that
assign the very same things to the very same times, then there can be
no co-located thinkers.  (But pluralities of variable embodiments will
overlap at any given time.)

However, even if there cannot be co-located thinkers---thinkers who
completely overlap---why can't there be partially overlapping
thinkers?  For example, Fine's theory may well predict the existence
of a variable embodiment that is composed of the various rigid
embodiments of Alex-as-passenger during a particular flight.  (That
is, since I change some of my parts during a flight, there are a
number of different rigid embodiments that may be called
Alex-as-passenger.  Then the question is whether there is a variable
embodiment composed of each of these rigid embodiments in turn.)  If
there is such a variable embodiment, why shouldn't we expect {\em it}
to think?

I think Fine will have to simply deny that such a thing could think.
There are a number of reasons that may be appealed to: the thing does
not have the right sort of history (it is at best a `restriction' of
me---the real thinking thing), or there is a better candidate (me) for
being the one and only thinking thing in that location.  (But I think
as a result of the functioning of my brain; my brain is also part of
the passenger.  How can only one of us think?  I will raise this
objection against Hovda as well in section \ref{problems3}.)

In any case, it seems simply bizarre that by boarding an airplane I
thereby cause a new thing to come into existence.  If I become a
judge, then according to Fine, a new {\em thing} has come into
existence.  Why not just say that a description is true of me that was
once not true of me?  For

\begin{squote}
suppose that Mary got married at noon.  Her marrying did not make a
wife come into existence: it merely made her become a wife.  Your
reaching the age of 20 did not make a teenager go out of existence; it
merely made you cease to be a teenager.  And so on
\citep[151]{thomson1998a}.
\end{squote}

The consequence of Fine's theory that passengers are things distinct
from people, coupled with the `explosion of reality' that occurs
simultaneously, is cause for concern.  I have in fact understated the
size of the explosion, for in addition to the pluralities of
co-located rigid embodiments, there is likely also a plurality of
variable embodiments, each corresponding to a possible function.

But these consequences are not limited to our first theory.  The
second theory, as we will see, results in a similar explosion.

\section{Second theory: composition operators}
\label{fine-c}
Recently Kit Fine has proposed a new analysis of things.  In ``Toward
a theory of part'' \citeyearpar{fine2010}, he suggests that not only
are there a plurality of mereological sums, but that there is a
plurality of {\em kinds of things}; sums are only one kind in a vast
``mereological firmament''.  Fine's theory is extremely interesting,
but ultimately it faces a particularly acute version of the problem of
co-location that faces the other two theories.  For while Fine's
theory of embodiments and Hovda's theory of tensed mereology (section
\ref{hovda}) predict a plurality of overlapping things, Fine's theory
of composition operators predicts, in addition, a plurality of {\em
  kinds} of things.

Fine's new theory has a number of connections with his theory of {\em
  rigid embodiments} (see section \ref{rigid}).  That theory posited
{\em relations} as parts of things, but as parts in a different {\em
  way}.  The relation of being bunched was supposed to be part of the
bunch of flowers, but in a different way than the flowers themselves
are part of the bunch.  But it was not explained {\em how} something
can have different parts in different ways.  

One might suppose that if we reject Fine's theory of embodiments, we
can reject this `pluralist' conception of parthood.  But as we saw in
section \ref{all-sum}, there are independent grounds for thinking that
there are different ways of being a part.  The way that letters are
parts of words is different from the way members are parts of sets,
and both are different from the way things are parts of sums.  Fine's
new theory begins by defending this `pluralist' claim about parthood.

\subsection{Problems for pluralists}
\label{sets}
There are a number of objections to Fine's pluralism about parthood.
The first objection is that while parthood is supposed to be
transitive, the membership relation of sets is not.  The letter `n' is
a member of the set \{`n',\{`n',`o'\}\}, but `o' is not.  The
objection claims that sets have {\em members}, not parts, and that
Fine has confused the two.

But while it is true that the membership relation is not the parthood
relation, this is no reason to think that sets do not have parts.  A
given set will have certain members---the $x$s---and certain
parts---the $y$s---and only sometimes will the $x$s and the $y$s be
the very same things.  The set \{`n',\{`n',`o'\}\} has two members
but three parts.  The parthood relation for sets can even be defined
in set-theoretic terms:

\begin{squote}
It may well be thought that the way in which a member is a part of a
set is given, not by the membership relation itself, but by the
ancestral of the membership relation, where this is the relation that
holds between $x$ and $y$ when $x$ is a member of $y$ or a member of a
member of $y$ or a member of a member of a member of $y$, and so on
\citep[563]{fine2010}.
\end{squote}

A second objection is that talk of parthood in connection with things
like sets is somehow metaphorical or non-literal.  We saw above that
van Inwagen admits that many different things are said to have parts.
However, he qualifies this in two ways.  First, he seems to have
doubts (or at least is sympathetic with those who have doubts) as to
whether the non-material things that are said to have parts really
exist:

\begin{squote}
The word `part' is applied to things that are clearly \emph{not}
material objects---or at least it is on the assumption that these
things really exist and that apparent reference to them is not a mere
manner of speaking \citep[19]{inwagen1995}.
\end{squote}

If there are no such things as tennis matches or poems or papers, then
of course they do not have parts.  But I think it is obviously true
that there are such things.  This being so, what does it mean to say
that they have parts?  This is where van Inwagen's second
qualification comes in.  For he suggests not only that the `parts' of
tennis matches and poems are parts in a different way than are the
parts of a table, but that these different relations of parthood are
only tenuously connected.  Van Inwagen says that the various relations
of parthood (if such there be) are connected only by the ``unity of
analogy'' \citeyearpar[19]{inwagen1995}.  If the only similarity
between the parthood relation for poems and the parthood relation for
chairs is that they share the `analogy' of parthood, then is there
anything important or interesting about `parts' of poems?  Is the
parthood relation for sets likewise only interesting because of the
analogy with the parthood relation for chairs?

At least in the case of parthood for sets, the notion does not appear
to be wholly metaphorical:

\begin{squote}
In the case of set-membership, there would appear to be nothing that
might plausibly be taken to indicate that the talk of part-whole is
not to be taken literally. A set is indeed composed of or built up
from its members, and we should add that we may meaningfully
talk---and in the intended way---of \emph{replacing} one member of a
set with another.  Thus Aristotle in the set \{Plato, Aristotle\} may
be replaced with Socrates to obtain the set \{Plato, Socrates\}, with
the given set becoming a different set from what it was. In the case
of sets, our conception of members as parts seems to extend all the
way \citep[564]{fine2010}.
\end{squote}

But the second worry raised by van Inwagen remains.  Why should we
think that there is any {\em real} similarity between these different
parthood relations, other than the fact that we call them all
`parthood'?

\subsection{Operationalism}
\label{operation}
Fine's theory of {\em operationalism} helps answer this worry.
Various {\em operations} produce different things---mereological
summation produces mereological sums or fusions, the set-builder
produces sets, and so forth.  Parts are therefore {\em things} that
have been `combined', through one or more such operations, into a
single {\em thing}.  What is common to all parthood relations is that
from each set of parts is produced a {\em whole} by means of a
composition operator.  From parts (letters, atoms) are made something
else (a word, a set, a chair).  What ties together all the ways of
being a part is that they are involved in a composition operation that
produces a single thing from a number of things:

\begin{squote}
In formulating the principles of mereology, it has been usual to take
the relation of part-whole or some associated relation (such as
overlap) as primitive.  But I believe that, in formulating a more
general theory, it is important to take the operation of composition
as primitive rather than the more familiar relation of part-whole.  In
the case of classical mereology, the operation of composition will
take some objects into the sum or fusion of those objects, while, in
the set-theoretic case, it will take some objects into the set of
those objects; and, in general, the operation of composition will be
the characteristic means (summation, set-builder, and so on) by which
a given kind of whole is formed from its parts \citep[565]{fine2010}.
\end{squote}

Each way of being a part can then be defined in terms of the related
composition operation:

\begin{squote}
Once given a compositional operation, a corresponding relation of part
may be defined in two steps.  We say first that $x$ is a {\em
  component} of $y$ if $y$ is the result of applying $\sum$ to $x$ or
to $x$ and some other objects.  In other words, $y$ should be of the
form $\sum (x_{1}, x_{2}, \mathellipsis )$, where at least one of
$x_1$, $x_2, \mathellipsis$ is $x$.  Thus when $\sum$ is mereological
summation the components of an object will be mere parts, and where
$\sum$ is the set-builder the components of an object will be its
members.  We may then define $x$ to be a part of $y$ if there is a
sequence of objects $x_1$, $x_2, \mathellipsis x_n$, $n$
\textgreater{} $0$, for which $x = x_1$, $y = x_n$, and $x_i$ is a
component of $x_{i+1}$ for $i = 1$, $2, \mathellipsis, n-1$. The parts
of an object are the object itself, or its components, or the
components of the components, and so on \citep[567--568]{fine2010}.
\end{squote}

The parthood relation for mereological sums can therefore be shown to
exhibit reflexivity, transitivity and anti-symmetry
\citep[568]{fine2010}:

\begin{description}
\item[Reflexivity] Each object is a part of itself.
\item[Transitivity] If $x$ is a part of $y$ and $y$ of $z$, then $x$
  is a part of $z$.
\item[Anti-symmetry] $x$ is a part of $y$ and $y$ of $x$ only when $x
  = y$.
\end{description}

But not all definitions of parthood that issue from a composition
operator will exhibit these features:

\begin{squote}
When the underlying operation is summation, each object will be a part
of itself, since the unit sum of any object is the object itself, but
when the underlying operation is the set-builder, no object will be a
part of itself, since no object is ever an ancestral member of itself
\citep[569]{fine2010}.
\end{squote}

In every case, how some thing is part of a whole (if it is) will
depend on the composition operation that produced the whole.  Other
properties, both of a whole and its parts, will be determined by the
nature of the composition operator that produced it.  Each composition
operation will, according to Fine, be governed by various principles.
The `formal principles' govern when composition occurs and when two
products of a composition operation are identical.  The `material
principles' govern both how the object `sits' in space and
time---whether it has spatial and/or temporal parts (see section
\ref{4d}) or not---and the specific characteristics of the object
(such as its color and weight).

\subsection{Fine's pluralist account of classical mereology}
\label{classical}
Of the principles sketched above, Fine gives most attention to the
identity conditions for composition operations.  The composition
operation used as a paradigm is the summation operation of classical
mereology.  Fine's exposition of identity conditions for sums relies
on the notion of `regularity':

\begin{squote}
Call an identity condition $s = t$ {\em regular} if the variables
appearing in $s$ and in $t$ are the same.  Thus $\sum (x, y) = \sum
(y, x)$ is regular while $\sum (x, y) = x$ is not
\citeyearpar[572]{fine2010}.
\end{squote}

With this notion in hand, Fine proposes this condition for identity of
sums:

\begin{description}
  \item[Summative Identity] $s = t$ whenever `$s = t$' is a regular
    identity \citeyearpar[572]{fine2010}.
\end{description}

One particularly interesting aspect of this condition is that it
entails four more principles of the summation operation:

\begin{description}
  \item[Absorption] $\sum (\mathellipsis, x, x, \mathellipsis,
    \mathellipsis, y, y, \mathellipsis, \mathellipsis = \sum (
    \mathellipsis, x, \mathellipsis, y, \mathellipsis )$;
\item[Collapse] $\sum (x) = x$;
\item[Leveling] $\sum (\mathellipsis, \sum (x, y, z, \mathellipsis ),
  \mathellipsis, \sum (u, v, w, \mathellipsis ), \mathellipsis ) \\ =
  \sum (\mathellipsis, x, y, z, \mathellipsis, \mathellipsis, u, v, w,
  \mathellipsis, \mathellipsis )$;
\item[Permutation] $\sum (x, y, z, \mathellipsis ) = \sum (y, z, x,
  \mathellipsis )$ (and similarly for all other permutations)
  \citep[573]{fine2010}.
\end{description}

We can define other compositional identity criteria (e.g., sequences)
in terms of which of these principles apply to their compositional
operation.  But we may also devise new principles by which we may then
define new types of composition:

\begin{squote}
We should note that there would appear to be no good reason to require
that the defining principles for the various operations should be
limited to the particular principles (C [collapse], L [leveling], A
[absorption], and P [permutation]) that we used in characterizing
sums; for any set of regular identities would appear to be equally
well suited to defining a basic form of composition, so long as they
conform to Anti-cyclicity.  Indeed, I would conjecture that any such
set of principles in fact will correspond to a form of composition and
a corresponding form of whole.  How the resulting forms of composition
and whole might be organized is an interesting question, but it should
be apparent that the approach will lead to an infinitude of forms of
composition, each differing from one another in how exactly the
identity of the resulting wholes is to be
determined. \citep[575--576]{fine2010}.
\end{squote}

It is at this point that the importance of Fine's theory becomes
obvious.  Above I stressed that things like teams and families are
really {\em things}; moreover I made this claim as part of an attempt
to motivate a sort of universalistic outlook on metaphysics.  I argued
that the term `composition' was potentially misleading, but that it
was nevertheless correct to say that things like dogbushes, wish
sandwiches, and teams are composed of their parts.  But now it is
apparent that `composition' will mean something different when applied
to each of these things.  Each thing will be the product of a
different composition operation.

Fine's theory reveals new {\em kinds} of universalism.  One might be
committed to the existence of dogbushes---and so to unrestricted
mereological composition---but deny the existence of some other kind
of thing (for example, groups---see section \ref{groups}).  Or one
might defend unrestricted composition of other kinds of things while
claiming a restriction on mereological composition.

\subsection{Composition operators and time}
\label{c-change}
In Fine's theory of embodiments (section \ref{fine-h}) he recognizes
at least two kinds of things: rigid and variable embodiments.  Rigid
embodiments have their parts `timelessly'.  They exist when and only
when their parts exist, and at all times during which they exist, they
have the same parts.  Rigid embodiments, therefore, cannot change
their parts.  Variable embodiments {\em can} change their parts,
however; what rigid embodiment a given variable embodiment is composed
of at a given time is determined by a function.

Fine's account of composition operators explains how the create things
that, like rigid embodiments, do not (and presumably cannot) change
their parts.  He does not address how composition operators might
produce things that, like variable embodiments, {\em can} change their
parts over time; he opens his paper on composition operators by saying
that ``it is not [his] aim to discuss either the notion of relative
part or its connection with the absolute notion''
\citeyearpar[559]{fine2010}.  However, I think we can imagine a few
ways in which Fine's theory of composition operators might be adapted
to relative or temporary parthood.

One way to adapt Fine's new theory so as to allow things to change
their parts would be to think of the mereological sum operator (the
sum-builder) as operating {\em not} on ordinary things but on
things-at-times.  By `things-at-times' I mean {\em temporal slices} of
things.  For example, a temporal slice of a chair is an object that
resembles a chair but has no temporal duration.  The sum-builder for a
chair would take such temporal slices and compose from them a chair.
The chair would have temporal duration and would be capable of
changing its parts.  (As in the four-dimensional picture presented in
section \ref{4de}, for a thing to gain or lose some part $x$ would be
analyzed as: having some but not all of $x$'s temporal parts as
parts.)

One problem with this proposal is that it presupposes {\em temporal
  parts} (see section \ref{4d}).  For it seems that composition
operators like the sum-builder operate on {\em things}.  If
sum-builder can operate on things-at-times, then we commit ourselves
to the claim that things-at-times are {\em things}.  And what things
could they be but temporal parts of other things?

If we don't want to presuppose temporal parts, the sum-builder has to
be somehow \emph{dynamic}. It can't just take things, compose them and
be done---it has to \emph{add and remove things over time}.

Making sense of a dynamic operator might allow us to avoid
presupposing {\em eternalism} as well.  If the sum-builder composes a
chair `in one go' out of different temporal slices, then the future
slices would have to already exist in some sense.  How else could the
sum-builder operate on them?

One way to make sense of a dynamic operator is by relativizing the
sum-builder to times.  We can think of the operator as taking some
things at a time and producing a sum: $G = \sum _{t} (S)$.  (There are
two interpretations of $\sum _{t}$: we might say that the composition
operator (re-)produces a sum at a number of different times $t$, or we
might say that there is a {\em different} composition operator at each
time $t$.  I will suppose that the former is correct.)

A second way to make sense of a dynamic operator is the way that Fine
makes sense of variable embodiments.  Variable embodiments were
composed of different things at different times according to a
function.  Likewise, a composition operator that produces a thing that
has different parts might do so by means of a function.  Rather than
operating directly on some things, the operator could apply to a
function.  Instead of

\begin{displaymath}
\sum (a, b, c, \mathellipsis )
\end{displaymath}
we would have something like this:

\begin{displaymath}
\sum ( f )
\end{displaymath}

On this understanding of a `dynamic operator', the only {\em
  component} (see section \ref{operation}) of the object is the
function, but at any given time it has as parts (in some sense)
whatever objects the function assigns to that time.

On either understanding of the `dynamic operator', things can change
their parts, but things can (and will) also be co-located.

\subsection{Problems with temporally relativized operators}
\label{problems2a}
Suppose we take the first suggestion and relativize the sum-builder to
a time.  The primary problem with this is that it leads to a great
plurality of co-located objects.  What is particularly objectionable
in this case is that the objects are all of different {\em kinds}.

To see why this is so, let us suppose that ordinary things like chairs
and statues are produced by means of the temporally relativized
sum-builder $\sum_{s_t}$.  A statue might then be produced thus:

\begin{displaymath}
ST = \sum_{s_t} (a, b, c, \mathellipsis )
\end{displaymath}

Now since we are assuming that universalism---assumption
\ref{ass-uni}---is true, there is an object composed of all the parts
of the statue except for the left hand:

\begin{displaymath}
LF = \sum_{s_t} (a, b, \mathellipsis )
\end{displaymath}

At $t$, these are obviously different things.  Since $ST$ and $LF$
have different parts, $ST \neq LF$.  But now suppose the statue
changes its parts---by losing its left hand---while the lump of clay
$LF$ remains the same.  We will have these two objects:

\begin{displaymath}
ST = \sum_{s_t} (a, b, c, \mathellipsis )
\end{displaymath}

\begin{displaymath}
LF = \sum_{s_t} (a, b, \mathellipsis )
\end{displaymath}

But now we are committed to it being the case that $ST = LF$.
Identity is not a temporary or contingent relation.  If any two things
are actually the same thing, they are necessarily so.  That is, $ST =
LF \rightarrow \square ST = LF$.  It cannot ever be the case that $ST
\neq LF$.  But at $t$, this was apparently so.

The way to avoid this contradiction is to deny that the statue and the
lump are produced by the same composition operator.  The statue must
be seen to be the product of the `statue-builder'---$\sum
_{st_t}$---and the lump the `lump-builder'---$\sum _{lump_t}$.

But the products of different operators are of different {\em kinds}.
The statue and the lump, therefore, are different kinds of things---to
say that both are `physical objects' or `ordinary things' is simply to
bring two heterogeneous kinds under one label.

Thus, in the same fashion as Fine's theory of embodiments, we find
ourselves rejecting non-co-location---assumption
\ref{ass-co}---without denying uniqueness.  Because things that are
co-located (like the statue and lump) are different kinds of things,
they have their parts in different ways.  When we say ``these things
are parts of the statue'' and ``these things are parts of the lump'',
we are using `parts' differently in each case.

But just as with Fine's previous theory, the theory of composition
operators creates an `explosion of reality', with the additional
strangeness of a plurality of {\em kinds} of things.  Since we have
allowed that the statue and the lump may be different things composed
of the same sums, why stop there?  We can introduce more composition
operators that produce distinct objects.  Where we see a statue and a
lump, why not suppose that there is a great plurality of objects, each
of a different kind and with slightly different properties?

This is not a particularly attractive position, but it is not
indefensible (see \citet[section 4]{bennett2004}).  Since we have
already allowed a plurality of scattered objects like archipelagos and
dogbushes, why not allow a plurality of co-located objects?

One additional difficulty for this theory is that it is unclear how
Fine would avoid there being co-located thinkers.  When discussing
Fine's theory of embodiments (section \ref{fine-h}) we saw that he has
to claim that `qua-objects' like airline passengers
(people-as-passengers) don't actually think, but that it is
nonetheless correct to say that passengers think.

Fine's theory of operators may have to include a similar clause.  If
the theory is correct, then there will no doubt be many things
composed of the same atoms that compose me, but none of them will
think.  Only I will be thinking.  Fine needs an explanation both of
why there can only be one thinking thing composed of any given
parts---why only one `thinker-builder' can apply to some things---and
of what the `thinker-builder' is.  What builds me?

\subsection{Problems with functional operators}
\label{problems2b}
The second way that I suggested we make sense of a `dynamic operator'
was to understand it as applying not to things but to a single
function:

\begin{displaymath}
G = \sum ( f )
\end{displaymath}

The function assigns certain things to certain times, so to determine
what is part of $G$ at a given time, we appeal to the function: what
thing or things does the function assign to that time?  In Fine's
theory of embodiments (section \ref{fine-h}), the things that composed
a variable embodiment $V$ at a given time was the rigid embodiment
that is determined by $V$'s function.  Non-dynamic operators (section
\ref{operation}) produce things that do not change their parts, much
as rigid embodiments do not change their parts.  The function of a
dynamic operator might therefore assign products of non-dynamic
operations to times, just as the functions of variable embodiments
assign rigid embodiments to times.

The major advantage of this kind of dynamic operator is that it does
not result in a plurality of different kinds of things.  The statue
and the lump can now be built from the same composition operator.
That operator, in producing them, will of course be operating on
different functions; the `statue function' will assign different
pieces of clay to different times than will the `lump function'.
Since the operator $\sum$ will be operating on two different functions
(rather than on the same objects), it will produce two different
things.

This version of the dynamic operator also blocks the entailment from
co-location to uniqueness.  The statue and the lump may be in the same
place at the same time, but they do not have all the same parts.  The
statue has the statue-function as a part, and the lump has the
lump-function as a part.  (If these functions were not parts, there
would be no way to distinguish the statue and the lump.)  This version
of the dynamic operator therefore treats functions as parts of things,
in much the same way as Fine's theory of embodiments treats relations
as parts of things.  Just as it seems false to say that relations are
parts of things (in any sense), so it seems false that functions are
parts of things.

Moreover, this version of the dynamic operator still leaves us with a
great plurality of objects.  Indeed, there seems no principled limit
to the functions that might, through an application of a composition
operator, give rise to new things.  If there is a `statue function'
that assigns pieces of clay to times for every moment at which the
statue in question exists, then it seems arbitrary to say that there
isn't a function that is identical but for its beginning 10 minutes
later.  If we apply the same operator to {\em that} function, do we
get another thing that overlaps the statue for most of its existence?
What about a function that leaves off the first 10.1 minutes?  What
about a function that assigns Tacitus to 100 AD and my cat to today?
Is there a thing that existed momentarily last year, then exists for
all of today, then ceases to be?

Another difficulty with relying on functions is that it may also
assume eternalism.  If the function seeks to assign objects to past or
future times, we may be thereby committed to the existence of those
past and future objects and times.\\

Whether we construe the dynamic operator as relative to time or
operating on functions, we get an absurd number of co-located things
of different kinds.  Of course, in addition to this plurality of
`dynamic' objects, there is {\em also} a plurality of `static'
objects.  This is analogous to the result of Fine's theory of
embodiments, with pluralities of both rigid and variable embodiments.

There are enough problems with Fine's theory to encourage us to look
for something better.  The third theory I will examine has a number of
similarities to those of Fine (especially his theory of embodiments),
but it has some advantages as well.  Unlike previous theories, Hovda's
theory of tensed mereology does not have the consequence that
relations or functions are parts of things.  Nor does not posit a
large number of different kinds of things.  But it does nonetheless
posit what I take to be an objectionable number of overlapping things.

\section{Third theory: tensed mereology}
\label{hovda}
Paul Hovda has proposed an amended version of classical mereology that
presupposes neither eternalism nor presentism, and that allows for
mereological sums to change their parts over time.  (In his paper
``Tensed mereology'' \citeyearpar{hovda2011} he in fact offers three
versions of his theory; I will focus on the first formulation.)

{\em Tensed mereology} is similar to Fine's theory of embodiments in
at least one important way.  According to both theories, some relation
or property is required to specify when and where a sum of some things
exist.  Fine's example was of a bunch of flowers; the flowers compose
the bunch when and only when the relation of being bunched holds of
them.

Hovda uses the term `condition' to cover both relations and
properties:

\begin{squote}
We will want a ``condition'' to be an open sentence that may have more
than one free variable, \emph{together with a specification of a
  target variable}. For example, we will want to consider
``conditions'' like ``$y$ loves $x$'', with ``$x$'' as target.  This
is because we want to consider, in effect, for each object that might
be a value of the variable ``$y$'', the property of being a thing
loved by that object.  The point of this may be brought out by an
example.  We want as an instance of the plenitude principle, roughly
this: that for every $y$, if $y$ loves at least one thing, then there
is a thing $b$ such that $b$ is a fusion of the condition (with
respect to $x$) ``$y$ loves $x$'' (i.e., a fusion of the condition of
being loved by $y$) \citeyearpar[sec. 1.1n2]{hovda2011}.
\end{squote}

With this notion in mind, Hovda replaces classical mereological sums
with `diachronic fusions':

\begin{description}
  \item[Diachronic fusion] An object $b$ is a ``diachronic fusion'' of
    a condition if and only if it is always the case that (1) every
    $x$ that meets the condition is part of $b$; and (2) every part of
    $b$ overlaps something that meets the condition
    \citeyearpar[sec. 1.1]{hovda2011}.
\end{description}

Like Fine, Hovda takes composition to be unrestricted: ``every
suitable condition has a diachronic fusion (where a condition is
suitable iff it is not always empty; i.e., it is suitable iff at some
time, at least one thing satisfies it)'' \citeyearpar[sec.
  3.1]{hovda2011}.  Not only does every suitable condition have a
diachronic fusion, but ``it is always the case that every suitable
condition has a diachronic fusion''
\citeyearpar[sec. 3.1.1]{hovda2011}.  In other words,

\begin{itemize}
  \item For every condition $K$,
  \item if it is ever the case that something satisfies $K$, then
  \item there is exactly one thing $b$ such that at at any time $t$
    during which anything satisfies $K$, all the things that satisfy
    $K$ at $t$ are parts of $b$ at $t$ and all the parts of $b$ at $t$
    overlap at least one of the things that satisfies $K$ at $t$.
\end{itemize}

This has the welcome consequence that there are no two things that are
{\em always} co-located.  However, it does mean that there will be
very many things that are co-located at some time or other, and this
may cause problems.  Things being co-located at a time will cause
problems if we make two plausible assumptions about how parts work.
These two assumptions are {\em strong supplementation} and {\em
  anti-symmetry}.

Strong supplementation is a common assumption in mereology to the
effect that if everything that overlaps one thing overlaps another
thing, then the first thing is part of the second.  It may be
formalized as:

\begin{displaymath}
\forall x \forall y ( \forall z ( z \circ x \rightarrow z \circ y )
\rightarrow x \leq y )
\end{displaymath}
(Here `$x \circ y$' means `$x$ overlaps $y$' and `$x \leq y$' means
`$x$ is part of $y$'.)

Anti-symmetry is the assumption that if two things are parts of each
other, it follows that they are really the {\em same thing}:

\begin{displaymath}
\forall x \forall y ( ( x \leq y \wedge y \leq x ) \rightarrow x = y )
\end{displaymath}

At this point, a problem arises:

\begin{squote}
Consider a (diachronic) fusion of the condition on $x$ that ``($x$ is
Socrates and Socrates is sitting) or ($x$ is Plato and Socrates is not
sitting).''  Suppose $\beta$ is such a fusion.  Then, when Socrates
and Plato are sitting at dinner, $\beta$ exists and it should hold
(then) that everything that overlaps Socrates overlaps $\beta$ and
vice-versa.  By strong supplementation, Socrates and $\beta$ then bear
$\leq$ to one another.  By anti-symmetry, they are then identical.
But later, when Socrates stands, $\beta$ will then (by similar
reasoning) be identical with Plato, yet Socrates won't be identical
with Plato, so Socrates and $\beta$ are then non-identical.  I take
this result to be unacceptable: once identical, always identical,
certainly if ``both'' exist \citeyearpar[sec. 3.1.2]{hovda2011}.
\end{squote}

One of our assumptions---strong supplementation or
anti-symmetry---must be withdrawn.  Hovda chooses to deny
anti-symmetry:

\begin{squote}
Instead of saying that it is always true that any mutual parts are
identical, [we] will say, roughly, that any things that are always
mutual parts are identical \citep[sec. 3.1.2]{hovda2011}.
\end{squote}

The rejection of anti-symmetry helps to show why Hovda's theory will
result in there being, at particular times, many co-located objects.
(Fine's theory of embodiments did not have to reject anti-symmetry
because his co-located objects did not share all the same parts---they
had different relations as parts.  Fine's theory of operators did not
have to reject anti-symmetry because his `co-located' things had the
same parts, but in different ways.)

\subsection{Problems with the third theory}
\label{problems3}
Hovda's tensed mereology avoids the conclusion that conditions or
relations or functions are parts of things.  The theory does this,
however, only by rejecting our assumption \ref{ass-co} of both
non-co-location {\em and} uniqueness.  When Tibbles is sitting, the
fusion of the condition of being Tibbles and the fusion of the
condition of being Tibbles sitting have all the same parts (in the
same sense of `part').  The two things are distinguished by the fact
that it {\em will be} or {\em was} the case that they do not share all
the same parts.

Like Fine's theory of embodiments, Hovda's theory produces a huge
number of things, most (perhaps all) of which are temporarily
co-located with other things.  For instance, when Tibbles the cat is
sitting, there are also an unknown number of other objects co-located
with Tibbles: the fusion of the condition of being Tibbles sitting,
the fusion of the condition of being Tibbles while less than 3 years
old, the fusion of the condition of being Tibbles with a full stomach,
etc.  We should therefore pose to Hovda the same objection, by
Thomson, that we posed to Fine's theory of embodiments in section
\ref{problems1}: is it really true that when a cat sits, it thereby
comes to pass that a new thing (a sitting-cat) comes into existence?
When (if) I graduate college, does a college-graduate pop into being?

One might object further that Hovda's theory results in temporarily
co-located thinkers, which would be a grave objection indeed.  We
should all agree that the object that fuses the condition of being
Tibbles surely thinks.  How, then can the object that fuses the
condition of being a sitting cat {\em not} think?  Like Fine in
section \ref{problems1}, Hovda must reply that the fusion of the
condition of being Tibbles sitting is just not the kind of thing that
can think.

It is not obviously false that the fusion of being a sitting cat
cannot think, but what about the fusion of being a {\em thinking} cat?
That is, can the fusion of being Tibbles while thinking itself think?
I am inclined to think so.

Hovda could object that this thing is identical with the fusion of the
condition of being Tibbles.  We saw above that any `two' diachronic
fusions that always share the same parts are really the very same
thing.  But is it true that Tibbles is {\em always} thinking?  Tibbles
probably does not think when unconscious (and not dreaming).
Moreover, depending on when the life of Tibbles begins, there may be a
period in which Tibbles {\em can't} think; for example, while a fetus.

If one wants to quibble about whether or how cats think, we can run
the same argument with people.  There is a fusion of the condition of
being Alex; this fusion is me, a thinking thing.  But there is also
the condition of being Alex thinking.

It might seem that {\em by definition} this thing thinks; after all,
the condition that is fused makes direct reference to thinking.  But
it is important to keep in mind that ``the formal theories behind the
idea of a fusion make no mention of change, or its absence, or of the
essential natures of [fusions]'' \citep[sec. 1]{hovda2011}.  The
condition being fused merely picks out the objects that satisfy it.
Moreover, any two fusions which always have the same parts are
identical; if it was true that we exist when and only when we think,
the fusion of the condition of being Alex thinking would {\em be} the
fusion of the condition of being Alex.

Nonetheless there is reason to think that the fusion of the condition
of being Alex thinking {\em does} think.  At any time during which
this fusion exists, it has all the same parts as I---the fusion of the
condition of being Alex---have.  If I think, it seems that this is as
a result of the functioning of my brain; and my brain is also part of
the fusion of the condition of being me thinking.  How can only one of
us think?  If we are not to have co-located thinkers, we must deny
that if a thing is thinking at a given time, it is in virtue of the
structure or functioning of its parts at that time.  We must instead
say that it is in virtue of its history (only the fusion of the
condition of being Alex is a {\em human}) or other properties.

There may be a satisfactory reply here, one that allows Hovda to avoid
co-located thinkers.  But his theory still entails a curious number of
overlapping things.  A theory that gave us fewer would, I think, be
better.

\section{Does the Supreme Court exist?}
\label{groups}
These three theories are bizarre---they predict a huge number of
co-located things always popping into and out of existence.  In the
case of Fine's theory of composition operators (section \ref{fine-c}),
there may be a swarm of different kinds of things as well.  

But bizarre as this is, there are some reasons to believe such a
theory.  Not only does it appear to follow from other theoretical
assumptions (universalism, change of parts, the negation of
four-dimensionalism), but it allows us to account for certain aspects
of ordinary belief.

For example, the three theories we have examined, and especially the
theory of composition operators, can help us describe the existence of
{\em groups} better than can the theory of classical mereology alone.

\subsection{What's a group?}
\label{what-g}
Groups, as I will understand them, are things that have other things
as members.  Families, sports teams, support groups, and committees
are all examples of groups.

But why should we suppose that groups are {\em things}?  For example,
one might claim that `the Dunns' refers {\em plurally} to me and the
other members of my family.  I say things like ``the Dunns are fine
people''; the term obviously does not function as a singular term.
The suggestion is that `the crew' behaves similarly.  When I say that
the crew exists, it would then {\em not} follow that there is a {\em
  thing} composed of the crewmembers.  Saying ``the crew exists''
would instead be equivalent to saying ``the crewmembers all exist''.

I do not think that this a plausible claim.  Recall the analogy I
tried to draw between `the crew' and `the Dunns'.  On closer
inspection, this analogy appears weak.  A stronger analogy would be
between a term like `the crew' and a term like `the Dunn family'.
`The Dunn family' is {\em not} a plural referring expression.  It is
used to refer to a {\em thing}.  If I talk about `the Dunn family', I
would say things like, ``The Dunn family is waning'', or ``The Dunn
family must regain its political power''.  The term `the Dunn family'
is a singular term that designates a thing---the family.

`The crew' appears to behave like `the Dunn family' and not like `the
Dunns'.  We say things like ``there is a skeleton crew on board'', or
``the crew is small for such a large ship'', and ``the crew is
abandoning the ship''.  We so also say things like ``the crew {\em
  are} abandoning the ship'', but this may be a case of {\em
  non-literal} speech; `the crew' is being used non-literally to refer
to the crewmembers.  If this is not non-literal speech, then it seems
most likely that `the crew' is {\em ambiguous}: it can be used to
refer either to {\em the crew} or to the crew{\em members}.  In the
former case, `the crew' is used as a singular term.

Similar considerations apply to terms like `team' as well.  `The Reed
College women's rugby team' is a singular term, for it behaves in the
same ways as do `the crew' and `the Dunn family'.  We say things like
``The Reed College women's rugby team is going to win'', or ``The Reed
College women's rugby team is in Seattle this weekend''.  However,
team names are often used (whether non-literally or not) to refer to
the team-members, rather than to the team itself.  This is often due to
pluralized team names.  The Reed College women's rugby team is called
``The Badass Sparkle Princesses''.  This leads us to say things like
``The Badass Sparkle Princesses are on a losing streak''.  Here we are
led by the plural construction to---perhaps unconsciously---use the
term to refer not to the team itself but to the players.  The Badass
Sparkle Princesses {\em is} a rugby team, but it is far more natural
to say that the Badass Sparkle Princesses are rugby players.

Yet even if it is agreed that groups are things, why shouldn't we
think they are just sets, or sums?  Why suppose that there is another
kind of thing?

\subsection{Groups and sets}
\label{group-set}
I have suggested that families, crews, and other {\em groups} are,
strictly speaking, things.  But one might object that there is no {\em
  need} to suppose that there are things called groups; we can
identify families, crews, courts, etc.\ with {\em sets}, and avoid the
`ontological clutter' that would result from the introduction of
groups.  Groups, it may be said, are really no different than sets.
When we speak of a group of people, we are actually referring to the
set of which they are members.

But there are some reasons why it seems incorrect to identify groups
with sets.  Take the Supreme Court.  It seems that any attempt to
identify the Supreme Court with the set of the Supreme Court justices
will not succeed.  This is because the membership of the Supreme Court
changes over time, while the members of a set do not.  The set
containing the 1990 justices is a {\em different} set from the set
containing the 2012 justices, but the 2012 Supreme Court is not a
different entity than the 1990 Court.  (We may of course say things
like ``it's a different court now'', but by that we mean only that it
is composed of different people, and so may rule differently---note
that we do {\em not} say ``it's a different Court now''.)

In section \ref{set-id} I will re-examine these arguments against
identifying groups with sets.  But let us suppose for know that they
are correct.

\subsection{Groups and sums}
\label{group-sum}
If one grants that groups such as the Supreme Court are not sets, it
may be objected that they are therefore mereological sums, like chairs
and people (see section \ref{tech}).  But it seems that

\begin{squote}
membership in the Supreme Court is very different from
the part-whole relation on material objects.  The part-whole relation
on material objects is a transitive relation.  Thus if one identified
the Supreme Court with a material object and Justice Breyer with a
part of it, then one would be forced to conclude that Justice Breyer's
arm must be a part of the Supreme Court as well.  Yet, it is plain
that Justice Breyer's arm is neither a part nor a member of the
Supreme Court \citep[136--137]{uzquiano2004a}.
\end{squote}

If we are going to attempt to account for groups with Fine's theory of
embodiments (section \ref{fine-h}) or Hovda's theory of tensed
mereology (section \ref{hovda}), we must accept this strange result,
and identify groups with sums.  But if we adopt Fine's theory of
composition operators (section \ref{fine-c}), we do not need to
identify groups with either sets or sums.

According to Fine's theory of embodiments, when we say ``the Supreme
Court has become more diverse over time'' we are referring to a
variable embodiment that is composed of different rigid embodiments at
different times.  These rigid embodiments are things composed of
justices (for example, Rehnquist, Stevens, O'Connor, Scalia, Kennedy,
Souter, Thomas, Ginsburg, and Breyer) in a certain relation (that of
being part of the Supreme Court).  The rigid embodiment $S =$
(Rehnquist, Stevens, O'Connor, Scalia, Kennedy, Souter, Thomas,
Ginsburg, Breyer)/$R$ exists when and only when those justices stand
in that relation; when Rehnquist died, $S$ ceased to exist.  But the
variable embodiment that is the Supreme Court did not cease to exist;
it was simply no longer composed of {\em that} rigid embodiment.

According to Hovda's mereology, the Supreme Court is the diachronic
fusion of the condition of being the Supreme Court.  It is a scattered
object that overlaps the individual justices as well as other things
at various times (like the fusion of the condition of being Rehnquist,
Stevens, O'Connor, Scalia, Kennedy, Souter, Thomas, Ginsburg, or
Breyer, with which it is temporarily co-located).

The treatment of groups from within the framework of Fine's theory of
composition operators is somewhat more involved.  Fine's theory has
the advantage of treating groups as different than sums or sets, but
it also requires treating different groups {\em themselves} as
different kinds of things.

\subsection{Composition operators and groups}
\label{group-fine}
To see why the theory of composition operators must treat different
groups as different kinds, we must first recognize that a set of
people can compose more than one group at a time.  Suppose that all
and only the members of the Supreme Court in 2004 are part of the
Special Committee on Judicial Ethics.  In this case ``the Supreme
Court share[s] all of its members with the Special Committee on
Judicial Ethics as of a certain time'' \citep[151]{uzquiano2004a}.  It
seems false to say that, in 2004, the Supreme Court was identical with
the Special Committee.  But if the Supreme Court, $G$, is $\sum _{t} (
S )$ and the Special Committee, $C$, is also $\sum _{t} ( S )$, then
how can we deny that $G = C$?

Just as the co-located statue and lump were produced by means of
different operators, so the Supreme Court and the Special Committee
must be produced by means of different operators.  The Supreme Court
will be the product of some operator $\sum _{sc}$ and the Special
Committee of $\sum _{sp}$.  Since these two things are the products of
different operators, they are not identical.

But just as the statue and the lump are therefore different kinds of
things, so the Supreme Court and the Special Committee must now be
recognized as not merely different groups, but as different {\em
  kinds}.  There may be a greater resemblance between their two kinds
than there is between things like sums and sets, but ultimately they
have been estranged.  Calling both `groups' is simply categorizing
both kinds under a common label.

Is there anything wrong with this conclusion?  It does seem bizarre in
some ways.  For it is clear enough that one person (or group) may be a
member of an indefinite number of groups; each of these `groups' will
therefore be a product of a different composition operator.  And each
will, strictly speaking, be a different kind of thing.

Previously we had a relatively tidy ontology.  There were sums, and
sets, and other well-known kinds; but now each task force or
subcommittee is potentially a kind unto itself.  Fine recognizes that
his approach ``will lead to an infinitude of forms of
composition\,\ldots a vast mereological firmament''
\citeyearpar[576]{fine2010}.  But he does not consider this to be a
drawback.

\section{Lessons}
\label{lessons-p}
The qualms I have vocalized in this section by no means show that any
of the three theories presented are false.  But I would prefer a
solution that does not postulate such bizarre pluralities.  In section
\ref{set-id}, therefore, I set aside the theories of this section and
look at a new possibility.  I will re-examine the thesis that
groups---all of them---really are identical with sets, and that
ordinary things are identical with sums.  This will lead to some
strange consequences, but it may be that they are {\em less} strange
than the `explosion of reality' that we otherwise face.

First, however, there is a question that arises for each theory: can
it explain (or be supplemented with an explanation) of why we believe
things that the theory denies?

\subsection{Can the plurality theories explain what we believe?}
\label{explain-p}
In section \ref{stroud} I claimed that a theory that denies that there
are chairs should be supplemented with an explanation of why we
nonetheless believe that there are chairs.

None of the theories I have proposed deny that there are chairs, but
they do make other unexpected claims that conflict with certain of our
beliefs.  Therefore, the theories should be supplemented with
explanations of why we hold these beliefs.

If the plurality thesis is right, why do we believe that there are
chairs when we {\em don't} believe that there are millions of other
objects?  Why, when there is a ten-pound chair in an otherwise empty
room, are we inclined to say that there is just {\em one} thing that
weighs ten pounds?  What is so special about the chair that promotes
it to our attention out of the many objects in the room?

It may be, in fact, that we {\em do} believe that there are co-located
things.  Many philosophers believe that there are co-located statues
and lumps of clay; do any normal people hold this belief too?
If so, then the question become: why do people believe in only {\em
  some} co-located things?

One answer might be simply that things like chairs, statues, and lumps
matter to us more than the other things.  This is similar to Trenton
Merricks' explanation of why we believe that there are chairs (section
\ref{universe}).  Merricks denies that there are chairs, but claimed
that we believe there are chairs because `things arranged chairwise'
matter to us.  Because they matter, we have terms to refer to them;
for ease of use (or some other reason) we use singular terms to refer
to things arranged chairwise, and so we are fooled by the grammar into
thinking that there are chairs.

Likewise, a philosopher like Fine or Hovda could claim that chairs
(which do exist) matter to us more than the plurality of co-located
objects that share parts with the chair.  We introduce terms to pick
out one object from among the plurality (how this happens is a
difficult question) and ignore the others.  The things that do not
matter of course remain; ``we just pay most of them no attention''
\citep[356]{bennett2004}.

\subsection{Is there an alternative to the plurality?}
We have examined three different theories that account for the
existence of chairs.  Each has the difficult consequence that there
are pluralities of co-located objects.  Once we have admitted that,
for instance, the same matter might compose a statue and a lump, we
have trouble resisting the idea that there might be {\em more} things,
of other kinds, composed of that same matter.  (Some theories that I
have not discussed, like Thomson's \citeyearpar{thomson1998a}, are
also forced to posit co-located kinds.)  Where we might take there to
be one thing (or maybe two), we now seem committed to there being a
huge number of co-located things.  I believe that this consequence is
a reason to reject each theory and look for an alternative.

In section \ref{essential} I will sketch an alternative theory.  This
theory identifies chairs and other ordinary things with mereological
sums in the classical sense.  Groups like the Supreme Court will be
identified with sets.  As I have argued, a sum is like a set in that
it does not change its parts.  Therefore when refer to `my chair', I
am referring to a sum.  If I replace the leg on my chair, I henceforth
use `my chair' to refer to a different sum.  Which sum I refer to with
`my chair' will be governed by convention.

\ifstandalone
\end{spacing}
\bibliography{everything}
\bibliographystyle{ChicagoReedweb}
\fi
\end{document}
