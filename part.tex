\documentclass[11pt]{article}
\usepackage{standalone} \newif\ifstandlone \standalonetrue
\usepackage[left=1.75in, right=1.75in, top=1.25in, bottom=1.25in]{geometry}
\geometry{letterpaper}
\usepackage{graphicx}
\usepackage{enumitem}
\usepackage{amssymb}
\usepackage{amsmath}
\usepackage{epstopdf}
\usepackage{verbatim}
\usepackage{setspace}
\usepackage{natbib}
\setcitestyle{aysep={}}
\usepackage{url}
\usepackage{hyperref}
\synctex=1

\DeclareSymbolFont{symbolsC}{U}{txsyc}{m}{n}
\DeclareMathSymbol{\strictif}{\mathrel}{symbolsC}{74}
\DeclareMathSymbol{\boxright}{\mathrel}{symbolsC}{128}

\newenvironment{squote}{%
\begin{spacing}{1}
\begin{list}{}{%
\setlength{\labelwidth}{0pt}%
\rightmargin\leftmargin%
}
\item\relax
}{%
\end{list}%
\end{spacing}
}

\title{How do chairs persist over time?}
\author{Alexander A. Dunn}
\begin{document}
\ifstandalone
\maketitle
\begin{spacing}{1.5}
\fi

\label{parts}

In the previous section I argued that not only do `material things'
like chairs exist, but other kinds of things---like teams and
families---exist as well.  In this section I will adopt Kit Fine's
theory of parthood, which claims that there are many different ways of
being a part of something.  However, I will argue that the way in
which people are members of groups is the way that that things are
parts of sets, and that teams and other groups are actually identical
with sets.  Since sets do not change their parts, which set is
identical with a group at a time is variable and dictated by {\em
  convention}.  I will argue that the situation is analogous with
material things: which `mereological sum' is identical with a chair at
a time is a conventional matter.

\section{Parthood and the language of composition}
\label{parthood}
Things like teams, crews, and families are indeed {\em things}.  They
are not disguised references to plurals, nor are they ``mere
collections'' \citep[29]{inwagen2009}.  Moreover, things like teams
are things with {\em parts}.  The rugby players are each {\em part} of
the Reed College women's rugby team.  The team is made up of---it is
composed of---the players.

When I say that the players are part of the team, or that the
crewmembers are part of the crew, or that I am part of my family, is
that use of `part' the same as when I say that the tree is part of the
dogbush, or that the seat is part of the chair?  Are {\em any} of
these uses of `part' the same?

Technical notions of composition are often defined in terms of
parthood.  How I am using the term `part' will therefore influence how
I construct formal equivalences for propositions like ``there are
chairs'', ``there are dogbushes'', and ``there are teams''.  Is the
appropriate formalization for each of these the same?  Can chairs,
dogbushes, and teams each `fit in' to the schema ``there is an $x$
such that it is composed of the $y$s''?  Or is `composition' one of
many {\em operations} that `produce' things?

\subsection{Van Inwagen's notion of parthood}
\label{van-part}
Van Inwagen defines his technical notion of composition (see section
\ref{scq}) in terms of a largely intuitive notion of parthood.  Van
Inwagen's interest, however, is restricted to `material' objects
(objects made exclusively of quarks and protons, or whatever the basic
atoms of the physical world turn out to be).  While he goes on to use
`part' only in reference to material objects, he recognizes that the
term has much wider application:

\begin{squote}
Parthood will occupy a central place in the present study of material
objects.  It is therefore worth noting that the word `part' is applied
to many things besides material objects.  We have already noted that
submicroscopic objects like quarks and protons are at least not clear
cases of material objects; nevertheless, every material object would
seem pretty clearly to have quarks and protons as \emph{parts}, and,
it would seem, in exactly the same sense of \emph{part} as that in
which a paradigmatic material object might have another paradigmatic
material object as a part.  A ``part,'' therefore, need not be a thing
that is clearly a material object.  Moreover, the word `part' is
applied to things that are clearly \emph{not} material objects---or at
least it is on the assumption that these things really exist and that
apparent reference to them is not a mere manner of speaking.  A stanza
is a part of a poem; Botvinnik was in trouble for part of the game;
the part of the curve that lies below the x-axis contains two minima;
parts of his story are hard to believe\,\ldots\,such examples can be
multiplied indefinitely.  Does this word `part' mean the same thing
when we speak of parts of cats, parts of poems, parts of games, parts
of curves, and parts of stories \citeyearpar[18--19]{inwagen1995}?
\end{squote} 

Van Inwagen suggests that `part' does have a number of different
meanings.  Later he says that ``there is one relation called
`parthood' whose field comprises material objects\,\ldots\,another
relation called `parthood' defined on events, another still defined on
stories, yet another defined on curves, and so on''
\citeyearpar[19]{inwagen1995}.

One reason why we might resist this conclusion is that it appears to
rule out the silly example of the wish sandwich.  A wish sandwich,
recall, is the kind of sandwich where you have two slices of bread and
wish you had some meat.  The slices of bread and the wish for meat are
all parts of the wish sandwich.  But if they are parts of the sandwich
in different ways, then in what sense did I use `part' in the
preceding sentence?  If it was part$_b$---the parthood relation for
foodstuffs---then the sentence was false, because a wish does not
partake in that sort of relation.  If the relation was that of
parthood$_w$---parthood for wishes---then the sentence would again be
false, because {\em that} relation does not govern foodstuffs.  I
could instead say ``the slices of bread are part$_b$ of the sandwich
and the wish is part$_w$ of the sandwich'', but it still seems to me
that the original sentence is {\em true}.  Might this suggest that
there is really just one parthood relation that both foodstuffs and
wishes partake in?

Moreover, our modified sentence still faces a difficulty.  For the
notion of composition is generally defined in terms of parthood.
Since van Inwagen's technical definition of composition is given in
terms of his notion of parthood, `composition' for van Inwagen can
only apply to things whose parts are all material things.  Van
Inwagen's notion of composition cannot make sense of the wish
sandwich, or any thing with both material and non-material parts.
(Van Inwagen does not see this as a disadvantage; he finds mysterious
the idea that there could be something composed of ``you and I and the
number two'' \citeyearpar[20]{inwagen1995}.)

We could, perhaps, define `composition' in terms of not just one
parthood relation but all of them.  Composition would take into
account all possible ways there are of being a part.  Kit Fine has
proposed a theory of parthood that takes seriously the possibility
that there are a plurality of different parthood relations.

\section{Fine's theory of part}
\label{fine}
Fine agrees with van Inwagen that the notion of parthood should not be
reserved only for material things:

\begin{squote}
Philosophers have often supposed the notion of part only has proper
application to material things or the like and that its application
to abstract objects such as sets or properties is somehow improper
and not sanctioned by ordinary use.  But I suspect that this is
something of a philosopher’s myth.  We happily talk of a sentence
being composed of words and of the words being composed of
letters---and not just the sentence and work tokens, mind, but also
the types.  And similarly, a symphony (and not just its performance)
will be composed of movements, a play of acts, a proof of steps.  I
wonder how many of these philosophers have said such things as ``this
paper is in three parts.''  When they have, then I very much doubt
that they would have any inclination, as ordinary speakers of the
language, to add ``but not, of course, in a strict or literal sense of
the term''; and the intended reference here is not primarily---or
perhaps not at all---to the tokens of the paper but to the type of
which they are the tokens.  The evidence concerning our ordinary talk
of part is mixed and complicated, but it does not seem especially to
favor taking material things to be the only true relata of the
relation \citeyearpar[561]{fine2010}.
\end{squote}

But even if one accepts the idea that there are things other than
`material things' that have parts, one might object that they are
still all parts in the same sense.  This might be the parthood
relation of classical mereology, or it might be some other, general
relation.  Moreover, one who maintained the univocality of parthood
can still concede that there are different ways of being a part.  But
for the believer in the univocality of parthood---the `monist'---these
are only {\em derivative} kinds of parthood.  For example, for any
given mereological sum, there are bigger and smaller parts of it. But
these are bigger and smaller parts of the same {\em kind}.  The
pluralist goes further and claims that there are parts of different
kinds. These different kinds are not derivative but \emph{basic}; they
are ``not definable in terms of other ways of being a part''
\citep[561]{fine2010}.  Fine gives a number of reasons to think that
there might be different basic parthood relations:

\begin{squote}
Now, on the face of it, there would appear to be a wide variety of
basic ways in which one object can be a part of another.  The letter
`n' would appear to be a part of the expression `no', for example, and
a particular pint of milk part of a particular quart; and if these two
relations of part are not themselves basic (perhaps through being
restricted to expressions or quantities), there would appear to be
basic relations of part that hold between `n' and `no' or the pint and
the quart.  It is also plausible that the way in which `n' is a part
of `no' is different from the way in which the pint is a part of the
quart.  For if the two ways were the same, then how could it be that
two pints were only capable of composing a single quart, while the two
letters `n' and `o' were capable of composing two expressions, `no'
and `on' \citeyearpar[562]{fine2010}?
\end{squote}

The parthood relation for sets is again different.  The set containing
the only the letters `n' and `o' has the letters as parts.  When the
letters are parts of a set, their order is irrelevant, but when the
letters are parts of a word, order matters; hence `no' and `on'.  The
parthood relation for sets is also different from the parthood
relation for quantities (of milk):

\begin{squote}
If four quarts compose a gallon the pints which compose the quarts
will compose the gallon in the same way in which they compose the
quarts, whereas, if four sets compose a further set the members of the
sets will not compose the further set in the same way in which they
compose the component sets.  Thus we would now appear to have three
different basic ways in which one object can be a part of another
(pint/gallon, letter/word, and member/set); and once these cases have
been granted, it is plausible that there will be many more
\citeyearpar[562]{fine2010}.
\end{squote}

One might, of course, refuse to grant these cases.  But one would have
to refuse them {\em all}; for if it can be established that there are
even two different (basic) ways of being a part, then the pluralist
position is established.  Once it is established that there are at
least two ways of being a part, it becomes much more plausible that
there might be three ways, or more.  Fine therefore attempts to
motivate the idea that the members of a set are, quite literally,
parts of the set.

\subsection{Parts of sets}
\label{sets}
The first objection is that while parthood is supposed to be
transitive, the membership relation of sets is not.  The letter `n' is
a member of the set \{`n',\{`n',`o'\}\}, but `o' is not.  The
objection claims that sets have {\em members}, not parts, and that
Fine has confused the two.

But while it is true that the membership relation is not the parthood
relation, this is no reason to think that sets do not have parts.  A
given set will have certain members---the $x$s---and certain
parts---the $y$s---and only sometimes will the $x$s and the $y$s be
the very same things.  The set \{`n',\{`n',`o'\}\} has two members
but three parts.  The parthood relation for sets can even be defined
in set-theoretic terms:

\begin{squote}
It may well be thought that the way in which a member is a part of a
set is given, not by the membership relation itself, but by the
ancestral of the membership relation, where this is the relation that
holds between $x$ and $y$ when $x$ is a member of $y$ or a member of a
member of $y$ or a member of a member of a member of $y$, and so on
\citep[563]{fine2010}.
\end{squote}

A second objection is that talk of parthood in connection with things
like sets is somehow metaphorical or non-literal.  We saw above that
van Inwagen admits that many different things are said to have parts.
However, he qualifies this in two ways.  First, he seems to have
doubts (or at least is sympathetic with those who have doubts) as to
whether the non-material things that are said to have parts really
exist:

\begin{squote}
The word `part' is applied to things that are clearly \emph{not}
material objects---or at least it is on the assumption that these
things really exist and that apparent reference to them is not a mere
manner of speaking \citep[19]{inwagen1995}.
\end{squote}

If there are no such things as tennis matches or poems or papers, then
of course they do not have parts.  But I think it is obviously true
that there are such things.  This being so, what does it mean to say
that they have parts?  This is where van Inwagen's second
qualification comes in.  For he suggests not only that the `parts' of
tennis matches and poems are parts in a different way than are the
parts of a table, but that these different relations of parthood are
only tenuously connected.  Van Inwagen says that the various relations
of parthood (if such there be) are connected only by the ``unity of
analogy'' \citeyearpar[19]{inwagen1995}.  If the only similarity
between the parthood relation for poems and the parthood relation for
chairs is that they share the `analogy' of parthood, then is there
anything important or interesting about `parts' of poems?  Is the
parthood relation for sets likewise only interesting because of the
analogy with the parthood relation for chairs?

At least in the case of parthood for sets, the notion does not appear
to be wholly metaphorical:

\begin{squote}
In the case of set-membership, there would appear to be nothing that
might plausibly be taken to indicate that the talk of part-whole is
not to be taken literally. A set is indeed composed of or built up
from its members, and we should add that we may meaningfully
talk---and in the intended way---of \emph{replacing} one member of a
set with another.  Thus Aristotle in the set \{Plato, Aristotle\} may
be replaced with Socrates to obtain the set \{Plato, Socrates\}, with
the given set becoming a different set from what it was. In the case
of sets, our conception of members as parts seems to extend all the
way \citep[564]{fine2010}.
\end{squote}

But the second worry raised by van Inwagen remains.  Why should we
think that there is any {\em real} similarity between these different
parthood relations, other than the fact that we call them all
`parthood'?

\subsection{Operationalism}
\label{operation}
Fine's theory of {\em operationalism} helps answer this worry.
Various {\em operations} produce different things---mereological
summation produces mereological sums or fusions, the set-builder
produces sets, and so forth.  Parts are therefore {\em things} that
have been `combined', through one or more such operations, into a
single {\em thing}.  What is common to all parthood relations is that
from each set of parts is produced a {\em whole}.  This may simply be
metaphorical, but it is nonetheless an accurate description of the
result of the composition operator.  From parts (letters, atoms) we
can `make' something new (a word, a set, a chair).  What ties together
all the ways of being a part is that they are involved in an operation
that produces a single thing from a number of things:

\begin{squote}
In formulating the principles of mereology, it has been usual to take
the relation of part-whole or some associated relation (such as
overlap) as primitive.  But I believe that, in formulating a more
general theory, it is important to take the operation of composition
as primitive rather than the more familiar relation of part-whole.  In
the case of classical mereology, the operation of composition will
take some objects into the sum or fusion of those objects, while, in
the set-theoretic case, it will take some objects into the set of
those objects; and, in general, the operation of composition will be
the characteristic means (summation, set-builder, and so on) by which
a given kind of whole is formed from its parts \citep[565]{fine2010}.
\end{squote}

Each way of being a part can then be defined in terms of the related
composition operation:

\begin{squote}
Once given a compositional operation, a corresponding relation of part
may be defined in two steps.  We say first that $x$ is a component of
$y$ if $y$ is the result of applying $\sum$ to $x$ or to $x$ and some
other objects.  In other words, $y$ should be of the form $\sum
(x_{1}, x_{2}, \mathellipsis )$, where at least one of $x_1$, $x_2,
\mathellipsis$ is $x$.  Thus when $\sum$ is mereological summation the
components of an object will be mere parts, and where $\sum$ is the
set-builder the components of an object will be its members.  We may
then define $x$ to be a part of $y$ if there is a sequence of objects
$x_1$, $x_2, \mathellipsis x_n$, $n$ \textgreater{} $0$, for which $x
= x_1$, $y = x_n$, and $x_i$ is a component of $x_{i+1}$ for $i = 1$,
$2, \mathellipsis, n-1$. The parts of an object are the object itself,
or its components, or the components of the components, and so on
\citep[567--568]{fine2010}.
\end{squote}

The parthood relation for summation can therefore be seen to exhibit
reflexivity, transitivity and anti-symmetry:

\begin{description}
\item[Reflexivity] Each object is a part of itself.
\item[Transitivity] If $x$ is a part of $y$ and $y$ of $z$, then $x$
  is a part of $z$.
\item[Anti-symmetry] $x$ is a part of $y$ and $y$ of $x$ only when $x
  = y$ \citep[568]{fine2010}.
\end{description}

But not all definitions of parthood that issue from a composition
operator will exhibit these features:

\begin{squote}
When the underlying operation is summation, each object will be a part
of itself, since the unit sum of any object is the object itself, but
when the underlying operation is the set-builder, no object will be a
part of itself, since no object is ever an ancestral member of itself
\citep[569]{fine2010}.
\end{squote}

\subsection{Principles}
\label{principle}
Each composition operation will, according to Fine, be governed by
various principles:

\begin{squote}
I believe that the principles governing the basic forms of composition
will conform to a general template.  Variations in the principles for
the different forms of composition will then arise from variations in
how the template is to be filled in.  The template will comprise two
broad categories of principle---the {\em formal} and the {\em
  material} (though not quite in the sense of Husserl).  Among the
formal principles, we may distinguish between those that provide
conditions of application for the operation and those that provide
identity conditions; among the material principles, we may distinguish
between those that provide conditions for the presence of a whole (in
space and time or at a world) and those that specify the descriptive
character of the whole.  The presence conditions, in their turn, may
concern either the existence of the whole or its extension
\citeyearpar[569--570]{fine2010}.
\end{squote}

\paragraph{Formal principles}
The formal principles govern when composition occurs and when two
products of a composition operation are identical:

\begin{description}
  \item[Application] The application conditions are ``the conditions
    under which there are wholes of a given sort---which, on the
    operational approach, is a matter of stating when the result of
    applying the compositional operation to various objects will be
    defined'' \citep[570]{fine2010}.  For the summation operation, the
    application conditions are very lax: for any $n$ physical objects
    (where $n$ \textgreater{} 1), there is a thing composed of those
    objects.  The application conditions for the set-builder will be
    more limited; and other operations will be more restricted
    still. \label{fine-app}
  \item[Identity] Setting out identity conditions on Fine's
    operational approach ``is a matter of stating when a whole formed
    in one way by means of the compositional operation is the same as
    a given object or a whole that has been formed in some other way''
    \citeyearpar[570]{fine2010}.  For example, suppose $y$ is the
    result of applying the summation operator to $x$, and suppose
    $y^\prime$ is the result of applying the set-builder to $x$.  If
    $y \neq y^\prime$, it will be because of some difference in the
    respective operations that produced these composites.  As we will
    see below, the summation operator is defined such that its
    application to a single thing ($x$) produces that very thing.  The
    set-builder, however, produces the singleton of $x$, which is not
    $x$.  And whereas applying the set-builder to nothing produces the
    null set, the result of applying the summation operator to nothing
    might be undefined---nothing would result.
\end{description}

\paragraph{Material principles}
As above, there are two subcategories:

\begin{description}
  \item[Presence] Fine claims that ``there are two fundamentally
    different ways in which an object might be present in space or
    time; it may \emph{exist} in space or time, or it may be
    \emph{extended} (or \emph{located}) in space or time. Thus a
    material thing will exist in time but be extended in space while
    an event will be extended in both space and time''
    \citeyearpar[570]{fine2010}.  Whether or not this is actually true
    seems to depend on whether a `three-dimensional' theory of
    persistence is correct.  However, that question will remain unasked.
  \item[Character] ``The character conditions will tend to have a much
    more ad hoc character than the other conditions that we have
    considered. The color of a house, for example, is the color of its
    siding; the color of an egg, the color of its shell; the color of
    a pencil, the color of its lead. In the case of the `intrinsic'
    character of a thing---such as its mass or color---the character
    of the whole will be some sort of function of the character of the
    parts. But the function in question will vary from case to case''
    \citep[571]{fine2010}.  What we decide is the color of a house
    will turn on our concept `house', rather than anything about the
    house itself (anything beyond its having that color to {\em some}
    extent).  When the composition operation is summation, interesting
    questions also arise about weight; is the weight of a sum equal to
    the combined weights of the parts?  This seems intuitively so, but
    if someone's parts weight 150 pounds, and therefore they weight
    150, it does not follow that when they (and their parts) stand on
    a scale, the scale reads 300 pounds.  Figuring out what to say in
    these cases may ultimately feel ad hoc.
\end{description}

\subsection{Fine's pluralist account of classical mereology}
\label{classical}
Of the principles sketched above, Fine gives most attention to the
identity conditions for composition operations.  The composition
operation used as a paradigm is the summation operation of classical
mereology.  Fine's exposition of identity conditions for sums relies
on the notion of `regularity':

\begin{squote}
Call an identity condition $s = t$ {\em regular} if the variables
appearing in $s$ and in $t$ are the same.  Thus $\sum (x, y) = \sum
(y, x)$ is regular while $\sum (x, y) = x$ is not
\citeyearpar[572]{fine2010}.
\end{squote}

With this notion in hand, Fine proposes this condition for identity of
sums:

\begin{description}
  \item[Summative Identity] $s = t$ whenever `$s = t$' is a regular
    identity \citeyearpar[572]{fine2010}.
\end{description}

One particularly interesting aspect of this condition is that it
entails four more principles of the summation operation:

\begin{description}
  \item[Absorption] $\sum (\mathellipsis, x, x, \mathellipsis,
    \mathellipsis, y, y, \mathellipsis, \mathellipsis = \sum (
    \mathellipsis, x, \mathellipsis, y, \mathellipsis )$;
\item[Collapse] $\sum (x) = x$;
\item[Leveling] $\sum (\mathellipsis, \sum (x, y, z, \mathellipsis ),
  \mathellipsis, \sum (u, v, w, \mathellipsis ), \mathellipsis ) \\ =
  \sum (\mathellipsis, x, y, z, \mathellipsis, \mathellipsis, u, v, w,
  \mathellipsis, \mathellipsis )$;
\item[Permutation] $\sum (x, y, z, \mathellipsis ) = \sum (y, z, x,
  \mathellipsis )$ (and similarly for all other permutations)
  \citep[573]{fine2010}.
\end{description}

We can define other compositional identity criteria (e.g., sequences)
in terms of which of these principles apply to their compositional
operation.  But we may also devise new principles by which we may then
define new types of composition:

\begin{squote}
We should note that there would appear to be no good reason to require
that the defining principles for the various operations should be
limited to the particular principles (C [collapse], L [leveling], A
[absorption], and P [permutation]) that we used in characterizing
sums; for any set of regular identities would appear to be equally
well suited to defining a basic form of composition, so long as they
conform to Anti-cyclicity.  Indeed, I would conjecture that any such
set of principles in fact will correspond to a form of composition and
a corresponding form of whole.  How the resulting forms of composition
and whole might be organized is an interesting question, but it should
be apparent that the approach will lead to an infinitude of forms of
composition, each differing from one another in how exactly the
identity of the resulting wholes is to be
determined. \citep[575--576]{fine2010}.
\end{squote}

It is at this point that the importance of Fine's theory becomes
obvious.  Above I stressed that things like teams and families are
really {\em things}; moreover I made this claim as part of an attempt
to motivate a sort of universalistic outlook on metaphysics.  I argued
that the term `composition' was potentially misleading, but that it
was nevertheless correct to say that things like dogbushes, wish
sandwiches, and teams are composed of their parts.  But now it is
apparent that `composition' will mean something different when applied
to each of these things.  Each thing will be the product of a
different composition operation.

Fine's theory reveals new {\em kinds} of universalism.  One might be
committed to the existence of dogbushes---and so to unrestricted
mereological composition---but deny the existence of teams, groups,
crews, and families.  Or one might defend unrestricted composition of
groups while claiming a restriction on mereological composition.

Below I will look at how a definition of the composition operator for
groups might be formulated.  But I will first return to Fine's theory.

\subsection{Hybrid parts}
\label{hybrid}
Above, we saw that one objection to the idea of sets having parts was
that parthood is transitive and set-membership is not.  Moreover it
was supposed (rather plausibly) that the only reason we think that
sets have parts is {\em because} they have members; it is the members
of a set that we are tempted to call parts.  But, the objection goes,
it is a mistake to think of members as part.  I am the only member of
my singleton (the singleton of $x$ is the set resulting from applying
the set-builder to $x$ alone).  My hand, for instance, is not a member
of my singleton.  But my hand is a part of me.  If I was a part of my
singleton, then---because parthood is transitive---my hand would be a
part of my singleton.  And if that means that my hand is a {\em
  member} of my singleton, that is clearly wrong.

Fine points out, of course, that the objection makes the mistake of
supposing that something (me, my hand) can be a part in only one way
(in this case, through set-membership).  Once we recognize that there
are a plurality of ways of being a part, it becomes clear that my hand
is part of the set in one way, but not in another:

\begin{squote}
Given the specific relations of part, we may derive various {\em
  hybrid} relations of part.  Suppose, for example, that we are given
the relations of set-theoretic and mereological part---which we may
designate as \textepsilon -part and $m$-part. We may then take one
object to be an \textepsilon ,$m$-part of another if it is an
\textepsilon -part or an $m$-part or an $m$-part of an \textepsilon
-part or an \textepsilon -part of an $m$-part, or an $m$-part of an
\textepsilon -part of an $m$-part, and so on. More generally, if $K$
is a family of specific ways of being a part, we may take an object to
be a {\em K-part} of another if $x$ and $y$ can be linked by
relationships of $k$-part for $k$ in $K$ \citep[579]{fine2010}.
\end{squote}

My hand is a \textepsilon ,$m$-part of my singleton, but not a
\textepsilon -part.

By conjoining every way of being a part, we arrive at the most general
notion of part:

\begin{squote}
Among the hybrid relations of part, of special interest is the
relation of $K$-part where $K$ is the family of {\em all} the specific
ways of being a part.  This is the relation of $K$-part that holds
between two objects when they may be linked by relationships of
$k$-part without restriction on $k$.  We might call it the {\em
  general} relation of part, and it is a relation that holds between
$x$ and $y$ whenever $x$ is in any way whatever a part of $y$
\citep[580]{fine2010}.
\end{squote}

\subsection{Generating kinds}
\label{generate}
On this theory, what kind a thing is depends on what operation
produced it.  If a chair or a dogbush is a mereological sum, then this
is because they are produced by the summation operation.  The Dunn
family is `produced' by the family operation.  Groups are produced by
the group operation (see section \ref{group}).

But there is a difficulty to be avoided here.  As we saw in section
\ref{classical}, the mereological sum of a single thing $x$ is just
$x$.  Therefore there is a sense in which every physical thing,
including every simple, is a mereological sum, for the application of
the summation operation would just produce that thing.  To avoid this
consequence Fine introduces the notion of a {\em generative}
application of an operation:

\begin{squote}
We might say that the application $y = \Gamma (x_1, x_2, x_3,
\mathellipsis )$ of an operation $\Gamma$ is {\em generative} if there
is an explanation of the identity of $y$ as $\Gamma (x_1, x_2, x_3,
\mathellipsis )$; and we might say that the operation $\Gamma$ is
itself {\em generative} if it permits a generative application. Thus
both the set-builder and the operation of predication will be
generative in this sense \citeyearpar[582]{fine2010}.
\end{squote}

Whether or not the summation operation is generative depends on the
things it is being applied to.  When summing a dog and a tree, it is
generative; when summing a dog by itself, it is not.

For any operation, there will be things it applies to that it cannot
produce.  The summation operator fuses simples, but cannot produce
them; the set-builder combines many things that it cannot produce
(like letters).  For any given operation, there is a `level 0'
consisting of the things that the operator itself cannot produce:

\begin{squote}
We suppose that certain objects are simply given.  These are the
objects whose identity does not require an explanation in terms of
$\Gamma$.  Thus, when $\Gamma$ is the set-builder, they are the
objects that are not sets and, when $\Gamma$ is summation, they are
the objects that are not sums or, rather, the objects that do not need
to be seen as sums.

We now `generate' objects in stages.  At stage 0 are the givens; at
stage 1, we add the objects that result from a single application of
the generative operation $\Gamma$ to the givens \citep[583]{fine2010}.
\end{squote}

An application can now be identified as generative in a strong or a
weak sense:

\begin{description}
  \item[Strong generative application] Also called `strict' by Fine, a
    ``[strong] generative application of $\Gamma$ to the objects $x_1,
    x_2, \mathellipsis$ can now be defined as one in which $y = \Gamma
    (x_1, x_2, \mathellipsis )$ is of a higher level than each of
    $x_1, x_2, \mathellipsis$'' \citeyearpar[584]{fine2010}.  For
    example, summing the simples $x$ and $y$ to produce the fusion $z$
    would be a strong generative application of the summation
    operator; the simples are level 0 and $z$ is level 1.  Summing two
    composites, or a composite and a simple, would not be strongly
    generative; one or both of the parts would be the same level (1)
    as the product.
  \item[Weak generative application] To illuminate this notion Fine
    introduces another, that of a {\em putative generative
      application}: ``Let us say, in the first place, that $y = \Gamma
    (x_1, x_2, \mathellipsis )$ is a putative generative application
    of $\Gamma$ if $y$ is of a higher or of the same level as each of
    $x_1, x_2, \mathellipsis$.  This gives us the notions of a
    putative prior component and of a putative prior in the usual way.
    We now say that the application $y = \Gamma (x_1, x_2,
    \mathellipsis )$ of $\Gamma$ is a {\em weak} generative
    application if it is the putative generative application and if
    $y$ is not putatively prior to any of $x1, x2, \mathellipsis$.  We
    can get from $x_1, x_2, \mathellipsis$ to $y$ without an ascent in
    level but not from $y$ to any of $x_1, x_2, \mathellipsis$''
    \citeyearpar[584]{fine2010}.
\end{description}

Applying the summation operator to a simple is neither strongly nor
weakly generative.  It is not strongly generative because the result
is a simple, which is at level 0---the same level as its part
(itself).  It is not weakly generative because the result of the
operation is putatively prior to its parts.

\section{Groups and sets}
\label{group}
There are a number of reasons to think that, along with sets, there
are also groups.  The most salient reason is the apparent fact that
groups can change their parts, while sets cannot.  If groups are
distinct from sets, we should be able to define a group-builder that
is distinct from a set-builder.  However, defining a group-builder
that allows a group to change its parts requires that we assume {\em
  eternalism}.  If we wish to remain neutral on the
presentism/eternalism debate, we have no means of distinguishing
groups from sets.

\subsection{Motivating groups}
The set-builder, on Fine's theory, takes things (such as jurists) and
produces a set composed of them.  There is a set $S$ composed of the
2004 Supreme Court justices:

\begin{squote}
$S = \sum _{\in}$ (Rehnquist, Stevens, O'Connor, Scalia, Kennedy,
  Souter, Thomas, \\ Ginsburg, Breyer) $ = $ \{Rehnquist, Stevens,
  O'Connor, Scalia, Kennedy, \\ Souter, Thomas, Ginsburg, Breyer\}
\end{squote}

But some claim that, in addition to sets, there are also {\em groups}.
There may, in addition to the set \{Rehnquist, Stevens, O'Connor,
Scalia, Kennedy, Souter, Thomas, Ginsburg, Breyer\}, a group
containing the very same people.

One might ask why this is necessary.  Groups, it might be objected,
are really no different than sets.  When we speak of a group of
people, we are actually referring to the set of which they are
members.

But there are some reasons why it seems incorrect to identify groups
with sets.  Take the Supreme Court.  It seems that any attempt to
identify the Supreme Court with the set of the Supreme Court justices
will not succeed.  This is because the membership of the Supreme Court
changes over time, while the members of a set do not.  The set
containing the 1990 justices is a {\em different} set from the set
containing the 2012 justices, but the 2012 Supreme Court is not a
different entity than the 1990 Court.  (We may of course say things
like ``it's a different court now'', but by that we mean only that it
is composed of different people, and so may rule differently---note
that we do {\em not} say ``it's a different Court now''.)

If one grants that groups such as the Supreme Court are not sets, it
may still be objected that they are therefore simply mereological
sums.  But 

\begin{squote}
membership in the Supreme Court is very different from
the part-whole relation on material objects.  The part-whole relation
on material objects is a transitive relation.  Thus if one identified
the Supreme Court with a material object and Justice Breyer with a
part of it, then one would be forced to conclude that Justice Breyer's
arm must be a part of the Supreme Court as well.  Yet, it is plain
that Justice Breyer's arm is neither a part nor a member of the
Supreme Court \citep[136--137]{uzquiano2004a}.
\end{squote}

If the Supreme Court were a mereological sum, it would behave very
strangely.  What its parts would be on a given occasion would depend
on the appointment decisions of the President.  (If we accept a
`four-dimensional' version of universalism, then objects have {\em
  temporal} as well as spatial parts.  There would then be a
mereological sum of the parts of the justices that existed during
their appointments.  This would be an object whose existence would not
depend on the President.)

There is at least some motivation to posit a new {\em kind} of thing
that is not a set or a sum.  This new kind is the group.
Unfortunately, once we try to spell out the identity conditions of
groups, we will see that, without assuming eternalism, we cannot
distinguish groups from sets.

\subsection{The place of the group in Fine's template}
\label{group-temp}
In section \ref{classical} above, Kit Fine showed how the summation
operator relates to four different properties: Collapse, Leveling,
Absorption, and Permutation.

\begin{description}
  \item[Absorption] $\sum (\mathellipsis, x, x, \mathellipsis,
    \mathellipsis, y, y, \mathellipsis, \mathellipsis = \sum (
    \mathellipsis, x, \mathellipsis, y, \mathellipsis )$;
\item[Collapse] $\sum (x) = x$;
\item[Leveling] $\sum (\mathellipsis, \sum (x, y, z, \mathellipsis ),
  \mathellipsis, \sum (u, v, w, \mathellipsis ), \mathellipsis ) \\ =
  \sum (\mathellipsis, x, y, z, \mathellipsis, \mathellipsis, u, v, w,
  \mathellipsis, \mathellipsis )$;
\item[Permutation] $\sum (x, y, z, \mathellipsis ) = \sum (y, z, x,
  \mathellipsis )$ (and similarly for all other permutations)
  \citep[573]{fine2010}.
\end{description}

Sums have all four properties, while sets have only Permutation and
Absorption.  We can begin to define our group operator by thinking
about which of these properties it has.

I tentatively suggest that groups mimic sets with regard to these
four properties.  Groups possess Absorption due to the fact that one
cannot be twice a member of the same group.  Groups possess
Permutation, since there is no `order' with regard to membership of a
group (there may be {\em temporal} order---I joined the group
first!---but that is not the same thing).  Groups do not possess
Collapse, since (I am inclined to think) a group can have a single
member without thereby {\em being} that member.  If, for example, a
task force is created and only one individual assigned to it, the
findings of the task force will be of the {\em task force} and not of
the individual.  Groups do not possess Leveling either, since there
can be groups made up of groups.

So far we have seen that groups are quite similar to sets.  But there
are some differences.  As we have seen, groups can change their
membership, while sets cannot.  Additionally, I suggest that the
application conditions (see section \ref{fine-app}) for groups is
different from that of sets.  While sets can have more or less
anything as members (letters, people, other sets), I propose that {\em
  groups may be composed only of living things and other groups}.

This claim is made on intuitive grounds.  It simply seems odd to talk
about a group of rocks, or a group of sets.  Talk of groups implies
some sort of activity, and so it is more natural to use `group' to
refer to people and other animals.  Even a `group' of trees is not
wholly bizarre.  And a group---for example, the Special Committee on
Judicial Ethics---might be part of another group---the Committee of
Ethics Committees \citep[145]{uzquiano2004a}.

%% Another point on application conditions: I think that group
%% composition is {\em unrestricted} among living things and groups.
%% That is, for any set of living things and/or groups, there is a
%% group of them.  Any restriction seems arbitrary.

\subsection{How do groups change their members?}
The most important apparent difference between sets and groups is that
while the members of a set are necessarily so, the members of a group
may change over time.  How does this work?

We could think of the group composition operator (the group-builder)
as operating {\em not} on things (living things and groups) but on
things-at-times.  The group-builder for the Supreme Court takes the
various justices during the times of their service and produces the
group---the Supreme Court---from those people-at-times.

One problem with this proposal is that it appears to presuppose {\em
  temporal parts}.  For if the group-builder works in similar fashion
to the set-builder and summation operator, then it operates on {\em
  things}.  If our group-builder is going to operate on
people-at-times, then we seem to commit ourselves to the claim that
people-at-times are {\em things}.  And what things could they be but
temporal parts of other things?

Thinking of groups as being composed of things-at-times rather than
things is unintuitive in any case.  Sandra Day O'Connor \emph{was} a
member of the Supreme Court.  Taking temporal parts seriously would
require us saying instead that her 1981---2006 part \emph{is} a member
of the Supreme Court.  But Sandra Day O'Connor is no longer a member
{\em at all}.

If we don't want to presuppose temporal parts, the group operator has
to be somehow \emph{dynamic}. It can't just take things, compose them
and be done---it has to \emph{add and remove things over time}.

Making sense of a dynamic group operator might allow us to avoid
presupposing {\em eternalism} as well (ultimately it will not).  If
the group-builder made the Supreme Court `in one go', then future
justices would have to already exist in some sense.  How else could
the group-builder operate on them?

The most readily apparent way of making sense of a dynamic operator is
by relativizing the group-builder to times.  We can think of the
operator as taking a set at a time and producing a group: $G = \sum
_{t} (S)$.  Following \citet{uzquiano2004a}, we can say that set $S$
composes group $G$ at time $t$ if and only if:

\begin{enumerate}[label=(\arabic*)]
  \item $\forall x\ (x \in S \leftrightarrow x\ \text{is a member of}\ G
  \  @\ t)$
  \item $\exists x {[} x\ \text{is a member
      of}\ G\ @\ t\ \wedge\ \square ( x \in S )\ \wedge
    \\ \diamond\ \exists t^{\prime} ( G\ \text{exists}\ @
    \ t^{\prime}\ \wedge\ \neg ( x\ \text{is a member
      of}\ G\ @\ t^{\prime} )) {]}\ \vee \\ \exists x^{\prime} {[}
    \neg ( x^{\prime}\ \text{is a member of}\ G\ @\ t)\ \wedge \\ \neg
    \diamond (x^{\prime} \in S ) \wedge \ \diamond \exists t^{\prime
      \prime} (x^{\prime} \text{is a member of}\ G\ @\ t^{\prime
      \prime}) {]}$\ \citeyearpar[150]{uzquiano2004a}
\end{enumerate}

I will assume that group composition is {\em unrestricted}.  That is,
for any people and/or groups at any time $t$, there is a group
composed of them at that time.

\subsection{Identity conditions}
Given that groups can change their parts over time, there will be
cases in which $G = \sum _{t_1} ( S )$ and $G^{\prime} = \sum _{t_2}
( S^{\prime} )$ and $G = G^{\prime}$ and $t_1 \neq t_2$ and $S \neq
S^{\prime}$.  When will this occur?  Under what conditions does $G =
G^{\prime}$?

For example, $t_1$ might be 2004, $t_2$ might be 2012, $S$ might be
\{Rehnquist, Stevens, O'Connor, Scalia, Kennedy, Souter, Thomas,
Ginsburg, Breyer\} and $S^{\prime}$ might be \{Roberts, Stevens,
O'Connor, Scalia, Kennedy, Souter, Thomas, Ginsburg, Breyer\}.  If we
suppose that $G$ is the Supreme Court in 2004 and that $G^{\prime}$
is the Supreme Court, then we want to be able to say that $G =
G^{\prime}$.  

But now we have hit a snag.  For as Uzquiano points out, a set of
people can compose more than one group at a time.  Suppose that all
and only the members of the Supreme Court in 2004 are part of the
Special Committee on Judicial Ethics.  In this case ``the Supreme
Court share[s] all of its members with the Special Committee on
Judicial Ethics as of a certain time [2004]''
\citep[151]{uzquiano2004a}.  It seems false to say that, in 2004, the
Supreme Court was identical with the Special Committee.  But if the
Supreme Court, $G$, is $\sum _{t} ( S )$ and the Special Committee,
$C$, is also $\sum _{t} ( S )$, then how can we deny that $G = C$?

%% One way is to allow that the group-builder can produce numerically
%% distinct groups from repeated applications of the same operation.

The obvious solution is to look at the past and future histories of
$G$ and $C$.  The Supreme Court is composed of $S^{\prime}$ in 2012,
while the Special Committee has been dissolved.  But I do not see how
we can appeal to identity across time without assuming eternalism.  If
we assume eternalism, we can say that $G = \sum ( \mathellipsis , \sum
_{2004}(S), \sum _{2012}(S^{\prime}), \mathellipsis )$.  But if we do
not assume eternalism, we will have to use temporal operators like
\textsc{was} and \textsc{will}.  We can therefore say only that
\textsc{was}($G = \sum (S)$), $G = \sum (S^{\prime})$ and
\textsc{will}($G = \sum (S^{\prime \prime})$).  We are still left with
its being presently the case that $G = C$.  Without assuming
eternalism, we seem forced to admit that, in 2004, the Supreme Court
{\em is} the Special Committee on Judicial Ethics.

The point of positing the existence of groups in addition to sets was
to avoid the identity of the Supreme Court with the Special
Committee.  But positing groups is not sufficient; we also need to
assume eternalism.  If we do not assume eternalism, then groups do not
help us.  Therefore, we should reconsider whether the positing of
groups is necessary.

\subsection{Re-examining the set identity thesis}
\label{set-id}
Given that we cannot seem to distinguish the Supreme Court and the
Special Committee, we might abandon groups altogether and claim that
the Supreme Court is not a group, but a set.  This view, of course,
has difficulties of its own.

The primary motivation cited above for positing groups was the fact
that the Supreme Court changes its members over time.  For example,
both of the following sentences are true:

\begin{enumerate}[label=(\arabic*)]
  \item The Supreme Court ruled on Roe vs.\ Wade in 1973. \label{roe1}

  \item The set of justices now serving as Supreme Court Justices did
    not rule on Roe vs.\ Wade in 1973
    \citep[135]{uzquiano2004a}. \label{roe2}
\end{enumerate}

One way to accommodate these facts is to ``insist that the Supreme
Court is a set, but to abandon the assumption that there is a single
set to which the phrase `the Supreme Court' refers in sentences
\ref{roe1} and \ref{roe2}'' \citep[138]{uzquiano2004a}.  To
successfully use the term `the Supreme Court' to refer to a set of
justices, there must be an implicit or explicit temporal reference.
If an utterance of \ref{roe1} is true it will be true because it the
speaker intends her audience to recognize her intention to refer to
the set of justices that was the Supreme Court in 1973.  If her
audience, for whatever reason, takes her to be referring to the
current Court, then they will evaluate \ref{roe1} as false.

Considered in this light, `the Supreme Court' is used to express a
relation between sets and times; ``$x$ is the Supreme Court at $t$''
\citep[140]{uzquiano2004a}.  There is some precedent for this sort of
interpretation:

\begin{squote}
Our use of the phrase `the Supreme Court' to express a relation a set
of justices bears to a time is much like our use of the phrase `the
president of the United States' to express a relation an individual
bears to a time.  Different persons may be the president of the United
States at different times, but there is at most one person that bears
that relation to each time \citep[138]{uzquiano2004a}.
\end{squote}

``But,'' it will be objected, ``there is an important difference here.
We use both terms---`the Supreme Court' and `the president'---to refer
to a past, present or future set that `is' the thing, but we also use
`the Supreme Court' to refer to {\em the Supreme Court}, which has
changed its membership over time.  If I say, `the Supreme Court has
become more conservative over the past century', there is no one set I
am referring to.  I must be referring to something else; the obvious
candidate is the {\em group} that is the Court.''

One reply here is to claim that all that what ``the Supreme Court has
become more conservative over the past century'' actually means is
that the members of the sets that have been the Supreme Court have
become more conservative.  Another, similar reply is that someone who
utters ``the Supreme Court has become more conservative over the past
century'' is saying something literally false (either because there is
no unique set that is being referred to, or because there is a unique
set referred to, but one that does not make the proposition true), but
can generally be understood to mean something else; namely, that the
members of the sets that have been the Supreme Court have become more
conservative.

Neither reply is {\em very} unintuitive; indeed, there is something
attractive about a thesis that reserves application of adjectives like
`conservative' for people, rather than other things like groups.

But there is a more pressing worry for the set identity thesis.
Recall that the set that is the Supreme Court at a given time might
also be the Special Committee on Judicial Ethics.  We must admit that
the Supreme Court in 2004 is the set \{Rehnquist, Stevens, O'Connor,
Scalia, Kennedy, Souter, Thomas, Ginsburg, Breyer\}, and the Special
Committee in 2004 is that very same set.  But now we are committed to
this argument:

\begin{enumerate}[ref=(\arabic*)]
  \item The Special Committee on Judicial Ethics is one of the
    committees assembled by the Senate.

  \item The Special Committee on Judicial Ethics is identical with the
    Supreme Court.

  \item {\em Therefore} the Supreme Court is one of the committees
    assembled by the
    Senate. \citep[144]{uzquiano2004a} \label{sup-com}
\end{enumerate}

And \ref{sup-com} seems false.

But it may be possible to argue that \ref{sup-com} is not false but
only {\em misleading} (indeed, very misleading).  For it
(conversationally) implies that future sets referred to by `the
Supreme Court' will be identical to future sets referred to by `the
Special Committee'.  And it is {\em this} that is certainly false.

\subsection{Set membership and implicature}
\label{implicate}
Suppose we arrive at a meeting of the Special Committee on Judicial
Ethics.  Rehnquist, Stevens, O'Connor, Scalia, Kennedy, Souter,
Thomas, Ginsburg, and Breyer are sitting around a center table.  As we
take our seats you turn to me and say, ``they look rather familiar,
don't they?''  I say ``that's also the Supreme Court.''

What am I referring to with the demonstrative expression ``that''?  If
one thinks that I am referring to a {\em group}---the Special
Committee---that is distinct from the Supreme Court, my utterance will
have to be interpreted as non-literal.  I will have to be understood
to mean that the {\em members} of the Special Committee are also the
members of the Supreme Court.

Suppose instead that you ask me who the members of the Special
Committee are.  I say ``Rehnquist, Stevens, O'Connor, Scalia, Kennedy,
Souter, Thomas, Ginsburg, and Breyer.  The Special Committee is just
the Supreme Court.''  Here again one could argue that I am speaking
non-literally; what I mean is that the members of the Special
Committee are just the members of the Supreme Court.

Now suppose that the Special Committee is dissolved in 2004.  In 2005,
we see the members of the Supreme Court (still Rehnquist, Stevens,
O'Connor, Scalia, Kennedy, Souter, Thomas, Ginsburg, and Breyer) out
to lunch together.  I point and say ``that was the Special Committee
on Judicial Ethics.''  Now what is the referent of ``that''?  It
cannot be the Special Committee, for that has ceased to be.  It must
either be the Supreme Court or the set \{Rehnquist, Stevens, O'Connor,
Scalia, Kennedy, Souter, Thomas, Ginsburg, and Breyer\}.  Either way,
the proponent of groups will have to interpret this utterance as
non-literal.

Now suppose that the Special Committee is dissolved in 2004 and
Rehnquist retired before dying in 2005 (let's pretend he retired in
May).  Now in August we see Rehnquist, Stevens, O'Connor, Scalia,
Kennedy, Souter, Thomas, Ginsburg, and Breyer out to lunch together.
I point and say ``that was the Supreme Court {\em and} the Special
Committee on Judicial Ethics.''  I can only be referring to the set of
justices.  Why not suppose that I have only {\em ever} been referring
to the set of justices?  If I am in fact referring to the set
\{Rehnquist, Stevens, O'Connor, Scalia, Kennedy, Souter, Thomas,
Ginsburg, Breyer\}, then when I say ``that was the Supreme Court {\em
  and} the Special Committee'', I say something literally true.

I also say something literally true when I say ``the Supreme Court is
one of the committees assembled by the Senate'' or ``the Supreme Court
is the Special Committee on Judicial Ethics''.  But it is very
misleading to say either.  By saying ``the Supreme Court is the
Special Committee'' I imply that future referents of `the Supreme
Court' will be identical to future referents of `the Special
Committee'.  It is less misleading to say ``the current Supreme Court
is the Special Committee on Judicial Ethics''.  (It is even less
misleading to say ``the current Supreme Court is also the Special
Committee''.)

We say above that `the Supreme Court' is sometimes used to refer
(whether literally or non-literally) to more than one set.  When I say
``the Supreme Court has become more diverse'' I mean that the members
of the sets that have been the Supreme Court have become more diverse.
It may be due to this fact that we so easily misinterpret uttered
propositions like ``the Supreme Court is the Special Committee''.  A
listener might take this to mean that the members of the sets that
have been the Supreme Court are identical with the members of the sets
that have been the Special Committee.  They would therefore evaluate
the utterance as false.

%% This is because it is mutually assumed that the set referred to by
%% `the Supreme Court' will change; the current set of justices will
%% not always be the Supreme Court.  It is a contingent fact that one
%% set---\{Rehnquist, Stevens, O'Connor, Scalia, Kennedy, Souter,
%% Thomas, Ginsburg, and Breyer\}---is both the Supreme Court and the
%% Special Committee in 2004, But since the form of the sentence ``the
%% Supreme Court is the Special Committee'' is that of an identity
%% statement, and sincethe audience will understandably

\subsection{Conventional identity conditions}
\label{set-convention}
Even supposing everything above is right, there is still more to be
said.  What are the identity conditions for the Supreme Court?  What
makes it true that one set is the Supreme Court in 2004 and a
different set is in 2012?

What makes it true that a given set is the Supreme Court at a given
time is simply our legal conventions.  The Constitution authorizes the
recognition of a set of justices as `the Supreme Court'.  Which set is
recognized as the Supreme Court is decided by the legislative and
executive branches.  The president nominates a set (the sitting
justices and the nominated justice) and the legislative branch votes.
The outcome of the vote make it true or false that a given set is the
Supreme Court.

This is analogous for all `groups'.  Teams, bands, militias---what
makes it true that a certain set is a team, band or militia is just
the conventions governing the group.  If I desert my militia and the
other members of the militia recognize my absence as a desertion, then
it is understood that I am no longer part of the militia; for that
reason it is then true that the set containing me is no longer the
militia.  A smaller set, not containing me, is now the militia.

It seems, then, that {\em identity conditions over time for groups are
  wholly conventional}.  This is plausible; groups are social
entities, and it makes sense that their composition should be a matter
of social convention.  But if this is true, it suggests something more
radical: that identity conditions over time for physical objects like
chairs are conventional as well.

\section{The conventions of ordinary things}
[Fine's definition of classical mereology; essentialism and its
  parallel with sets; objects are identical with sums; when we use
  `chair' we express a relation between a sum and a time; which sum is
  the referent of chair at any given time is conventional; the
  persistence of any given `chair' over time is conventional.]

%%%%%%
\begin{comment}
\section{Deflationary metaphysics}
Kathrin Koslicki has an interesting objection to universalist theses
such as the one I appear committed to.  Her objection amounts to this:
if every `collection' of objects (such as the London Bridge, a
particle in the moon, and Cal Ripkin, Jr.) is a thing in its own
right, then metaphysics becomes uninteresting.  There is no longer any
debate about whether chairs or dogbushes are more `real' or have a
stronger claim to existence.  They both (obviously) exist, and the
difference between chairs and dogbushes is not ontological but
conceptual: `chair' is more embedded in our talk, and so chairs have
greater importance to {\em us}.  But metaphysically, or ontologically,
chairs and dogbushes are on the same level.  There is no sense in
which chairs exist and dogbushes do not.

In the quoted material below, Koslicki is criticizing a version of
four-dimensionalism that Sider has previously defended.  Sider's
position was that any collection of objects-at-times is a thing in its
own right.  Sider calls these things `fusions'.  For example, a chair
is a fusion of a large number of {\em temporal part} of things (wood
molecules, or atoms, or simples).  Each thing (wood molecule, atom, or
simple) is a fusion of {\em its} temporal parts.  Each temporal part
of the chair is also a thing (a fusion).

I take no stance on whether objects have temporal parts or rather
`endure' through time.  Moreover, if we accept Fine's theory of parts
then we reject the idea that there is just one composition operation;
the operation that produces fusions is one among many.  But Koslicki's
comments are relevant nonetheless:

\begin{squote}
There is room, in Sider's theory, for {\em some} genuine ontological
disagreements: for example, the universalist, the nihilist and the
holder of the intermediary position genuinely disagree over how many
and which fusions that exist.  But the only genuine ontological
disagreements for which there is room, in Sider's world, are ones that
concern disagreements over `bare' fusions, so to speak.  What has
happened to the houses, trees, people, and cars, the familiar concrete
objects of common-sense, whose persistence this account set out to
analyze?  There are no `deep' ontological facts as to whether a given
fusion should count as a house or not\,\ldots

[By claiming that there can be genuine ontological disputes,] Sider is
guilty of a bit of false advertising: his account is really a way of
saying that, at the end of the day, there is no interesting {\em
  ontological} story to be told about the persistence of our familiar
concrete objects of common-sense; whatever there is to say about the
persistence of houses, trees, people and cars concerns the
organization of our conceptual household
\citeyearpar[124--125]{koslicki2003}.
\end{squote}

Koslicki seems to think that we ought to be able to find some
ontological difference between ``the familiar concrete objects of
common-sense'' and things like dogbushes or chairs-at-times.  But as I
remarked above, why should what interests us (familiar objects like
chairs) be a guide to what exists?  The conclusion that ``the
persistence [and other properties] of houses, trees, people and cars
concerns the organization of our conceptual household'' seems to be a
most welcome one.

However, there {\em is} an ontological difference between some things,
if no between chairs and dogbushes.  One lesson of Kit Fine's theory
of parts is that mereological sums are not the only kind of composite
thing.  There are sets as well, and groups, and sequences, and perhaps
infinitely many other types of thing.  The difference between a set
and a sum is an ontological difference.  Within each type, however,
we must rely on our own conceptual `scheme' to organize things.

In section \ref{lessons-v} I considered Jay Rosenberg's claim that
the Special Composition Question is the wrong question to be asking.
Rosenberg's position seems to be that there is {\em no} answer to the
Special Composition Question.  Rather, he thinks what it takes to
`compose' something depends on what that something is---making a chair
is not like making a pie.

This insight of Rosenberg's can be connected with the insights of
Fine's theory.  If we understand `composition' in the Special
Composition Question to mean {\em mereological composition}, then
Rosenberg was wrong if he held that there is no correct answer to the
Special Composition Question.  It seems intuitively true that
mereological composition is unrestricted.  But if take `composition'
in the Special Composition Question to be $K$-composition---any
composition operator at all---then Rosenberg was {\em right} that
there is no answer.  What the application conditions are for a
composition operator depends on {\em which} composition operation is
being applied.  

Moreover, determining these application conditions, and determining
identity conditions, and determining the other properties of these
various composition operations is a task for metaphysics.  The field
is not then so barren as Koslicki seems to have feared.  But it is
true that perhaps the most interesting questions---When are we willing
to call something a chair, and why?  What conditions must be
fulfilled?---are not ontological questions anymore.  They are
questions about our ``conceptual household.''

\end{comment}

\ifstandalone
\end{spacing}
\bibliography{everything}
\bibliographystyle{ChicagoReedweb}
\fi
\end{document}
