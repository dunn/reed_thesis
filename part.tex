\documentclass[11pt]{article}
\usepackage{standalone} \newif\ifstandlone \standalonetrue
\usepackage[left=1.75in, right=1.75in, top=1.25in, bottom=1.25in]{geometry}
\geometry{letterpaper}
\usepackage{graphicx}
\usepackage{enumitem}
\usepackage{amssymb}
\usepackage{amsmath}
\usepackage{epstopdf}
\usepackage{verbatim}
\usepackage{setspace}
\usepackage{natbib}
\setcitestyle{aysep={}}
\usepackage{url}
\usepackage{hyperref}
\synctex=1

\DeclareSymbolFont{symbolsC}{U}{txsyc}{m}{n}
\DeclareMathSymbol{\strictif}{\mathrel}{symbolsC}{74}
\DeclareMathSymbol{\boxright}{\mathrel}{symbolsC}{128}

\newenvironment{squote}{%
\begin{spacing}{1}
\begin{list}{}{%
\setlength{\labelwidth}{0pt}%
\rightmargin\leftmargin%
}
\item\relax
}{%
\end{list}%
\end{spacing}
}

\title{What are the parts of a chair?}
\author{Alexander A. Dunn}
\begin{document}
\ifstandalone
\maketitle
\begin{spacing}{1.5}
\fi

\label{parts}

In the previous section I argued that not only are there ordinary
things like chairs, but that there are more unusual things like
archipelagos, scattered works of art, and perhaps even dogbushes.  I
tried to make it at least plausible to assume that some version of
{\em universalism} is true---that for any material things, there is
some material object made up of them.

But I also argued that in addition to this plurality of `material
objects', there are other {\em kinds} of things, such as teams,
families, and judicial bodies.  For example, in addition to the
material object made up of the justices on the Supreme Court, there is
also the Supreme Court itself.  The former thing is located at each
point at which the justices are located, but it seems odd to say that
the Supreme Court is a large, scattered object.  It is more natural to
say that the Court is a {\em group} of physical objects (the
justices), not a physical object itself.  The same goes for the Reed
College women's rugby team, and my family.  There is at least some
reason to recognize kinds of things in addition to material objects.

How, though, do these new kinds of things differ from material
objects?  The material object made up of the Supreme Court justices
and the Supreme Court itself appear to have the same parts: the
justices.  But we are supposing that they are not the same thing.  If
they do not differ in their parts, they must differ in some other way.

In this section we will assess several different theories.  These
theories have to do with how different kinds of things exist and how
they relate to their parts.  We will see that each theory, in order to
differentiate different kinds of things (like the material object made
up of the justices and the Supreme Court), has the consequence that
there are potentially {\em infinite} different kinds of things, and,
worse, potentially infinite kinds of things in any given location.

A theory that posits such {\em co-located objects} has some merits.
For example, some philosophers claim that a statue is an object
distinct from the clay of which it is made.  A problem for this view
is how to distinguish the statue from the clay, when both have the
same parts (the parallel with the Supreme Court and the justices is
obvious).  The theories that we will examine each solve this problem,
though in different ways.

There are, however, reasons to be suspicious of a theory that entails
such a `plurality' of kinds of objects.  The alternative, which I will
present in section \ref{essential}, is to maintain that there are only
a small number of kinds---just mereological sums and sets, perhaps.
Groups like the Supreme Court may be identified with sets, and
material objects like chairs may be identified with sums.  The
difficulty with this view is that sets do not change their parts over
time, and (given certain assumptions) neither do sums.

If we defend this `essentialist' answer, we will have to interpret
talk about `the same chair' over time as being talk about {\em
  different} sums over time.  The appearance of persistence over time
(and through change) is a product of {\em conventions} that govern
terms like `chair'.  What mereological sum is the referent of `the
chair' from one day to the next depends not upon the actual identity
of the two referents, but upon convention.

\section{Parthood and composition}
\label{parthood}
I argued in section \ref{plural-ref} that things like teams, crews,
and families are indeed {\em things}.  Terms like `team', `crew', and
`family' are not disguised references to plurals.  Moreover, things
like teams are things with {\em parts}.  The rugby players are each
{\em part} of the Reed College women's rugby team.  The team is made
up of---it is composed of---the players.

When I say that the players are part of the team, or that the
crewmembers are part of the crew, or that I am part of my family, is
that use of `part' the same as when I say that the tree is part of the
dogbush, or that the seat is part of the chair?  Are {\em any} of
these uses of `part' the same?

\subsection{Parthood and classical mereology}
\label{tech}
Technical notions of composition are often defined in terms of
parthood, which is itself left undefined.  Peter van Inwagen provides
the following definitions:

\begin{displaymath}
x\ \text{is a part of}\ y =_{df} x\ \text{is a proper part
  of}\ y\ \text{or}\ x = y
\end{displaymath}

It is assumed that everything is a part of itself.  A `proper part' of
some $x$ is a part that is not $x$ itself.

\begin{displaymath}
x\ \text{overlaps}\ y =_{df} \text{For some}\ z, z\ \text{is a part
  of}\ x\ \text{and}\ z\ \text{is a part of}\ y
\end{displaymath}

Just as everything is a part of itself, everything overlaps itself
(when $x = y = z$).

\begin{squote}
$x$ is a mereological sum of the $y$s $=_{df}$ For all $z$ (if $z$ is
  one of the $y$s, $z$ is a part of $x$) and for all $z$ (if $z$ is a
  part of $x$, then for some $w$, ($w$ is one of the $y$s and $z$
  overlaps $w$)) \citeyearpar[618--619]{inwagen2006}.
\end{squote}

The first part of this definition---``if $z$ is one of the $y$s, $z$
is a part of $x$''---specifies that all of the $y$s are part of $x$.
But we want to be able to say that all and {\em only} the $y$s are
part of $x$.  The second part of the definition---``if $z$ is a part
of $x$, then for some $w$, $w$ is one of the $y$s and $z$ overlaps
$w$''---gets us this, by specifying that every part of $x$ overlaps
one of the $y$s.

%% \begin{squote}
%% $x$ is a mereological sum $=_{df}$ For some $y$s, $x$ is a
%%   mereological sum of those $y$s.
%% \end{squote}

There are at least two limitations to this standard formulation of
mereology.  First, it entails that everything that has parts is a
mereological sum.  Second, it does not explain how, if at all,
mereological sums can change their parts.

\subsection{Is everything a mereological sum?}
\label{all-sum}
One problem with the standard formulations of mereology is that---like
van Inwagen's definitions above---they entail that everything that has
parts is a mereological sum.  And though it may be an unreflective
prejudice, I am inclined to believe that mereological sums are {\em
  physical}, or material, things.  Material things certainly have
parts, but they are not the only things:

\begin{squote}
The word `part' is applied to many things besides material objects.
We have already noted that submicroscopic objects like quarks and
protons are at least not clear cases of material objects;
nevertheless, every material object would seem pretty clearly to have
quarks and protons as \emph{parts}, and, it would seem, in exactly the
same sense of \emph{part} as that in which a paradigmatic material
object might have another paradigmatic material object as a part.  A
``part,'' therefore, need not be a thing that is clearly a material
object.  Moreover, the word `part' is applied to things that are
clearly \emph{not} material objects---or at least it is on the
assumption that these things really exist and that apparent reference
to them is not a mere manner of speaking.  A stanza is a part of a
poem; Botvinnik was in trouble for part of the game; the part of the
curve that lies below the x-axis contains two minima; parts of his
story are hard to believe\,\ldots\,such examples can be multiplied
indefinitely \citeyearpar[18--19]{inwagen1995}.
\end{squote}

Under our current conception of a mereological sum, things like poems
seem to be included.  For recall van Inwagen's definition:

\begin{squote}
$x$ is a mereological sum of the $y$s $=_{df}$ For all $z$ (if $z$ is
  one of the $y$s, $z$ is a part of $x$) and for all $z$ (if $z$ is a
  part of $x$, then for some $w$, ($w$ is one of the $y$s and $z$
  overlaps $w$)) \citeyearpar[618--619]{inwagen2006}.
\end{squote}

Stanzas are parts of the poem, as are lines, words, and letters.  For
simplicity's sake, though, let us pretend that only words are parts of
poems.  Let the $y$s therefore be all the words in a poem $x$.

Suppose $z$ is the word `bear'.  The word `bear' is a word in the
poem, so it is one of the $y$s.  The first part of van Inwagen's
definition tells us that `bear' is therefore part of the poem.  Now
take the second part of the definition.  We have established that $z$
(`bear') is part of $x$ (the poem), so the antecedent of the
conditional (``if $z$ is a part of $x$'') is true.  If the poem is a
mereological sum, then the consequent must also be true.  There must
be some one of the $y$s that overlaps $z$.  Since $z$ is one of the
$y$s, and everything overlaps itself, the consequent is true.

We can follow the same steps for every word in the poem.  It seems,
therefore, that poems are mereological sums.  Is this acceptable?

I can think of no principled reason to deny that poems in particular
are mereological sums, but there are good reasons to think that not
{\em everything} is a mereological sum.  This is because there are
some things that are correctly said to be parts of another thing, but
are apparently parts in a {\em different way} than are the parts of
mereological sums:

\begin{squote}
Now, on the face of it, there would appear to be a wide variety of
basic ways in which one object can be a part of another.  The letter
`n' would appear to be a part of the expression `no', for example, and
a particular pint of milk part of a particular quart; and if these two
relations of part are not themselves basic (perhaps through being
restricted to expressions or quantities), there would appear to be
basic relations of part that hold between `n' and `no' or the pint and
the quart.  It is also plausible that the way in which `n' is a part
of `no' is different from the way in which the pint is a part of the
quart.  For if the two ways were the same, then how could it be that
two pints were only capable of composing a single quart, while the two
letters `n' and `o' were capable of composing two expressions, `no'
and `on' \citep[562]{fine2010}?
\end{squote}

The parthood relation for sets is again different.  The set containing
the only the letters `n' and `o' has the letters as parts.  When the
letters are parts of a set, their order is irrelevant, but when the
letters are parts of a word, order matters; hence `no' and `on'.  The
parthood relation for sets is also different from the parthood
relation for quantities (of milk):

\begin{squote}
If four quarts compose a gallon the pints which compose the quarts
will compose the gallon in the same way in which they compose the
quarts, whereas, if four sets compose a further set the members of the
sets will not compose the further set in the same way in which they
compose the component sets.  Thus we would now appear to have three
different basic ways in which one object can be a part of another
(pint/gallon, letter/word, and member/set); and once these cases have
been granted, it is plausible that there will be many more
\citep[562]{fine2010}.
\end{squote}

The standard mereology does not appear to be capable of handling the
parthood relation that applies to sets, or that which applies to
words.  But if these things (sets, words) have parts, and if they are
not sums, then the standard mereology is flawed, for it entails that
everything that has parts is a sum.

\subsection{Can mereological sums change their parts?}
\label{change}
Many philosophers believe that mereological sums cannot change their
parts.  Since many also believe that chairs and other ordinary things
{\em can} change their parts, concerns arise about the utility of
mereological sums.  If chairs and other ordinary things change their
parts, then they are not sums; what then {\em are} sums?

But not all philosophers do believe that sums cannot change their
parts.  Peter van Inwagen is one.  His argument is very
straightforward:  just as it follows from the definition of
`mereological sum' that things like poems are sums, so it follows that
things like chairs are sums.  Things like chairs can change their
parts.  Therefore, sums can change their parts.

This simple argument requires some supplementation, for there is an
(almost) equally simple argument that purports to show that sums {\em
  cannot} change their parts:

\begin{squote}
Consider an object $\alpha$ that is the mereological sum of $A$, $B$,
and $C$ (that is $\alpha = A + B + C$).  We suppose that $A$, $B$, and
$C$ are simples (that they have no proper parts), and that none of
them overlaps either of the others.  And let us suppose that nothing
{\em else} exists---that nothing exists besides $A$, $B$, $C$, $A +
B$, $B + C$, $A + C$, and $A + B + C$.  Now suppose that a little time
has passed since we supposed this, and that, during this brief
interval, $C$ has been annihilated (and that nothing has been created
{\em ex nihilo}).  Can it be that $\alpha$ still exists?  Well, here
is a complete inventory of the things that now exist: $A$, $B$, and $A
+ B$.  And $\alpha$ is none of these things, for, before the
annihilation of $C$, they existed and $\alpha$ existed and $\alpha$
was was not identical with any of them (all three of them were then
proper parts of $\alpha$).  And nothing can become identical with
something else: $x \neq y \rightarrow \square\ x \neq y$; a thing and
another thing cannot become a thing and itself.  We do not, in fact,
have to appeal to any modal principle to establish this conclusion,
for if $\alpha$ were (now) identical with, say, $A + B$, that identity
would constitute a violation of Leibniz's Law, since the object that
is both $\alpha$ and $A + B$ would both have and lack the property
``once having had $C$ as a part'' \citep[628]{inwagen2006}.
\end{squote}

Two assumptions are required for this argument to be valid.  First,
the thing that is the sum of $A$ and $B$ before $C$ is destroyed must
be the same thing that is the sum of $A$ and $B$ after $C$ is
destroyed.  That is,

\begin{squote}
If $A$ and $B$ had a unique mereological sum before the annihilation
of $C$, and if $A$ and $B$ had a unique mereological sum after the
annihilation of $C$, the object that was their sum before the
annihilation of $C$ and the object that was their sum after the
annihilation of $C$ are identical \citep[629]{inwagen2006}.
\end{squote}

This assumption seems very plausible.  If $C$ had not been destroyed,
we would have had little or no inclination to say that the sum of $A$
and $B$ at the earlier time is not identical with the sum of $A$ and
$B$ at the later time.  So I do not see why we should think that {\em
  if} $C$ is destroyed, then the sum of $A$ and $B$ at the earlier
time is not identical with the sum of $A$ and $B$ at the later time.

The second assumption is that composition is unrestricted (that is,
that universalism is true).  Van Inwagen escapes the conclusion that
mereological sums cannot change their parts by denying that
mereological composition is unrestricted.  He denies that for any
things, there is an object composed of them.  In the example above,
therefore, van Inwagen might deny that, before the annihilation of
$C$, there was a sum of $A + B$.  He would therefore be able to
maintain that $\alpha$ loses a part, going from $A + B + C$ to $A +
B$.  There would be no preexisting $A + B$ to compete with.

As I have argued in section \ref{universe}, I think unrestricted
composition---universalism---is true.  Therefore I cannot follow van
Inwagen's escape route, and I conclude that mereological sums, in
their standard formulation, cannot change their parts.

\subsection{Expanding classical mereology}
\label{expand}
Since not everything that has parts seems to be a mereological sum,
and since mereological sums cannot change their parts, there is some
motivation to amend standard mereology to account for these
divergences.

I will look at three different theories that attempt to provide a more
satisfactory mereology.  Section \ref{fine-h} will discuss Kit Fine's
theory of `embodiments'.  Section \ref{fine-c} will discuss Fine's
more recent theory of composition operators.  Section \ref{hovda} will
discuss Paul Hovda's theory of temporal mereology.  What is common to
these theories (and, I suspect, most other modifications of standard
mereology) is that they entail the existence of a plurality of
completely overlapping (co-located) objects.  If such co-location is
unacceptable, we will have to reject all three theories.

\section{First theory: rigid and variable embodiments}
\label{fine-h}
An outline of Kit Fine's `hylomorphic' theory is presented in his
paper ``Things and their parts'' \citeyearpar{fine1999}.  His
objective in that paper is not to present a framework for talking
about things like groups and teams, although what he says can be put
to such a purpose.  Rather, his objective is to present a satisfactory
account of how things persist (or don't persist) over time.

In section \ref{change} I explained why I am assuming that
mereological sums do not change their parts.  Moreover, I am assuming
that mereological composition is {\em unrestricted}---for any
(material) things, there is another thing composed of them (this is
the assumption of universalism that I attempted to motivate in section
\ref{universe}).

Given these assumptions, a mereological sum exists whenever the things
that compose it exist.  The mereological sum of $a, b, c$ exists
whenever (and wherever) $a, b, c$ exist.

Now, however, there are some composite objects that do not seem to
obey these assumptions.  Take a ham sandwich, for example.  It has two
slices of bread and a piece of ham as parts.  It seems to fit the
definition of a mereological sum.  But

\begin{squote}
the sum $a + b + c + \mathellipsis $ will exist {\em whenever} any of
its components $a, b, c, \mathellipsis $ exists (just as it is
located, at any time, {\em wherever} any of its components are
located).  It follows that, under the proposed analysis of the ham
sandwich, it will exist as soon as the piece of ham or either slice of
bread exists.  Yet surely this is not so.  Surely the ham sandwich
will not exist until the ham is actually placed between the two slices
of bread.  After all, one {\em makes} a ham sandwich; and to make
something is to bring into existence something that formerly did not
exist \citep[62]{fine1999}.
\end{squote}

If it is true that the sandwich comes into existence only when the
bread and meat are put together, then the sandwich cannot be a
mereological sum in the classical sense.  How, then, is it composed?

\subsection{Composition relations}
\label{rigid}
Fine's suggestion is that things like the sandwich be seen not merely
as the sum of the bread and meat, but as an object composed of the
bread and the meat {\em standing in the relation of `betweenness'}.
Likewise, a bunch of flowers is not merely the sum of the individual
flowers, but as an object composed of the flowers {\em in the relation
  of being bunched}:

\begin{squote}
Given objects $a, b, c, \mathellipsis $ and given a relation $R$ that
may hold or fail to hold of those objects at any given time, we
suppose that there is a new object---what one may call ``the objects
$a, b, c, \mathellipsis $ in the relation $R$.''  So, for example,
given some flowers and given the relation of being bunched, there will
be a new object---the flowers in the relation of being bunched (what
might ordinarily be called a ``bunch of flowers'')
\citeyearpar[65]{fine1999}.
\end{squote}

The addition of a relation or property that holds of the objects
allows us to distinguish co-located objects like the statue and the
lump of clay.  The statue might be composed of the clay when
possessing the property of being a statue (or just a work of art).
The property of being a statue would be understood to {\em not} apply
to squashed lumps of clay.  This would be a {\em character postulate}
of a statue:

\begin{squote}
The character postulates state what descriptive properties a [thing]
will have---whether, for example, it is red or heavy or much admired.
What these postulates are will depend upon the properties and the sort
of [thing] in question \citep[67]{fine1999}.
\end{squote}

We can apply this theory to groups like the Supreme Court.  The
Supreme Court is a thing composed of people (justices) who stand in
the relation of being jointly endowed with the power of interpreting
the Constitution.  We can specify the character postulates for the
Supreme Court however we like.  Whether the Supreme Court is located
wherever the justices are, whether the weight of the Supreme Court is
equal to the combined weight of the justices or if, strictly speaking,
it weighs nothing at all---the answers to these questions are more or
less up to us.

\subsection{How things change their parts}
\label{h-part}
This proposal by Fine---to think of objects as having parts that stand
in a certain relation---does not allow for things to change their
parts over time.  Fine stipulates that a thing $x$ composed of $a, b,
c$ in relation $R$ exists at a time $t$ if and only if $R$ holds of
$a, b, c$ at $t$.  If $x$ exists at $t_1$, it is because $a, b, c$ are
in $R$ at that time.  If at $t_2$, $a, b, c$ are not in $R$---say that
only $b, c$ are in that relation---then $x$ does not exist.

I am not sure, but I assume that Fine introduces this stipulation
because he is a universalist and believes that composition is
unrestricted.  If composition is unrestricted, then for any things
($z$s) in a relation $R$, there is an object composed of the $z$s in
that relation.  Suppose, as above, that there is an object $x$
composed of $a, b, c$ in relation $R$.  If $a, b$ alone also stand in
$R$, then, {\em if} composition is unrestricted, there is also an
object $y$ composed of $a, b$ in relation $R$.  Objects $x$ and $y$
have different parts and are therefore different things.

Now suppose $c$ is destroyed or somehow no longer stands in $R$ with
$a, b$.  If we assume that composition is unrestricted and that the
object composed of $a, b$ in $R$ before $c$ is destroyed is identical
with the object composed of $a, b$ in $R$ after $c$ is destroyed, then
we cannot say that $x$ has lost a part and is now composed of $a, b$
in $R$.  There is already an object composed of $a, b$ in $R$---the
object $y$.  If we said that $x$ has lost a part, we would be
committed to the claim that $x = y$, even though previously $x \neq
y$.  If $y$ exists, then we must say that $x$ ceases to exist when it
loses a part ($c$).  Fine calls things like these---things that cannot
change their parts---{\em rigid embodiments}.

(Here I am assuming that relations like $R$ are {\em not} fixed
polyadic relations.  That is, there is not one relation $R$ that can
apply to three things---the schema being $Rxyz$---and a different
relation $R^{\prime}$ that can apply to two things---$Rxy$.  Rather, I
am assuming that relations like $R$ have a single variable `slot' that
can accommodate {\em plural variables}.  The schema is something like
$Rx$s, where $x$s is a plural variable that can designate any number
of things.  Therefore it is the {\em same} relation $R$ that applies
to $a$, $b$ and to $a$, $b$, $c$.)

Therefore Fine has a separate proposal for objects that can change
their parts.  These things Fine calls {\em variable embodiments}.
Variable embodiments have, at different times, different {\em rigid}
embodiments as parts.  What part a variable embodiment has at a given
time is determined by a function that assigns rigid embodiments to
times.  Fine illustrates this with the water of a river.  There is the
quantity of water that currently composes the river, but there is also
the `variable' water, that consists of different quantities of water
at different times:

\begin{squote}
I take it that the water in the river in the second sense---what we may
call the variable water---is now constituted by one quantity of water
and now by another. But what is the variable water?\,\ldots

I would like to take the bold step of supposing that there is here a
hitherto unrecognized method by which wholes may be formed from parts.
In the case of the variable water, there is a function, or
``principle,'' that determines which quantity of water constitutes the
variable water at any given time \citeyearpar[68]{fine1999}.
\end{squote}

In effect, {\em the water} of the river---the thing that is the
variable embodiment---is composed of other things---rigid embodiments
that are in turn composed of water molecules.  The water molecules are
not `directly' part of {\em the water}, but they are parts of its
parts.

This theory allows us to explain how a thing like the Supreme Court
changes its members (parts) over time.  When we say ``the Supreme
Court has become more diverse over time'' we are referring to a
variable embodiment that is composed of different rigid embodiments at
different times.  These rigid embodiments are things composed of
justices (for example, Rehnquist, Stevens, O'Connor, Scalia, Kennedy,
Souter, Thomas, Ginsburg, and Breyer) in a certain relation (that of
being part of the Supreme Court).  The rigid embodiment $S =$
(Rehnquist, Stevens, O'Connor, Scalia, Kennedy, Souter, Thomas,
Ginsburg, Breyer)/$R$ exists when and only when those justices stand
in that relation; when Rehnquist died, $S$ ceased to exist.  But the
variable embodiment that is the Supreme Court did not cease to exist;
it was simply no longer composed of {\em that} rigid embodiment.

\subsection{Problems with the first theory}
\label{problems1}
There are two problems with this theory.  First, it has the
consequence that relations (like `being bunched') are actually {\em
  parts} of things (the relation of being bunched is part of the bunch
of flowers).  Second, it produces a plurality of co-located objects.

It is certainly not true that a relation is part of a bunch of flowers
in the same way that the flowers are part of the bunch.  Fine
recognizes this; it constitutes one of his objections to a possible
extension of standard mereology.  He observes that one could claim
that mereological sums are made up of things like bread and meat as
well as {\em tropes}, or relations.  But

\begin{squote}
even if we grant that the trope is a part of the sandwich, it is hard
to believe that it is a part in the same way as the standard
ingredients.  Thus we should not regard the sandwich as a
straightforward mereological sum of $s_1$, $s_2$, $h$, and $r$, but in
some other way that has yet to be made clear \citep[64]{fine1999}.
\end{squote}

Fine's theory of embodiments recognizes relations as parts of things,
but in a different way than things like slices of bread are parts of
things.  This is suggested by his notation for a rigid embodiment of
$a$, $b$, and $c$ in relation $R$: $a, b, c / R$.  But this does not
explain in {\em what} way relations are parts of things.  Moreover, it
just seems false that the relation of being bunched {\em is} a part of
the bunch of flowers in any way.  The relation {\em holds} of the
flowers, and it explains why the flowers are a bunch, but that does
not convince me that the relation is in fact part of the bunch.  In
section \ref{all-sum} we saw examples of many different kinds of
things that have many different kinds of parts.  Tennis matches have
sets, sets have members, poems have stanzas, stanzas have lines, lines
have words.  But {\em relations} were not included in this catalog of
parts.  A theory that has the consequence that relations are parts is,
at least, unintuitive.  (Fine's theory can be modified to avoid this
consequence, as we will see in section \ref{fine-c}.)

The second problem with Fine's theory is one that will plague all
three theories: it posits a plurality of co-located objects.  Suppose
$a$, $b$, and $c$ are pieces of clay that have been formed into a
statue.  There is an object $S = ( a, b, c / R )$, where $R$ is the
relation of being formed into a statue.  $S$ is a statue.  But there
is also an object $L = ( a, b, c / R^{\prime} ) $, where $R^{\prime}$
is the relation of being lumped together.  $L$ is a lump of clay.  If
the statue is squashed, $R^{\prime}$ (the relation of begin lumped)
still holds of $a$, $b$, and $c$, but $R$ (the relation of being
formed into a statue) does not.  $R$ and $R^{\prime}$ therefore have
different {\em persistence conditions} with regard to $a$, $b$, and
$c$.

For every relation $R^{\prime \prime}$ that has different persistence
conditions with regard to $a$, $b$, and $c$, there is an object
composed of $a$, $b$, and $c$ in the relation $R^{\prime \prime}$.
Fine illustrates how this might occur with people as well as lumps:

\begin{squote}
An especially important class of cases are those in which the
principle of embodiment is a property $P$ rather than a polyadic
relation $R$.  The rigid embodiment is then of the form ``$a/P$'' and
may be read as ``$a$ qua $P$'' or as ``$a$ under the description
$P$.''  An airline passenger, for example, is not the same as the
person who is the passenger since, in counting the passengers who pass
through an airport on a given weekend, we may legitimately count the
same person several times.  This therefore suggests that we should
take an airline passenger to be someone under the description of being
flown on such and such a flight.  And similarly for mayors and judges
and other ``personages'' of this sort \citeyearpar[67--68]{fine1999}.
\end{squote}

One might take this to be an unacceptable consequence of Fine's
theory.  For persons can think, and airline passengers can think as
well.  Are we therefore being asked to accept that there are at least
{\em two} thinking things in every seat on the airplane?

This objection, however, comes from confusing rigid and variable
embodiments.  Rigid embodiments, like the person-as-passenger, cannot
change their parts.  As soon as the person-as-passenger loses {\em
  any} of its parts, it ceases to exist.  I think Fine would say that
rigid embodiments, because they cannot undergo change, cannot properly
be said to think.  Things that {\em do} think are variable
embodiments; for example, the human person that at one time is
composed of some of the same parts as the person-as-passenger (but not
all of the same parts, for the person-as-passenger has a relation as a
part).  If it is only variable embodiments that can think, then a
variable embodiment overlapped by one or more rigid embodiments cannot
result in co-located thinkers.

Moreover, if functions are identified by their assignments of things
to times---that is, extensionally---then there may not be {\em always}
co-located variable embodiments.  If there are no two functions that
assign the very same things to the very same times, then there can be
no co-located thinkers.  (But pluralities of variable embodiments will
overlap at any given time.)

However, even if there cannot be co-located thinkers---thinkers who
completely overlap---why can't there be partially overlapping
thinkers?  For example, Fine's theory may well predict the existence
of a variable embodiment that is composed of the various rigid
embodiments of Alex-as-passenger during a particular flight.  (That
is, since I change some of my parts during a flight, there are a
number of different rigid embodiments that may be called
Alex-as-passenger.  Then the question is whether there is a variable
embodiment composed of each of these rigid embodiments in turn.)  If
there is such a variable embodiment, why shouldn't we expect {\em it}
to think?

I think Fine will have to simply deny that such a thing could think.
There are a number of reasons that may be appealed to: the thing does
not have the right sort of history (it is at best a `restriction' of
me---the real thinking thing), or there is a better candidate (me) for
being the one and only thinking thing in that location.

In any case, it seems simply bizarre that by boarding an airplane I
thereby cause a new thing to come into existence.  If I become a
judge, then according to Fine, a new {\em thing} has come into
existence.  Why not just say that a description is true of me that was
once not true of me?  For

\begin{squote}
suppose that Mary got married at noon.  Her marrying did not make a
wife come into existence: it merely made her become a wife.  Your
reaching the age of 20 did not make a teenager go out of existence; it
merely made you cease to be a teenager.  And so on
\citep[151]{thomson1998a}.
\end{squote}

This strange consequence---that passengers are things distinct from
people---coupled with the `explosion of reality', is cause for
concern.  I have in fact understated the size of the explosion, for in
addition to the pluralities of co-located rigid embodiments, there is
no doubt also a plurality of variable embodiments, each corresponding
to a possible function.

But these consequences are not limited to our first theory.  The
second theory, as we will see, results in a similar explosion.

\subsection{Fine's new theory}
\label{new-old}
Recently Kit Fine has proposed a new analysis of things.  In ``Toward
a theory of part'' \citeyearpar{fine2010}, he suggests that not only
are there a plurality of mereological sums, but that there is a
plurality of {\em kinds of things}; sums are only one kind in a vast
``mereological firmament''.  Fine's theory is extremely interesting,
but ultimately it faces a particularly acute version of the problem of
co-location that faces the other two theories.  For while Fine's
theory of embodiments and Hovda's theory of tensed mereology (section
\ref{hovda}) predict a plurality of overlapping things, Fine's theory
predicts in addition a plurality of {\em kinds} of things.

\section{Second theory: composition operators}
\label{fine-c}
Fine has recently published a theory of parthood that has many
connections with his theory of {\em rigid embodiments} (see section
\ref{rigid}).  That theory, while including relations as parts of
things, did not explain how different things (like words, or sets)
have their parts in different ways.  But (as I claim in section
\ref{all-sum}) this seems to be the case.  The way that letters are
parts of words is different from the way members are parts of sets,
and both are different from the way things are parts of sums.  Fine's
new theory begins by emphasizing this `pluralist' claim about
parthood.

\subsection{Problems for pluralists}
\label{sets}
There are a number of objections to Fine's pluralism about parthood.
The first objection is that while parthood is supposed to be
transitive, the membership relation of sets is not.  The letter `n' is
a member of the set \{`n',\{`n',`o'\}\}, but `o' is not.  The
objection claims that sets have {\em members}, not parts, and that
Fine has confused the two.

But while it is true that the membership relation is not the parthood
relation, this is no reason to think that sets do not have parts.  A
given set will have certain members---the $x$s---and certain
parts---the $y$s---and only sometimes will the $x$s and the $y$s be
the very same things.  The set \{`n',\{`n',`o'\}\} has two members
but three parts.  The parthood relation for sets can even be defined
in set-theoretic terms:

\begin{squote}
It may well be thought that the way in which a member is a part of a
set is given, not by the membership relation itself, but by the
ancestral of the membership relation, where this is the relation that
holds between $x$ and $y$ when $x$ is a member of $y$ or a member of a
member of $y$ or a member of a member of a member of $y$, and so on
\citep[563]{fine2010}.
\end{squote}

A second objection is that talk of parthood in connection with things
like sets is somehow metaphorical or non-literal.  We saw above that
van Inwagen admits that many different things are said to have parts.
However, he qualifies this in two ways.  First, he seems to have
doubts (or at least is sympathetic with those who have doubts) as to
whether the non-material things that are said to have parts really
exist:

\begin{squote}
The word `part' is applied to things that are clearly \emph{not}
material objects---or at least it is on the assumption that these
things really exist and that apparent reference to them is not a mere
manner of speaking \citep[19]{inwagen1995}.
\end{squote}

If there are no such things as tennis matches or poems or papers, then
of course they do not have parts.  But I think it is obviously true
that there are such things.  This being so, what does it mean to say
that they have parts?  This is where van Inwagen's second
qualification comes in.  For he suggests not only that the `parts' of
tennis matches and poems are parts in a different way than are the
parts of a table, but that these different relations of parthood are
only tenuously connected.  Van Inwagen says that the various relations
of parthood (if such there be) are connected only by the ``unity of
analogy'' \citeyearpar[19]{inwagen1995}.  If the only similarity
between the parthood relation for poems and the parthood relation for
chairs is that they share the `analogy' of parthood, then is there
anything important or interesting about `parts' of poems?  Is the
parthood relation for sets likewise only interesting because of the
analogy with the parthood relation for chairs?

At least in the case of parthood for sets, the notion does not appear
to be wholly metaphorical:

\begin{squote}
In the case of set-membership, there would appear to be nothing that
might plausibly be taken to indicate that the talk of part-whole is
not to be taken literally. A set is indeed composed of or built up
from its members, and we should add that we may meaningfully
talk---and in the intended way---of \emph{replacing} one member of a
set with another.  Thus Aristotle in the set \{Plato, Aristotle\} may
be replaced with Socrates to obtain the set \{Plato, Socrates\}, with
the given set becoming a different set from what it was. In the case
of sets, our conception of members as parts seems to extend all the
way \citep[564]{fine2010}.
\end{squote}

But the second worry raised by van Inwagen remains.  Why should we
think that there is any {\em real} similarity between these different
parthood relations, other than the fact that we call them all
`parthood'?

\subsection{Operationalism}
\label{operation}
Fine's theory of {\em operationalism} helps answer this worry.
Various {\em operations} produce different things---mereological
summation produces mereological sums or fusions, the set-builder
produces sets, and so forth.  Parts are therefore {\em things} that
have been `combined', through one or more such operations, into a
single {\em thing}.  What is common to all parthood relations is that
from each set of parts is produced a {\em whole} by means of a
composition operator.  From parts (letters, atoms) are made something
else (a word, a set, a chair).  What ties together all the ways of
being a part is that they are involved in a composition operation that
produces a single thing from a number of things:

\begin{squote}
In formulating the principles of mereology, it has been usual to take
the relation of part-whole or some associated relation (such as
overlap) as primitive.  But I believe that, in formulating a more
general theory, it is important to take the operation of composition
as primitive rather than the more familiar relation of part-whole.  In
the case of classical mereology, the operation of composition will
take some objects into the sum or fusion of those objects, while, in
the set-theoretic case, it will take some objects into the set of
those objects; and, in general, the operation of composition will be
the characteristic means (summation, set-builder, and so on) by which
a given kind of whole is formed from its parts \citep[565]{fine2010}.
\end{squote}

Each way of being a part can then be defined in terms of the related
composition operation:

\begin{squote}
Once given a compositional operation, a corresponding relation of part
may be defined in two steps.  We say first that $x$ is a {\em
  component} of $y$ if $y$ is the result of applying $\sum$ to $x$ or
to $x$ and some other objects.  In other words, $y$ should be of the
form $\sum (x_{1}, x_{2}, \mathellipsis )$, where at least one of
$x_1$, $x_2, \mathellipsis$ is $x$.  Thus when $\sum$ is mereological
summation the components of an object will be mere parts, and where
$\sum$ is the set-builder the components of an object will be its
members.  We may then define $x$ to be a part of $y$ if there is a
sequence of objects $x_1$, $x_2, \mathellipsis x_n$, $n$
\textgreater{} $0$, for which $x = x_1$, $y = x_n$, and $x_i$ is a
component of $x_{i+1}$ for $i = 1$, $2, \mathellipsis, n-1$. The parts
of an object are the object itself, or its components, or the
components of the components, and so on \citep[567--568]{fine2010}.
\end{squote}

The parthood relation for mereological sums can therefore be shown to
exhibit reflexivity, transitivity and anti-symmetry:

\begin{description}
\item[Reflexivity] Each object is a part of itself.
\item[Transitivity] If $x$ is a part of $y$ and $y$ of $z$, then $x$
  is a part of $z$.
\item[Anti-symmetry] $x$ is a part of $y$ and $y$ of $x$ only when $x
  = y$ \citep[568]{fine2010}.
\end{description}

But not all definitions of parthood that issue from a composition
operator will exhibit these features:

\begin{squote}
When the underlying operation is summation, each object will be a part
of itself, since the unit sum of any object is the object itself, but
when the underlying operation is the set-builder, no object will be a
part of itself, since no object is ever an ancestral member of itself
\citep[569]{fine2010}.
\end{squote}

In every case, how some thing is part of a whole (if it is) will
depend on the composition operation that produced the whole.  Other
properties, both of a whole and its parts, will be determined by the
nature of the composition operator that produced it.  Each composition
operation will, according to Fine, be governed by various principles.
The `formal principles' govern when composition occurs and when two
products of a composition operation are identical.  The `material
principles' govern both how the object `sits' in space and
time---whether it has spatial and/or temporal parts (see section
\ref{4d}) or not---and the specific characteristics of the object
(such as its color and weight).

\subsection{Fine's pluralist account of classical mereology}
\label{classical}
Of the principles sketched above, Fine gives most attention to the
identity conditions for composition operations.  The composition
operation used as a paradigm is the summation operation of classical
mereology.  Fine's exposition of identity conditions for sums relies
on the notion of `regularity':

\begin{squote}
Call an identity condition $s = t$ {\em regular} if the variables
appearing in $s$ and in $t$ are the same.  Thus $\sum (x, y) = \sum
(y, x)$ is regular while $\sum (x, y) = x$ is not
\citeyearpar[572]{fine2010}.
\end{squote}

With this notion in hand, Fine proposes this condition for identity of
sums:

\begin{description}
  \item[Summative Identity] $s = t$ whenever `$s = t$' is a regular
    identity \citeyearpar[572]{fine2010}.
\end{description}

One particularly interesting aspect of this condition is that it
entails four more principles of the summation operation:

\begin{description}
  \item[Absorption] $\sum (\mathellipsis, x, x, \mathellipsis,
    \mathellipsis, y, y, \mathellipsis, \mathellipsis = \sum (
    \mathellipsis, x, \mathellipsis, y, \mathellipsis )$;
\item[Collapse] $\sum (x) = x$;
\item[Leveling] $\sum (\mathellipsis, \sum (x, y, z, \mathellipsis ),
  \mathellipsis, \sum (u, v, w, \mathellipsis ), \mathellipsis ) \\ =
  \sum (\mathellipsis, x, y, z, \mathellipsis, \mathellipsis, u, v, w,
  \mathellipsis, \mathellipsis )$;
\item[Permutation] $\sum (x, y, z, \mathellipsis ) = \sum (y, z, x,
  \mathellipsis )$ (and similarly for all other permutations)
  \citep[573]{fine2010}.
\end{description}

We can define other compositional identity criteria (e.g., sequences)
in terms of which of these principles apply to their compositional
operation.  But we may also devise new principles by which we may then
define new types of composition:

\begin{squote}
We should note that there would appear to be no good reason to require
that the defining principles for the various operations should be
limited to the particular principles (C [collapse], L [leveling], A
[absorption], and P [permutation]) that we used in characterizing
sums; for any set of regular identities would appear to be equally
well suited to defining a basic form of composition, so long as they
conform to Anti-cyclicity.  Indeed, I would conjecture that any such
set of principles in fact will correspond to a form of composition and
a corresponding form of whole.  How the resulting forms of composition
and whole might be organized is an interesting question, but it should
be apparent that the approach will lead to an infinitude of forms of
composition, each differing from one another in how exactly the
identity of the resulting wholes is to be
determined. \citep[575--576]{fine2010}.
\end{squote}

It is at this point that the importance of Fine's theory becomes
obvious.  Above I stressed that things like teams and families are
really {\em things}; moreover I made this claim as part of an attempt
to motivate a sort of universalistic outlook on metaphysics.  I argued
that the term `composition' was potentially misleading, but that it
was nevertheless correct to say that things like dogbushes, wish
sandwiches, and teams are composed of their parts.  But now it is
apparent that `composition' will mean something different when applied
to each of these things.  Each thing will be the product of a
different composition operation.

Fine's theory reveals new {\em kinds} of universalism.  One might be
committed to the existence of dogbushes---and so to unrestricted
mereological composition---but deny the existence of teams, groups,
crews, and families.  Or one might defend unrestricted composition of
groups while claiming a restriction on mereological composition.

\subsection{Hybrid parts}
\label{hybrid}
The idea that there are different ways of being a part, corresponding
to the different kinds of things produced by different composition
operators, allows us to solve certain puzzles about parthood.

For example, I am the only member of my singleton (the singleton of
$x$ is the set resulting from applying the set-builder to $x$ alone).
My hand, for instance, is not a member of my singleton.  But my hand
is a part of me.  If I was a part of my singleton, then---because
parthood is transitive---my hand would be a part of my singleton.  And
if that means that my hand is a {\em member} of my singleton, that is
clearly wrong.

Fine points out, of course, that the objection makes the mistake of
supposing that something (me, my hand) can be a part in only one way
(in this case, through set-membership).  Once we recognize that there
are a plurality of ways of being a part, it becomes clear that my hand
is part of the set in one way, but not in another:

\begin{squote}
Given the specific relations of part, we may derive various {\em
  hybrid} relations of part.  Suppose, for example, that we are given
the relations of set-theoretic and mereological part---which we may
designate as \textepsilon -part and $m$-part. We may then take one
object to be an \textepsilon ,$m$-part of another if it is an
\textepsilon -part or an $m$-part or an $m$-part of an \textepsilon
-part or an \textepsilon -part of an $m$-part, or an $m$-part of an
\textepsilon -part of an $m$-part, and so on. More generally, if $K$
is a family of specific ways of being a part, we may take an object to
be a {\em K-part} of another if $x$ and $y$ can be linked by
relationships of $k$-part for $k$ in $K$ \citep[579]{fine2010}.
\end{squote}

My hand is a \textepsilon ,$m$-part of my singleton, but not a
\textepsilon -part.

By conjoining every way of being a part, we arrive at the most general
notion of part:

\begin{squote}
Among the hybrid relations of part, of special interest is the
relation of $K$-part where $K$ is the family of {\em all} the specific
ways of being a part.  This is the relation of $K$-part that holds
between two objects when they may be linked by relationships of
$k$-part without restriction on $k$.  We might call it the {\em
  general} relation of part, and it is a relation that holds between
$x$ and $y$ whenever $x$ is in any way whatever a part of $y$
\citep[580]{fine2010}.
\end{squote}

So in addition to saying that my hand is a \textepsilon ,$m$-part of
my singleton, it is also a $K$-part of my singleton.

%% Related to this notion of hybrid parts is the distinction Fine draws
%% between {\em parts} and {\em components}.  To illustrate this, take
%% the set \{Socrates, \{Socrates, Plato\} \}.  This set was produced by
%% applying the set-builder to Socrates and \{Socrates, Plato\}.  These
%% two objects are the {\em components} of the set: ``$x$ is a {\em
%%   component} of $y$ if $y$ is the result of applying $\sum$ to $x$ or
%% to $x$ and some other objects'' \citep[567]{fine2010}.  A {\em part},
%% as 

\subsection{Generating kinds}
\label{generate}
On this theory, what kind a thing is depends on what operation
produced it.  If a chair or a dogbush is a mereological sum, then this
is because they are produced by the summation operation.  The Dunn
family is `produced' by the family operation.  Groups are produced by
the group operation (see section \ref{group}).

But there is a difficulty to be avoided here.  As we saw in section
\ref{classical}, the mereological sum of a single thing $x$ is just
$x$.  Therefore there is a sense in which every physical thing,
including every simple, is a mereological sum, for the application of
the summation operation would just produce that thing.  To avoid this
consequence Fine introduces the notion of a {\em generative}
application of an operation:

\begin{squote}
We might say that the application $y = \Gamma (x_1, x_2, x_3,
\mathellipsis )$ of an operation $\Gamma$ is {\em generative} if there
is an explanation of the identity of $y$ as $\Gamma (x_1, x_2, x_3,
\mathellipsis )$; and we might say that the operation $\Gamma$ is
itself {\em generative} if it permits a generative application. Thus
both the set-builder and the operation of predication will be
generative in this sense \citeyearpar[582]{fine2010}.
\end{squote}

Whether or not the summation operation is generative depends on the
things it is being applied to.  When summing a dog and a tree, it is
generative; when summing a dog by itself, it is not.

For any operation, there will be things it applies to that it cannot
produce.  The summation operator fuses simples, but cannot produce
them; the set-builder combines many things that it cannot produce
(like letters).  For any given operation, there is a `level 0'
consisting of the things that the operator itself cannot produce:

\begin{squote}
We suppose that certain objects are simply given.  These are the
objects whose identity does not require an explanation in terms of
$\Gamma$.  Thus, when $\Gamma$ is the set-builder, they are the
objects that are not sets and, when $\Gamma$ is summation, they are
the objects that are not sums or, rather, the objects that do not need
to be seen as sums.

We now `generate' objects in stages.  At stage 0 are the givens; at
stage 1, we add the objects that result from a single application of
the generative operation $\Gamma$ to the givens \citep[583]{fine2010}.
\end{squote}

An application can now be identified as generative in a strong or a
weak sense:

\begin{description}
  \item[Strong generative application] Also called `strict' by Fine, a
    ``[strong] generative application of $\Gamma$ to the objects $x_1,
    x_2, \mathellipsis$ can now be defined as one in which $y = \Gamma
    (x_1, x_2, \mathellipsis )$ is of a higher level than each of
    $x_1, x_2, \mathellipsis$'' \citeyearpar[584]{fine2010}.  For
    example, summing the simples $x$ and $y$ to produce the fusion $z$
    would be a strong generative application of the summation
    operator; the simples are level 0 and $z$ is level 1.  Summing two
    composites, or a composite and a simple, would not be strongly
    generative; one or both of the parts would be the same level (1)
    as the product.
  \item[Weak generative application] To illuminate this notion Fine
    introduces another, that of a {\em putative generative
      application}: ``Let us say, in the first place, that $y = \Gamma
    (x_1, x_2, \mathellipsis )$ is a putative generative application
    of $\Gamma$ if $y$ is of a higher or of the same level as each of
    $x_1, x_2, \mathellipsis$.  This gives us the notions of a
    putative prior component and of a putative prior in the usual way.
    We now say that the application $y = \Gamma (x_1, x_2,
    \mathellipsis )$ of $\Gamma$ is a {\em weak} generative
    application if it is the putative generative application and if
    $y$ is not putatively prior to any of $x1, x2, \mathellipsis$.  We
    can get from $x_1, x_2, \mathellipsis$ to $y$ without an ascent in
    level but not from $y$ to any of $x_1, x_2, \mathellipsis$''
    \citeyearpar[584]{fine2010}.
\end{description}

Applying the summation operator to a simple is neither strongly nor
weakly generative.  It is not strongly generative because the result
is a simple, which is at level 0---the same level as its part
(itself).  It is not weakly generative because the result of the
operation is putatively prior to its parts.

\subsection{How do things change their parts?}
\label{c-change}
In Fine's theory of embodiments (section \ref{fine-h}) he recognizes
at least two kinds of things: rigid and variable embodiments.  Rigid
embodiments have their parts `timelessly'.  They exist when and only
when their parts exist, and at all times during which they exist, they
have the same parts.  Rigid embodiments, therefore, cannot change
their parts.  Variable embodiments {\em can} change their parts,
however; what rigid embodiment a given variable embodiment is composed
of at a given time is determined by a function.

Fine's account of composition operators explains how the create things
that, like rigid embodiments, do not (and presumably cannot) change
their parts.  He does not address how composition operators might
produce things that, like variable embodiments, {\em can} change their
parts over time; he opens his paper on composition operators by saying
that ``it is not [his] aim to discuss either the notion of relative
part or its connection with the absolute notion''
\citeyearpar[559]{fine2010}.  However, I think we can imagine a few
ways in which Fine's theory of composition operators might be adapted
to relative or temporary parthood.

%%%%%%%

We could think of the group composition operator (the group-builder)
as operating {\em not} on things like people but on things-at-times.
The group-builder for the Supreme Court takes the various justices
during the times of their service and produces the group---the Supreme
Court---from those people-at-times.

One problem with this proposal is that it appears to presuppose {\em
  temporal parts} (see section \ref{4d}).  For if the group-builder
works in similar fashion to the set-builder and summation operator,
then it operates on {\em things}.  If our group-builder is going to
operate on people-at-times, then we seem to commit ourselves to the
claim that people-at-times are {\em things}.  And what things could
they be but temporal parts of other things?

Thinking of groups as being composed of things-at-times rather than
things is unintuitive in any case.  Sandra Day O'Connor \emph{was} a
member of the Supreme Court.  Taking temporal parts seriously would
require us saying instead that her 1981--2006 part \emph{is} a member
of the Supreme Court.  But, intuitively, Sandra Day O'Connor is no
longer a member {\em at all}.

If we don't want to presuppose temporal parts, the group operator has
to be somehow \emph{dynamic}. It can't just take things, compose them
and be done---it has to \emph{add and remove things over time}.

Making sense of a dynamic group operator might allow us to avoid
presupposing {\em eternalism} as well.  If the group-builder made the
Supreme Court `in one go', then future justices would have to already
exist in some sense.  How else could the group-builder operate on
them?

One way to make sense of a dynamic operator is by relativizing the
group-builder to times.  We can think of the operator as taking a set
at a time and producing a group: $G = \sum _{t} (S)$.  (There are two
interpretations of $\sum _{t}$: we might say that the composition
operator as (re-)producing a group at a number of different times $t$,
or we might say that there is a {\em different} composition operator
at each time $t$.  I will suppose that the former is correct.)

A second way to make sense of a dynamic operator is the way that Fine
makes sense of variable embodiments.  Variable embodiments were
composed of different things at different times according to a
function.  Likewise, a composition operator that produces a thing that
has different parts might do so by means of a function.  Rather than
operating directly on some things, the operator could apply to a
function.  Instead of

\begin{displaymath}
\sum (a, b, c, \mathellipsis )
\end{displaymath}
we would have something like this:

\begin{displaymath}
\sum ( f )
\end{displaymath}

On this understanding of a `variable builder', the only {\em
  component} (see section \ref{operation}) of the object is the
function, but at any given time it has as parts (in some sense)
whatever objects the function assigns to that time.

%% Following \citet{uzquiano2004a}, we can say that set $S$
%% composes group $G$ at time $t$ if and only if:

%% \begin{enumerate}[label=(\arabic*)]
%%   \item $\forall x\ (x \in S \leftrightarrow x\ \text{is a member of}\ G
%%   \  @\ t)$
%%   \item $\exists x {[} x\ \text{is a member
%%       of}\ G\ @\ t\ \wedge\ \square ( x \in S )\ \wedge
%%     \\ \diamond\ \exists t^{\prime} ( G\ \text{exists}\ @
%%     \ t^{\prime}\ \wedge\ \neg ( x\ \text{is a member
%%       of}\ G\ @\ t^{\prime} )) {]}\ \vee \\ \exists x^{\prime} {[}
%%     \neg ( x^{\prime}\ \text{is a member of}\ G\ @\ t)\ \wedge \\ \neg
%%     \diamond (x^{\prime} \in S ) \wedge \ \diamond \exists t^{\prime
%%       \prime} (x^{\prime} \text{is a member of}\ G\ @\ t^{\prime
%%       \prime}) {]}$\ \citeyearpar[150]{uzquiano2004a}
%% \end{enumerate}

%% I will assume that group composition is {\em unrestricted}.  That is,
%% for any things at any time $t$, there is a group composed of them at
%% that time.

\subsection{Problems with temporally relativized operators}
\label{problems2a}
Suppose we take the first suggestion and relativize the group operator
to a time.  The primary problem with this is that it leads to a great
plurality of co-located objects.  What is particularly objectionable
in this case is that the objects are all of different {\em kinds}.

First we should recognize that a set of people can compose more than
one group at a time.  Suppose that all and only the members of the
Supreme Court in 2004 are part of the Special Committee on Judicial
Ethics.  In this case ``the Supreme Court share[s] all of its members
with the Special Committee on Judicial Ethics as of a certain time''
\citep[151]{uzquiano2004a}.  It seems false to say that, in 2004, the
Supreme Court was identical with the Special Committee.  But if the
Supreme Court, $G$, is $\sum _{t} ( S )$ and the Special Committee,
$C$, is also $\sum _{t} ( S )$, then how can we deny that $G = C$?

It seems apparent that the Supreme Court and the Special Committee
have different properties.  The Supreme Court can interpret the
Constitution and the Special Committee cannot.  They two groups have
different {\em modal} properties as well.  If we look at {\em
  possible} past and future histories of the Supreme Court and the
Special Committee, we see that it is not necessary that they always be
composed of the same members.

But in order to have such different properties, especially modal
properties, the Supreme Court and the Special Committee must be
produced by means of different operators.  The Supreme Court will be
the product of some operator $\sum _{sc}$ and the Special Committee of
$\sum _{sp}$.  And since these two things are the products of
different operators, they need not be identical.

But just as sets and sums are different kinds of things, so the
Supreme Court and the Special Committee must now be recognized as not
merely different groups, but as different {\em kinds}.  There may be a
greater resemblance between their two kinds than there is between
things like sums and sets, but ultimately they have been estranged.
Calling both `groups' is simply categorizing both kinds under a common
label.

Is there anything wrong with this conclusion?  It does seem bizarre in
some ways.  For it is clear enough that one person (or group) may be a
member of an infinite number of groups; each of these `groups' will
therefore be a product of a different composition operator.  And each
will, strictly speaking, be a different kind of thing.

Previously we had a relatively tidy ontology.  There were sums, and
sets, and other well-known kinds; but now each task force or
subcommittee is potentially a kind unto itself.  Fine recognizes that
his approach ``will lead to an infinitude of forms of
composition\,\ldots a vast mereological firmament''
\citeyearpar[576]{fine2010}.  But he does not consider this to be a
drawback.

However, Fine's theory also seems to have the consequence that
ordinary things---chairs, statues, lumps of clay, and possibly even
people---are different kinds of things.  And at least in the case of
the statue and the lump, they seem to be co-located things.

To see why this is so, first we may observe that ordinary things do
not appear to be sums.  Sums, like sets, cannot change their parts
(see section \ref{change}).  Ordinary things like statues, however,
{\em can} apparently change their parts.  If I break the nose of my
statue, I can replace it.  Do I thereby have a new statue?  It seems I
do not; rather, I have fixed my old statue.

At this point one might say that statues are therefore not
mereological sums, and posit a type of entity that includes statues
and {\em can} change its parts over time.  Perhaps a statue would then
be composed by a sum at a time.  We could relativize the composition
operator for statues to times.  The `object-builder' would then be an
operator that takes sums at times and produces an object---in this
case, a statue: $O = \sum _{t} ( S )$.

But now trouble arises.  For it seems plausible that some things might
compose more than one object.  Just as the members of the set
\{Rehnquist, Stevens, O'Connor, Scalia, Kennedy, Souter, Thomas,
Ginsburg, Breyer\} composed both the Supreme Court and the Special
Committee on Judicial Ethics in 2004, so some things might compose
both a statue and a lump of clay.  Just as the Supreme Court has
different powers and a different history than the Special Committee,
so the statue and the lump have (apparently) different properties and
histories.  The statue cannot survive a squashing; the lump can.  The
lump may have been formed on Monday; the statue might not have been
shaped until Tuesday.

It would be natural to assume that the lump is also produced by means
of the object-builder: $O^{\prime} = \sum _{t} ( S )$.  But now we
face the same problems that plagued the group operator.  Above, we
produced the statue by means of the same object-builder operation: $O
= \sum _{t} ( S )$.  How can we deny that $O = O^{\prime}$, that the
statue is identical with the lump?

We can perhaps make this point more clearly by recalling Fine's notion
of $K$-part from section \ref{hybrid}.  The $K$-part is ``the {\em
  general} relation of part\,\ldots a relation that holds between $x$
and $y$ whenever $x$ is in any way whatever a part of $y$''
\citep[580]{fine2010}.  The very same things (atoms, bits of clay)
that are $K$-parts of the statue are $K$-parts of the lump.  These
things are $K$-parts of at least two different objects.  The following
argument suggests itself:

\begin{enumerate}
  \item Suppose $x \neq y$ but $\forall z$\,($z$ is a $K$-part of $x$
    if and only if $z$ is a $K$-part of $y$)
  \item Then for some $\sum_{i}$ and some $\sum_{j}$, $\sum_{i} \neq
    \sum_{j}$ and $x = \sum_{i}(a, b, c, \mathellipsis ,)$ and $y =
    \sum_{j}(a, b, c, \mathellipsis , )$
\end{enumerate}

%% I say ``perhaps'' because several qualifications are required.  First,
%% the applications of $\sum_{i}$ and $\sum_{j}$ that produce $x$ and
%% $y$, respectively, must be {\em generative} applications (see section
%% \ref{generate}).  

One might claim that repeated applications of a particular operator on
the same things may produce multiple distinct objects.  For example,
one might claim that applying the sum operator to some things $a$,
$b$, $c$\,\ldots will result in a lump on one application and a statue
on another.  But this is inconsistent with the rest of Fine's theory.
Call the lump $L$ and the statue $S$.  Fine represents the application
of a composition operator as $L = \sum (a, b, c, \mathellipsis , )$.
The product $L$ {\em just is} the application of the operator; they
are {\em identical}.  If the application of the sum operator produces
both the statue and the lump, they are therefore identical:

\begin{enumerate}
  \item $L = \sum_{s} (a, b, c, \mathellipsis , )$
  \item $S = \sum_{s} (a, b, c, \mathellipsis , )$
  \item[$\therefore$] $L = S$
\end{enumerate}

But we are supposing that the statue and the lump are {\em not}
identical.  Therefore they must be composed by different operators;
the statue will be the product of the statue-builder $\sum _{st}$ and
the lump the lump-builder $\sum _{lump}$.  We have seen that for the
Supreme Court and the Special Committee to be the products of
different operators means that they are different {\em kinds} of
things---to say that both are `groups' is simply to bring two
heterogeneous kinds under one label.  Likewise, the statue and the
lump are different kinds of things---to say that both are `physical
objects' or `ordinary things' is simply to bring two heterogeneous
kinds under one label.

Thus we get the same `explosion of reality' that we get with the
previous two theories, except with the additional strangeness of a
plurality of {\em kinds} of things.  Since we have allowed that the
statue and the lump may be different things composed of the same sums,
why stop there?  We can introduce more composition operators that
produce distinct objects.  Where we see a statue and a lump, why not
suppose that there is an infinity of objects, each of a different kind
and with slightly different properties?

This is not a particularly attractive position, but it is not
indefensible (see \citet[section 4]{bennett2004}).  Since we have
already allowed a plurality of scattered objects like archipelagos and
dogbushes, why not allow a plurality of co-located objects?

% which (one and only) builder makes thinking people?

One additional difficulty for this theory is that it is unclear how
Fine would avoid there being co-located thinkers.  When discussing
Fine's theory of embodiments (section \ref{fine-h}) we saw that he has
to claim that `qua-objects' like airline passengers
(people-as-passengers) don't actually think, but that it is
nonetheless correct to say that passengers think.

Fine's theory of operators may have to include a similar clause.
If Fine is correct then there will no doubt be many things composed of
the same atoms that compose me, but none of them will think.  Only I
will be thinking.  Fine needs an explanation both of why there can
only be one thinking thing composed of any given parts---why only one
`thinking-builder' can apply to some things---and of what the
`thinking-builder' is.  What builds me?

%% more?  not the sum builder, but wait until we talk about temporary
%% parts (ie the bit where I try to combine both Fines)

\subsection{Problems with functional operators}
\label{problems2b}
The second way that I suggested we make sense of a `dynamic operator'
was to understand it as applying not to things but to a single
function:

\begin{displaymath}
G = \sum ( f )
\end{displaymath}

The function assigns certain things to certain times, so to determine
what is part of $G$ at a given time, we refer to the function: what
thing or things does the function assign to that time?  In Fine's
theory of embodiments (section \ref{fine-h}), the things that composed
a variable embodiment $V$ at a given time was the rigid embodiment
that is determined by $V$'s function.  Non-dynamic operators (section
\ref{operation}) produce things that do not change their parts, much as
rigid embodiments do not change their parts.  The function of a
dynamic operator might therefore assign products of non-dynamic
operations to time, just as the functions of variable embodiments
assign rigid embodiments to times.

The major advantage of this kind of dynamic operator is that it does
not result in a plurality of different kinds of things.  The Supreme
Court and the Special Committee can, here, be build from the same
composition operator.  That operator, in producing them, will of
course be operating on different functions; the `Supreme Court
function' will assign the various justices to the times of their
terms, and the `Special Committee function' will assign the members of
the committee to the times of their service.  Since the operator
$\sum$ will be operating on two different functions (rather than on
the same objects), it will produce two different things.

%%% is there a problem with time?

But we are left with a great plurality of objects.  Indeed, there
seems no principled limit to the functions that might, through an
application of a composition operator, give rise to new things.  If
there is a `Supreme Court function', then it seems arbitrary to say
that there isn't a function that is identical but for its beginning 10
years later.  If we apply the same operator to {\em that} function, do
we get another group that overlaps the Supreme Court for most of its
existence?  What about a function that leaves off the first 10.1
years?  What about a function that assigns Tacitus to 100 AD and my
cat to today?  Is there a group that existed momentarily, then
re-convened today, then disbands?

This kind of dynamic operator has the advantage of not producing
co-located objects of the same kind, but it seems to produce
co-located objects of different kinds.  I suggested in section
\ref{problems1} that functions might be defined by what things they
assign to each time.  Therefore, by definition, there would be no two
functions with the same assignments.  Therefore no single composition
operator will produce co-located objects.  But if two operators can
operate on the same function, they will produce co-located objects.

Whether we construe the dynamic operator as relative to time or
operating on functions, we get an absurd number of co-located things
of different kinds.  Of course, in addition to this plurality of
`dynamic' objects, there is {\em also} a plurality of `static'
objects.  This is analogous to the result of Fine's theory of
embodiments, with pluralities of both rigid and variable embodiments.

The third theory has a number of similarities to those of Fine
(especially his theory of embodiments), but it has some advantages as
well.  Unlike the theory of embodiments, Hovda's theory does not have
the consequence that relations or `conditions' are parts of things.
And unlike the theory of operators, it does not posit a large number
of different kinds of things.  But it does posit what I take to be an
objectionable number of overlapping things.

\section{Third theory: tensed mereology}
\label{hovda}
Paul Hovda has proposed an amended version of classical mereology that
presupposes neither eternalism nor presentism, and that allows for
mereological sums to change their parts over time.  (In his paper
``Tensed mereology'' \citeyearpar{hovda2011} he in fact offers three
versions of his theory; I will focus on the first formulation.)

{\em Tensed mereology} is similar to Fine's theory of embodiments in
at least one important way.  According to both theories, some relation
or property is required to specify when and where a sum of some things
exist.  Fine's example was of a bunch of flowers; the flowers compose
the bunch when and only when the relation of being bunched holds of
them.

Hovda uses the term `condition' to cover both relations and
properties:

\begin{squote}
We will want a ``condition'' to be an open sentence that may have more
than one free variable, \emph{together with a specification of a
  target variable}. For example, we will want to consider
``conditions'' like ``$y$ loves $x$'', with ``$x$'' as target.  This
is because we want to consider, in effect, for each object that might
be a value of the variable ``$y$'', the property of being a thing
loved by that object.  The point of this may be brought out by an
example.  We want as an instance of the plenitude principle, roughly
this: that for every $y$, if $y$ loves at least one thing, then there
is a thing $b$ such that $b$ is a fusion of the condition (with
respect to $x$) ``$y$ loves $x$'' (i.e., a fusion of the condition of
being loved by $y$) \citeyearpar[3n]{hovda2011}.
\end{squote}

With this notion in mind, Hovda replaces classical mereological sums
with `diachronic fusions':

\begin{description}
  \item[Diachronic fusion] An object $b$ is a ``diachronic fusion'' of
    a condition if and only if it is always the case that (1) every
    $x$ that meets the condition is part of $b$; and (2) every part of
    $b$ overlaps something that meets the condition
    \citeyearpar[3--4]{hovda2011}.
\end{description}

Like Fine, Hovda takes composition to be unrestricted: ``every
suitable condition has a diachronic fusion (where a condition is
suitable iff it is not always empty; i.e., it is suitable iff at some
time, at least one thing satisfies it)'' \citeyearpar[14]{hovda2011}.
Not only does every suitable condition have a diachronic fusion, but
``it is always the case that every suitable condition has a diachronic
fusion'' \citeyearpar[16]{hovda2011}.

In other words,

\begin{itemize}
  \item For every condition $K$,
  \item if it is ever the case that something satisfies $K$, then
  \item there is exactly one thing $b$ such that at at any time $t$
    during which anything satisfies $K$, all and only the things that
    satisfy $K$ at $t$ are parts of $b$ at $t$.
\end{itemize}

This has the welcome consequence that there are no two things that are
{\em always} co-located.  However, it does mean that there will be
very many things that are co-located at some time or other, and this
may cause problems.  Things being co-located at a time will cause
problems if we make two plausible assumptions about how parts work.
These two assumptions are {\em strong supplementation} and {\em
  anti-symmetry}.

Strong supplementation is a common assumption in mereology to the
effect that if everything that overlaps one thing overlaps another
thing, then the first thing is part of the second.  It may be
formalized as:

\begin{displaymath}
\forall x \forall y ( \forall z ( z \circ x \rightarrow z \circ y )
\rightarrow x \leq y )
\end{displaymath}
(Here `$x \circ y$' means `$x$ overlaps $y$' and `$x \leq y$' means
`$x$ is part of $y$'.)

Anti-symmetry is the assumption that if two things are parts of each
other, it follows that they are really the {\em same thing}:

\begin{displaymath}
\forall x \forall y ( ( x \leq y \wedge y \leq x ) \rightarrow x = y )
\end{displaymath}

Now take this example:

\begin{squote}
Consider a (diachronic) fusion of the condition on $x$ that ``($x$ is
Socrates and Socrates is sitting) or ($x$ is Plato and Socrates is not
sitting).''  Suppose $\beta$ is such a fusion.  Then, when Socrates
and Plato are sitting at dinner, $\beta$ exists and it should hold
(then) that everything that overlaps Socrates overlaps $\beta$ and
vice-versa.  By strong supplementation, Socrates and $\beta$ then bear
$\leq$ to one another.  By anti-symmetry, they are then identical.
But later, when Socrates stands, $\beta$ will then (by similar
reasoning) be identical with Plato, yet Socrates won’t be identical
with Plato, so Socrates and $\beta$ are then non-identical.  I take
this result to be unacceptable: once identical, always identical,
certainly if ``both'' exist \citeyearpar[17]{hovda2011}.
\end{squote}

One of our assumptions---strong supplementation or
anti-symmetry---must be withdrawn.  Hovda chooses to deny
anti-symmetry:

\begin{squote}
Instead of saying that it is always true that any mutual parts are
identical, [we] will say, roughly, that any things that are always
mutual parts are identical \citep[17]{hovda2011}.
\end{squote}

The rejection of anti-symmetry helps to show why Hovda's theory will
result in there being, at particular times, many co-located objects.
(Fine's theory of embodiments did not have to reject anti-symmetry
because his co-located objects were not, strictly speaking,
co-located; they had different relations as parts.  Fine's theory of
operators did not have to reject anti-symmetry because his
`co-located' things had the same parts, but in different ways.)

\subsection{Problems with the third theory}
\label{problems3}
Hovda's tensed mereology avoids the conclusion that conditions or
relations are parts of things, but like Fine's theory of embodiments,
it produces a huge number of things, most (perhaps all) of which are
temporarily co-located with other things.  For instance, when Tibbles
the cat is sitting, there are also an unknown number of other objects
co-located with Tibbles: the fusion of the condition of being Tibbles
sitting, the fusion of the condition of being Tibbles while less than
3 years old, the fusion of the condition of being Tibbles with a full
stomach, etc.  We should therefore pose to Hovda the same objection,
by Thomson, that we posed to Fine's theory of embodiments: is it
really true that when a cat fits, it thereby comes to pass that a new
thing (a sitting-cat) comes into existence?  When (if) I graduate
college, does a college-graduate pop into being?

One might object further that Hovda's theory results in temporarily
co-located thinkers, which would be a grave objection indeed.  We
should all agree the object that fuses the condition of being Tibbles
surely thinks.  How, then can the object that fuses the condition of
being a sitting cat {\em not} think?  Like Fine in \ref{problems1},
Hovda can reply that since the fusion of the condition of being
Tibbles sitting is just not the kind of thing that can think; it is
too transient, for it ceases to exist whenever the cat stands up, then
re-emerges when the cat sits down.  And since there are no diachronic
fusions that always share the same parts, the there can be no thinkers
{\em always} co-located with Tibbles.

\subsection{Onward}
\label{hovda-o}
These qualms by no means show that these theories are wrong.  But I
would prefer a theory that does not postulate such a bizarre
plurality.  In section \ref{set-id}, therefore, I set aside the
theories of this section and look at a new possibility.  I will
re-examine the thesis that groups---all of them---really are identical
with sets, and that ordinary things are identical with sums.  This
will lead to some strange consequences, but it may be that they are
{\em less} strange than the `explosion of reality' that we otherwise
face.

First, however, there is a question that arises for each theory: can
it explain (or be supplemented with an explanation) of why we believe
things that the theory denies?

\section{Can the plurality theories explain what we believe?}
\label{explain-p}
In section \ref{stroud} I claimed that a theory that denies that there
are chairs should be supplemented with an explanation of why we
nonetheless believe that there are chairs.

None of the theories I have proposed deny that there are chairs, but
they do make other unexpected claims that conflict with certain of our
beliefs.  Therefore, the theories should be supplemented with
explanations of why we hold these beliefs.

If the plurality thesis is right, why do we believe that there are
chairs when we {\em don't} believe that there are millions of other
objects?  Why, when there is a ten-pound chair in an otherwise empty
room, are we inclined to say that there is just {\em one} thing that
weighs ten pounds?

\subsection{Why do we believe in single things?}
\label{exp-single}
If we want to defend the theory that there are many overlapping things
where we ordinarily believe there to be just one, we should have some
account of why we are generally oblivious to this plurality of
objects.  Why do we think there is {\em just} a chair in the room, and
why do we think there is just a {\em chair} in the room?  What is so
special about the chair that promotes it to our attention out of the
many objects in the room?

It may be, in fact, that we {\em do} believe that there are co-located
things.  Many philosophers believe that there are co-located statues
and lumps of clay; do any normal people hold this belief too?
If so, then the question become: why do people believe in only {\em
  some} co-located things?

One answer might be simply that things like chairs, statues, and lumps
matter to us more than the other things.  This is similar to Trenton
Merricks' explanation of why we believe that there are chairs (section
\ref{universe}).  Merricks denies that there are chairs, but claimed
that we believe there are chairs because `things arranged chairwise'
matter to us.  Because they matter, we have terms to refer to them;
for ease of use (or some other reason) we use singular terms to refer
to things arranged chairwise, and so we are fooled by the grammar into
thinking that there are chairs.

Likewise, a pluralist like Fine or Hovda could claim that chairs
(which do exist) matter to us more than the plurality of co-located
objects that share parts with the chair.  We introduce terms to pick
out one object from among the plurality (how this happens is a
difficult question) and ignore the others.

\section{Lessons}
\label{lessons-p}
We have examined three different theories that account for the
existence of chairs.  Each has the difficult consequence that there
are pluralities of co-located objects.  Once we have admitted that,
for instance, the same matter might compose a statue and a lump, we
have trouble resisting the idea that there might be {\em more} things,
of other kinds, composed of that same matter.  (Some theories that I
have not discussed, like Thomson's \citeyearpar{thomson1998a}, are
also forced to posit co-located kinds.)  Where we might take there to
be one thing (or maybe two), we now seem committed to there being a
huge number of co-located things.  I believe that this consequence is
a reason to reject each theory and look for an alternative.

In section \ref{essential} I will sketch an alternative theory.  This
theory identifies chairs and other ordinary things with mereological
sums in the classical sense.  Groups like the Supreme Court will be
identified with sets.  As I have argued, a sum is like a set in that
it does not change its parts.  Therefore when refer to `my chair', I
am referring to a sum.  If I replace the leg on my chair, I henceforth
use `my chair' to refer to a different sum.  Which sum I refer to with
`my chair' will be governed by convention.

\ifstandalone
\end{spacing}
\bibliography{everything}
\bibliographystyle{ChicagoReedweb}
\fi
\end{document}
