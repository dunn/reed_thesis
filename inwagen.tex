\documentclass[11pt]{article}
\usepackage{standalone} \newif\ifstandlone \standalonetrue
\usepackage[left=1.75in, right=1.75in, top=1.25in, bottom=1.25in]{geometry}
\geometry{letterpaper}
\usepackage{verbatim}
\usepackage{graphicx}
\usepackage{enumitem}
%\usepackage{amssymb}
\usepackage{amsmath}
\usepackage{epstopdf}
\usepackage{setspace}
\usepackage{natbib}
\setcitestyle{aysep={}}
\usepackage{hyperref}
\usepackage{url}
\synctex=1

\DeclareSymbolFont{symbolsC}{U}{txsyc}{m}{n}
\DeclareMathSymbol{\strictif}{\mathrel}{symbolsC}{74}
\DeclareMathSymbol{\boxright}{\mathrel}{symbolsC}{128}

\newenvironment{squote}{%
\begin{spacing}{1}
\begin{list}{}{%
    \setlength{\labelwidth}{0pt}%
    \rightmargin\leftmargin%
  }
\item\relax
}{%
\end{list}%
\end{spacing}
}

\title{Paraphrases}
\author{Alexander A. Dunn}
\begin{document}
\ifstandalone
\maketitle
\begin{spacing}{1.5}
\fi

\section{Paraphrases}
\label{van-paraphrase}
I have proposed that any attempt to deny the existence of ordinary
things such as tables and chairs must be supplemented by an
explanation as to why we believe in the existence of ordinary things.
As we will see, Peter Unger's claim that there are no ordinary things
and his claim that propositions like ``that is a chair'' are uniformly
false leaves it quite mysterious why we take there to be chairs in the
first place.  Some nihilistic philosophers, therefore, have attempted
to maintain Unger's first thesis---that there are no ordinary
things---while rejecting the second---that ordinary thing discourse is
invariably false.  Such a philosopher will claim that such discourse
is {\em compatible} with the nonexistence of chairs.  This may involve
the claim that while we take ourselves to have beliefs about chairs
and other ordinary things, our beliefs do not actually concern such
(non-existent) entities.  Rather, our thought and talk is (or should
be seen as) relating to such things as do exist.  Strategies that
follow this pattern can be called paraphrasing strategies, and Peter
van Inwagen has presented a well-known version.

Like Unger, van Inwagen claims that (necessarily) there are simply no
such things as tables and chairs in the world.  But unlike Unger, he
does not claim that when we take ourselves to be thinking and talking
about such things, we are thinking and talking about nothing at all.
At least, ``when people say things in the ordinary business of life by
uttering sentences that start `There are chairs\,\ldots ' or `There
are stars\,\ldots ', they very often say things that are literally
true'' \citep[102]{inwagen1995}.  

One might assume that if such statements are true, then it follows
that there are chairs and stars.  But van Inwagen denies that chairs
and stars exist.  How can he claim, then, that what was said was true?
What van Inwagen does is attempt to show that the statements in
question can be {\em paraphrased}---they can be reformulated to show
that they have no `ontological commitments'.  According to van
Inwagen, one can assert that there is a chair without being committed
to the existence of chairs.

Section~\ref{comp} will summarize the motivation for van Inwagen's
denial.  Section~\ref{inwagen} will introduce and criticize van
Inwagen's paraphrasing strategy.

\subsection{Composition}
\label{comp}
Van Inwagen's conclusion that there are no chairs is a consequence of
his views on {\em composition} (or `constitution').  Some things are
said to compose another thing if the former are {\em parts} of the
latter; the latter is `made up of' the former.  Van Inwagen believes
that ``the metaphysically puzzling features of material objects are
connected in deep and essential ways with metaphysically puzzling
features of the constitution of material objects by their
parts''~\citep[18]{inwagen1995}.  An example of such a puzzle is the
Ship of Theseus.  The Ship of Theseus is (presumably) an object
composed of many parts, including planks of wood.  As the planks (and
other parts of the ship) wear out, they are replaced.  These
replacements happen each by themselves; the entire ship (or even a
large section) is not replaced all at once.  But eventually no part of
the original ship remains; it is build of entirely different planks,
nails, rigging, etc.  And yet we would commonly say that it is still
the same ship.  But why should we think that the present ship is
identical with a past ship with which it shares no parts?

\subsection{The Special Composition Question}
\label{scq}
Answering the question `why is this ship identical with that past
ship?' requires first figuring out why (and how) these planks and
rigging and sails (et.\ al.) compose a ship in the first place.  Van
Inwagen asks ``in what circumstances do planks\footnote{For
  simplicity's sake, van Inwagen ignores the rigging and sails.}
compose (add up to, form) something?'' (\citeyear[21]{inwagen1995}) 
For some $x$s, then, van Inwagen asks us to consider when
\begin{equation}
\exists y\ \text{the}\ x\text{s compose}\ y
\end{equation}
is true.%
\footnote{Van Inwagen explains in some detail how plural referring
  expressions (like ``the planks'') can be given a logical
  formalization (\citeyear[23--28]{inwagen1995}), but suffice to say
  they work just as one would expect.}
%
\ Less formally, van Inwagen asks: ``suppose one had certain
(nonoverlapping) objects, the $x$s, at one's disposal; what would one
have to do---what {\em could} one do---to get the $x$s to compose
something?'' (\citeyear[31]{inwagen1995})  This is the Special
Composition Question.

(`Composition' is used in a technical sense with regard to the Special
Composition Question.  Van Inwagen defines it thus: ``the $x$s compose
$y$'' means that ``the $x$s are all parts of $y$ and no two of the
$x$s overlap and every part of $y$ overlaps at least one of the
$x$s\,\ldots\,a thing {\em overlaps} a thing---or: they overlap---if
they have a common part'' (\citeyear[29]{inwagen1995}).  For van
Inwagen, everything is a part of itself; some $x$ is a {\em proper}
part of some $y$ only if $x \neq y$.)

\subsection{The usual answers}
\label{scq-ans}
There are several prominent answers to the Special Composition
Question, including the following (These formulations are from
\citet{markosian1998a}):
\begin{description}
	\item[Nihilism] Necessarily, for any $x$s, there is an object
          composed of the $x$s iff there is only one of the $x$s,
          i.e., the only objects that exist are
          simples (\citeyear[219]{markosian1998a}).
	%
	\footnote{\label{flip} Note that this may not be Unger's view.
          He denies that people, apples, cheese, tables, chairs, and
          other ``ordinary things'' are nonexistence but he does not,
          as far as I know, take a stand on whether anything at all
          exists.  His view can be (flippantly) summarized thus: ``if
          we have a word for it, it doesn't exist.''}
	%\footnote{\label{gunk} Of course, it may be that the world is
	%not fundamentally particulate, and is filled not with simples
	%but with `gunk'; see \citet{schaffer2003}.  Nihilism (and van
	%Inwagen's second condition below) can be formulated to take
	%this possibility into account: ``for any quantity of gunk,
	%there is nothing composed of it.''}%
	%
	\item[Universalism] Necessarily, for any $x$s, there is an
          object composed of the $x$s iff no two of the $x$s
          overlap (\citeyear[227]{markosian1998a}).
	\item[Van Inwagenism] Necessarily, for any $x$s, there is an
          object composed of the $x$s iff either (i) the activity of
          the $x$s constitutes a life or (ii) there is only one of the
          $x$s (\citeyear[221]{markosian1998a}).
\end{description}

We will discuss Unger's version of nihilism in section \ref{unger}.
As I will argue, any version of nihilism that does not explain our
beliefs in the existence of ordinary objects is problematic.

Universalism raises a number of problems in relation to Peter Unger's
`problem of the many' (see section \ref{many}).  I will therefore
postpone discussion of this view until later.

Van Inwagen examines and rejects universalism and the version of
nihilism given above.  He also rejects a number of other answers to
the Special Composition Question.  Some are too strong: `some $x$s
compose a $y$ iff the $x$s are in contact' would entail that two
people shaking hands will result in a new object coming into being.
Others are too strong in some ways and too weak in others: `some $x$s
compose a $y$ iff the $x$s are fastened together' would entail that
two people being glued together would result in a new object; and it
would deny that an object can be composed without fastening its parts
together (such as when building a house of cards).  The only answer
van Inwagen finds consistent is what we have dubbed {\em van
  Inwagenism}, which entails that tables and chairs do not exist.

Because of this consequence, van Inwagenism should include an
explanation why we nonetheless believe that there are tables and
chairs.  Happily, van Inwagen recognizes this and is prepared with a
{\em paraphrasing strategy}.  This strategy aims to show that the
beliefs that we take to be about tables and chairs are really about
something else, and are not true beliefs.  If such beliefs are true,
then it should be relatively easy to explain why hold them: they are
true, and we learn of them through some reliable means (like our
eyes).

Unfortunately, van Inwagen's paraphrasing strategy does not work.

\subsection{Van Inwagen's paraphrasing strategy}
\label{inwagen}
Peter Unger maintains that terms like `chair' are incoherent; if this
is so, a statement involving the phrase `There is a chair\,\ldots '
could surely not be true.  Van Inwagen, on the other hand, admits that
``when people say things in the ordinary business of life by uttering
sentences that start `There are chairs\,\ldots ' or `There are
stars\,\ldots ', they very often say things that are literally true''
\cite[102]{inwagen1995}.  It does not seem unreasonable to assume that
if what people say with ``There are chairs\,\ldots '' and the like are
true, then chairs exist.  But van Inwagen denies this entailment.

How can van Inwagen maintain this?  Someone can say, truly, ``There
are simples arranged chairwise\,\ldots '' without committing oneself
to the existence of chairs.  Van Inwagen might then claim that when
someone says ``There is a chair\,\ldots '' she {\em means} `There are
simples arranged chairwise.'  This is, of course, a bold hypothesis
about the speech practices of ordinary speakers.  Certainly very few
speakers would, if asked, affirm that what they meant to say had
anything to do with simples; they would say that when they said that
there was a chair, they meant just that.  Van Inwagen recognizes that
this is not a viable position: ``The only thing I have to say about
what the ordinary man really means by `There are two valuable chairs
in the next room' is that he really means that there are two valuable
chairs in the next room'' (\citeyear[106]{inwagen1995}).

One might then assume that van Inwagen is thinking in analogy with
Russell.  He could attempt to claim that, despite the surface
appearance of language (`There is a chair\,\ldots '), the underlying
logical form does not make any mention of chairs (or tables); the
offending concept is analyzed away, leaving `There are simples
arranged chairwise\,\ldots '.  Van Inwagen notes that his ``suggested
technique of paraphrasing enables us to escape some of the more
embarrassing consequences of this position.  When someone says `Some
tables are heavier than some chairs,' there is obviously something
right about what he says.  Our technique of paraphrasis enables us to
capture what it is that is right about what he says''
(\citeyear[111]{inwagen1995}).  However, on the very next page he
admits that the ordinary language proposition and his paraphrased
version are different propositions: ``When the ordinary man utters the
sentence `Some chairs are heavier than some tables' (in an appropriate
context, and so on and so on), he expresses a certain proposition, and
one that is almost certainly true.  But I do not claim that this
proposition {\em is} the proposition that, for some $x$s, those $x$s
are arranged chairwise and for some $y$s, those $y$s are arranged
tablewise, and the $x$s are heavier than the $y$s''
(\citeyear[112]{inwagen1995}).  So van Inwagen is not making an appeal
to some notion of `logical form'.  But then what is the purpose of the
paraphrasing project?

Van Inwagen attempts to justify his method of paraphrasis by asserting
the following parallels between the original and paraphrased
propositions:
\begin{enumerate}[label=(\Alph*)]
	\item The paraphrase describes the same fact as the
          original.  \label{para-a}
	\item The paraphrase, unlike the original, does not even
          appear to imply that there are any objects that occupy
          chair-receptacles.  \label{para-b}
	\item The paraphrase is neutral with respect to competing
          metaphysical theories, {\em viz}.  the ``received'' theory,
          that there are objects that occupy chair-receptacles, and
          the theory I have proposed, according to which there are no
          such objects.  \label{para-c}
	\item The original, though it doubtless does not express the
          same proposition as the paraphrase, has the feature ascribed
          to the paraphrase in \ref{para-c}: It is neutral with
          respect to the question whether there are objects that fit
          exactly into
          chair-receptacles~(\citeyear[113]{inwagen1995}).  \label{para-d}
\end{enumerate}

I am rather dubious as to the truth of \ref{para-a}, but I am quite
sure that \ref{para-d} is false, and van Inwagen's thesis appears to
depend on it.  He admits in \ref{para-b} that the original sentence
(`There are chairs\,\ldots ') {\em implies} that there are chairs, but
claims in \ref{para-d} that it does not {\em entail} this.  But why
wouldn't it?

\subsection{Propositions and ontological commitment}
Let us review the situation.  First, van Inwagen agrees that when
someone says things like ``There is a chair\,\ldots '' they mean just
that.  Second, he admits that his `paraphrases' of such propositions
are not so faithful to the original that they can be called the same
proposition; the original and the paraphrase are two different
propositions.  Third, he claims nonetheless that {\em neither} the
original nor the paraphrase entail the existence of chairs.

This seems obviously untrue.  How can he claim that when someone says
``There is a chair\,\ldots '' and means just that, that the
proposition they express does not entail the existence of chairs?  To
defend his claim, van Inwagen appeals to his `Copernican analogy':

\begin{squote}
I accept the Copernican Hypothesis.  One day you hear me say, ``It was
cooler in the garden after the sun had moved behind the elms.''  You
say, ``You see, you can't consistently maintain your Copernicanism
outside the astronomer's study.  You say that the sun moved behind the
elms; yet, according to your official theory, the sun does not move.''
I reply that the proposition I expressed by saying ``It was cooler in
the garden after the sun had moved behind the elms'' is consistent
with the Copernican Hypothesis (\citeyear[101]{inwagen1995}).
\end{squote}
That is, van Inwagen claims that the proposition he expressed with
``It was cooler in the garden after the sun had moved behind the
elms'' does not entail that the sun actually moved.  And he argues
that this is analogous to our talk of chairs: most propositions
expressed with ``There is a chair\,\ldots '' do not entail that chairs
actually exist.

Does the proposition van Inwagen expresses with ``The sun moved behind
the elms'' entail that the sun moved? I am inclined to say that it
does.  If I were to say simply ``The sun moved'' (meaning just that),
I think I would have committed myself to the movement of the sun.  Why
should we think that the addition of `behind the elms' defeats this
entailment?  Without some explanation of what the difference is, I see
no reason to think that saying ``The sun moved behind the elms'' (and
meaning it) does not entail the movement of the sun.  Likewise, if
``there are chairs in the next room'' does not entail that there are
chairs, then it would appear that ``there are chairs'' does not entail
that there are chairs.

Before we dismiss van Inwagen's paraphrasing strategy, we should
examine another, perhaps more plausible, analogy.  This analogy
involves an imaginary planet called Pluralia where there is a
``creature'' known as a bliger.  The bliger, according to van Inwagen,
is what happens when a monkey, four owls, and a tiger (or something
like that) attach themselves together temporarily.  The conglomeration
appears to the untrained observer to be a single animal.  Gullible
farmers have dubbed this conglomeration a `bliger'.  Van Inwagen's
point is that there are no bligers, but that a farmer saying ``there's
a bliger'' when pointing at such a conglomeration would be saying
something true.  Even though there are no bligers (according to van
Inwagen), someone saying ``there's a bliger'' says something true
because she reports a fact.  The fact being reported by ``there's a
bliger'' is the fact that a monkey, four owls, and a tiger are there.
If she has instead said ``that bliger just exploded'', what she said
would be false, because there is no fact that her proposition reports.

People believe that there are bligers because they mistake the group
of animals for a single thing, which has been dubbed `bliger'.
Likewise, van Inwagen maintains that people mistake chairwise
arrangements of simples for chairs.  When someone says ``there's a
chair'' what she says is true because it reports a fact.  The fact
being reported is that there is a chairwise arrangement of simples
there.  People believe that there are chairs because they mistake the
group of animals for a single thing, which has been dubbed `chair'.

I agree with van Inwagen that these cases are analogous.  However,
where van Inwagen takes this analogy to show that there are no chairs,
I take it to show that there {\em are} bligers in van Inwagen's
imaginary scenario.  When it is discovered that bligers are built up
from six other creatures, we are learning something about bligers:

\begin{squote}
\ldots {\em of course} there are bligers in [van Inwagen's] story.
Bligers are what the story is about.  The zoologists do not report
that there are no bligers.  Rather they tell us what a bliger is.
They explain that a bliger is not a single large carnivorous animal
but a transient symbiotic union of six animals
\citep[704]{rosenberg1993}.
\end{squote}

In short, van Inwagen's analogy does not provide us with an
explanation of why we would believe in chairs even if there were none.

\subsection{Backfire}
Van Inwagen should be thankful that his analogy does not succeed.  If
he showed that ``there are chairs'' does not entail that there are
chairs, then the whole notion of ontological commitment would be
undermined.  If ``there are chairs'' did not entail that there are
chairs, then why should any proposition of the form ``there are $x$s''
entail that there are $x$s?

It is surely true that van Inwagen would affirm ``there are simples
arranged chairwise''.  And no doubt he would affirm that it follows
from the truth of that proposition that there are simples arranged
chairwise.  But how can he affirm this, if he denies that ``there are
chairs'' entails that there are chairs?

If ``there are chairs in the next room'' does not entail that there
are chairs and if ``the sun moved behind the trees'' does not entail
sun moved (nor that it exists), then how can van Inwagen maintain that
``there are simples arranged chairwise'' entails that there are
simples, or that they are arranged chairwise?  He has given us no
reason to believe one and not the other.

\subsection{Has van Inwagen gone astray?}
Van Inwagen has not succeeded in explaining why we believe that there
are chairs when (according to him) there are none.  This gives us a
good reason to reject his conclusion---that the only composite objects
are lives.

If van Inwagen's conclusion is false, there must be something wrong
with his argument.  One possibility is that he has overlooked a better
answer to the Special Composition Question.  But it is also possible
that the Special Composition Question is itself the wrong question to
be asking.  Jay Rosenberg brings out this worry nicely.  When van
Inwagen asks ``suppose one had certain (nonoverlapping) objects, the
$x$s, at one's disposal; what would one have to do---what {\em could}
one do---to get the $x$s to compose something?''
(\citeyear[31]{inwagen1995}), Rosenberg says this:

\begin{quote}
To me it just seems obvious that the answer to such a question will
always depend on what sorts of things one has at one's disposal and
what sort of thing one is trying to get them to compose. If the $x$s
are, for example, ``a lot of wooden blocks that one may do with as one
wills'', then to get them to compose, for example, a wall, it may be
sufficient to stack them up in the manner we call ``building a wall.''
To get them to compose a wooden raft,on the other hand,one would
surely need to fasten them together more securely, e.g., by gluing
them to one another. And there's nothing at all one could do with them
to get them to add up to a fish or a clock or a sports car
(\citeyear[705]{rosenberg1993}).
\end{quote}

Van Inwagen's approach to nihilism is not the only one, however.
Trenton Merricks has proposed a very similar thesis---that the only
composite objects are human beings---for very different reasons.  We
will examine his proposal now.

\ifstandalone
\end{spacing}
\bibliography{everything}
\bibliographystyle{ChicagoReedweb}
\fi

\end{document}
