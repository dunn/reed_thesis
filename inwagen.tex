\documentclass[11pt]{standalone}
\usepackage{standalone} \newif\ifstandlone \standalonetrue
\usepackage[left=1.75in, right=1.75in, top=1.25in, bottom=1.25in]{geometry}
\geometry{letterpaper}
\usepackage{graphicx}
%\usepackage{tipa}
%\usepackage{exaccent}
%\usepackage{txfonts}
%\usepackage{pxfonts}
\usepackage{enumitem}
%\usepackage{amssymb}
\usepackage{amsmath}
\usepackage{epstopdf}
\usepackage{setspace}
\usepackage{natbib}
\setcitestyle{aysep={}}
\usepackage{hyperref}
\usepackage{url}
\synctex=1

\DeclareSymbolFont{symbolsC}{U}{txsyc}{m}{n}
\DeclareMathSymbol{\strictif}{\mathrel}{symbolsC}{74}
\DeclareMathSymbol{\boxright}{\mathrel}{symbolsC}{128}

                \newenvironment{squote}{%
\begin{spacing}{1}
       	\begin{list}{}{%
\setlength{\labelwidth}{0pt}%
\rightmargin\leftmargin%
}
\item\relax
}{%
\end{list}%
\end{spacing}
}

\title{1.25}
\author{Alexander A. Dunn}
\begin{document}
\ifstandalone
\maketitle
\begin{spacing}{1.5}
\fi

\section{Paraphrases}
In section~\ref{unger} we found that an attempt to deny the existence
of ordinary things such as tables and chairs must be able to explain
why we believe in the existence of ordinary things.  Unger's claim
that there are no ordinary things and his claim that propositions like
``that is a chair'' are uniformly false leaves it wholly mysterious
why we take there to be chairs in the first place.  Some nihilistic
philosophers, therefore, have attempted to maintain Unger's first
thesis---that there are no ordinary things---while rejecting the
second---that ordinary-thing-discourse is invariably false.  Such a
philosopher will claim that such discourse is {\em compatible} with
the nonexistence of chairs.  They may then claim that while we take
ourselves to have beliefs about chairs and other ordinary things, our
beliefs do not actually concern such (non-existent) entities.  Rather,
our thought and talk is (or should be seen as) relating to such things
as do exist.  Strategies that follow this pattern can be called
paraphrasing strategies, and Peter van Inwagen has presented a
well-known version.

Like Unger, van Inwagen claims that (necessarily) there are simply no
such things as tables and chairs in the world.  But unlike Unger, he
does not claim that when we take ourselves to be thinking and talking
about such things, we are thinking and talking about nothing at all.
Or at least ``when people say things in the ordinary business of life
by uttering sentences that start `There are chairs\,\ldots\,' or
`There are stars\,\ldots\,', they very often say things that are
literally true''~\citep[102]{inwagen1995}.  One would generally assume
that if such statements are true, then it follows that chairs and
stars exist.  But van Inwagen denies that chairs and stars exist.  How
can he claim, then, that what was said was true?  As we mentioned in
section~\ref{paraphrase}, van Inwagen attempts to show that the
statements in question can be {\em paraphrased}---they can be
reformulated to show that they have no ``ontological commitments''.
According to van Inwagen, one can assert that there is a chair without
being committed to the existence of chairs.

Section~\ref{comp} will summarize the motivation for van Inwagen's
denial.  Section~\ref{inwagen} will introduce and criticize van
Inwagen's paraphrasing strategy.

\subsection{Composition}
\label{comp}
The Special Composition Question was given a precise formulation by
van Inwagen, who finds that ``the metaphysically puzzling features of
material objects are connected in deep and essential ways with
metaphysically puzzling features of the constitution of material
objects by their parts''~\citep[18]{inwagen1995}.  A ready example is
the Ship of Theseus: the planks and rigging and sails (and every part
of the ship) are replaced as they individually wear out.  These
replacements happen each by themselves; it's not the case that the
entire ship (or even a large section) is swapped out all at once.  But
eventually no part of the original ship remains.  And yet we would
commonly say that it is still the same ship.  But why should we think
that the present ship is identical with a past ship with which it
shares no parts?

\subsection{The question}
\label{scq}
Answering the question `why is this ship identical with that past
ship?' requires first figuring out why (and how) these planks and
rigging and sails (et.\ al.) compose a ship in the first place.  Van
Inwagen asks ``in what circumstances do planks\footnote{For
  simplicity's sake, van Inwagen ignores the rigging and sails.}
compose (add up to, form) something?''~(\citeyear[21]{inwagen1995}) 
For some $x$s, then, van Inwagen asks us to consider when
\begin{equation}
\exists y\ \text{the}\ x\text{s compose}\ y
\end{equation}
is true.%
\footnote{Van Inwagen explains in some detail how plural referring
  expressions (like ``the planks'') can be given a logical
  formalization (\citeyear[23--28]{inwagen1995}), but suffice to say
  they work just as one would expect.}
%
\ Less formally, van Inwagen asks: ``suppose one had certain
(nonoverlapping) objects, the $x$s, at one's disposal; what would one
have to do---what {\em could} one do---to get the $x$s to compose
something?''~(\citeyear[31]{inwagen1995})

({\em Composition} is a technical term for van Inwagen.  He
understands it thus: ``the $x$s compose $y$'' means that ``the $x$s
are all parts of $y$ and no two of the $x$s overlap and every part of
$y$ overlaps at least one of the $x$s\,\ldots\,a thing {\em overlaps}
a thing---or: they overlap---if they have a common
part''~(\citeyear[29]{inwagen1995}).  For van Inwagen, everything is a
part of itself; some $x$ is a {\em proper} part of some $y$ only if $x
\neq y$.)

\subsection{The usual answers}
There are several prominent answers to the Special Composition
Question, including the following:\footnote{These formulations are
  from~\citet{markosian1998a}.}
\begin{description}
	\item[Nihilism] Necessarily, for any $x$s, there is an object
          composed of the $x$s iff there is only one of the $x$s,
          i.e., the only objects that exist are
          simples~(\citeyear[219]{markosian1998a}).
	%
	\footnote{\label{flip} Note that this may not be Unger's view.
          He denies that people, apples, cheese, tables, chairs, and
          other ``ordinary things'' are nonexistence but he does not,
          as far as I know, take a stand on whether anything at all
          exists.  His view can be (flippantly) summarized thus: ``if
          we have a word for it, it doesn't exist.''}
	%\footnote{\label{gunk} Of course, it may be that the world is
	%not fundamentally particulate, and is filled not with simples
	%but with `gunk'; see \citet{schaffer2003}.  Nihilism (and van
	%Inwagen's second condition below) can be formulated to take
	%this possibility into account: ``for any quantity of gunk,
	%there is nothing composed of it.''}%
	%
	\item[Universalism] Necessarily, for any $x$s, there is an
          object composed of the $x$s iff no two of the $x$s
          overlap~(\citeyear[227]{markosian1998a}).
	\item[Van Inwagenism] Necessarily, for any $x$s, there is an
          object composed of the $x$s iff either (i) the activity of
          the $x$s constitutes a life or (ii) there is only one of the
          $x$s~(\citeyear[221]{markosian1998a}).
\end{description}

As we have seen, any version of nihilism that does not explain our
beliefs in the existence of ordinary objects is problematic.

Universalism raises a number of problems in relation to the problem of
the many.  The view entails that where we suppose a single chair
(stone, cloud) to be, there exist a plurality of other things.  Given
that these things differ minimally from stones (chairs, clouds), and
not in any relevant way, there is strong pressure to admit that these
many other objects are also stones (chairs, clouds).  But we agreed
that this is a bizarre and unacceptable consequence.  An acceptable
version of universalism must therefore avoid this consequence.  We
will (hopefully) encounter such a modified universalism in a later
section.

Van Inwagen examines and rejects universalism and the strict version
of nihilism above.  He also rejects a number of other answers to the
Special Composition Question.  Some are too strong: `some $x$s compose
a $y$ iff the $x$s are in contact' would entail that two people
shaking hands will result in a new object coming into being.  Others
are too strong in some ways and too weak in others: `some $x$s compose
a $y$ iff the $x$s are fastened together' would entail that two people
being glued together would result in a new object; and it would deny
that an object can be composed without fastening its parts together
(such as when building a house of cards).  The only answer van Inwagen
finds consistent is what we have dubbed {\em van Inwagenism}, which
entails that tables and chairs do not exist.

However, without an account of our beliefs about tables and chairs,
the fact that van Inwagenism entails the nonexistence of tables and
chairs only gives us a reason to reject van Inwagenism.  Happily,
though, van Inwagen recognizes this and is prepared with a
paraphrasing strategy aimed to show that the beliefs that we take to
be about tables and chairs are really about something else.

\section{Van Inwagen's paraphrasing strategy}
\label{inwagen}
Van Inwagen distances himself from the kind of resolute denial we saw
Unger attempting in section~\ref{unger}.  Unger maintained that terms
like `chair' are incoherent; were this so, a statement involving the
phrase `There is a chair\,\ldots ' could surely not be true.  Van
Inwagen, on the other hand, admits that ``when people say things in
the ordinary business of life by uttering sentences that start `There
are chairs\,\ldots\,' or `There are stars\,\ldots\,', they very often
say things that are literally true''~\cite[102]{inwagen1995}.  One
would generally assume that if what people say with ``There are
chairs\,\ldots '' and the like are true, then chairs exist.  But van
Inwagen denies this entailment.

How can van Inwagen maintain this?  He claims that one can also say,
truly, ``There are simples arranged chairwise\,\ldots '' without
committing oneself to the existence of chairs.  He could, therefore,
claim that when someone says ``There is a chair\,\ldots '' she {\em
  means} `There are simples arranged chairwise.'  This is, of course,
a bold hypothesis about the speech practices of ordinary speakers.
Certainly very few speakers would, if asked, affirm that what they
meant to say had anything to do with simples; they would say that when
they said that there was a chair, they meant just that.  Van Inwagen
recognizes that this is not a viable position: ``The only thing I have
to say about what the ordinary man really means by `There are two
valuable chairs in the next room' is that he really means that there
are two valuable chairs in the next
room''~(\citeyear[106]{inwagen1995}).

One might then assume that van Inwagen is thinking in analogy with
Russell.  He could attempt to claim that, despite the surface
appearance of language (`There is a chair\,\ldots '), the underlying
logical form does not make any mention of chairs (or tables); the
offending concept is analyzed away, leaving `There are simples
arranged chairwise\,\ldots '.  Van Inwagen notes that his ``suggested
technique of paraphrasing enables us to escape some of the more
embarrassing consequences of this position.  When someone says `Some
tables are heavier than some chairs,' there is obviously something
right about what he says.  Our technique of paraphrasis enables us to
capture what it is that is right about what he
says''~(\citeyear[111]{inwagen1995}).  However, on the very next page
he admits that the ordinary language proposition and his paraphrased
version are different propositions: ``When the ordinary man utters the
sentence `Some chairs are heavier than some tables' (in an appropriate
context, and so on and so on), he expresses a certain proposition, and
one that is almost certainly true.  But I do not claim that this
proposition {\em is} the proposition that, for some $x$s, those $x$s
are arranged chairwise and for some $y$s, those $y$s are arranged
tablewise, and the $x$s are heavier than the
$y$s''~(\citeyear[112]{inwagen1995}).  So van Inwagen is not making an
appeal to some notion of `logical form'.  But then what is the purpose
of the paraphrasing project?

Van Inwagen attempts to justify his method of paraphrasis by asserting
the following parallels between the original and paraphrased
propositions:
\begin{enumerate}[label=(\Alph*)]
	\item The paraphrase describes the same fact as the
          original.  \label{para-a}
	\item The paraphrase, unlike the original, does not even
          appear to imply that there are any objects that occupy
          chair-receptacles.  \label{para-b}
	\item The paraphrase is neutral with respect to competing
          metaphysical theories, {\em viz}.  the ``received'' theory,
          that there are objects that occupy chair-receptacles, and
          the theory I have proposed, according to which there are no
          such objects.  \label{para-c}
	\item The original, though it doubtless does not express the
          same proposition as the paraphrase, has the feature ascribed
          to the paraphrase in \ref{para-c}: It is neutral with
          respect to the question whether there are objects that fit
          exactly into
          chair-receptacles~(\citeyear[113]{inwagen1995}).  \label{para-d}
\end{enumerate}
I am rather dubious as to the truth of \ref{para-a}, but I am quite
sure that \ref{para-d} is false, and van Inwagen's thesis appears to
depend on it.  He admits in~\ref{para-b} that the original sentence
(e.g., `There are chairs\,\ldots ') {\em implies} that there are
chairs, but claims in~\ref{para-d} that it does not {\em entail} this.
But why wouldn't it?

\subsection{Propositions and ontological commitment}
Let us review the situation.  First, van Inwagen agrees that when
someone says things like ``There is a chair\,\ldots '' they mean just
that.  Second, he admits that his `paraphrases' of such propositions
are not so faithful to the original that they can be called the same
proposition; the original and the paraphrase are two different
propositions.  Third, he claims nonetheless that {\em neither} the
original nor the paraphrase entail the existence of chairs.

This seems obviously untrue.  How can he claim that when someone says
``There is a chair\,\ldots '' and means just that, that the
proposition they express does not entail the existence of chairs?  To
defend his claim, van Inwagen appeals to his `Copernican analogy':
\begin{squote}
I accept the Copernican Hypothesis.  One day you hear me say, ``It was
cooler in the garden after the sun had moved behind the elms.''  You
say, ``You see, you can't consistently maintain your Copernicanism
outside the astronomer's study.  You say that the sun moved behind the
elms; yet, according to your official theory, the sun does not move.''
I reply that the proposition I expressed by saying ``It was cooler in
the garden after the sun had moved behind the elms'' is consistent
with the Copernican Hypothesis~(\citeyear[101]{inwagen1995}).
\end{squote}
That is, van Inwagen claims that the proposition he expressed with
``It was cooler in the garden after the sun had moved behind the
elms'' does not entail that the sun actually moved.  And he argues
that this is analogous to our talk of chairs: most propositions
expressed with ``There is a chair\,\ldots '' do not entail that chairs
actually exist.

First, does the proposition van Inwagen expresses with ``The sun moved
behind the elms'' entail that the sun moved? I am inclined to say that
it does.  If I were to say simply ``The sun moved'' (meaning just
that), I think I would have committed myself to the movement of the
sun.  Why should we think that the addition of `behind the elms'
defeats this entailment?  Without some explanation of what the
difference is, I see no reason to think that saying ``The sun moved
behind the elms'' (and meaning it) does not entail the movement of the
sun.  But van Inwagen may be forced to say here that neither
proposition entails that the sun moved.  For he certainly won't allow
that either entails that the sun {\em exists.}

There is an analogy here, though perhaps not the one van Inwagen had
in mind.  He claims that a proposition expressed by ``There are two
very valuable chairs in the next room'' does not necessarily entail
the existence of chairs.  If this proposition does not entail that
chairs exist, then what about `There are two valuable chairs left in
the world' or `There are at least two chairs in the world' or `There
are at least two chairs' or simply `There are chairs'?  Van Inwagen
appears committed to the claim that the proposition I would express
with ``There are chairs'' does not entail that there are chairs.

Why on earth should this be? Does not the proposition expressed by my
saying ``There are simples arranged chairwise\,\ldots '' entail the
existence of simples?  If van Inwagen says that there are simples
arranged chairwise, and means just that, then it would appear to
follow that there are simples.  Indeed, van Inwagen's argument relies
rather heavily on the assumption that simples exist.
%
%% \footnote{Ted Sider takes him to task for this
%%   assumption~(\citeyear{sider1993}), claiming that the possibility of
%%   `gunk'---the possibility that the matter of the world is not
%%   fundamentally particulate but infinitely divisible---falsifies van
%%   Inwagen's thesis.  I think it may be possible for van Inwagen to
%%   adapt to a gunky world (he might be able to claim that nothing
%%   exists but organisms, who are composed of other organisms and/or
%%   gunk), but I think van Inwagen's thesis is false either way.}
%
\ But if `There are chairs' does not entail that there are chairs and
if `The sun moved behind the trees' entails neither that the sun moved
nor that the sun exists, then how can van Inwagen maintain that `There
are simples arranged chairwise' entails that there are simples, or
that they are arranged chairwise?  He has given us no reason to
believe one and not the other.

\subsection{Loose truth, again}
\label{loose-v}
When criticizing Peter Unger's nihilism, we imagined a defense of his
thesis based on the notion of `loose truth'.  Our Unger partisan
claimed that while such claims as `There is a chair\,\ldots ' are
invariably false (because incoherent), they may be loosely true.  But
that defense remained incapable of explaining why we believe that
there are chairs at all, as opposed to chair-wise arrangements of
matter.

Might van Inwagen also appeal to loose truths?  He admits it as a
last-ditch possibility:

\begin{squote}
I can say this [that `There are chairs\,\ldots ' can be true yet not
  entail that there are chairs] because I accept certain theses in the
philosophy of language.  I can say this because I accept certain
theses in the philosophy of language.  Some people, I suppose, would
reject these theses.  These people would say that when I
said\,\ldots\,`The sun moved behind the elms,' I said something
false\,\ldots\,If someone maintains that `The sun moved behind the
elms' expresses a falsehood, he must still have some way to
distinguish between this sentence and those sentences (like `The sun
exploded' and `The sun turned green') that the vulgar would regard as
the sentences that expressed falsehoods about the sun\,\ldots\,[if I
  took this position,] I should not be willing to say that people who
uttered things like `There are two valuable chairs in the next room'
very often said what was true.  I should be willing to say only that
they very often say what might be treated as a truth for all practical
purposes~(\citeyear[102--103]{inwagen1995}).
\end{squote}

Van Inwagen admits that `There are two very valuable chairs in the
next room', ``when it is successfully used to report a fact, does
report a fact about the existence of {\em something}''
(\citeyear[102]{inwagen1995}).  Presumably van Inwagen thinks that
`something' is the chairwise arrangements of simples.  Van Inwagen
must therefore (like our pseudo-Unger above) explain how a chairwise
arrangement of simples can cause people to believe that there are
chairs.  Van Inwagen attempts to bolster his case by drawing an
analogy with the imaginary bliger.  The bliger, according to van
Inwagen, is what happens when a monkey, four owls, and a tiger attach
themselves together temporarily.  The conglomeration appears to the
untrained observer to be a single animal, and gullible farmers dubbed
it a `bliger'.  Van Inwagen's point is that there are no bligers, but
that a farmer saying ``there's a bliger'' when pointing at such a
conglomeration would be saying something loosely true.  The fact being
reported by ``there's a bliger'' is the fact that a monkey, four owls,
and a tiger are there.  People believe that there are bligers because
they mistake the group of animals for a single thing, which has been
dubbed `bliger'.

Likewise, van Inwagen maintains that people mistake chairwise
arrangements of simples for chairs.  When someone says ``there's a
chair'' what she says is loosely true because there is a chairwise
arrangement of simples there.  People believe that there are chairs
because they mistake the group of animals for a single thing, which
has been dubbed `chair'.

I agree with van Inwagen that these cases are analogous.  However,
where van Inwagen takes this analogy to show that there are no chairs,
I take it to show that there {\em are} bligers in van Inwagen's
imaginary scenario.  When it is discovered that bligers are built up
from six other creatures, we are learning something about bligers:

\begin{squote}
\ldots {\em of course} there are bligers in [van Inwagen's] story.
Bligers are what the story is about.  The zoologists do not report
that there are no bligers.  Rather they tell us what a bliger is.
They explain that a bliger is not a single large carnivorous animal
but a transient symbiotic union of six animals
\citep[704]{rosenberg1993}.
\end{squote}

In short, van Inwagen's analogy does not provide us with an
explanation of why we would believe in chairs even if there were none.
Just as the Ungerian treatment of loose truth failed to explain our
beliefs in chairs, van Inwagen's appeal to loose truth does not
explain why we believe that there are chairs, rather than chairwise
arrangements of simples.

\section{A stand-off?}
Neither Unger or van Inwagen's brands of nihilism are equipped to
explain why we believe in chairs and other ordinary things.  This
explanatory deficiency is a significant obstacle to the acceptance of
a nihilistic thesis.  But this does not show that nihilism is false.
The considerations which led philosophers to embrace nihilistic theses
have not been answered.  The sorites paradox and the problem of the
many still pose problems for our ordinary concepts.  In order to be in
a position to reject nihilism, we must examine these problems more
closely.  It may be that from these considerations, we can learn more
about our concepts for ordinary things.

\ifstandalone
\end{spacing}
\fi
\end{document}
