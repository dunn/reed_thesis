%\documentclass[11pt]{article}
%\usepackage[margin=1.25in]{geometry}
%\geometry{letterpaper}
%\usepackage{graphicx}
%%\usepackage{tipa}
%%\usepackage{exaccent}
%%\usepackage{txfonts}
%%\usepackage{pxfonts}
%\usepackage{enumitem}
%%\usepackage{amssymb}
%\usepackage{amsmath}
%\usepackage{epstopdf}
%\usepackage{setspace}
%\usepackage{natbib}
%\setcitestyle{aysep={}}
%\synctex=1
%
%\DeclareSymbolFont{symbolsC}{U}{txsyc}{m}{n}
%\DeclareMathSymbol{\strictif}{\mathrel}{symbolsC}{74}
%\DeclareMathSymbol{\boxright}{\mathrel}{symbolsC}{128}
%
%\newenvironment{squote}{\begin{quote}\begin{singlespace}}{\end{singlespace}\end{quote}}
%
%\newcommand{\stager}[4]%
%{%
%	\begin{spacing}{1}%
%	\vspace{0pt}
%		\begin{description}[style=nextline, noitemsep, parsep=0pt, topsep=0pt, leftmargin=15mm, itemindent=-10mm, font=\mdseries]
%			\item[\textsc{#1} \emph{#2}] #3
%			\item[]%
%			\begin{flushright}#4\end{flushright}
%		\end{description}%
%	\end{spacing}%
%}
%
%\newcommand{\stage}[3]%
%{%
%	\begin{spacing}{1}%
%	\vspace{0pt}
%		\begin{description}[style=nextline, parsep=0pt, leftmargin=15mm, itemindent=-10mm, font=\mdseries]
%			\item[\textsc{#1} \emph{#2}] #3
%		\end{description}%
%	\end{spacing}%
%}
%
%\title{Alien Explorers and Conceptual Schemes}
%\author{Alexander A. Dunn}
%\begin{document}
%\maketitle
%\begin{spacing}{1}

\chapter{A story about things}
\chapterpig{A Story About Things}

\section{Intuitions}
Gareth Evans once claimed that ``with pliant enough intuitions you can swallow anything in philosophy''~(\citeyear[192]{evans1973}). That's an hypothesis that demands testing. So I will claim that, if humans did not exist, neither would dogs, cats, mountains, forests, lakes, rivers, pigs, apples, shrubs, stars, and pretty much everything else. But before I try to get you to swallow this, I'm going to tell a story.

\subsection{Alien explorers meet a metaphysician}
{\em (Our protagonist sits in a chair in front of the fire.)} \\

\stager{Protagonist}{}{Suppose some years in the future we explore a distant planet very unlike our own. There seems to be organic life, but whether it is intelligent or not is hard to say. There seem to be artificial modifications to the environment, but again it's difficult to say what is natural and what isn't. But our intrepid scientists refuse to be baffled by this strange world. After some years of research they manage to pin down the physical processes and produce a model of the planet with really quite good predictive capabilities.}{{\em (A child's head droops.)}}

\stager{Protagonist}{(continuing)}{We've classified some phenomena as organisms with their set of biological underpinnings, and we've classified others as natural though non-living processes, such as weather patterns and geological change. There might be some things we've overlooked, but it doesn't look like there are going to be many more surprises. So I think it's safe to say that we've recognized most of what's on this planet, don't you?}{{\em (Vague nods. A sleeping \textsc{metaphysician} stirs briefly.)}}

\stager{Protagonist}{}{ Now I'm afraid I must admit that I haven't been entirely honest with you. The explorers in this story are not us, but an intelligent alien species. The planet is not distant at all; they are exploring our own Earth. And yet I think we must agree that we were correct---}{{\em (The metaphysician awakes with a start.)}}

\stage{Metaphysician}{(rising angrily)}{Now look here! These fool aliens don't recognize their own ignorance, let alone half of what exists on Earth! They forgot about adzes and Axminsters, boats and books, catalogs and cups, doors and dumbwaiters, earrings and elegance! Their ontology would fit in a knapsack!}

And so the question is asked to you, readers: are the things that went unseen by the aliens {\em really there?}

%\bibliography{everything}
%\bibliographystyle{ChicagoReedweb}
%
%\end{spacing}
%\end{document}
