\documentclass[11pt]{standalone}
\usepackage{standalone} \newif\ifstandlone \standalonetrue
\usepackage[left=1.75in, right=1.75in, top=1.25in, bottom=1.25in]{geometry}
\geometry{letterpaper}
\usepackage{graphicx}
%\usepackage{tipa}
%\usepackage{exaccent}
%\usepackage{txfonts}
%\usepackage{pxfonts}
\usepackage{enumitem}
%\usepackage{amssymb}
\usepackage{amsmath}
\usepackage{epstopdf}
\usepackage{setspace}
\usepackage{natbib}
\setcitestyle{aysep={}}
\usepackage{hyperref}
\usepackage{url}
\synctex=1

\DeclareSymbolFont{symbolsC}{U}{txsyc}{m}{n}
\DeclareMathSymbol{\strictif}{\mathrel}{symbolsC}{74}
\DeclareMathSymbol{\boxright}{\mathrel}{symbolsC}{128}

                \newenvironment{squote}{%
\begin{spacing}{1}
       	\begin{list}{}{%
\setlength{\labelwidth}{0pt}%
\rightmargin\leftmargin%
}
\item\relax
}{%
\end{list}%
\end{spacing}
}

\title{Vague terms and some other stuff}
\author{Alexander A. Dunn}
\begin{document}
\ifstandalone
\maketitle
\begin{spacing}{1.5}
\fi

\section{The tolerance of ordinary terms}
There is a feature of ordinary terms that Unger's arguments exploit.
This feature is what Crispin Wright calls ``tolerance''.  A term is
tolerance if, given that it applies to a certain situation, it would
also apply to a situation minutely (or indiscriminably) different from
the first situation.  `Stone' is tolerance because, for any given
object to which `stone' applies, we can remove a single atom (or even
a larger speck) from that object and remain justified in applying the
term to the slightly smaller object.

Why should we think that `stone' applies to the slightly smaller
object?  If we could apply the term to the first object but withhold
it from the second (slightly smaller) object, we would not be so
easily ensnared.

What compels us to apply the term to both objects is the pressure of
consistency.  Because there is no relevant difference between the two
objects that would justify our application of `stone' to one and not
the other, we feel that the term must therefore apply to both.  We
imagine that our language-use must be consistent and {\em regular}:

\begin{squote}
We suppose our use of language to be fundamentally {\em regular}; we
picture the learning of language as the acquisition [or] grasp of a
set of rules for the combination and application of expressions
\citep[326]{wright1975}.
\end{squote}

This is the first of two assumptions that are required by
sorites-style arguments.  The second is that we can discover these
(consistent, regular) rules of language-use by examining our language
``from within'':

\begin{squote}
The question now arises, what means are legitimate in the attempt to
discover features of the \emph{substantial} rules for expressions in
our language, the rules which determine specifically the senses of
such expressions?  The view of the matter with which we are centrally
concerned in this paper is that we may legitimately approach our use
of language from within, that is, reflectively as self-conscious
masters of it, rather than externally, equipped only with behavioural
notions.  We may appeal to our conception of what justifies the
application of a particular expression; we may appeal to our
conception of what we should count as an adequate explanation of the
sense of a particular expression; to the limitations imposed by our
senses and memories on the kind of instruction which we can actually
implement; and to the kind of consequence which we attach to the
application of a given predicate, to what we conceive as the point of
the classification which the predicate effects.  The notion that forms
the primary concern of this paper---henceforward referred to as the
\emph{governing view}---is that we can derive from such considerations
a reflective awareness of how expressions in our language are
understood, and so of the nature of the rules which determine their
correct use.  The governing view, then, is a conjunction of two
theses: that our use of language is properly seen, like a game, as an
activity in which the allowability of a move is determined by rule,
and that properties of the rules may be discovered by means of the
sorts of consideration just described. \citep[327]{wright1975}.
\end{squote}

Wright uses the term `red' to illustrate how this second assumption
functions to generate sorites paradoxes.  If we reflect on how the
term `red' is taught, learned and used, we see that it is defined
emph{ostensively}.  We learn what `red' means by being shown red
things, and we cannot be said to understand the term unless we can,
for example, point out the red tiles on a quilt of several colors.  It
therefore appears that the criteria for applying the term `red' are
observational: if something looks red then, as long as one is not
deceived by strange lighting, `red' applies to that thing.  But
because my vision is far from perfect, I am unable to distinguish very
subtle differences in color.  Two patches of red might look identical
to me, but be slightly different shades.  But because \emph{I} cannot
tell them apart, and because the application-criteria for `red' are
observational, `red' will presumably apply to both patches of color,
if it applies to either.

But now suppose there is a long series of color patches on a wall; the
leftmost one is definitely red and the rightmost definitely orange.
There are enough patches in between that the difference in color
between any two adjacent patches is indiscernible to a human observer.
Now if `red' applies to the leftmost patch, then it applies to the
patch immediately to the right; the two patches are indistinguishable
in terms of their color.  But if `red' applies to this second patch,
then it applies to the third, because \emph{those} are
indistinguishable.  Eventually we will find ourselves applying `red'
to the rightmost patch, which, by stipulation, is \emph{not} red but
orange.  The sorites paradox is established.

The arguments involving stones and other things can be understood as
involving these two assumptions as well.  `Stone', `chair' and other
ordinary terms are also defined ostensively; the application-criteria
for these terms are observational.  `Observational' may over-emphasize
the role of vision, but the other senses play a role too.  For
example, here Austin lists some application-criteria for `telephone':

\begin{squote}
\ldots\,you tell me there's a telephone in the next room, and,
(feeling mistrustful) I decide to verify this\,\ldots\,I go into the
next room, and certainly there's something there that looks exactly
like a telephone.  But is it a case perhaps of \emph{trompe l'oeil}
painting?  I can soon settle that.  Is it just a dummy perhaps, not
connected up and with no proper works?  Well, I can take it to pieces
a bit and find out, or actually use it to ring somebody up---and
perhaps get them to ring me up too, just to make sure.  And of course,
if I do all these things, I \emph{do} make sure; what more could
possibly be required?  This object has already stood up to amply
enough tests to establish that it really is a telephone; and it isn't
just that, for everyday or practical or ordinary purposes, enough is
\emph{as good as} a telephone; what meets all these tests just
\emph{is} a telephone, no doubt about it \citep[118--119]{austin1964}.
\end{squote}

If (let us suppose) that it really is a telephone, then something
minutely different---perhaps the back of the receiver sanded down a
touch---cannot fail to also be a telephone.  For us to deny that the
second object is a telephone would be to violate our first assumption,
that our application-criteria are consistent.  But if we keep sanding
down the telephone ever so slowly, eventually there will be nothing
left.  So the sorites paradox is clearly still with us.

(Our two assumptions are behind the `success' of the problem of the
many as well.  Because we cannot distinguish between the various sets
and various boundaries of objects, we must (if we are to be
consistent) apply `cloud' and the other terms to each of the
candidates.)

\section{Getting out}
It appears that to escape the sorites paradoxes as well as the
problems of the many, we will have to reject or revise one or both of
the theses of the governing view.  Much of what follows will be
examinations of various strategies.

(The solution to the sorites paradox and the problem of the many is
not to deny that there are, in fact, chairs.  Instead we must look at
the two assumptions of the governing view, and decide what is to be
done with them.)

\section{The second thesis}
oink

\ifstandalone
\end{spacing}
\bibliography{everything}
\bibliographystyle{ChicagoReedweb}
\fi
\end{document}
