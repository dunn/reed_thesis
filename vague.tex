\documentclass[11pt]{article}
\usepackage{standalone} \newif\ifstandlone \standalonetrue
\usepackage[left=1.75in, right=1.75in, top=1.25in, bottom=1.25in]{geometry}
\geometry{letterpaper}
\usepackage{graphicx}
%\usepackage{tipa}
%\usepackage{exaccent}
%\usepackage{txfonts}
%\usepackage{pxfonts}
\usepackage{enumitem}
%\usepackage{amssymb}
\usepackage{amsmath}
\usepackage{epstopdf}
\usepackage{setspace}
\usepackage{natbib}
\setcitestyle{aysep={}}
\usepackage{hyperref}
\usepackage{url}
\synctex=1

\DeclareSymbolFont{symbolsC}{U}{txsyc}{m}{n}
\DeclareMathSymbol{\strictif}{\mathrel}{symbolsC}{74}
\DeclareMathSymbol{\boxright}{\mathrel}{symbolsC}{128}

                \newenvironment{squote}{%
\begin{spacing}{1}
       	\begin{list}{}{%
\setlength{\labelwidth}{0pt}%
\rightmargin\leftmargin%
}
\item\relax
}{%
\end{list}%
\end{spacing}
}

\title{Vague terms and some other stuff}
\author{Alexander A. Dunn}
\begin{document}
\ifstandalone
\maketitle
\begin{spacing}{1.5}
\fi

\section{Unger's intolerance}
Peter Unger denies that chairs exist.  (It is possible that he has
since recanted.)  Unger claims that this is a consequence of the
incoherency of the term `chair'.  As he argues, `chair' is an
incoherent term; being incoherent, he says, it cannot have any
application to things in the world; therefore he concludes that there
are no chairs.

The strength of Unger's argument for the incoherency of `chair' rests
on two things: the sorites paradox and the 'problem of the many'.

\subsection{Sorites paradoxes}
A typical case of a sorites paradox involves some object
(paradigmatically, a heap of sand) from which a minute quantity of
matter is removed.  If we are inclined to suppose that the initial
quantity of matter (in this case, sand) was really a heap, then the
removal of a single grain of sand should leave the heap intact.  At
least on an intuitive level, it seems false that the removal of a
single grain of sand could {\em ever} transform a heap to something
less than a heap.  The heap will of course be a slightly smaller heap,
but it seems that it must be a heap nonetheless.

Once we have conceded these two points, however, we have unwittingly
put our foot in it.  For if the removal of a grain of sand {\em never}
transforms a heap into a non-heap, then by repeatedly removing single
grains of sand, we will eventually find ourselves with a heap of sand
that consists of absolutely no sand at all!

Unger follows this line of reasoning with respect to not only chairs,
but stones as well:

\begin{squote}
Consider a stone, consisting of a certain finite number of atoms.  If
we or some physical process should remove one atom, without
replacement, then there is left that number minus one, presumably
constituting a stone still\,\ldots\,after another atom is removed,
there is that original number minus two; so far, so good.  But after
that certain number has been removed, in similar stepwise fashion,
there are no atoms at all in the situation, while we must still be
supposing that there is a stone there.  But as we have already agreed,
if there is a stone present, then there must be some atoms\,\ldots\,I
suggest that any adequate response to this contradiction must
include\,\ldots\,the denial of the existence of even a single
stone.~\citep[121--122]{unger1979}
\end{squote}

Unger generalizes this argument and denies the existence of all
``ordinary things''.  An object that is generally thought to endure
minuscule losses of matter is therefore banished.

\subsection{The problem of the many}
Here we find a rather similar method.  A cloud is, presumably,
composed of molecules.  There is probably then a set of molecules, the
members of which compose the cloud.  Call that set $S$.  Now consider
$S_1$.  This is a set of molecules that includes all of the members of
$S$ as well as one additional molecule.  Do the members of $S_1$
compose a cloud?  Surely they are just as well suited to do so.  Now
consider $S_2$\,\ldots

Because these numerous 'candidates' are equally (or nearly equally)
well suited to be clouds, we seem forced to conclude that there are
either many clouds where we supposed there to be one, or rather no
clouds at all:

\begin{squote}
No matter where we start, the complex first chosen has nothing
objectively in its favor to make it a better candidate for cloudhood
than so many of its overlappers are.  Putting the matter somewhat
personally, each one's claim to be a cloud is just as good, no better
and no worse, than each of the many others.  And, by all odds, each
complex has \emph{at least} as good a claim as any still further real
entity in the situation.  So, either \emph{all} of \emph{them} make it or else
emph{nothing} does; in this real situation, either there are many clouds
or else there really are no clouds at all \citep[415--??]{unger1980a}.
\end{squote}

The problem of the many can also arise by considering the {\em
  boundary} of a given cloud.  It is natural to suppose that a cloud
has a determinate boundary.  But if we look at the edge of the cloud,
where we suppose the boundary to be, ``we may find, side by side, or
themselves overlapping, a great many potential boundaries for
clouds\,\ldots\,if our alleged typical item {[}the cloud{]} is indeed
a typical cloud, then many of these candidates, millions at least, do
not fail to be clouds altogether but are clouds of some
sort''~\citep[420--421]{unger1980a}.

The pattern of argumentation is the same for both approaches.  For a
certain cloud, a given set of members or a given boundary is supposed,
and it is argued that a set or boundary that differs minimally from
the original must also compose our bound a cloud.  The new set or
boundary does not appear to differ from the original in any relevant
way; there seems no principled way to deny that if the first set's
members compose a cloud, the second set's members do too.  And since
there are a great deal of very similar sets and boundaries, we find
ourselves with a plurality of clouds.

And of course, Unger does not rest content with applying the problem
of the many to clouds.  All ordinary objects get the same treatment;
he concludes that either there are a great many of them, or there are
none at all.  He claims, predictably, that the latter disjunct is
preferable.

\section{The tolerance of ordinary terms}
There is a feature of ordinary terms that Unger's arguments exploit.
This feature is what Crispin Wright calls ``tolerance''.  A term is
tolerance if, given that it applies to a certain situation, it would
also apply to a situation minutely (or indiscriminably) different from
the first situation.  `Stone' is tolerance because, for any given
object to which `stone' applies, we can remove a single atom (or even
a larger speck) from that object and remain justified in applying the
term to the slightly smaller object.

Why should we think that `stone' applies to the slightly smaller
object?  If we could apply the term to the first object but withhold
it from the second (slightly smaller) object, we would not be so
easily ensnared.

What compels us to apply the term to both objects is the pressure of
consistency.  Because there is no relevant difference between the two
objects that would justify our application of `stone' to one and not
the other, we feel that the term must therefore apply to both.  We
imagine that our language-use must be consistent and {\em regular}:

\begin{squote}
We suppose our use of language to be fundamentally {\em regular}; we
picture the learning of language as the acquisition [or] grasp of a
set of rules for the combination and application of expressions
\citep[326]{wright1975}.

This is the first of two assumptions that are required by
sorites-style arguments.  The second is that we can discover these
(consistent, regular) rules of language-use by examining our language
``from within'':

\begin{squote}
The question now arises, what means are legitimate in the attempt to
discover features of the _substantial_ rules for expressions in our
language, the rules which determine specifically the senses of such
expressions?  The view of the matter with which we are centrally
concerned in this paper is that we may legitimately approach our use
of language from within, that is, reflectively as self-conscious
masters of it, rather than externally, equipped only with behavioural
notions.  We may appeal to our conception of what justifies the
application of a particular expression; we may appeal to our
conception of what we should count as an adequate explanation of the
sense of a particular expression; to the limitations imposed by our
senses and memories on the kind of instruction which we can actually
implement; and to the kind of consequence which we attach to the
application of a given predicate, to what we conceive as the point of
the classification which the predicate effects.  The notion that forms
the primary concern of this paper---henceforward referred to as the
_governing view_---is that we can derive from such considerations a
reflective awareness of how expressions in our language are
understood, and so of the nature of the rules which determine their
correct use.  The governing view, then, is a conjunction of two
theses: that our use of language is properly seen, like a game, as an
activity in which the allowability of a move is determined by rule,
and that properties of the rules may be discovered by means of the
sorts of consideration just described. \citep[327]{wright1975}.
\end{squote}

Wright uses the term `red' to illustrate how this second assumption
functions to generate sorites paradoxes.  If we reflect on how the
term `red' is taught, learned and used, we see that it is defined
_ostensively_.  We learn what `red' means by being shown red things,
and we cannot be said to understand the term unless we can, for
example, point out the red tiles on a quilt of several colors.  It
therefore appears that the criteria for applying the term `red' are
observational: if something looks red then, as long as one is not
deceived by strange lighting, `red' applies to that thing.  But
because my vision is far from perfect, I am unable to distinguish very
subtle differences in color.  Two patches of red might look identical
to me, but be slightly different shades.  But because _I_ cannot tell
them apart, and because the
application-criteria for `red' are observational, `red' will
presumably apply to both patches of color, if it applies to either.

But now suppose there is a long series of color patches on a wall; the
leftmost one is definitely red and the rightmost definitely orange.
There are enough patches in between that the difference in color
between any two adjacent patches is indiscernible to a human
observer.  Now if `red' applies to the leftmost patch, then it applies
to the patch immediately to the right; the two patches are
indistinguishable in terms of their color.  But if `red' applies to
this second patch, then it applies to the third, because _those_ are
indistinguishable.  Eventually we will find ourselves applying `red'
to the rightmost patch, which, by stipulation, is _not_ red but
orange.  The sorites paradox is established.

The arguments involving stones and other things can be understood as
involving these two assumptions as well.  `Stone', `chair' and other
ordinary terms are also defined ostensively; the application-criteria
for these terms are observational.  `Observational' may over-emphasize
the role of vision, but the other senses play a role too.  For
example, here Austin lists some application-criteria for `telephone':

\begin{squote}
\ldots\,you tell me there's a telephone in the next room, and,
(feeling mistrustful) I decide to verify this\,\ldots\, I go into the
next room, and certainly there's something there that looks exactly
like a telephone.  But is it a case perhaps of _trompe l'oeil_
painting?  I can soon settle that.  Is it just a dummy perhaps, not
connected up and with no proper works?  Well, I can take it to pieces
a bit and find out, or actually use it to ring somebody up---and
perhaps get them to ring me up too, just to make sure.  And of course,
if I do all these things, I _do_ make sure; what more could possibly
be required?  This object has already stood up to amply enough tests
to establish that it really is a telephone; and it isn't just that,
for everyday or practical or ordinary purposes, enough is _as good as_
a telephone; what meets all these tests just _is_ a telephone, no
doubt about it \citep[115?]{austin1965}.
\end{squote}

If (let us suppose) that it really is a telephone, then something
minutely different---perhaps the back of the receiver is sanded down a
touch---cannot fail to also be a telephone.  For us to deny that the
second object is a telephone would be to violate our first assumption,
that our application-criteria are consistent.  But if we keep sanding
down the telephone ever so slowly, eventually there will be nothing
left.  So the sorites paradox is clearly still with us.

(Our two assumptions are behind the `success' of the problem of the
many as well.  Because we cannot distinguish between the various sets
and various boundaries of objects, we must (if we are to be
consistent) apply `cloud' and the other terms to each of the
candidates.)

\section{Getting out}
It appears that to escape the sorites paradoxes as well as the
problems of the many, we will have to reject or revise one or both of
the theses of the governing view.  Much of what follows will be
examinations of various strategies.

First, however, something must be said as to why we need to `get out'
in the first place.  Why not conclude with Unger that there just
aren't any chairs?

\subsection{Unger's skeptical solution}
Unger's objective, in his series of papers on ordinary things, is to
show that there are no chairs, tables, etc.  He therefore claims that
the sorites paradox is not a paradox at all; it simply shows that
there are no chairs et.\ al.:

\begin{squote}
While Eubulides' contribution has often been labeled `the sorites
paradox', there is nothing here which is a paradox in any
philosophically important sense\,\ldots\,Accepting our negative
conclusions here does not mean important logical trouble for us; we
only think we have troubles while we refuse to admit their validity
(\citeyear[145]{unger1979}).
\end{squote}

But things are not nearly so simple.  One problem is that Unger has no
way to explain why we _believe_ there to be chairs.  Another problem
is that Unger must deny that our use of ordinary terms follows any
sort of pattern or displays any competence at all.

\subsection{Sources of belief}
(Integrate old material)

\subsection{Competence and correctness}
Setting aside whether or not expressions of propositions like ``that's
a chair'' are ever _true_, it seems right to say that there are at
least correct and incorrect uses of the terms.  How else can it be
that we have to _learn_ how to use the terms?  But Unger denies that
we can use such terms correctly:

\begin{squote}
Concerning words and kinds, now, we might say this.  First, we might
say that it is in connection with _semantics_ that our reasonings have
what are their most obvious implications and, second, that their most
obvious semantic implications concern certain _sortal nouns_, namely,
those which purport to denote ordinary things.  Thus, it appears quite
obvious to us now that there will be no application to things for such
nouns as `stone' and `rock', `twig' and `log', `planet' and `sun',
`mountain' and `lake', `sweater' and `cardigan', `telescope' and
`microscope', and so on, and so forth.  Simple positive sentences
containing these terms will never, given their current meanings,
express anything true, correct, accurate, etc., or even anything which
is anywhere close to being any of those things
(\citeyear[148]{unger1979}).

This seems simply bizarre.  On what grounds, then, do parents correct
their children with respect to their use of ordinary terms?  Are they
compelled by some irrational force to consider certain utterances
correct and others incorrect?  Unger's view is extremely implausible,
and it becomes more so when we consider color words.  As mentioned
above, Crispin Wright uses `red' as a example of a vague term.  It is,
according to the governing view, semantically incoherent in just the
same way that ordinary terms like `chair' and `table' are.  Unger does
not address color words, but it seems that he would be compelled to
treat them as incoherent.  He would have to conclude that, as `red' is
incoherent, it has no application; he would have to say that there are
no red things, and that no expressions of propositions like ``that is
red'' could ever express anything correct or accurate, let alone true.
But what brings us to consider the problem of vagueness with regard to
terms like `red' is that they _do_ seem to be applied in correct and
incorrect ways.  It seemed correct to apply `red' to the leftmost
color patch and incorrect to apply it to the rightmost one.  The
difficulties that arise in the middle of the spectrum are only
troubling _because_ they conflict with the obvious correctness of the
application (or withholding) of `red' at the edge of the spectrum.
The sorites paradox might make us doubt whether our application of
certain terms is consistent, but it should not convince us that we do
not apply the terms _correctly_:

\begin{squote}
It is, to begin with, unclear how far our use of e.g. the vocabulary
  of colours _is_ consistent.  The descriptions given of awkward cases
  may vary from occasion to occasion.  Besides that, the notion of
  using a predicate consistently would appear to require some
  objective criteria for variation in relevant respects among items to
  be described in terms of it; but what is distinctive about
  observational predicates is exactly the lack of such criteria.  So
  it would be unwise to lean too heavily, as though it were a matter
  of hard fact, upon the consistency of our employment of colour
  predicates.  What, however, may be depended upon is that our use of
  these predicates is largely _successful_; the expectations which we
  form on the basis of others' ascriptions of colour are not usually
  disappointed.  Agreement is generally possible about how colours are
  to be described; and this, of course, is equivalent to saying that
  others _seem_ to use colour predicates in a largely consistent way
  \citep[361]{wright1975}.
\end{squote}

Just as Unger cannot deny that we believe there to be chairs, so he
cannot deny that there is a pattern of use for ordinary terms such
that some uses are correct and others incorrect.

The solution to the sorites paradox and the problem of the many is not
to deny that there are, in fact, chairs.  Instead we must look at the
two assumptions of the governing view, and decide what is to be done
with them.

\section{The second thesis}
oink

\ifstandalone
\end{spacing}
\fi
\end{document}
