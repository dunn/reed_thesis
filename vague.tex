\documentclass[11pt]{standalone}
\usepackage{standalone} \newif\ifstandlone \standalonetrue
\usepackage[left=1.75in, right=1.75in, top=1.25in, bottom=1.25in]{geometry}
\geometry{letterpaper}
\usepackage{graphicx}
%\usepackage{tipa}
%\usepackage{exaccent}
%\usepackage{txfonts}
%\usepackage{pxfonts}
\usepackage{enumitem}
%\usepackage{amssymb}
\usepackage{amsmath}
\usepackage{epstopdf}
\usepackage{setspace}
\usepackage{natbib}
\setcitestyle{aysep={}}
\usepackage{hyperref}
\usepackage{url}
\synctex=1

\DeclareSymbolFont{symbolsC}{U}{txsyc}{m}{n}
\DeclareMathSymbol{\strictif}{\mathrel}{symbolsC}{74}
\DeclareMathSymbol{\boxright}{\mathrel}{symbolsC}{128}

                \newenvironment{squote}{%
\begin{spacing}{1}
       	\begin{list}{}{%
\setlength{\labelwidth}{0pt}%
\rightmargin\leftmargin%
}
\item\relax
}{%
\end{list}%
\end{spacing}
}

\title{Vague terms and some other stuff}
\author{Alexander A. Dunn}
\begin{document}
\ifstandalone
\maketitle
\begin{spacing}{1.5}
\fi

\section{The tolerance of ordinary terms}
There is a feature of ordinary terms that is exploited by the sorites
paradoxes and by the problem of the many.  This feature is what
Crispin Wright calls ``tolerance''.  A term is tolerance if, given
that it applies to a certain situation, it would also apply to a
situation minutely (or indiscriminably) different from the first
situation.  `Stone' is tolerant because, for any given object to which
`stone' applies, we can remove a single atom (or even a larger speck)
from that object and remain justified in applying the term to the
slightly smaller object.

Why should we think that `stone' applies to the slightly smaller
object?  If we could apply the term to the first object but withhold
it from the second (slightly smaller) object, we would not be so
easily ensnared by the sorites paradox.

What compels us to apply the term to both objects is the pressure of
consistency.  Because there is no relevant difference between the two
objects that would justify our application of `stone' to one and not
the other, we feel that the term must therefore apply to both.  We
imagine that our language-use must be consistent and {\em regular}:

\begin{squote}
We suppose our use of language to be fundamentally {\em regular}; we
picture the learning of language as the acquisition [or] grasp of a
set of rules for the combination and application of expressions
\citep[326]{wright1975}.
\end{squote}

This is the first of two assumptions that are required by
sorites-style arguments.  The second is that we can discover these
(consistent, regular) rules of language-use by examining our language
``from within'':

\begin{squote}
The question now arises, what means are legitimate in the attempt to
discover features of the \emph{substantial} rules for expressions in
our language, the rules which determine specifically the senses of
such expressions?  The view of the matter with which we are centrally
concerned in this paper is that we may legitimately approach our use
of language from within, that is, reflectively as self-conscious
masters of it, rather than externally, equipped only with behavioural
notions.  We may appeal to our conception of what justifies the
application of a particular expression; we may appeal to our
conception of what we should count as an adequate explanation of the
sense of a particular expression; to the limitations imposed by our
senses and memories on the kind of instruction which we can actually
implement; and to the kind of consequence which we attach to the
application of a given predicate, to what we conceive as the point of
the classification which the predicate effects.  The notion that forms
the primary concern of this paper---henceforward referred to as the
\emph{governing view}---is that we can derive from such considerations
a reflective awareness of how expressions in our language are
understood, and so of the nature of the rules which determine their
correct use.  The governing view, then, is a conjunction of two
theses: that our use of language is properly seen, like a game, as an
activity in which the allowability of a move is determined by rule,
and that properties of the rules may be discovered by means of the
sorts of consideration just described. \citep[327]{wright1975}.
\end{squote}

Wright uses the term `red' to illustrate how this second assumption
functions to generate sorites paradoxes.  If we reflect on how the
term `red' is taught, learned and used, we see that it is defined
\emph{ostensively}.  We learn what `red' means by being shown red
things, and we cannot be said to understand the term unless we can,
for example, point out the red tiles on a quilt of several colors.  It
therefore appears that the criteria for applying the term `red' are
observational: if something looks red then, as long as one is not
deceived by strange lighting, `red' applies to that thing.  But
because my vision is far from perfect, I am unable to distinguish very
subtle differences in color.  Two patches of red might look identical
to me, but be slightly different shades.  But because \emph{I} cannot
tell them apart, and because the application-criteria for `red' are
observational, `red' will presumably apply to both patches of color,
if it applies to either.

But now suppose there is a long series of color patches on a wall; the
leftmost one is definitely red and the rightmost definitely orange.
There are enough patches in between that the difference in color
between any two adjacent patches is indiscernible to a human observer.
Now if `red' applies to the leftmost patch, then it applies to the
patch immediately to the right; the two patches are indistinguishable
in terms of their color.  But if `red' applies to this second patch,
then it applies to the third, because \emph{those} are
indistinguishable.  Eventually we will find ourselves applying `red'
to the rightmost patch, which, by stipulation, is \emph{not} red but
orange.  The sorites paradox is established.

The arguments involving stones and other things can be understood as
involving these two assumptions as well.  `Stone', `chair' and other
ordinary terms are also defined ostensively; the application-criteria
for these terms are observational.  `Observational' may over-emphasize
the role of vision; the other senses play a role too.  For example,
here Austin lists some application-criteria for `telephone':

\begin{squote}
\ldots\,you tell me there's a telephone in the next room, and,
(feeling mistrustful) I decide to verify this\,\ldots\,I go into the
next room, and certainly there's something there that looks exactly
like a telephone.  But is it a case perhaps of \emph{trompe l'oeil}
painting?  I can soon settle that.  Is it just a dummy perhaps, not
connected up and with no proper works?  Well, I can take it to pieces
a bit and find out, or actually use it to ring somebody up---and
perhaps get them to ring me up too, just to make sure.  And of course,
if I do all these things, I \emph{do} make sure; what more could
possibly be required?  This object has already stood up to amply
enough tests to establish that it really is a telephone; and it isn't
just that, for everyday or practical or ordinary purposes, enough is
\emph{as good as} a telephone; what meets all these tests just
\emph{is} a telephone, no doubt about it \citep[118--119]{austin1964}.
\end{squote}

Of course, if we slowly remove bits of matter from the telephone,
there will probably be a point at which it suddenly stops working;
there may be a sharp cut-off between a working telephone and a broken
one.  But we could theoretically construct a case in which this is not
so; perhaps the volume is decreased by an indiscriminable amount
before finally cutting out completely.  By the end of this process the
phone is obviously not functional, but when should we say that it
stops working?  A sorites paradox can, it seems, be constructed for
most (if not all) terms that rely on observational criteria of
application.

(The two theses of the governing view are behind the success of the
problem of the many as well.  Because we cannot distinguish between
the various sets and various boundaries of objects, we must (if we are
to be consistent) apply `cloud' and the other terms to each of the
candidates.)

\section{Getting out}
It appears that to escape the sorites paradoxes as well as the
problems of the many, we will have to reject or revise one or both of
the theses of the governing view.  Much of what follows will be
examinations of various strategies.

(The solution to the sorites paradox and the problem of the many is
not to deny that there are, in fact, chairs.  Instead we must look at
the two assumptions of the governing view, and decide what is to be
done with them.)

\subsection{Working within the constraints of the governing view}
I think certain brands of supervaluationism might be attempting to
solve the sorites paradox from within the constraints of the governing
view.  These versions of supervaluationism rely on our ability to make
vague terms precise.  `Red', for instance, might, on a certain
precisification, be defined as a hue within a certain range of patches
on a color-chart.  Then the application-criteria for `red' will no
longer be observational.  Whether or not `red' applies in a given
situation will not depend on what something looks like, but where it
falls on the chart.

This might make our use of color-terms less convenient.  If we really
want to know whether or not something is red, we can't just trust our
senses; we have to consult the chart.  But this is similar to what
goes on with our use of terms for length.  We may ultimately have to
appeal to a ruler to determine whether or not `more than 1 foot'
applies to a given situation, but that does not mean that we cannot
often tell without a ruler.  It is only with regard to the borderline
cases that we must pull out our rulers (and color-charts).

But

\begin{squote}
the possibility of our dispensing with paradigms [rulers and
  color-charts] for most practical purposes depends upon our capacity
e.g. to distinguish between cases where we could tell whether or not
`red' applied just by looking and cases where we could not, where we
should have recourse to a chart.  If we are able to make such a
distinction, what objection can there be to introducing a predicate to
express it?  But then, it seems, the semantics of this predicate will
have to be observational \citep[359]{wright1975}.
\end{squote}

A predicate like ``looks as if it were red'' cannot be made precise.
It is specifically introduced to rely on observational
criteria---whether or not something {\em looks} red.  Now this
predicate is susceptible to the sorites paradox.  Recall the series of
color-patches, moving indiscriminably from red to orange.  The
leftmost (red) patch certainly looks as if it were red, and the
rightmost (orange) patch certainly does not look as if it were red.
But there is no point in between at which we can say that the patches
cease to look as if they were red.  So if we continue to hold both
theses of the governing view, we are forced to conclude that this
predicate is incoherent.  It seems that we must, after all, reject one
of the theses of the governing view.

\subsection{The second thesis}
Our options:
\begin{enumerate}
  \item Epistemicism seems to involve a rejection of the second thesis
    of the governing view.  It involves the claim that the meanings of
    vague terms like `heap' and `red' {\em are} precise, but that we
    do not (indeed, cannot) know exactly where their applicability
    ends.  So there is no question of deducing their
    application-criteria from the sort of internal examination that is
    called for by the second thesis.  This is an interesting view, but
    does seem pretty implausible that our (probably inconsistent) use
    of vague terms determines their meaning with such precision.
    Unger judgement as to what is and is not absurd may not be the
    most reliable, but I think he might be right here:

    \begin{squote}
      Does anyone imagine that our concept of a swizzle stick
      discriminates at the required atomic level?  Surely, this is
      quite absurd.  But, then, it is just as absurd in the case of
      tables, and of stones (\citeyear[126]{unger1979}).
    \end{squote}

  \item One might attempt to apply Markosian's theory of `brutal
    composition' to semantics.  On his theory of mereology
    (\citeyear{markosian1998a}), whether or not certain objects
    compose another object is a brute, unexplained fact---there is no
    finite, non-trivial answer to van Inwagen's Special Composition
    Question.  Correspondingly one might claim that whether or not a
    term (like `red') applies to a given situation is also a brute
    fact.  There would be no underlying principle from which one could
    deduce the application-criteria for such a term---and certainly no
    principle discoverable by means of the internal examination called
    for by the second thesis.

    This might work for ordinary object terms, but it less appropriate
    for color terms.  As Wright points out, we may in fact apply color
    terms inconsistently, but what `ontological import' does this
    have?  If ``\,`x is red' is true iff x is red'' is true, then if
    we allow any inconsistency in our (correct) application of ``r is
    red'', then we may end up committed to some color being red which
    is closer on the color-spectrum to orange than some color which is
    not red but orange.  This doesn't make much sense (including this
    explanation).
\end{enumerate}

\subsection{The first thesis}
What would be involved in rejecting the notion that our language-use
is governed by consistent rules?  Earlier we criticized Unger for
denying that ordinary terms could ever have correct or accurate uses,
but if there are no rules governing them, on what grounds can we say
that a given use is correct?
\ifstandalone
\end{spacing}
\bibliography{everything}
\bibliographystyle{ChicagoReedweb}
\fi
\end{document}
